\changefontsizes{20pt}
\chapter{Initial Controller Design}
\label{cha:chap3}
\changefontsizes{12pt}
For the controller design, it is imperitive that a concrete control objective is set so that controller performance can be judged. The CF Bike is the plant for our control system and it is a MIMO (Multi-input, Multi-output) system. The inputs of our plant are the X-Y Planar forces and torque about Z axis, applied at the centre of gravity of the bike. The outputs are the translational positions of the bike and the rotation of the bike. Since each output would be coupled with other inputs (for example, the translation position of the bike would depend on the planar control forces and the control torque as well), our plant is quite interative. Irrespective of that, we use \emph{Decentralized Control}, i.e. the MIMO system is treated as three independent SISO systems.

% --------------------------------------------------------------------------
% 		Section 3.1
% --------------------------------------------------------------------------
\section{Control Objective}
\label{sec:chap3sec1}
The control objective is derived from the aim of SAA: mimicing the behaviour of the actual bike ride (or the reference simulation ride) using control. Thus, our primary control performance parameters will be the closeness of CF and RS bike loads at the validation points mentioned in section \ref{sec:kap2ab2}.
Thus, we are not necessarily very bothered about the traditional control system performance criteria like overshoot and other time domain peformance characteristics for regulation. Moreover, while using classical control methods, our tuning of the controller parameters can't include tradional tuning strategies like open/closed-loop shaping since reducing regulation/tracking error is not our goal. 
Thus, as the first iteration, imbalance compensation is attempted. Since the CF Bike experiences external excitations, there will be some net imbalance created, both translational and rotational, on the bike frame. The net rotational imbalance can be modelled as shown in Figure \ref{fig:imbalance1}.
\begin{figure}[h!]
	\centering
	\tikzset{->-/.style={decoration={
			markings,
			mark=at position #1 with {\arrow{>}}},postaction={decorate}}}
	
\begin{tikzpicture}
	%% ROADSURFACE BIKE
	% Coordinates
	\coordinate (rh) at (-4,0); % Rearhub
	\coordinate (fh) at (4,0); % Fronthub
	\coordinate (d1) at ($(rh)!0.25!(fh)$); % (1,0)
	\coordinate (sb) at ($(d1)!0.33!90:(fh)$); % Seat Bottom
	\coordinate (st) at ($(d1)!0.5!90:(fh)$); % Seat Top
	\coordinate (sr) at ($(st)!0.3!90:(sb)$); % Seat Right
	\coordinate (sl) at ($(st)!0.3!-90:(sb)$); % Seat Left
	\coordinate (d2) at ($(rh)!0.5!(fh)$); % (2,0)
	\coordinate (p) at ($(d2)!0.15!-90:(fh)$); % Pedal
	\coordinate (d3) at ($(rh)!0.875!(fh)$); % (3.5,0)
	\coordinate (hbb) at ($(d3)!1!-90:(d2)$); % Handlebar Bottom
	\coordinate (d4) at ($(rh)!0.75!(fh)$); % (3,0)
	\coordinate (hbt) at ($(d4)!1!-90:(rh)$); % Handlebar Top
	\coordinate (d5) at ($(fh)!1!90:(d4)$); % (4,-1)
	\coordinate (d6) at ($(rh)!1!-90:(d2)$); % (0,-1)
	
	% Center of gravity
	\coordinate (cogR) at ($(rh)!0.375!20:(fh)$);
	\coordinate (d7) at ($(rh)!0.375!(fh)$);
	\coordinate (d8) at ($(rh)!0.342!90:(d7)$);
	\coordinate (cog) at ($(d8)!0.956!(cogR)$);
	\coordinate (cogL) at ($(d8)!0.911!(cogR)$);
	\coordinate (cogN) at ($(cog)!1!90:(cogR)$);
	\coordinate (cogS) at ($(cog)!1!-90:(cogR)$);
	
	% Moment Arms
	\draw[dotted, ->-=0.5,ultra thick] (hbt) -- (cog);
	\draw[dotted, ->-=0.5,ultra thick] (rh) -- (cog);
	\draw[dotted, ->-=0.5,ultra thick] (fh) -- (cog);
	\draw[dotted, ->-=0.5,ultra thick] (p) -- (cog);
	
	
	% Text for Moment Arms
	\coordinate (rRH) at ($(rh)!0.6!8:(cog)$);
	\coordinate (rP) at ($(p)!0.5!-30:(cog)$);
	\coordinate (rH) at ($(hbt)!0.5!-5:(cog)$);
	\coordinate (rFH) at ($(fh)!0.5!10:(cog)$);
	\node [text width = 1cm] at (rRH) {$\mathbf{r}_{RH}$};
	\node [text width = 1cm] at (rP) {$\mathbf{r}_{P}$};
	\node [text width = 1cm] at (rH) {$\mathbf{r}_{H}$};
	\node [text width = 1cm] at (rFH) {$\mathbf{r}_{FH}$};
	
	% External Forces
	\coordinate (fRH) at ($(rh)!0.5!-120:(cog)$);
	\coordinate (fP) at ($(p)!0.6!-140:(cog)$);
	\coordinate (fH) at ($(hbt)!0.2!120:(fh)$);
	\coordinate (fFH) at ($(fh)!0.3!120:(cog)$);
	\coordinate (fRHt) at ($(rh)!0.6!-120:(cog)$);
	\coordinate (fPt) at ($(p)!0.9!-140:(cog)$);
	\coordinate (fHt) at ($(hbt)!0.3!120:(fh)$);
	\coordinate (fFHt) at ($(fh)!0.4!120:(cog)$);
	
	\draw[->,ultra thick] (fRH) -- (rh);
	\node[text width = 1cm] at (fRHt) {$\mathbf{F}_{RH}$};
	\draw[->,ultra thick] (fP) -- (p);
	\node[text width = 1cm] at (fPt) {$\mathbf{F}_{P}$};
	\draw[->,ultra thick] (fH) -- (hbt);
	\node[text width = 1cm] at (fHt) {$\mathbf{F}_{H}$};
	\draw[->,ultra thick] (fFH) -- (fh);
	\node[text width = 1cm] at (fFHt) {$\mathbf{F}_{FH}$};
	

	\fill[black] (cogR) arc (0:90:0.125) -- (cog) -- cycle;
	\draw (cogR) arc (0:180:0.125) -- (cog) -- cycle;
	\draw (cogR) arc (0:270:0.125) -- (cog) -- cycle;
	\draw (cogR) arc (0:360:0.125) -- (cog) -- cycle;
	\fill[yellow] (cogN) to (cogL) to (cog);
	\fill[black] (cogL) to (cogS) to (cog);
	\fill[yellow] (cogS) to (cogR) to (cog);
	
	% wheels
	\draw[very thick](rh) circle (1);
	\draw[very thick](fh) circle (1);
	
	% frame and seat
	\draw (rh) -- (sb) -- (p) -- (rh);
	\draw (sb) -- (st);
	\draw (sl) -- (sr);
	\draw (sb) -- (hbb) -- (fh) -- (p) -- (sb);
	\draw (hbb) -- (hbt);
	
	% road
	\draw[ultra thick](-5,-1) .. controls (1,-0.5) and (2,-3) .. (6,0);

	
\end{tikzpicture}
	\caption{Net Imbalance force and torque on CF Bike}
	\label{fig:imbalance1}
\end{figure}
 It is important to note that the moment arm for each corresponding force will continuously change during the reference ride due to various inter-moving elements of the bike (for example, fork translates with respect to the front frame due to a spring connecting the two). For the control design in this thesis, which depends on a simulation reference, it is quite easy to extract the time variance of each moment arm, but when the loads are measured on the actual bike, recording the changing moving arms requires usage of additional sensors like IMUs which would record the position change of the different components of the bike. Keeping in mind economic feasibility and using minimal amount of sensors, this might be not the most suitable option. In such a scenario, an alternative is to measure the moment arms when the bike is static, and then assume that the change in the moment arms with time is negligible. The error in the actual imbalance on the bike and the imbalance estimated would then depend on various factors like the terrain and the manner of riding the bike, which would naturally affect the change of moment arms.


% --------------------------------------------------------------------------
% 		Section 3.2
% --------------------------------------------------------------------------
\section{Pure Feedforward for Imbalance Compensation}
\label{sec:chap3sec2}
A decentralized feedforward controller is used to feed in the negative of the imbalances estimated as shown in Figure \ref{fig:feedfwd}. The idea is to compensate whatever net imbalance is created from the excitations provided to the CF Bike.

It is quite clear from Figure \ref{fig:pureFeedFwdB}, which compares the loads from CF and RS bikes at point B, that such an approach is infact quite unsuitable. The bike frame becomes completely unstable in 3D space as a result of which its rotation degree of freedom explodes and hence the RS Bike forces are reflected very poorly. One plausible explanation for this would be the application method of the feedforward control forces on the CF Bike, which are localised only on the center of gravity of the bike. On the contrary, the excitations from the RS Bike are fed to the CF Bike at different discrete points with a given eccentricity from the center of gravity. In such a load application configuration, it might be possible that the expected compensation is not happening properly and the feedforward control forces are acting as additional excitations to the unconstrained CF Bike, hence inducing its uncontrollable and unstable movement in 3D space.
Thus, we can conclude that a basic position control using feedback may be necessary in order to maintain the stability of the CF Bike in 3D space.
\begin{figure}[h!]
  \centering
  \tikzstyle{block}     = [draw, rectangle, minimum height=1.5cm, minimum width=1.6cm]
    \tikzstyle{branch}    = [circle, inner sep=0pt, minimum size=1mm, fill=black, draw=black]
    \tikzstyle{connector} = [->, thick]
    \tikzstyle{dummy}     = [inner sep=0pt, minimum size=0pt]
    \tikzstyle{inout}     = []
    \tikzstyle{sum}       = [circle, inner sep=0pt, minimum size=2mm, draw=black, thick]
    \begin{tikzpicture}[auto, node distance=3cm, >=stealth']
      \node[block] (bike) {CF};
      \node[block, left of = bike] (feedfwd) {$\begin{matrix}
      -1&0&0\\
      0&-1&0\\
      0&0&-1
      \end{matrix}$};
      \node[inout,right of = bike] (y) {\textbf{y}};
      \node[inout,left of = feedfwd] (d) {\textbf{d}};

      \draw[connector] (d) -- (feedfwd);
      \draw[connector] (feedfwd) -- (bike);
      \draw[connector] (bike) -- (y);
    \end{tikzpicture}
	  \caption{Pure feedforward Control}
    \label{fig:feedfwd}
\end{figure}


\begin{figure}[h!]
     \centering
     \scalebox{1.0}{
     \begin{tikzpicture}
        % This file was created by matlab2tikz.
%
%The latest updates can be retrieved from
%  http://www.mathworks.com/matlabcentral/fileexchange/22022-matlab2tikz-matlab2tikz
%where you can also make suggestions and rate matlab2tikz.
%
\begin{tikzpicture}

\begin{axis}[%
width=4.521in,
height=1.476in,
at={(0.758in,2.571in)},
scale only axis,
xmin=0,
xmax=10,
xlabel style={font=\color{white!15!black}},
xlabel={Time (s)},
ymin=-241.685852050781,
ymax=20000,
ylabel style={font=\color{white!15!black}},
ylabel={FX (N)},
axis background/.style={fill=white},
xmajorgrids,
ymajorgrids,
legend style={at={(0.85,1)}, anchor=north east, legend cell align=left, align=left, draw=black}
]
\addplot [color=black, dashed, line width=2.0pt]
  table[row sep=crcr]{%
0.0949999988079071	2.00570869445801\\
0.100000001490116	1.89491093158722\\
0.104999996721745	1.80077695846558\\
0.109999999403954	1.72017312049866\\
0.115000002086163	1.65045356750488\\
0.119999997317791	3.31705713272095\\
0.125	9.30066204071045\\
0.129999995231628	12.6379013061523\\
0.135000005364418	15.031044960022\\
0.140000000596046	16.648458480835\\
0.144999995827675	17.653865814209\\
0.150000005960464	18.2128238677979\\
0.155000001192093	18.4613170623779\\
0.159999996423721	18.6960277557373\\
0.165000006556511	18.8919277191162\\
0.170000001788139	18.8828105926514\\
0.174999997019768	18.6068382263184\\
0.180000007152557	18.0122547149658\\
0.185000002384186	17.12184715271\\
0.189999997615814	15.8765964508057\\
0.194999992847443	219.090148925781\\
0.200000002980232	344.076721191406\\
0.204999998211861	414.820587158203\\
0.209999993443489	441.413787841797\\
0.215000003576279	436.748596191406\\
0.219999998807907	431.400604248047\\
0.224999994039536	399.495666503906\\
0.230000004172325	346.075988769531\\
0.234999999403954	281.704620361328\\
0.239999994635582	219.60368347168\\
0.245000004768372	172.671371459961\\
0.25	155.471939086914\\
0.254999995231628	187.228012084961\\
0.259999990463257	240.139022827148\\
0.264999985694885	300.491485595703\\
0.270000010728836	357.91162109375\\
0.275000005960464	404.962158203125\\
0.280000001192093	436.891387939453\\
0.284999996423721	453.309051513672\\
0.28999999165535	454.831726074219\\
0.294999986886978	448.932006835938\\
0.300000011920929	436.77587890625\\
0.305000007152557	414.071258544922\\
0.310000002384186	384.729400634766\\
0.314999997615814	353.256378173828\\
0.319999992847443	324.247436523438\\
0.324999988079071	301.532501220703\\
0.330000013113022	287.603790283203\\
0.33500000834465	283.388427734375\\
0.340000003576279	291.562408447266\\
0.344999998807907	301.522674560547\\
0.349999994039536	310.160217285156\\
0.354999989271164	315.057922363281\\
0.360000014305115	314.628509521484\\
0.365000009536743	309.861785888672\\
0.370000004768372	302.006530761719\\
0.375	288.123168945313\\
0.379999995231628	269.010589599609\\
0.384999990463257	246.183563232422\\
0.389999985694885	221.722351074219\\
0.395000010728836	197.795837402344\\
0.400000005960464	176.876647949219\\
0.405000001192093	159.266265869141\\
0.409999996423721	147.165252685547\\
0.41499999165535	139.638900756836\\
0.419999986886978	136.132537841797\\
0.425000011920929	136.124496459961\\
0.430000007152557	135.614196777344\\
0.435000002384186	133.431045532227\\
0.439999997615814	130.358978271484\\
0.444999992847443	124.595504760742\\
0.449999988079071	115.962257385254\\
0.455000013113022	104.813758850098\\
0.46000000834465	91.9112014770508\\
0.465000003576279	78.3416442871094\\
0.469999998807907	65.2714233398438\\
0.474999994039536	53.9079856872559\\
0.479999989271164	45.1242752075195\\
0.485000014305115	39.4535484313965\\
0.490000009536743	36.9612197875977\\
0.495000004768372	38.5592346191406\\
0.5	41.5151519775391\\
0.504999995231628	44.6676292419434\\
0.509999990463257	47.298412322998\\
0.514999985694885	48.9513702392578\\
0.519999980926514	49.4721794128418\\
0.524999976158142	49.0688781738281\\
0.529999971389771	48.5936164855957\\
0.535000026226044	47.3401718139648\\
0.540000021457672	45.8112945556641\\
0.545000016689301	44.5954322814941\\
0.550000011920929	44.1814727783203\\
0.555000007152557	45.6632995605469\\
0.560000002384186	48.5185279846191\\
0.564999997615814	52.5384635925293\\
0.569999992847443	57.499942779541\\
0.574999988079071	63.0450630187988\\
0.579999983310699	68.898551940918\\
0.584999978542328	74.821044921875\\
0.589999973773956	80.583381652832\\
0.595000028610229	86.1193542480469\\
0.600000023841858	91.3947601318359\\
0.605000019073486	96.4056015014648\\
0.610000014305115	101.28466796875\\
0.615000009536743	106.092948913574\\
0.620000004768372	110.959754943848\\
0.625	115.920127868652\\
0.629999995231628	121.076271057129\\
0.634999990463257	126.39249420166\\
0.639999985694885	131.905288696289\\
0.644999980926514	137.56364440918\\
0.649999976158142	143.291564941406\\
0.654999971389771	149.016189575195\\
0.660000026226044	154.655792236328\\
0.665000021457672	160.132339477539\\
0.670000016689301	165.384033203125\\
0.675000011920929	170.356903076172\\
0.680000007152557	175.0224609375\\
0.685000002384186	179.378692626953\\
0.689999997615814	183.433135986328\\
0.694999992847443	187.201950073242\\
0.699999988079071	190.703399658203\\
0.704999983310699	193.976486206055\\
0.709999978542328	197.024200439453\\
0.714999973773956	199.815460205078\\
0.720000028610229	202.375762939453\\
0.725000023841858	204.742172241211\\
0.730000019073486	206.754989624023\\
0.735000014305115	208.55110168457\\
0.740000009536743	210.014190673828\\
0.745000004768372	211.199264526367\\
0.75	212.053085327148\\
0.754999995231628	212.609558105469\\
0.759999990463257	212.830596923828\\
0.764999985694885	212.833297729492\\
0.769999980926514	212.652404785156\\
0.774999976158142	212.186752319336\\
0.779999971389771	211.404907226563\\
0.785000026226044	210.340942382813\\
0.790000021457672	209.024978637695\\
0.795000016689301	207.478149414063\\
0.800000011920929	205.718704223633\\
0.805000007152557	203.764343261719\\
0.810000002384186	201.632858276367\\
0.814999997615814	199.341842651367\\
0.819999992847443	196.907501220703\\
0.824999988079071	194.344421386719\\
0.829999983310699	191.654678344727\\
0.834999978542328	188.854125976563\\
0.839999973773956	185.962921142578\\
0.845000028610229	182.991775512695\\
0.850000023841858	179.946563720703\\
0.855000019073486	176.84602355957\\
0.860000014305115	173.730514526367\\
0.865000009536743	170.618316650391\\
0.870000004768372	167.517166137695\\
0.875	164.446578979492\\
0.879999995231628	161.419296264648\\
0.884999990463257	158.44596862793\\
0.889999985694885	155.53776550293\\
0.894999980926514	152.706619262695\\
0.899999976158142	149.962982177734\\
0.904999971389771	147.316909790039\\
0.910000026226044	144.780075073242\\
0.915000021457672	142.357696533203\\
0.920000016689301	140.046691894531\\
0.925000011920929	137.860107421875\\
0.930000007152557	135.80322265625\\
0.935000002384186	133.874160766602\\
0.939999997615814	132.081512451172\\
0.944999992847443	130.434814453125\\
0.949999988079071	128.94938659668\\
0.954999983310699	127.629615783691\\
0.959999978542328	126.474990844727\\
0.964999973773956	125.484062194824\\
0.970000028610229	124.664642333984\\
0.975000023841858	124.009674072266\\
0.980000019073486	123.518074035645\\
0.985000014305115	123.191101074219\\
0.990000009536743	123.021697998047\\
0.995000004768372	123.010063171387\\
1	123.15705871582\\
1.00499999523163	123.463218688965\\
1.00999999046326	123.922584533691\\
1.01499998569489	124.475692749023\\
1.01999998092651	125.122398376465\\
1.02499997615814	125.876655578613\\
1.02999997138977	126.740592956543\\
1.0349999666214	127.704521179199\\
1.03999996185303	128.767562866211\\
1.04499995708466	129.920043945313\\
1.04999995231628	131.158554077148\\
1.05499994754791	132.473648071289\\
1.05999994277954	133.858856201172\\
1.06500005722046	135.310699462891\\
1.07000005245209	136.816772460938\\
1.07500004768372	138.370498657227\\
1.08000004291534	139.959823608398\\
1.08500003814697	141.57763671875\\
1.0900000333786	143.217056274414\\
1.09500002861023	144.86930847168\\
1.10000002384186	146.526245117188\\
1.10500001907349	148.181045532227\\
1.11000001430511	149.829284667969\\
1.11500000953674	151.461807250977\\
1.12000000476837	153.071334838867\\
1.125	154.654769897461\\
1.12999999523163	156.205368041992\\
1.13499999046326	157.715972900391\\
1.13999998569489	159.181198120117\\
1.14499998092651	160.598907470703\\
1.14999997615814	161.961685180664\\
1.15499997138977	163.264038085938\\
1.1599999666214	164.503723144531\\
1.16499996185303	165.676452636719\\
1.16999995708466	166.77668762207\\
1.17499995231628	167.800857543945\\
1.17999994754791	168.751983642578\\
1.18499994277954	169.623489379883\\
1.19000005722046	170.412155151367\\
1.19500005245209	171.117294311523\\
1.20000004768372	171.738067626953\\
1.20500004291534	172.272079467773\\
1.21000003814697	172.718475341797\\
1.2150000333786	173.082717895508\\
1.22000002861023	173.364059448242\\
1.22500002384186	173.56462097168\\
1.23000001907349	173.690567016602\\
1.23500001430511	173.749771118164\\
1.24000000953674	173.747329711914\\
1.24500000476837	173.690887451172\\
1.25	173.553314208984\\
1.25499999523163	173.345260620117\\
1.25999999046326	173.063842773438\\
1.26499998569489	172.711944580078\\
1.26999998092651	172.29704284668\\
1.27499997615814	171.815002441406\\
1.27999997138977	171.270858764648\\
1.2849999666214	170.682144165039\\
1.28999996185303	170.046203613281\\
1.29499995708466	169.36784362793\\
1.29999995231628	168.656616210938\\
1.30499994754791	167.916122436523\\
1.30999994277954	167.151870727539\\
1.31500005722046	166.366317749023\\
1.32000005245209	165.565078735352\\
1.32500004768372	164.753204345703\\
1.33000004291534	163.932846069336\\
1.33500003814697	163.107727050781\\
1.3400000333786	162.282562255859\\
1.34500002861023	161.461639404297\\
1.35000002384186	160.649383544922\\
1.35500001907349	159.849975585938\\
1.36000001430511	159.066772460938\\
1.36500000953674	158.300933837891\\
1.37000000476837	157.556289672852\\
1.375	156.834777832031\\
1.37999999523163	156.134307861328\\
1.38499999046326	155.458648681641\\
1.38999998569489	154.8095703125\\
1.39499998092651	154.193481445313\\
1.39999997615814	153.611465454102\\
1.40499997138977	153.067001342773\\
1.4099999666214	152.574111938477\\
1.41499996185303	152.134140014648\\
1.41999995708466	151.748840332031\\
1.42499995231628	151.405776977539\\
1.42999994754791	151.111892700195\\
1.43499994277954	150.866088867188\\
1.44000005722046	150.664657592773\\
1.44500005245209	150.508407592773\\
1.45000004768372	150.393188476563\\
1.45500004291534	150.313232421875\\
1.46000003814697	150.267822265625\\
1.4650000333786	150.258651733398\\
1.47000002861023	150.288482666016\\
1.47500002384186	150.357620239258\\
1.48000001907349	150.468887329102\\
1.48500001430511	150.625778198242\\
1.49000000953674	150.832565307617\\
1.49500000476837	151.071243286133\\
1.5	151.343734741211\\
1.50499999523163	151.650650024414\\
1.50999999046326	151.994094848633\\
1.51499998569489	152.371887207031\\
1.51999998092651	152.781005859375\\
1.52499997615814	153.219360351563\\
1.52999997138977	153.685668945313\\
1.5349999666214	154.178085327148\\
1.53999996185303	154.694900512695\\
1.54499995708466	155.231994628906\\
1.54999995231628	155.784713745117\\
1.55499994754791	156.350967407227\\
1.55999994277954	156.924835205078\\
1.56500005722046	157.503295898438\\
1.57000005245209	157.774826049805\\
1.57500004768372	158.106552124023\\
1.58000004291534	158.3349609375\\
1.58500003814697	158.455032348633\\
1.5900000333786	158.562149047852\\
1.59500002861023	158.699279785156\\
1.60000002384186	158.898147583008\\
1.60500001907349	159.194869995117\\
1.61000001430511	159.606918334961\\
1.61500000953674	160.094284057617\\
1.62000000476837	160.673751831055\\
1.625	161.24382019043\\
1.62999999523163	161.758865356445\\
1.63499999046326	162.158660888672\\
1.63999998569489	162.412780761719\\
1.64499998092651	162.656784057617\\
1.64999997615814	162.831069946289\\
1.65499997138977	162.940475463867\\
1.6599999666214	162.999328613281\\
1.66499996185303	163.016067504883\\
1.66999995708466	162.98420715332\\
1.67499995231628	162.943710327148\\
1.67999994754791	162.927841186523\\
1.68499994277954	162.903411865234\\
1.69000005722046	162.856079101563\\
1.69500005245209	162.7939453125\\
1.70000004768372	162.716583251953\\
1.70500004291534	162.610885620117\\
1.71000003814697	162.475875854492\\
1.7150000333786	162.318954467773\\
1.72000002861023	162.143142700195\\
1.72500002384186	161.93359375\\
1.73000001907349	161.695770263672\\
1.73500001430511	161.426498413086\\
1.74000000953674	161.126281738281\\
1.74500000476837	160.798095703125\\
1.75	160.44758605957\\
1.75499999523163	160.071670532227\\
1.75999999046326	159.675231933594\\
1.76499998569489	159.257369995117\\
1.76999998092651	158.820266723633\\
1.77499997615814	158.372222900391\\
1.77999997138977	157.913665771484\\
1.7849999666214	157.442138671875\\
1.78999996185303	156.959197998047\\
1.79499995708466	156.46598815918\\
1.79999995231628	155.962387084961\\
1.80499994754791	155.449401855469\\
1.80999994277954	154.926483154297\\
1.81500005722046	154.394012451172\\
1.82000005245209	153.852462768555\\
1.82500004768372	153.302459716797\\
1.83000004291534	152.744293212891\\
1.83500003814697	152.177612304688\\
1.8400000333786	151.602447509766\\
1.84500002861023	151.019653320313\\
1.85000002384186	150.437591552734\\
1.85500001907349	149.85205078125\\
1.86000001430511	149.26286315918\\
1.86500000953674	148.669616699219\\
1.87000000476837	148.072555541992\\
1.875	147.479461669922\\
1.87999999523163	146.896377563477\\
1.88499999046326	146.318405151367\\
1.88999998569489	145.744964599609\\
1.89499998092651	145.157241821289\\
1.89999997615814	144.565887451172\\
1.90499997138977	143.970886230469\\
1.9099999666214	143.372360229492\\
1.91499996185303	142.770263671875\\
1.91999995708466	142.164611816406\\
1.92499995231628	141.560745239258\\
1.92999994754791	140.958267211914\\
1.93499994277954	140.355850219727\\
1.94000005722046	139.750228881836\\
1.94500005245209	139.14030456543\\
1.95000004768372	138.527603149414\\
1.95500004291534	137.917007446289\\
1.96000003814697	137.31608581543\\
1.9650000333786	136.724044799805\\
1.97000002861023	136.132263183594\\
1.97500002384186	135.510513305664\\
1.98000001907349	134.866073608398\\
1.98500001430511	134.196853637695\\
1.99000000953674	133.522857666016\\
1.99500000476837	132.833511352539\\
2	132.129302978516\\
2.00500011444092	131.42951965332\\
2.00999999046326	130.730575561523\\
2.01500010490417	130.036224365234\\
2.01999998092651	129.383316040039\\
2.02500009536743	128.774993896484\\
2.02999997138977	128.2177734375\\
2.03500008583069	127.676963806152\\
2.03999996185303	127.168975830078\\
2.04500007629395	126.719451904297\\
2.04999995231628	126.360618591309\\
2.0550000667572	126.091903686523\\
2.05999994277954	125.908462524414\\
2.06500005722046	125.818687438965\\
2.0699999332428	125.80638885498\\
2.07500004768372	125.87580871582\\
2.07999992370605	126.029640197754\\
2.08500003814697	126.265434265137\\
2.08999991416931	126.572875976563\\
2.09500002861023	126.948692321777\\
2.09999990463257	127.376487731934\\
2.10500001907349	127.850875854492\\
2.10999989509583	128.378768920898\\
2.11500000953674	128.961013793945\\
2.11999988555908	129.599960327148\\
2.125	130.301727294922\\
2.13000011444092	131.032516479492\\
2.13499999046326	131.775741577148\\
2.14000010490417	132.493850708008\\
2.14499998092651	133.170043945313\\
2.15000009536743	133.862121582031\\
2.15499997138977	134.45329284668\\
2.16000008583069	134.768341064453\\
2.16499996185303	135.033630371094\\
2.17000007629395	135.239944458008\\
2.17499995231628	135.273025512695\\
2.1800000667572	135.223388671875\\
2.18499994277954	135.129776000977\\
2.19000005722046	135.054779052734\\
2.1949999332428	134.975708007813\\
2.20000004768372	134.901565551758\\
2.20499992370605	134.839263916016\\
2.21000003814697	134.788131713867\\
2.21499991416931	134.749771118164\\
2.22000002861023	134.729904174805\\
2.22499990463257	134.736328125\\
2.23000001907349	134.707595825195\\
2.23499989509583	134.540588378906\\
2.24000000953674	134.401718139648\\
2.24499988555908	134.282104492188\\
2.25	134.147979736328\\
2.25500011444092	134.042846679688\\
2.25999999046326	134.017303466797\\
2.26500010490417	134.150909423828\\
2.26999998092651	134.494918823242\\
2.27500009536743	135.060546875\\
2.27999997138977	135.807434082031\\
2.28500008583069	136.623138427734\\
2.28999996185303	137.544418334961\\
2.29500007629395	138.577682495117\\
2.29999995231628	139.680999755859\\
2.3050000667572	140.865921020508\\
2.30999994277954	142.122497558594\\
2.31500005722046	143.438583374023\\
2.3199999332428	144.79997253418\\
2.32500004768372	146.195892333984\\
2.32999992370605	147.614242553711\\
2.33500003814697	149.040710449219\\
2.33999991416931	150.500869750977\\
2.34500002861023	152.005706787109\\
2.34999990463257	153.503601074219\\
2.35500001907349	155.006195068359\\
2.35999989509583	156.482315063477\\
2.36500000953674	157.882125854492\\
2.36999988555908	159.16975402832\\
2.375	160.356262207031\\
2.38000011444092	161.427124023438\\
2.38499999046326	162.318450927734\\
2.39000010490417	163.002487182617\\
2.39499998092651	163.45654296875\\
2.40000009536743	163.692352294922\\
2.40499997138977	163.707122802734\\
2.41000008583069	163.545150756836\\
2.41499996185303	163.263473510742\\
2.42000007629395	162.912399291992\\
2.42499995231628	162.618621826172\\
2.4300000667572	162.350814819336\\
2.43499994277954	162.190048217773\\
2.44000005722046	162.169509887695\\
2.4449999332428	162.082275390625\\
2.45000004768372	162.342514038086\\
2.45499992370605	162.674743652344\\
2.46000003814697	163.177856445313\\
2.46499991416931	163.894439697266\\
2.47000002861023	164.759796142578\\
2.47499990463257	165.755233764648\\
2.48000001907349	166.864761352539\\
2.48499989509583	168.053649902344\\
2.49000000953674	169.347946166992\\
2.49499988555908	170.779647827148\\
2.5	172.273483276367\\
2.50500011444092	173.633651733398\\
2.50999999046326	174.84635925293\\
2.51500010490417	176.045303344727\\
2.51999998092651	177.115234375\\
2.52500009536743	178.057846069336\\
2.52999997138977	178.902954101563\\
2.53500008583069	179.663146972656\\
2.53999996185303	180.325653076172\\
2.54500007629395	180.914276123047\\
2.54999995231628	181.396469116211\\
2.5550000667572	181.761993408203\\
2.55999994277954	181.985565185547\\
2.56500005722046	182.030792236328\\
2.5699999332428	181.876556396484\\
2.57500004768372	181.501770019531\\
2.57999992370605	180.904220581055\\
2.58500003814697	180.101623535156\\
2.58999991416931	179.022338867188\\
2.59500002861023	177.815765380859\\
2.59999990463257	176.477096557617\\
2.60500001907349	174.960906982422\\
2.60999989509583	173.335037231445\\
2.61500000953674	171.605209350586\\
2.61999988555908	169.810455322266\\
2.625	167.958755493164\\
2.63000011444092	166.053192138672\\
2.63499999046326	163.998596191406\\
2.64000010490417	161.952224731445\\
2.64499998092651	159.843841552734\\
2.65000009536743	157.453796386719\\
2.65499997138977	154.894332885742\\
2.66000008583069	152.149215698242\\
2.66499996185303	149.137466430664\\
2.67000007629395	145.870101928711\\
2.67499995231628	142.509796142578\\
2.6800000667572	139.177490234375\\
2.68499994277954	135.663070678711\\
2.69000005722046	132.217254638672\\
2.6949999332428	128.853073120117\\
2.70000004768372	125.592864990234\\
2.70499992370605	122.461578369141\\
2.71000003814697	119.448249816895\\
2.71499991416931	116.53092956543\\
2.72000002861023	113.716567993164\\
2.72499990463257	110.989570617676\\
2.73000001907349	108.326904296875\\
2.73499989509583	105.729782104492\\
2.74000000953674	103.199882507324\\
2.74499988555908	100.759735107422\\
2.75	98.4458999633789\\
2.75500011444092	96.3092727661133\\
2.75999999046326	94.4518356323242\\
2.76500010490417	92.9323806762695\\
2.76999998092651	91.7209167480469\\
2.77500009536743	90.7720260620117\\
2.77999997138977	90.0856094360352\\
2.78500008583069	89.6837921142578\\
2.78999996185303	89.5269012451172\\
2.79500007629395	89.5503997802734\\
2.79999995231628	89.6921920776367\\
2.8050000667572	89.9189300537109\\
2.80999994277954	90.1763534545898\\
2.81500005722046	90.4523773193359\\
2.8199999332428	90.7192611694336\\
2.82500004768372	90.9946594238281\\
2.82999992370605	91.2961044311523\\
2.83500003814697	91.7384490966797\\
2.83999991416931	92.3672103881836\\
2.84500002861023	93.2755889892578\\
2.84999990463257	94.610725402832\\
2.85500001907349	96.3661956787109\\
2.85999989509583	98.6754913330078\\
2.86500000953674	101.631713867188\\
2.86999988555908	105.15941619873\\
2.875	109.419845581055\\
2.88000011444092	114.379676818848\\
2.88499999046326	120.082290649414\\
2.89000010490417	126.568962097168\\
2.89499998092651	133.837432861328\\
2.90000009536743	141.877075195313\\
2.90499997138977	150.7158203125\\
2.91000008583069	160.311569213867\\
2.91499996185303	170.60563659668\\
2.92000007629395	181.464080810547\\
2.92499995231628	192.677062988281\\
2.9300000667572	203.995498657227\\
2.93499994277954	215.127868652344\\
2.94000005722046	225.801071166992\\
2.9449999332428	235.7353515625\\
2.95000004768372	244.759841918945\\
2.95499992370605	252.642822265625\\
2.96000003814697	259.387023925781\\
2.96499991416931	265.459930419922\\
2.97000002861023	270.384338378906\\
2.97499990463257	273.899475097656\\
2.98000001907349	276.094268798828\\
2.98499989509583	277.056640625\\
2.99000000953674	276.963836669922\\
2.99499988555908	275.943084716797\\
3	274.181549072266\\
3.00500011444092	271.777557373047\\
3.00999999046326	268.761413574219\\
3.01500010490417	265.107604980469\\
3.01999998092651	260.725341796875\\
3.02500009536743	255.550659179688\\
3.02999997138977	249.405075073242\\
3.03500008583069	242.138290405273\\
3.03999996185303	233.812759399414\\
3.04500007629395	224.477249145508\\
3.04999995231628	214.40510559082\\
3.0550000667572	203.880615234375\\
3.05999994277954	193.030822753906\\
3.06500005722046	182.28759765625\\
3.0699999332428	171.993270874023\\
3.07500004768372	161.910705566406\\
3.07999992370605	152.33479309082\\
3.08500003814697	143.202926635742\\
3.08999991416931	134.318817138672\\
3.09500002861023	125.753257751465\\
3.09999990463257	117.525001525879\\
3.10500001907349	109.51879119873\\
3.10999989509583	101.875938415527\\
3.11500000953674	94.6744842529297\\
3.11999988555908	88.0386276245117\\
3.125	82.2196731567383\\
3.13000011444092	77.6504669189453\\
3.13499999046326	74.8792190551758\\
3.14000010490417	72.9592514038086\\
3.14499998092651	71.5766830444336\\
3.15000009536743	70.3588027954102\\
3.15499997138977	69.1042404174805\\
3.16000008583069	67.7724761962891\\
3.16499996185303	66.4133911132813\\
3.17000007629395	65.0668563842773\\
3.17499995231628	63.7933540344238\\
3.1800000667572	62.9157867431641\\
3.18499994277954	62.6241683959961\\
3.19000005722046	62.8889999389648\\
3.1949999332428	63.4821281433105\\
3.20000004768372	64.567626953125\\
3.20499992370605	66.4948272705078\\
3.21000003814697	69.0147094726563\\
3.21499991416931	72.6471786499023\\
3.22000002861023	77.4419250488281\\
3.22499990463257	83.3674621582031\\
3.23000001907349	90.4710464477539\\
3.23499989509583	98.7725143432617\\
3.24000000953674	108.256622314453\\
3.24499988555908	118.826103210449\\
3.25	130.334289550781\\
3.25500011444092	142.568481445313\\
3.25999999046326	155.304809570313\\
3.26500010490417	168.24089050293\\
3.26999998092651	181.113037109375\\
3.27500009536743	193.607467651367\\
3.27999997138977	205.565841674805\\
3.28500008583069	216.851181030273\\
3.28999996185303	227.350967407227\\
3.29500007629395	237.104934692383\\
3.29999995231628	246.082183837891\\
3.3050000667572	254.222457885742\\
3.30999994277954	261.817626953125\\
3.31500005722046	268.543701171875\\
3.3199999332428	274.487762451172\\
3.32500004768372	279.643280029297\\
3.32999992370605	283.839813232422\\
3.33500003814697	287.012878417969\\
3.33999991416931	289.255279541016\\
3.34500002861023	290.494812011719\\
3.34999990463257	290.290252685547\\
3.35500001907349	288.497985839844\\
3.35999989509583	285.100952148438\\
3.36500000953674	280.066864013672\\
3.36999988555908	273.527740478516\\
3.375	265.655334472656\\
3.38000011444092	256.694122314453\\
3.38499999046326	246.91015625\\
3.39000010490417	236.523071289063\\
3.39499998092651	225.746704101563\\
3.40000009536743	214.725479125977\\
3.40499997138977	203.505508422852\\
3.41000008583069	192.150650024414\\
3.41499996185303	180.676467895508\\
3.42000007629395	169.108734130859\\
3.42499995231628	157.479522705078\\
3.4300000667572	145.906616210938\\
3.43499994277954	134.534759521484\\
3.44000005722046	123.591552734375\\
3.4449999332428	113.228996276855\\
3.45000004768372	103.662063598633\\
3.45499992370605	95.1396331787109\\
3.46000003814697	88.7467498779297\\
3.46499991416931	83.8145751953125\\
3.47000002861023	79.2780914306641\\
3.47499990463257	75.0530395507813\\
3.48000001907349	70.9610137939453\\
3.48499989509583	67.0062789916992\\
3.49000000953674	62.7481002807617\\
3.49499988555908	58.9881362915039\\
3.5	56.1394424438477\\
3.50500011444092	53.7524185180664\\
3.50999999046326	51.8958854675293\\
3.51500010490417	51.0255317687988\\
3.51999998092651	50.5438766479492\\
3.52500009536743	51.0989151000977\\
3.52999997138977	52.1108779907227\\
3.53500008583069	53.7985610961914\\
3.53999996185303	56.3729019165039\\
3.54500007629395	60.3594512939453\\
3.54999995231628	67.6429824829102\\
3.5550000667572	78.8632965087891\\
3.55999994277954	94.018928527832\\
3.56500005722046	113.096382141113\\
3.5699999332428	135.532867431641\\
3.57500004768372	160.10368347168\\
3.57999992370605	185.335830688477\\
3.58500003814697	209.535339355469\\
3.58999991416931	231.335327148438\\
3.59500002861023	249.646957397461\\
3.59999990463257	263.978271484375\\
3.60500001907349	274.106384277344\\
3.60999989509583	281.633026123047\\
3.61500000953674	286.941253662109\\
3.61999988555908	288.723541259766\\
3.625	288.13525390625\\
3.63000011444092	286.387359619141\\
3.63499999046326	284.921539306641\\
3.64000010490417	284.653900146484\\
3.64499998092651	285.968872070313\\
3.65000009536743	289.674987792969\\
3.65499997138977	293.124053955078\\
3.66000008583069	294.872528076172\\
3.66499996185303	294.624969482422\\
3.67000007629395	291.767303466797\\
3.67499995231628	284.890594482422\\
3.6800000667572	273.902374267578\\
3.68499994277954	259.308929443359\\
3.69000005722046	242.195648193359\\
3.6949999332428	223.844055175781\\
3.70000004768372	205.886276245117\\
3.70499992370605	189.238952636719\\
3.71000003814697	175.049942016602\\
3.71499991416931	163.192321777344\\
3.72000002861023	153.426208496094\\
3.72499990463257	145.046417236328\\
3.73000001907349	137.199890136719\\
3.73499989509583	129.143264770508\\
3.74000000953674	120.389381408691\\
3.74499988555908	110.862342834473\\
3.75	102.324325561523\\
3.75500011444092	95.0264587402344\\
3.75999999046326	88.9060974121094\\
3.76500010490417	82.74609375\\
3.76999998092651	76.2732467651367\\
3.77500009536743	70.1796493530273\\
3.77999997138977	64.7241516113281\\
3.78500008583069	60.0224761962891\\
3.78999996185303	57.0250205993652\\
3.79500007629395	55.8133926391602\\
3.79999995231628	55.4323043823242\\
3.8050000667572	55.5197601318359\\
3.80999994277954	55.5367774963379\\
3.81500005722046	55.1324195861816\\
3.8199999332428	54.0803298950195\\
3.82500004768372	52.6568832397461\\
3.82999992370605	51.7557411193848\\
3.83500003814697	52.7793197631836\\
3.83999991416931	58.9007301330566\\
3.84500002861023	72.8449478149414\\
3.84999990463257	92.6566543579102\\
3.85500001907349	117.798820495605\\
3.85999989509583	146.856430053711\\
3.86500000953674	177.524154663086\\
3.86999988555908	207.449554443359\\
3.875	234.53076171875\\
3.88000011444092	257.220581054688\\
3.88499999046326	274.7412109375\\
3.89000010490417	286.929901123047\\
3.89499998092651	294.727844238281\\
3.90000009536743	301.165893554688\\
3.90499997138977	302.965728759766\\
3.91000008583069	302.093566894531\\
3.91499996185303	300.4208984375\\
3.92000007629395	299.456939697266\\
3.92499995231628	300.475738525391\\
3.9300000667572	305.021667480469\\
3.93499994277954	311.170074462891\\
3.94000005722046	316.144866943359\\
3.9449999332428	318.410125732422\\
3.95000004768372	317.401702880859\\
3.95499992370605	313.976898193359\\
3.96000003814697	305.472839355469\\
3.96499991416931	291.767944335938\\
3.97000002861023	273.617645263672\\
3.97499990463257	252.383743286133\\
3.98000001907349	229.831359863281\\
3.98499989509583	207.732009887695\\
3.99000000953674	187.496734619141\\
3.99499988555908	170.03190612793\\
4	155.590805053711\\
4.00500011444092	143.724563598633\\
4.01000022888184	133.593475341797\\
4.0149998664856	124.2041015625\\
4.01999998092651	114.61979675293\\
4.02500009536743	104.521850585938\\
4.03000020980835	94.8766403198242\\
4.03499984741211	86.8752059936523\\
4.03999996185303	79.5986557006836\\
4.04500007629395	72.4842147827148\\
4.05000019073486	65.2733383178711\\
4.05499982833862	58.016227722168\\
4.05999994277954	50.9921989440918\\
4.06500005722046	45.2315673828125\\
4.07000017166138	41.1755561828613\\
4.07499980926514	39.3788604736328\\
4.07999992370605	38.8784790039063\\
4.08500003814697	39.4177703857422\\
4.09000015258789	40.0131225585938\\
4.09499979019165	39.1825675964355\\
4.09999990463257	35.6866264343262\\
4.10500001907349	29.6143989562988\\
4.1100001335144	22.5737609863281\\
4.11499977111816	21.2202205657959\\
4.11999988555908	24.6610298156738\\
4.125	47.4759635925293\\
4.13000011444092	90.599250793457\\
4.13500022888184	141.384918212891\\
4.1399998664856	196.112731933594\\
4.14499998092651	248.35514831543\\
4.15000009536743	292.986602783203\\
4.15500020980835	326.586303710938\\
4.15999984741211	347.723785400391\\
4.16499996185303	356.948486328125\\
4.17000007629395	364.107238769531\\
4.17500019073486	358.292907714844\\
4.17999982833862	342.119445800781\\
4.18499994277954	320.284057617188\\
4.19000005722046	298.914093017578\\
4.19500017166138	283.652709960938\\
4.19999980926514	278.986938476563\\
4.20499992370605	292.368560791016\\
4.21000003814697	311.120269775391\\
4.21500015258789	328.684509277344\\
4.21999979019165	340.260589599609\\
4.22499990463257	342.410919189453\\
4.23000001907349	337.328125\\
4.2350001335144	324.913970947266\\
4.23999977111816	301.547912597656\\
4.24499988555908	268.183715820313\\
4.25	231.28450012207\\
4.25500011444092	193.524826049805\\
4.26000022888184	159.991180419922\\
4.2649998664856	134.11393737793\\
4.26999998092651	117.400039672852\\
4.27500009536743	109.679107666016\\
4.28000020980835	110.867164611816\\
4.28499984741211	111.867980957031\\
4.28999996185303	110.231956481934\\
4.29500007629395	105.295600891113\\
4.30000019073486	98.8982925415039\\
4.30499982833862	90.5534820556641\\
4.30999994277954	77.5890045166016\\
4.31500005722046	62.8579216003418\\
4.32000017166138	49.737735748291\\
4.32499980926514	40.0819511413574\\
4.32999992370605	34.9606246948242\\
4.33500003814697	34.8547325134277\\
4.34000015258789	37.7982406616211\\
4.34499979019165	42.2776069641113\\
4.34999990463257	45.8519897460938\\
4.35500001907349	45.7499847412109\\
4.3600001335144	39.7116508483887\\
4.36499977111816	27.5075912475586\\
4.36999988555908	17.9860610961914\\
4.375	22.6482486724854\\
4.38000011444092	26.5124740600586\\
4.38500022888184	29.0138893127441\\
4.3899998664856	30.1168537139893\\
4.39499998092651	102.447448730469\\
4.40000009536743	176.54817199707\\
4.40500020980835	248.600723266602\\
4.40999984741211	310.420013427734\\
4.41499996185303	356.309356689453\\
4.42000007629395	384.414764404297\\
4.42500019073486	395.693664550781\\
4.42999982833862	402.540496826172\\
4.43499994277954	394.277160644531\\
4.44000005722046	371.044677734375\\
4.44500017166138	339.197174072266\\
4.44999980926514	307.311553955078\\
4.45499992370605	283.979431152344\\
4.46000003814697	275.610595703125\\
4.46500015258789	292.746887207031\\
4.46999979019165	319.512786865234\\
4.47499990463257	344.446960449219\\
4.48000001907349	361.791870117188\\
4.4850001335144	366.702301025391\\
4.48999977111816	359.838134765625\\
4.49499988555908	346.457946777344\\
4.5	318.3408203125\\
4.50500011444092	278.359558105469\\
4.51000022888184	231.810302734375\\
4.5149998664856	184.896041870117\\
4.51999998092651	143.415618896484\\
4.52500009536743	112.345893859863\\
4.53000020980835	93.1472702026367\\
4.53499984741211	86.0232772827148\\
4.53999996185303	90.7100601196289\\
4.54500007629395	94.569450378418\\
4.55000019073486	95.3938674926758\\
4.55499982833862	92.4172210693359\\
4.55999994277954	87.1980667114258\\
4.56500005722046	77.1299896240234\\
4.57000017166138	62.7049903869629\\
4.57499980926514	47.4546775817871\\
4.57999992370605	34.1531753540039\\
4.58500003814697	24.0710716247559\\
4.59000015258789	18.74196434021\\
4.59499979019165	18.9871234893799\\
4.59999990463257	23.6764621734619\\
4.60500001907349	31.283411026001\\
4.6100001335144	38.2868194580078\\
4.61499977111816	39.0984039306641\\
4.61999988555908	29.2156238555908\\
4.625	15.9044342041016\\
4.63000011444092	27.0273742675781\\
4.63500022888184	35.284969329834\\
4.6399998664856	39.8284797668457\\
4.64499998092651	41.1792030334473\\
4.65000009536743	41.0564918518066\\
4.65500020980835	39.9752388000488\\
4.65999984741211	37.368953704834\\
4.66499996185303	266.228546142578\\
4.67000007629395	380.182861328125\\
4.67500019073486	456.426208496094\\
4.67999982833862	494.932342529297\\
4.68499994277954	500.949737548828\\
4.69000005722046	504.085021972656\\
4.69500017166138	472.275421142578\\
4.69999980926514	408.271057128906\\
4.70499992370605	324.433044433594\\
4.71000003814697	241.131729125977\\
4.71500015258789	180.085815429688\\
4.71999979019165	156.458602905273\\
4.72499990463257	198.494277954102\\
4.73000001907349	259.729064941406\\
4.7350001335144	320.764251708984\\
4.73999977111816	365.496459960938\\
4.74499988555908	384.703552246094\\
4.75	375.578674316406\\
4.75500011444092	357.944122314453\\
4.76000022888184	314.171203613281\\
4.7649998664856	249.163040161133\\
4.76999998092651	173.0859375\\
4.77500009536743	98.9974136352539\\
4.78000020980835	38.891414642334\\
4.78499984741211	9.73081588745117\\
4.78999996185303	8.25672245025635\\
4.79500007629395	6.3941798210144\\
4.80000019073486	48.1665725708008\\
4.80499982833862	75.172119140625\\
4.80999994277954	82.1851348876953\\
4.81500005722046	67.2914733886719\\
4.82000017166138	46.776683807373\\
4.82499980926514	30.6417007446289\\
4.82999992370605	24.5009136199951\\
4.83500003814697	28.8527412414551\\
4.84000015258789	41.7755393981934\\
4.84499979019165	58.3023910522461\\
4.84999990463257	70.6645889282227\\
4.85500001907349	67.8956756591797\\
4.8600001335144	41.6819038391113\\
4.86499977111816	3.50963258743286\\
4.86999988555908	8.72592735290527\\
4.875	19.2996692657471\\
4.88000011444092	30.2236099243164\\
4.88500022888184	38.0127334594727\\
4.8899998664856	42.115837097168\\
4.89499998092651	42.9221725463867\\
4.90000009536743	42.9892883300781\\
4.90500020980835	41.4789276123047\\
4.90999984741211	38.4977531433105\\
4.91499996185303	34.3838844299316\\
4.92000007629395	264.081512451172\\
4.92500019073486	404.044219970703\\
4.92999982833862	479.460296630859\\
4.93499994277954	502.226226806641\\
4.94000005722046	490.668395996094\\
4.94500017166138	477.807983398438\\
4.94999980926514	430.759033203125\\
4.95499992370605	361.606994628906\\
4.96000003814697	286.333221435547\\
4.96500015258789	222.673965454102\\
4.96999979019165	184.696807861328\\
4.97499990463257	187.635223388672\\
4.98000001907349	223.591598510742\\
4.9850001335144	264.617095947266\\
4.98999977111816	300.175506591797\\
4.99499988555908	323.075500488281\\
5	330.795684814453\\
5.00500011444092	323.479064941406\\
5.01000022888184	308.6064453125\\
5.0149998664856	287.122528076172\\
5.01999998092651	256.564300537109\\
5.02500009536743	220.436553955078\\
5.03000020980835	182.640975952148\\
5.03499984741211	146.462509155273\\
5.03999996185303	114.509689331055\\
5.04500007629395	88.2855529785156\\
5.05000019073486	68.5273742675781\\
5.05499982833862	55.0981407165527\\
5.05999994277954	47.1034240722656\\
5.06500005722046	43.0321388244629\\
5.07000017166138	41.0981369018555\\
5.07499980926514	39.5217552185059\\
5.07999992370605	36.739818572998\\
5.08500003814697	32.1769943237305\\
5.09000015258789	24.8106555938721\\
5.09499979019165	14.6214590072632\\
5.09999990463257	2.60986280441284\\
5.10500001907349	-0.315004736185074\\
5.1100001335144	-0.288657367229462\\
5.11499977111816	-0.250287264585495\\
5.11999988555908	-0.20632740855217\\
5.125	-0.160908922553062\\
5.13000011444092	-0.121495857834816\\
5.13500022888184	9.23101329803467\\
5.1399998664856	17.3196296691895\\
5.14499998092651	21.9304885864258\\
5.15000009536743	23.9719581604004\\
5.15500020980835	24.098819732666\\
5.15999984741211	125.03409576416\\
5.16499996185303	194.907958984375\\
5.17000007629395	247.96614074707\\
5.17500019073486	281.864715576172\\
5.17999982833862	297.305389404297\\
5.18499994277954	298.226196289063\\
5.19000005722046	291.115905761719\\
5.19500017166138	279.9345703125\\
5.19999980926514	261.224975585938\\
5.20499992370605	239.421752929688\\
5.21000003814697	218.659240722656\\
5.21500015258789	202.149063110352\\
5.21999979019165	191.712387084961\\
5.22499990463257	189.400054931641\\
5.23000001907349	193.55290222168\\
5.2350001335144	199.318664550781\\
5.23999977111816	204.998870849609\\
5.24499988555908	209.472686767578\\
5.25	211.905899047852\\
5.25500011444092	211.951263427734\\
5.26000022888184	211.859481811523\\
5.2649998664856	209.177429199219\\
5.26999998092651	203.336807250977\\
5.27500009536743	194.377593994141\\
5.28000020980835	182.718063354492\\
5.28499984741211	169.220138549805\\
5.28999996185303	155.050323486328\\
5.29500007629395	141.555068969727\\
5.30000019073486	130.2998046875\\
5.30499982833862	122.449928283691\\
5.30999994277954	119.247909545898\\
5.31500005722046	119.846626281738\\
5.32000017166138	128.466751098633\\
5.32499980926514	137.951446533203\\
5.32999992370605	146.794815063477\\
5.33500003814697	153.419952392578\\
5.34000015258789	156.765350341797\\
5.34499979019165	158.286605834961\\
5.34999990463257	154.658996582031\\
5.35500001907349	145.503143310547\\
5.3600001335144	131.455459594727\\
5.36499977111816	114.205307006836\\
5.36999988555908	95.7335891723633\\
5.375	79.3859710693359\\
5.38000011444092	66.5707702636719\\
5.38500022888184	58.9736595153809\\
5.3899998664856	57.0937881469727\\
5.39499998092651	60.8976898193359\\
5.40000009536743	63.6553192138672\\
5.40500020980835	63.0057640075684\\
5.40999984741211	60.5737113952637\\
5.41499996185303	52.6996765136719\\
5.42000007629395	38.7300262451172\\
5.42500019073486	19.509464263916\\
5.42999982833862	2.40218997001648\\
5.43499994277954	1.91859352588654\\
5.44000005722046	1.41803848743439\\
5.44500017166138	1.14787900447845\\
5.44999980926514	1.08561456203461\\
5.45499992370605	1.03774189949036\\
5.46000003814697	1.00113129615784\\
5.46500015258789	0.974171578884125\\
5.46999979019165	0.948079347610474\\
5.47499990463257	0.940666973590851\\
5.48000001907349	0.903235018253326\\
5.4850001335144	0.904013156890869\\
5.48999977111816	0.883900582790375\\
5.49499988555908	0.876433253288269\\
5.5	0.864398181438446\\
5.50500011444092	0.860158145427704\\
5.51000022888184	0.861392676830292\\
5.5149998664856	0.850037753582001\\
5.51999998092651	1.44086730480194\\
5.52500009536743	3.18753671646118\\
5.53000020980835	4.17417812347412\\
5.53499984741211	5.02417802810669\\
5.53999996185303	5.71588897705078\\
5.54500007629395	6.26162147521973\\
5.55000019073486	6.66233682632446\\
5.55499982833862	6.9447135925293\\
5.55999994277954	7.16957569122314\\
5.56500005722046	7.31390285491943\\
5.57000017166138	7.33972501754761\\
5.57499980926514	7.27773284912109\\
5.57999992370605	7.1020622253418\\
5.58500003814697	6.85220193862915\\
5.59000015258789	6.5023627281189\\
5.59499979019165	6.15174722671509\\
5.59999990463257	5.73155689239502\\
5.60500001907349	5.32458162307739\\
5.6100001335144	4.97348022460938\\
5.61499977111816	4.65623188018799\\
5.61999988555908	4.37753009796143\\
5.625	4.18023109436035\\
5.63000011444092	4.05389213562012\\
5.63500022888184	3.92036390304565\\
5.6399998664856	3.87912154197693\\
5.64499998092651	3.86516547203064\\
5.65000009536743	3.848313331604\\
5.65500020980835	3.83613610267639\\
5.65999984741211	3.82295155525208\\
5.66499996185303	3.81722831726074\\
5.67000007629395	8.42273807525635\\
5.67500019073486	108.884826660156\\
5.67999982833862	131.818023681641\\
5.68499994277954	133.046981811523\\
5.69000005722046	127.265655517578\\
5.69500017166138	116.336921691895\\
5.69999980926514	100.558929443359\\
5.70499992370605	85.6963043212891\\
5.71000003814697	75.7878646850586\\
5.71500015258789	74.6244964599609\\
5.71999979019165	80.0309143066406\\
5.72499990463257	86.1121826171875\\
5.73000001907349	90.2329635620117\\
5.7350001335144	91.1660919189453\\
5.73999977111816	88.7475204467773\\
5.74499988555908	83.4331130981445\\
5.75	76.2499694824219\\
5.75500011444092	68.3889694213867\\
5.76000022888184	60.6371078491211\\
5.7649998664856	53.1919822692871\\
5.76999998092651	45.3028602600098\\
5.77500009536743	36.6014556884766\\
5.78000020980835	25.9125194549561\\
5.78499984741211	12.8845205307007\\
5.78999996185303	-2.59577393531799\\
5.79500007629395	-20.7190589904785\\
5.80000019073486	-41.1919364929199\\
5.80499982833862	-63.3027305603027\\
5.80999994277954	-86.3926086425781\\
5.81500005722046	-109.437438964844\\
5.82000017166138	-131.168075561523\\
5.82499980926514	-150.029663085938\\
5.82999992370605	-164.598190307617\\
5.83500003814697	-173.135314941406\\
5.84000015258789	-174.089019775391\\
5.84499979019165	-166.25\\
5.84999990463257	-149.206817626953\\
5.85500001907349	-124.481506347656\\
5.8600001335144	-90.577033996582\\
5.86499977111816	-50.7368431091309\\
5.86999988555908	-9.68663597106934\\
5.875	24.5480117797852\\
5.88000011444092	43.5589561462402\\
5.88500022888184	40.639533996582\\
5.8899998664856	19.8940353393555\\
5.89499998092651	19.9154376983643\\
5.90000009536743	20.5759658813477\\
5.90500020980835	22.1011772155762\\
5.90999984741211	25.5714302062988\\
5.91499996185303	31.0854110717773\\
5.92000007629395	37.2313995361328\\
5.92500019073486	42.5552635192871\\
5.92999982833862	46.2691535949707\\
5.93499994277954	48.0176239013672\\
5.94000005722046	48.1363792419434\\
5.94500017166138	47.2665061950684\\
5.94999980926514	46.1516227722168\\
5.95499992370605	45.607063293457\\
5.96000003814697	46.889835357666\\
5.96500015258789	49.8283920288086\\
5.96999979019165	54.0823745727539\\
5.97499990463257	59.2559280395508\\
5.98000001907349	476.367401123047\\
5.9850001335144	760.791809082031\\
5.98999977111816	913.1806640625\\
5.99499988555908	956.175048828125\\
6	940.124389648438\\
6.00500011444092	917.226013183594\\
6.01000022888184	886.286376953125\\
6.0149998664856	814.450378417969\\
6.01999998092651	701.931213378906\\
6.02500009536743	563.838562011719\\
6.03000020980835	425.777801513672\\
6.03499984741211	317.043365478516\\
6.03999996185303	264.754547119141\\
6.04500007629395	307.075836181641\\
6.05000019073486	399.521820068359\\
6.05499982833862	500.212249755859\\
6.05999994277954	587.75390625\\
6.06500005722046	647.788757324219\\
6.07000017166138	676.229248046875\\
6.07499980926514	673.713989257813\\
6.07999992370605	651.992309570313\\
6.08500003814697	624.95068359375\\
6.09000015258789	575.867370605469\\
6.09499979019165	509.698333740234\\
6.09999990463257	433.058624267578\\
6.10500001907349	354.100555419922\\
6.1100001335144	283.235076904297\\
6.11499977111816	227.258651733398\\
6.11999988555908	191.658111572266\\
6.125	177.2529296875\\
6.13000011444092	189.721908569336\\
6.13500022888184	207.776718139648\\
6.1399998664856	222.440628051758\\
6.14499998092651	228.95295715332\\
6.15000009536743	224.937133789063\\
6.15500020980835	215.006927490234\\
6.15999984741211	197.365707397461\\
6.16499996185303	169.445205688477\\
6.17000007629395	133.205490112305\\
6.17500019073486	91.8606033325195\\
6.17999982833862	49.3978958129883\\
6.18499994277954	10.1494178771973\\
6.19000005722046	4.19211673736572\\
6.19500017166138	3.67328977584839\\
6.19999980926514	2.83171725273132\\
6.20499992370605	1.75673162937164\\
6.21000003814697	0.561404407024384\\
6.21500015258789	-0.0483319871127605\\
6.21999979019165	-0.105352811515331\\
6.22499990463257	-0.140487089753151\\
6.23000001907349	-0.158906400203705\\
6.2350001335144	-0.165811866521835\\
6.23999977111816	-0.164734423160553\\
6.24499988555908	-0.159442394971848\\
6.25	-0.152413591742516\\
6.25500011444092	-0.143329650163651\\
6.26000022888184	-0.133984833955765\\
6.2649998664856	-0.123582690954208\\
6.26999998092651	-0.113170750439167\\
6.27500009536743	-0.103666186332703\\
6.28000020980835	-0.0947950407862663\\
6.28499984741211	-0.0862003266811371\\
6.28999996185303	-0.0777106881141663\\
6.29500007629395	-0.0697755739092827\\
6.30000019073486	-0.0628884434700012\\
6.30499982833862	-0.0568633340299129\\
6.30999994277954	-0.0511793978512287\\
6.31500005722046	-0.0459954217076302\\
6.32000017166138	-0.0416448377072811\\
6.32499980926514	-0.0380459688603878\\
6.32999992370605	-0.0348630547523499\\
6.33500003814697	-0.0313087441027164\\
6.34000015258789	-0.0270224492996931\\
6.34499979019165	-0.0232679452747107\\
6.34999990463257	-0.0201354827731848\\
6.35500001907349	-0.0179150626063347\\
6.3600001335144	48.4339599609375\\
6.36499977111816	68.6913223266602\\
6.36999988555908	81.7798690795898\\
6.375	88.1875381469727\\
6.38000011444092	88.7521362304688\\
6.38500022888184	86.964469909668\\
6.3899998664856	88.8522415161133\\
6.39499998092651	85.0227813720703\\
6.40000009536743	82.0942230224609\\
6.40500020980835	84.4209671020508\\
6.40999984741211	97.1643447875977\\
6.41499996185303	115.381713867188\\
6.42000007629395	137.724838256836\\
6.42500019073486	162.122451782227\\
6.42999982833862	186.366592407227\\
6.43499994277954	208.980133056641\\
6.44000005722046	228.608474731445\\
6.44500017166138	244.714813232422\\
6.44999980926514	257.045227050781\\
6.45499992370605	266.140899658203\\
6.46000003814697	272.421630859375\\
6.46500015258789	276.593505859375\\
6.46999979019165	279.541717529297\\
6.47499990463257	281.872375488281\\
6.48000001907349	284.118286132813\\
6.4850001335144	286.774810791016\\
6.48999977111816	290.025787353516\\
6.49499988555908	293.872283935547\\
6.5	298.181274414063\\
6.50500011444092	302.699859619141\\
6.51000022888184	307.065093994141\\
6.5149998664856	310.968109130859\\
6.51999998092651	314.120422363281\\
6.52500009536743	316.2841796875\\
6.53000020980835	317.253082275391\\
6.53499984741211	317.182312011719\\
6.53999996185303	316.005920410156\\
6.54500007629395	313.820953369141\\
6.55000019073486	310.788269042969\\
6.55499982833862	307.163940429688\\
6.55999994277954	303.238128662109\\
6.56500005722046	298.819305419922\\
6.57000017166138	293.930969238281\\
6.57499980926514	288.641784667969\\
6.57999992370605	283.374694824219\\
6.58500003814697	277.939392089844\\
6.59000015258789	272.459411621094\\
6.59499979019165	267.006896972656\\
6.59999990463257	261.508178710938\\
6.60500001907349	255.93391418457\\
6.6100001335144	250.311462402344\\
6.61499977111816	244.669967651367\\
6.61999988555908	238.833129882813\\
6.625	232.944961547852\\
6.63000011444092	226.951553344727\\
6.63500022888184	220.898223876953\\
6.6399998664856	214.932922363281\\
6.64499998092651	208.951416015625\\
6.65000009536743	203.077926635742\\
6.65500020980835	197.387054443359\\
6.65999984741211	191.829879760742\\
6.66499996185303	186.505142211914\\
6.67000007629395	181.579193115234\\
6.67500019073486	176.939514160156\\
6.67999982833862	172.757705688477\\
6.68499994277954	169.017562866211\\
6.69000005722046	165.611221313477\\
6.69500017166138	162.902770996094\\
6.69999980926514	161.014556884766\\
6.70499992370605	159.584686279297\\
6.71000003814697	158.6025390625\\
6.71500015258789	157.935546875\\
6.71999979019165	157.648742675781\\
6.72499990463257	157.725997924805\\
6.73000001907349	158.051467895508\\
6.7350001335144	158.677917480469\\
6.73999977111816	159.603485107422\\
6.74499988555908	160.79606628418\\
6.75	162.323974609375\\
6.75500011444092	164.097915649414\\
6.76000022888184	165.936172485352\\
6.7649998664856	168.182052612305\\
6.76999998092651	170.61149597168\\
6.77500009536743	173.31298828125\\
6.78000020980835	176.27099609375\\
6.78499984741211	179.402114868164\\
6.78999996185303	182.728713989258\\
6.79500007629395	186.200775146484\\
6.80000019073486	189.766845703125\\
6.80499982833862	193.390380859375\\
6.80999994277954	197.093643188477\\
6.81500005722046	200.750640869141\\
6.82000017166138	204.40657043457\\
6.82499980926514	208.008605957031\\
6.82999992370605	211.517364501953\\
6.83500003814697	214.899688720703\\
6.84000015258789	218.126602172852\\
6.84499979019165	221.275985717773\\
6.84999990463257	224.265594482422\\
6.85500001907349	227.079833984375\\
6.8600001335144	229.703964233398\\
6.86499977111816	232.029678344727\\
6.86999988555908	234.080795288086\\
6.875	235.812622070313\\
6.88000011444092	237.126571655273\\
6.88500022888184	237.98063659668\\
6.8899998664856	238.308471679688\\
6.89499998092651	238.066970825195\\
6.90000009536743	237.236968994141\\
6.90500020980835	235.817535400391\\
6.90999984741211	233.806335449219\\
6.91499996185303	231.247619628906\\
6.92000007629395	228.190444946289\\
6.92500019073486	224.691040039063\\
6.92999982833862	220.819915771484\\
6.93499994277954	216.649887084961\\
6.94000005722046	212.233688354492\\
6.94500017166138	207.608871459961\\
6.94999980926514	202.804092407227\\
6.95499992370605	197.81982421875\\
6.96000003814697	192.685317993164\\
6.96500015258789	187.414794921875\\
6.96999979019165	181.980728149414\\
6.97499990463257	176.347961425781\\
6.98000001907349	170.563629150391\\
6.9850001335144	164.597213745117\\
6.98999977111816	158.457641601563\\
6.99499988555908	152.194030761719\\
7	145.849761962891\\
7.00500011444092	139.46418762207\\
7.01000022888184	133.091720581055\\
7.0149998664856	126.79679107666\\
7.01999998092651	120.628196716309\\
7.02500009536743	114.616676330566\\
7.03000020980835	108.788452148438\\
7.03499984741211	103.167106628418\\
7.03999996185303	97.7684555053711\\
7.04500007629395	92.5998153686523\\
7.05000019073486	87.6615829467773\\
7.05499982833862	82.9475555419922\\
7.05999994277954	78.4457855224609\\
7.06500005722046	74.1484298706055\\
7.07000017166138	70.0484848022461\\
7.07499980926514	66.144775390625\\
7.07999992370605	62.4408493041992\\
7.08500003814697	58.947338104248\\
7.09000015258789	55.6816825866699\\
7.09499979019165	52.6595840454102\\
7.09999990463257	49.9011154174805\\
7.10500001907349	47.4253082275391\\
7.1100001335144	45.2497711181641\\
7.11499977111816	43.3811683654785\\
7.11999988555908	41.8315200805664\\
7.125	40.6060943603516\\
7.13000011444092	39.6675834655762\\
7.13500022888184	39.0082855224609\\
7.1399998664856	38.6156845092773\\
7.14499998092651	38.4542770385742\\
7.15000009536743	38.5246963500977\\
7.15500020980835	38.8852005004883\\
7.15999984741211	39.3950347900391\\
7.16499996185303	39.983211517334\\
7.17000007629395	40.757381439209\\
7.17500019073486	41.6884155273438\\
7.17999982833862	42.7767562866211\\
7.18499994277954	44.0389976501465\\
7.19000005722046	45.4634971618652\\
7.19500017166138	47.0481643676758\\
7.19999980926514	48.7838859558105\\
7.20499992370605	50.6364288330078\\
7.21000003814697	52.6104698181152\\
7.21500015258789	54.6754760742188\\
7.21999979019165	56.7843704223633\\
7.22499990463257	58.9538154602051\\
7.23000001907349	61.1586418151855\\
7.2350001335144	63.3831024169922\\
7.23999977111816	65.6245269775391\\
7.24499988555908	67.8619079589844\\
7.25	70.0848922729492\\
7.25500011444092	72.2975311279297\\
7.26000022888184	74.4739456176758\\
7.2649998664856	76.6178512573242\\
7.26999998092651	78.7344436645508\\
7.27500009536743	80.7988357543945\\
7.28000020980835	82.797119140625\\
7.28499984741211	84.7305603027344\\
7.28999996185303	86.5858993530273\\
7.29500007629395	88.3665237426758\\
7.30000019073486	90.037841796875\\
7.30499982833862	91.6108093261719\\
7.30999994277954	93.0749893188477\\
7.31500005722046	94.4325408935547\\
7.32000017166138	95.6681518554688\\
7.32499980926514	96.7859573364258\\
7.32999992370605	97.773078918457\\
7.33500003814697	98.6277618408203\\
7.34000015258789	99.3520202636719\\
7.34499979019165	99.9427871704102\\
7.34999990463257	100.401672363281\\
7.35500001907349	100.729957580566\\
7.3600001335144	100.939086914063\\
7.36499977111816	101.031341552734\\
7.36999988555908	101.013664245605\\
7.375	100.894508361816\\
7.38000011444092	100.678894042969\\
7.38500022888184	100.339729309082\\
7.3899998664856	99.8604583740234\\
7.39499998092651	99.2475814819336\\
7.40000009536743	98.5346069335938\\
7.40500020980835	97.7135391235352\\
7.40999984741211	96.7948532104492\\
7.41499996185303	95.7850799560547\\
7.42000007629395	94.692985534668\\
7.42500019073486	93.5287551879883\\
7.42999982833862	92.2984313964844\\
7.43499994277954	91.0040893554688\\
7.44000005722046	89.6535797119141\\
7.44500017166138	88.2558135986328\\
7.44999980926514	86.8171005249023\\
7.45499992370605	85.3490829467773\\
7.46000003814697	83.8643264770508\\
7.46500015258789	82.3708038330078\\
7.46999979019165	80.870849609375\\
7.47499990463257	79.3727188110352\\
7.48000001907349	77.8824157714844\\
7.4850001335144	76.4047317504883\\
7.48999977111816	74.9452743530273\\
7.49499988555908	73.5085296630859\\
7.5	72.0976715087891\\
7.50500011444092	70.7149047851563\\
7.51000022888184	69.3647613525391\\
7.5149998664856	68.0590972900391\\
7.51999998092651	66.8051147460938\\
7.52500009536743	65.6245651245117\\
7.53000020980835	64.4814987182617\\
7.53499984741211	63.4151077270508\\
7.53999996185303	62.4089813232422\\
7.54500007629395	61.4832382202148\\
7.55000019073486	60.6312484741211\\
7.55499982833862	59.8519554138184\\
7.55999994277954	59.149242401123\\
7.56500005722046	58.5219383239746\\
7.57000017166138	57.9667358398438\\
7.57499980926514	57.4840202331543\\
7.57999992370605	57.0736083984375\\
7.58500003814697	56.7346839904785\\
7.59000015258789	56.4684944152832\\
7.59499979019165	56.2747383117676\\
7.59999990463257	56.1517944335938\\
7.60500001907349	56.1019287109375\\
7.6100001335144	56.138744354248\\
7.61499977111816	56.2765884399414\\
7.61999988555908	56.4892616271973\\
7.625	56.7458534240723\\
7.63000011444092	57.0548477172852\\
7.63500022888184	57.4193115234375\\
7.6399998664856	57.836841583252\\
7.64499998092651	58.3069915771484\\
7.65000009536743	58.818431854248\\
7.65500020980835	59.3679351806641\\
7.65999984741211	59.9545021057129\\
7.66499996185303	60.5734596252441\\
7.67000007629395	61.2199859619141\\
7.67500019073486	61.892391204834\\
7.67999982833862	62.5911521911621\\
7.68499994277954	63.3135604858398\\
7.69000005722046	64.0574111938477\\
7.69500017166138	64.8125\\
7.69999980926514	65.5752868652344\\
7.70499992370605	66.3443603515625\\
7.71000003814697	67.1189193725586\\
7.71500015258789	67.9044342041016\\
7.71999979019165	68.689338684082\\
7.72499990463257	69.4609603881836\\
7.73000001907349	70.2180023193359\\
7.7350001335144	70.9559173583984\\
7.73999977111816	71.6854476928711\\
7.74499988555908	72.3982238769531\\
7.75	73.078010559082\\
7.75500011444092	73.7332992553711\\
7.76000022888184	74.3673553466797\\
7.7649998664856	74.9720077514648\\
7.76999998092651	75.5449523925781\\
7.77500009536743	76.0877838134766\\
7.78000020980835	76.6001892089844\\
7.78499984741211	77.0795745849609\\
7.78999996185303	77.5275421142578\\
7.79500007629395	77.9440841674805\\
7.80000019073486	78.3256378173828\\
7.80499982833862	78.6735763549805\\
7.80999994277954	78.9881973266602\\
7.81500005722046	79.2679214477539\\
7.82000017166138	79.5125579833984\\
7.82499980926514	79.7226867675781\\
7.82999992370605	79.8997268676758\\
7.83500003814697	80.0452728271484\\
7.84000015258789	80.1580505371094\\
7.84499979019165	80.2388153076172\\
7.84999990463257	80.2914886474609\\
7.85500001907349	80.3153457641602\\
7.8600001335144	80.3112945556641\\
7.86499977111816	80.2817840576172\\
7.86999988555908	80.2274398803711\\
7.875	80.1494293212891\\
7.88000011444092	80.052734375\\
7.88500022888184	79.9396896362305\\
7.8899998664856	79.8117523193359\\
7.89499998092651	79.6695098876953\\
7.90000009536743	79.5132751464844\\
7.90500020980835	79.3454895019531\\
7.90999984741211	79.1670913696289\\
7.91499996185303	78.9776916503906\\
7.92000007629395	78.7791290283203\\
7.92500019073486	78.5732803344727\\
7.92999982833862	78.3736877441406\\
7.93499994277954	78.1726837158203\\
7.94000005722046	77.9702682495117\\
7.94500017166138	77.7808074951172\\
7.94999980926514	77.6003036499023\\
7.95499992370605	77.4267501831055\\
7.96000003814697	77.2685470581055\\
7.96500015258789	77.1283111572266\\
7.96999979019165	77.0021286010742\\
7.97499990463257	76.8956832885742\\
7.98000001907349	76.8179321289063\\
7.9850001335144	76.7623748779297\\
7.98999977111816	76.7293014526367\\
7.99499988555908	76.7206497192383\\
8	76.7353134155273\\
8.00500011444092	76.7732772827148\\
8.01000022888184	76.8431701660156\\
8.01500034332275	76.9443588256836\\
8.02000045776367	77.0782089233398\\
8.02499961853027	77.2438430786133\\
8.02999973297119	77.4418029785156\\
8.03499984741211	77.6738204956055\\
8.03999996185303	77.9399337768555\\
8.04500007629395	78.2396545410156\\
8.05000019073486	78.5741729736328\\
8.05500030517578	78.9435501098633\\
8.0600004196167	79.3462066650391\\
8.0649995803833	79.7830352783203\\
8.06999969482422	80.2540817260742\\
8.07499980926514	80.7620697021484\\
8.07999992370605	81.3065032958984\\
8.08500003814697	81.8879089355469\\
8.09000015258789	82.5024185180664\\
8.09500026702881	83.149528503418\\
8.10000038146973	83.8293914794922\\
8.10499954223633	84.5427017211914\\
8.10999965667725	85.2901992797852\\
8.11499977111816	86.0711288452148\\
8.11999988555908	86.8848724365234\\
8.125	87.7298431396484\\
8.13000011444092	88.6061859130859\\
8.13500022888184	89.5134506225586\\
8.14000034332275	90.4516830444336\\
8.14500045776367	91.420295715332\\
8.14999961853027	92.4189071655273\\
8.15499973297119	93.4452209472656\\
8.15999984741211	94.498779296875\\
8.16499996185303	95.5789489746094\\
8.17000007629395	96.6879043579102\\
8.17500019073486	97.8268356323242\\
8.18000030517578	98.9948806762695\\
8.1850004196167	100.190361022949\\
8.1899995803833	101.409736633301\\
8.19499969482422	102.653755187988\\
8.19999980926514	103.923377990723\\
8.20499992370605	105.220375061035\\
8.21000003814697	106.543724060059\\
8.21500015258789	107.889045715332\\
8.22000026702881	109.252143859863\\
8.22500038146973	110.632888793945\\
8.22999954223633	112.037300109863\\
8.23499965667725	113.468223571777\\
8.23999977111816	114.921173095703\\
8.24499988555908	116.393936157227\\
8.25	117.889633178711\\
8.25500011444092	119.410972595215\\
8.26000022888184	120.95671081543\\
8.26500034332275	122.527114868164\\
8.27000045776367	124.121681213379\\
8.27499961853027	125.739906311035\\
8.27999973297119	127.378311157227\\
8.28499984741211	129.037673950195\\
8.28999996185303	130.721420288086\\
8.29500007629395	132.418533325195\\
8.30000019073486	134.122787475586\\
8.30500030517578	135.831848144531\\
8.3100004196167	137.548583984375\\
8.3149995803833	139.365051269531\\
8.31999969482422	141.169631958008\\
8.32499980926514	142.995376586914\\
8.32999992370605	144.845840454102\\
8.33500003814697	146.715362548828\\
8.34000015258789	148.602874755859\\
8.34500026702881	150.507843017578\\
8.35000038146973	152.432098388672\\
8.35499954223633	154.352935791016\\
8.35999965667725	156.309326171875\\
8.36499977111816	158.29768371582\\
8.36999988555908	160.275268554688\\
8.375	162.255920410156\\
8.38000011444092	164.23762512207\\
8.38500022888184	166.211944580078\\
8.39000034332275	168.183486938477\\
8.39500045776367	170.130249023438\\
8.39999961853027	172.104690551758\\
8.40499973297119	174.052734375\\
8.40999984741211	175.975967407227\\
8.41499996185303	177.876556396484\\
8.42000007629395	179.747756958008\\
8.42500019073486	181.596710205078\\
8.43000030517578	183.411209106445\\
8.4350004196167	185.193099975586\\
8.4399995803833	186.945068359375\\
8.44499969482422	188.632476806641\\
8.44999980926514	190.252151489258\\
8.45499992370605	191.79035949707\\
8.46000003814697	193.214920043945\\
8.46500015258789	194.530975341797\\
8.47000026702881	195.685623168945\\
8.47500038146973	196.694595336914\\
8.47999954223633	197.561630249023\\
8.48499965667725	198.298126220703\\
8.48999977111816	198.866775512695\\
8.49499988555908	199.313385009766\\
8.5	199.650634765625\\
8.50500011444092	199.863037109375\\
8.51000022888184	200.064758300781\\
8.51500034332275	200.267044067383\\
8.52000045776367	200.464691162109\\
8.52499961853027	200.830429077148\\
8.52999973297119	201.236862182617\\
8.53499984741211	201.830032348633\\
8.53999996185303	202.632583618164\\
8.54500007629395	203.682952880859\\
8.55000019073486	205.028106689453\\
8.55500030517578	206.687194824219\\
8.5600004196167	208.766311645508\\
8.5649995803833	211.224136352539\\
8.56999969482422	214.074996948242\\
8.57499980926514	217.32585144043\\
8.57999992370605	220.838302612305\\
8.58500003814697	224.585021972656\\
8.59000015258789	228.466125488281\\
8.59500026702881	232.161163330078\\
8.60000038146973	235.5341796875\\
8.60499954223633	238.564682006836\\
8.60999965667725	241.185485839844\\
8.61499977111816	243.419403076172\\
8.61999988555908	245.325408935547\\
8.625	246.910263061523\\
8.63000011444092	248.250686645508\\
8.63500022888184	249.353927612305\\
8.64000034332275	250.301376342773\\
8.64500045776367	251.020172119141\\
8.64999961853027	251.721374511719\\
8.65499973297119	251.899810791016\\
8.65999984741211	252.618469238281\\
8.66499996185303	252.539657592773\\
8.67000007629395	252.237243652344\\
8.67500019073486	251.439041137695\\
8.68000030517578	250.374954223633\\
8.6850004196167	248.856658935547\\
8.6899995803833	246.801055908203\\
8.69499969482422	244.298477172852\\
8.69999980926514	241.476608276367\\
8.70499992370605	238.348785400391\\
8.71000003814697	234.895126342773\\
8.71500015258789	231.14289855957\\
8.72000026702881	227.428405761719\\
8.72500038146973	223.654739379883\\
8.72999954223633	219.866149902344\\
8.73499965667725	216.169357299805\\
8.73999977111816	212.617904663086\\
8.74499988555908	209.146301269531\\
8.75	205.699356079102\\
8.75500011444092	202.008712768555\\
8.76000022888184	198.505859375\\
8.76500034332275	194.834884643555\\
8.77000045776367	191.119445800781\\
8.77499961853027	187.312881469727\\
8.77999973297119	183.48503112793\\
8.78499984741211	179.599548339844\\
8.78999996185303	175.720230102539\\
8.79500007629395	171.910125732422\\
8.80000019073486	168.224502563477\\
8.80500030517578	164.712432861328\\
8.8100004196167	161.422866821289\\
8.8149995803833	158.397552490234\\
8.81999969482422	155.631698608398\\
8.82499980926514	153.13606262207\\
8.82999992370605	150.900665283203\\
8.83500003814697	148.871932983398\\
8.84000015258789	147.012573242188\\
8.84500026702881	145.304901123047\\
8.85000038146973	143.677978515625\\
8.85499954223633	142.160217285156\\
8.85999965667725	140.72265625\\
8.86499977111816	139.373672485352\\
8.86999988555908	138.128921508789\\
8.875	137.018707275391\\
8.88000011444092	136.077789306641\\
8.88500022888184	135.335113525391\\
8.89000034332275	134.815979003906\\
8.89500045776367	134.514083862305\\
8.89999961853027	134.439071655273\\
8.90499973297119	134.621459960938\\
8.90999984741211	134.973007202148\\
8.91499996185303	135.383651733398\\
8.92000007629395	135.928756713867\\
8.92500019073486	136.618835449219\\
8.93000030517578	137.423736572266\\
8.9350004196167	138.261947631836\\
8.9399995803833	138.987060546875\\
8.94499969482422	139.775299072266\\
8.94999980926514	140.646057128906\\
8.95499992370605	141.580047607422\\
8.96000003814697	142.592041015625\\
8.96500015258789	143.689315795898\\
8.97000026702881	144.894790649414\\
8.97500038146973	146.164886474609\\
8.97999954223633	147.483016967773\\
8.98499965667725	148.962219238281\\
8.98999977111816	150.426544189453\\
8.99499988555908	151.808898925781\\
9	153.294448852539\\
9.00500011444092	154.773101806641\\
9.01000022888184	156.221405029297\\
9.01500034332275	157.630950927734\\
9.02000045776367	159.008209228516\\
9.02499961853027	160.341781616211\\
9.02999973297119	161.624130249023\\
9.03499984741211	162.855270385742\\
9.03999996185303	164.03108215332\\
9.04500007629395	165.148803710938\\
9.05000019073486	166.205657958984\\
9.05500030517578	167.197448730469\\
9.0600004196167	168.117797851563\\
9.0649995803833	168.963439941406\\
9.06999969482422	169.728652954102\\
9.07499980926514	170.409118652344\\
9.07999992370605	171.00163269043\\
9.08500003814697	171.502853393555\\
9.09000015258789	171.903503417969\\
9.09500026702881	172.18278503418\\
9.10000038146973	172.3359375\\
9.10499954223633	172.423934936523\\
9.10999965667725	172.422653198242\\
9.11499977111816	172.303558349609\\
9.11999988555908	172.058990478516\\
9.125	171.701522827148\\
9.13000011444092	171.239730834961\\
9.13500022888184	170.686233520508\\
9.14000034332275	170.091827392578\\
9.14500045776367	169.442886352539\\
9.14999961853027	168.769500732422\\
9.15499973297119	168.073028564453\\
9.15999984741211	167.355026245117\\
9.16499996185303	166.611297607422\\
9.17000007629395	165.833084106445\\
9.17500019073486	165.017013549805\\
9.18000030517578	164.157165527344\\
9.1850004196167	163.25569152832\\
9.1899995803833	162.311111450195\\
9.19499969482422	161.321487426758\\
9.19999980926514	160.296768188477\\
9.20499992370605	159.248321533203\\
9.21000003814697	158.194229125977\\
9.21500015258789	157.148056030273\\
9.22000026702881	156.106292724609\\
9.22500038146973	155.082626342773\\
9.22999954223633	154.075454711914\\
9.23499965667725	153.08837890625\\
9.23999977111816	152.137588500977\\
9.24499988555908	151.257080078125\\
9.25	150.477676391602\\
9.25500011444092	149.834396362305\\
9.26000022888184	149.39143371582\\
9.26500034332275	149.24382019043\\
9.27000045776367	149.658584594727\\
9.27499961853027	150.544219970703\\
9.27999973297119	151.891632080078\\
9.28499984741211	153.710189819336\\
9.28999996185303	155.994857788086\\
9.29500007629395	158.727722167969\\
9.30000019073486	161.871688842773\\
9.30500030517578	165.465072631836\\
9.3100004196167	169.371459960938\\
9.3149995803833	173.579071044922\\
9.31999969482422	177.999740600586\\
9.32499980926514	182.532455444336\\
9.32999992370605	187.069793701172\\
9.33500003814697	191.425994873047\\
9.34000015258789	195.385162353516\\
9.34500026702881	198.848510742188\\
9.35000038146973	201.411911010742\\
9.35499954223633	202.815811157227\\
9.35999965667725	202.925277709961\\
9.36499977111816	201.458038330078\\
9.36999988555908	198.356155395508\\
9.375	193.827117919922\\
9.38000011444092	188.066635131836\\
9.38500022888184	181.404067993164\\
9.39000034332275	174.384033203125\\
9.39500045776367	167.248519897461\\
9.39999961853027	160.71842956543\\
9.40499973297119	155.056213378906\\
9.40999984741211	150.416946411133\\
9.41499996185303	146.72705078125\\
9.42000007629395	143.524566650391\\
9.42500019073486	140.067276000977\\
9.43000030517578	135.22932434082\\
9.4350004196167	127.676567077637\\
9.4399995803833	115.750984191895\\
9.44499969482422	97.9403991699219\\
9.44999980926514	72.702522277832\\
9.45499992370605	39.1442031860352\\
9.46000003814697	-2.43955445289612\\
9.46500015258789	-50.9143600463867\\
9.47000026702881	-102.740707397461\\
9.47500038146973	-153.678863525391\\
9.47999954223633	-198.554412841797\\
9.48499965667725	-230.644836425781\\
9.48999977111816	-241.606262207031\\
9.49499988555908	-241.685852050781\\
9.5	-227.452987670898\\
9.50500011444092	-201.974456787109\\
9.51000022888184	-169.214492797852\\
9.51500034332275	-132.908157348633\\
9.52000045776367	-96.8308029174805\\
9.52499961853027	-63.4038543701172\\
9.52999973297119	-36.3394660949707\\
9.53499984741211	-15.8066263198853\\
9.53999996185303	-3.1326916217804\\
9.54500007629395	1.35464763641357\\
9.55000019073486	-1.79051518440247\\
9.55500030517578	-1.3484719991684\\
9.5600004196167	-1.23700499534607\\
9.5649995803833	-1.12290644645691\\
9.56999969482422	-0.987666964530945\\
9.57499980926514	-0.841245412826538\\
9.57999992370605	-0.74242490530014\\
9.58500003814697	-0.597982466220856\\
9.59000015258789	-0.47397917509079\\
9.59500026702881	-0.374574065208435\\
9.60000038146973	-0.290358990430832\\
9.60499954223633	-0.206613704562187\\
9.60999965667725	-0.124602638185024\\
9.61499977111816	-0.0598373413085938\\
9.61999988555908	-0.000270512886345387\\
9.625	0.0480994060635567\\
9.63000011444092	0.0988239198923111\\
9.63500022888184	0.135431706905365\\
9.64000034332275	0.171691790223122\\
9.64500045776367	0.211730092763901\\
9.64999961853027	0.239277243614197\\
9.65499973297119	0.258926600217819\\
9.65999984741211	0.284768015146255\\
9.66499996185303	0.305459469556808\\
9.67000007629395	0.322387456893921\\
9.67500019073486	0.338313847780228\\
9.68000030517578	0.352400988340378\\
9.6850004196167	0.365059226751328\\
9.6899995803833	0.37596008181572\\
9.69499969482422	0.384878844022751\\
9.69999980926514	0.393013656139374\\
9.70499992370605	0.401162087917328\\
9.71000003814697	0.408085584640503\\
9.71500015258789	0.412478506565094\\
9.72000026702881	0.416557461023331\\
9.72500038146973	0.421744048595428\\
9.72999954223633	0.426762372255325\\
9.73499965667725	0.431613117456436\\
9.73999977111816	0.435913681983948\\
9.74499988555908	0.439018070697784\\
9.75	0.442071706056595\\
9.75500011444092	0.444036871194839\\
9.76000022888184	0.444481939077377\\
9.76500034332275	0.444396734237671\\
9.77000045776367	0.445190399885178\\
9.77499961853027	0.446731120347977\\
9.77999973297119	0.449579060077667\\
9.78499984741211	0.450949370861053\\
9.78999996185303	0.451823174953461\\
9.79500007629395	0.451838672161102\\
9.80000019073486	0.451670050621033\\
9.80500030517578	0.452192574739456\\
9.8100004196167	0.453167170286179\\
9.8149995803833	0.454828709363937\\
9.81999969482422	0.455675005912781\\
9.82499980926514	0.456137001514435\\
9.82999992370605	0.455936193466187\\
9.83500003814697	0.455537468194962\\
9.84000015258789	0.455504804849625\\
9.84500026702881	0.455656439065933\\
9.85000038146973	0.456152558326721\\
9.85499954223633	0.456454038619995\\
9.85999965667725	0.456872791051865\\
9.86499977111816	0.457439035177231\\
9.86999988555908	0.457760006189346\\
9.875	0.45752939581871\\
9.88000011444092	0.45683091878891\\
9.88500022888184	0.455704689025879\\
9.89000034332275	0.456065326929092\\
9.89500045776367	0.45730397105217\\
9.89999961853027	0.459676086902618\\
9.90499973297119	0.461231142282486\\
9.90999984741211	0.460579186677933\\
9.91499996185303	0.458225071430206\\
9.92000007629395	0.454185426235199\\
9.92500019073486	0.453592270612717\\
9.93000030517578	0.454615563154221\\
9.9350004196167	0.457819283008575\\
9.9399995803833	0.460287302732468\\
9.94499969482422	0.460875481367111\\
9.94999980926514	0.461125493049622\\
9.95499992370605	0.461037367582321\\
9.96000003814697	0.460611045360565\\
9.96500015258789	0.459846585988998\\
9.97000026702881	0.458743959665298\\
9.97500038146973	0.458205312490463\\
9.97999954223633	0.458175390958786\\
9.98499965667725	0.458185851573944\\
9.98999977111816	0.45823672413826\\
9.99499988555908	0.458328008651733\\
10	0.458459734916687\\
};
\addlegendentry{RS}

\addplot [color=red, line width=1.0pt]
  table[row sep=crcr]{%
0.0949999988079071	7.17677164077759\\
0.100000001490116	6.78368330001831\\
0.104999996721745	6.45834398269653\\
0.109999999403954	6.15628910064697\\
0.115000002086163	5.90958023071289\\
0.119999997317791	-0.154913529753685\\
0.125	18.925594329834\\
0.129999995231628	28.1416187286377\\
0.135000005364418	36.0824279785156\\
0.140000000596046	43.347541809082\\
0.144999995827675	50.751350402832\\
0.150000005960464	58.7220344543457\\
0.155000001192093	67.6747970581055\\
0.159999996423721	77.9475250244141\\
0.165000006556511	89.8627090454102\\
0.170000001788139	102.257034301758\\
0.174999997019768	114.167297363281\\
0.180000007152557	124.735473632813\\
0.185000002384186	133.337615966797\\
0.189999997615814	139.320587158203\\
0.194999992847443	313.982482910156\\
0.200000002980232	399.786865234375\\
0.204999998211861	434.257904052734\\
0.209999993443489	438.749938964844\\
0.215000003576279	427.238983154297\\
0.219999998807907	424.73876953125\\
0.224999994039536	403.132598876953\\
0.230000004172325	365.589141845703\\
0.234999999403954	321.205993652344\\
0.239999994635582	283.158081054688\\
0.245000004768372	264.949401855469\\
0.25	280.829498291016\\
0.254999995231628	347.970703125\\
0.259999990463257	436.407623291016\\
0.264999985694885	527.342346191406\\
0.270000010728836	608.812316894531\\
0.275000005960464	673.137084960938\\
0.280000001192093	716.871398925781\\
0.284999996423721	741.54248046875\\
0.28999999165535	750.276733398438\\
0.294999986886978	754.134704589844\\
0.300000011920929	753.805297851563\\
0.305000007152557	746.464294433594\\
0.310000002384186	738.572082519531\\
0.314999997615814	735.92578125\\
0.319999992847443	743.246948242188\\
0.324999988079071	763.210632324219\\
0.330000013113022	796.595275878906\\
0.33500000834465	841.382629394531\\
0.340000003576279	896.503173828125\\
0.344999998807907	948.678588867188\\
0.349999994039536	993.236694335938\\
0.354999989271164	1027.31896972656\\
0.360000014305115	1050.08825683594\\
0.365000009536743	1064.09558105469\\
0.370000004768372	1072.59240722656\\
0.375	1075.85607910156\\
0.379999995231628	1077.11584472656\\
0.384999990463257	1079.28576660156\\
0.389999985694885	1084.71411132813\\
0.395000010728836	1094.94140625\\
0.400000005960464	1110.82788085938\\
0.405000001192093	1130.6669921875\\
0.409999996423721	1154.47546386719\\
0.41499999165535	1179.52221679688\\
0.419999986886978	1203.80358886719\\
0.425000011920929	1226.03076171875\\
0.430000007152557	1241.7724609375\\
0.435000002384186	1250.85241699219\\
0.439999997615814	1254.7685546875\\
0.444999992847443	1252.58190917969\\
0.449999988079071	1245.79113769531\\
0.455000013113022	1236.37182617188\\
0.46000000834465	1226.60314941406\\
0.465000003576279	1218.72497558594\\
0.469999998807907	1214.53295898438\\
0.474999994039536	1214.94848632813\\
0.479999989271164	1219.87622070313\\
0.485000014305115	1228.45007324219\\
0.490000009536743	1239.21899414063\\
0.495000004768372	1251.1005859375\\
0.5	1261.30944824219\\
0.504999995231628	1268.70422363281\\
0.509999990463257	1273.24572753906\\
0.514999985694885	1275.32312011719\\
0.519999980926514	1275.65454101563\\
0.524999976158142	1275.41430664063\\
0.529999971389771	1276.01721191406\\
0.535000026226044	1276.60083007813\\
0.540000021457672	1278.08764648438\\
0.545000016689301	1281.29797363281\\
0.550000011920929	1286.60864257813\\
0.555000007152557	1294.88330078125\\
0.560000002384186	1305.22277832031\\
0.564999997615814	1317.15600585938\\
0.569999992847443	1330.23669433594\\
0.574999988079071	1344.07250976563\\
0.579999983310699	1358.26538085938\\
0.584999978542328	1372.8974609375\\
0.589999973773956	1388.03698730469\\
0.595000028610229	1403.96850585938\\
0.600000023841858	1420.97094726563\\
0.605000019073486	1439.24609375\\
0.610000014305115	1458.99084472656\\
0.615000009536743	1480.1513671875\\
0.620000004768372	1502.79992675781\\
0.625	1526.5712890625\\
0.629999995231628	1551.544921875\\
0.634999990463257	1577.3525390625\\
0.639999985694885	1603.87036132813\\
0.644999980926514	1630.9775390625\\
0.649999976158142	1658.68994140625\\
0.654999971389771	1686.86694335938\\
0.660000026226044	1715.71069335938\\
0.665000021457672	1745.16943359375\\
0.670000016689301	1775.65490722656\\
0.675000011920929	1807.25024414063\\
0.680000007152557	1840.35107421875\\
0.685000002384186	1875.09936523438\\
0.689999997615814	1911.60693359375\\
0.694999992847443	1949.82067871094\\
0.699999988079071	1989.56652832031\\
0.704999983310699	2030.51708984375\\
0.709999978542328	2072.31665039063\\
0.714999973773956	2114.3681640625\\
0.720000028610229	2156.41650390625\\
0.725000023841858	2198.04760742188\\
0.730000019073486	2239.04223632813\\
0.735000014305115	2279.30834960938\\
0.740000009536743	2319.13134765625\\
0.745000004768372	2358.60595703125\\
0.75	2398.26147460938\\
0.754999995231628	2438.51977539063\\
0.759999990463257	2479.76049804688\\
0.764999985694885	2522.36108398438\\
0.769999980926514	2566.46728515625\\
0.774999976158142	2611.96020507813\\
0.779999971389771	2658.37280273438\\
0.785000026226044	2704.95532226563\\
0.790000021457672	2751.294921875\\
0.795000016689301	2796.69409179688\\
0.800000011920929	2840.70849609375\\
0.805000007152557	2882.76391601563\\
0.810000002384186	2923.42895507813\\
0.814999997615814	2962.33276367188\\
0.819999992847443	3000.65795898438\\
0.824999988079071	3038.4306640625\\
0.829999983310699	3076.51440429688\\
0.834999978542328	3115.45581054688\\
0.839999973773956	3155.3837890625\\
0.845000028610229	3196.21362304688\\
0.850000023841858	3237.47094726563\\
0.855000019073486	3278.66845703125\\
0.860000014305115	3318.92626953125\\
0.865000009536743	3357.69946289063\\
0.870000004768372	3394.42749023438\\
0.875	3429.06518554688\\
0.879999995231628	3461.86499023438\\
0.884999990463257	3493.33935546875\\
0.889999985694885	3524.28051757813\\
0.894999980926514	3555.48559570313\\
0.899999976158142	3587.470703125\\
0.904999971389771	3620.58227539063\\
0.910000026226044	3654.74243164063\\
0.915000021457672	3689.26684570313\\
0.920000016689301	3723.71069335938\\
0.925000011920929	3757.31811523438\\
0.930000007152557	3789.33422851563\\
0.935000002384186	3819.59887695313\\
0.939999997615814	3848.09716796875\\
0.944999992847443	3875.37963867188\\
0.949999988079071	3902.26025390625\\
0.954999983310699	3929.58984375\\
0.959999978542328	3957.9873046875\\
0.964999973773956	3988.13598632813\\
0.970000028610229	4019.74462890625\\
0.975000023841858	4052.43774414063\\
0.980000019073486	4085.65869140625\\
0.985000014305115	4118.41015625\\
0.990000009536743	4150.03369140625\\
0.995000004768372	4180.28662109375\\
1	4209.2919921875\\
1.00499999523163	4237.6337890625\\
1.00999999046326	4266.17578125\\
1.01499998569489	4295.8349609375\\
1.01999998092651	4327.43212890625\\
1.02499997615814	4361.32080078125\\
1.02999997138977	4397.173828125\\
1.0349999666214	4434.43017578125\\
1.03999996185303	4472.26708984375\\
1.04499995708466	4509.45361328125\\
1.04999995231628	4545.40869140625\\
1.05499994754791	4580.2939453125\\
1.05999994277954	4614.3017578125\\
1.06500005722046	4648.47607421875\\
1.07000005245209	4683.89404296875\\
1.07500004768372	4721.52001953125\\
1.08000004291534	4761.80419921875\\
1.08500003814697	4804.6396484375\\
1.0900000333786	4849.28564453125\\
1.09500002861023	4894.4931640625\\
1.10000002384186	4939.10546875\\
1.10500001907349	4982.36572265625\\
1.11000001430511	5024.1552734375\\
1.11500000953674	5065.11279296875\\
1.12000000476837	5106.443359375\\
1.125	5149.37060546875\\
1.12999999523163	5195.07470703125\\
1.13499999046326	5243.72412109375\\
1.13999998569489	5295.0439453125\\
1.14499998092651	5347.5419921875\\
1.14999997615814	5399.896484375\\
1.15499997138977	5450.767578125\\
1.1599999666214	5499.5390625\\
1.16499996185303	5546.72021484375\\
1.16999995708466	5593.52197265625\\
1.17499995231628	5641.62158203125\\
1.17999994754791	5692.2890625\\
1.18499994277954	5746.0556640625\\
1.19000005722046	5802.50439453125\\
1.19500005245209	5860.34619140625\\
1.20000004768372	5917.58544921875\\
1.20500004291534	5972.876953125\\
1.21000003814697	6025.57373046875\\
1.2150000333786	6076.24462890625\\
1.22000002861023	6126.4013671875\\
1.22500002384186	6178.123046875\\
1.23000001907349	6232.4384765625\\
1.23500001430511	6290.14794921875\\
1.24000000953674	6349.8095703125\\
1.24500000476837	6410.05419921875\\
1.25	6468.65380859375\\
1.25499999523163	6524.248046875\\
1.25999999046326	6577.09326171875\\
1.26499998569489	6628.16455078125\\
1.26999998092651	6679.66552734375\\
1.27499997615814	6733.45263671875\\
1.27999997138977	6790.63818359375\\
1.2849999666214	6850.22265625\\
1.28999996185303	6910.5771484375\\
1.29499995708466	6969.3154296875\\
1.29999995231628	7024.95166015625\\
1.30499994754791	7077.302734375\\
1.30999994277954	7127.7666015625\\
1.31500005722046	7178.38720703125\\
1.32000005245209	7231.31884765625\\
1.32500004768372	7287.49951171875\\
1.33000004291534	7346.0146484375\\
1.33500003814697	7405.1240234375\\
1.3400000333786	7462.3095703125\\
1.34500002861023	7516.07275390625\\
1.35000002384186	7566.5322265625\\
1.35500001907349	7615.4208984375\\
1.36000001430511	7665.05029296875\\
1.36500000953674	7717.427734375\\
1.37000000476837	7773.04150390625\\
1.375	7830.7900390625\\
1.37999999523163	7888.03564453125\\
1.38499999046326	7942.5205078125\\
1.38999998569489	7993.5625\\
1.39499998092651	8041.822265625\\
1.39999997615814	8089.91064453125\\
1.40499997138977	8140.001953125\\
1.4099999666214	8193.677734375\\
1.41499996185303	8250.408203125\\
1.41999995708466	8307.9365234375\\
1.42499995231628	8363.3544921875\\
1.42999994754791	8415.125\\
1.43499994277954	8463.666015625\\
1.44000005722046	8511.6083984375\\
1.44500005245209	8561.009765625\\
1.45000004768372	8614.1298828125\\
1.45500004291534	8670.7685546875\\
1.46000003814697	8728.7890625\\
1.4650000333786	8785.17578125\\
1.47000002861023	8838.142578125\\
1.47500002384186	8887.7646484375\\
1.48000001907349	8936.3056640625\\
1.48500001430511	8986.7998046875\\
1.49000000953674	9041.3193359375\\
1.49500000476837	9099.6083984375\\
1.5	9159.5126953125\\
1.50499999523163	9217.5634765625\\
1.50999999046326	9272.1806640625\\
1.51499998569489	9323.341796875\\
1.51999998092651	9373.873046875\\
1.52499997615814	9426.94921875\\
1.52999997138977	9484.1455078125\\
1.5349999666214	9545.2568359375\\
1.53999996185303	9607.1142578125\\
1.54499995708466	9666.486328125\\
1.54999995231628	9721.8603515625\\
1.55499994754791	9774.2236328125\\
1.55999994277954	9826.8974609375\\
1.56500005722046	9882.8125\\
1.57000005245209	9941.486328125\\
1.57500004768372	10002.587890625\\
1.58000004291534	10064.08984375\\
1.58500003814697	10122.611328125\\
1.5900000333786	10177.375\\
1.59500002861023	10230.9677734375\\
1.60000002384186	10287.1376953125\\
1.60500001907349	10348.15625\\
1.61000001430511	10413.310546875\\
1.61500000953674	10479.5283203125\\
1.62000000476837	10542.8125\\
1.625	10601.0087890625\\
1.62999999523163	10655.505859375\\
1.63499999046326	10710.263671875\\
1.63999998569489	10768.7568359375\\
1.64499998092651	10832.015625\\
1.64999997615814	10897.529296875\\
1.65499997138977	10961.41015625\\
1.6599999666214	11020.5068359375\\
1.66499996185303	11075.0751953125\\
1.66999995708466	11129.1044921875\\
1.67499995231628	11186.8203125\\
1.67999994754791	11249.599609375\\
1.68499994277954	11315.1787109375\\
1.69000005722046	11379.3515625\\
1.69500005245209	11438.638671875\\
1.70000004768372	11492.9951171875\\
1.70500004291534	11545.8046875\\
1.71000003814697	11601.498046875\\
1.7150000333786	11662.0234375\\
1.72000002861023	11725.427734375\\
1.72500002384186	11787.6005859375\\
1.73000001907349	11844.1201171875\\
1.73500001430511	11895.494140625\\
1.74000000953674	11945.6796875\\
1.74500000476837	11998.3671875\\
1.75	12056.1953125\\
1.75499999523163	12116.568359375\\
1.75999999046326	12174.75390625\\
1.76499998569489	12227.2119140625\\
1.76999998092651	12274.388671875\\
1.77499997615814	12321.408203125\\
1.77999997138977	12370.259765625\\
1.7849999666214	12424.48046875\\
1.78999996185303	12481.1025390625\\
1.79499995708466	12533.103515625\\
1.79999995231628	12579.455078125\\
1.80499994754791	12620.8251953125\\
1.80999994277954	12662.38671875\\
1.81500005722046	12707.3251953125\\
1.82000005245209	12756.5234375\\
1.82500004768372	12806.7158203125\\
1.83000004291534	12851.2138671875\\
1.83500003814697	12889.0380859375\\
1.8400000333786	12923.1572265625\\
1.84500002861023	12958.408203125\\
1.85000002384186	12998.1640625\\
1.85500001907349	13041.0361328125\\
1.86000001430511	13082.560546875\\
1.86500000953674	13117.3359375\\
1.87000000476837	13145.91015625\\
1.875	13172.08984375\\
1.87999999523163	13201.630859375\\
1.88499999046326	13234.4189453125\\
1.88999998569489	13268.8271484375\\
1.89499998092651	13299.59375\\
1.89999997615814	13322.4091796875\\
1.90499997138977	13340.4345703125\\
1.9099999666214	13358.8154296875\\
1.91499996185303	13380.310546875\\
1.91999995708466	13405.5361328125\\
1.92499995231628	13431.021484375\\
1.92999994754791	13449.90234375\\
1.93499994277954	13459.609375\\
1.94000005722046	13467.517578125\\
1.94500005245209	13477.3388671875\\
1.95000004768372	13491.7138671875\\
1.95500004291534	13508.0283203125\\
1.96000003814697	13519.66796875\\
1.9650000333786	13524.7509765625\\
1.97000002861023	13524.044921875\\
1.97500002384186	13523.115234375\\
1.98000001907349	13525.8115234375\\
1.98500001430511	13532.216796875\\
1.99000000953674	13537.9931640625\\
1.99500000476837	13537.611328125\\
2	13530.853515625\\
2.00500011444092	13520.3896484375\\
2.00999999046326	13512.654296875\\
2.01500010490417	13509.9951171875\\
2.01999998092651	13508.1005859375\\
2.02500009536743	13503.400390625\\
2.02999997138977	13492.01171875\\
2.03500008583069	13475.6943359375\\
2.03999996185303	13459.525390625\\
2.04500007629395	13448.451171875\\
2.04999995231628	13440.9033203125\\
2.0550000667572	13434.056640625\\
2.05999994277954	13422.611328125\\
2.06500005722046	13406.02734375\\
2.0699999332428	13388.0771484375\\
2.07500004768372	13372.84765625\\
2.07999992370605	13364.259765625\\
2.08500003814697	13359.072265625\\
2.08999991416931	13352.6904296875\\
2.09500002861023	13341.3974609375\\
2.09999990463257	13327.3232421875\\
2.10500001907349	13314.31640625\\
2.10999989509583	13306.6015625\\
2.11500000953674	13303.28515625\\
2.11999988555908	13304.982421875\\
2.125	13295.5\\
2.13000011444092	13284.9423828125\\
2.13499999046326	13272.2783203125\\
2.14000010490417	13262.5693359375\\
2.14499998092651	13255.251953125\\
2.15000009536743	13250.8466796875\\
2.15499997138977	13245.20703125\\
2.16000008583069	13234.599609375\\
2.16499996185303	13220.154296875\\
2.17000007629395	13206.494140625\\
2.17499995231628	13194.529296875\\
2.1800000667572	13186.9638671875\\
2.18499994277954	13179.244140625\\
2.19000005722046	13169.330078125\\
2.1949999332428	13154.388671875\\
2.20000004768372	13137.0498046875\\
2.20499992370605	13120.482421875\\
2.21000003814697	13106.7734375\\
2.21499991416931	13095.314453125\\
2.22000002861023	13081.5341796875\\
2.22499990463257	13064.2978515625\\
2.23000001907349	13043.1806640625\\
2.23499989509583	13021.99609375\\
2.24000000953674	13001.9033203125\\
2.24499988555908	12986.8046875\\
2.25	12973.7509765625\\
2.25500011444092	12958.9189453125\\
2.25999999046326	12940.8447265625\\
2.26500010490417	12921.671875\\
2.26999998092651	12902.521484375\\
2.27500009536743	12888.52734375\\
2.27999997138977	12875.9130859375\\
2.28500008583069	12865.2236328125\\
2.28999996185303	12852.580078125\\
2.29500007629395	12836.61328125\\
2.29999995231628	12819.68359375\\
2.3050000667572	12805.146484375\\
2.30999994277954	12795.931640625\\
2.31500005722046	12788.58203125\\
2.3199999332428	12782.2548828125\\
2.32500004768372	12771.5263671875\\
2.32999992370605	12758.375\\
2.33500003814697	12749.0498046875\\
2.33999991416931	12736.9423828125\\
2.34500002861023	12733.408203125\\
2.34999990463257	12732.9326171875\\
2.35500001907349	12728.533203125\\
2.35999989509583	12721.4951171875\\
2.36500000953674	12710.0302734375\\
2.36999988555908	12701.5556640625\\
2.375	12695.8125\\
2.38000011444092	12693.462890625\\
2.38499999046326	12691.7265625\\
2.39000010490417	12686.0322265625\\
2.39499998092651	12674.8115234375\\
2.40000009536743	12661.7763671875\\
2.40499997138977	12651.5888671875\\
2.41000008583069	12647.6796875\\
2.41499996185303	12648.4892578125\\
2.42000007629395	12648.91015625\\
2.42499995231628	12645.3974609375\\
2.4300000667572	12638.185546875\\
2.43499994277954	12632.7197265625\\
2.44000005722046	12632.75390625\\
2.4449999332428	12640.0908203125\\
2.45000004768372	12650.23828125\\
2.45499992370605	12658.455078125\\
2.46000003814697	12662.7265625\\
2.46499991416931	12661.6220703125\\
2.47000002861023	12664.9716796875\\
2.47499990463257	12672.087890625\\
2.48000001907349	12683.8876953125\\
2.48499989509583	12694.759765625\\
2.49000000953674	12701.740234375\\
2.49499988555908	12701.4580078125\\
2.5	12699.998046875\\
2.50500011444092	12700.548828125\\
2.50999999046326	12706.865234375\\
2.51500010490417	12715.62109375\\
2.51999998092651	12721.8017578125\\
2.52500009536743	12722.8603515625\\
2.52999997138977	12719.1748046875\\
2.53500008583069	12717.712890625\\
2.53999996185303	12715.3212890625\\
2.54500007629395	12717.8310546875\\
2.54999995231628	12720.9267578125\\
2.5550000667572	12718.7265625\\
2.55999994277954	12710.5654296875\\
2.56500005722046	12699.1572265625\\
2.5699999332428	12687.759765625\\
2.57500004768372	12681.220703125\\
2.57999992370605	12674.8349609375\\
2.58500003814697	12667.8056640625\\
2.58999991416931	12655.685546875\\
2.59500002861023	12639.5068359375\\
2.59999990463257	12621.962890625\\
2.60500001907349	12606.5517578125\\
2.60999989509583	12594.5986328125\\
2.61500000953674	12583.5498046875\\
2.61999988555908	12569.83203125\\
2.625	12551.6591796875\\
2.63000011444092	12530.4384765625\\
2.63499999046326	12508.8330078125\\
2.64000010490417	12490.3125\\
2.64499998092651	12474.333984375\\
2.65000009536743	12459.12109375\\
2.65499997138977	12441.4501953125\\
2.66000008583069	12421.5966796875\\
2.66499996185303	12399.7900390625\\
2.67000007629395	12381.3046875\\
2.67499995231628	12366.0341796875\\
2.6800000667572	12352.9912109375\\
2.68499994277954	12341.73828125\\
2.69000005722046	12328.595703125\\
2.6949999332428	12313.703125\\
2.70000004768372	12298.7861328125\\
2.70499992370605	12285.1240234375\\
2.71000003814697	12275.638671875\\
2.71499991416931	12267.484375\\
2.72000002861023	12259.3115234375\\
2.72499990463257	12248.9619140625\\
2.73000001907349	12236.5732421875\\
2.73499989509583	12223.9580078125\\
2.74000000953674	12214.046875\\
2.74499988555908	12207.80859375\\
2.75	12204.9853515625\\
2.75500011444092	12199.8720703125\\
2.75999999046326	12194.0703125\\
2.76500010490417	12185.71484375\\
2.76999998092651	12178.6044921875\\
2.77500009536743	12175.44140625\\
2.77999997138977	12174.2236328125\\
2.78500008583069	12174.81640625\\
2.78999996185303	12174.1884765625\\
2.79500007629395	12171.546875\\
2.79999995231628	12165.3564453125\\
2.8050000667572	12162.19140625\\
2.80999994277954	12163.2099609375\\
2.81500005722046	12169.7021484375\\
2.8199999332428	12179.2568359375\\
2.82500004768372	12188.1484375\\
2.82999992370605	12195.984375\\
2.83500003814697	12205.4853515625\\
2.83999991416931	12221.4443359375\\
2.84500002861023	12243.73828125\\
2.84999990463257	12273.7529296875\\
2.85500001907349	12302.55078125\\
2.85999989509583	12330.7001953125\\
2.86500000953674	12356.42578125\\
2.86999988555908	12384.6875\\
2.875	12418.46484375\\
2.88000011444092	12458.3330078125\\
2.88499999046326	12501.3330078125\\
2.89000010490417	12540.9775390625\\
2.89499998092651	12575.1064453125\\
2.90000009536743	12606.775390625\\
2.90499997138977	12638.2861328125\\
2.91000008583069	12672.052734375\\
2.91499996185303	12702.3134765625\\
2.92000007629395	12727.333984375\\
2.92499995231628	12741.0732421875\\
2.9300000667572	12744.646484375\\
2.93499994277954	12742.580078125\\
2.94000005722046	12738.9326171875\\
2.9449999332428	12737.4658203125\\
2.95000004768372	12737.39453125\\
2.95499992370605	12731.6826171875\\
2.96000003814697	12721.8056640625\\
2.96499991416931	12706.91796875\\
2.97000002861023	12692.8095703125\\
2.97499990463257	12683.54296875\\
2.98000001907349	12679.1884765625\\
2.98499989509583	12676.802734375\\
2.99000000953674	12671.701171875\\
2.99499988555908	12661.0439453125\\
3	12648.4072265625\\
3.00500011444092	12634.4296875\\
3.00999999046326	12624.2626953125\\
3.01500010490417	12614.1455078125\\
3.01999998092651	12602.119140625\\
3.02500009536743	12584.9189453125\\
3.02999997138977	12563.77734375\\
3.03500008583069	12539.3359375\\
3.03999996185303	12517.3564453125\\
3.04500007629395	12499.796875\\
3.04999995231628	12485.3671875\\
3.0550000667572	12469.64453125\\
3.05999994277954	12452.3154296875\\
3.06500005722046	12431.517578125\\
3.0699999332428	12411.87109375\\
3.07500004768372	12391.7666015625\\
3.07999992370605	12375.443359375\\
3.08500003814697	12356.0791015625\\
3.08999991416931	12335.1376953125\\
3.09500002861023	12310.7744140625\\
3.09999990463257	12283.962890625\\
3.10500001907349	12256.869140625\\
3.10999989509583	12232.2431640625\\
3.11500000953674	12210.22265625\\
3.11999988555908	12190.486328125\\
3.125	12168.9189453125\\
3.13000011444092	12143.8115234375\\
3.13499999046326	12119.48828125\\
3.14000010490417	12088.5888671875\\
3.14499998092651	12061.1103515625\\
3.15000009536743	12036.822265625\\
3.15499997138977	12015.556640625\\
3.16000008583069	11993.640625\\
3.16499996185303	11972.8740234375\\
3.17000007629395	11957.8134765625\\
3.17499995231628	11954.390625\\
3.1800000667572	11964.9677734375\\
3.18499994277954	11988.1123046875\\
3.19000005722046	12018.8505859375\\
3.1949999332428	12053.2998046875\\
3.20000004768372	12091.904296875\\
3.20499992370605	12137.7578125\\
3.21000003814697	12191.44140625\\
3.21499991416931	12250.8095703125\\
3.22000002861023	12311.4482421875\\
3.22499990463257	12364.1162109375\\
3.23000001907349	12411.1962890625\\
3.23499989509583	12452.859375\\
3.24000000953674	12491.8916015625\\
3.24499988555908	12529.0615234375\\
3.25	12561.9814453125\\
3.25500011444092	12586.6728515625\\
3.25999999046326	12600.5322265625\\
3.26500010490417	12604.94921875\\
3.26999998092651	12604.775390625\\
3.27500009536743	12604.8505859375\\
3.27999997138977	12607.3291015625\\
3.28500008583069	12610.6748046875\\
3.28999996185303	12609.66015625\\
3.29500007629395	12604.54296875\\
3.29999995231628	12593.4501953125\\
3.3050000667572	12586.3935546875\\
3.30999994277954	12578.982421875\\
3.31500005722046	12576.2666015625\\
3.3199999332428	12573.1455078125\\
3.32500004768372	12567.078125\\
3.32999992370605	12555.0068359375\\
3.33500003814697	12539.9345703125\\
3.33999991416931	12524.125\\
3.34500002861023	12511.609375\\
3.34999990463257	12502.08984375\\
3.35500001907349	12491.7578125\\
3.35999989509583	12478.0908203125\\
3.36500000953674	12461.744140625\\
3.36999988555908	12444.439453125\\
3.375	12430.375\\
3.38000011444092	12421.8642578125\\
3.38499999046326	12416.46875\\
3.39000010490417	12408.91796875\\
3.39499998092651	12397.953125\\
3.40000009536743	12382.998046875\\
3.40499997138977	12366.91015625\\
3.41000008583069	12351.71875\\
3.41499996185303	12337.0634765625\\
3.42000007629395	12322.556640625\\
3.42499995231628	12305.7412109375\\
3.4300000667572	12284.4951171875\\
3.43499994277954	12260.65234375\\
3.44000005722046	12236.0888671875\\
3.4449999332428	12211.7490234375\\
3.45000004768372	12188.693359375\\
3.45499992370605	12166.658203125\\
3.46000003814697	12138.2294921875\\
3.46499991416931	12104.294921875\\
3.47000002861023	12064.115234375\\
3.47499990463257	12023.349609375\\
3.48000001907349	11986.392578125\\
3.48499989509583	11958.2060546875\\
3.49000000953674	11937.8984375\\
3.49499988555908	11923.2119140625\\
3.5	11913.05078125\\
3.50500011444092	11912.5302734375\\
3.50999999046326	11922.7138671875\\
3.51500010490417	11951.7841796875\\
3.51999998092651	11996.68359375\\
3.52500009536743	12053.1015625\\
3.52999997138977	12117.3798828125\\
3.53500008583069	12190.1640625\\
3.53999996185303	12275.666015625\\
3.54500007629395	12373.921875\\
3.54999995231628	12478.427734375\\
3.5550000667572	12572.0322265625\\
3.55999994277954	12642.29296875\\
3.56500005722046	12687.0517578125\\
3.5699999332428	12708.482421875\\
3.57500004768372	12710.359375\\
3.57999992370605	12695.4033203125\\
3.58500003814697	12665.921875\\
3.58999991416931	12623.5126953125\\
3.59500002861023	12574.9814453125\\
3.59999990463257	12527.673828125\\
3.60500001907349	12487.677734375\\
3.60999989509583	12461.4921875\\
3.61500000953674	12452.240234375\\
3.61999988555908	12458.0849609375\\
3.625	12465.9228515625\\
3.63000011444092	12476.4501953125\\
3.63499999046326	12487.509765625\\
3.64000010490417	12502.0869140625\\
3.64499998092651	12510.607421875\\
3.65000009536743	12515.9970703125\\
3.65499997138977	12512.63671875\\
3.66000008583069	12498.17578125\\
3.66499996185303	12474.71875\\
3.67000007629395	12446.169921875\\
3.67499995231628	12417.4462890625\\
3.6800000667572	12395.021484375\\
3.68499994277954	12380.806640625\\
3.69000005722046	12374.5771484375\\
3.6949999332428	12368.326171875\\
3.70000004768372	12363.3037109375\\
3.70499992370605	12357.8681640625\\
3.71000003814697	12353.072265625\\
3.71499991416931	12344.865234375\\
3.72000002861023	12334.322265625\\
3.72499990463257	12310.33203125\\
3.73000001907349	12280.51171875\\
3.73499989509583	12243.2138671875\\
3.74000000953674	12200.591796875\\
3.74499988555908	12156.6572265625\\
3.75	12114.0966796875\\
3.75500011444092	12071.810546875\\
3.75999999046326	12023.90234375\\
3.76500010490417	11967.142578125\\
3.76999998092651	11903.1640625\\
3.77500009536743	11842.78515625\\
3.77999997138977	11796.3486328125\\
3.78500008583069	11771.310546875\\
3.78999996185303	11765.5908203125\\
3.79500007629395	11773.8662109375\\
3.79999995231628	11796.48828125\\
3.8050000667572	11832.4697265625\\
3.80999994277954	11894.0517578125\\
3.81500005722046	11981.7626953125\\
3.8199999332428	12091.1904296875\\
3.82500004768372	12216.3408203125\\
3.82999992370605	12345.1630859375\\
3.83500003814697	12476.1103515625\\
3.83999991416931	12602.693359375\\
3.84500002861023	12713.0625\\
3.84999990463257	12797.169921875\\
3.85500001907349	12844.251953125\\
3.85999989509583	12848.3896484375\\
3.86500000953674	12820.1123046875\\
3.86999988555908	12765.47265625\\
3.875	12696.6591796875\\
3.88000011444092	12623.7763671875\\
3.88499999046326	12562.1630859375\\
3.89000010490417	12516.2470703125\\
3.89499998092651	12482.7490234375\\
3.90000009536743	12458.5673828125\\
3.90499997138977	12442.7783203125\\
3.91000008583069	12442.537109375\\
3.91499996185303	12455.109375\\
3.92000007629395	12473.291015625\\
3.92499995231628	12489.3818359375\\
3.9300000667572	12495.1591796875\\
3.93499994277954	12487.7138671875\\
3.94000005722046	12469.29296875\\
3.9449999332428	12443.8427734375\\
3.95000004768372	12417.4599609375\\
3.95499992370605	12391.0517578125\\
3.96000003814697	12363.72265625\\
3.96499991416931	12341.2705078125\\
3.97000002861023	12317.662109375\\
3.97499990463257	12304.201171875\\
3.98000001907349	12298.501953125\\
3.98499989509583	12300.69921875\\
3.99000000953674	12304.96484375\\
3.99499988555908	12303.4443359375\\
4	12294.04296875\\
4.00500011444092	12276.12109375\\
4.01000022888184	12247.525390625\\
4.0149998664856	12211.4970703125\\
4.01999998092651	12168.8642578125\\
4.02500009536743	12123.201171875\\
4.03000020980835	12072.23828125\\
4.03499984741211	12013.8359375\\
4.03999996185303	11944.9951171875\\
4.04500007629395	11869.609375\\
4.05000019073486	11795.5478515625\\
4.05499982833862	11730.2470703125\\
4.05999994277954	11675.6484375\\
4.06500005722046	11631.5595703125\\
4.07000017166138	11597.0947265625\\
4.07499980926514	11575.3798828125\\
4.07999992370605	11578.3779296875\\
4.08500003814697	11614.3544921875\\
4.09000015258789	11685.4228515625\\
4.09499979019165	11787.53515625\\
4.09999990463257	11924.0087890625\\
4.10500001907349	12098.2734375\\
4.1100001335144	12314.228515625\\
4.11499977111816	12553.015625\\
4.11999988555908	12782.154296875\\
4.125	12949.97265625\\
4.13000011444092	13033.7412109375\\
4.13500022888184	13048.791015625\\
4.1399998664856	12997.583984375\\
4.14499998092651	12890.052734375\\
4.15000009536743	12743.833984375\\
4.15500020980835	12588.81640625\\
4.15999984741211	12453.49609375\\
4.16499996185303	12356.1103515625\\
4.17000007629395	12290.1015625\\
4.17500019073486	12258.501953125\\
4.17999982833862	12270.1162109375\\
4.18499994277954	12320.5654296875\\
4.19000005722046	12387.8955078125\\
4.19500017166138	12456.1259765625\\
4.19999980926514	12517.6513671875\\
4.20499992370605	12559.1640625\\
4.21000003814697	12571.7919921875\\
4.21500015258789	12552.0908203125\\
4.21999979019165	12509.5234375\\
4.22499990463257	12455.3310546875\\
4.23000001907349	12399.1220703125\\
4.2350001335144	12346.4951171875\\
4.23999977111816	12299.5751953125\\
4.24499988555908	12270.5859375\\
4.25	12263.8349609375\\
4.25500011444092	12273.9912109375\\
4.26000022888184	12293.6650390625\\
4.2649998664856	12311.638671875\\
4.26999998092651	12319.150390625\\
4.27500009536743	12312.8427734375\\
4.28000020980835	12286.412109375\\
4.28499984741211	12237.416015625\\
4.28999996185303	12172.845703125\\
4.29500007629395	12095.3818359375\\
4.30000019073486	12005.3955078125\\
4.30499982833862	11894.2177734375\\
4.30999994277954	11767.6650390625\\
4.31500005722046	11641.830078125\\
4.32000017166138	11536.5771484375\\
4.32499980926514	11460.91015625\\
4.32999992370605	11417.1435546875\\
4.33500003814697	11392.9853515625\\
4.34000015258789	11391.2294921875\\
4.34499979019165	11424.63671875\\
4.34999990463257	11503.0888671875\\
4.35500001907349	11632.921875\\
4.3600001335144	11812.3349609375\\
4.36499977111816	12037.845703125\\
4.36999988555908	12291.58984375\\
4.375	12545.6982421875\\
4.38000011444092	12799.576171875\\
4.38500022888184	13034.9365234375\\
4.3899998664856	13218.9951171875\\
4.39499998092651	13292.193359375\\
4.40000009536743	13249.0703125\\
4.40500020980835	13114.5009765625\\
4.40999984741211	12914.916015625\\
4.41499996185303	12686.64453125\\
4.42000007629395	12474.306640625\\
4.42500019073486	12324.25\\
4.42999982833862	12235.91796875\\
4.43499994277954	12193.931640625\\
4.44000005722046	12194.6953125\\
4.44500017166138	12241.958984375\\
4.44999980926514	12323.2685546875\\
4.45499992370605	12417.013671875\\
4.46000003814697	12504.1025390625\\
4.46500015258789	12561.7490234375\\
4.46999979019165	12578.0615234375\\
4.47499990463257	12551.5361328125\\
4.48000001907349	12495.482421875\\
4.4850001335144	12415.625\\
4.48999977111816	12338.7373046875\\
4.49499988555908	12275.0166015625\\
4.5	12226.7060546875\\
4.50500011444092	12196.20703125\\
4.51000022888184	12186.640625\\
4.5149998664856	12199.447265625\\
4.51999998092651	12228.51171875\\
4.52500009536743	12259.1953125\\
4.53000020980835	12276.96484375\\
4.53499984741211	12275.228515625\\
4.53999996185303	12246.671875\\
4.54500007629395	12188.5263671875\\
4.55000019073486	12104.052734375\\
4.55499982833862	12001.369140625\\
4.55999994277954	11895.7685546875\\
4.56500005722046	11792.63671875\\
4.57000017166138	11680.5556640625\\
4.57499980926514	11563.599609375\\
4.57999992370605	11454.26171875\\
4.58500003814697	11365.8388671875\\
4.59000015258789	11301.5771484375\\
4.59499979019165	11256.873046875\\
4.59999990463257	11230.328125\\
4.60500001907349	11221.22265625\\
4.6100001335144	11230.259765625\\
4.61499977111816	11266.45703125\\
4.61999988555908	11429.0986328125\\
4.625	11747.35546875\\
4.63000011444092	12118.9345703125\\
4.63500022888184	12471.4013671875\\
4.6399998664856	12786.1259765625\\
4.64499998092651	13074.4912109375\\
4.65000009536743	13332.572265625\\
4.65500020980835	13518.052734375\\
4.65999984741211	13596.669921875\\
4.66499996185303	13445.283203125\\
4.67000007629395	13126.1162109375\\
4.67500019073486	12740.013671875\\
4.67999982833862	12358.052734375\\
4.68499994277954	12064.5263671875\\
4.69000005722046	11897.9052734375\\
4.69500017166138	11855.421875\\
4.69999980926514	11904.1142578125\\
4.70499992370605	12016.5771484375\\
4.71000003814697	12180.5322265625\\
4.71500015258789	12375.79296875\\
4.71999979019165	12572.7587890625\\
4.72499990463257	12716.3798828125\\
4.73000001907349	12766.5712890625\\
4.7350001335144	12730.6201171875\\
4.73999977111816	12633.109375\\
4.74499988555908	12491.556640625\\
4.75	12340.810546875\\
4.75500011444092	12212.50390625\\
4.76000022888184	12122.3369140625\\
4.7649998664856	12081.7041015625\\
4.76999998092651	12080.720703125\\
4.77500009536743	12112.6142578125\\
4.78000020980835	12169.6650390625\\
4.78499984741211	12235.87890625\\
4.78999996185303	12281.59765625\\
4.79500007629395	12286.5029296875\\
4.80000019073486	12248.470703125\\
4.80499982833862	12144.5078125\\
4.80999994277954	11971.88671875\\
4.81500005722046	11770.626953125\\
4.82000017166138	11577.5888671875\\
4.82499980926514	11412.7373046875\\
4.82999992370605	11285.77734375\\
4.83500003814697	11188.6220703125\\
4.84000015258789	11112.1572265625\\
4.84499979019165	11064.07421875\\
4.84999990463257	11052.802734375\\
4.85500001907349	11084.4951171875\\
4.8600001335144	11163.806640625\\
4.86499977111816	11267.2861328125\\
4.86999988555908	11514.9072265625\\
4.875	11832.802734375\\
4.88000011444092	12178.7900390625\\
4.88500022888184	12557.5419921875\\
4.8899998664856	12931.28125\\
4.89499998092651	13231.9716796875\\
4.90000009536743	13448.677734375\\
4.90500020980835	13585.1533203125\\
4.90999984741211	13654.9892578125\\
4.91499996185303	13655.134765625\\
4.92000007629395	13429.337890625\\
4.92500019073486	12980.701171875\\
4.92999982833862	12477.984375\\
4.93499994277954	12046.3232421875\\
4.94000005722046	11752.4931640625\\
4.94500017166138	11567.140625\\
4.94999980926514	11506.89453125\\
4.95499992370605	11588.078125\\
4.96000003814697	11776.8076171875\\
4.96500015258789	12007.966796875\\
4.96999979019165	12228.94921875\\
4.97499990463257	12421.5810546875\\
4.98000001907349	12549.1650390625\\
4.9850001335144	12585.9814453125\\
4.98999977111816	12534.107421875\\
4.99499988555908	12416.4560546875\\
5	12272.3994140625\\
5.00500011444092	12130.583984375\\
5.01000022888184	12006.375\\
5.0149998664856	11904.755859375\\
5.01999998092651	11850.3056640625\\
5.02500009536743	11849.5869140625\\
5.03000020980835	11893.8720703125\\
5.03499984741211	11963.4814453125\\
5.03999996185303	12048.2353515625\\
5.04500007629395	12133.916015625\\
5.05000019073486	12210.7314453125\\
5.05499982833862	12268.0615234375\\
5.05999994277954	12296.64453125\\
5.06500005722046	12294.3876953125\\
5.07000017166138	12261.662109375\\
5.07499980926514	12203.4208984375\\
5.07999992370605	12125.4609375\\
5.08500003814697	12066.2841796875\\
5.09000015258789	12024.8291015625\\
5.09499979019165	12003.169921875\\
5.09999990463257	11996.9423828125\\
5.10500001907349	11994.35546875\\
5.1100001335144	11989.140625\\
5.11499977111816	11983.501953125\\
5.11999988555908	11980.224609375\\
5.125	11980.73828125\\
5.13000011444092	11981.5048828125\\
5.13500022888184	12115.0244140625\\
5.1399998664856	12272.478515625\\
5.14499998092651	12416.66015625\\
5.15000009536743	12558.859375\\
5.15500020980835	12689.9248046875\\
5.15999984741211	12722.5869140625\\
5.16499996185303	12646.4296875\\
5.17000007629395	12500.0146484375\\
5.17500019073486	12318.0703125\\
5.17999982833862	12135.009765625\\
5.18499994277954	11986.390625\\
5.19000005722046	11889.8076171875\\
5.19500017166138	11854.2041015625\\
5.19999980926514	11874.404296875\\
5.20499992370605	11931.576171875\\
5.21000003814697	12008.9775390625\\
5.21500015258789	12097.1123046875\\
5.21999979019165	12188.486328125\\
5.22499990463257	12269.3359375\\
5.23000001907349	12331.318359375\\
5.2350001335144	12369.6630859375\\
5.23999977111816	12382.697265625\\
5.24499988555908	12373.57421875\\
5.25	12350.525390625\\
5.25500011444092	12323.40234375\\
5.26000022888184	12301.48046875\\
5.2649998664856	12293.9326171875\\
5.26999998092651	12302.5\\
5.27500009536743	12323.994140625\\
5.28000020980835	12357.134765625\\
5.28499984741211	12403.3525390625\\
5.28999996185303	12462.6220703125\\
5.29500007629395	12533.9990234375\\
5.30000019073486	12609.3857421875\\
5.30499982833862	12676.13671875\\
5.30999994277954	12730.0673828125\\
5.31500005722046	12767.021484375\\
5.32000017166138	12786.279296875\\
5.32499980926514	12785.900390625\\
5.32999992370605	12768.2255859375\\
5.33500003814697	12736.9482421875\\
5.34000015258789	12695.4306640625\\
5.34499979019165	12649.556640625\\
5.34999990463257	12603.255859375\\
5.35500001907349	12564.982421875\\
5.3600001335144	12539.5810546875\\
5.36499977111816	12526.25390625\\
5.36999988555908	12520.1474609375\\
5.375	12516.3427734375\\
5.38000011444092	12512.923828125\\
5.38500022888184	12508.171875\\
5.3899998664856	12496.27734375\\
5.39499998092651	12473.375\\
5.40000009536743	12436.2763671875\\
5.40500020980835	12390.2724609375\\
5.40999984741211	12340.3896484375\\
5.41499996185303	12289.5810546875\\
5.42000007629395	12243.6640625\\
5.42500019073486	12204.5361328125\\
5.42999982833862	12176.849609375\\
5.43499994277954	12156.875\\
5.44000005722046	12134.4482421875\\
5.44500017166138	12114.123046875\\
5.44999980926514	12096.8310546875\\
5.45499992370605	12083.12890625\\
5.46000003814697	12073.2490234375\\
5.46500015258789	12065.474609375\\
5.46999979019165	12058.890625\\
5.47499990463257	12051.708984375\\
5.48000001907349	12044.3837890625\\
5.4850001335144	12037.9853515625\\
5.48999977111816	12034.16796875\\
5.49499988555908	12031.1806640625\\
5.5	12028.3173828125\\
5.50500011444092	12025.8896484375\\
5.51000022888184	12023.447265625\\
5.5149998664856	12020.462890625\\
5.51999998092651	12028.875\\
5.52500009536743	12052.6337890625\\
5.53000020980835	12081.0478515625\\
5.53499984741211	12108.3603515625\\
5.53999996185303	12131.7724609375\\
5.54500007629395	12151.919921875\\
5.55000019073486	12169.697265625\\
5.55499982833862	12187.068359375\\
5.55999994277954	12201.4345703125\\
5.56500005722046	12211.9873046875\\
5.57000017166138	12215.8427734375\\
5.57499980926514	12214.4365234375\\
5.57999992370605	12210.7822265625\\
5.58500003814697	12203.626953125\\
5.59000015258789	12195.6982421875\\
5.59499979019165	12186.2431640625\\
5.59999990463257	12172.732421875\\
5.60500001907349	12156.646484375\\
5.6100001335144	12139.7548828125\\
5.61499977111816	12124.2626953125\\
5.61999988555908	12108.6201171875\\
5.625	12094.9580078125\\
5.63000011444092	12083.6767578125\\
5.63500022888184	12074.85546875\\
5.6399998664856	12066.091796875\\
5.64499998092651	12058.291015625\\
5.65000009536743	12052.1103515625\\
5.65500020980835	12043.9951171875\\
5.65999984741211	12038.9453125\\
5.66499996185303	12035.9345703125\\
5.67000007629395	12034.8203125\\
5.67500019073486	12070.248046875\\
5.67999982833862	12050.8564453125\\
5.68499994277954	12021.3271484375\\
5.69000005722046	12019.658203125\\
5.69500017166138	12028.5673828125\\
5.69999980926514	12051.2568359375\\
5.70499992370605	12080.2080078125\\
5.71000003814697	12105.783203125\\
5.71500015258789	12126.611328125\\
5.71999979019165	12141.802734375\\
5.72499990463257	12144.50390625\\
5.73000001907349	12135.25390625\\
5.7350001335144	12116.970703125\\
5.73999977111816	12096.6474609375\\
5.74499988555908	12078.0537109375\\
5.75	12061.939453125\\
5.75500011444092	12045.4775390625\\
5.76000022888184	12029.7314453125\\
5.7649998664856	12017.412109375\\
5.76999998092651	12011.7275390625\\
5.77500009536743	12010.2265625\\
5.78000020980835	12009.7705078125\\
5.78499984741211	12004.24609375\\
5.78999996185303	11993.0615234375\\
5.79500007629395	11980.3447265625\\
5.80000019073486	11971.9736328125\\
5.80499982833862	11973.181640625\\
5.80999994277954	11977.7724609375\\
5.81500005722046	11980.2177734375\\
5.82000017166138	11971.095703125\\
5.82499980926514	11954.7822265625\\
5.82999992370605	11939.1962890625\\
5.83500003814697	11928.681640625\\
5.84000015258789	11925.4990234375\\
5.84499979019165	11919.16015625\\
5.84999990463257	11903.373046875\\
5.85500001907349	11872.5771484375\\
5.8600001335144	11838.0615234375\\
5.86499977111816	11811.48828125\\
5.86999988555908	11801.3408203125\\
5.875	11807.517578125\\
5.88000011444092	11831.7109375\\
5.88500022888184	11869.841796875\\
5.8899998664856	11919.494140625\\
5.89499998092651	11943.9921875\\
5.90000009536743	11997.728515625\\
5.90500020980835	12126.28515625\\
5.90999984741211	12358.2939453125\\
5.91499996185303	12686.052734375\\
5.92000007629395	13100.654296875\\
5.92500019073486	13622.8046875\\
5.92999982833862	14241.1875\\
5.93499994277954	14889.283203125\\
5.94000005722046	15486.2529296875\\
5.94500017166138	16004.7939453125\\
5.94999980926514	16451.857421875\\
5.95499992370605	16813.6875\\
5.96000003814697	17066.673828125\\
5.96500015258789	17163.939453125\\
5.96999979019165	17087.478515625\\
5.97499990463257	16855.498046875\\
5.98000001907349	16238.7578125\\
5.9850001335144	15218.0595703125\\
5.98999977111816	14064.9169921875\\
5.99499988555908	13030.4140625\\
6	12266.77734375\\
6.00500011444092	11833.59765625\\
6.01000022888184	11651.8671875\\
6.0149998664856	11686.8505859375\\
6.01999998092651	11935.66796875\\
6.02500009536743	12368.060546875\\
6.03000020980835	12901.337890625\\
6.03499984741211	13431.5146484375\\
6.03999996185303	13880.1435546875\\
6.04500007629395	14208.888671875\\
6.05000019073486	14351.203125\\
6.05499982833862	14289.0107421875\\
6.05999994277954	14057.6591796875\\
6.06500005722046	13718.1640625\\
6.07000017166138	13340.90625\\
6.07499980926514	12979.8251953125\\
6.07999992370605	12669.7490234375\\
6.08500003814697	12432.8076171875\\
6.09000015258789	12309.7080078125\\
6.09499979019165	12302.369140625\\
6.09999990463257	12379.1787109375\\
6.10500001907349	12497.6025390625\\
6.1100001335144	12633.0029296875\\
6.11499977111816	12775.1611328125\\
6.11999988555908	12898.94140625\\
6.125	12975.689453125\\
6.13000011444092	12977.2529296875\\
6.13500022888184	12902.4150390625\\
6.1399998664856	12771.1923828125\\
6.14499998092651	12606.8271484375\\
6.15000009536743	12433.787109375\\
6.15500020980835	12268.9755859375\\
6.15999984741211	12134.9443359375\\
6.16499996185303	12043.6298828125\\
6.17000007629395	11996.255859375\\
6.17500019073486	11986.0185546875\\
6.17999982833862	12007.2177734375\\
6.18499994277954	12056.9306640625\\
6.19000005722046	12099.9404296875\\
6.19500017166138	12128.5439453125\\
6.19999980926514	12137.9208984375\\
6.20499992370605	12128.490234375\\
6.21000003814697	12105.2431640625\\
6.21500015258789	12089.099609375\\
6.21999979019165	12079.19921875\\
6.22499990463257	12075.072265625\\
6.23000001907349	12073.9580078125\\
6.2350001335144	12072.087890625\\
6.23999977111816	12068.7314453125\\
6.24499988555908	12064.0908203125\\
6.25	12060.580078125\\
6.25500011444092	12059.4013671875\\
6.26000022888184	12060.220703125\\
6.2649998664856	12061.88671875\\
6.26999998092651	12061.46484375\\
6.27500009536743	12060.556640625\\
6.28000020980835	12059.2177734375\\
6.28499984741211	12059.2919921875\\
6.28999996185303	12059.193359375\\
6.29500007629395	12059.72265625\\
6.30000019073486	12060.21875\\
6.30499982833862	12060.4599609375\\
6.30999994277954	12060.3115234375\\
6.31500005722046	12059.623046875\\
6.32000017166138	12059.9921875\\
6.32499980926514	12059.7607421875\\
6.32999992370605	12061.267578125\\
6.33500003814697	12060.3671875\\
6.34000015258789	12060.798828125\\
6.34499979019165	12060.1943359375\\
6.34999990463257	12061.796875\\
6.35500001907349	12060.259765625\\
6.3600001335144	12007.0244140625\\
6.36499977111816	11948.9091796875\\
6.36999988555908	11892.0009765625\\
6.375	11838.0439453125\\
6.38000011444092	11790.28125\\
6.38500022888184	11764.068359375\\
6.3899998664856	11860.4560546875\\
6.39499998092651	11989.05078125\\
6.40000009536743	12130.4462890625\\
6.40500020980835	12260.8330078125\\
6.40999984741211	12367.0966796875\\
6.41499996185303	12447.0732421875\\
6.42000007629395	12502.5048828125\\
6.42500019073486	12532.5361328125\\
6.42999982833862	12537.9580078125\\
6.43499994277954	12524.8798828125\\
6.44000005722046	12496.814453125\\
6.44500017166138	12461.5458984375\\
6.44999980926514	12427.7333984375\\
6.45499992370605	12408.408203125\\
6.46000003814697	12409.8955078125\\
6.46500015258789	12429.853515625\\
6.46999979019165	12458.89453125\\
6.47499990463257	12487.8623046875\\
6.48000001907349	12515.525390625\\
6.4850001335144	12546.7265625\\
6.48999977111816	12583.23828125\\
6.49499988555908	12623.6943359375\\
6.5	12658.087890625\\
6.50500011444092	12676.6884765625\\
6.51000022888184	12679.7548828125\\
6.5149998664856	12672.1044921875\\
6.51999998092651	12662.9150390625\\
6.52500009536743	12658.9833984375\\
6.53000020980835	12658.9912109375\\
6.53499984741211	12656.5927734375\\
6.53999996185303	12645.162109375\\
6.54500007629395	12626.7646484375\\
6.55000019073486	12607.7841796875\\
6.55499982833862	12598.7607421875\\
6.55999994277954	12601.138671875\\
6.56500005722046	12608.0859375\\
6.57000017166138	12609.953125\\
6.57499980926514	12602.685546875\\
6.57999992370605	12587.0673828125\\
6.58500003814697	12574.5986328125\\
6.59000015258789	12570.4921875\\
6.59499979019165	12572.1455078125\\
6.59999990463257	12571.318359375\\
6.60500001907349	12559.72265625\\
6.6100001335144	12537.7041015625\\
6.61499977111816	12512.16015625\\
6.61999988555908	12492.208984375\\
6.625	12482.986328125\\
6.63000011444092	12479.0908203125\\
6.63500022888184	12471.4140625\\
6.6399998664856	12454.1962890625\\
6.64499998092651	12430.568359375\\
6.65000009536743	12411.55078125\\
6.65500020980835	12401.4736328125\\
6.65999984741211	12403.3779296875\\
6.66499996185303	12404.6572265625\\
6.67000007629395	12398.9658203125\\
6.67500019073486	12384.93359375\\
6.67999982833862	12364.3955078125\\
6.68499994277954	12352.830078125\\
6.69000005722046	12351.197265625\\
6.69500017166138	12359.41796875\\
6.69999980926514	12361.8623046875\\
6.70499992370605	12354.884765625\\
6.71000003814697	12338.9990234375\\
6.71500015258789	12326.1533203125\\
6.71999979019165	12323.634765625\\
6.72499990463257	12331.0966796875\\
6.73000001907349	12340.69140625\\
6.7350001335144	12340.8837890625\\
6.73999977111816	12332.0283203125\\
6.74499988555908	12315.796875\\
6.75	12304.4140625\\
6.75500011444092	12303.51171875\\
6.76000022888184	12304.7724609375\\
6.7649998664856	12300.9794921875\\
6.76999998092651	12283.953125\\
6.77500009536743	12256.873046875\\
6.78000020980835	12227.3017578125\\
6.78499984741211	12204.5478515625\\
6.78999996185303	12187.5712890625\\
6.79500007629395	12170.4580078125\\
6.80000019073486	12143.61328125\\
6.80499982833862	12105.17578125\\
6.80999994277954	12060.4052734375\\
6.81500005722046	12019.544921875\\
6.82000017166138	11987.08203125\\
6.82499980926514	11960.330078125\\
6.82999992370605	11931.3359375\\
6.83500003814697	11892.6552734375\\
6.84000015258789	11845.9306640625\\
6.84499979019165	11797.2822265625\\
6.84999990463257	11756.30078125\\
6.85500001907349	11724.794921875\\
6.8600001335144	11697.4755859375\\
6.86499977111816	11665.9609375\\
6.86999988555908	11625.5537109375\\
6.875	11580.0419921875\\
6.88000011444092	11537.1689453125\\
6.88500022888184	11504.5498046875\\
6.8899998664856	11478.6083984375\\
6.89499998092651	11457.1474609375\\
6.90000009536743	11432.24609375\\
6.90500020980835	11400.7763671875\\
6.90999984741211	11367.74609375\\
6.91499996185303	11338.060546875\\
6.92000007629395	11318.9111328125\\
6.92500019073486	11308.2421875\\
6.92999982833862	11299.6142578125\\
6.93499994277954	11286.953125\\
6.94000005722046	11269.87890625\\
6.94500017166138	11251.8076171875\\
6.94999980926514	11239.4873046875\\
6.95499992370605	11235.0009765625\\
6.96000003814697	11236.39453125\\
6.96500015258789	11239.6796875\\
6.96999979019165	11237.8349609375\\
6.97499990463257	11233.177734375\\
6.98000001907349	11229.208984375\\
6.9850001335144	11230.6669921875\\
6.98999977111816	11237.7841796875\\
6.99499988555908	11250.3486328125\\
7	11262.5419921875\\
7.00500011444092	11272.513671875\\
7.01000022888184	11278.8486328125\\
7.0149998664856	11286.673828125\\
7.01999998092651	11300.3408203125\\
7.02500009536743	11316.5732421875\\
7.03000020980835	11336.3876953125\\
7.03499984741211	11355.2900390625\\
7.03999996185303	11370.8828125\\
7.04500007629395	11385.3564453125\\
7.05000019073486	11399.580078125\\
7.05499982833862	11418.2177734375\\
7.05999994277954	11438.5126953125\\
7.06500005722046	11460.451171875\\
7.07000017166138	11484.78125\\
7.07499980926514	11497.609375\\
7.07999992370605	11512.9287109375\\
7.08500003814697	11529.1083984375\\
7.09000015258789	11547.107421875\\
7.09499979019165	11566.7607421875\\
7.09999990463257	11585.9521484375\\
7.10500001907349	11603.7880859375\\
7.1100001335144	11617.5458984375\\
7.11499977111816	11629.6025390625\\
7.11999988555908	11641.830078125\\
7.125	11656.0439453125\\
7.13000011444092	11670.9091796875\\
7.13500022888184	11683.4541015625\\
7.1399998664856	11693.6162109375\\
7.14499998092651	11700.3134765625\\
7.15000009536743	11706.4521484375\\
7.15500020980835	11711.0009765625\\
7.15999984741211	11717.546875\\
7.16499996185303	11723.9677734375\\
7.17000007629395	11728.61328125\\
7.17500019073486	11729.9619140625\\
7.17999982833862	11728.0869140625\\
7.18499994277954	11725.26171875\\
7.19000005722046	11724.0439453125\\
7.19500017166138	11722.388671875\\
7.19999980926514	11720.794921875\\
7.20499992370605	11717.205078125\\
7.21000003814697	11710.6640625\\
7.21500015258789	11701.146484375\\
7.21999979019165	11691.4013671875\\
7.22499990463257	11685.0458984375\\
7.23000001907349	11675.255859375\\
7.2350001335144	11668.56640625\\
7.23999977111816	11658.2724609375\\
7.24499988555908	11645.8505859375\\
7.25	11631.005859375\\
7.25500011444092	11616.11328125\\
7.26000022888184	11603.173828125\\
7.2649998664856	11592.1865234375\\
7.26999998092651	11582.2939453125\\
7.27500009536743	11568.40625\\
7.28000020980835	11552.134765625\\
7.28499984741211	11534.75\\
7.28999996185303	11517.9873046875\\
7.29500007629395	11503.8427734375\\
7.30000019073486	11492.5302734375\\
7.30499982833862	11480.3564453125\\
7.30999994277954	11466.37890625\\
7.31500005722046	11449.591796875\\
7.32000017166138	11431.66796875\\
7.32499980926514	11415.9091796875\\
7.32999992370605	11402.951171875\\
7.33500003814697	11392.244140625\\
7.34000015258789	11381.6435546875\\
7.34499979019165	11368.6826171875\\
7.34999990463257	11354.2451171875\\
7.35500001907349	11338.6513671875\\
7.3600001335144	11324.9248046875\\
7.36499977111816	11314.875\\
7.36999988555908	11307.3427734375\\
7.375	11300.0693359375\\
7.38000011444092	11289.8994140625\\
7.38500022888184	11278.3388671875\\
7.3899998664856	11265.97265625\\
7.39499998092651	11256.3486328125\\
7.40000009536743	11250.04296875\\
7.40500020980835	11246.1328125\\
7.40999984741211	11241.7333984375\\
7.41499996185303	11235.853515625\\
7.42000007629395	11228.330078125\\
7.42500019073486	11220.23046875\\
7.42999982833862	11215.140625\\
7.43499994277954	11212.9873046875\\
7.44000005722046	11212.1572265625\\
7.44500017166138	11212.2119140625\\
7.44999980926514	11210.8125\\
7.45499992370605	11206.5244140625\\
7.46000003814697	11203.7001953125\\
7.46500015258789	11201.2294921875\\
7.46999979019165	11202.267578125\\
7.47499990463257	11205.5263671875\\
7.48000001907349	11207.8671875\\
7.4850001335144	11208.8837890625\\
7.48999977111816	11207.6259765625\\
7.49499988555908	11206.8037109375\\
7.5	11207.60546875\\
7.50500011444092	11211.9501953125\\
7.51000022888184	11216.15625\\
7.5149998664856	11220.2197265625\\
7.51999998092651	11222.39453125\\
7.52500009536743	11222.8740234375\\
7.53000020980835	11223.53515625\\
7.53499984741211	11225.5869140625\\
7.53999996185303	11229.6533203125\\
7.54500007629395	11234.912109375\\
7.55000019073486	11239.1591796875\\
7.55499982833862	11240.8740234375\\
7.55999994277954	11243.0537109375\\
7.56500005722046	11240.720703125\\
7.57000017166138	11242.3134765625\\
7.57499980926514	11245.6123046875\\
7.57999992370605	11248.783203125\\
7.58500003814697	11255.8564453125\\
7.59000015258789	11251.7939453125\\
7.59499979019165	11249.8671875\\
7.59999990463257	11247.4365234375\\
7.60500001907349	11246.6279296875\\
7.6100001335144	11247.255859375\\
7.61499977111816	11248.2763671875\\
7.61999988555908	11247.9775390625\\
7.625	11246.009765625\\
7.63000011444092	11240.228515625\\
7.63500022888184	11234.9111328125\\
7.6399998664856	11230.57421875\\
7.64499998092651	11228.59375\\
7.65000009536743	11225.5673828125\\
7.65500020980835	11224.1650390625\\
7.65999984741211	11215.2529296875\\
7.66499996185303	11206.99609375\\
7.67000007629395	11197.7685546875\\
7.67500019073486	11189.5986328125\\
7.67999982833862	11183.7578125\\
7.68499994277954	11177.9765625\\
7.69000005722046	11172.216796875\\
7.69500017166138	11161.134765625\\
7.69999980926514	11148.1689453125\\
7.70499992370605	11135.6396484375\\
7.71000003814697	11124.65234375\\
7.71500015258789	11115.498046875\\
7.71999979019165	11106.7802734375\\
7.72499990463257	11097.025390625\\
7.73000001907349	11083.4033203125\\
7.7350001335144	11069.93359375\\
7.73999977111816	11052.837890625\\
7.74499988555908	11038.958984375\\
7.75	11027.458984375\\
7.75500011444092	11016.4248046875\\
7.76000022888184	11004.0390625\\
7.7649998664856	10988.7607421875\\
7.76999998092651	10971.748046875\\
7.77500009536743	10953.8876953125\\
7.78000020980835	10938.177734375\\
7.78499984741211	10925.279296875\\
7.78999996185303	10912.7529296875\\
7.79500007629395	10903.173828125\\
7.80000019073486	10882.98046875\\
7.80499982833862	10864.3037109375\\
7.80999994277954	10845.5283203125\\
7.81500005722046	10829.2783203125\\
7.82000017166138	10814.841796875\\
7.82499980926514	10801.1279296875\\
7.82999992370605	10787.609375\\
7.83500003814697	10770.0498046875\\
7.84000015258789	10750.7685546875\\
7.84499979019165	10731.267578125\\
7.84999990463257	10713.6328125\\
7.85500001907349	10698.52734375\\
7.8600001335144	10684.9951171875\\
7.86499977111816	10670.6630859375\\
7.86999988555908	10654.203125\\
7.875	10635.2109375\\
7.88000011444092	10615.30078125\\
7.88500022888184	10597.013671875\\
7.8899998664856	10581.423828125\\
7.89499998092651	10567.7041015625\\
7.90000009536743	10553.6611328125\\
7.90500020980835	10537.982421875\\
7.90999984741211	10518.583984375\\
7.91499996185303	10498.44140625\\
7.92000007629395	10479.4892578125\\
7.92500019073486	10463.185546875\\
7.92999982833862	10449.3388671875\\
7.93499994277954	10434.953125\\
7.94000005722046	10418.9921875\\
7.94500017166138	10400.0341796875\\
7.94999980926514	10379.3935546875\\
7.95499992370605	10359.658203125\\
7.96000003814697	10342.3115234375\\
7.96500015258789	10327.1650390625\\
7.96999979019165	10312.560546875\\
7.97499990463257	10296.189453125\\
7.98000001907349	10277.3232421875\\
7.9850001335144	10256.0732421875\\
7.98999977111816	10234.40234375\\
7.99499988555908	10214.8359375\\
8	10198.1123046875\\
8.00500011444092	10182.228515625\\
8.01000022888184	10165.0087890625\\
8.01500034332275	10145.658203125\\
8.02000045776367	10122.38671875\\
8.02499961853027	10098.60546875\\
8.02999973297119	10076.2841796875\\
8.03499984741211	10056.267578125\\
8.03999996185303	10038.4677734375\\
8.04500007629395	10019.73046875\\
8.05000019073486	9998.3525390625\\
8.05500030517578	9973.73828125\\
8.0600004196167	9946.9208984375\\
8.0649995803833	9920.6015625\\
8.06999969482422	9896.4873046875\\
8.07499980926514	9874.7216796875\\
8.07999992370605	9853.6396484375\\
8.08500003814697	9832.51953125\\
8.09000015258789	9804.03125\\
8.09500026702881	9773.9736328125\\
8.10000038146973	9743.8369140625\\
8.10499954223633	9713.884765625\\
8.10999965667725	9687.1474609375\\
8.11499977111816	9661.9521484375\\
8.11999988555908	9636.1845703125\\
8.125	9608.1171875\\
8.13000011444092	9574.1279296875\\
8.13500022888184	9538.576171875\\
8.14000034332275	9503.744140625\\
8.14500045776367	9470.291015625\\
8.14999961853027	9440.203125\\
8.15499973297119	9411.0234375\\
8.15999984741211	9379.5791015625\\
8.16499996185303	9344.546875\\
8.17000007629395	9304.6083984375\\
8.17500019073486	9263.470703125\\
8.18000030517578	9223.5615234375\\
8.1850004196167	9186.6220703125\\
8.1899995803833	9152.5\\
8.19499969482422	9117.919921875\\
8.19999980926514	9080.462890625\\
8.20499992370605	9038.7021484375\\
8.21000003814697	8993.830078125\\
8.21500015258789	8946.3037109375\\
8.22000026702881	8902.1005859375\\
8.22500038146973	8860.73828125\\
8.22999954223633	8822.2431640625\\
8.23499965667725	8783.6982421875\\
8.23999977111816	8741.4697265625\\
8.24499988555908	8694.2568359375\\
8.25	8643.3935546875\\
8.25500011444092	8591.33984375\\
8.26000022888184	8542.181640625\\
8.26500034332275	8497.19921875\\
8.27000045776367	8455.3779296875\\
8.27499961853027	8413.5732421875\\
8.27999973297119	8368.3466796875\\
8.28499984741211	8317.8994140625\\
8.28999996185303	8263.2451171875\\
8.29500007629395	8207.484375\\
8.30000019073486	8154.5439453125\\
8.30500030517578	8106.86865234375\\
8.3100004196167	8063.2783203125\\
8.3149995803833	8021.375\\
8.31999969482422	7977.00732421875\\
8.32499980926514	7927.38525390625\\
8.32999992370605	7872.88818359375\\
8.33500003814697	7816.37158203125\\
8.34000015258789	7762.26123046875\\
8.34500026702881	7713.646484375\\
8.35000038146973	7671.595703125\\
8.35499954223633	7633.63525390625\\
8.35999965667725	7596.0107421875\\
8.36499977111816	7554.52587890625\\
8.36999988555908	7508.0732421875\\
8.375	7458.04541015625\\
8.38000011444092	7408.32275390625\\
8.38500022888184	7364.3076171875\\
8.39000034332275	7328.85986328125\\
8.39500045776367	7301.9619140625\\
8.39999961853027	7280.3369140625\\
8.40499973297119	7259.47998046875\\
8.40999984741211	7235.18701171875\\
8.41499996185303	7206.6376953125\\
8.42000007629395	7176.45849609375\\
8.42500019073486	7149.79150390625\\
8.43000030517578	7132.48486328125\\
8.4350004196167	7127.3779296875\\
8.4399995803833	7134.34765625\\
8.44499969482422	7149.2705078125\\
8.44999980926514	7166.30029296875\\
8.45499992370605	7180.71435546875\\
8.46000003814697	7190.96728515625\\
8.46500015258789	7200.220703125\\
8.47000026702881	7214.138671875\\
8.47500038146973	7238.92822265625\\
8.47999954223633	7278.0517578125\\
8.48499965667725	7330.79248046875\\
8.48999977111816	7391.58544921875\\
8.49499988555908	7453.3115234375\\
8.5	7510.03173828125\\
8.50500011444092	7560.02587890625\\
8.51000022888184	7606.90478515625\\
8.51500034332275	7657.4697265625\\
8.52000045776367	7718.89208984375\\
8.52499961853027	7794.1748046875\\
8.52999973297119	7881.0869140625\\
8.53499984741211	7972.27294921875\\
8.53999996185303	8058.71630859375\\
8.54500007629395	8134.771484375\\
8.55000019073486	8200.7763671875\\
8.55500030517578	8263.013671875\\
8.5600004196167	8330.4033203125\\
8.5649995803833	8409.80859375\\
8.56999969482422	8502.248046875\\
8.57499980926514	8601.7177734375\\
8.57999992370605	8698.921875\\
8.58500003814697	8785.4990234375\\
8.59000015258789	8860.130859375\\
8.59500026702881	8928.619140625\\
8.60000038146973	9001.9970703125\\
8.60499954223633	9089.2109375\\
8.60999965667725	9192.9130859375\\
8.61499977111816	9307.15625\\
8.61999988555908	9421.763671875\\
8.625	9527.83203125\\
8.63000011444092	9624.1943359375\\
8.63500022888184	9717.080078125\\
8.64000034332275	9816.7099609375\\
8.64500045776367	9930.2421875\\
8.64999961853027	10056.708984375\\
8.65499973297119	10186.8701171875\\
8.65999984741211	10310.3427734375\\
8.66499996185303	10418.759765625\\
8.67000007629395	10515.232421875\\
8.67500019073486	10608.052734375\\
8.68000030517578	10707.70703125\\
8.6850004196167	10818.955078125\\
8.6899995803833	10937.287109375\\
8.69499969482422	11053.13671875\\
8.69999980926514	11157.431640625\\
8.70499992370605	11249.873046875\\
8.71000003814697	11337.4697265625\\
8.71500015258789	11430.87109375\\
8.72000026702881	11535.953125\\
8.72500038146973	11649.2890625\\
8.72999954223633	11760.8466796875\\
8.73499965667725	11861.185546875\\
8.73999977111816	11947.9599609375\\
8.74499988555908	12028.07421875\\
8.75	12110.3671875\\
8.75500011444092	12200.3369140625\\
8.76000022888184	12295.2841796875\\
8.76500034332275	12385.890625\\
8.77000045776367	12464.619140625\\
8.77499961853027	12531.212890625\\
8.77999973297119	12592.4052734375\\
8.78499984741211	12657.77734375\\
8.78999996185303	12731.8955078125\\
8.79500007629395	12810.396484375\\
8.80000019073486	12884.232421875\\
8.80500030517578	12947.9501953125\\
8.8100004196167	13001.9755859375\\
8.8149995803833	13053.978515625\\
8.81999969482422	13111.384765625\\
8.82499980926514	13175.3544921875\\
8.82999992370605	13239.7744140625\\
8.83500003814697	13296.5693359375\\
8.84000015258789	13341.6640625\\
8.84500026702881	13379.908203125\\
8.85000038146973	13418.544921875\\
8.85499954223633	13463.662109375\\
8.85999965667725	13513.751953125\\
8.86499977111816	13562.0146484375\\
8.86999988555908	13601.9072265625\\
8.875	13633.251953125\\
8.88000011444092	13662.1337890625\\
8.88500022888184	13695.7353515625\\
8.89000034332275	13736.5361328125\\
8.89500045776367	13779.9248046875\\
8.89999961853027	13816.9306640625\\
8.90499973297119	13845.5166015625\\
8.90999984741211	13867.0595703125\\
8.91499996185303	13889.591796875\\
8.92000007629395	13918.1337890625\\
8.92500019073486	13951.46484375\\
8.93000030517578	13982.4775390625\\
8.9350004196167	14005.83203125\\
8.9399995803833	14021.078125\\
8.94499969482422	14034.1904296875\\
8.94999980926514	14052.322265625\\
8.95499992370605	14077.0458984375\\
8.96000003814697	14104.23828125\\
8.96500015258789	14125.337890625\\
8.97000026702881	14137.8603515625\\
8.97500038146973	14145.1787109375\\
8.97999954223633	14155.099609375\\
8.98499965667725	14171.3447265625\\
8.98999977111816	14190.4931640625\\
8.99499988555908	14206.8798828125\\
9	14214.453125\\
9.00500011444092	14214.87109375\\
9.01000022888184	14214.150390625\\
9.01500034332275	14218.4052734375\\
9.02000045776367	14228.2109375\\
9.02499961853027	14237.390625\\
9.02999973297119	14239.5009765625\\
9.03499984741211	14232.66015625\\
9.03999996185303	14222.8076171875\\
9.04500007629395	14214.5888671875\\
9.05000019073486	14212.7470703125\\
9.05500030517578	14213.0732421875\\
9.0600004196167	14208.388671875\\
9.0649995803833	14194.458984375\\
9.06999969482422	14174.447265625\\
9.07499980926514	14154.62890625\\
9.07999992370605	14139.876953125\\
9.08500003814697	14129.4052734375\\
9.09000015258789	14116.671875\\
9.09500026702881	14097.017578125\\
9.10000038146973	14068.341796875\\
9.10499954223633	14037.095703125\\
9.10999965667725	14010.0224609375\\
9.11499977111816	13989.365234375\\
9.11999988555908	13968.5048828125\\
9.125	13944.037109375\\
9.13000011444092	13909.0966796875\\
9.13500022888184	13871.00390625\\
9.14000034332275	13835.5771484375\\
9.14500045776367	13805.556640625\\
9.14999961853027	13779.3251953125\\
9.15499973297119	13751.0615234375\\
9.15999984741211	13715.5869140625\\
9.16499996185303	13674.82421875\\
9.17000007629395	13632.896484375\\
9.17500019073486	13596.2392578125\\
9.18000030517578	13565.7119140625\\
9.1850004196167	13536.166015625\\
9.1899995803833	13503.9384765625\\
9.19499969482422	13464.0185546875\\
9.19999980926514	13423.4462890625\\
9.20499992370605	13383.8876953125\\
9.21000003814697	13351.6884765625\\
9.21500015258789	13324.9951171875\\
9.22000026702881	13296.2021484375\\
9.22500038146973	13264.8828125\\
9.22999954223633	13227.2041015625\\
9.23499965667725	13190.7138671875\\
9.23999977111816	13160.7109375\\
9.24499988555908	13138.525390625\\
9.25	13121.2744140625\\
9.25500011444092	13104.27734375\\
9.26000022888184	13084.4560546875\\
9.26500034332275	13066.3701171875\\
9.27000045776367	13055.4609375\\
9.27499961853027	13056.03125\\
9.27999973297119	13067.701171875\\
9.28499984741211	13081.9658203125\\
9.28999996185303	13095.8251953125\\
9.29500007629395	13109.1962890625\\
9.30000019073486	13129.7900390625\\
9.30500030517578	13162.134765625\\
9.3100004196167	13212.8125\\
9.3149995803833	13276.708984375\\
9.31999969482422	13346.7421875\\
9.32499980926514	13418.4560546875\\
9.32999992370605	13496.2197265625\\
9.33500003814697	13591.0283203125\\
9.34000015258789	13711.88671875\\
9.34500026702881	13863.3759765625\\
9.35000038146973	14037.4814453125\\
9.35499954223633	14223.2822265625\\
9.35999965667725	14417.0390625\\
9.36499977111816	14629.140625\\
9.36999988555908	14869.3681640625\\
9.375	15144.1591796875\\
9.38000011444092	15446.6494140625\\
9.38500022888184	15750.5625\\
9.39000034332275	16031.421875\\
9.39500045776367	16280.119140625\\
9.39999961853027	16505.23828125\\
9.40499973297119	16714.625\\
9.40999984741211	16894.666015625\\
9.41499996185303	17009.029296875\\
9.42000007629395	17026.89453125\\
9.42500019073486	16944.26953125\\
9.43000030517578	16788.076171875\\
9.4350004196167	16578.529296875\\
9.4399995803833	16315.7021484375\\
9.44499969482422	15977.9892578125\\
9.44999980926514	15563.326171875\\
9.45499992370605	15098.662109375\\
9.46000003814697	14623.99609375\\
9.46500015258789	14166.2392578125\\
9.47000026702881	13732.53125\\
9.47500038146973	13337.283203125\\
9.47999954223633	13003.337890625\\
9.48499965667725	12764.798828125\\
9.48999977111816	12628.4326171875\\
9.49499988555908	12552.576171875\\
9.5	12534.314453125\\
9.50500011444092	12551.7607421875\\
9.51000022888184	12575.98046875\\
9.51500034332275	12587.494140625\\
9.52000045776367	12581.4248046875\\
9.52499961853027	12562.1591796875\\
9.52999973297119	12535.82421875\\
9.53499984741211	12508.5673828125\\
9.53999996185303	12476.783203125\\
9.54500007629395	12440.30859375\\
9.55000019073486	12397.3515625\\
9.55500030517578	12345.35546875\\
9.5600004196167	12294.3466796875\\
9.5649995803833	12258.052734375\\
9.56999969482422	12232.630859375\\
9.57499980926514	12210.4755859375\\
9.57999992370605	12187.2880859375\\
9.58500003814697	12163.4794921875\\
9.59000015258789	12143.203125\\
9.59500026702881	12129.0830078125\\
9.60000038146973	12120.2744140625\\
9.60499954223633	12114.7919921875\\
9.60999965667725	12107.826171875\\
9.61499977111816	12099.7109375\\
9.61999988555908	12091.732421875\\
9.625	12085.2724609375\\
9.63000011444092	12082.4384765625\\
9.63500022888184	12079.0732421875\\
9.64000034332275	12077.4775390625\\
9.64500045776367	12074.78125\\
9.64999961853027	12071.6748046875\\
9.65499973297119	12069.8056640625\\
9.65999984741211	12066.7900390625\\
9.66499996185303	12065.7119140625\\
9.67000007629395	12064.861328125\\
9.67500019073486	12064.2919921875\\
9.68000030517578	12063.861328125\\
9.6850004196167	12062.9716796875\\
9.6899995803833	12060.9326171875\\
9.69499969482422	12060.599609375\\
9.69999980926514	12060.078125\\
9.70499992370605	12059.6650390625\\
9.71000003814697	12059.72265625\\
9.71500015258789	12059.6396484375\\
9.72000026702881	12059.400390625\\
9.72500038146973	12058.37890625\\
9.72999954223633	12058.107421875\\
9.73499965667725	12057.91796875\\
9.73999977111816	12057.98828125\\
9.74499988555908	12058.7216796875\\
9.75	12057.5966796875\\
9.75500011444092	12057.4814453125\\
9.76000022888184	12057.4150390625\\
9.76500034332275	12057.341796875\\
9.77000045776367	12057.1474609375\\
9.77499961853027	12057.3349609375\\
9.77999973297119	12057.0751953125\\
9.78499984741211	12056.9072265625\\
9.78999996185303	12057.619140625\\
9.79500007629395	12056.8427734375\\
9.80000019073486	12056.8310546875\\
9.80500030517578	12058.0859375\\
9.8100004196167	12057.1474609375\\
9.8149995803833	12056.9130859375\\
9.81999969482422	12057.4482421875\\
9.82499980926514	12056.6162109375\\
9.82999992370605	12057.1201171875\\
9.83500003814697	12056.603515625\\
9.84000015258789	12056.8193359375\\
9.84500026702881	12056.888671875\\
9.85000038146973	12056.6923828125\\
9.85499954223633	12056.638671875\\
9.85999965667725	12056.619140625\\
9.86499977111816	12056.6123046875\\
9.86999988555908	12056.65234375\\
9.875	12057.212890625\\
9.88000011444092	12056.7568359375\\
9.88500022888184	12057.2880859375\\
9.89000034332275	12056.517578125\\
9.89500045776367	12057.8115234375\\
9.89999961853027	12056.6806640625\\
9.90499973297119	12056.6923828125\\
9.90999984741211	12056.96484375\\
9.91499996185303	12056.8154296875\\
9.92000007629395	12056.78515625\\
9.92500019073486	12056.5927734375\\
9.93000030517578	12056.5517578125\\
9.9350004196167	12057.0458984375\\
9.9399995803833	12057.5673828125\\
9.94499969482422	12056.7021484375\\
9.94999980926514	12057.6552734375\\
9.95499992370605	12056.22265625\\
9.96000003814697	12056.447265625\\
9.96500015258789	12057.1494140625\\
9.97000026702881	12056.6875\\
9.97500038146973	12056.662109375\\
9.97999954223633	12056.2939453125\\
9.98499965667725	12056.548828125\\
9.98999977111816	12056.57421875\\
9.99499988555908	12057.251953125\\
10	12056.8935546875\\
};
\addlegendentry{CF}

\end{axis}

\begin{axis}[%
width=4.521in,
height=1.476in,
at={(0.758in,0.498in)},
scale only axis,
xmin=0,
xmax=10,
xlabel style={font=\color{white!15!black}},
xlabel={Time (s)},
ymin=-25.4022064208984,
ymax=2000,
ylabel style={font=\color{white!15!black}},
ylabel={FY (N)},
axis background/.style={fill=white},
xmajorgrids,
ymajorgrids,
legend style={at={(0.85,1)}, anchor=north east, legend cell align=left, align=left, draw=black}
]
\addplot [color=black, dashed, line width=2.0pt]
  table[row sep=crcr]{%
0.0949999988079071	0.210808470845222\\
0.100000001490116	0.199163168668747\\
0.104999996721745	0.18926927447319\\
0.109999999403954	0.180797472596169\\
0.115000002086163	0.173469662666321\\
0.119999997317791	0.348636746406555\\
0.125	0.97753894329071\\
0.129999995231628	1.32829701900482\\
0.135000005364418	1.57982647418976\\
0.140000000596046	1.74982357025146\\
0.144999995827675	1.85549604892731\\
0.150000005960464	1.91424489021301\\
0.155000001192093	1.94036269187927\\
0.159999996423721	1.96503162384033\\
0.165000006556511	1.98562169075012\\
0.170000001788139	1.98466336727142\\
0.174999997019768	1.95565736293793\\
0.180000007152557	1.89316427707672\\
0.185000002384186	1.79957854747772\\
0.189999997615814	1.66869747638702\\
0.194999992847443	23.0273017883301\\
0.200000002980232	36.1639213562012\\
0.204999998211861	43.5993995666504\\
0.209999993443489	46.394458770752\\
0.215000003576279	45.9041290283203\\
0.219999998807907	45.3420333862305\\
0.224999994039536	41.9886894226074\\
0.230000004172325	36.3740539550781\\
0.234999999403954	29.6083507537842\\
0.239999994635582	23.0812759399414\\
0.245000004768372	18.1484928131104\\
0.25	16.3407592773438\\
0.254999995231628	19.6784572601318\\
0.259999990463257	25.2396278381348\\
0.264999985694885	31.5829277038574\\
0.270000010728836	37.6180267333984\\
0.275000005960464	42.5632362365723\\
0.280000001192093	45.9191360473633\\
0.284999996423721	47.6446990966797\\
0.28999999165535	47.8047409057617\\
0.294999986886978	47.1846542358398\\
0.300000011920929	45.9069938659668\\
0.305000007152557	43.5206413269043\\
0.310000002384186	40.4366912841797\\
0.314999997615814	37.1287422180176\\
0.319999992847443	34.0797805786133\\
0.324999988079071	31.6923427581787\\
0.330000013113022	30.2283763885498\\
0.33500000834465	29.7853240966797\\
0.340000003576279	30.6444435119629\\
0.344999998807907	31.6913108825684\\
0.349999994039536	32.5991516113281\\
0.354999989271164	33.1139221191406\\
0.360000014305115	33.068790435791\\
0.365000009536743	32.5677871704102\\
0.370000004768372	31.742166519165\\
0.375	30.2829647064209\\
0.379999995231628	28.2741546630859\\
0.384999990463257	25.8749351501465\\
0.389999985694885	23.3039588928223\\
0.395000010728836	20.7891807556152\\
0.400000005960464	18.5904846191406\\
0.405000001192093	16.739559173584\\
0.409999996423721	15.4676923751831\\
0.41499999165535	14.6766405105591\\
0.419999986886978	14.3081064224243\\
0.425000011920929	14.3072605133057\\
0.430000007152557	14.2536268234253\\
0.435000002384186	14.0241680145264\\
0.439999997615814	13.7012805938721\\
0.444999992847443	13.0955152511597\\
0.449999988079071	12.1881246566772\\
0.455000013113022	11.0163707733154\\
0.46000000834465	9.66025638580322\\
0.465000003576279	8.23403835296631\\
0.469999998807907	6.86030292510986\\
0.474999994039536	5.66595792770386\\
0.479999989271164	4.74275255203247\\
0.485000014305115	4.14673519134521\\
0.490000009536743	3.88478088378906\\
0.495000004768372	4.05273866653442\\
0.5	4.3634181022644\\
0.504999995231628	4.69475698471069\\
0.509999990463257	4.97126340866089\\
0.514999985694885	5.14499616622925\\
0.519999980926514	5.19973564147949\\
0.524999976158142	5.15734720230103\\
0.529999971389771	5.10739469528198\\
0.535000026226044	4.97565269470215\\
0.540000021457672	4.81496095657349\\
0.545000016689301	4.68716907501221\\
0.550000011920929	4.64366006851196\\
0.555000007152557	4.79940605163574\\
0.560000002384186	5.09950256347656\\
0.564999997615814	5.52201509475708\\
0.569999992847443	6.04348754882813\\
0.574999988079071	6.62630319595337\\
0.579999983310699	7.24152994155884\\
0.584999978542328	7.86400890350342\\
0.589999973773956	8.46965503692627\\
0.595000028610229	9.05150890350342\\
0.600000023841858	9.60597610473633\\
0.605000019073486	10.1326370239258\\
0.610000014305115	10.6454477310181\\
0.615000009536743	11.1508188247681\\
0.620000004768372	11.6623401641846\\
0.625	12.1836967468262\\
0.629999995231628	12.7256288528442\\
0.634999990463257	13.2843866348267\\
0.639999985694885	13.8638038635254\\
0.644999980926514	14.4585218429565\\
0.649999976158142	15.0605497360229\\
0.654999971389771	15.6622333526611\\
0.660000026226044	16.2549800872803\\
0.665000021457672	16.830587387085\\
0.670000016689301	17.3825626373291\\
0.675000011920929	17.9052314758301\\
0.680000007152557	18.395601272583\\
0.685000002384186	18.8534603118896\\
0.689999997615814	19.2796001434326\\
0.694999992847443	19.6757183074951\\
0.699999988079071	20.0437355041504\\
0.704999983310699	20.3877506256104\\
0.709999978542328	20.7080783843994\\
0.714999973773956	21.0014514923096\\
0.720000028610229	21.2705497741699\\
0.725000023841858	21.5192699432373\\
0.730000019073486	21.7308254241943\\
0.735000014305115	21.9196033477783\\
0.740000009536743	22.0733814239502\\
0.745000004768372	22.1979370117188\\
0.75	22.2876758575439\\
0.754999995231628	22.3461647033691\\
0.759999990463257	22.3693981170654\\
0.764999985694885	22.3696823120117\\
0.769999980926514	22.3506698608398\\
0.774999976158142	22.3017272949219\\
0.779999971389771	22.2195510864258\\
0.785000026226044	22.107723236084\\
0.790000021457672	21.969409942627\\
0.795000016689301	21.8068313598633\\
0.800000011920929	21.6219062805176\\
0.805000007152557	21.4164943695068\\
0.810000002384186	21.1924667358398\\
0.814999997615814	20.9516716003418\\
0.819999992847443	20.6958122253418\\
0.824999988079071	20.4264221191406\\
0.829999983310699	20.1437187194824\\
0.834999978542328	19.8493690490723\\
0.839999973773956	19.5454902648926\\
0.845000028610229	19.2332096099854\\
0.850000023841858	18.9131469726563\\
0.855000019073486	18.5872669219971\\
0.860000014305115	18.2598133087158\\
0.865000009536743	17.9327087402344\\
0.870000004768372	17.6067638397217\\
0.875	17.2840328216553\\
0.879999995231628	16.9658527374268\\
0.884999990463257	16.6533432006836\\
0.889999985694885	16.3476791381836\\
0.894999980926514	16.0501117706299\\
0.899999976158142	15.7617454528809\\
0.904999971389771	15.4836311340332\\
0.910000026226044	15.2169990539551\\
0.915000021457672	14.9623966217041\\
0.920000016689301	14.7195014953613\\
0.925000011920929	14.4896812438965\\
0.930000007152557	14.2734937667847\\
0.935000002384186	14.0707416534424\\
0.939999997615814	13.8823261260986\\
0.944999992847443	13.7092514038086\\
0.949999988079071	13.553126335144\\
0.954999983310699	13.4144124984741\\
0.959999978542328	13.2930574417114\\
0.964999973773956	13.1889066696167\\
0.970000028610229	13.1027822494507\\
0.975000023841858	13.0339412689209\\
0.980000019073486	12.9822731018066\\
0.985000014305115	12.9479064941406\\
0.990000009536743	12.9301013946533\\
0.995000004768372	12.9288778305054\\
1	12.9443292617798\\
1.00499999523163	12.9765071868896\\
1.00999999046326	13.024787902832\\
1.01499998569489	13.0829219818115\\
1.01999998092651	13.1508941650391\\
1.02499997615814	13.2301692962646\\
1.02999997138977	13.3209733963013\\
1.0349999666214	13.4222860336304\\
1.03999996185303	13.5340166091919\\
1.04499995708466	13.6551475524902\\
1.04999995231628	13.7853193283081\\
1.05499994754791	13.9235410690308\\
1.05999994277954	14.0691318511963\\
1.06500005722046	14.2217273712158\\
1.07000005245209	14.3800220489502\\
1.07500004768372	14.5433254241943\\
1.08000004291534	14.7103700637817\\
1.08500003814697	14.8804092407227\\
1.0900000333786	15.0527181625366\\
1.09500002861023	15.2263784408569\\
1.10000002384186	15.4005289077759\\
1.10500001907349	15.5744562149048\\
1.11000001430511	15.7476921081543\\
1.11500000953674	15.9192771911621\\
1.12000000476837	16.0884456634521\\
1.125	16.2548713684082\\
1.12999999523163	16.4178466796875\\
1.13499999046326	16.5766162872314\\
1.13999998569489	16.7306175231934\\
1.14499998092651	16.8796253204346\\
1.14999997615814	17.0228595733643\\
1.15499997138977	17.1597423553467\\
1.1599999666214	17.2900371551514\\
1.16499996185303	17.4132957458496\\
1.16999995708466	17.5289363861084\\
1.17499995231628	17.6365814208984\\
1.17999994754791	17.7365493774414\\
1.18499994277954	17.828145980835\\
1.19000005722046	17.911039352417\\
1.19500005245209	17.9851512908936\\
1.20000004768372	18.0503978729248\\
1.20500004291534	18.1065254211426\\
1.21000003814697	18.1534442901611\\
1.2150000333786	18.1917266845703\\
1.22000002861023	18.2212963104248\\
1.22500002384186	18.2423782348633\\
1.23000001907349	18.2556133270264\\
1.23500001430511	18.2618370056152\\
1.24000000953674	18.2615795135498\\
1.24500000476837	18.2556476593018\\
1.25	18.2411880493164\\
1.25499999523163	18.2193202972412\\
1.25999999046326	18.1897430419922\\
1.26499998569489	18.1527576446533\\
1.26999998092651	18.1091480255127\\
1.27499997615814	18.0584850311279\\
1.27999997138977	18.001293182373\\
1.2849999666214	17.939416885376\\
1.28999996185303	17.8725757598877\\
1.29499995708466	17.8012771606445\\
1.29999995231628	17.7265243530273\\
1.30499994754791	17.6486949920654\\
1.30999994277954	17.5683708190918\\
1.31500005722046	17.4858055114746\\
1.32000005245209	17.40159034729\\
1.32500004768372	17.3162593841553\\
1.33000004291534	17.2300357818604\\
1.33500003814697	17.1433124542236\\
1.3400000333786	17.0565853118896\\
1.34500002861023	16.9703025817871\\
1.35000002384186	16.8849296569824\\
1.35500001907349	16.8009090423584\\
1.36000001430511	16.7185916900635\\
1.36500000953674	16.6380977630615\\
1.37000000476837	16.5598335266113\\
1.375	16.4839992523193\\
1.37999999523163	16.4103755950928\\
1.38499999046326	16.3393611907959\\
1.38999998569489	16.2711429595947\\
1.39499998092651	16.2063884735107\\
1.39999997615814	16.1452159881592\\
1.40499997138977	16.0879898071289\\
1.4099999666214	16.0361862182617\\
1.41499996185303	15.9899415969849\\
1.41999995708466	15.9494457244873\\
1.42499995231628	15.9133882522583\\
1.42999994754791	15.8824996948242\\
1.43499994277954	15.8566646575928\\
1.44000005722046	15.8354940414429\\
1.44500005245209	15.81907081604\\
1.45000004768372	15.806960105896\\
1.45500004291534	15.7985582351685\\
1.46000003814697	15.7937850952148\\
1.4650000333786	15.792820930481\\
1.47000002861023	15.795955657959\\
1.47500002384186	15.8032236099243\\
1.48000001907349	15.8149175643921\\
1.48500001430511	15.8314065933228\\
1.49000000953674	15.8531408309937\\
1.49500000476837	15.878228187561\\
1.5	15.906867980957\\
1.50499999523163	15.9391260147095\\
1.50999999046326	15.9752225875854\\
1.51499998569489	16.0149307250977\\
1.51999998092651	16.0579319000244\\
1.52499997615814	16.10400390625\\
1.52999997138977	16.1530151367188\\
1.5349999666214	16.2047710418701\\
1.53999996185303	16.2590885162354\\
1.54499995708466	16.3155403137207\\
1.54999995231628	16.3736343383789\\
1.55499994754791	16.4331474304199\\
1.55999994277954	16.493465423584\\
1.56500005722046	16.5542640686035\\
1.57000005245209	16.5828037261963\\
1.57500004768372	16.6176681518555\\
1.58000004291534	16.641674041748\\
1.58500003814697	16.6542949676514\\
1.5900000333786	16.6655540466309\\
1.59500002861023	16.6799659729004\\
1.60000002384186	16.7008686065674\\
1.60500001907349	16.7320537567139\\
1.61000001430511	16.7753620147705\\
1.61500000953674	16.826587677002\\
1.62000000476837	16.8874912261963\\
1.625	16.9474086761475\\
1.62999999523163	17.0015430450439\\
1.63499999046326	17.0435619354248\\
1.63999998569489	17.0702705383301\\
1.64499998092651	17.0959167480469\\
1.64999997615814	17.1142349243164\\
1.65499997138977	17.1257343292236\\
1.6599999666214	17.1319198608398\\
1.66499996185303	17.1336784362793\\
1.66999995708466	17.1303310394287\\
1.67499995231628	17.1260738372803\\
1.67999994754791	17.1244068145752\\
1.68499994277954	17.1218395233154\\
1.69000005722046	17.1168632507324\\
1.69500005245209	17.1103324890137\\
1.70000004768372	17.102201461792\\
1.70500004291534	17.0910930633545\\
1.71000003814697	17.0769023895264\\
1.7150000333786	17.0604095458984\\
1.72000002861023	17.0419311523438\\
1.72500002384186	17.0199069976807\\
1.73000001907349	16.994909286499\\
1.73500001430511	16.9666080474854\\
1.74000000953674	16.9350547790527\\
1.74500000476837	16.9005603790283\\
1.75	16.8637199401855\\
1.75499999523163	16.8242092132568\\
1.75999999046326	16.782543182373\\
1.76499998569489	16.7386245727539\\
1.76999998092651	16.6926822662354\\
1.77499997615814	16.6455917358398\\
1.77999997138977	16.5973949432373\\
1.7849999666214	16.5478343963623\\
1.78999996185303	16.4970760345459\\
1.79499995708466	16.4452381134033\\
1.79999995231628	16.3923072814941\\
1.80499994754791	16.3383903503418\\
1.80999994277954	16.2834300994873\\
1.81500005722046	16.2274646759033\\
1.82000005245209	16.1705455780029\\
1.82500004768372	16.1127376556396\\
1.83000004291534	16.0540714263916\\
1.83500003814697	15.9945106506348\\
1.8400000333786	15.9340600967407\\
1.84500002861023	15.8728055953979\\
1.85000002384186	15.8116273880005\\
1.85500001907349	15.7500848770142\\
1.86000001430511	15.6881589889526\\
1.86500000953674	15.6258058547974\\
1.87000000476837	15.5630521774292\\
1.875	15.5007162094116\\
1.87999999523163	15.4394311904907\\
1.88499999046326	15.37868309021\\
1.88999998569489	15.318413734436\\
1.89499998092651	15.2566413879395\\
1.89999997615814	15.194486618042\\
1.90499997138977	15.1319494247437\\
1.9099999666214	15.0690422058105\\
1.91499996185303	15.0057601928711\\
1.91999995708466	14.9421033859253\\
1.92499995231628	14.8786344528198\\
1.92999994754791	14.8153104782104\\
1.93499994277954	14.7519941329956\\
1.94000005722046	14.6883411407471\\
1.94500005245209	14.6242351531982\\
1.95000004768372	14.5598382949829\\
1.95500004291534	14.4956607818604\\
1.96000003814697	14.4325017929077\\
1.9650000333786	14.3702754974365\\
1.97000002861023	14.3080778121948\\
1.97500002384186	14.2427291870117\\
1.98000001907349	14.1749954223633\\
1.98500001430511	14.1046581268311\\
1.99000000953674	14.0338172912598\\
1.99500000476837	13.9613647460938\\
2	13.8873491287231\\
2.00500011444092	13.8137998580933\\
2.00999999046326	13.7403364181519\\
2.01500010490417	13.6673583984375\\
2.01999998092651	13.5987339019775\\
2.02500009536743	13.5347967147827\\
2.02999997138977	13.4762306213379\\
2.03500008583069	13.4193897247314\\
2.03999996185303	13.3659982681274\\
2.04500007629395	13.318751335144\\
2.04999995231628	13.2810363769531\\
2.0550000667572	13.2527933120728\\
2.05999994277954	13.233512878418\\
2.06500005722046	13.2240772247314\\
2.0699999332428	13.2227840423584\\
2.07500004768372	13.2300806045532\\
2.07999992370605	13.2462482452393\\
2.08500003814697	13.271032333374\\
2.08999991416931	13.3033447265625\\
2.09500002861023	13.3428449630737\\
2.09999990463257	13.3878087997437\\
2.10500001907349	13.437668800354\\
2.10999989509583	13.4931526184082\\
2.11500000953674	13.5543479919434\\
2.11999988555908	13.6215047836304\\
2.125	13.6952638626099\\
2.13000011444092	13.7720727920532\\
2.13499999046326	13.8501882553101\\
2.14000010490417	13.9256639480591\\
2.14499998092651	13.9967365264893\\
2.15000009536743	14.0694761276245\\
2.15499997138977	14.1316108703613\\
2.16000008583069	14.1647233963013\\
2.16499996185303	14.19260597229\\
2.17000007629395	14.2142915725708\\
2.17499995231628	14.2177677154541\\
2.1800000667572	14.2125511169434\\
2.18499994277954	14.2027111053467\\
2.19000005722046	14.1948299407959\\
2.1949999332428	14.1865186691284\\
2.20000004768372	14.1787261962891\\
2.20499992370605	14.1721782684326\\
2.21000003814697	14.1668033599854\\
2.21499991416931	14.1627712249756\\
2.22000002861023	14.1606845855713\\
2.22499990463257	14.1613578796387\\
2.23000001907349	14.1583385467529\\
2.23499989509583	14.1407852172852\\
2.24000000953674	14.1261901855469\\
2.24499988555908	14.1136178970337\\
2.25	14.0995206832886\\
2.25500011444092	14.0884704589844\\
2.25999999046326	14.0857858657837\\
2.26500010490417	14.0998287200928\\
2.26999998092651	14.135986328125\\
2.27500009536743	14.1954364776611\\
2.27999997138977	14.2739362716675\\
2.28500008583069	14.3596696853638\\
2.28999996185303	14.4565010070801\\
2.29500007629395	14.5651016235352\\
2.29999995231628	14.6810655593872\\
2.3050000667572	14.8056058883667\\
2.30999994277954	14.9376773834229\\
2.31500005722046	15.076003074646\\
2.3199999332428	15.2190895080566\\
2.32500004768372	15.3658065795898\\
2.32999992370605	15.5148820877075\\
2.33500003814697	15.6648101806641\\
2.33999991416931	15.8182783126831\\
2.34500002861023	15.9764423370361\\
2.34999990463257	16.1338787078857\\
2.35500001907349	16.2918071746826\\
2.35999989509583	16.4469528198242\\
2.36500000953674	16.5940799713135\\
2.36999988555908	16.7294158935547\\
2.375	16.8541221618652\\
2.38000011444092	16.9666748046875\\
2.38499999046326	17.0603561401367\\
2.39000010490417	17.132251739502\\
2.39499998092651	17.1799755096436\\
2.40000009536743	17.2047595977783\\
2.40499997138977	17.2063121795654\\
2.41000008583069	17.1892890930176\\
2.41499996185303	17.1596813201904\\
2.42000007629395	17.1227836608887\\
2.42499995231628	17.0919055938721\\
2.4300000667572	17.063756942749\\
2.43499994277954	17.0468616485596\\
2.44000005722046	17.0447025299072\\
2.4449999332428	17.0355339050293\\
2.45000004768372	17.0628852844238\\
2.45499992370605	17.0978050231934\\
2.46000003814697	17.1506824493408\\
2.46499991416931	17.2259998321533\\
2.47000002861023	17.316951751709\\
2.47499990463257	17.4215774536133\\
2.48000001907349	17.5381927490234\\
2.48499989509583	17.6631507873535\\
2.49000000953674	17.7991847991943\\
2.49499988555908	17.9496650695801\\
2.5	18.1066722869873\\
2.50500011444092	18.2496318817139\\
2.50999999046326	18.3770923614502\\
2.51500010490417	18.5031070709229\\
2.51999998092651	18.6155624389648\\
2.52500009536743	18.7146339416504\\
2.52999997138977	18.8034591674805\\
2.53500008583069	18.8833560943604\\
2.53999996185303	18.9529895782471\\
2.54500007629395	19.014856338501\\
2.54999995231628	19.0655384063721\\
2.5550000667572	19.1039543151855\\
2.55999994277954	19.1274547576904\\
2.56500005722046	19.1322078704834\\
2.5699999332428	19.1159954071045\\
2.57500004768372	19.0766048431396\\
2.57999992370605	19.0137996673584\\
2.58500003814697	18.929443359375\\
2.58999991416931	18.8160057067871\\
2.59500002861023	18.6891899108887\\
2.59999990463257	18.548490524292\\
2.60500001907349	18.3891334533691\\
2.60999989509583	18.2182464599609\\
2.61500000953674	18.036434173584\\
2.61999988555908	17.8477973937988\\
2.625	17.6531772613525\\
2.63000011444092	17.4528942108154\\
2.63499999046326	17.2369480133057\\
2.64000010490417	17.0218658447266\\
2.64499998092651	16.8002643585205\\
2.65000009536743	16.5490608215332\\
2.65499997138977	16.28005027771\\
2.66000008583069	15.9915266036987\\
2.66499996185303	15.6749801635742\\
2.67000007629395	15.3315658569336\\
2.67499995231628	14.97838306427\\
2.6800000667572	14.6281442642212\\
2.68499994277954	14.2587642669678\\
2.69000005722046	13.8965940475464\\
2.6949999332428	13.5430040359497\\
2.70000004768372	13.2003421783447\\
2.70499992370605	12.8712310791016\\
2.71000003814697	12.5545167922974\\
2.71499991416931	12.2478942871094\\
2.72000002861023	11.9520931243896\\
2.72499990463257	11.6654739379883\\
2.73000001907349	11.3856163024902\\
2.73499989509583	11.1126480102539\\
2.74000000953674	10.8467445373535\\
2.74499988555908	10.590274810791\\
2.75	10.3470811843872\\
2.75500011444092	10.1225128173828\\
2.75999999046326	9.92728805541992\\
2.76500010490417	9.76758670806885\\
2.76999998092651	9.64025688171387\\
2.77500009536743	9.54052448272705\\
2.77999997138977	9.46837902069092\\
2.78500008583069	9.42614650726318\\
2.78999996185303	9.4096565246582\\
2.79500007629395	9.41212558746338\\
2.79999995231628	9.42702865600586\\
2.8050000667572	9.45086002349854\\
2.80999994277954	9.4779167175293\\
2.81500005722046	9.50692749023438\\
2.8199999332428	9.53497886657715\\
2.82500004768372	9.56392383575439\\
2.82999992370605	9.59560775756836\\
2.83500003814697	9.64209938049316\\
2.83999991416931	9.70818519592285\\
2.84500002861023	9.80365943908691\\
2.84999990463257	9.94398784637451\\
2.85500001907349	10.1284952163696\\
2.85999989509583	10.3712120056152\\
2.86500000953674	10.681923866272\\
2.86999988555908	11.0527000427246\\
2.875	11.5004892349243\\
2.88000011444092	12.0217885971069\\
2.88499999046326	12.6211576461792\\
2.89000010490417	13.3029336929321\\
2.89499998092651	14.0668802261353\\
2.90000009536743	14.9118814468384\\
2.90499997138977	15.8408718109131\\
2.91000008583069	16.8494262695313\\
2.91499996185303	17.93137550354\\
2.92000007629395	19.0726432800293\\
2.92499995231628	20.2511749267578\\
2.9300000667572	21.4407901763916\\
2.93499994277954	22.6108493804932\\
2.94000005722046	23.7326488494873\\
2.9449999332428	24.776782989502\\
2.95000004768372	25.7252960205078\\
2.95499992370605	26.5538291931152\\
2.96000003814697	27.262674331665\\
2.96499991416931	27.9009628295898\\
2.97000002861023	28.4185390472412\\
2.97499990463257	28.7879943847656\\
2.98000001907349	29.0186767578125\\
2.98499989509583	29.1198272705078\\
2.99000000953674	29.1100730895996\\
2.99499988555908	29.0027866363525\\
3	28.8176422119141\\
3.00500011444092	28.5649719238281\\
3.00999999046326	28.2479629516602\\
3.01500010490417	27.8639316558838\\
3.01999998092651	27.4033374786377\\
3.02500009536743	26.8594570159912\\
3.02999997138977	26.213529586792\\
3.03500008583069	25.4497585296631\\
3.03999996185303	24.5747108459473\\
3.04500007629395	23.5935096740723\\
3.04999995231628	22.5348854064941\\
3.0550000667572	21.4287147521973\\
3.05999994277954	20.2883567810059\\
3.06500005722046	19.1591987609863\\
3.0699999332428	18.077220916748\\
3.07500004768372	17.0174999237061\\
3.07999992370605	16.0110321044922\\
3.08500003814697	15.0512342453003\\
3.08999991416931	14.1174764633179\\
3.09500002861023	13.2172002792358\\
3.09999990463257	12.3523759841919\\
3.10500001907349	11.5108880996704\\
3.10999989509583	10.7075929641724\\
3.11500000953674	9.9506893157959\\
3.11999988555908	9.25323295593262\\
3.125	8.64163589477539\\
3.13000011444092	8.16139316558838\\
3.13499999046326	7.8701229095459\\
3.14000010490417	7.66832590103149\\
3.14499998092651	7.52301216125488\\
3.15000009536743	7.3950080871582\\
3.15499997138977	7.26314783096313\\
3.16000008583069	7.1231746673584\\
3.16499996185303	6.98032855987549\\
3.17000007629395	6.83880233764648\\
3.17499995231628	6.70495176315308\\
3.1800000667572	6.61271572113037\\
3.18499994277954	6.58206558227539\\
3.19000005722046	6.60989999771118\\
3.1949999332428	6.67224073410034\\
3.20000004768372	6.78633117675781\\
3.20499992370605	6.98888826370239\\
3.21000003814697	7.25373792648315\\
3.21499991416931	7.63552618026733\\
3.22000002861023	8.1394739151001\\
3.22499990463257	8.76227378845215\\
3.23000001907349	9.50889015197754\\
3.23499989509583	10.3814096450806\\
3.24000000953674	11.3782291412354\\
3.24499988555908	12.4891271591187\\
3.25	13.6986856460571\\
3.25500011444092	14.9845514297485\\
3.25999999046326	16.3231945037842\\
3.26500010490417	17.6828289031982\\
3.26999998092651	19.0357475280762\\
3.27500009536743	20.3489646911621\\
3.27999997138977	21.6058406829834\\
3.28500008583069	22.7919769287109\\
3.28999996185303	23.8955497741699\\
3.29500007629395	24.9207324981689\\
3.29999995231628	25.8642807006836\\
3.3050000667572	26.719856262207\\
3.30999994277954	27.5181407928467\\
3.31500005722046	28.2250804901123\\
3.3199999332428	28.8498268127441\\
3.32500004768372	29.3916912078857\\
3.32999992370605	29.8327655792236\\
3.33500003814697	30.1662693023682\\
3.33999991416931	30.4019546508789\\
3.34500002861023	30.5322341918945\\
3.34999990463257	30.5107345581055\\
3.35500001907349	30.322359085083\\
3.35999989509583	29.9653186798096\\
3.36500000953674	29.4362144470215\\
3.36999988555908	28.7489242553711\\
3.375	27.921501159668\\
3.38000011444092	26.9796390533447\\
3.38499999046326	25.9513034820557\\
3.39000010490417	24.8595771789551\\
3.39499998092651	23.7269344329834\\
3.40000009536743	22.5685577392578\\
3.40499997138977	21.3892917633057\\
3.41000008583069	20.1958465576172\\
3.41499996185303	18.9898624420166\\
3.42000007629395	17.7740440368652\\
3.42499995231628	16.5517654418945\\
3.4300000667572	15.3354034423828\\
3.43499994277954	14.140172958374\\
3.44000005722046	12.9899959564209\\
3.4449999332428	11.9008474349976\\
3.45000004768372	10.8953218460083\\
3.45499992370605	9.99957847595215\\
3.46000003814697	9.32765865325928\\
3.46499991416931	8.80926609039307\\
3.47000002861023	8.33246326446533\\
3.47499990463257	7.88839244842529\\
3.48000001907349	7.45830297470093\\
3.48499989509583	7.04264354705811\\
3.49000000953674	6.59509134292603\\
3.49499988555908	6.19990301132202\\
3.5	5.90049314498901\\
3.50500011444092	5.64960670471191\\
3.50999999046326	5.45447731018066\\
3.51500010490417	5.3629994392395\\
3.51999998092651	5.31237554550171\\
3.52500009536743	5.37071228027344\\
3.52999997138977	5.47707414627075\\
3.53500008583069	5.654456615448\\
3.53999996185303	5.92503070831299\\
3.54500007629395	6.34403419494629\\
3.54999995231628	7.10956382751465\\
3.5550000667572	8.28886699676514\\
3.55999994277954	9.88178730010986\\
3.56500005722046	11.886908531189\\
3.5699999332428	14.245078086853\\
3.57500004768372	16.8275756835938\\
3.57999992370605	19.4795799255371\\
3.58500003814697	22.0230522155762\\
3.58999991416931	24.314323425293\\
3.59500002861023	26.2389526367188\\
3.59999990463257	27.7452354431152\\
3.60500001907349	28.809741973877\\
3.60999989509583	29.6008243560791\\
3.61500000953674	30.1587390899658\\
3.61999988555908	30.3460655212402\\
3.625	30.284236907959\\
3.63000011444092	30.1005268096924\\
3.63499999046326	29.9464626312256\\
3.64000010490417	29.9183292388916\\
3.64499998092651	30.0565395355225\\
3.65000009536743	30.4460678100586\\
3.65499997138977	30.8085784912109\\
3.66000008583069	30.9923496246338\\
3.66499996185303	30.9663314819336\\
3.67000007629395	30.665979385376\\
3.67499995231628	29.943208694458\\
3.6800000667572	28.7883014678955\\
3.68499994277954	27.254467010498\\
3.69000005722046	25.45578956604\\
3.6949999332428	23.5269584655762\\
3.70000004768372	21.6395206451416\\
3.70499992370605	19.8898162841797\\
3.71000003814697	18.3984909057617\\
3.71499991416931	17.1522045135498\\
3.72000002861023	16.1257457733154\\
3.72499990463257	15.2449932098389\\
3.73000001907349	14.4202899932861\\
3.73499989509583	13.573504447937\\
3.74000000953674	12.6534337997437\\
3.74499988555908	11.6521024703979\\
3.75	10.7547206878662\\
3.75500011444092	9.98768329620361\\
3.75999999046326	9.34440803527832\\
3.76500010490417	8.69696521759033\\
3.76999998092651	8.01664161682129\\
3.77500009536743	7.37617874145508\\
3.77999997138977	6.80278205871582\\
3.78500008583069	6.30861616134644\\
3.78999996185303	5.99357128143311\\
3.79500007629395	5.86622381210327\\
3.79999995231628	5.82616996765137\\
3.8050000667572	5.83536195755005\\
3.80999994277954	5.83715057373047\\
3.81500005722046	5.79465055465698\\
3.8199999332428	5.68407154083252\\
3.82500004768372	5.5344614982605\\
3.82999992370605	5.43974733352661\\
3.83500003814697	5.54732990264893\\
3.83999991416931	6.19071626663208\\
3.84500002861023	7.65631198883057\\
3.84999990463257	9.73860645294189\\
3.85500001907349	12.3811550140381\\
3.85999989509583	15.4352321624756\\
3.86500000953674	18.658540725708\\
3.86999988555908	21.8038272857666\\
3.875	24.6501750946045\\
3.88000011444092	27.0349731445313\\
3.88499999046326	28.87646484375\\
3.89000010490417	30.1575469970703\\
3.89499998092651	30.9771461486816\\
3.90000009536743	31.65380859375\\
3.90499997138977	31.842981338501\\
3.91000008583069	31.7513122558594\\
3.91499996185303	31.5755081176758\\
3.92000007629395	31.4741916656494\\
3.92499995231628	31.5812721252441\\
3.9300000667572	32.0590705871582\\
3.93499994277954	32.7052955627441\\
3.94000005722046	33.2281646728516\\
3.9449999332428	33.466251373291\\
3.95000004768372	33.3602638244629\\
3.95499992370605	33.000301361084\\
3.96000003814697	32.1064910888672\\
3.96499991416931	30.6660480499268\\
3.97000002861023	28.7583713531494\\
3.97499990463257	26.5265998840332\\
3.98000001907349	24.1562480926514\\
3.98499989509583	21.8335132598877\\
3.99000000953674	19.7067012786865\\
3.99499988555908	17.871072769165\\
4	16.3532524108887\\
4.00500011444092	15.1060600280762\\
4.01000022888184	14.0412397384644\\
4.0149998664856	13.0543766021729\\
4.01999998092651	12.047025680542\\
4.02500009536743	10.985689163208\\
4.03000020980835	9.97193717956543\\
4.03499984741211	9.13095188140869\\
4.03999996185303	8.36615562438965\\
4.04500007629395	7.61839818954468\\
4.05000019073486	6.86050415039063\\
4.05499982833862	6.09775114059448\\
4.05999994277954	5.35949611663818\\
4.06500005722046	4.75402927398682\\
4.07000017166138	4.32772541046143\\
4.07499980926514	4.13888502120972\\
4.07999992370605	4.08629274368286\\
4.08500003814697	4.14297485351563\\
4.09000015258789	4.20554876327515\\
4.09499979019165	4.11825370788574\\
4.09999990463257	3.75081562995911\\
4.10500001907349	3.11259889602661\\
4.1100001335144	2.37259793281555\\
4.11499977111816	2.2303352355957\\
4.11999988555908	2.59197854995728\\
4.125	4.98992490768433\\
4.13000011444092	9.52236557006836\\
4.13500022888184	14.8601531982422\\
4.1399998664856	20.6122779846191\\
4.14499998092651	26.103178024292\\
4.15000009536743	30.794132232666\\
4.15500020980835	34.3256034851074\\
4.15999984741211	36.5472412109375\\
4.16499996185303	37.5167999267578\\
4.17000007629395	38.2692108154297\\
4.17500019073486	37.6581039428711\\
4.17999982833862	35.9582023620605\\
4.18499994277954	33.6632118225098\\
4.19000005722046	31.4171371459961\\
4.19500017166138	29.8131008148193\\
4.19999980926514	29.3227100372314\\
4.20499992370605	30.729175567627\\
4.21000003814697	32.7000579833984\\
4.21500015258789	34.5461349487305\\
4.21999979019165	35.7628288269043\\
4.22499990463257	35.9888381958008\\
4.23000001907349	35.4546127319336\\
4.2350001335144	34.1498336791992\\
4.23999977111816	31.693962097168\\
4.24499988555908	28.1872425079346\\
4.25	24.3089809417725\\
4.25500011444092	20.3402786254883\\
4.26000022888184	16.8157501220703\\
4.2649998664856	14.0959424972534\\
4.26999998092651	12.3392419815063\\
4.27500009536743	11.527738571167\\
4.28000020980835	11.65260887146\\
4.28499984741211	11.7577981948853\\
4.28999996185303	11.5858459472656\\
4.29500007629395	11.0670137405396\\
4.30000019073486	10.3946294784546\\
4.30499982833862	9.51755428314209\\
4.30999994277954	8.15493297576904\\
4.31500005722046	6.60663414001465\\
4.32000017166138	5.2276463508606\\
4.32499980926514	4.21278285980225\\
4.32999992370605	3.67450976371765\\
4.33500003814697	3.66338014602661\\
4.34000015258789	3.97275519371033\\
4.34499979019165	4.44355583190918\\
4.34999990463257	4.81923866271973\\
4.35500001907349	4.80851697921753\\
4.3600001335144	4.17386293411255\\
4.36499977111816	2.89116430282593\\
4.36999988555908	1.89041125774384\\
4.375	2.38042688369751\\
4.38000011444092	2.78657341003418\\
4.38500022888184	3.04948282241821\\
4.3899998664856	3.16540884971619\\
4.39499998092651	10.7676610946655\\
4.40000009536743	18.5559616088867\\
4.40500020980835	26.1289901733398\\
4.40999984741211	32.6264572143555\\
4.41499996185303	37.4496231079102\\
4.42000007629395	40.403621673584\\
4.42500019073486	41.5890808105469\\
4.42999982833862	42.3087120056152\\
4.43499994277954	41.4402008056641\\
4.44000005722046	38.9983673095703\\
4.44500017166138	35.6510581970215\\
4.44999980926514	32.2997436523438\\
4.45499992370605	29.8474407196045\\
4.46000003814697	28.9678401947021\\
4.46500015258789	30.7689361572266\\
4.46999979019165	33.582145690918\\
4.47499990463257	36.2028350830078\\
4.48000001907349	38.0258560180664\\
4.4850001335144	38.5419654846191\\
4.48999977111816	37.8205108642578\\
4.49499988555908	36.4141998291016\\
4.5	33.4589691162109\\
4.50500011444092	29.2567691802979\\
4.51000022888184	24.3642444610596\\
4.5149998664856	19.4333572387695\\
4.51999998092651	15.0735893249512\\
4.52500009536743	11.8080291748047\\
4.53000020980835	9.7901725769043\\
4.53499984741211	9.04141139984131\\
4.53999996185303	9.534010887146\\
4.54500007629395	9.93964958190918\\
4.55000019073486	10.0262994766235\\
4.55499982833862	9.71344184875488\\
4.55999994277954	9.16488647460938\\
4.56500005722046	8.10668849945068\\
4.57000017166138	6.59055995941162\\
4.57499980926514	4.98768758773804\\
4.57999992370605	3.58964323997498\\
4.58500003814697	2.52997159957886\\
4.59000015258789	1.96985983848572\\
4.59499979019165	1.9956271648407\\
4.59999990463257	2.48849630355835\\
4.60500001907349	3.28801894187927\\
4.6100001335144	4.02410697937012\\
4.61499977111816	4.10940790176392\\
4.61999988555908	3.07068586349487\\
4.625	1.67162334918976\\
4.63000011444092	2.84069156646729\\
4.63500022888184	3.70859956741333\\
4.6399998664856	4.18614196777344\\
4.64499998092651	4.32810878753662\\
4.65000009536743	4.31521129608154\\
4.65500020980835	4.20156717300415\\
4.65999984741211	3.92763543128967\\
4.66499996185303	27.9817485809326\\
4.67000007629395	39.9588279724121\\
4.67500019073486	47.9723281860352\\
4.67999982833862	52.0194854736328\\
4.68499994277954	52.6519393920898\\
4.69000005722046	52.9814682006836\\
4.69500017166138	49.6381492614746\\
4.69999980926514	42.911018371582\\
4.70499992370605	34.0992851257324\\
4.71000003814697	25.3439655303955\\
4.71500015258789	18.9277820587158\\
4.71999979019165	16.4444618225098\\
4.72499990463257	20.8625888824463\\
4.73000001907349	27.2986259460449\\
4.7350001335144	33.7136840820313\\
4.73999977111816	38.415225982666\\
4.74499988555908	40.4339714050293\\
4.75	39.474910736084\\
4.75500011444092	37.6214447021484\\
4.76000022888184	33.0207214355469\\
4.7649998664856	26.1880912780762\\
4.76999998092651	18.192066192627\\
4.77500009536743	10.405047416687\\
4.78000020980835	4.0876522064209\\
4.78499984741211	1.02275002002716\\
4.78999996185303	0.867816507816315\\
4.79500007629395	0.672055423259735\\
4.80000019073486	5.06251096725464\\
4.80499982833862	7.90090799331665\\
4.80999994277954	8.63800525665283\\
4.81500005722046	7.07261896133423\\
4.82000017166138	4.91642761230469\\
4.82499980926514	3.22057247161865\\
4.82999992370605	2.57514977455139\\
4.83500003814697	3.03254532814026\\
4.84000015258789	4.39078617095947\\
4.84499979019165	6.1278281211853\\
4.84999990463257	7.42714786529541\\
4.85500001907349	7.13612270355225\\
4.8600001335144	4.38094472885132\\
4.86499977111816	0.36887726187706\\
4.86999988555908	0.917131900787354\\
4.875	2.02847695350647\\
4.88000011444092	3.17662954330444\\
4.88500022888184	3.99529933929443\\
4.8899998664856	4.42655277252197\\
4.89499998092651	4.51130199432373\\
4.90000009536743	4.51835632324219\\
4.90500020980835	4.35961103439331\\
4.90999984741211	4.04627656936646\\
4.91499996185303	3.61389183998108\\
4.92000007629395	27.7560844421387\\
4.92500019073486	42.4667587280273\\
4.92999982833862	50.3933067321777\\
4.93499994277954	52.7861022949219\\
4.94000005722046	51.5713272094727\\
4.94500017166138	50.2196426391602\\
4.94999980926514	45.274600982666\\
4.95499992370605	38.0064277648926\\
4.96000003814697	30.0948333740234\\
4.96500015258789	23.4039764404297\\
4.96999979019165	19.4124164581299\\
4.97499990463257	19.7212562561035\\
4.98000001907349	23.5004234313965\\
4.9850001335144	27.8123779296875\\
4.98999977111816	31.5497169494629\\
4.99499988555908	33.956600189209\\
5	34.7680244445801\\
5.00500011444092	33.9990196228027\\
5.01000022888184	32.4358444213867\\
5.0149998664856	30.1777935028076\\
5.01999998092651	26.9659957885742\\
5.02500009536743	23.168815612793\\
5.03000020980835	19.1963405609131\\
5.03499984741211	15.3938302993774\\
5.03999996185303	12.0354537963867\\
5.04500007629395	9.27918529510498\\
5.05000019073486	7.20251750946045\\
5.05499982833862	5.7910475730896\\
5.05999994277954	4.95076942443848\\
5.06500005722046	4.52286005020142\\
5.07000017166138	4.31958818435669\\
5.07499980926514	4.15390396118164\\
5.07999992370605	3.86151051521301\\
5.08500003814697	3.38193845748901\\
5.09000015258789	2.60770511627197\\
5.09499979019165	1.53677725791931\\
5.09999990463257	0.274307608604431\\
5.10500001907349	-0.0331083312630653\\
5.1100001335144	-0.0303391125053167\\
5.11499977111816	-0.0263062529265881\\
5.11999988555908	-0.0216858852654696\\
5.125	-0.0169122088700533\\
5.13000011444092	-0.0127697288990021\\
5.13500022888184	0.970218598842621\\
5.1399998664856	1.82036638259888\\
5.14499998092651	2.30498719215393\\
5.15000009536743	2.51955437660217\\
5.15500020980835	2.53288793563843\\
5.15999984741211	13.1416130065918\\
5.16499996185303	20.4856510162354\\
5.17000007629395	26.0622901916504\\
5.17500019073486	29.6251773834229\\
5.17999982833862	31.2480545043945\\
5.18499994277954	31.3448352813721\\
5.19000005722046	30.5975151062012\\
5.19500017166138	29.4223098754883\\
5.19999980926514	27.4558506011963\\
5.20499992370605	25.1642398834229\\
5.21000003814697	22.9820117950439\\
5.21500015258789	21.2467231750488\\
5.21999979019165	20.1497840881348\\
5.22499990463257	19.9067478179932\\
5.23000001907349	20.3432292938232\\
5.2350001335144	20.9492359161377\\
5.23999977111816	21.5462493896484\\
5.24499988555908	22.0164661407471\\
5.25	22.2722072601318\\
5.25500011444092	22.2769756317139\\
5.26000022888184	22.2673282623291\\
5.2649998664856	21.9854335784912\\
5.26999998092651	21.3715591430664\\
5.27500009536743	20.4299087524414\\
5.28000020980835	19.2044429779053\\
5.28499984741211	17.7857532501221\\
5.28999996185303	16.2964458465576\\
5.29500007629395	14.8780374526978\\
5.30000019073486	13.6950616836548\\
5.30499982833862	12.870005607605\\
5.30999994277954	12.5334606170654\\
5.31500005722046	12.5963878631592\\
5.32000017166138	13.5024003982544\\
5.32499980926514	14.4992809295654\\
5.32999992370605	15.4287567138672\\
5.33500003814697	16.1250877380371\\
5.34000015258789	16.4767017364502\\
5.34499979019165	16.6365928649902\\
5.34999990463257	16.2553157806396\\
5.35500001907349	15.2929964065552\\
5.3600001335144	13.8165245056152\\
5.36499977111816	12.0034618377686\\
5.36999988555908	10.0620059967041\\
5.375	8.34380149841309\\
5.38000011444092	6.99687004089355\\
5.38500022888184	6.1983814239502\\
5.3899998664856	6.00079917907715\\
5.39499998092651	6.40060520172119\\
5.40000009536743	6.69044351577759\\
5.40500020980835	6.62217283248901\\
5.40999984741211	6.36655378341675\\
5.41499996185303	5.53895902633667\\
5.42000007629395	4.07068967819214\\
5.42500019073486	2.05052733421326\\
5.42999982833862	0.252480328083038\\
5.43499994277954	0.201652303338051\\
5.44000005722046	0.149041846394539\\
5.44500017166138	0.120646946132183\\
5.44999980926514	0.114102683961391\\
5.45499992370605	0.10907106846571\\
5.46000003814697	0.105223141610622\\
5.46500015258789	0.102389559149742\\
5.46999979019165	0.099647156894207\\
5.47499990463257	0.0988680794835091\\
5.48000001907349	0.0949338227510452\\
5.4850001335144	0.0950156152248383\\
5.48999977111816	0.0929016917943954\\
5.49499988555908	0.0921168476343155\\
5.5	0.0908519104123116\\
5.50500011444092	0.0904062688350677\\
5.51000022888184	0.0905360206961632\\
5.5149998664856	0.0893425643444061\\
5.51999998092651	0.151441261172295\\
5.52500009536743	0.335023611783981\\
5.53000020980835	0.438723802566528\\
5.53499984741211	0.528062403202057\\
5.53999996185303	0.600764155387878\\
5.54500007629395	0.658122956752777\\
5.55000019073486	0.700239777565002\\
5.55499982833862	0.729918777942657\\
5.55999994277954	0.753552794456482\\
5.56500005722046	0.768722116947174\\
5.57000017166138	0.771436214447021\\
5.57499980926514	0.7649205327034\\
5.57999992370605	0.746456801891327\\
5.58500003814697	0.720195412635803\\
5.59000015258789	0.683425843715668\\
5.59499979019165	0.646574676036835\\
5.59999990463257	0.602410912513733\\
5.60500001907349	0.559636116027832\\
5.6100001335144	0.522733807563782\\
5.61499977111816	0.489389687776566\\
5.61999988555908	0.460096955299377\\
5.625	0.439359992742538\\
5.63000011444092	0.426081210374832\\
5.63500022888184	0.412046849727631\\
5.6399998664856	0.40771210193634\\
5.64499998092651	0.40624526143074\\
5.65000009536743	0.404474020004272\\
5.65500020980835	0.40319412946701\\
5.65999984741211	0.401808381080627\\
5.66499996185303	0.401206851005554\\
5.67000007629395	0.885265469551086\\
5.67500019073486	11.4442567825317\\
5.67999982833862	13.8546323776245\\
5.68499994277954	13.9838008880615\\
5.69000005722046	13.3761596679688\\
5.69500017166138	12.227502822876\\
5.69999980926514	10.5691699981689\\
5.70499992370605	9.00704479217529\\
5.71000003814697	7.96562576293945\\
5.71500015258789	7.84335088729858\\
5.71999979019165	8.41158866882324\\
5.72499990463257	9.05075550079346\\
5.73000001907349	9.48386669158936\\
5.7350001335144	9.58194255828857\\
5.73999977111816	9.32773971557617\\
5.74499988555908	8.76917362213135\\
5.75	8.01419448852539\\
5.75500011444092	7.18797016143799\\
5.76000022888184	6.37321710586548\\
5.7649998664856	5.59070253372192\\
5.76999998092651	4.76152229309082\\
5.77500009536743	3.84696793556213\\
5.78000020980835	2.72351551055908\\
5.78499984741211	1.35421764850616\\
5.78999996185303	-0.272826820611954\\
5.79500007629395	-2.17766094207764\\
5.80000019073486	-4.3294472694397\\
5.80499982833862	-6.65338516235352\\
5.80999994277954	-9.08022975921631\\
5.81500005722046	-11.5023384094238\\
5.82000017166138	-13.786319732666\\
5.82499980926514	-15.7687530517578\\
5.82999992370605	-17.2999668121338\\
5.83500003814697	-18.1972541809082\\
5.84000015258789	-18.297492980957\\
5.84499979019165	-17.4735794067383\\
5.84999990463257	-15.6822690963745\\
5.85500001907349	-13.0835332870483\\
5.8600001335144	-9.52003002166748\\
5.86499977111816	-5.33265733718872\\
5.86999988555908	-1.01810646057129\\
5.875	2.58010005950928\\
5.88000011444092	4.57823085784912\\
5.88500022888184	4.27138710021973\\
5.8899998664856	2.09094738960266\\
5.89499998092651	2.09319686889648\\
5.90000009536743	2.16262125968933\\
5.90500020980835	2.32292747497559\\
5.90999984741211	2.6876654624939\\
5.91499996185303	3.26720833778381\\
5.92000007629395	3.91317772865295\\
5.92500019073486	4.47273874282837\\
5.92999982833862	4.86308431625366\\
5.93499994277954	5.04685544967651\\
5.94000005722046	5.05933713912964\\
5.94500017166138	4.9679102897644\\
5.94999980926514	4.85073089599609\\
5.95499992370605	4.79349517822266\\
5.96000003814697	4.92832040786743\\
5.96500015258789	5.23717498779297\\
5.96999979019165	5.68428659439087\\
5.97499990463257	6.22804880142212\\
5.98000001907349	50.0682334899902\\
5.9850001335144	79.9624404907227\\
5.98999977111816	95.9791564941406\\
5.99499988555908	100.498046875\\
6	98.8110504150391\\
6.00500011444092	96.4043426513672\\
6.01000022888184	93.1524505615234\\
6.0149998664856	85.6021881103516\\
6.01999998092651	73.7759399414063\\
6.02500009536743	59.2618217468262\\
6.03000020980835	44.751049041748\\
6.03499984741211	33.3226013183594\\
6.03999996185303	27.8268222808838\\
6.04500007629395	32.2749710083008\\
6.05000019073486	41.9914360046387\\
6.05499982833862	52.5744247436523\\
6.05999994277954	61.7754249572754\\
6.06500005722046	68.0853424072266\\
6.07000017166138	71.0745620727539\\
6.07499980926514	70.8101959228516\\
6.07999992370605	68.5271530151367\\
6.08500003814697	65.6849594116211\\
6.09000015258789	60.5261039733887\\
6.09499979019165	53.5714530944824\\
6.09999990463257	45.5162963867188\\
6.10500001907349	37.2174682617188\\
6.1100001335144	29.7692070007324\\
6.11499977111816	23.8858470916748\\
6.11999988555908	20.144079208374\\
6.125	18.630033493042\\
6.13000011444092	19.9405765533447\\
6.13500022888184	21.8382129669189\\
6.1399998664856	23.379451751709\\
6.14499998092651	24.0639247894287\\
6.15000009536743	23.641845703125\\
6.15500020980835	22.5981388092041\\
6.15999984741211	20.7439708709717\\
6.16499996185303	17.8094081878662\\
6.17000007629395	14.0004615783691\\
6.17500019073486	9.65493869781494\\
6.17999982833862	5.19192790985107\\
6.18499994277954	1.06674683094025\\
6.19000005722046	0.440609216690063\\
6.19500017166138	0.386078298091888\\
6.19999980926514	0.297625482082367\\
6.20499992370605	0.184639945626259\\
6.21000003814697	0.0590059794485569\\
6.21500015258789	-0.0050798961892724\\
6.21999979019165	-0.0110730268061161\\
6.22499990463257	-0.0147657878696918\\
6.23000001907349	-0.0167017355561256\\
6.2350001335144	-0.0174275301396847\\
6.23999977111816	-0.0173142850399017\\
6.24499988555908	-0.0167580712586641\\
6.25	-0.0160193145275116\\
6.25500011444092	-0.0150645533576608\\
6.26000022888184	-0.0140823740512133\\
6.2649998664856	-0.0129890646785498\\
6.26999998092651	-0.0118947252631187\\
6.27500009536743	-0.0108957551419735\\
6.28000020980835	-0.00996336061507463\\
6.28499984741211	-0.00906001962721348\\
6.28999996185303	-0.00816772226244211\\
6.29500007629395	-0.00733370799571276\\
6.30000019073486	-0.00660984171554446\\
6.30499982833862	-0.00597657728940248\\
6.30999994277954	-0.00537917157635093\\
6.31500005722046	-0.0048343138769269\\
6.32000017166138	-0.00437704892829061\\
6.32499980926514	-0.00399879273027182\\
6.32999992370605	-0.00366425467655063\\
6.33500003814697	-0.00329068140126765\\
6.34000015258789	-0.00284017389640212\\
6.34499979019165	-0.0024455594830215\\
6.34999990463257	-0.00211632461287081\\
6.35500001907349	-0.00188294902909547\\
6.3600001335144	5.09061431884766\\
6.36499977111816	7.21974897384644\\
6.36999988555908	8.59541034698486\\
6.375	9.26888370513916\\
6.38000011444092	9.32822513580322\\
6.38500022888184	9.1403341293335\\
6.3899998664856	9.33874702453613\\
6.39499998092651	8.93625450134277\\
6.40000009536743	8.62845039367676\\
6.40500020980835	8.87300109863281\\
6.40999984741211	10.212384223938\\
6.41499996185303	12.1271076202393\\
6.42000007629395	14.4754638671875\\
6.42500019073486	17.0397567749023\\
6.42999982833862	19.5879173278809\\
6.43499994277954	21.9646968841553\\
6.44000005722046	24.027717590332\\
6.44500017166138	25.7205638885498\\
6.44999980926514	27.0165424346924\\
6.45499992370605	27.9725360870361\\
6.46000003814697	28.6326694488525\\
6.46500015258789	29.0711479187012\\
6.46999979019165	29.3810176849365\\
6.47499990463257	29.6259822845459\\
6.48000001907349	29.8620338439941\\
6.4850001335144	30.1412487030029\\
6.48999977111816	30.4829406738281\\
6.49499988555908	30.8872203826904\\
6.5	31.3401145935059\\
6.50500011444092	31.815034866333\\
6.51000022888184	32.2738418579102\\
6.5149998664856	32.6840629577637\\
6.51999998092651	33.0153846740723\\
6.52500009536743	33.242805480957\\
6.53000020980835	33.3446426391602\\
6.53499984741211	33.3372039794922\\
6.53999996185303	33.2135581970215\\
6.54500007629395	32.9839096069336\\
6.55000019073486	32.6651649475098\\
6.55499982833862	32.2842292785645\\
6.55999994277954	31.8716106414795\\
6.56500005722046	31.4071750640869\\
6.57000017166138	30.8933887481689\\
6.57499980926514	30.337474822998\\
6.57999992370605	29.7838802337646\\
6.58500003814697	29.2126064300537\\
6.59000015258789	28.6366367340088\\
6.59499979019165	28.0635566711426\\
6.59999990463257	27.48561668396\\
6.60500001907349	26.8997383117676\\
6.6100001335144	26.3087940216064\\
6.61499977111816	25.7158508300781\\
6.61999988555908	25.1023750305176\\
6.625	24.4835014343262\\
6.63000011444092	23.8535690307617\\
6.63500022888184	23.2173385620117\\
6.6399998664856	22.5903606414795\\
6.64499998092651	21.9616794586182\\
6.65000009536743	21.3443508148193\\
6.65500020980835	20.7462139129639\\
6.65999984741211	20.1621341705322\\
6.66499996185303	19.6024799346924\\
6.67000007629395	19.0847434997559\\
6.67500019073486	18.5970916748047\\
6.67999982833862	18.1575660705566\\
6.68499994277954	17.764461517334\\
6.69000005722046	17.4064407348633\\
6.69500017166138	17.1217727661133\\
6.69999980926514	16.9233112335205\\
6.70499992370605	16.7730274200439\\
6.71000003814697	16.6697978973389\\
6.71500015258789	16.5996952056885\\
6.71999979019165	16.5695495605469\\
6.72499990463257	16.5776691436768\\
6.73000001907349	16.6118793487549\\
6.7350001335144	16.6777210235596\\
6.73999977111816	16.7750015258789\\
6.74499988555908	16.9003467559814\\
6.75	17.0609378814697\\
6.75500011444092	17.247386932373\\
6.76000022888184	17.4405956268311\\
6.7649998664856	17.6766471862793\\
6.76999998092651	17.9319915771484\\
6.77500009536743	18.2159290313721\\
6.78000020980835	18.5268287658691\\
6.78499984741211	18.8559226989746\\
6.78999996185303	19.2055625915527\\
6.79500007629395	19.5704898834229\\
6.80000019073486	19.9452991485596\\
6.80499982833862	20.3261470794678\\
6.80999994277954	20.7153778076172\\
6.81500005722046	21.0997428894043\\
6.82000017166138	21.4839954376221\\
6.82499980926514	21.862585067749\\
6.82999992370605	22.2313709259033\\
6.83500003814697	22.5868682861328\\
6.84000015258789	22.9260292053223\\
6.84499979019165	23.2570419311523\\
6.84999990463257	23.5712642669678\\
6.85500001907349	23.8670520782471\\
6.8600001335144	24.1428604125977\\
6.86499977111816	24.3873023986816\\
6.86999988555908	24.6028823852539\\
6.875	24.7849063873291\\
6.88000011444092	24.9230079650879\\
6.88500022888184	25.0127735137939\\
6.8899998664856	25.0472297668457\\
6.89499998092651	25.0218467712402\\
6.90000009536743	24.9346103668213\\
6.90500020980835	24.78542137146\\
6.90999984741211	24.5740356445313\\
6.91499996185303	24.305103302002\\
6.92000007629395	23.9837818145752\\
6.92500019073486	23.6159801483154\\
6.92999982833862	23.2091083526611\\
6.93499994277954	22.7708206176758\\
6.94000005722046	22.3066596984863\\
6.94500017166138	21.8205718994141\\
6.94999980926514	21.3155689239502\\
6.95499992370605	20.7917003631592\\
6.96000003814697	20.2520427703857\\
6.96500015258789	19.6980876922607\\
6.96999979019165	19.1269454956055\\
6.97499990463257	18.5349178314209\\
6.98000001907349	17.9269599914551\\
6.9850001335144	17.2998638153076\\
6.98999977111816	16.6545677185059\\
6.99499988555908	15.9962368011475\\
7	15.3294277191162\\
7.00500011444092	14.6582775115967\\
7.01000022888184	13.9885034561157\\
7.0149998664856	13.3268804550171\\
7.01999998092651	12.6785345077515\\
7.02500009536743	12.0466985702515\\
7.03000020980835	11.4341268539429\\
7.03499984741211	10.8432998657227\\
7.03999996185303	10.27587890625\\
7.04500007629395	9.73263263702393\\
7.05000019073486	9.21360397338867\\
7.05499982833862	8.7181396484375\\
7.05999994277954	8.24498462677002\\
7.06500005722046	7.79331398010254\\
7.07000017166138	7.36239242553711\\
7.07499980926514	6.9520959854126\\
7.07999992370605	6.56279754638672\\
7.08500003814697	6.1956148147583\\
7.09000015258789	5.85238075256348\\
7.09499979019165	5.53474521636963\\
7.09999990463257	5.24481868743896\\
7.10500001907349	4.98460054397583\\
7.1100001335144	4.75594282150269\\
7.11499977111816	4.55954456329346\\
7.11999988555908	4.39666986465454\\
7.125	4.26787281036377\\
7.13000011444092	4.16923093795776\\
7.13500022888184	4.09993600845337\\
7.1399998664856	4.05867195129395\\
7.14499998092651	4.04170751571655\\
7.15000009536743	4.04910850524902\\
7.15500020980835	4.08699893951416\\
7.15999984741211	4.14058494567871\\
7.16499996185303	4.20240449905396\\
7.17000007629395	4.28377342224121\\
7.17500019073486	4.38162899017334\\
7.17999982833862	4.496018409729\\
7.18499994277954	4.62868499755859\\
7.19000005722046	4.77840614318848\\
7.19500017166138	4.94496154785156\\
7.19999980926514	5.12739276885986\\
7.20499992370605	5.32210350036621\\
7.21000003814697	5.52958345413208\\
7.21500015258789	5.74662399291992\\
7.21999979019165	5.96827793121338\\
7.22499990463257	6.19629573822021\\
7.23000001907349	6.42803239822388\\
7.2350001335144	6.66183233261108\\
7.23999977111816	6.89741563796997\\
7.24499988555908	7.1325740814209\\
7.25	7.36621904373169\\
7.25500011444092	7.59877681732178\\
7.26000022888184	7.82752704620361\\
7.2649998664856	8.05286026000977\\
7.26999998092651	8.27532291412354\\
7.27500009536743	8.49230003356934\\
7.28000020980835	8.70232772827148\\
7.28499984741211	8.90554046630859\\
7.28999996185303	9.10054492950439\\
7.29500007629395	9.28769588470459\\
7.30000019073486	9.46335792541504\\
7.30499982833862	9.62868404388428\\
7.30999994277954	9.7825756072998\\
7.31500005722046	9.92526054382324\\
7.32000017166138	10.0551280975342\\
7.32499980926514	10.1726140975952\\
7.32999992370605	10.2763643264771\\
7.33500003814697	10.3661956787109\\
7.34000015258789	10.4423179626465\\
7.34499979019165	10.5044097900391\\
7.34999990463257	10.552640914917\\
7.35500001907349	10.5871458053589\\
7.3600001335144	10.6091260910034\\
7.36499977111816	10.618821144104\\
7.36999988555908	10.6169633865356\\
7.375	10.6044406890869\\
7.38000011444092	10.5817775726318\\
7.38500022888184	10.5461301803589\\
7.3899998664856	10.4957571029663\\
7.39499998092651	10.4313411712646\\
7.40000009536743	10.3564052581787\\
7.40500020980835	10.2701072692871\\
7.40999984741211	10.1735486984253\\
7.41499996185303	10.0674180984497\\
7.42000007629395	9.95263385772705\\
7.42500019073486	9.83026790618896\\
7.42999982833862	9.70095634460449\\
7.43499994277954	9.56491470336914\\
7.44000005722046	9.42297172546387\\
7.44500017166138	9.27606010437012\\
7.44999980926514	9.12484455108643\\
7.45499992370605	8.97055053710938\\
7.46000003814697	8.81449604034424\\
7.46500015258789	8.65752029418945\\
7.46999979019165	8.49986934661865\\
7.47499990463257	8.34240913391113\\
7.48000001907349	8.18577194213867\\
7.4850001335144	8.03046131134033\\
7.48999977111816	7.87706613540649\\
7.49499988555908	7.72605752944946\\
7.5	7.57777070999146\\
7.50500011444092	7.43243598937988\\
7.51000022888184	7.29053068161011\\
7.5149998664856	7.1532998085022\\
7.51999998092651	7.02150058746338\\
7.52500009536743	6.89741945266724\\
7.53000020980835	6.77727842330933\\
7.53499984741211	6.66519641876221\\
7.53999996185303	6.5594482421875\\
7.54500007629395	6.46214866638184\\
7.55000019073486	6.37260103225708\\
7.55499982833862	6.29069423675537\\
7.55999994277954	6.21683597564697\\
7.56500005722046	6.15090370178223\\
7.57000017166138	6.09254932403564\\
7.57499980926514	6.04181432723999\\
7.57999992370605	5.99867820739746\\
7.58500003814697	5.96305561065674\\
7.59000015258789	5.93507766723633\\
7.59499979019165	5.91471338272095\\
7.59999990463257	5.9017915725708\\
7.60500001907349	5.89655017852783\\
7.6100001335144	5.90041971206665\\
7.61499977111816	5.91490793228149\\
7.61999988555908	5.93726062774658\\
7.625	5.96422958374023\\
7.63000011444092	5.99670648574829\\
7.63500022888184	6.03501272201538\\
7.6399998664856	6.07889699935913\\
7.64499998092651	6.12831163406372\\
7.65000009536743	6.18206644058228\\
7.65500020980835	6.239821434021\\
7.65999984741211	6.30147218704224\\
7.66499996185303	6.36652755737305\\
7.67000007629395	6.43447971343994\\
7.67500019073486	6.50515270233154\\
7.67999982833862	6.57859516143799\\
7.68499994277954	6.65452337265015\\
7.69000005722046	6.73270559310913\\
7.69500017166138	6.81206798553467\\
7.69999980926514	6.89224004745483\\
7.70499992370605	6.97307348251343\\
7.71000003814697	7.05448293685913\\
7.71500015258789	7.13704347610474\\
7.71999979019165	7.2195405960083\\
7.72499990463257	7.30064153671265\\
7.73000001907349	7.38020944595337\\
7.7350001335144	7.45776748657227\\
7.73999977111816	7.53444385528564\\
7.74499988555908	7.60935974121094\\
7.75	7.68080806732178\\
7.75500011444092	7.74968194961548\\
7.76000022888184	7.81632423400879\\
7.7649998664856	7.87987565994263\\
7.76999998092651	7.94009447097778\\
7.77500009536743	7.99714851379395\\
7.78000020980835	8.05100440979004\\
7.78499984741211	8.10138988494873\\
7.78999996185303	8.14847278594971\\
7.79500007629395	8.19225311279297\\
7.80000019073486	8.23235607147217\\
7.80499982833862	8.26892566680908\\
7.80999994277954	8.30199432373047\\
7.81500005722046	8.33139419555664\\
7.82000017166138	8.35710620880127\\
7.82499980926514	8.37919235229492\\
7.82999992370605	8.39779949188232\\
7.83500003814697	8.4130973815918\\
7.84000015258789	8.42495059967041\\
7.84499979019165	8.43343925476074\\
7.84999990463257	8.43897533416748\\
7.85500001907349	8.44148349761963\\
7.8600001335144	8.4410572052002\\
7.86499977111816	8.43795585632324\\
7.86999988555908	8.43224334716797\\
7.875	8.42404460906982\\
7.88000011444092	8.41388130187988\\
7.88500022888184	8.40200042724609\\
7.8899998664856	8.38855266571045\\
7.89499998092651	8.37360286712646\\
7.90000009536743	8.35718154907227\\
7.90500020980835	8.3395471572876\\
7.90999984741211	8.32079696655273\\
7.91499996185303	8.30088996887207\\
7.92000007629395	8.28002071380615\\
7.92500019073486	8.25838470458984\\
7.92999982833862	8.23740673065186\\
7.93499994277954	8.21627998352051\\
7.94000005722046	8.19500541687012\\
7.94500017166138	8.17509174346924\\
7.94999980926514	8.15612030029297\\
7.95499992370605	8.13787937164307\\
7.96000003814697	8.12125205993652\\
7.96500015258789	8.10651206970215\\
7.96999979019165	8.0932502746582\\
7.97499990463257	8.08206176757813\\
7.98000001907349	8.07389068603516\\
7.9850001335144	8.06805038452148\\
7.98999977111816	8.06457424163818\\
7.99499988555908	8.06366539001465\\
8	8.06520652770996\\
8.00500011444092	8.0691967010498\\
8.01000022888184	8.07654285430908\\
8.01500034332275	8.08717823028564\\
8.02000045776367	8.10124588012695\\
8.02499961853027	8.11865520477295\\
8.02999973297119	8.13946151733398\\
8.03499984741211	8.16384792327881\\
8.03999996185303	8.19181728363037\\
8.04500007629395	8.2233190536499\\
8.05000019073486	8.25847911834717\\
8.05500030517578	8.29730129241943\\
8.0600004196167	8.33962249755859\\
8.0649995803833	8.38553524017334\\
8.06999969482422	8.43504333496094\\
8.07499980926514	8.48843574523926\\
8.07999992370605	8.54565715789795\\
8.08500003814697	8.60676670074463\\
8.09000015258789	8.67135334014893\\
8.09500026702881	8.73936748504639\\
8.10000038146973	8.81082439422607\\
8.10499954223633	8.88579654693604\\
8.10999965667725	8.9643611907959\\
8.11499977111816	9.04644012451172\\
8.11999988555908	9.13196849822998\\
8.125	9.220778465271\\
8.13000011444092	9.31288528442383\\
8.13500022888184	9.40824222564697\\
8.14000034332275	9.50685501098633\\
8.14500045776367	9.60866069793701\\
8.14999961853027	9.71361827850342\\
8.15499973297119	9.82148838043213\\
8.15999984741211	9.93222236633301\\
8.16499996185303	10.0457525253296\\
8.17000007629395	10.1623077392578\\
8.17500019073486	10.2820148468018\\
8.18000030517578	10.4047813415527\\
8.1850004196167	10.5304317474365\\
8.1899995803833	10.6585931777954\\
8.19499969482422	10.7893447875977\\
8.19999980926514	10.9227876663208\\
8.20499992370605	11.0591068267822\\
8.21000003814697	11.1981964111328\\
8.21500015258789	11.3395957946777\\
8.22000026702881	11.4828624725342\\
8.22500038146973	11.6279850006104\\
8.22999954223633	11.7755947113037\\
8.23499965667725	11.9259910583496\\
8.23999977111816	12.0787019729614\\
8.24499988555908	12.2334957122803\\
8.25	12.3906993865967\\
8.25500011444092	12.5505990982056\\
8.26000022888184	12.713062286377\\
8.26500034332275	12.8781185150146\\
8.27000045776367	13.0457143783569\\
8.27499961853027	13.2157964706421\\
8.27999973297119	13.3879995346069\\
8.28499984741211	13.562406539917\\
8.28999996185303	13.7393760681152\\
8.29500007629395	13.9177484512329\\
8.30000019073486	14.0968742370605\\
8.30500030517578	14.2765026092529\\
8.3100004196167	14.4569387435913\\
8.3149995803833	14.6478576660156\\
8.31999969482422	14.8375263214111\\
8.32499980926514	15.0294189453125\\
8.32999992370605	15.2239112854004\\
8.33500003814697	15.4204063415527\\
8.34000015258789	15.6187915802002\\
8.34500026702881	15.8190116882324\\
8.35000038146973	16.0212593078613\\
8.35499954223633	16.2231464385986\\
8.35999965667725	16.4287719726563\\
8.36499977111816	16.6377563476563\\
8.36999988555908	16.845609664917\\
8.375	17.0537853240967\\
8.38000011444092	17.2620697021484\\
8.38500022888184	17.4695796966553\\
8.39000034332275	17.6767959594727\\
8.39500045776367	17.8814105987549\\
8.39999961853027	18.0889320373535\\
8.40499973297119	18.29368019104\\
8.40999984741211	18.4958190917969\\
8.41499996185303	18.6955795288086\\
8.42000007629395	18.8922500610352\\
8.42500019073486	19.0865840911865\\
8.43000030517578	19.2772941589355\\
8.4350004196167	19.46457862854\\
8.4399995803833	19.648717880249\\
8.44499969482422	19.8260726928711\\
8.44999980926514	19.9963054656982\\
8.45499992370605	20.15797996521\\
8.46000003814697	20.3077068328857\\
8.46500015258789	20.4460296630859\\
8.47000026702881	20.5673866271973\\
8.47500038146973	20.6734352111816\\
8.47999954223633	20.7645645141602\\
8.48499965667725	20.8419723510742\\
8.48999977111816	20.9017391204834\\
8.49499988555908	20.9486808776855\\
8.5	20.9841270446777\\
8.50500011444092	21.0064506530762\\
8.51000022888184	21.0276527404785\\
8.51500034332275	21.0489139556885\\
8.52000045776367	21.0696887969971\\
8.52499961853027	21.1081275939941\\
8.52999973297119	21.1508464813232\\
8.53499984741211	21.213191986084\\
8.53999996185303	21.2975425720215\\
8.54500007629395	21.4079418182373\\
8.55000019073486	21.5493240356445\\
8.55500030517578	21.7236995697021\\
8.5600004196167	21.9422245025635\\
8.5649995803833	22.2005500793457\\
8.56999969482422	22.5001888275146\\
8.57499980926514	22.8418674468994\\
8.57999992370605	23.2110404968262\\
8.58500003814697	23.6048374176025\\
8.59000015258789	24.0127563476563\\
8.59500026702881	24.4011211395264\\
8.60000038146973	24.7556381225586\\
8.60499954223633	25.0741577148438\\
8.60999965667725	25.3496150970459\\
8.61499977111816	25.5844097137451\\
8.61999988555908	25.784740447998\\
8.625	25.9513149261475\\
8.63000011444092	26.0921993255615\\
8.63500022888184	26.2081546783447\\
8.64000034332275	26.3077335357666\\
8.64500045776367	26.3832836151123\\
8.64999961853027	26.4569835662842\\
8.65499973297119	26.475736618042\\
8.65999984741211	26.5512714385986\\
8.66499996185303	26.5429878234863\\
8.67000007629395	26.5112037658691\\
8.67500019073486	26.4273071289063\\
8.68000030517578	26.3154678344727\\
8.6850004196167	26.1558876037598\\
8.6899995803833	25.9398365020752\\
8.69499969482422	25.6768035888672\\
8.69999980926514	25.3802146911621\\
8.70499992370605	25.0514659881592\\
8.71000003814697	24.6884727478027\\
8.71500015258789	24.2940979003906\\
8.72000026702881	23.9036884307861\\
8.72500038146973	23.5070610046387\\
8.72999954223633	23.1088638305664\\
8.73499965667725	22.7203140258789\\
8.73999977111816	22.3470420837402\\
8.74499988555908	21.9821624755859\\
8.75	21.619873046875\\
8.75500011444092	21.2319717407227\\
8.76000022888184	20.8638076782227\\
8.76500034332275	20.477970123291\\
8.77000045776367	20.0874633789063\\
8.77499961853027	19.6873779296875\\
8.77999973297119	19.2850532531738\\
8.78499984741211	18.876672744751\\
8.78999996185303	18.4689407348633\\
8.79500007629395	18.0684833526611\\
8.80000019073486	17.6811065673828\\
8.80500030517578	17.311975479126\\
8.8100004196167	16.9662265777588\\
8.8149995803833	16.6482543945313\\
8.81999969482422	16.357551574707\\
8.82499980926514	16.0952491760254\\
8.82999992370605	15.8602981567383\\
8.83500003814697	15.6470699310303\\
8.84000015258789	15.4516439437866\\
8.84500026702881	15.2721614837646\\
8.85000038146973	15.1011638641357\\
8.85499954223633	14.9416408538818\\
8.85999965667725	14.7905473709106\\
8.86499977111816	14.6487636566162\\
8.86999988555908	14.5179347991943\\
8.875	14.4012460708618\\
8.88000011444092	14.3023519515991\\
8.88500022888184	14.2242937088013\\
8.89000034332275	14.1697301864624\\
8.89500045776367	14.1379995346069\\
8.89999961853027	14.1301164627075\\
8.90499973297119	14.1492862701416\\
8.90999984741211	14.1862344741821\\
8.91499996185303	14.2293949127197\\
8.92000007629395	14.2866888046265\\
8.92500019073486	14.3592185974121\\
8.93000030517578	14.4438171386719\\
8.9350004196167	14.5319156646729\\
8.9399995803833	14.6081285476685\\
8.94499969482422	14.6909761428833\\
8.94999980926514	14.7824964523315\\
8.95499992370605	14.8806629180908\\
8.96000003814697	14.9870281219482\\
8.96500015258789	15.1023569107056\\
8.97000026702881	15.2290573120117\\
8.97500038146973	15.3625497817993\\
8.97999954223633	15.5010890960693\\
8.98499965667725	15.6565608978271\\
8.98999977111816	15.8104677200317\\
8.99499988555908	15.9557571411133\\
9	16.1118946075439\\
9.00500011444092	16.2673072814941\\
9.01000022888184	16.4195308685303\\
9.01500034332275	16.5676803588867\\
9.02000045776367	16.7124366760254\\
9.02499961853027	16.8526000976563\\
9.02999973297119	16.9873809814453\\
9.03499984741211	17.1167793273926\\
9.03999996185303	17.2403621673584\\
9.04500007629395	17.3578395843506\\
9.05000019073486	17.4689178466797\\
9.05500030517578	17.5731601715088\\
9.0600004196167	17.6698913574219\\
9.0649995803833	17.7587738037109\\
9.06999969482422	17.8391990661621\\
9.07499980926514	17.9107208251953\\
9.07999992370605	17.9729957580566\\
9.08500003814697	18.0256767272949\\
9.09000015258789	18.0677852630615\\
9.09500026702881	18.0971393585205\\
9.10000038146973	18.1132373809814\\
9.10499954223633	18.122486114502\\
9.10999965667725	18.122350692749\\
9.11499977111816	18.1098327636719\\
9.11999988555908	18.0841293334961\\
9.125	18.046558380127\\
9.13000011444092	17.9980201721191\\
9.13500022888184	17.9398460388184\\
9.14000034332275	17.8773727416992\\
9.14500045776367	17.8091659545898\\
9.14999961853027	17.7383899688721\\
9.15499973297119	17.6651859283447\\
9.15999984741211	17.5897216796875\\
9.16499996185303	17.5115528106689\\
9.17000007629395	17.429759979248\\
9.17500019073486	17.3439865112305\\
9.18000030517578	17.2536144256592\\
9.1850004196167	17.1588649749756\\
9.1899995803833	17.0595855712891\\
9.19499969482422	16.9555721282959\\
9.19999980926514	16.8478698730469\\
9.20499992370605	16.7376728057861\\
9.21000003814697	16.6268844604492\\
9.21500015258789	16.5169258117676\\
9.22000026702881	16.4074325561523\\
9.22500038146973	16.299840927124\\
9.22999954223633	16.1939830780029\\
9.23499965667725	16.0902366638184\\
9.23999977111816	15.9903049468994\\
9.24499988555908	15.8977603912354\\
9.25	15.8158416748047\\
9.25500011444092	15.7482299804688\\
9.26000022888184	15.7016725540161\\
9.26500034332275	15.6861581802368\\
9.27000045776367	15.7297506332397\\
9.27499961853027	15.8228349685669\\
9.27999973297119	15.9644546508789\\
9.28499984741211	16.1555919647217\\
9.28999996185303	16.3957214355469\\
9.29500007629395	16.6829566955566\\
9.30000019073486	17.0133991241455\\
9.30500030517578	17.3910808563232\\
9.3100004196167	17.8016586303711\\
9.3149995803833	18.2438945770264\\
9.31999969482422	18.7085266113281\\
9.32499980926514	19.1849327087402\\
9.32999992370605	19.6618270874023\\
9.33500003814697	20.1196842193604\\
9.34000015258789	20.5358085632324\\
9.34500026702881	20.8998203277588\\
9.35000038146973	21.1692447662354\\
9.35499954223633	21.316801071167\\
9.35999965667725	21.3283061981201\\
9.36499977111816	21.17409324646\\
9.36999988555908	20.848072052002\\
9.375	20.3720512390137\\
9.38000011444092	19.7665996551514\\
9.38500022888184	19.0663356781006\\
9.39000034332275	18.3285007476807\\
9.39500045776367	17.5785274505615\\
9.39999961853027	16.8921871185303\\
9.40499973297119	16.2970638275146\\
9.40999984741211	15.8094577789307\\
9.41499996185303	15.4216346740723\\
9.42000007629395	15.0850400924683\\
9.42500019073486	14.7216634750366\\
9.43000030517578	14.213173866272\\
9.4350004196167	13.4193477630615\\
9.4399995803833	12.1659183502197\\
9.44499969482422	10.2939510345459\\
9.44999980926514	7.64134311676025\\
9.45499992370605	4.11422157287598\\
9.46000003814697	-0.256407499313354\\
9.46500015258789	-5.35131502151489\\
9.47000026702881	-10.7984838485718\\
9.47500038146973	-16.1522998809814\\
9.47999954223633	-20.8689098358154\\
9.48499965667725	-24.2417488098145\\
9.48999977111816	-25.3938426971436\\
9.49499988555908	-25.4022064208984\\
9.5	-23.9062728881836\\
9.50500011444092	-21.2283706665039\\
9.51000022888184	-17.7851600646973\\
9.51500034332275	-13.9692106246948\\
9.52000045776367	-10.1773271560669\\
9.52499961853027	-6.66401338577271\\
9.52999973297119	-3.8194317817688\\
9.53499984741211	-1.66134345531464\\
9.53999996185303	-0.329259157180786\\
9.54500007629395	0.142379194498062\\
9.55000019073486	-0.18819072842598\\
9.55500030517578	-0.141730114817619\\
9.5600004196167	-0.130014464259148\\
9.5649995803833	-0.118022225797176\\
9.56999969482422	-0.103807978332043\\
9.57499980926514	-0.0884184539318085\\
9.57999992370605	-0.0780320018529892\\
9.58500003814697	-0.0628504902124405\\
9.59000015258789	-0.049817219376564\\
9.59500026702881	-0.0393693223595619\\
9.60000038146973	-0.0305179599672556\\
9.60499954223633	-0.0217159762978554\\
9.60999965667725	-0.0130962645635009\\
9.61499977111816	-0.00628915801644325\\
9.61999988555908	-2.84320503851632e-05\\
9.625	0.00505545129999518\\
9.63000011444092	0.0103868125006557\\
9.63500022888184	0.0142344459891319\\
9.64000034332275	0.0180455334484577\\
9.64500045776367	0.022253729403019\\
9.64999961853027	0.0251490511000156\\
9.65499973297119	0.0272142831236124\\
9.65999984741211	0.0299303233623505\\
9.66499996185303	0.0321050845086575\\
9.67000007629395	0.0338842868804932\\
9.67500019073486	0.0355582199990749\\
9.68000030517578	0.0370388366281986\\
9.6850004196167	0.0383692719042301\\
9.6899995803833	0.039514996111393\\
9.69499969482422	0.0404523946344852\\
9.69999980926514	0.0413074009120464\\
9.70499992370605	0.0421638339757919\\
9.71000003814697	0.0428915247321129\\
9.71500015258789	0.0433532372117043\\
9.72000026702881	0.0437819510698318\\
9.72500038146973	0.0443270839750767\\
9.72999954223633	0.0448545329272747\\
9.73499965667725	0.0453643687069416\\
9.73999977111816	0.0458163730800152\\
9.74499988555908	0.0461426563560963\\
9.75	0.0464636087417603\\
9.75500011444092	0.0466701574623585\\
9.76000022888184	0.046716932207346\\
9.76500034332275	0.0467079766094685\\
9.77000045776367	0.0467913970351219\\
9.77499961853027	0.0469533316791058\\
9.77999973297119	0.0472526662051678\\
9.78499984741211	0.0473966896533966\\
9.78999996185303	0.0474885292351246\\
9.79500007629395	0.0474901609122753\\
9.80000019073486	0.0474724359810352\\
9.80500030517578	0.0475273542106152\\
9.8100004196167	0.047629788517952\\
9.8149995803833	0.0478044264018536\\
9.81999969482422	0.0478933714330196\\
9.82499980926514	0.0479419305920601\\
9.82999992370605	0.047920823097229\\
9.83500003814697	0.0478789173066616\\
9.84000015258789	0.0478754825890064\\
9.84500026702881	0.0478914231061935\\
9.85000038146973	0.0479435659945011\\
9.85499954223633	0.0479752533137798\\
9.85999965667725	0.0480192638933659\\
9.86499977111816	0.0480787791311741\\
9.86999988555908	0.048112515360117\\
9.875	0.0480882786214352\\
9.88000011444092	0.0480148643255234\\
9.88500022888184	0.0478964932262897\\
9.89000034332275	0.0479343980550766\\
9.89500045776367	0.0480645820498466\\
9.89999961853027	0.0483139045536518\\
9.90499973297119	0.0484773479402065\\
9.90999984741211	0.0484088249504566\\
9.91499996185303	0.0481613948941231\\
9.92000007629395	0.0477368123829365\\
9.92500019073486	0.0476744696497917\\
9.93000030517578	0.0477820187807083\\
9.9350004196167	0.048118744045496\\
9.9399995803833	0.0483781471848488\\
9.94499969482422	0.0484399646520615\\
9.94999980926514	0.0484662428498268\\
9.95499992370605	0.0484569780528545\\
9.96000003814697	0.0484121739864349\\
9.96500015258789	0.0483318231999874\\
9.97000026702881	0.0482159331440926\\
9.97500038146973	0.0481593199074268\\
9.97999954223633	0.0481561720371246\\
9.98499965667725	0.048157274723053\\
9.98999977111816	0.0481626205146313\\
9.99499988555908	0.0481722168624401\\
10	0.0481860600411892\\
};
\addlegendentry{RS}

\addplot [color=red, line width=1.0pt]
  table[row sep=crcr]{%
0.0949999988079071	0.754309117794037\\
0.100000001490116	0.712993860244751\\
0.104999996721745	0.678799331188202\\
0.109999999403954	0.647052049636841\\
0.115000002086163	0.621121883392334\\
0.119999997317791	-0.0162820685654879\\
0.125	1.98916018009186\\
0.129999995231628	2.95780324935913\\
0.135000005364418	3.79241585731506\\
0.140000000596046	4.55601024627686\\
0.144999995827675	5.3341817855835\\
0.150000005960464	6.17193412780762\\
0.155000001192093	7.11290788650513\\
0.159999996423721	8.1926155090332\\
0.165000006556511	9.4449520111084\\
0.170000001788139	10.7476472854614\\
0.174999997019768	11.9994668960571\\
0.180000007152557	13.1102266311646\\
0.185000002384186	14.0143480300903\\
0.189999997615814	14.6431837081909\\
0.194999992847443	33.0008888244629\\
0.200000002980232	42.0192947387695\\
0.204999998211861	45.6423454284668\\
0.209999993443489	46.1144752502441\\
0.215000003576279	44.9046287536621\\
0.219999998807907	44.641845703125\\
0.224999994039536	42.3709449768066\\
0.230000004172325	38.4249687194824\\
0.234999999403954	33.7601089477539\\
0.239999994635582	29.7611141204834\\
0.245000004768372	27.8473052978516\\
0.25	29.5163707733154\\
0.254999995231628	36.5731925964355\\
0.259999990463257	45.8682899475098\\
0.264999985694885	55.425910949707\\
0.270000010728836	63.9887504577637\\
0.275000005960464	70.7495574951172\\
0.280000001192093	75.3462219238281\\
0.284999996423721	77.9392547607422\\
0.28999999165535	78.8572616577148\\
0.294999986886978	79.2627563476563\\
0.300000011920929	79.2281265258789\\
0.305000007152557	78.4565582275391\\
0.310000002384186	77.6270523071289\\
0.314999997615814	77.3489151000977\\
0.319999992847443	78.1184005737305\\
0.324999988079071	80.216667175293\\
0.330000013113022	83.7255401611328\\
0.33500000834465	88.4328842163086\\
0.340000003576279	94.2262802124023\\
0.344999998807907	99.7101364135742\\
0.349999994039536	104.393379211426\\
0.354999989271164	107.975570678711\\
0.360000014305115	110.368721008301\\
0.365000009536743	111.840957641602\\
0.370000004768372	112.734001159668\\
0.375	113.077033996582\\
0.379999995231628	113.209442138672\\
0.384999990463257	113.437507629395\\
0.389999985694885	114.008041381836\\
0.395000010728836	115.082977294922\\
0.400000005960464	116.752716064453\\
0.405000001192093	118.837890625\\
0.409999996423721	121.340255737305\\
0.41499999165535	123.972778320313\\
0.419999986886978	126.524856567383\\
0.425000011920929	128.861022949219\\
0.430000007152557	130.515548706055\\
0.435000002384186	131.469879150391\\
0.439999997615814	131.881484985352\\
0.444999992847443	131.651657104492\\
0.449999988079071	130.937927246094\\
0.455000013113022	129.947906494141\\
0.46000000834465	128.921188354492\\
0.465000003576279	128.093154907227\\
0.469999998807907	127.652557373047\\
0.474999994039536	127.696235656738\\
0.479999989271164	128.214157104492\\
0.485000014305115	129.115310668945\\
0.490000009536743	130.247161865234\\
0.495000004768372	131.495971679688\\
0.5	132.568969726563\\
0.504999995231628	133.34619140625\\
0.509999990463257	133.823516845703\\
0.514999985694885	134.041870117188\\
0.519999980926514	134.076705932617\\
0.524999976158142	134.051452636719\\
0.529999971389771	134.114822387695\\
0.535000026226044	134.176147460938\\
0.540000021457672	134.332412719727\\
0.545000016689301	134.669845581055\\
0.550000011920929	135.22802734375\\
0.555000007152557	136.097717285156\\
0.560000002384186	137.184448242188\\
0.564999997615814	138.438674926758\\
0.569999992847443	139.813507080078\\
0.574999988079071	141.267715454102\\
0.579999983310699	142.75944519043\\
0.584999978542328	144.297332763672\\
0.589999973773956	145.888565063477\\
0.595000028610229	147.563034057617\\
0.600000023841858	149.350051879883\\
0.605000019073486	151.270858764648\\
0.610000014305115	153.346115112305\\
0.615000009536743	155.570175170898\\
0.620000004768372	157.950637817383\\
0.625	160.449096679688\\
0.629999995231628	163.073944091797\\
0.634999990463257	165.786422729492\\
0.639999985694885	168.573577880859\\
0.644999980926514	171.422653198242\\
0.649999976158142	174.335342407227\\
0.654999971389771	177.296859741211\\
0.660000026226044	180.328460693359\\
0.665000021457672	183.424697875977\\
0.670000016689301	186.628845214844\\
0.675000011920929	189.949661254883\\
0.680000007152557	193.428695678711\\
0.685000002384186	197.08088684082\\
0.689999997615814	200.917984008789\\
0.694999992847443	204.934417724609\\
0.699999988079071	209.111862182617\\
0.704999983310699	213.415954589844\\
0.709999978542328	217.80924987793\\
0.714999973773956	222.229034423828\\
0.720000028610229	226.648498535156\\
0.725000023841858	231.024124145508\\
0.730000019073486	235.332824707031\\
0.735000014305115	239.564956665039\\
0.740000009536743	243.750518798828\\
0.745000004768372	247.899490356445\\
0.75	252.067428588867\\
0.754999995231628	256.298736572266\\
0.759999990463257	260.633331298828\\
0.764999985694885	265.11083984375\\
0.769999980926514	269.74658203125\\
0.774999976158142	274.528076171875\\
0.779999971389771	279.40625\\
0.785000026226044	284.30224609375\\
0.790000021457672	289.172760009766\\
0.795000016689301	293.944396972656\\
0.800000011920929	298.570495605469\\
0.805000007152557	302.990692138672\\
0.810000002384186	307.264770507813\\
0.814999997615814	311.353729248047\\
0.819999992847443	315.3818359375\\
0.824999988079071	319.351928710938\\
0.829999983310699	323.354675292969\\
0.834999978542328	327.447601318359\\
0.839999973773956	331.644195556641\\
0.845000028610229	335.935577392578\\
0.850000023841858	340.271911621094\\
0.855000019073486	344.601928710938\\
0.860000014305115	348.833190917969\\
0.865000009536743	352.908447265625\\
0.870000004768372	356.768707275391\\
0.875	360.409271240234\\
0.879999995231628	363.856689453125\\
0.884999990463257	367.164764404297\\
0.889999985694885	370.416809082031\\
0.894999980926514	373.696594238281\\
0.899999976158142	377.058380126953\\
0.904999971389771	380.538513183594\\
0.910000026226044	384.12890625\\
0.915000021457672	387.757568359375\\
0.920000016689301	391.377746582031\\
0.925000011920929	394.910034179688\\
0.930000007152557	398.275085449219\\
0.935000002384186	401.456024169922\\
0.939999997615814	404.451324462891\\
0.944999992847443	407.318817138672\\
0.949999988079071	410.144073486328\\
0.954999983310699	413.016540527344\\
0.959999978542328	416.001220703125\\
0.964999973773956	419.169982910156\\
0.970000028610229	422.4921875\\
0.975000023841858	425.928375244141\\
0.980000019073486	429.420043945313\\
0.985000014305115	432.862335205078\\
0.990000009536743	436.186096191406\\
0.995000004768372	439.365814208984\\
1	442.414428710938\\
1.00499999523163	445.393280029297\\
1.00999999046326	448.393127441406\\
1.01499998569489	451.510437011719\\
1.01999998092651	454.831451416016\\
1.02499997615814	458.393280029297\\
1.02999997138977	462.161590576172\\
1.0349999666214	466.077392578125\\
1.03999996185303	470.05419921875\\
1.04499995708466	473.962677001953\\
1.04999995231628	477.741729736328\\
1.05499994754791	481.408294677734\\
1.05999994277954	484.982635498047\\
1.06500005722046	488.574523925781\\
1.07000005245209	492.297088623047\\
1.07500004768372	496.251770019531\\
1.08000004291534	500.485778808594\\
1.08500003814697	504.987976074219\\
1.0900000333786	509.680450439453\\
1.09500002861023	514.431945800781\\
1.10000002384186	519.120910644531\\
1.10500001907349	523.667724609375\\
1.11000001430511	528.059997558594\\
1.11500000953674	532.364807128906\\
1.12000000476837	536.708801269531\\
1.125	541.220642089844\\
1.12999999523163	546.024353027344\\
1.13499999046326	551.137573242188\\
1.13999998569489	556.531555175781\\
1.14499998092651	562.04931640625\\
1.14999997615814	567.551940917969\\
1.15499997138977	572.898803710938\\
1.1599999666214	578.024841308594\\
1.16499996185303	582.983764648438\\
1.16999995708466	587.90283203125\\
1.17499995231628	592.958312988281\\
1.17999994754791	598.28369140625\\
1.18499994277954	603.934753417969\\
1.19000005722046	609.867797851563\\
1.19500005245209	615.947204589844\\
1.20000004768372	621.963256835938\\
1.20500004291534	627.774658203125\\
1.21000003814697	633.313354492188\\
1.2150000333786	638.639038085938\\
1.22000002861023	643.910705566406\\
1.22500002384186	649.346862792969\\
1.23000001907349	655.055725097656\\
1.23500001430511	661.121215820313\\
1.24000000953674	667.391906738281\\
1.24500000476837	673.723815917969\\
1.25	679.882934570313\\
1.25499999523163	685.726135253906\\
1.25999999046326	691.280334472656\\
1.26499998569489	696.648132324219\\
1.26999998092651	702.061157226563\\
1.27499997615814	707.714416503906\\
1.27999997138977	713.724853515625\\
1.2849999666214	719.987365722656\\
1.28999996185303	726.330932617188\\
1.29499995708466	732.504577636719\\
1.29999995231628	738.352172851563\\
1.30499994754791	743.8544921875\\
1.30999994277954	749.158508300781\\
1.31500005722046	754.478881835938\\
1.32000005245209	760.042236328125\\
1.32500004768372	765.947082519531\\
1.33000004291534	772.097229003906\\
1.33500003814697	778.309875488281\\
1.3400000333786	784.320373535156\\
1.34500002861023	789.971069335938\\
1.35000002384186	795.274597167969\\
1.35500001907349	800.412963867188\\
1.36000001430511	805.629272460938\\
1.36500000953674	811.134338378906\\
1.37000000476837	816.979614257813\\
1.375	823.049194335938\\
1.37999999523163	829.06591796875\\
1.38499999046326	834.792541503906\\
1.38999998569489	840.157287597656\\
1.39499998092651	845.229553222656\\
1.39999997615814	850.283874511719\\
1.40499997138977	855.548706054688\\
1.4099999666214	861.190246582031\\
1.41499996185303	867.15283203125\\
1.41999995708466	873.199340820313\\
1.42499995231628	879.02392578125\\
1.42999994754791	884.46533203125\\
1.43499994277954	889.567138671875\\
1.44000005722046	894.606079101563\\
1.44500005245209	899.798400878906\\
1.45000004768372	905.381530761719\\
1.45500004291534	911.334533691406\\
1.46000003814697	917.432739257813\\
1.4650000333786	923.359130859375\\
1.47000002861023	928.926147460938\\
1.47500002384186	934.141662597656\\
1.48000001907349	939.243591308594\\
1.48500001430511	944.550720214844\\
1.49000000953674	950.280944824219\\
1.49500000476837	956.407409667969\\
1.5	962.70361328125\\
1.50499999523163	968.804992675781\\
1.50999999046326	974.54541015625\\
1.51499998569489	979.922729492188\\
1.51999998092651	985.233764648438\\
1.52499997615814	990.812255859375\\
1.52999997138977	996.823852539063\\
1.5349999666214	1003.24688720703\\
1.53999996185303	1009.74841308594\\
1.54499995708466	1015.98858642578\\
1.54999995231628	1021.80865478516\\
1.55499994754791	1027.31225585938\\
1.55999994277954	1032.8486328125\\
1.56500005722046	1038.72546386719\\
1.57000005245209	1044.89233398438\\
1.57500004768372	1051.31433105469\\
1.58000004291534	1057.77856445313\\
1.58500003814697	1063.92932128906\\
1.5900000333786	1069.68518066406\\
1.59500002861023	1075.31811523438\\
1.60000002384186	1081.22180175781\\
1.60500001907349	1087.63500976563\\
1.61000001430511	1094.48303222656\\
1.61500000953674	1101.44274902344\\
1.62000000476837	1108.09423828125\\
1.625	1114.2109375\\
1.62999999523163	1119.93884277344\\
1.63499999046326	1125.69409179688\\
1.63999998569489	1131.84191894531\\
1.64499998092651	1138.49072265625\\
1.64999997615814	1145.37646484375\\
1.65499997138977	1152.09069824219\\
1.6599999666214	1158.30200195313\\
1.66499996185303	1164.03723144531\\
1.66999995708466	1169.71606445313\\
1.67499995231628	1175.7822265625\\
1.67999994754791	1182.38061523438\\
1.68499994277954	1189.27319335938\\
1.69000005722046	1196.01794433594\\
1.69500005245209	1202.24938964844\\
1.70000004768372	1207.96240234375\\
1.70500004291534	1213.51293945313\\
1.71000003814697	1219.36657714844\\
1.7150000333786	1225.72802734375\\
1.72000002861023	1232.39208984375\\
1.72500002384186	1238.9267578125\\
1.73000001907349	1244.8671875\\
1.73500001430511	1250.26672363281\\
1.74000000953674	1255.54150390625\\
1.74500000476837	1261.07922363281\\
1.75	1267.15710449219\\
1.75499999523163	1273.50268554688\\
1.75999999046326	1279.6181640625\\
1.76499998569489	1285.13171386719\\
1.76999998092651	1290.09020996094\\
1.77499997615814	1295.03210449219\\
1.77999997138977	1300.16674804688\\
1.7849999666214	1305.86547851563\\
1.78999996185303	1311.81677246094\\
1.79499995708466	1317.28234863281\\
1.79999995231628	1322.15405273438\\
1.80499994754791	1326.50219726563\\
1.80999994277954	1330.87048339844\\
1.81500005722046	1335.59362792969\\
1.82000005245209	1340.7646484375\\
1.82500004768372	1346.04016113281\\
1.83000004291534	1350.71704101563\\
1.83500003814697	1354.69250488281\\
1.8400000333786	1358.27856445313\\
1.84500002861023	1361.98352050781\\
1.85000002384186	1366.162109375\\
1.85500001907349	1370.66809082031\\
1.86000001430511	1375.03247070313\\
1.86500000953674	1378.6875\\
1.87000000476837	1381.69079589844\\
1.875	1384.4423828125\\
1.87999999523163	1387.54724121094\\
1.88499999046326	1390.99353027344\\
1.88999998569489	1394.60986328125\\
1.89499998092651	1397.84362792969\\
1.89999997615814	1400.24157714844\\
1.90499997138977	1402.13610839844\\
1.9099999666214	1404.06811523438\\
1.91499996185303	1406.32727050781\\
1.91999995708466	1408.97863769531\\
1.92499995231628	1411.6572265625\\
1.92999994754791	1413.64172363281\\
1.93499994277954	1414.66198730469\\
1.94000005722046	1415.4931640625\\
1.94500005245209	1416.525390625\\
1.95000004768372	1418.03625488281\\
1.95500004291534	1419.7509765625\\
1.96000003814697	1420.97436523438\\
1.9650000333786	1421.50866699219\\
1.97000002861023	1421.43432617188\\
1.97500002384186	1421.33666992188\\
1.98000001907349	1421.6201171875\\
1.98500001430511	1422.29333496094\\
1.99000000953674	1422.900390625\\
1.99500000476837	1422.86022949219\\
2	1422.15002441406\\
2.00500011444092	1421.05017089844\\
2.00999999046326	1420.23718261719\\
2.01500010490417	1419.95764160156\\
2.01999998092651	1419.75866699219\\
2.02500009536743	1419.26452636719\\
2.02999997138977	1418.06750488281\\
2.03500008583069	1416.3525390625\\
2.03999996185303	1414.65307617188\\
2.04500007629395	1413.48913574219\\
2.04999995231628	1412.69592285156\\
2.0550000667572	1411.97631835938\\
2.05999994277954	1410.77319335938\\
2.06500005722046	1409.0302734375\\
2.0699999332428	1407.1435546875\\
2.07500004768372	1405.54284667969\\
2.07999992370605	1404.64038085938\\
2.08500003814697	1404.09509277344\\
2.08999991416931	1403.42431640625\\
2.09500002861023	1402.23742675781\\
2.09999990463257	1400.75805664063\\
2.10500001907349	1399.39111328125\\
2.10999989509583	1398.58020019531\\
2.11500000953674	1398.23156738281\\
2.11999988555908	1398.41003417969\\
2.125	1397.41333007813\\
2.13000011444092	1396.3037109375\\
2.13499999046326	1394.97265625\\
2.14000010490417	1393.9521484375\\
2.14499998092651	1393.18310546875\\
2.15000009536743	1392.72009277344\\
2.15499997138977	1392.12731933594\\
2.16000008583069	1391.01245117188\\
2.16499996185303	1389.49426269531\\
2.17000007629395	1388.05847167969\\
2.17499995231628	1386.80090332031\\
2.1800000667572	1386.00573730469\\
2.18499994277954	1385.1943359375\\
2.19000005722046	1384.15234375\\
2.1949999332428	1382.58203125\\
2.20000004768372	1380.75964355469\\
2.20499992370605	1379.01818847656\\
2.21000003814697	1377.57739257813\\
2.21499991416931	1376.373046875\\
2.22000002861023	1374.92456054688\\
2.22499990463257	1373.11303710938\\
2.23000001907349	1370.8935546875\\
2.23499989509583	1368.66687011719\\
2.24000000953674	1366.55505371094\\
2.24499988555908	1364.96813964844\\
2.25	1363.59619140625\\
2.25500011444092	1362.03723144531\\
2.25999999046326	1360.13757324219\\
2.26500010490417	1358.12243652344\\
2.26999998092651	1356.10961914063\\
2.27500009536743	1354.63879394531\\
2.27999997138977	1353.31298828125\\
2.28500008583069	1352.189453125\\
2.28999996185303	1350.86059570313\\
2.29500007629395	1349.18237304688\\
2.29999995231628	1347.40307617188\\
2.3050000667572	1345.87512207031\\
2.30999994277954	1344.90661621094\\
2.31500005722046	1344.13415527344\\
2.3199999332428	1343.46911621094\\
2.32500004768372	1342.34155273438\\
2.32999992370605	1340.95922851563\\
2.33500003814697	1339.97912597656\\
2.33999991416931	1338.70654296875\\
2.34500002861023	1338.33508300781\\
2.34999990463257	1338.28515625\\
2.35500001907349	1337.82275390625\\
2.35999989509583	1337.0830078125\\
2.36500000953674	1335.8779296875\\
2.36999988555908	1334.9873046875\\
2.375	1334.38366699219\\
2.38000011444092	1334.13671875\\
2.38499999046326	1333.95422363281\\
2.39000010490417	1333.35571289063\\
2.39499998092651	1332.17639160156\\
2.40000009536743	1330.80639648438\\
2.40499997138977	1329.73547363281\\
2.41000008583069	1329.32470703125\\
2.41499996185303	1329.40979003906\\
2.42000007629395	1329.4541015625\\
2.42499995231628	1329.08483886719\\
2.4300000667572	1328.32690429688\\
2.43499994277954	1327.75231933594\\
2.44000005722046	1327.75598144531\\
2.4449999332428	1328.52709960938\\
2.45000004768372	1329.59362792969\\
2.45499992370605	1330.45715332031\\
2.46000003814697	1330.90612792969\\
2.46499991416931	1330.79016113281\\
2.47000002861023	1331.14221191406\\
2.47499990463257	1331.89013671875\\
2.48000001907349	1333.13024902344\\
2.48499989509583	1334.27294921875\\
2.49000000953674	1335.00671386719\\
2.49499988555908	1334.97705078125\\
2.5	1334.82360839844\\
2.50500011444092	1334.88146972656\\
2.50999999046326	1335.54541015625\\
2.51500010490417	1336.46569824219\\
2.51999998092651	1337.115234375\\
2.52500009536743	1337.22644042969\\
2.52999997138977	1336.83911132813\\
2.53500008583069	1336.685546875\\
2.53999996185303	1336.43420410156\\
2.54500007629395	1336.69787597656\\
2.54999995231628	1337.02319335938\\
2.5550000667572	1336.79211425781\\
2.55999994277954	1335.93420410156\\
2.56500005722046	1334.73522949219\\
2.5699999332428	1333.53723144531\\
2.57500004768372	1332.84997558594\\
2.57999992370605	1332.17883300781\\
2.58500003814697	1331.43994140625\\
2.58999991416931	1330.16613769531\\
2.59500002861023	1328.46569824219\\
2.59999990463257	1326.62170410156\\
2.60500001907349	1325.001953125\\
2.60999989509583	1323.74572753906\\
2.61500000953674	1322.58435058594\\
2.61999988555908	1321.142578125\\
2.625	1319.23254394531\\
2.63000011444092	1317.00207519531\\
2.63499999046326	1314.73132324219\\
2.64000010490417	1312.78479003906\\
2.64499998092651	1311.10534667969\\
2.65000009536743	1309.50634765625\\
2.65499997138977	1307.64904785156\\
2.66000008583069	1305.56237792969\\
2.66499996185303	1303.2705078125\\
2.67000007629395	1301.32763671875\\
2.67499995231628	1299.72253417969\\
2.6800000667572	1298.35168457031\\
2.68499994277954	1297.1689453125\\
2.69000005722046	1295.78759765625\\
2.6949999332428	1294.22241210938\\
2.70000004768372	1292.65454101563\\
2.70499992370605	1291.21862792969\\
2.71000003814697	1290.22155761719\\
2.71499991416931	1289.36462402344\\
2.72000002861023	1288.50549316406\\
2.72499990463257	1287.41772460938\\
2.73000001907349	1286.11572265625\\
2.73499989509583	1284.78979492188\\
2.74000000953674	1283.748046875\\
2.74499988555908	1283.09240722656\\
2.75	1282.79565429688\\
2.75500011444092	1282.25817871094\\
2.75999999046326	1281.6484375\\
2.76500010490417	1280.77026367188\\
2.76999998092651	1280.02294921875\\
2.77500009536743	1279.6904296875\\
2.77999997138977	1279.5625\\
2.78500008583069	1279.62475585938\\
2.78999996185303	1279.55883789063\\
2.79500007629395	1279.28112792969\\
2.79999995231628	1278.63049316406\\
2.8050000667572	1278.2978515625\\
2.80999994277954	1278.40490722656\\
2.81500005722046	1279.08728027344\\
2.8199999332428	1280.09143066406\\
2.82500004768372	1281.02600097656\\
2.82999992370605	1281.849609375\\
2.83500003814697	1282.84814453125\\
2.83999991416931	1284.52551269531\\
2.84500002861023	1286.86865234375\\
2.84999990463257	1290.0234375\\
2.85500001907349	1293.05017089844\\
2.85999989509583	1296.0087890625\\
2.86500000953674	1298.71276855469\\
2.86999988555908	1301.68310546875\\
2.875	1305.23327636719\\
2.88000011444092	1309.42358398438\\
2.88499999046326	1313.94299316406\\
2.89000010490417	1318.10986328125\\
2.89499998092651	1321.69702148438\\
2.90000009536743	1325.02551269531\\
2.90499997138977	1328.33740234375\\
2.91000008583069	1331.88635253906\\
2.91499996185303	1335.06701660156\\
2.92000007629395	1337.69677734375\\
2.92499995231628	1339.14074707031\\
2.9300000667572	1339.51635742188\\
2.93499994277954	1339.29907226563\\
2.94000005722046	1338.91577148438\\
2.9449999332428	1338.76159667969\\
2.95000004768372	1338.75402832031\\
2.95499992370605	1338.15380859375\\
2.96000003814697	1337.11560058594\\
2.96499991416931	1335.55090332031\\
2.97000002861023	1334.06811523438\\
2.97499990463257	1333.09411621094\\
2.98000001907349	1332.63647460938\\
2.98499989509583	1332.38562011719\\
2.99000000953674	1331.84948730469\\
2.99499988555908	1330.72937011719\\
3	1329.40112304688\\
3.00500011444092	1327.93212890625\\
3.00999999046326	1326.86352539063\\
3.01500010490417	1325.80004882813\\
3.01999998092651	1324.5361328125\\
3.02500009536743	1322.72827148438\\
3.02999997138977	1320.50622558594\\
3.03500008583069	1317.93725585938\\
3.03999996185303	1315.62719726563\\
3.04500007629395	1313.78161621094\\
3.04999995231628	1312.26501464844\\
3.0550000667572	1310.61242675781\\
3.05999994277954	1308.79113769531\\
3.06500005722046	1306.60510253906\\
3.0699999332428	1304.54016113281\\
3.07500004768372	1302.42712402344\\
3.07999992370605	1300.71154785156\\
3.08500003814697	1298.67626953125\\
3.08999991416931	1296.47521972656\\
3.09500002861023	1293.91455078125\\
3.09999990463257	1291.09655761719\\
3.10500001907349	1288.24890136719\\
3.10999989509583	1285.66052246094\\
3.11500000953674	1283.34606933594\\
3.11999988555908	1281.27172851563\\
3.125	1279.0048828125\\
3.13000011444092	1276.36608886719\\
3.13499999046326	1273.8095703125\\
3.14000010490417	1270.56188964844\\
3.14499998092651	1267.673828125\\
3.15000009536743	1265.12097167969\\
3.15499997138977	1262.88586425781\\
3.16000008583069	1260.58239746094\\
3.16499996185303	1258.39978027344\\
3.17000007629395	1256.81689453125\\
3.17499995231628	1256.45703125\\
3.1800000667572	1257.56884765625\\
3.18499994277954	1260.00134277344\\
3.19000005722046	1263.23205566406\\
3.1949999332428	1266.85290527344\\
3.20000004768372	1270.91027832031\\
3.20499992370605	1275.72973632813\\
3.21000003814697	1281.3720703125\\
3.21499991416931	1287.61193847656\\
3.22000002861023	1293.9853515625\\
3.22499990463257	1299.52099609375\\
3.23000001907349	1304.46923828125\\
3.23499989509583	1308.84826660156\\
3.24000000953674	1312.95068359375\\
3.24499988555908	1316.857421875\\
3.25	1320.31750488281\\
3.25500011444092	1322.91259765625\\
3.25999999046326	1324.36926269531\\
3.26500010490417	1324.83349609375\\
3.26999998092651	1324.81530761719\\
3.27500009536743	1324.82312011719\\
3.27999997138977	1325.08374023438\\
3.28500008583069	1325.43542480469\\
3.28999996185303	1325.32861328125\\
3.29500007629395	1324.79089355469\\
3.29999995231628	1323.625\\
3.3050000667572	1322.88330078125\\
3.30999994277954	1322.10437011719\\
3.31500005722046	1321.81884765625\\
3.3199999332428	1321.49084472656\\
3.32500004768372	1320.85314941406\\
3.32999992370605	1319.58447265625\\
3.33500003814697	1318.00024414063\\
3.33999991416931	1316.33862304688\\
3.34500002861023	1315.02319335938\\
3.34999990463257	1314.02258300781\\
3.35500001907349	1312.93664550781\\
3.35999989509583	1311.50012207031\\
3.36500000953674	1309.78198242188\\
3.36999988555908	1307.96325683594\\
3.375	1306.48510742188\\
3.38000011444092	1305.59057617188\\
3.38499999046326	1305.0234375\\
3.39000010490417	1304.22985839844\\
3.39499998092651	1303.07739257813\\
3.40000009536743	1301.50549316406\\
3.40499997138977	1299.81469726563\\
3.41000008583069	1298.21789550781\\
3.41499996185303	1296.67761230469\\
3.42000007629395	1295.15283203125\\
3.42499995231628	1293.38549804688\\
3.4300000667572	1291.15246582031\\
3.43499994277954	1288.646484375\\
3.44000005722046	1286.06469726563\\
3.4449999332428	1283.50659179688\\
3.45000004768372	1281.08337402344\\
3.45499992370605	1278.76733398438\\
3.46000003814697	1275.77941894531\\
3.46499991416931	1272.21264648438\\
3.47000002861023	1267.98962402344\\
3.47499990463257	1263.70495605469\\
3.48000001907349	1259.82055664063\\
3.48499989509583	1256.85815429688\\
3.49000000953674	1254.72375488281\\
3.49499988555908	1253.18005371094\\
3.5	1252.11206054688\\
3.50500011444092	1252.05737304688\\
3.50999999046326	1253.12780761719\\
3.51500010490417	1256.18322753906\\
3.51999998092651	1260.90222167969\\
3.52500009536743	1266.83203125\\
3.52999997138977	1273.587890625\\
3.53500008583069	1281.23791503906\\
3.53999996185303	1290.22448730469\\
3.54500007629395	1300.55163574219\\
3.54999995231628	1311.53552246094\\
3.5550000667572	1321.37377929688\\
3.55999994277954	1328.75854492188\\
3.56500005722046	1333.462890625\\
3.5699999332428	1335.71533203125\\
3.57500004768372	1335.91259765625\\
3.57999992370605	1334.34057617188\\
3.58500003814697	1331.24206542969\\
3.58999991416931	1326.78466796875\\
3.59500002861023	1321.68383789063\\
3.59999990463257	1316.71154785156\\
3.60500001907349	1312.5078125\\
3.60999989509583	1309.75561523438\\
3.61500000953674	1308.783203125\\
3.61999988555908	1309.39758300781\\
3.625	1310.22131347656\\
3.63000011444092	1311.32775878906\\
3.63499999046326	1312.49011230469\\
3.64000010490417	1314.02233886719\\
3.64499998092651	1314.91784667969\\
3.65000009536743	1315.48425292969\\
3.65499997138977	1315.13110351563\\
3.66000008583069	1313.61120605469\\
3.66499996185303	1311.14575195313\\
3.67000007629395	1308.14514160156\\
3.67499995231628	1305.12622070313\\
3.6800000667572	1302.76928710938\\
3.68499994277954	1301.27514648438\\
3.69000005722046	1300.62048339844\\
3.6949999332428	1299.96350097656\\
3.70000004768372	1299.435546875\\
3.70499992370605	1298.8642578125\\
3.71000003814697	1298.36010742188\\
3.71499991416931	1297.49768066406\\
3.72000002861023	1296.38952636719\\
3.72499990463257	1293.86804199219\\
3.73000001907349	1290.73376464844\\
3.73499989509583	1286.81359863281\\
3.74000000953674	1282.33386230469\\
3.74499988555908	1277.71618652344\\
3.75	1273.24279785156\\
3.75500011444092	1268.79846191406\\
3.75999999046326	1263.76306152344\\
3.76500010490417	1257.79736328125\\
3.76999998092651	1251.07287597656\\
3.77500009536743	1244.72692871094\\
3.77999997138977	1239.84619140625\\
3.78500008583069	1237.21459960938\\
3.78999996185303	1236.61340332031\\
3.79500007629395	1237.48327636719\\
3.79999995231628	1239.86083984375\\
3.8050000667572	1243.64270019531\\
3.80999994277954	1250.115234375\\
3.81500005722046	1259.333984375\\
3.8199999332428	1270.83532714844\\
3.82500004768372	1283.98913574219\\
3.82999992370605	1297.52893066406\\
3.83500003814697	1311.2919921875\\
3.83999991416931	1324.59643554688\\
3.84500002861023	1336.19665527344\\
3.84999990463257	1345.03674316406\\
3.85500001907349	1349.9853515625\\
3.85999989509583	1350.42016601563\\
3.86500000953674	1347.44812011719\\
3.86999988555908	1341.70532226563\\
3.875	1334.47265625\\
3.88000011444092	1326.81237792969\\
3.88499999046326	1320.33654785156\\
3.89000010490417	1315.51062011719\\
3.89499998092651	1311.98974609375\\
3.90000009536743	1309.44812011719\\
3.90499997138977	1307.78869628906\\
3.91000008583069	1307.76330566406\\
3.91499996185303	1309.08471679688\\
3.92000007629395	1310.99572753906\\
3.92499995231628	1312.68701171875\\
3.9300000667572	1313.29406738281\\
3.93499994277954	1312.51159667969\\
3.94000005722046	1310.57556152344\\
3.9449999332428	1307.90051269531\\
3.95000004768372	1305.12768554688\\
3.95499992370605	1302.35205078125\\
3.96000003814697	1299.47961425781\\
3.96499991416931	1297.11987304688\\
3.97000002861023	1294.63842773438\\
3.97499990463257	1293.2236328125\\
3.98000001907349	1292.62463378906\\
3.98499989509583	1292.85559082031\\
3.99000000953674	1293.30383300781\\
3.99499988555908	1293.14404296875\\
4	1292.15600585938\\
4.00500011444092	1290.27233886719\\
4.01000022888184	1287.26684570313\\
4.0149998664856	1283.48010253906\\
4.01999998092651	1278.99914550781\\
4.02500009536743	1274.19982910156\\
4.03000020980835	1268.84338378906\\
4.03499984741211	1262.705078125\\
4.03999996185303	1255.46960449219\\
4.04500007629395	1247.54614257813\\
4.05000019073486	1239.76196289063\\
4.05499982833862	1232.89868164063\\
4.05999994277954	1227.16015625\\
4.06500005722046	1222.52624511719\\
4.07000017166138	1218.90380859375\\
4.07499980926514	1216.62145996094\\
4.07999992370605	1216.9365234375\\
4.08500003814697	1220.71789550781\\
4.09000015258789	1228.1875\\
4.09499979019165	1238.919921875\\
4.09999990463257	1253.26379394531\\
4.10500001907349	1271.57983398438\\
4.1100001335144	1294.27758789063\\
4.11499977111816	1319.37512207031\\
4.11999988555908	1343.45849609375\\
4.125	1361.09692382813\\
4.13000011444092	1369.9013671875\\
4.13500022888184	1371.48315429688\\
4.1399998664856	1366.10119628906\\
4.14499998092651	1354.79907226563\\
4.15000009536743	1339.43090820313\\
4.15500020980835	1323.13793945313\\
4.15999984741211	1308.91516113281\\
4.16499996185303	1298.67956542969\\
4.17000007629395	1291.74169921875\\
4.17500019073486	1288.42053222656\\
4.17999982833862	1289.64123535156\\
4.18499994277954	1294.94360351563\\
4.19000005722046	1302.02038574219\\
4.19500017166138	1309.19152832031\\
4.19999980926514	1315.658203125\\
4.20499992370605	1320.02136230469\\
4.21000003814697	1321.34851074219\\
4.21500015258789	1319.27783203125\\
4.21999979019165	1314.80395507813\\
4.22499990463257	1309.10803222656\\
4.23000001907349	1303.2001953125\\
4.2350001335144	1297.6689453125\\
4.23999977111816	1292.73742675781\\
4.24499988555908	1289.69055175781\\
4.25	1288.98095703125\\
4.25500011444092	1290.04846191406\\
4.26000022888184	1292.1162109375\\
4.2649998664856	1294.00537109375\\
4.26999998092651	1294.794921875\\
4.27500009536743	1294.13195800781\\
4.28000020980835	1291.35400390625\\
4.28499984741211	1286.20422363281\\
4.28999996185303	1279.41760253906\\
4.29500007629395	1271.27587890625\\
4.30000019073486	1261.81799316406\\
4.30499982833862	1250.13256835938\\
4.30999994277954	1236.83142089844\\
4.31500005722046	1223.60571289063\\
4.32000017166138	1212.54309082031\\
4.32499980926514	1204.59020996094\\
4.32999992370605	1199.990234375\\
4.33500003814697	1197.45104980469\\
4.34000015258789	1197.26647949219\\
4.34499979019165	1200.77770996094\\
4.34999990463257	1209.02331542969\\
4.35500001907349	1222.66943359375\\
4.3600001335144	1241.5263671875\\
4.36499977111816	1265.22863769531\\
4.36999988555908	1291.89819335938\\
4.375	1318.60595703125\\
4.38000011444092	1345.28967285156\\
4.38500022888184	1370.02709960938\\
4.3899998664856	1389.37243652344\\
4.39499998092651	1397.06579589844\\
4.40000009536743	1392.53344726563\\
4.40500020980835	1378.3896484375\\
4.40999984741211	1357.41235351563\\
4.41499996185303	1333.42004394531\\
4.42000007629395	1311.1025390625\\
4.42500019073486	1295.33093261719\\
4.42999982833862	1286.04675292969\\
4.43499994277954	1281.6337890625\\
4.44000005722046	1281.71411132813\\
4.44500017166138	1286.68176269531\\
4.44999980926514	1295.22778320313\\
4.45499992370605	1305.08068847656\\
4.46000003814697	1314.23413085938\\
4.46500015258789	1320.29309082031\\
4.46999979019165	1322.00756835938\\
4.47499990463257	1319.21960449219\\
4.48000001907349	1313.328125\\
4.4850001335144	1304.93481445313\\
4.48999977111816	1296.853515625\\
4.49499988555908	1290.15625\\
4.5	1285.07861328125\\
4.50500011444092	1281.873046875\\
4.51000022888184	1280.86755371094\\
4.5149998664856	1282.21350097656\\
4.51999998092651	1285.26831054688\\
4.52500009536743	1288.49340820313\\
4.53000020980835	1290.36108398438\\
4.53499984741211	1290.17858886719\\
4.53999996185303	1287.17712402344\\
4.54500007629395	1281.06567382813\\
4.55000019073486	1272.18713378906\\
4.55499982833862	1261.39477539063\\
4.55999994277954	1250.29565429688\\
4.56500005722046	1239.4560546875\\
4.57000017166138	1227.67590332031\\
4.57499980926514	1215.38330078125\\
4.57999992370605	1203.89135742188\\
4.58500003814697	1194.59790039063\\
4.59000015258789	1187.84362792969\\
4.59499979019165	1183.14501953125\\
4.59999990463257	1180.35510253906\\
4.60500001907349	1179.39807128906\\
4.6100001335144	1180.34790039063\\
4.61499977111816	1184.15234375\\
4.61999988555908	1201.24658203125\\
4.625	1234.69677734375\\
4.63000011444092	1273.75134277344\\
4.63500022888184	1310.79711914063\\
4.6399998664856	1343.8759765625\\
4.64499998092651	1374.18432617188\\
4.65000009536743	1401.30981445313\\
4.65500020980835	1420.8046875\\
4.65999984741211	1429.06762695313\\
4.66499996185303	1413.15612792969\\
4.67000007629395	1379.61047363281\\
4.67500019073486	1339.02941894531\\
4.67999982833862	1298.88366699219\\
4.68499994277954	1268.03283691406\\
4.69000005722046	1250.52026367188\\
4.69500017166138	1246.05505371094\\
4.69999980926514	1251.1728515625\\
4.70499992370605	1262.9931640625\\
4.71000003814697	1280.2255859375\\
4.71500015258789	1300.74816894531\\
4.71999979019165	1321.4501953125\\
4.72499990463257	1336.54541015625\\
4.73000001907349	1341.82067871094\\
4.7350001335144	1338.04211425781\\
4.73999977111816	1327.79333496094\\
4.74499988555908	1312.91552734375\\
4.75	1297.07141113281\\
4.75500011444092	1283.5859375\\
4.76000022888184	1274.10900878906\\
4.7649998664856	1269.83825683594\\
4.76999998092651	1269.73498535156\\
4.77500009536743	1273.08703613281\\
4.78000020980835	1279.08337402344\\
4.78499984741211	1286.04272460938\\
4.78999996185303	1290.84790039063\\
4.79500007629395	1291.36352539063\\
4.80000019073486	1287.36608886719\\
4.80499982833862	1276.43920898438\\
4.80999994277954	1258.29602050781\\
4.81500005722046	1237.14270019531\\
4.82000017166138	1216.853515625\\
4.82499980926514	1199.52697753906\\
4.82999992370605	1186.18298339844\\
4.83500003814697	1175.97155761719\\
4.84000015258789	1167.93481445313\\
4.84499979019165	1162.88110351563\\
4.84999990463257	1161.69641113281\\
4.85500001907349	1165.02734375\\
4.8600001335144	1173.36340332031\\
4.86499977111816	1184.23950195313\\
4.86999988555908	1210.26550292969\\
4.875	1243.677734375\\
4.88000011444092	1280.04248046875\\
4.88500022888184	1319.85083007813\\
4.8899998664856	1359.13244628906\\
4.89499998092651	1390.73620605469\\
4.90000009536743	1413.51293945313\\
4.90500020980835	1427.85705566406\\
4.90999984741211	1435.197265625\\
4.91499996185303	1435.21240234375\\
4.92000007629395	1411.48034667969\\
4.92500019073486	1364.32666015625\\
4.92999982833862	1311.48901367188\\
4.93499994277954	1266.11962890625\\
4.94000005722046	1235.23681640625\\
4.94500017166138	1215.75549316406\\
4.94999980926514	1209.42333984375\\
4.95499992370605	1217.95617675781\\
4.96000003814697	1237.79235839844\\
4.96500015258789	1262.08813476563\\
4.96999979019165	1285.31433105469\\
4.97499990463257	1305.56079101563\\
4.98000001907349	1318.97033691406\\
4.9850001335144	1322.83996582031\\
4.98999977111816	1317.38781738281\\
4.99499988555908	1305.02209472656\\
5	1289.88110351563\\
5.00500011444092	1274.97583007813\\
5.01000022888184	1261.9208984375\\
5.0149998664856	1251.24035644531\\
5.01999998092651	1245.51733398438\\
5.02500009536743	1245.44177246094\\
5.03000020980835	1250.09631347656\\
5.03499984741211	1257.41259765625\\
5.03999996185303	1266.32055664063\\
5.04500007629395	1275.32592773438\\
5.05000019073486	1283.39953613281\\
5.05499982833862	1289.42529296875\\
5.05999994277954	1292.42944335938\\
5.06500005722046	1292.19213867188\\
5.07000017166138	1288.75256347656\\
5.07499980926514	1282.63122558594\\
5.07999992370605	1274.43725585938\\
5.08500003814697	1268.21765136719\\
5.09000015258789	1263.86047363281\\
5.09499979019165	1261.583984375\\
5.09999990463257	1260.92944335938\\
5.10500001907349	1260.65747070313\\
5.1100001335144	1260.109375\\
5.11499977111816	1259.51672363281\\
5.11999988555908	1259.17236328125\\
5.125	1259.22631835938\\
5.13000011444092	1259.30700683594\\
5.13500022888184	1273.34033203125\\
5.1399998664856	1289.88940429688\\
5.14499998092651	1305.04357910156\\
5.15000009536743	1319.9892578125\\
5.15500020980835	1333.76489257813\\
5.15999984741211	1337.19775390625\\
5.16499996185303	1329.193359375\\
5.17000007629395	1313.80444335938\\
5.17500019073486	1294.68139648438\\
5.17999982833862	1275.44091796875\\
5.18499994277954	1259.82043457031\\
5.19000005722046	1249.66918945313\\
5.19500017166138	1245.92700195313\\
5.19999980926514	1248.05017089844\\
5.20499992370605	1254.05920410156\\
5.21000003814697	1262.1943359375\\
5.21500015258789	1271.45776367188\\
5.21999979019165	1281.0615234375\\
5.22499990463257	1289.55908203125\\
5.23000001907349	1296.07373046875\\
5.2350001335144	1300.10400390625\\
5.23999977111816	1301.47399902344\\
5.24499988555908	1300.51501464844\\
5.25	1298.09252929688\\
5.25500011444092	1295.24169921875\\
5.26000022888184	1292.93774414063\\
5.2649998664856	1292.14440917969\\
5.26999998092651	1293.04479980469\\
5.27500009536743	1295.30395507813\\
5.28000020980835	1298.78723144531\\
5.28499984741211	1303.64489746094\\
5.28999996185303	1309.87426757813\\
5.29500007629395	1317.37634277344\\
5.30000019073486	1325.2998046875\\
5.30499982833862	1332.31567382813\\
5.30999994277954	1337.98400878906\\
5.31500005722046	1341.86804199219\\
5.32000017166138	1343.89208984375\\
5.32499980926514	1343.85229492188\\
5.32999992370605	1341.99462890625\\
5.33500003814697	1338.70715332031\\
5.34000015258789	1334.34350585938\\
5.34499979019165	1329.52197265625\\
5.34999990463257	1324.65563964844\\
5.35500001907349	1320.6328125\\
5.3600001335144	1317.96313476563\\
5.36499977111816	1316.56237792969\\
5.36999988555908	1315.92053222656\\
5.375	1315.52062988281\\
5.38000011444092	1315.16137695313\\
5.38500022888184	1314.66186523438\\
5.3899998664856	1313.41162109375\\
5.39499998092651	1311.00451660156\\
5.40000009536743	1307.10534667969\\
5.40500020980835	1302.27014160156\\
5.40999984741211	1297.02722167969\\
5.41499996185303	1291.68701171875\\
5.42000007629395	1286.86096191406\\
5.42500019073486	1282.74841308594\\
5.42999982833862	1279.83850097656\\
5.43499994277954	1277.73901367188\\
5.44000005722046	1275.3818359375\\
5.44500017166138	1273.24572753906\\
5.44999980926514	1271.42822265625\\
5.45499992370605	1269.98803710938\\
5.46000003814697	1268.94970703125\\
5.46500015258789	1268.13244628906\\
5.46999979019165	1267.44055175781\\
5.47499990463257	1266.68566894531\\
5.48000001907349	1265.91564941406\\
5.4850001335144	1265.2431640625\\
5.48999977111816	1264.84204101563\\
5.49499988555908	1264.52807617188\\
5.5	1264.22705078125\\
5.50500011444092	1263.97192382813\\
5.51000022888184	1263.71533203125\\
5.5149998664856	1263.40161132813\\
5.51999998092651	1264.28576660156\\
5.52500009536743	1266.78283691406\\
5.53000020980835	1269.76928710938\\
5.53499984741211	1272.64001464844\\
5.53999996185303	1275.1005859375\\
5.54500007629395	1277.21826171875\\
5.55000019073486	1279.08679199219\\
5.55499982833862	1280.91247558594\\
5.55999994277954	1282.42236328125\\
5.56500005722046	1283.53161621094\\
5.57000017166138	1283.93676757813\\
5.57499980926514	1283.78894042969\\
5.57999992370605	1283.40490722656\\
5.58500003814697	1282.65295410156\\
5.59000015258789	1281.81958007813\\
5.59499979019165	1280.82580566406\\
5.59999990463257	1279.40563964844\\
5.60500001907349	1277.71508789063\\
5.6100001335144	1275.93957519531\\
5.61499977111816	1274.31127929688\\
5.61999988555908	1272.66723632813\\
5.625	1271.23132324219\\
5.63000011444092	1270.04565429688\\
5.63500022888184	1269.11840820313\\
5.6399998664856	1268.197265625\\
5.64499998092651	1267.37744140625\\
5.65000009536743	1266.72790527344\\
5.65500020980835	1265.87487792969\\
5.65999984741211	1265.34411621094\\
5.66499996185303	1265.02770996094\\
5.67000007629395	1264.91064453125\\
5.67500019073486	1268.63415527344\\
5.67999982833862	1266.59606933594\\
5.68499994277954	1263.49243164063\\
5.69000005722046	1263.31689453125\\
5.69500017166138	1264.25329589844\\
5.69999980926514	1266.63818359375\\
5.70499992370605	1269.68103027344\\
5.71000003814697	1272.369140625\\
5.71500015258789	1274.55822753906\\
5.71999979019165	1276.15490722656\\
5.72499990463257	1276.43872070313\\
5.73000001907349	1275.46655273438\\
5.7350001335144	1273.544921875\\
5.73999977111816	1271.40881347656\\
5.74499988555908	1269.45458984375\\
5.75	1267.76086425781\\
5.75500011444092	1266.03063964844\\
5.76000022888184	1264.37573242188\\
5.7649998664856	1263.08093261719\\
5.76999998092651	1262.48352050781\\
5.77500009536743	1262.32568359375\\
5.78000020980835	1262.27783203125\\
5.78499984741211	1261.69702148438\\
5.78999996185303	1260.521484375\\
5.79500007629395	1259.18493652344\\
5.80000019073486	1258.30517578125\\
5.80499982833862	1258.43212890625\\
5.80999994277954	1258.91467285156\\
5.81500005722046	1259.17163085938\\
5.82000017166138	1258.21276855469\\
5.82499980926514	1256.49829101563\\
5.82999992370605	1254.86010742188\\
5.83500003814697	1253.7548828125\\
5.84000015258789	1253.42053222656\\
5.84499979019165	1252.75415039063\\
5.84999990463257	1251.09497070313\\
5.85500001907349	1247.85815429688\\
5.8600001335144	1244.23034667969\\
5.86499977111816	1241.43737792969\\
5.86999988555908	1240.37097167969\\
5.875	1241.02014160156\\
5.88000011444092	1243.56286621094\\
5.88500022888184	1247.57067871094\\
5.8899998664856	1252.78930664063\\
5.89499998092651	1255.36413574219\\
5.90000009536743	1261.01208496094\\
5.90500020980835	1274.52392578125\\
5.90999984741211	1298.90893554688\\
5.91499996185303	1333.35791015625\\
5.92000007629395	1376.93432617188\\
5.92500019073486	1431.814453125\\
5.92999982833862	1496.80908203125\\
5.93499994277954	1564.9267578125\\
5.94000005722046	1627.67077636719\\
5.94500017166138	1682.17163085938\\
5.94999980926514	1729.15991210938\\
5.95499992370605	1767.18981933594\\
5.96000003814697	1793.77966308594\\
5.96500015258789	1804.00280761719\\
5.96999979019165	1795.96630859375\\
5.97499990463257	1771.58435058594\\
5.98000001907349	1706.76220703125\\
5.9850001335144	1599.48254394531\\
5.98999977111816	1478.28234863281\\
5.99499988555908	1369.55163574219\\
6	1289.29028320313\\
6.00500011444092	1243.76123046875\\
6.01000022888184	1224.66064453125\\
6.0149998664856	1228.33752441406\\
6.01999998092651	1254.4892578125\\
6.02500009536743	1299.935546875\\
6.03000020980835	1355.98522949219\\
6.03499984741211	1411.70910644531\\
6.03999996185303	1458.86193847656\\
6.04500007629395	1493.41442871094\\
6.05000019073486	1508.37231445313\\
6.05499982833862	1501.83557128906\\
6.05999994277954	1477.51953125\\
6.06500005722046	1441.83715820313\\
6.07000017166138	1402.18579101563\\
6.07499980926514	1364.23461914063\\
6.07999992370605	1331.64428710938\\
6.08500003814697	1306.74072265625\\
6.09000015258789	1293.80249023438\\
6.09499979019165	1293.03112792969\\
6.09999990463257	1301.10412597656\\
6.10500001907349	1313.55090332031\\
6.1100001335144	1327.78210449219\\
6.11499977111816	1342.72351074219\\
6.11999988555908	1355.73327636719\\
6.125	1363.79992675781\\
6.13000011444092	1363.96423339844\\
6.13500022888184	1356.09838867188\\
6.1399998664856	1342.30639648438\\
6.14499998092651	1325.03088378906\\
6.15000009536743	1306.84362792969\\
6.15500020980835	1289.52124023438\\
6.15999984741211	1275.43408203125\\
6.16499996185303	1265.83642578125\\
6.17000007629395	1260.85729980469\\
6.17500019073486	1259.78125\\
6.17999982833862	1262.00939941406\\
6.18499994277954	1267.23449707031\\
6.19000005722046	1271.75500488281\\
6.19500017166138	1274.76135253906\\
6.19999980926514	1275.74682617188\\
6.20499992370605	1274.75573730469\\
6.21000003814697	1272.31225585938\\
6.21500015258789	1270.61560058594\\
6.21999979019165	1269.57495117188\\
6.22499990463257	1269.14123535156\\
6.23000001907349	1269.02404785156\\
6.2350001335144	1268.82751464844\\
6.23999977111816	1268.47473144531\\
6.24499988555908	1267.98693847656\\
6.25	1267.61804199219\\
6.25500011444092	1267.494140625\\
6.26000022888184	1267.58020019531\\
6.2649998664856	1267.75537109375\\
6.26999998092651	1267.7109375\\
6.27500009536743	1267.61560058594\\
6.28000020980835	1267.47485351563\\
6.28499984741211	1267.48266601563\\
6.28999996185303	1267.47229003906\\
6.29500007629395	1267.52795410156\\
6.30000019073486	1267.580078125\\
6.30499982833862	1267.60534667969\\
6.30999994277954	1267.58984375\\
6.31500005722046	1267.51745605469\\
6.32000017166138	1267.55615234375\\
6.32499980926514	1267.53186035156\\
6.32999992370605	1267.69030761719\\
6.33500003814697	1267.59558105469\\
6.34000015258789	1267.64111328125\\
6.34499979019165	1267.57751464844\\
6.34999990463257	1267.74597167969\\
6.35500001907349	1267.58447265625\\
6.3600001335144	1261.98913574219\\
6.36499977111816	1255.88098144531\\
6.36999988555908	1249.89965820313\\
6.375	1244.228515625\\
6.38000011444092	1239.20849609375\\
6.38500022888184	1236.45336914063\\
6.3899998664856	1246.58422851563\\
6.39499998092651	1260.09997558594\\
6.40000009536743	1274.96130371094\\
6.40500020980835	1288.66552734375\\
6.40999984741211	1299.83422851563\\
6.41499996185303	1308.24011230469\\
6.42000007629395	1314.06616210938\\
6.42500019073486	1317.22265625\\
6.42999982833862	1317.79248046875\\
6.43499994277954	1316.41784667969\\
6.44000005722046	1313.46813964844\\
6.44500017166138	1309.76123046875\\
6.44999980926514	1306.20739746094\\
6.45499992370605	1304.17626953125\\
6.46000003814697	1304.33264160156\\
6.46500015258789	1306.43029785156\\
6.46999979019165	1309.48254394531\\
6.47499990463257	1312.52722167969\\
6.48000001907349	1315.43469238281\\
6.4850001335144	1318.71411132813\\
6.48999977111816	1322.55163574219\\
6.49499988555908	1326.8037109375\\
6.5	1330.41870117188\\
6.50500011444092	1332.37365722656\\
6.51000022888184	1332.69604492188\\
6.5149998664856	1331.89184570313\\
6.51999998092651	1330.92602539063\\
6.52500009536743	1330.5126953125\\
6.53000020980835	1330.51354980469\\
6.53499984741211	1330.26159667969\\
6.53999996185303	1329.06005859375\\
6.54500007629395	1327.12634277344\\
6.55000019073486	1325.13159179688\\
6.55499982833862	1324.18310546875\\
6.55999994277954	1324.43310546875\\
6.56500005722046	1325.16320800781\\
6.57000017166138	1325.35949707031\\
6.57499980926514	1324.59558105469\\
6.57999992370605	1322.9541015625\\
6.58500003814697	1321.6435546875\\
6.59000015258789	1321.21203613281\\
6.59499979019165	1321.3857421875\\
6.59999990463257	1321.298828125\\
6.60500001907349	1320.080078125\\
6.6100001335144	1317.76574707031\\
6.61499977111816	1315.0810546875\\
6.61999988555908	1312.98400878906\\
6.625	1312.01477050781\\
6.63000011444092	1311.60534667969\\
6.63500022888184	1310.79846191406\\
6.6399998664856	1308.98876953125\\
6.64499998092651	1306.50537109375\\
6.65000009536743	1304.50659179688\\
6.65500020980835	1303.44738769531\\
6.65999984741211	1303.64758300781\\
6.66499996185303	1303.78198242188\\
6.67000007629395	1303.18383789063\\
6.67500019073486	1301.708984375\\
6.67999982833862	1299.55041503906\\
6.68499994277954	1298.33471679688\\
6.69000005722046	1298.16320800781\\
6.69500017166138	1299.02722167969\\
6.69999980926514	1299.28405761719\\
6.70499992370605	1298.55078125\\
6.71000003814697	1296.88098144531\\
6.71500015258789	1295.53100585938\\
6.71999979019165	1295.26611328125\\
6.72499990463257	1296.05041503906\\
6.73000001907349	1297.05883789063\\
6.7350001335144	1297.0791015625\\
6.73999977111816	1296.14831542969\\
6.74499988555908	1294.4423828125\\
6.75	1293.24609375\\
6.75500011444092	1293.15112304688\\
6.76000022888184	1293.28369140625\\
6.7649998664856	1292.88500976563\\
6.76999998092651	1291.09545898438\\
6.77500009536743	1288.24926757813\\
6.78000020980835	1285.14111328125\\
6.78499984741211	1282.74963378906\\
6.78999996185303	1280.96533203125\\
6.79500007629395	1279.16662597656\\
6.80000019073486	1276.34521484375\\
6.80499982833862	1272.30517578125\\
6.80999994277954	1267.59973144531\\
6.81500005722046	1263.30505371094\\
6.82000017166138	1259.89306640625\\
6.82499980926514	1257.08129882813\\
6.82999992370605	1254.03393554688\\
6.83500003814697	1249.96838378906\\
6.84000015258789	1245.05749511719\\
6.84499979019165	1239.9443359375\\
6.84999990463257	1235.63708496094\\
6.85500001907349	1232.32568359375\\
6.8600001335144	1229.45422363281\\
6.86499977111816	1226.14196777344\\
6.86999988555908	1221.89489746094\\
6.875	1217.11145019531\\
6.88000011444092	1212.60534667969\\
6.88500022888184	1209.17700195313\\
6.8899998664856	1206.45043945313\\
6.89499998092651	1204.19470214844\\
6.90000009536743	1201.57739257813\\
6.90500020980835	1198.26989746094\\
6.90999984741211	1194.79821777344\\
6.91499996185303	1191.67822265625\\
6.92000007629395	1189.66552734375\\
6.92500019073486	1188.54418945313\\
6.92999982833862	1187.63732910156\\
6.93499994277954	1186.306640625\\
6.94000005722046	1184.51208496094\\
6.94500017166138	1182.61267089844\\
6.94999980926514	1181.31774902344\\
6.95499992370605	1180.84619140625\\
6.96000003814697	1180.99267578125\\
6.96500015258789	1181.33801269531\\
6.96999979019165	1181.14404296875\\
6.97499990463257	1180.65454101563\\
6.98000001907349	1180.23742675781\\
6.9850001335144	1180.390625\\
6.98999977111816	1181.138671875\\
6.99499988555908	1182.45935058594\\
7	1183.74084472656\\
7.00500011444092	1184.78894042969\\
7.01000022888184	1185.45471191406\\
7.0149998664856	1186.27722167969\\
7.01999998092651	1187.71362304688\\
7.02500009536743	1189.41979980469\\
7.03000020980835	1191.50231933594\\
7.03499984741211	1193.48901367188\\
7.03999996185303	1195.1279296875\\
7.04500007629395	1196.64916992188\\
7.05000019073486	1198.14404296875\\
7.05499982833862	1200.10302734375\\
7.05999994277954	1202.23608398438\\
7.06500005722046	1204.5419921875\\
7.07000017166138	1207.09912109375\\
7.07499980926514	1208.44750976563\\
7.07999992370605	1210.0576171875\\
7.08500003814697	1211.75817871094\\
7.09000015258789	1213.64990234375\\
7.09499979019165	1215.71557617188\\
7.09999990463257	1217.73266601563\\
7.10500001907349	1219.60729980469\\
7.1100001335144	1221.05322265625\\
7.11499977111816	1222.32043457031\\
7.11999988555908	1223.60559082031\\
7.125	1225.099609375\\
7.13000011444092	1226.66198730469\\
7.13500022888184	1227.98046875\\
7.1399998664856	1229.04858398438\\
7.14499998092651	1229.75244140625\\
7.15000009536743	1230.39770507813\\
7.15500020980835	1230.87585449219\\
7.15999984741211	1231.56384277344\\
7.16499996185303	1232.23864746094\\
7.17000007629395	1232.72692871094\\
7.17500019073486	1232.86865234375\\
7.17999982833862	1232.67163085938\\
7.18499994277954	1232.37463378906\\
7.19000005722046	1232.24658203125\\
7.19500017166138	1232.07275390625\\
7.19999980926514	1231.90515136719\\
7.20499992370605	1231.52783203125\\
7.21000003814697	1230.84033203125\\
7.21500015258789	1229.84008789063\\
7.21999979019165	1228.81579589844\\
7.22499990463257	1228.14782714844\\
7.23000001907349	1227.11889648438\\
7.2350001335144	1226.41577148438\\
7.23999977111816	1225.33386230469\\
7.24499988555908	1224.02819824219\\
7.25	1222.46789550781\\
7.25500011444092	1220.90270996094\\
7.26000022888184	1219.54272460938\\
7.2649998664856	1218.38781738281\\
7.26999998092651	1217.34814453125\\
7.27500009536743	1215.88842773438\\
7.28000020980835	1214.17834472656\\
7.28499984741211	1212.35107421875\\
7.28999996185303	1210.58923339844\\
7.29500007629395	1209.10266113281\\
7.30000019073486	1207.91357421875\\
7.30499982833862	1206.63415527344\\
7.30999994277954	1205.1650390625\\
7.31500005722046	1203.40063476563\\
7.32000017166138	1201.51672363281\\
7.32499980926514	1199.86047363281\\
7.32999992370605	1198.49853515625\\
7.33500003814697	1197.37316894531\\
7.34000015258789	1196.25891113281\\
7.34499979019165	1194.89672851563\\
7.34999990463257	1193.37927246094\\
7.35500001907349	1191.740234375\\
7.3600001335144	1190.29760742188\\
7.36499977111816	1189.2412109375\\
7.36999988555908	1188.44958496094\\
7.375	1187.68518066406\\
7.38000011444092	1186.6162109375\\
7.38500022888184	1185.40124511719\\
7.3899998664856	1184.10144042969\\
7.39499998092651	1183.08984375\\
7.40000009536743	1182.42712402344\\
7.40500020980835	1182.01623535156\\
7.40999984741211	1181.55383300781\\
7.41499996185303	1180.93579101563\\
7.42000007629395	1180.14501953125\\
7.42500019073486	1179.29370117188\\
7.42999982833862	1178.7587890625\\
7.43499994277954	1178.53247070313\\
7.44000005722046	1178.44519042969\\
7.44500017166138	1178.45092773438\\
7.44999980926514	1178.30395507813\\
7.45499992370605	1177.85314941406\\
7.46000003814697	1177.55639648438\\
7.46500015258789	1177.29663085938\\
7.46999979019165	1177.40576171875\\
7.47499990463257	1177.74829101563\\
7.48000001907349	1177.99426269531\\
7.4850001335144	1178.10119628906\\
7.48999977111816	1177.96899414063\\
7.49499988555908	1177.88256835938\\
7.5	1177.966796875\\
7.50500011444092	1178.42346191406\\
7.51000022888184	1178.86560058594\\
7.5149998664856	1179.29260253906\\
7.51999998092651	1179.52124023438\\
7.52500009536743	1179.57153320313\\
7.53000020980835	1179.64111328125\\
7.53499984741211	1179.85668945313\\
7.53999996185303	1180.28405761719\\
7.54500007629395	1180.8369140625\\
7.55000019073486	1181.283203125\\
7.55499982833862	1181.46350097656\\
7.55999994277954	1181.69250488281\\
7.56500005722046	1181.44738769531\\
7.57000017166138	1181.61474609375\\
7.57499980926514	1181.96154785156\\
7.57999992370605	1182.29479980469\\
7.58500003814697	1183.03820800781\\
7.59000015258789	1182.61120605469\\
7.59499979019165	1182.40869140625\\
7.59999990463257	1182.15319824219\\
7.60500001907349	1182.06823730469\\
7.6100001335144	1182.13415527344\\
7.61499977111816	1182.24145507813\\
7.61999988555908	1182.21008300781\\
7.625	1182.00329589844\\
7.63000011444092	1181.39562988281\\
7.63500022888184	1180.83666992188\\
7.6399998664856	1180.380859375\\
7.64499998092651	1180.17272949219\\
7.65000009536743	1179.85473632813\\
7.65500020980835	1179.70727539063\\
7.65999984741211	1178.77062988281\\
7.66499996185303	1177.90270996094\\
7.67000007629395	1176.93286132813\\
7.67500019073486	1176.07421875\\
7.67999982833862	1175.46032714844\\
7.68499994277954	1174.85266113281\\
7.69000005722046	1174.24731445313\\
7.69500017166138	1173.08251953125\\
7.69999980926514	1171.7197265625\\
7.70499992370605	1170.40283203125\\
7.71000003814697	1169.248046875\\
7.71500015258789	1168.28588867188\\
7.71999979019165	1167.36975097656\\
7.72499990463257	1166.34436035156\\
7.73000001907349	1164.91259765625\\
7.7350001335144	1163.49694824219\\
7.73999977111816	1161.70007324219\\
7.74499988555908	1160.24133300781\\
7.75	1159.03271484375\\
7.75500011444092	1157.87280273438\\
7.76000022888184	1156.57116699219\\
7.7649998664856	1154.96533203125\\
7.76999998092651	1153.17724609375\\
7.77500009536743	1151.29992675781\\
7.78000020980835	1149.64880371094\\
7.78499984741211	1148.29309082031\\
7.78999996185303	1146.9765625\\
7.79500007629395	1145.9697265625\\
7.80000019073486	1143.84741210938\\
7.80499982833862	1141.88427734375\\
7.80999994277954	1139.91101074219\\
7.81500005722046	1138.20300292969\\
7.82000017166138	1136.68566894531\\
7.82499980926514	1135.24426269531\\
7.82999992370605	1133.82336425781\\
7.83500003814697	1131.97790527344\\
7.84000015258789	1129.95129394531\\
7.84499979019165	1127.90173339844\\
7.84999990463257	1126.04809570313\\
7.85500001907349	1124.46057128906\\
7.8600001335144	1123.03820800781\\
7.86499977111816	1121.53198242188\\
7.86999988555908	1119.80187988281\\
7.875	1117.8056640625\\
7.88000011444092	1115.71301269531\\
7.88500022888184	1113.791015625\\
7.8899998664856	1112.15246582031\\
7.89499998092651	1110.71044921875\\
7.90000009536743	1109.23449707031\\
7.90500020980835	1107.58666992188\\
7.90999984741211	1105.54772949219\\
7.91499996185303	1103.4306640625\\
7.92000007629395	1101.43872070313\\
7.92500019073486	1099.72509765625\\
7.92999982833862	1098.26977539063\\
7.93499994277954	1096.75769042969\\
7.94000005722046	1095.08020019531\\
7.94500017166138	1093.08764648438\\
7.94999980926514	1090.91821289063\\
7.95499992370605	1088.84387207031\\
7.96000003814697	1087.02075195313\\
7.96500015258789	1085.42883300781\\
7.96999979019165	1083.89379882813\\
7.97499990463257	1082.17309570313\\
7.98000001907349	1080.19018554688\\
7.9850001335144	1077.95678710938\\
7.98999977111816	1075.67907714844\\
7.99499988555908	1073.62255859375\\
8	1071.86486816406\\
8.00500011444092	1070.1953125\\
8.01000022888184	1068.38549804688\\
8.01500034332275	1066.35168457031\\
8.02000045776367	1063.90576171875\\
8.02499961853027	1061.40625\\
8.02999973297119	1059.06018066406\\
8.03499984741211	1056.95629882813\\
8.03999996185303	1055.08557128906\\
8.04500007629395	1053.11608886719\\
8.05000019073486	1050.869140625\\
8.05500030517578	1048.28210449219\\
8.0600004196167	1045.46350097656\\
8.0649995803833	1042.697265625\\
8.06999969482422	1040.16271972656\\
8.07499980926514	1037.87512207031\\
8.07999992370605	1035.65930175781\\
8.08500003814697	1033.439453125\\
8.09000015258789	1030.44519042969\\
8.09500026702881	1027.28601074219\\
8.10000038146973	1024.11853027344\\
8.10499954223633	1020.97045898438\\
8.10999965667725	1018.16021728516\\
8.11499977111816	1015.51202392578\\
8.11999988555908	1012.80383300781\\
8.125	1009.85375976563\\
8.13000011444092	1006.28143310547\\
8.13500022888184	1002.54473876953\\
8.14000034332275	998.883728027344\\
8.14500045776367	995.367736816406\\
8.14999961853027	992.205322265625\\
8.15499973297119	989.138427734375\\
8.15999984741211	985.83349609375\\
8.16499996185303	982.151428222656\\
8.17000007629395	977.953796386719\\
8.17500019073486	973.630065917969\\
8.18000030517578	969.435424804688\\
8.1850004196167	965.552917480469\\
8.1899995803833	961.966491699219\\
8.19499969482422	958.331970214844\\
8.19999980926514	954.395141601563\\
8.20499992370605	950.005859375\\
8.21000003814697	945.289672851563\\
8.21500015258789	940.294372558594\\
8.22000026702881	935.648498535156\\
8.22500038146973	931.301086425781\\
8.22999954223633	927.255126953125\\
8.23499965667725	923.203857421875\\
8.23999977111816	918.765441894531\\
8.24499988555908	913.803283691406\\
8.25	908.457275390625\\
8.25500011444092	902.986206054688\\
8.26000022888184	897.819519042969\\
8.26500034332275	893.091674804688\\
8.27000045776367	888.696105957031\\
8.27499961853027	884.30224609375\\
8.27999973297119	879.548645019531\\
8.28499984741211	874.246459960938\\
8.28999996185303	868.502075195313\\
8.29500007629395	862.641296386719\\
8.30000019073486	857.077087402344\\
8.30500030517578	852.066223144531\\
8.3100004196167	847.484680175781\\
8.3149995803833	843.080444335938\\
8.31999969482422	838.417236328125\\
8.32499980926514	833.201782226563\\
8.32999992370605	827.473876953125\\
8.33500003814697	821.533752441406\\
8.34000015258789	815.846557617188\\
8.34500026702881	810.736877441406\\
8.35000038146973	806.317199707031\\
8.35499954223633	802.327392578125\\
8.35999965667725	798.372924804688\\
8.36499977111816	794.0126953125\\
8.36999988555908	789.130310058594\\
8.375	783.872192382813\\
8.38000011444092	778.646057128906\\
8.38500022888184	774.019897460938\\
8.39000034332275	770.294250488281\\
8.39500045776367	767.467102050781\\
8.39999961853027	765.194213867188\\
8.40499973297119	763.002075195313\\
8.40999984741211	760.448791503906\\
8.41499996185303	757.448181152344\\
8.42000007629395	754.276184082031\\
8.42500019073486	751.473388671875\\
8.43000030517578	749.654357910156\\
8.4350004196167	749.117553710938\\
8.4399995803833	749.850158691406\\
8.44499969482422	751.418579101563\\
8.44999980926514	753.20849609375\\
8.45499992370605	754.723510742188\\
8.46000003814697	755.801147460938\\
8.46500015258789	756.773681640625\\
8.47000026702881	758.236572265625\\
8.47500038146973	760.842041015625\\
8.47999954223633	764.9541015625\\
8.48499965667725	770.497375488281\\
8.48999977111816	776.886901855469\\
8.49499988555908	783.374633789063\\
8.5	789.336120605469\\
8.50500011444092	794.590698242188\\
8.51000022888184	799.517883300781\\
8.51500034332275	804.83251953125\\
8.52000045776367	811.288208007813\\
8.52499961853027	819.200744628906\\
8.52999973297119	828.335571289063\\
8.53499984741211	837.919677734375\\
8.53999996185303	847.005187988281\\
8.54500007629395	854.998962402344\\
8.55000019073486	861.936279296875\\
8.55500030517578	868.477661132813\\
8.5600004196167	875.560668945313\\
8.5649995803833	883.906494140625\\
8.56999969482422	893.622314453125\\
8.57499980926514	904.076965332031\\
8.57999992370605	914.293518066406\\
8.58500003814697	923.393188476563\\
8.59000015258789	931.2373046875\\
8.59500026702881	938.435607910156\\
8.60000038146973	946.148010253906\\
8.60499954223633	955.314514160156\\
8.60999965667725	966.214050292969\\
8.61499977111816	978.221557617188\\
8.61999988555908	990.267272949219\\
8.625	1001.41552734375\\
8.63000011444092	1011.54364013672\\
8.63500022888184	1021.30633544922\\
8.64000034332275	1031.77783203125\\
8.64500045776367	1043.71057128906\\
8.64999961853027	1057.00268554688\\
8.65499973297119	1070.68322753906\\
8.65999984741211	1083.66064453125\\
8.66499996185303	1095.05578613281\\
8.67000007629395	1105.19543457031\\
8.67500019073486	1114.95129394531\\
8.68000030517578	1125.42529296875\\
8.6850004196167	1137.11804199219\\
8.6899995803833	1149.55517578125\\
8.69499969482422	1161.7314453125\\
8.69999980926514	1172.693359375\\
8.70499992370605	1182.40930175781\\
8.71000003814697	1191.61608886719\\
8.71500015258789	1201.43298339844\\
8.72000026702881	1212.4775390625\\
8.72500038146973	1224.3896484375\\
8.72999954223633	1236.11474609375\\
8.73499965667725	1246.66088867188\\
8.73999977111816	1255.78112792969\\
8.74499988555908	1264.20153808594\\
8.75	1272.85083007813\\
8.75500011444092	1282.30712890625\\
8.76000022888184	1292.28637695313\\
8.76500034332275	1301.8095703125\\
8.77000045776367	1310.08435058594\\
8.77499961853027	1317.08349609375\\
8.77999973297119	1323.51513671875\\
8.78499984741211	1330.38598632813\\
8.78999996185303	1338.17614746094\\
8.79500007629395	1346.42700195313\\
8.80000019073486	1354.18737792969\\
8.80500030517578	1360.88439941406\\
8.8100004196167	1366.56274414063\\
8.8149995803833	1372.02844238281\\
8.81999969482422	1378.06201171875\\
8.82499980926514	1384.78552246094\\
8.82999992370605	1391.55639648438\\
8.83500003814697	1397.52575683594\\
8.84000015258789	1402.26550292969\\
8.84500026702881	1406.28503417969\\
8.85000038146973	1410.34594726563\\
8.85499954223633	1415.087890625\\
8.85999965667725	1420.3525390625\\
8.86499977111816	1425.42517089844\\
8.86999988555908	1429.61804199219\\
8.875	1432.91247558594\\
8.88000011444092	1435.94812011719\\
8.88500022888184	1439.47973632813\\
8.89000034332275	1443.76806640625\\
8.89500045776367	1448.32836914063\\
8.89999961853027	1452.21789550781\\
8.90499973297119	1455.22241210938\\
8.90999984741211	1457.48669433594\\
8.91499996185303	1459.85498046875\\
8.92000007629395	1462.85473632813\\
8.92500019073486	1466.35803222656\\
8.93000030517578	1469.61755371094\\
8.9350004196167	1472.072265625\\
8.9399995803833	1473.67468261719\\
8.94499969482422	1475.05285644531\\
8.94999980926514	1476.95849609375\\
8.95499992370605	1479.55712890625\\
8.96000003814697	1482.41528320313\\
8.96500015258789	1484.6328125\\
8.97000026702881	1485.94909667969\\
8.97500038146973	1486.71826171875\\
8.97999954223633	1487.76098632813\\
8.98499965667725	1489.46826171875\\
8.98999977111816	1491.48095703125\\
8.99499988555908	1493.20324707031\\
9	1493.99914550781\\
9.00500011444092	1494.04309082031\\
9.01000022888184	1493.96740722656\\
9.01500034332275	1494.41467285156\\
9.02000045776367	1495.44519042969\\
9.02499961853027	1496.41003417969\\
9.02999973297119	1496.6318359375\\
9.03499984741211	1495.91284179688\\
9.03999996185303	1494.87731933594\\
9.04500007629395	1494.01342773438\\
9.05000019073486	1493.81994628906\\
9.05500030517578	1493.85424804688\\
9.0600004196167	1493.36181640625\\
9.0649995803833	1491.89770507813\\
9.06999969482422	1489.79443359375\\
9.07499980926514	1487.71142578125\\
9.07999992370605	1486.16088867188\\
9.08500003814697	1485.06042480469\\
9.09000015258789	1483.72192382813\\
9.09500026702881	1481.65625\\
9.10000038146973	1478.64233398438\\
9.10499954223633	1475.35827636719\\
9.10999965667725	1472.5126953125\\
9.11499977111816	1470.34155273438\\
9.11999988555908	1468.14904785156\\
9.125	1465.57739257813\\
9.13000011444092	1461.90502929688\\
9.13500022888184	1457.90124511719\\
9.14000034332275	1454.177734375\\
9.14500045776367	1451.0224609375\\
9.14999961853027	1448.26538085938\\
9.15499973297119	1445.29479980469\\
9.15999984741211	1441.56628417969\\
9.16499996185303	1437.28198242188\\
9.17000007629395	1432.87524414063\\
9.17500019073486	1429.02233886719\\
9.18000030517578	1425.81384277344\\
9.1850004196167	1422.70837402344\\
9.1899995803833	1419.32116699219\\
9.19499969482422	1415.12536621094\\
9.19999980926514	1410.86108398438\\
9.20499992370605	1406.70324707031\\
9.21000003814697	1403.31909179688\\
9.21500015258789	1400.51342773438\\
9.22000026702881	1397.48718261719\\
9.22500038146973	1394.19543457031\\
9.22999954223633	1390.23510742188\\
9.23499965667725	1386.39990234375\\
9.23999977111816	1383.24645996094\\
9.24499988555908	1380.91467285156\\
9.25	1379.1015625\\
9.25500011444092	1377.31506347656\\
9.26000022888184	1375.23168945313\\
9.26500034332275	1373.33093261719\\
9.27000045776367	1372.18432617188\\
9.27499961853027	1372.244140625\\
9.27999973297119	1373.470703125\\
9.28499984741211	1374.97009277344\\
9.28999996185303	1376.42663574219\\
9.29500007629395	1377.83203125\\
9.30000019073486	1379.99645996094\\
9.30500030517578	1383.39611816406\\
9.3100004196167	1388.72253417969\\
9.3149995803833	1395.43835449219\\
9.31999969482422	1402.79919433594\\
9.32499980926514	1410.33654785156\\
9.32999992370605	1418.509765625\\
9.33500003814697	1428.474609375\\
9.34000015258789	1441.17736816406\\
9.34500026702881	1457.099609375\\
9.35000038146973	1475.39880371094\\
9.35499954223633	1494.92712402344\\
9.35999965667725	1515.29187011719\\
9.36499977111816	1537.58471679688\\
9.36999988555908	1562.83361816406\\
9.375	1591.71520996094\\
9.38000011444092	1623.50830078125\\
9.38500022888184	1655.45080566406\\
9.39000034332275	1684.97033691406\\
9.39500045776367	1711.10949707031\\
9.39999961853027	1734.7705078125\\
9.40499973297119	1756.77783203125\\
9.40999984741211	1775.70080566406\\
9.41499996185303	1787.72094726563\\
9.42000007629395	1789.59875488281\\
9.42500019073486	1780.91442871094\\
9.43000030517578	1764.49792480469\\
9.4350004196167	1742.4736328125\\
9.4399995803833	1714.84936523438\\
9.44499969482422	1679.35437011719\\
9.44999980926514	1635.771484375\\
9.45499992370605	1586.93334960938\\
9.46000003814697	1537.0439453125\\
9.46500015258789	1488.93176269531\\
9.47000026702881	1443.34716796875\\
9.47500038146973	1401.80493164063\\
9.47999954223633	1366.70593261719\\
9.48499965667725	1341.63439941406\\
9.48999977111816	1327.3017578125\\
9.49499988555908	1319.32897949219\\
9.5	1317.40954589844\\
9.50500011444092	1319.2431640625\\
9.51000022888184	1321.78881835938\\
9.51500034332275	1322.9990234375\\
9.52000045776367	1322.36096191406\\
9.52499961853027	1320.33605957031\\
9.52999973297119	1317.56823730469\\
9.53499984741211	1314.70349121094\\
9.53999996185303	1311.36279296875\\
9.54500007629395	1307.52905273438\\
9.55000019073486	1303.01416015625\\
9.55500030517578	1297.54919433594\\
9.5600004196167	1292.18798828125\\
9.5649995803833	1288.37316894531\\
9.56999969482422	1285.70129394531\\
9.57499980926514	1283.37268066406\\
9.57999992370605	1280.935546875\\
9.58500003814697	1278.43322753906\\
9.59000015258789	1276.30212402344\\
9.59500026702881	1274.81799316406\\
9.60000038146973	1273.89208984375\\
9.60499954223633	1273.31591796875\\
9.60999965667725	1272.58374023438\\
9.61499977111816	1271.73083496094\\
9.61999988555908	1270.89233398438\\
9.625	1270.21337890625\\
9.63000011444092	1269.91552734375\\
9.63500022888184	1269.56176757813\\
9.64000034332275	1269.39404296875\\
9.64500045776367	1269.11071777344\\
9.64999961853027	1268.78405761719\\
9.65499973297119	1268.58764648438\\
9.65999984741211	1268.27075195313\\
9.66499996185303	1268.15734863281\\
9.67000007629395	1268.06799316406\\
9.67500019073486	1268.00817871094\\
9.68000030517578	1267.962890625\\
9.6850004196167	1267.86938476563\\
9.6899995803833	1267.65502929688\\
9.69499969482422	1267.6201171875\\
9.69999980926514	1267.56530761719\\
9.70499992370605	1267.52185058594\\
9.71000003814697	1267.52795410156\\
9.71500015258789	1267.51916503906\\
9.72000026702881	1267.494140625\\
9.72500038146973	1267.38671875\\
9.72999954223633	1267.35815429688\\
9.73499965667725	1267.33825683594\\
9.73999977111816	1267.345703125\\
9.74499988555908	1267.42272949219\\
9.75	1267.30444335938\\
9.75500011444092	1267.29235839844\\
9.76000022888184	1267.28540039063\\
9.76500034332275	1267.27770996094\\
9.77000045776367	1267.25732421875\\
9.77499961853027	1267.27697753906\\
9.77999973297119	1267.24963378906\\
9.78499984741211	1267.23205566406\\
9.78999996185303	1267.30688476563\\
9.79500007629395	1267.22521972656\\
9.80000019073486	1267.22399902344\\
9.80500030517578	1267.35583496094\\
9.8100004196167	1267.25732421875\\
9.8149995803833	1267.23266601563\\
9.81999969482422	1267.28881835938\\
9.82499980926514	1267.20141601563\\
9.82999992370605	1267.25439453125\\
9.83500003814697	1267.20007324219\\
9.84000015258789	1267.22277832031\\
9.84500026702881	1267.23010253906\\
9.85000038146973	1267.20947265625\\
9.85499954223633	1267.20373535156\\
9.85999965667725	1267.20178222656\\
9.86499977111816	1267.20104980469\\
9.86999988555908	1267.20520019531\\
9.875	1267.26403808594\\
9.88000011444092	1267.21618652344\\
9.88500022888184	1267.27209472656\\
9.89000034332275	1267.19104003906\\
9.89500045776367	1267.3271484375\\
9.89999961853027	1267.20825195313\\
9.90499973297119	1267.20947265625\\
9.90999984741211	1267.23803710938\\
9.91499996185303	1267.22241210938\\
9.92000007629395	1267.21923828125\\
9.92500019073486	1267.19897460938\\
9.93000030517578	1267.19470214844\\
9.9350004196167	1267.24658203125\\
9.9399995803833	1267.30139160156\\
9.94499969482422	1267.21044921875\\
9.94999980926514	1267.31066894531\\
9.95499992370605	1267.16003417969\\
9.96000003814697	1267.18371582031\\
9.96500015258789	1267.25744628906\\
9.97000026702881	1267.208984375\\
9.97500038146973	1267.20617675781\\
9.97999954223633	1267.16760253906\\
9.98499965667725	1267.1943359375\\
9.98999977111816	1267.19702148438\\
9.99499988555908	1267.26831054688\\
10	1267.23046875\\
};
\addlegendentry{CF}

\end{axis}
\end{tikzpicture}%
    \end{tikzpicture}}
    \caption{Loads at point B of CF under pure feedforward control}
    \label{fig:pureFeedFwdB}
\end{figure}



% --------------------------------------------------------------------------
% 		Section 3.3
% --------------------------------------------------------------------------
\section{Classical Feedback Control}
\label{sec:chap3sec3}
Figure \ref{fig:feedback} shows a decentralized feedback controller which generates the controller effort based on the output of the plant; the translational and angular position coordinates of the CF Bike. It is important to note that a positive feedback is used due to certain constraints in Simpack, and for now regulation is attempted, i.e. the reference position is zero. 

\subsection*{PD Control}
The feedback controller is set such that:
$$K_x = K_y = K_z = 100 + 500s$$

The tuning of the controller is done such that the derivative control is sufficiently high so as to increase the response speed and to counter imbalances with higher rates of change. The validation results can be visualized in Figures \ref{fig:pureFeedbkPDA} to \ref{fig:pureFeedbkPDC}, while the errors between CF and RS has been quantified in Tables \ref{tab:pureFeedbkPDA} to \ref{tab:pureFeedbkPDC}. 

\begin{figure}[h]
	\centering
	\tikzstyle{block}     = [draw, rectangle, minimum height=1.5cm, minimum width=1.6cm]
	\tikzstyle{branch}    = [circle, inner sep=0pt, minimum size=0.01mm, fill=black, draw=black]
	\tikzstyle{connector} = [->, thick]
	\tikzstyle{dummy}     = [inner sep=0pt, minimum size=0pt]
	\tikzstyle{inout}     = []
	\tikzstyle{sum}       = [circle, inner sep=0pt, minimum size=2mm, draw=black, thick]
	\begin{tikzpicture}[auto, node distance=3cm, >=stealth']
		\node[block] (bike) {CF};
		\node[block, left of = bike] (PD) {$\begin{matrix}
				K_x&&\\
				&K_y&\\
				&&K_z
			\end{matrix}$};
		\node[inout,right of = bike] (y) {\textbf{y}};
		\node[sum,left of = PD] (s) {};
		\node[branch,right of = bike,node distance = 1.5cm] (b1) {};
		\node[branch,below of = b1] (b2) {};
		
		\draw[connector] (PD) -- node{\textbf{u}} (bike);
		\draw[connector] (bike) -- (y);
		\draw[thick] (b1) -- (b2);
		\draw[connector] (b2) -| (s);
		\draw[connector] (s) -- (PD);
	\end{tikzpicture}
	\caption{Classical Feedback Control}
	\label{fig:feedback}
\end{figure}

Thus, PD Control is promising for forces at point B and X component of forces at point A and C, while it shows mediocre performance for Y components of forces at A and C.

 \begin{table}[h!]
	\centering
	\begin{tabular}{ |c|c|c|c| } 
		\hline
		Forces & Mean Error (\%) & RMSE & $R_2$\\ 
		\hline
		FX & 3&127&0.98\\ 
		FY & 99&906&0 \\ 
		\hline
	\end{tabular}
	\caption{Error Tabulation of Loads at Point A of CF under PD Feedback Control}
	\label{tab:pureFeedbkPDA}
\end{table}

\begin{table}[h!]
	\centering
	\begin{tabular}{ |c|c|c|c| } 
		\hline
		Forces & Mean Error (\%) & RMSE & $R_2$\\ 
		\hline
		FX & 3&15&0.99\\
		FY&3&2&0.99\\
		\hline
	\end{tabular}
	\caption{Error Tabulation of Loads at Point B of CF under PD Feedback Control}
	\label{tab:pureFeedbkPDB}
\end{table}

\begin{table}[h!]
	\centering
	\begin{tabular}{ |c|c|c|c| } 
		\hline
		Forces & Mean Error (\%) & RMSE & $R_2$\\ 
		\hline
		FX & 4&149&0.95\\ 
		FY & 56&901&0.05 \\ 
		\hline
	\end{tabular}
	\caption{Error Tabulation of Loads at Point C of CF under PD Feedback Control}
	\label{tab:pureFeedbkPDC}
\end{table}

\begin{figure}[h!]
    \centering
    \scalebox{0.8}{
    \begin{tikzpicture}
        % This file was created by matlab2tikz.
%
%The latest updates can be retrieved from
%  http://www.mathworks.com/matlabcentral/fileexchange/22022-matlab2tikz-matlab2tikz
%where you can also make suggestions and rate matlab2tikz.
%
\begin{tikzpicture}

\begin{axis}[%
width=4.521in,
height=1.476in,
at={(0.758in,2.571in)},
scale only axis,
xmin=0,
xmax=10,
xlabel style={font=\color{white!15!black}},
xlabel={Time (s)},
ymin=-70.0275421142578,
ymax=6000,
ylabel style={font=\color{white!15!black}},
ylabel={FX (N)},
axis background/.style={fill=white},
xmajorgrids,
ymajorgrids,
legend style={at={(0.85,1)}, anchor=north east, legend cell align=left, align=left, draw=black}
]
\addplot [color=black, dashed, line width=2.0pt]
  table[row sep=crcr]{%
0.0949999988079071	15.8128757476807\\
0.100000001490116	13.5325384140015\\
0.104999996721745	11.553840637207\\
0.109999999403954	9.83138465881348\\
0.115000002086163	8.33064746856689\\
0.119999997317791	241.457473754883\\
0.125	545.687744140625\\
0.129999995231628	782.91650390625\\
0.135000005364418	971.380676269531\\
0.140000000596046	1111.84790039063\\
0.144999995827675	1206.85559082031\\
0.150000005960464	1259.75073242188\\
0.155000001192093	1273.865234375\\
0.159999996423721	1267.48400878906\\
0.165000006556511	1232.30285644531\\
0.170000001788139	1162.44714355469\\
0.174999997019768	1059.94775390625\\
0.180000007152557	929.114501953125\\
0.185000002384186	775.30615234375\\
0.189999997615814	605.3291015625\\
0.194999992847443	507.519195556641\\
0.200000002980232	487.125366210938\\
0.204999998211861	502.143249511719\\
0.209999993443489	656.599975585938\\
0.215000003576279	913.404174804688\\
0.219999998807907	1241.50134277344\\
0.224999994039536	1600.14453125\\
0.230000004172325	1950.97143554688\\
0.234999999403954	2263.14721679688\\
0.239999994635582	2513.54858398438\\
0.245000004768372	2687.57348632813\\
0.25	2780.49365234375\\
0.254999995231628	2801.94165039063\\
0.259999990463257	2800.33056640625\\
0.264999985694885	2747.60107421875\\
0.270000010728836	2646.53955078125\\
0.275000005960464	2522.55908203125\\
0.280000001192093	2403.4287109375\\
0.284999996423721	2311.9716796875\\
0.28999999165535	2261.88549804688\\
0.294999986886978	2274.046875\\
0.300000011920929	2331.79541015625\\
0.305000007152557	2402.70849609375\\
0.310000002384186	2470.48901367188\\
0.314999997615814	2521.54077148438\\
0.319999992847443	2545.62353515625\\
0.324999988079071	2536.87817382813\\
0.330000013113022	2510.89990234375\\
0.33500000834465	2457.54736328125\\
0.340000003576279	2372.53881835938\\
0.344999998807907	2262.09594726563\\
0.349999994039536	2136.60034179688\\
0.354999989271164	2007.76867675781\\
0.360000014305115	1886.05541992188\\
0.365000009536743	1779.15637207031\\
0.370000004768372	1691.68542480469\\
0.375	1623.36535644531\\
0.379999995231628	1571.14392089844\\
0.384999990463257	1529.43896484375\\
0.389999985694885	1492.29345703125\\
0.395000010728836	1452.31481933594\\
0.400000005960464	1404.74157714844\\
0.405000001192093	1345.31884765625\\
0.409999996423721	1273.0029296875\\
0.41499999165535	1188.55444335938\\
0.419999986886978	1095.02856445313\\
0.425000011920929	997.367248535156\\
0.430000007152557	900.260681152344\\
0.435000002384186	808.484313964844\\
0.439999997615814	726.714111328125\\
0.444999992847443	657.466491699219\\
0.449999988079071	601.405578613281\\
0.455000013113022	557.660766601563\\
0.46000000834465	523.948852539063\\
0.465000003576279	497.415161132813\\
0.469999998807907	474.441253662109\\
0.474999994039536	452.282165527344\\
0.479999989271164	428.593688964844\\
0.485000014305115	402.142364501953\\
0.490000009536743	372.830108642578\\
0.495000004768372	342.247314453125\\
0.5	311.842437744141\\
0.504999995231628	283.43017578125\\
0.509999990463257	259.353729248047\\
0.514999985694885	241.675308227539\\
0.519999980926514	231.731582641602\\
0.524999976158142	231.725280761719\\
0.529999971389771	243.398681640625\\
0.535000026226044	259.713195800781\\
0.540000021457672	279.501007080078\\
0.545000016689301	301.19580078125\\
0.550000011920929	323.785949707031\\
0.555000007152557	347.092620849609\\
0.560000002384186	370.847473144531\\
0.564999997615814	394.982513427734\\
0.569999992847443	419.795562744141\\
0.574999988079071	445.720886230469\\
0.579999983310699	473.269683837891\\
0.584999978542328	502.888336181641\\
0.589999973773956	534.819213867188\\
0.595000028610229	569.128356933594\\
0.600000023841858	605.685791015625\\
0.605000019073486	644.196350097656\\
0.610000014305115	684.252075195313\\
0.615000009536743	725.32470703125\\
0.620000004768372	766.879150390625\\
0.625	808.521301269531\\
0.629999995231628	849.854125976563\\
0.634999990463257	890.603149414063\\
0.639999985694885	930.506164550781\\
0.644999980926514	969.407104492188\\
0.649999976158142	1007.24786376953\\
0.654999971389771	1044.009765625\\
0.660000026226044	1079.69921875\\
0.665000021457672	1114.3212890625\\
0.670000016689301	1147.87023925781\\
0.675000011920929	1180.31591796875\\
0.680000007152557	1211.5830078125\\
0.685000002384186	1241.55407714844\\
0.689999997615814	1270.08154296875\\
0.694999992847443	1296.99230957031\\
0.699999988079071	1322.1181640625\\
0.704999983310699	1345.31298828125\\
0.709999978542328	1366.45935058594\\
0.714999973773956	1385.47399902344\\
0.720000028610229	1402.32116699219\\
0.725000023841858	1417.0087890625\\
0.730000019073486	1429.50866699219\\
0.735000014305115	1439.89465332031\\
0.740000009536743	1448.19323730469\\
0.745000004768372	1454.47351074219\\
0.75	1458.77514648438\\
0.754999995231628	1461.15893554688\\
0.759999990463257	1461.66162109375\\
0.764999985694885	1460.9453125\\
0.769999980926514	1458.82690429688\\
0.774999976158142	1454.93518066406\\
0.779999971389771	1449.13818359375\\
0.785000026226044	1441.51635742188\\
0.790000021457672	1432.18530273438\\
0.795000016689301	1421.24914550781\\
0.800000011920929	1408.8095703125\\
0.805000007152557	1394.96813964844\\
0.810000002384186	1379.85229492188\\
0.814999997615814	1363.57409667969\\
0.819999992847443	1346.26379394531\\
0.824999988079071	1328.04174804688\\
0.829999983310699	1309.04211425781\\
0.834999978542328	1289.38647460938\\
0.839999973773956	1269.1904296875\\
0.845000028610229	1248.5751953125\\
0.850000023841858	1227.65014648438\\
0.855000019073486	1206.51721191406\\
0.860000014305115	1185.24890136719\\
0.865000009536743	1163.93676757813\\
0.870000004768372	1142.68041992188\\
0.875	1121.56518554688\\
0.879999995231628	1100.68347167969\\
0.884999990463257	1080.12268066406\\
0.889999985694885	1059.97058105469\\
0.894999980926514	1040.31640625\\
0.899999976158142	1021.23425292969\\
0.904999971389771	1002.79235839844\\
0.910000026226044	985.079162597656\\
0.915000021457672	968.158874511719\\
0.920000016689301	952.072631835938\\
0.925000011920929	936.905944824219\\
0.930000007152557	922.71533203125\\
0.935000002384186	909.507080078125\\
0.939999997615814	897.360412597656\\
0.944999992847443	886.282592773438\\
0.949999988079071	876.221557617188\\
0.954999983310699	867.238037109375\\
0.959999978542328	859.328918457031\\
0.964999973773956	852.488342285156\\
0.970000028610229	846.735595703125\\
0.975000023841858	842.054748535156\\
0.980000019073486	838.443237304688\\
0.985000014305115	835.909423828125\\
0.990000009536743	834.473388671875\\
0.995000004768372	834.134887695313\\
1	834.940368652344\\
1.00499999523163	836.931640625\\
1.00999999046326	840.085693359375\\
1.01499998569489	843.941589355469\\
1.01999998092651	848.52978515625\\
1.02499997615814	853.818969726563\\
1.02999997138977	859.83203125\\
1.0349999666214	866.488647460938\\
1.03999996185303	873.778686523438\\
1.04499995708466	881.638671875\\
1.04999995231628	890.044189453125\\
1.05499994754791	898.933837890625\\
1.05999994277954	908.264404296875\\
1.06500005722046	918.013854980469\\
1.07000005245209	928.100463867188\\
1.07500004768372	938.489135742188\\
1.08000004291534	949.115966796875\\
1.08500003814697	959.931518554688\\
1.0900000333786	970.880981445313\\
1.09500002861023	981.910339355469\\
1.10000002384186	992.965515136719\\
1.10500001907349	1003.99652099609\\
1.11000001430511	1014.96075439453\\
1.11500000953674	1025.80102539063\\
1.12000000476837	1036.46569824219\\
1.125	1046.923828125\\
1.12999999523163	1057.12646484375\\
1.13499999046326	1067.0224609375\\
1.13999998569489	1076.57495117188\\
1.14499998092651	1085.78002929688\\
1.14999997615814	1094.57946777344\\
1.15499997138977	1102.93493652344\\
1.1599999666214	1110.84558105469\\
1.16499996185303	1118.28686523438\\
1.16999995708466	1125.21948242188\\
1.17499995231628	1131.62170410156\\
1.17999994754791	1137.5126953125\\
1.18499994277954	1142.85180664063\\
1.19000005722046	1147.61901855469\\
1.19500005245209	1151.82360839844\\
1.20000004768372	1155.47302246094\\
1.20500004291534	1158.55688476563\\
1.21000003814697	1161.07751464844\\
1.2150000333786	1163.02893066406\\
1.22000002861023	1164.42004394531\\
1.22500002384186	1165.25793457031\\
1.23000001907349	1165.58508300781\\
1.23500001430511	1165.45751953125\\
1.24000000953674	1164.89929199219\\
1.24500000476837	1163.95678710938\\
1.25	1162.47705078125\\
1.25499999523163	1160.52404785156\\
1.25999999046326	1158.08508300781\\
1.26499998569489	1155.18664550781\\
1.26999998092651	1151.88452148438\\
1.27499997615814	1148.15612792969\\
1.27999997138977	1144.03771972656\\
1.2849999666214	1139.64074707031\\
1.28999996185303	1134.95532226563\\
1.29499995708466	1130.01501464844\\
1.29999995231628	1124.86279296875\\
1.30499994754791	1119.52990722656\\
1.30999994277954	1114.05151367188\\
1.31500005722046	1108.42724609375\\
1.32000005245209	1102.69287109375\\
1.32500004768372	1096.87829589844\\
1.33000004291534	1090.99731445313\\
1.33500003814697	1085.07141113281\\
1.3400000333786	1079.12731933594\\
1.34500002861023	1073.19274902344\\
1.35000002384186	1067.29809570313\\
1.35500001907349	1061.46875\\
1.36000001430511	1055.74304199219\\
1.36500000953674	1050.18200683594\\
1.37000000476837	1044.82153320313\\
1.375	1039.6962890625\\
1.37999999523163	1034.73620605469\\
1.38499999046326	1030.00659179688\\
1.38999998569489	1025.52758789063\\
1.39499998092651	1021.24743652344\\
1.39999997615814	1017.19683837891\\
1.40499997138977	1013.38421630859\\
1.4099999666214	1009.83258056641\\
1.41499996185303	1006.54571533203\\
1.41999995708466	1003.53466796875\\
1.42499995231628	1000.81591796875\\
1.42999994754791	998.398681640625\\
1.43499994277954	996.287414550781\\
1.44000005722046	994.474914550781\\
1.44500005245209	992.973999023438\\
1.45000004768372	991.7802734375\\
1.45500004291534	990.884582519531\\
1.46000003814697	990.292724609375\\
1.4650000333786	990.022094726563\\
1.47000002861023	990.094055175781\\
1.47500002384186	990.520568847656\\
1.48000001907349	991.191162109375\\
1.48500001430511	992.031860351563\\
1.49000000953674	992.988342285156\\
1.49500000476837	994.291625976563\\
1.5	995.902587890625\\
1.50499999523163	997.788146972656\\
1.50999999046326	999.869323730469\\
1.51499998569489	1002.16186523438\\
1.51999998092651	1004.63464355469\\
1.52499997615814	1007.26324462891\\
1.52999997138977	1010.03356933594\\
1.5349999666214	1012.96185302734\\
1.53999996185303	1016.02587890625\\
1.54499995708466	1019.19030761719\\
1.54999995231628	1022.41442871094\\
1.55499994754791	1025.68359375\\
1.55999994277954	1028.97814941406\\
1.56500005722046	1032.27331542969\\
1.57000005245209	1024.22045898438\\
1.57500004768372	1027.57092285156\\
1.58000004291534	1031.93737792969\\
1.58500003814697	1036.58154296875\\
1.5900000333786	1041.11889648438\\
1.59500002861023	1045.33557128906\\
1.60000002384186	1049.12194824219\\
1.60500001907349	1052.42443847656\\
1.61000001430511	1055.22058105469\\
1.61500000953674	1057.62829589844\\
1.62000000476837	1059.74792480469\\
1.625	1061.68774414063\\
1.62999999523163	1063.53735351563\\
1.63499999046326	1065.35571289063\\
1.63999998569489	1067.18566894531\\
1.64499998092651	1069.03759765625\\
1.64999997615814	1070.90270996094\\
1.65499997138977	1072.73400878906\\
1.6599999666214	1074.50109863281\\
1.66499996185303	1076.14660644531\\
1.66999995708466	1077.59802246094\\
1.67499995231628	1078.86145019531\\
1.67999994754791	1079.92590332031\\
1.68499994277954	1080.77575683594\\
1.69000005722046	1081.28723144531\\
1.69500005245209	1081.64196777344\\
1.70000004768372	1081.84606933594\\
1.70500004291534	1081.88781738281\\
1.71000003814697	1081.84545898438\\
1.7150000333786	1081.70544433594\\
1.72000002861023	1081.48059082031\\
1.72500002384186	1081.13354492188\\
1.73000001907349	1080.68408203125\\
1.73500001430511	1080.14538574219\\
1.74000000953674	1079.51513671875\\
1.74500000476837	1078.80090332031\\
1.75	1078.02185058594\\
1.75499999523163	1077.16760253906\\
1.75999999046326	1076.24877929688\\
1.76499998569489	1075.26354980469\\
1.76999998092651	1074.21166992188\\
1.77499997615814	1073.0966796875\\
1.77999997138977	1071.91870117188\\
1.7849999666214	1070.68127441406\\
1.78999996185303	1069.38586425781\\
1.79499995708466	1068.04077148438\\
1.79999995231628	1066.66540527344\\
1.80499994754791	1065.26245117188\\
1.80999994277954	1063.84875488281\\
1.81500005722046	1062.43432617188\\
1.82000005245209	1061.029296875\\
1.82500004768372	1059.64306640625\\
1.83000004291534	1058.28674316406\\
1.83500003814697	1056.97631835938\\
1.8400000333786	1055.70642089844\\
1.84500002861023	1054.47631835938\\
1.85000002384186	1053.28942871094\\
1.85500001907349	1052.14404296875\\
1.86000001430511	1051.04272460938\\
1.86500000953674	1049.99194335938\\
1.87000000476837	1048.98791503906\\
1.875	1048.0458984375\\
1.87999999523163	1047.177734375\\
1.88499999046326	1046.37377929688\\
1.88999998569489	1045.63439941406\\
1.89499998092651	1044.96643066406\\
1.89999997615814	1044.36596679688\\
1.90499997138977	1043.8330078125\\
1.9099999666214	1043.37756347656\\
1.91499996185303	1042.99597167969\\
1.91999995708466	1042.6875\\
1.92499995231628	1042.46484375\\
1.92999994754791	1042.32702636719\\
1.93499994277954	1042.27099609375\\
1.94000005722046	1042.28637695313\\
1.94500005245209	1042.36987304688\\
1.95000004768372	1042.52600097656\\
1.95500004291534	1042.75634765625\\
1.96000003814697	1043.0625\\
1.9650000333786	1043.4443359375\\
1.97000002861023	1043.89868164063\\
1.97500002384186	1044.41467285156\\
1.98000001907349	1044.99462890625\\
1.98500001430511	1045.63452148438\\
1.99000000953674	1046.3212890625\\
1.99500000476837	1047.05419921875\\
2	1047.82836914063\\
2.00500011444092	1048.63427734375\\
2.00999999046326	1049.46765136719\\
2.01500010490417	1050.32287597656\\
2.01999998092651	1051.25378417969\\
2.02500009536743	1052.25354003906\\
2.02999997138977	1053.32531738281\\
2.03500008583069	1054.46899414063\\
2.03999996185303	1055.69311523438\\
2.04500007629395	1056.99829101563\\
2.04999995231628	1058.3837890625\\
2.0550000667572	1059.84411621094\\
2.05999994277954	1061.33984375\\
2.06500005722046	1062.8544921875\\
2.0699999332428	1064.29870605469\\
2.07500004768372	1065.63635253906\\
2.07999992370605	1066.76257324219\\
2.08500003814697	1067.6162109375\\
2.08999991416931	1068.08642578125\\
2.09500002861023	1068.09655761719\\
2.09999990463257	1067.56115722656\\
2.10500001907349	1066.408203125\\
2.10999989509583	1064.75561523438\\
2.11500000953674	1062.62475585938\\
2.11999988555908	1059.67736816406\\
2.125	1055.79809570313\\
2.13000011444092	1051.12768554688\\
2.13499999046326	1045.61206054688\\
2.14000010490417	1039.31518554688\\
2.14499998092651	1032.35852050781\\
2.15000009536743	1025.02624511719\\
2.15499997138977	1017.28656005859\\
2.16000008583069	1008.97802734375\\
2.16499996185303	1000.38073730469\\
2.17000007629395	991.520385742188\\
2.17499995231628	982.345825195313\\
2.1800000667572	972.874816894531\\
2.18499994277954	963.105346679688\\
2.19000005722046	953.071899414063\\
2.1949999332428	942.826110839844\\
2.20000004768372	932.4365234375\\
2.20499992370605	921.996276855469\\
2.21000003814697	911.629821777344\\
2.21499991416931	901.491333007813\\
2.22000002861023	891.765625\\
2.22499990463257	882.599243164063\\
2.23000001907349	874.098876953125\\
2.23499989509583	866.323364257813\\
2.24000000953674	859.391662597656\\
2.24499988555908	853.34033203125\\
2.25	848.174499511719\\
2.25500011444092	843.919921875\\
2.25999999046326	840.551147460938\\
2.26500010490417	838.035400390625\\
2.26999998092651	836.423950195313\\
2.27500009536743	835.7587890625\\
2.27999997138977	836.141662597656\\
2.28500008583069	837.817565917969\\
2.28999996185303	840.807189941406\\
2.29500007629395	844.925231933594\\
2.29999995231628	849.341552734375\\
2.3050000667572	854.932495117188\\
2.30999994277954	861.46142578125\\
2.31500005722046	868.740051269531\\
2.3199999332428	876.765014648438\\
2.32500004768372	885.648376464844\\
2.32999992370605	895.385498046875\\
2.33500003814697	905.967407226563\\
2.33999991416931	918.796020507813\\
2.34500002861023	933.480834960938\\
2.34999990463257	947.479797363281\\
2.35500001907349	962.243530273438\\
2.35999989509583	977.635131835938\\
2.36500000953674	993.548645019531\\
2.36999988555908	1010.240234375\\
2.375	1027.57458496094\\
2.38000011444092	1045.59020996094\\
2.38499999046326	1064.40368652344\\
2.39000010490417	1083.9033203125\\
2.39499998092651	1104.10546875\\
2.40000009536743	1125.00427246094\\
2.40499997138977	1146.50329589844\\
2.41000008583069	1168.51538085938\\
2.41499996185303	1190.90661621094\\
2.42000007629395	1213.49743652344\\
2.42499995231628	1236.11437988281\\
2.4300000667572	1258.50659179688\\
2.43499994277954	1280.45861816406\\
2.44000005722046	1301.72631835938\\
2.4449999332428	1321.97277832031\\
2.45000004768372	1341.09020996094\\
2.45499992370605	1358.8095703125\\
2.46000003814697	1374.98742675781\\
2.46499991416931	1389.51354980469\\
2.47000002861023	1402.29553222656\\
2.47499990463257	1413.28405761719\\
2.48000001907349	1422.46899414063\\
2.48499989509583	1429.84887695313\\
2.49000000953674	1435.49633789063\\
2.49499988555908	1439.48315429688\\
2.5	1441.86499023438\\
2.50500011444092	1442.65466308594\\
2.50999999046326	1442.17236328125\\
2.51500010490417	1440.70581054688\\
2.51999998092651	1437.53283691406\\
2.52500009536743	1432.43896484375\\
2.52999997138977	1425.61962890625\\
2.53500008583069	1416.40197753906\\
2.53999996185303	1405.57421875\\
2.54500007629395	1392.89782714844\\
2.54999995231628	1378.4482421875\\
2.5550000667572	1362.36193847656\\
2.55999994277954	1344.71655273438\\
2.56500005722046	1325.74462890625\\
2.5699999332428	1305.54516601563\\
2.57500004768372	1284.29333496094\\
2.57999992370605	1262.18493652344\\
2.58500003814697	1239.43994140625\\
2.58999991416931	1216.25927734375\\
2.59500002861023	1192.62048339844\\
2.59999990463257	1168.64184570313\\
2.60500001907349	1144.39038085938\\
2.60999989509583	1119.83020019531\\
2.61500000953674	1094.9814453125\\
2.61999988555908	1069.89038085938\\
2.625	1044.62609863281\\
2.63000011444092	1019.37463378906\\
2.63499999046326	994.228820800781\\
2.64000010490417	969.17138671875\\
2.64499998092651	944.469970703125\\
2.65000009536743	920.362182617188\\
2.65499997138977	896.796691894531\\
2.66000008583069	873.76220703125\\
2.66499996185303	851.363952636719\\
2.67000007629395	829.406555175781\\
2.67499995231628	807.8388671875\\
2.6800000667572	786.861938476563\\
2.68499994277954	766.275390625\\
2.69000005722046	746.192321777344\\
2.6949999332428	726.553588867188\\
2.70000004768372	707.403198242188\\
2.70499992370605	688.775573730469\\
2.71000003814697	670.74462890625\\
2.71499991416931	653.391723632813\\
2.72000002861023	636.914794921875\\
2.72499990463257	621.422607421875\\
2.73000001907349	607.013122558594\\
2.73499989509583	593.808959960938\\
2.74000000953674	581.963500976563\\
2.74499988555908	571.737121582031\\
2.75	563.006958007813\\
2.75500011444092	555.803100585938\\
2.75999999046326	550.397277832031\\
2.76500010490417	547.299438476563\\
2.76999998092651	546.638793945313\\
2.77500009536743	548.49267578125\\
2.77999997138977	552.589477539063\\
2.78500008583069	559.092590332031\\
2.78999996185303	568.1513671875\\
2.79500007629395	579.890869140625\\
2.79999995231628	594.4130859375\\
2.8050000667572	611.800964355469\\
2.80999994277954	632.122863769531\\
2.81500005722046	655.379821777344\\
2.8199999332428	681.509094238281\\
2.82500004768372	710.400939941406\\
2.82999992370605	741.913269042969\\
2.83500003814697	775.884643554688\\
2.83999991416931	812.127014160156\\
2.84500002861023	850.476989746094\\
2.84999990463257	890.821105957031\\
2.85500001907349	933.011291503906\\
2.85999989509583	976.817749023438\\
2.86500000953674	1022.24786376953\\
2.86999988555908	1069.30798339844\\
2.875	1117.76818847656\\
2.88000011444092	1167.97546386719\\
2.88499999046326	1219.81384277344\\
2.89000010490417	1273.3388671875\\
2.89499998092651	1327.16870117188\\
2.90000009536743	1379.53405761719\\
2.90499997138977	1429.63317871094\\
2.91000008583069	1476.26489257813\\
2.91499996185303	1518.24853515625\\
2.92000007629395	1555.35327148438\\
2.92499995231628	1587.53747558594\\
2.9300000667572	1615.04187011719\\
2.93499994277954	1638.34106445313\\
2.94000005722046	1657.8486328125\\
2.9449999332428	1673.97009277344\\
2.95000004768372	1686.72192382813\\
2.95499992370605	1696.52880859375\\
2.96000003814697	1703.75048828125\\
2.96499991416931	1707.19116210938\\
2.97000002861023	1707.06237792969\\
2.97499990463257	1702.20751953125\\
2.98000001907349	1691.58837890625\\
2.98499989509583	1679.50317382813\\
2.99000000953674	1661.75561523438\\
2.99499988555908	1638.37646484375\\
3	1609.76391601563\\
3.00500011444092	1576.27905273438\\
3.00999999046326	1538.40734863281\\
3.01500010490417	1496.73901367188\\
3.01999998092651	1451.96765136719\\
3.02500009536743	1404.81665039063\\
3.02999997138977	1355.92309570313\\
3.03500008583069	1305.85681152344\\
3.03999996185303	1254.93908691406\\
3.04500007629395	1203.33923339844\\
3.04999995231628	1151.05993652344\\
3.0550000667572	1097.99169921875\\
3.05999994277954	1044.07153320313\\
3.06500005722046	989.247131347656\\
3.0699999332428	933.641784667969\\
3.07500004768372	877.682312011719\\
3.07999992370605	821.753601074219\\
3.08500003814697	766.457336425781\\
3.08999991416931	712.467163085938\\
3.09500002861023	660.497924804688\\
3.09999990463257	611.024841308594\\
3.10500001907349	564.431091308594\\
3.10999989509583	521.263854980469\\
3.11500000953674	482.264465332031\\
3.11999988555908	447.480407714844\\
3.125	417.194610595703\\
3.13000011444092	392.199554443359\\
3.13499999046326	373.150939941406\\
3.14000010490417	360.297821044922\\
3.14499998092651	354.065185546875\\
3.15000009536743	356.539733886719\\
3.15499997138977	368.509887695313\\
3.16000008583069	387.435119628906\\
3.16499996185303	412.961547851563\\
3.17000007629395	444.52001953125\\
3.17499995231628	481.472717285156\\
3.1800000667572	523.169677734375\\
3.18499994277954	568.849792480469\\
3.19000005722046	617.700927734375\\
3.1949999332428	668.632202148438\\
3.20000004768372	720.811645507813\\
3.20499992370605	773.594421386719\\
3.21000003814697	826.24169921875\\
3.21499991416931	878.692687988281\\
3.22000002861023	930.677368164063\\
3.22499990463257	982.299377441406\\
3.23000001907349	1033.49853515625\\
3.23499989509583	1084.05395507813\\
3.24000000953674	1133.82861328125\\
3.24499988555908	1181.68200683594\\
3.25	1228.037109375\\
3.25500011444092	1272.28979492188\\
3.25999999046326	1314.61926269531\\
3.26500010490417	1355.03002929688\\
3.26999998092651	1393.693359375\\
3.27500009536743	1430.59558105469\\
3.27999997138977	1465.75720214844\\
3.28500008583069	1498.89318847656\\
3.28999996185303	1529.49536132813\\
3.29500007629395	1556.96911621094\\
3.29999995231628	1580.71984863281\\
3.3050000667572	1600.17822265625\\
3.30999994277954	1614.84521484375\\
3.31500005722046	1624.52160644531\\
3.3199999332428	1631.31481933594\\
3.32500004768372	1633.71252441406\\
3.32999992370605	1630.31408691406\\
3.33500003814697	1621.31396484375\\
3.33999991416931	1607.298828125\\
3.34500002861023	1588.84069824219\\
3.34999990463257	1566.39086914063\\
3.35500001907349	1540.31909179688\\
3.35999989509583	1510.81652832031\\
3.36500000953674	1477.93115234375\\
3.36999988555908	1441.53430175781\\
3.375	1401.60498046875\\
3.38000011444092	1358.05151367188\\
3.38499999046326	1310.83959960938\\
3.39000010490417	1260.07202148438\\
3.39499998092651	1205.9814453125\\
3.40000009536743	1148.88049316406\\
3.40499997138977	1089.46850585938\\
3.41000008583069	1028.2900390625\\
3.41499996185303	966.00146484375\\
3.42000007629395	903.354553222656\\
3.42499995231628	840.926452636719\\
3.4300000667572	779.432922363281\\
3.43499994277954	719.366821289063\\
3.44000005722046	661.266235351563\\
3.4449999332428	605.891723632813\\
3.45000004768372	554.060119628906\\
3.45499992370605	506.659729003906\\
3.46000003814697	464.779449462891\\
3.46499991416931	429.277069091797\\
3.47000002861023	400.453430175781\\
3.47499990463257	378.839965820313\\
3.48000001907349	364.918518066406\\
3.48499989509583	359.536865234375\\
3.49000000953674	366.757781982422\\
3.49499988555908	381.72509765625\\
3.5	404.279541015625\\
3.50500011444092	434.211456298828\\
3.50999999046326	471.186981201172\\
3.51500010490417	515.648376464844\\
3.51999998092651	567.861877441406\\
3.52500009536743	628.648193359375\\
3.52999997138977	698.199035644531\\
3.53500008583069	777.4716796875\\
3.53999996185303	868.736938476563\\
3.54500007629395	969.425659179688\\
3.54999995231628	1073.322265625\\
3.5550000667572	1169.3525390625\\
3.55999994277954	1250.8896484375\\
3.56500005722046	1313.35729980469\\
3.5699999332428	1361.15832519531\\
3.57500004768372	1395.04736328125\\
3.57999992370605	1424.3037109375\\
3.58500003814697	1447.35400390625\\
3.58999991416931	1466.39208984375\\
3.59500002861023	1488.39636230469\\
3.59999990463257	1519.67443847656\\
3.60500001907349	1554.89001464844\\
3.60999989509583	1591.57592773438\\
3.61500000953674	1626.67407226563\\
3.61999988555908	1655.84606933594\\
3.625	1675.62573242188\\
3.63000011444092	1683.57690429688\\
3.63499999046326	1681.94128417969\\
3.64000010490417	1672.30505371094\\
3.64499998092651	1649.05297851563\\
3.65000009536743	1614.07067871094\\
3.65499997138977	1569.53381347656\\
3.66000008583069	1518.83142089844\\
3.66499996185303	1465.50402832031\\
3.67000007629395	1412.50537109375\\
3.67499995231628	1360.91796875\\
3.6800000667572	1310.80407714844\\
3.68499994277954	1261.18994140625\\
3.69000005722046	1210.47692871094\\
3.6949999332428	1156.77856445313\\
3.70000004768372	1098.42358398438\\
3.70499992370605	1034.94494628906\\
3.71000003814697	965.716796875\\
3.71499991416931	892.542724609375\\
3.72000002861023	816.571716308594\\
3.72499990463257	740.018493652344\\
3.73000001907349	665.057250976563\\
3.73499989509583	594.183959960938\\
3.74000000953674	528.536560058594\\
3.74499988555908	469.7353515625\\
3.75	420.641510009766\\
3.75500011444092	380.961059570313\\
3.75999999046326	352.182159423828\\
3.76500010490417	334.154571533203\\
3.76999998092651	327.182250976563\\
3.77500009536743	335.664733886719\\
3.77999997138977	356.84228515625\\
3.78500008583069	388.046813964844\\
3.78999996185303	427.925140380859\\
3.79500007629395	476.881988525391\\
3.79999995231628	533.878479003906\\
3.8050000667572	598.156127929688\\
3.80999994277954	669.067687988281\\
3.81500005722046	746.486022949219\\
3.8199999332428	830.262084960938\\
3.82500004768372	920.722229003906\\
3.82999992370605	1018.31744384766\\
3.83500003814697	1119.12512207031\\
3.83999991416931	1213.54235839844\\
3.84500002861023	1294.826171875\\
3.84999990463257	1356.16809082031\\
3.85500001907349	1397.00866699219\\
3.85999989509583	1427.89526367188\\
3.86500000953674	1449.94909667969\\
3.86999988555908	1462.10778808594\\
3.875	1471.87329101563\\
3.88000011444092	1486.37817382813\\
3.88499999046326	1518.29467773438\\
3.89000010490417	1562.89318847656\\
3.89499998092651	1612.41296386719\\
3.90000009536743	1662.40368652344\\
3.90499997138977	1705.87744140625\\
3.91000008583069	1737.63977050781\\
3.91499996185303	1754.07775878906\\
3.92000007629395	1756.89221191406\\
3.92499995231628	1750.53759765625\\
3.9300000667572	1727.50378417969\\
3.93499994277954	1689.56701660156\\
3.94000005722046	1640.10681152344\\
3.9449999332428	1583.603515625\\
3.95000004768372	1523.95446777344\\
3.95499992370605	1465.00537109375\\
3.96000003814697	1407.94006347656\\
3.96499991416931	1352.54296875\\
3.97000002861023	1297.19702148438\\
3.97499990463257	1239.84020996094\\
3.98000001907349	1177.61657714844\\
3.98499989509583	1108.61572265625\\
3.99000000953674	1032.00817871094\\
3.99499988555908	948.249938964844\\
4	858.366271972656\\
4.00500011444092	763.957275390625\\
4.01000022888184	668.714172363281\\
4.0149998664856	576.072631835938\\
4.01999998092651	487.542572021484\\
4.02500009536743	405.929870605469\\
4.03000020980835	333.629577636719\\
4.03499984741211	272.285888671875\\
4.03999996185303	222.915954589844\\
4.04500007629395	185.552841186523\\
4.05000019073486	160.414047241211\\
4.05499982833862	146.890625\\
4.05999994277954	145.334106445313\\
4.06500005722046	160.456848144531\\
4.07000017166138	186.130783081055\\
4.07499980926514	222.76887512207\\
4.07999992370605	270.867462158203\\
4.08500003814697	332.312347412109\\
4.09000015258789	410.413757324219\\
4.09499979019165	510.737579345703\\
4.09999990463257	643.526123046875\\
4.10500001907349	828.257202148438\\
4.1100001335144	1057.02758789063\\
4.11499977111816	1276.857421875\\
4.11999988555908	1446.53869628906\\
4.125	1546.84790039063\\
4.13000011444092	1599.06591796875\\
4.13500022888184	1627.34448242188\\
4.1399998664856	1607.11584472656\\
4.14499998092651	1556.96801757813\\
4.15000009536743	1502.912109375\\
4.15500020980835	1468.55908203125\\
4.15999984741211	1474.11767578125\\
4.16499996185303	1541.71569824219\\
4.17000007629395	1633.029296875\\
4.17500019073486	1729.47802734375\\
4.17999982833862	1813.98193359375\\
4.18499994277954	1873.23034667969\\
4.19000005722046	1898.62292480469\\
4.19500017166138	1889.62976074219\\
4.19999980926514	1867.34716796875\\
4.20499992370605	1814.8232421875\\
4.21000003814697	1735.19213867188\\
4.21500015258789	1638.10168457031\\
4.21999979019165	1535.83544921875\\
4.22499990463257	1438.58178710938\\
4.23000001907349	1353.58581542969\\
4.2350001335144	1283.33605957031\\
4.23999977111816	1223.828125\\
4.24499988555908	1169.34313964844\\
4.25	1113.6318359375\\
4.25500011444092	1050.01892089844\\
4.26000022888184	974.773864746094\\
4.2649998664856	885.689025878906\\
4.26999998092651	784.263244628906\\
4.27500009536743	674.471496582031\\
4.28000020980835	562.258911132813\\
4.28499984741211	451.798217773438\\
4.28999996185303	349.127746582031\\
4.29500007629395	258.766754150391\\
4.30000019073486	185.162185668945\\
4.30499982833862	131.81315612793\\
4.30999994277954	98.5380020141602\\
4.31500005722046	84.663215637207\\
4.32000017166138	90.2949600219727\\
4.32499980926514	113.04150390625\\
4.32999992370605	166.440246582031\\
4.33500003814697	228.797882080078\\
4.34000015258789	298.478546142578\\
4.34499979019165	379.844268798828\\
4.34999990463257	475.007293701172\\
4.35500001907349	589.048278808594\\
4.3600001335144	729.229309082031\\
4.36499977111816	914.205810546875\\
4.36999988555908	1148.75073242188\\
4.375	1394.05480957031\\
4.38000011444092	1592.3505859375\\
4.38500022888184	1708.08239746094\\
4.3899998664856	1737.1689453125\\
4.39499998092651	1752.93664550781\\
4.40000009536743	1710.23608398438\\
4.40500020980835	1618.92785644531\\
4.40999984741211	1514.85400390625\\
4.41499996185303	1435.3212890625\\
4.42000007629395	1406.95166015625\\
4.42500019073486	1476.23132324219\\
4.42999982833862	1596.15832519531\\
4.43499994277954	1732.02709960938\\
4.44000005722046	1857.05151367188\\
4.44500017166138	1950.03564453125\\
4.44999980926514	1997.36010742188\\
4.45499992370605	1993.73657226563\\
4.46000003814697	1971.97216796875\\
4.46500015258789	1911.31774902344\\
4.46999979019165	1813.11791992188\\
4.47499990463257	1691.59643554688\\
4.48000001907349	1563.33544921875\\
4.4850001335144	1443.62670898438\\
4.48999977111816	1342.31652832031\\
4.49499988555908	1263.73205566406\\
4.5	1201.59948730469\\
4.50500011444092	1147.51208496094\\
4.51000022888184	1091.8642578125\\
4.5149998664856	1025.78479003906\\
4.51999998092651	943.663024902344\\
4.52500009536743	842.367004394531\\
4.53000020980835	724.800720214844\\
4.53499984741211	595.467529296875\\
4.53999996185303	463.204223632813\\
4.54500007629395	332.930389404297\\
4.55000019073486	211.756439208984\\
4.55499982833862	105.878433227539\\
4.55999994277954	60.9680862426758\\
4.56500005722046	62.1962509155273\\
4.57000017166138	69.8994903564453\\
4.57499980926514	80.4328460693359\\
4.57999992370605	91.4052581787109\\
4.58500003814697	100.681655883789\\
4.59000015258789	107.438613891602\\
4.59499979019165	112.047401428223\\
4.59999990463257	114.459426879883\\
4.60500001907349	112.05241394043\\
4.6100001335144	102.69221496582\\
4.61499977111816	83.4552764892578\\
4.61999988555908	317.652496337891\\
4.625	824.225402832031\\
4.63000011444092	1376.61254882813\\
4.63500022888184	1815.64025878906\\
4.6399998664856	2099.58569335938\\
4.64499998092651	2214.66455078125\\
4.65000009536743	2224.83569335938\\
4.65500020980835	2122.86962890625\\
4.65999984741211	1888.46862792969\\
4.66499996185303	1663.33251953125\\
4.67000007629395	1407.01892089844\\
4.67500019073486	1202.58349609375\\
4.67999982833862	1101.57312011719\\
4.68499994277954	1182.36315917969\\
4.69000005722046	1382.33374023438\\
4.69500017166138	1615.03198242188\\
4.69999980926514	1828.82763671875\\
4.70499992370605	1984.72351074219\\
4.71000003814697	2057.79809570313\\
4.71500015258789	2040.62231445313\\
4.71999979019165	1989.49340820313\\
4.72499990463257	1884.43103027344\\
4.73000001907349	1719.84289550781\\
4.7350001335144	1523.029296875\\
4.73999977111816	1325.07800292969\\
4.74499988555908	1152.40209960938\\
4.75	1019.55267333984\\
4.75500011444092	934.620483398438\\
4.76000022888184	884.010559082031\\
4.7649998664856	855.194763183594\\
4.76999998092651	815.257690429688\\
4.77500009536743	763.955017089844\\
4.78000020980835	685.310546875\\
4.78499984741211	584.172058105469\\
4.78999996185303	464.312255859375\\
4.79500007629395	322.627227783203\\
4.80000019073486	186.124694824219\\
4.80499982833862	57.6118812561035\\
4.80999994277954	54.7035064697266\\
4.81500005722046	76.4421463012695\\
4.82000017166138	101.376220703125\\
4.82499980926514	125.766090393066\\
4.82999992370605	147.659286499023\\
4.83500003814697	163.911499023438\\
4.84000015258789	174.496673583984\\
4.84499979019165	175.949462890625\\
4.84999990463257	163.450912475586\\
4.85500001907349	132.50520324707\\
4.8600001335144	80.9447555541992\\
4.86499977111816	15.9731845855713\\
4.86999988555908	452.451446533203\\
4.875	1012.03656005859\\
4.88000011444092	1558.61865234375\\
4.88500022888184	1974.61828613281\\
4.8899998664856	2223.53344726563\\
4.89499998092651	2299.6494140625\\
4.90000009536743	2292.50952148438\\
4.90500020980835	2142.85620117188\\
4.90999984741211	1859.16162109375\\
4.91499996185303	1477.40625\\
4.92000007629395	1125.15319824219\\
4.92500019073486	833.9677734375\\
4.92999982833862	647.272521972656\\
4.93499994277954	646.439331054688\\
4.94000005722046	884.747009277344\\
4.94500017166138	1214.54809570313\\
4.94999980926514	1562.35864257813\\
4.95499992370605	1869.26086425781\\
4.96000003814697	2094.20727539063\\
4.96500015258789	2214.20727539063\\
4.96999979019165	2225.66577148438\\
4.97499990463257	2190.15991210938\\
4.98000001907349	2097.2392578125\\
4.9850001335144	1936.45043945313\\
4.98999977111816	1741.53039550781\\
4.99499988555908	1546.3837890625\\
5	1379.77001953125\\
5.00500011444092	1257.70056152344\\
5.01000022888184	1185.26843261719\\
5.0149998664856	1158.78894042969\\
5.01999998092651	1169.669921875\\
5.02500009536743	1173.19836425781\\
5.03000020980835	1157.72448730469\\
5.03499984741211	1116.20092773438\\
5.03999996185303	1061.37841796875\\
5.04500007629395	982.32080078125\\
5.05000019073486	878.241271972656\\
5.05499982833862	754.822082519531\\
5.05999994277954	617.544616699219\\
5.06500005722046	473.185638427734\\
5.07000017166138	326.963775634766\\
5.07499980926514	184.053482055664\\
5.07999992370605	48.9029159545898\\
5.08500003814697	15.6395597457886\\
5.09000015258789	4.54384517669678\\
5.09499979019165	-4.47182607650757\\
5.09999990463257	-13.8551483154297\\
5.10500001907349	-18.9757709503174\\
5.1100001335144	-19.4506359100342\\
5.11499977111816	-18.3916110992432\\
5.11999988555908	-16.7593097686768\\
5.125	-15.0208930969238\\
5.13000011444092	-13.3896179199219\\
5.13500022888184	411.320648193359\\
5.1399998664856	769.033935546875\\
5.14499998092651	998.2060546875\\
5.15000009536743	1108.86877441406\\
5.15500020980835	1120.845703125\\
5.15999984741211	1146.70788574219\\
5.16499996185303	1123.23596191406\\
5.17000007629395	1074.43786621094\\
5.17500019073486	1034.40405273438\\
5.17999982833862	1032.32800292969\\
5.18499994277954	1108.23742675781\\
5.19000005722046	1214.40368652344\\
5.19500017166138	1335.75622558594\\
5.19999980926514	1453.54626464844\\
5.20499992370605	1554.11987304688\\
5.21000003814697	1628.86413574219\\
5.21500015258789	1673.60876464844\\
5.21999979019165	1688.95056152344\\
5.22499990463257	1679.4560546875\\
5.23000001907349	1662.00842285156\\
5.2350001335144	1631.939453125\\
5.23999977111816	1588.14611816406\\
5.24499988555908	1539.17138671875\\
5.25	1491.19421386719\\
5.25500011444092	1448.35095214844\\
5.26000022888184	1415.07946777344\\
5.2649998664856	1392.26489257813\\
5.26999998092651	1383.75512695313\\
5.27500009536743	1380.16491699219\\
5.28000020980835	1376.40966796875\\
5.28499984741211	1369.32043457031\\
5.28999996185303	1357.02160644531\\
5.29500007629395	1338.12780761719\\
5.30000019073486	1315.97717285156\\
5.30499982833862	1288.91906738281\\
5.30999994277954	1253.04736328125\\
5.31500005722046	1208.27868652344\\
5.32000017166138	1157.5908203125\\
5.32499980926514	1101.43896484375\\
5.32999992370605	1043.58703613281\\
5.33500003814697	986.912353515625\\
5.34000015258789	933.768920898438\\
5.34499979019165	886.848693847656\\
5.34999990463257	846.240173339844\\
5.35500001907349	810.841918945313\\
5.3600001335144	778.560729980469\\
5.36499977111816	746.47314453125\\
5.36999988555908	711.255920410156\\
5.375	670.278381347656\\
5.38000011444092	621.594909667969\\
5.38500022888184	564.853942871094\\
5.3899998664856	501.482330322266\\
5.39499998092651	434.717163085938\\
5.40000009536743	366.306793212891\\
5.40500020980835	299.507263183594\\
5.40999984741211	238.281112670898\\
5.41499996185303	183.712417602539\\
5.42000007629395	135.60221862793\\
5.42500019073486	92.6330413818359\\
5.42999982833862	54.3288688659668\\
5.43499994277954	26.3126373291016\\
5.44000005722046	0.576142907142639\\
5.44500017166138	-10.6391553878784\\
5.44999980926514	-11.9291973114014\\
5.45499992370605	-11.7547941207886\\
5.46000003814697	-10.9870452880859\\
5.46500015258789	-10.0311164855957\\
5.46999979019165	-9.10385131835938\\
5.47499990463257	-8.25202465057373\\
5.48000001907349	-7.46887922286987\\
5.4850001335144	-6.76615810394287\\
5.48999977111816	-6.14298391342163\\
5.49499988555908	-5.60813760757446\\
5.5	-5.1096978187561\\
5.50500011444092	-4.6788444519043\\
5.51000022888184	-4.31125783920288\\
5.5149998664856	-3.97868132591248\\
5.51999998092651	28.155200958252\\
5.52500009536743	99.8635711669922\\
5.53000020980835	141.901412963867\\
5.53499984741211	174.98811340332\\
5.53999996185303	198.993591308594\\
5.54500007629395	214.349563598633\\
5.55000019073486	221.774719238281\\
5.55499982833862	222.609603881836\\
5.55999994277954	219.656448364258\\
5.56500005722046	212.726272583008\\
5.57000017166138	200.816040039063\\
5.57499980926514	184.919540405273\\
5.57999992370605	165.839904785156\\
5.58500003814697	144.231964111328\\
5.59000015258789	121.083763122559\\
5.59499979019165	97.7242279052734\\
5.59999990463257	75.1830215454102\\
5.60500001907349	54.5372848510742\\
5.6100001335144	36.6168022155762\\
5.61499977111816	21.7925643920898\\
5.61999988555908	10.2144956588745\\
5.625	1.74973106384277\\
5.63000011444092	-3.99846220016479\\
5.63500022888184	-7.68318319320679\\
5.6399998664856	-9.02161026000977\\
5.64499998092651	-9.09969902038574\\
5.65000009536743	-9.07180500030518\\
5.65500020980835	-8.96457099914551\\
5.65999984741211	-8.77997493743896\\
5.66499996185303	-8.58776187896729\\
5.67000007629395	-7.21260404586792\\
5.67500019073486	54.9024772644043\\
5.67999982833862	81.2165069580078\\
5.68499994277954	99.0744400024414\\
5.69000005722046	218.492065429688\\
5.69500017166138	299.077087402344\\
5.69999980926514	361.345855712891\\
5.70499992370605	399.495056152344\\
5.71000003814697	411.451873779297\\
5.71500015258789	405.884033203125\\
5.71999979019165	392.567932128906\\
5.72499990463257	360.577087402344\\
5.73000001907349	317.033050537109\\
5.7350001335144	270.055053710938\\
5.73999977111816	226.688827514648\\
5.74499988555908	192.442459106445\\
5.75	169.629196166992\\
5.75500011444092	157.868591308594\\
5.76000022888184	156.868759155273\\
5.7649998664856	161.205688476563\\
5.76999998092651	166.629058837891\\
5.77500009536743	172.400802612305\\
5.78000020980835	178.548721313477\\
5.78499984741211	185.247650146484\\
5.78999996185303	193.118911743164\\
5.79500007629395	202.226974487305\\
5.80000019073486	212.943588256836\\
5.80499982833862	225.212249755859\\
5.80999994277954	238.833786010742\\
5.81500005722046	253.44189453125\\
5.82000017166138	268.727508544922\\
5.82499980926514	284.561157226563\\
5.82999992370605	300.413146972656\\
5.83500003814697	315.723999023438\\
5.84000015258789	329.899108886719\\
5.84499979019165	341.699340820313\\
5.84999990463257	349.23681640625\\
5.85500001907349	351.0107421875\\
5.8600001335144	346.601501464844\\
5.86499977111816	332.733459472656\\
5.86999988555908	303.34326171875\\
5.875	255.465805053711\\
5.88000011444092	187.495803833008\\
5.88500022888184	103.395927429199\\
5.8899998664856	16.5651798248291\\
5.89499998092651	5.07563877105713\\
5.90000009536743	84.6855239868164\\
5.90500020980835	296.027770996094\\
5.90999984741211	695.457763671875\\
5.91499996185303	1243.24182128906\\
5.92000007629395	1864.1376953125\\
5.92500019073486	2494.55053710938\\
5.92999982833862	3086.91821289063\\
5.93499994277954	3614.81884765625\\
5.94000005722046	4055.095703125\\
5.94500017166138	4397.142578125\\
5.94999980926514	4633.1416015625\\
5.95499992370605	4751.38232421875\\
5.96000003814697	4804.201171875\\
5.96500015258789	4734.146484375\\
5.96999979019165	4502.06494140625\\
5.97499990463257	4093.48974609375\\
5.98000001907349	3512.97119140625\\
5.9850001335144	2872.1767578125\\
5.98999977111816	2280.67163085938\\
5.99499988555908	1871.65625\\
6	1746.19177246094\\
6.00500011444092	2103.05444335938\\
6.01000022888184	2667.900390625\\
6.0149998664856	3326.62939453125\\
6.01999998092651	3983.12548828125\\
6.02500009536743	4553.76318359375\\
6.03000020980835	4976.216796875\\
6.03499984741211	5210.09033203125\\
6.03999996185303	5239.5439453125\\
6.04500007629395	5167.07958984375\\
6.05000019073486	4932.46923828125\\
6.05499982833862	4537.25146484375\\
6.05999994277954	4049.65112304688\\
6.06500005722046	3550.24267578125\\
6.07000017166138	3110.07788085938\\
6.07499980926514	2777.861328125\\
6.07999992370605	2570.330078125\\
6.08500003814697	2480.70678710938\\
6.09000015258789	2523.87329101563\\
6.09499979019165	2615.32421875\\
6.09999990463257	2696.06225585938\\
6.10500001907349	2734.67944335938\\
6.1100001335144	2712.52124023438\\
6.11499977111816	2654.97875976563\\
6.11999988555908	2531.51391601563\\
6.125	2335.40087890625\\
6.13000011444092	2079.94702148438\\
6.13500022888184	1788.90576171875\\
6.1399998664856	1488.27160644531\\
6.14499998092651	1203.06225585938\\
6.15000009536743	953.265563964844\\
6.15500020980835	751.135070800781\\
6.15999984741211	600.183776855469\\
6.16499996185303	496.875427246094\\
6.17000007629395	430.214202880859\\
6.17500019073486	387.124176025391\\
6.17999982833862	353.243103027344\\
6.18499994277954	315.989624023438\\
6.19000005722046	275.782928466797\\
6.19500017166138	226.590698242188\\
6.19999980926514	163.621505737305\\
6.20499992370605	90.3087310791016\\
6.21000003814697	12.9181499481201\\
6.21500015258789	-20.0268058776855\\
6.21999979019165	-23.3345260620117\\
6.22499990463257	-23.20285987854\\
6.23000001907349	-21.6861152648926\\
6.2350001335144	-19.6431465148926\\
6.23999977111816	-17.5441150665283\\
6.24499988555908	-15.5846452713013\\
6.25	-13.8008737564087\\
6.25500011444092	-12.1974973678589\\
6.26000022888184	-10.764518737793\\
6.2649998664856	-9.4924488067627\\
6.26999998092651	-8.36860752105713\\
6.27500009536743	-7.37819385528564\\
6.28000020980835	-6.50318813323975\\
6.28499984741211	-5.73168516159058\\
6.28999996185303	-5.05098342895508\\
6.29500007629395	-4.45027446746826\\
6.30000019073486	-3.91920709609985\\
6.30499982833862	-3.45007610321045\\
6.30999994277954	-3.03709268569946\\
6.31500005722046	-2.67484521865845\\
6.32000017166138	-2.35531735420227\\
6.32499980926514	-2.07514023780823\\
6.32999992370605	-1.82884681224823\\
6.33500003814697	-1.60954833030701\\
6.34000015258789	-1.4134202003479\\
6.34499979019165	-1.24309599399567\\
6.34999990463257	-1.09758853912354\\
6.35500001907349	-0.974028885364532\\
6.3600001335144	35.062126159668\\
6.36499977111816	65.4983978271484\\
6.36999988555908	88.6946105957031\\
6.375	103.274063110352\\
6.38000011444092	109.190017700195\\
6.38500022888184	138.562393188477\\
6.3899998664856	533.832763671875\\
6.39499998092651	768.248291015625\\
6.40000009536743	971.201477050781\\
6.40500020980835	1141.13610839844\\
6.40999984741211	1282.34741210938\\
6.41499996185303	1396.47680664063\\
6.42000007629395	1488.15588378906\\
6.42500019073486	1563.31567382813\\
6.42999982833862	1628.33374023438\\
6.43499994277954	1689.54309082031\\
6.44000005722046	1751.88244628906\\
6.44500017166138	1819.15173339844\\
6.44999980926514	1893.33435058594\\
6.45499992370605	1974.75622558594\\
6.46000003814697	2062.07983398438\\
6.46500015258789	2152.87280273438\\
6.46999979019165	2244.10034179688\\
6.47499990463257	2332.49755859375\\
6.48000001907349	2415.125\\
6.4850001335144	2489.66918945313\\
6.48999977111816	2554.52709960938\\
6.49499988555908	2608.97265625\\
6.5	2653.0625\\
6.50500011444092	2687.45190429688\\
6.51000022888184	2713.3232421875\\
6.5149998664856	2732.15405273438\\
6.51999998092651	2745.51733398438\\
6.52500009536743	2754.8447265625\\
6.53000020980835	2761.28833007813\\
6.53499984741211	2765.6328125\\
6.53999996185303	2768.31225585938\\
6.54500007629395	2769.47802734375\\
6.55000019073486	2768.9384765625\\
6.55499982833862	2766.4658203125\\
6.55999994277954	2762.11083984375\\
6.56500005722046	2755.98413085938\\
6.57000017166138	2747.09790039063\\
6.57499980926514	2734.91748046875\\
6.57999992370605	2719.1318359375\\
6.58500003814697	2699.89013671875\\
6.59000015258789	2677.30615234375\\
6.59499979019165	2651.75634765625\\
6.59999990463257	2623.79614257813\\
6.60500001907349	2594.01953125\\
6.6100001335144	2563.14990234375\\
6.61499977111816	2531.90844726563\\
6.61999988555908	2500.89868164063\\
6.625	2470.7724609375\\
6.63000011444092	2441.9482421875\\
6.63500022888184	2414.69580078125\\
6.6399998664856	2389.51123046875\\
6.64499998092651	2366.2177734375\\
6.65000009536743	2345.0419921875\\
6.65500020980835	2325.86938476563\\
6.65999984741211	2308.48315429688\\
6.66499996185303	2293.0263671875\\
6.67000007629395	2279.76782226563\\
6.67500019073486	2268.1630859375\\
6.67999982833862	2258.50610351563\\
6.68499994277954	2251.16162109375\\
6.69000005722046	2245.4658203125\\
6.69500017166138	2242.62915039063\\
6.69999980926514	2243.45532226563\\
6.70499992370605	2247.20385742188\\
6.71000003814697	2253.083984375\\
6.71500015258789	2261.17041015625\\
6.71999979019165	2271.1953125\\
6.72499990463257	2282.93481445313\\
6.73000001907349	2296.24536132813\\
6.7350001335144	2310.88525390625\\
6.73999977111816	2325.90942382813\\
6.74499988555908	2341.3388671875\\
6.75	2356.70532226563\\
6.75500011444092	2372.140625\\
6.76000022888184	2388.48559570313\\
6.7649998664856	2403.58984375\\
6.76999998092651	2417.89086914063\\
6.77500009536743	2431.13452148438\\
6.78000020980835	2443.82641601563\\
6.78499984741211	2455.88208007813\\
6.78999996185303	2466.75244140625\\
6.79500007629395	2476.97705078125\\
6.80000019073486	2486.52368164063\\
6.80499982833862	2495.37646484375\\
6.80999994277954	2503.63061523438\\
6.81500005722046	2511.23486328125\\
6.82000017166138	2518.02856445313\\
6.82499980926514	2524.19165039063\\
6.82999992370605	2529.59692382813\\
6.83500003814697	2534.90454101563\\
6.84000015258789	2537.52758789063\\
6.84499979019165	2538.97265625\\
6.84999990463257	2538.02685546875\\
6.85500001907349	2534.27490234375\\
6.8600001335144	2527.81958007813\\
6.86499977111816	2516.37670898438\\
6.86999988555908	2500.84643554688\\
6.875	2483.06494140625\\
6.88000011444092	2461.21411132813\\
6.88500022888184	2436.2236328125\\
6.8899998664856	2408.32495117188\\
6.89499998092651	2377.72412109375\\
6.90000009536743	2344.71728515625\\
6.90500020980835	2309.61303710938\\
6.90999984741211	2272.01000976563\\
6.91499996185303	2232.3544921875\\
6.92000007629395	2190.6044921875\\
6.92500019073486	2146.63354492188\\
6.92999982833862	2100.615234375\\
6.93499994277954	2052.298828125\\
6.94000005722046	2001.95056152344\\
6.94500017166138	1949.56469726563\\
6.94999980926514	1895.14013671875\\
6.95499992370605	1837.70849609375\\
6.96000003814697	1778.74975585938\\
6.96500015258789	1719.65930175781\\
6.96999979019165	1659.00732421875\\
6.97499990463257	1597.99975585938\\
6.98000001907349	1536.84619140625\\
6.9850001335144	1476.197265625\\
6.98999977111816	1415.79333496094\\
6.99499988555908	1355.64343261719\\
7	1296.79931640625\\
7.00500011444092	1238.7041015625\\
7.01000022888184	1181.53332519531\\
7.0149998664856	1125.30322265625\\
7.01999998092651	1070.09985351563\\
7.02500009536743	1016.11077880859\\
7.03000020980835	963.348815917969\\
7.03499984741211	911.887145996094\\
7.03999996185303	861.854309082031\\
7.04500007629395	813.403381347656\\
7.05000019073486	766.713195800781\\
7.05499982833862	722.007446289063\\
7.05999994277954	679.26611328125\\
7.06500005722046	638.72705078125\\
7.07000017166138	600.583618164063\\
7.07499980926514	564.879211425781\\
7.07999992370605	531.668395996094\\
7.08500003814697	501.001098632813\\
7.09000015258789	472.900146484375\\
7.09499979019165	447.356750488281\\
7.09999990463257	424.257049560547\\
7.10500001907349	403.514434814453\\
7.1100001335144	385.045288085938\\
7.11499977111816	368.881134033203\\
7.11999988555908	354.917236328125\\
7.125	343.054107666016\\
7.13000011444092	333.310729980469\\
7.13500022888184	325.639434814453\\
7.1399998664856	320.004150390625\\
7.14499998092651	316.415954589844\\
7.15000009536743	314.953735351563\\
7.15500020980835	315.942474365234\\
7.15999984741211	319.282257080078\\
7.16499996185303	324.763610839844\\
7.17000007629395	331.402130126953\\
7.17500019073486	339.349304199219\\
7.17999982833862	348.584075927734\\
7.18499994277954	358.991058349609\\
7.19000005722046	370.430694580078\\
7.19500017166138	382.8330078125\\
7.19999980926514	396.130981445313\\
7.20499992370605	410.129760742188\\
7.21000003814697	424.906372070313\\
7.21500015258789	440.315093994141\\
7.21999979019165	456.071411132813\\
7.22499990463257	472.326965332031\\
7.23000001907349	488.916473388672\\
7.2350001335144	505.734649658203\\
7.23999977111816	522.736145019531\\
7.24499988555908	539.767272949219\\
7.25	556.726867675781\\
7.25500011444092	573.582885742188\\
7.26000022888184	590.132263183594\\
7.2649998664856	606.3544921875\\
7.26999998092651	622.24267578125\\
7.27500009536743	637.614440917969\\
7.28000020980835	652.344848632813\\
7.28499984741211	666.444213867188\\
7.28999996185303	679.845825195313\\
7.29500007629395	692.565063476563\\
7.30000019073486	704.362487792969\\
7.30499982833862	715.31640625\\
7.30999994277954	725.412536621094\\
7.31500005722046	734.649108886719\\
7.32000017166138	742.9599609375\\
7.32499980926514	750.360534667969\\
7.32999992370605	756.775756835938\\
7.33500003814697	762.19677734375\\
7.34000015258789	766.646301269531\\
7.34499979019165	770.106384277344\\
7.34999990463257	772.577026367188\\
7.35500001907349	774.074279785156\\
7.3600001335144	774.694702148438\\
7.36499977111816	774.457885742188\\
7.36999988555908	773.384948730469\\
7.375	771.483764648438\\
7.38000011444092	768.792663574219\\
7.38500022888184	765.184509277344\\
7.3899998664856	760.580810546875\\
7.39499998092651	755.0146484375\\
7.40000009536743	748.686950683594\\
7.40500020980835	741.558044433594\\
7.40999984741211	733.727416992188\\
7.41499996185303	725.250793457031\\
7.42000007629395	716.203491210938\\
7.42500019073486	706.675048828125\\
7.42999982833862	696.716430664063\\
7.43499994277954	686.340026855469\\
7.44000005722046	675.61328125\\
7.44500017166138	664.57666015625\\
7.44999980926514	653.280456542969\\
7.45499992370605	641.762268066406\\
7.46000003814697	630.057983398438\\
7.46500015258789	618.228210449219\\
7.46999979019165	606.374145507813\\
7.47499990463257	594.560241699219\\
7.48000001907349	582.813049316406\\
7.4850001335144	571.2021484375\\
7.48999977111816	559.760192871094\\
7.49499988555908	548.525207519531\\
7.5	537.528930664063\\
7.50500011444092	526.795959472656\\
7.51000022888184	516.359313964844\\
7.5149998664856	506.296234130859\\
7.51999998092651	496.660766601563\\
7.52500009536743	487.707946777344\\
7.53000020980835	478.981872558594\\
7.53499984741211	470.785675048828\\
7.53999996185303	463.068511962891\\
7.54500007629395	455.859313964844\\
7.55000019073486	449.181610107422\\
7.55499982833862	443.054779052734\\
7.55999994277954	437.495941162109\\
7.56500005722046	432.518157958984\\
7.57000017166138	428.128509521484\\
7.57499980926514	424.338836669922\\
7.57999992370605	421.143524169922\\
7.58500003814697	418.553192138672\\
7.59000015258789	416.55126953125\\
7.59499979019165	415.139709472656\\
7.59999990463257	414.311126708984\\
7.60500001907349	414.065399169922\\
7.6100001335144	414.455718994141\\
7.61499977111816	415.540344238281\\
7.61999988555908	417.152648925781\\
7.625	419.099822998047\\
7.63000011444092	421.432922363281\\
7.63500022888184	424.180084228516\\
7.6399998664856	427.313873291016\\
7.64499998092651	430.827087402344\\
7.65000009536743	434.695220947266\\
7.65500020980835	438.902801513672\\
7.65999984741211	443.423126220703\\
7.66499996185303	448.233306884766\\
7.67000007629395	453.301361083984\\
7.67500019073486	458.600341796875\\
7.67999982833862	464.098785400391\\
7.68499994277954	469.767578125\\
7.69000005722046	475.576995849609\\
7.69500017166138	481.494506835938\\
7.69999980926514	487.489959716797\\
7.70499992370605	493.534881591797\\
7.71000003814697	499.604217529297\\
7.71500015258789	505.681335449219\\
7.71999979019165	511.740325927734\\
7.72499990463257	517.757568359375\\
7.73000001907349	523.708251953125\\
7.7350001335144	529.567260742188\\
7.73999977111816	535.311462402344\\
7.74499988555908	540.937255859375\\
7.75	546.410461425781\\
7.75500011444092	551.727478027344\\
7.76000022888184	556.868469238281\\
7.7649998664856	561.798889160156\\
7.76999998092651	566.503967285156\\
7.77500009536743	570.993774414063\\
7.78000020980835	575.265869140625\\
7.78499984741211	579.2998046875\\
7.78999996185303	583.108032226563\\
7.79500007629395	586.6904296875\\
7.80000019073486	590.015441894531\\
7.80499982833862	593.095092773438\\
7.80999994277954	595.931884765625\\
7.81500005722046	598.521057128906\\
7.82000017166138	600.861938476563\\
7.82499980926514	602.956237792969\\
7.82999992370605	604.812622070313\\
7.83500003814697	606.440795898438\\
7.84000015258789	607.832946777344\\
7.84499979019165	608.996215820313\\
7.84999990463257	609.96875\\
7.85500001907349	610.743530273438\\
7.8600001335144	611.329162597656\\
7.86499977111816	611.726501464844\\
7.86999988555908	611.945556640625\\
7.875	611.993469238281\\
7.88000011444092	611.911560058594\\
7.88500022888184	611.71728515625\\
7.8899998664856	611.420104980469\\
7.89499998092651	611.025939941406\\
7.90000009536743	610.538879394531\\
7.90500020980835	609.976684570313\\
7.90999984741211	609.348449707031\\
7.91499996185303	608.656127929688\\
7.92000007629395	607.913391113281\\
7.92500019073486	607.134704589844\\
7.92999982833862	606.404541015625\\
7.93499994277954	605.674438476563\\
7.94000005722046	604.944458007813\\
7.94500017166138	604.326721191406\\
7.94999980926514	603.7900390625\\
7.95499992370605	603.318786621094\\
7.96000003814697	603.005187988281\\
7.96500015258789	602.877868652344\\
7.96999979019165	602.893981933594\\
7.97499990463257	603.0732421875\\
7.98000001907349	603.44677734375\\
7.9850001335144	603.992004394531\\
7.98999977111816	604.71875\\
7.99499988555908	605.692016601563\\
8	606.874328613281\\
8.00500011444092	608.265625\\
8.01000022888184	609.948974609375\\
8.01500034332275	611.918212890625\\
8.02000045776367	614.186584472656\\
8.02499961853027	616.736083984375\\
8.02999973297119	619.567260742188\\
8.03499984741211	622.696594238281\\
8.03999996185303	626.124328613281\\
8.04500007629395	629.847229003906\\
8.05000019073486	633.873291015625\\
8.05500030517578	638.204345703125\\
8.0600004196167	642.836608886719\\
8.0649995803833	647.774780273438\\
8.06999969482422	653.021240234375\\
8.07499980926514	658.641540527344\\
8.07999992370605	664.625061035156\\
8.08500003814697	670.985717773438\\
8.09000015258789	677.66357421875\\
8.09500026702881	684.655212402344\\
8.10000038146973	691.968322753906\\
8.10499954223633	699.620544433594\\
8.10999965667725	707.630187988281\\
8.11499977111816	715.989685058594\\
8.11999988555908	724.694274902344\\
8.125	733.722290039063\\
8.13000011444092	743.08203125\\
8.13500022888184	752.7734375\\
8.14000034332275	762.8310546875\\
8.14500045776367	773.2451171875\\
8.14999961853027	784.0205078125\\
8.15499973297119	795.112426757813\\
8.15999984741211	806.523681640625\\
8.16499996185303	818.252746582031\\
8.17000007629395	830.384704589844\\
8.17500019073486	842.97509765625\\
8.18000030517578	856.015991210938\\
8.1850004196167	869.450561523438\\
8.1899995803833	883.1396484375\\
8.19499969482422	897.132690429688\\
8.19999980926514	911.46240234375\\
8.20499992370605	926.185302734375\\
8.21000003814697	941.275085449219\\
8.21500015258789	956.698608398438\\
8.22000026702881	972.424133300781\\
8.22500038146973	988.456115722656\\
8.22999954223633	1005.18493652344\\
8.23499965667725	1022.83978271484\\
8.23999977111816	1040.94091796875\\
8.24499988555908	1059.13427734375\\
8.25	1077.68786621094\\
8.25500011444092	1096.78271484375\\
8.26000022888184	1116.35083007813\\
8.26500034332275	1136.4228515625\\
8.27000045776367	1157.15551757813\\
8.27499961853027	1178.56066894531\\
8.27999973297119	1200.41357421875\\
8.28499984741211	1222.82727050781\\
8.28999996185303	1246.06225585938\\
8.29500007629395	1269.90014648438\\
8.30000019073486	1294.19836425781\\
8.30500030517578	1318.65551757813\\
8.3100004196167	1343.39245605469\\
8.3149995803833	1372.09838867188\\
8.31999969482422	1399.91772460938\\
8.32499980926514	1428.33020019531\\
8.32999992370605	1457.62182617188\\
8.33500003814697	1488.00134277344\\
8.34000015258789	1519.28161621094\\
8.34500026702881	1551.34399414063\\
8.35000038146973	1584.50695800781\\
8.35499954223633	1617.90637207031\\
8.35999965667725	1653.54516601563\\
8.36499977111816	1691.34509277344\\
8.36999988555908	1729.23107910156\\
8.375	1768.31884765625\\
8.38000011444092	1808.99206542969\\
8.38500022888184	1850.60437011719\\
8.39000034332275	1893.53723144531\\
8.39500045776367	1936.83044433594\\
8.39999961853027	1983.59826660156\\
8.40499973297119	2031.27490234375\\
8.40999984741211	2079.91796875\\
8.41499996185303	2130.2802734375\\
8.42000007629395	2181.57153320313\\
8.42500019073486	2233.69287109375\\
8.43000030517578	2285.85571289063\\
8.4350004196167	2338.27734375\\
8.4399995803833	2391.34033203125\\
8.44499969482422	2442.71826171875\\
8.44999980926514	2492.5\\
8.45499992370605	2540.55053710938\\
8.46000003814697	2585.44018554688\\
8.46500015258789	2628.19506835938\\
8.47000026702881	2668.03881835938\\
8.47500038146973	2704.94702148438\\
8.47999954223633	2738.884765625\\
8.48499965667725	2769.41137695313\\
8.48999977111816	2795.84228515625\\
8.49499988555908	2819.42944335938\\
8.5	2839.36279296875\\
8.50500011444092	2855.85961914063\\
8.51000022888184	2868.68701171875\\
8.51500034332275	2878.78125\\
8.52000045776367	2887.26953125\\
8.52499961853027	2890.49243164063\\
8.52999973297119	2892.23583984375\\
8.53499984741211	2890.72485351563\\
8.53999996185303	2886.4287109375\\
8.54500007629395	2879.78979492188\\
8.55000019073486	2871.26196289063\\
8.55500030517578	2861.09301757813\\
8.5600004196167	2849.6611328125\\
8.5649995803833	2837.7841796875\\
8.56999969482422	2825.79809570313\\
8.57499980926514	2813.7041015625\\
8.57999992370605	2801.8515625\\
8.58500003814697	2790.15063476563\\
8.59000015258789	2779.25708007813\\
8.59500026702881	2769.08837890625\\
8.60000038146973	2759.3974609375\\
8.60499954223633	2749.95239257813\\
8.60999965667725	2740.37719726563\\
8.61499977111816	2730.1591796875\\
8.61999988555908	2718.70556640625\\
8.625	2705.3935546875\\
8.63000011444092	2689.53173828125\\
8.63500022888184	2670.80151367188\\
8.64000034332275	2648.53076171875\\
8.64500045776367	2622.65869140625\\
8.64999961853027	2593.17944335938\\
8.65499973297119	2560.2216796875\\
8.65999984741211	2524.26782226563\\
8.66499996185303	2485.96142578125\\
8.67000007629395	2445.79809570313\\
8.67500019073486	2404.4140625\\
8.68000030517578	2362.42749023438\\
8.6850004196167	2320.19604492188\\
8.6899995803833	2278.03515625\\
8.69499969482422	2236.08862304688\\
8.69999980926514	2194.27612304688\\
8.70499992370605	2152.48046875\\
8.71000003814697	2110.42041015625\\
8.71500015258789	2067.85571289063\\
8.72000026702881	2024.69653320313\\
8.72500038146973	1980.76647949219\\
8.72999954223633	1935.99816894531\\
8.73499965667725	1890.45922851563\\
8.73999977111816	1844.32202148438\\
8.74499988555908	1797.83728027344\\
8.75	1751.41430664063\\
8.75500011444092	1705.46887207031\\
8.76000022888184	1660.38977050781\\
8.76500034332275	1616.53039550781\\
8.77000045776367	1574.16857910156\\
8.77499961853027	1533.48071289063\\
8.77999973297119	1494.55261230469\\
8.78499984741211	1457.37377929688\\
8.78999996185303	1421.85375976563\\
8.79500007629395	1387.89819335938\\
8.80000019073486	1355.38000488281\\
8.80500030517578	1324.16455078125\\
8.8100004196167	1294.13879394531\\
8.8149995803833	1265.24829101563\\
8.81999969482422	1237.47827148438\\
8.82499980926514	1210.89697265625\\
8.82999992370605	1185.62084960938\\
8.83500003814697	1161.78247070313\\
8.84000015258789	1139.53991699219\\
8.84500026702881	1119.05310058594\\
8.85000038146973	1100.44409179688\\
8.85499954223633	1083.77600097656\\
8.85999965667725	1069.05139160156\\
8.86499977111816	1056.21350097656\\
8.86999988555908	1045.15576171875\\
8.875	1035.73022460938\\
8.88000011444092	1027.7626953125\\
8.88500022888184	1021.06854248047\\
8.89000034332275	1015.47387695313\\
8.89500045776367	1010.86474609375\\
8.89999961853027	1007.15576171875\\
8.90499973297119	1004.33929443359\\
8.90999984741211	1002.39471435547\\
8.91499996185303	1001.34692382813\\
8.92000007629395	1001.22357177734\\
8.92500019073486	1002.14056396484\\
8.93000030517578	1004.45782470703\\
8.9350004196167	1007.90124511719\\
8.9399995803833	1011.93817138672\\
8.94499969482422	1016.81384277344\\
8.94999980926514	1022.39349365234\\
8.95499992370605	1028.63256835938\\
8.96000003814697	1035.41259765625\\
8.96500015258789	1042.63879394531\\
8.97000026702881	1050.21411132813\\
8.97500038146973	1058.07006835938\\
8.97999954223633	1066.15417480469\\
8.98499965667725	1074.462890625\\
8.98999977111816	1082.91284179688\\
8.99499988555908	1091.44995117188\\
9	1100.09777832031\\
9.00500011444092	1108.79821777344\\
9.01000022888184	1117.51049804688\\
9.01500034332275	1126.18823242188\\
9.02000045776367	1134.77319335938\\
9.02499961853027	1143.20642089844\\
9.02999973297119	1151.42724609375\\
9.03499984741211	1159.3798828125\\
9.03999996185303	1167.00646972656\\
9.04500007629395	1174.25402832031\\
9.05000019073486	1181.06945800781\\
9.05500030517578	1187.40539550781\\
9.0600004196167	1193.23852539063\\
9.0649995803833	1198.51611328125\\
9.06999969482422	1203.24011230469\\
9.07499980926514	1207.36962890625\\
9.07999992370605	1210.89453125\\
9.08500003814697	1213.80310058594\\
9.09000015258789	1216.08154296875\\
9.09500026702881	1217.74523925781\\
9.10000038146973	1218.77978515625\\
9.10499954223633	1219.2607421875\\
9.10999965667725	1219.15759277344\\
9.11499977111816	1218.53063964844\\
9.11999988555908	1217.39587402344\\
9.125	1215.67309570313\\
9.13000011444092	1213.22521972656\\
9.13500022888184	1210.13073730469\\
9.14000034332275	1206.41467285156\\
9.14500045776367	1202.08129882813\\
9.14999961853027	1197.18664550781\\
9.15499973297119	1191.8046875\\
9.15999984741211	1185.98522949219\\
9.16499996185303	1179.79833984375\\
9.17000007629395	1173.33813476563\\
9.17500019073486	1166.64282226563\\
9.18000030517578	1159.78503417969\\
9.1850004196167	1152.85498046875\\
9.1899995803833	1145.84069824219\\
9.19499969482422	1138.76647949219\\
9.19999980926514	1131.68859863281\\
9.20499992370605	1124.63598632813\\
9.21000003814697	1117.61889648438\\
9.21500015258789	1110.56921386719\\
9.22000026702881	1103.50549316406\\
9.22500038146973	1096.59631347656\\
9.22999954223633	1089.82214355469\\
9.23499965667725	1083.53430175781\\
9.23999977111816	1078.75866699219\\
9.24499988555908	1077.18334960938\\
9.25	1079.49011230469\\
9.25500011444092	1085.20666503906\\
9.26000022888184	1094.74609375\\
9.26500034332275	1108.66882324219\\
9.27000045776367	1126.74865722656\\
9.27499961853027	1149.30529785156\\
9.27999973297119	1176.82641601563\\
9.28499984741211	1209.17810058594\\
9.28999996185303	1247.10986328125\\
9.29500007629395	1291.11303710938\\
9.30000019073486	1341.25256347656\\
9.30500030517578	1402.11193847656\\
9.3100004196167	1468.51989746094\\
9.3149995803833	1543.5361328125\\
9.31999969482422	1627.72485351563\\
9.32499980926514	1721.66198730469\\
9.32999992370605	1827.41430664063\\
9.33500003814697	1943.56726074219\\
9.34000015258789	2069.87280273438\\
9.34500026702881	2215.1611328125\\
9.35000038146973	2370.03881835938\\
9.35499954223633	2536.37451171875\\
9.35999965667725	2722.2509765625\\
9.36499977111816	2916.6689453125\\
9.36999988555908	3119.88598632813\\
9.375	3331.21948242188\\
9.38000011444092	3547.18237304688\\
9.38500022888184	3765.5859375\\
9.39000034332275	3986.373046875\\
9.39500045776367	4206.89599609375\\
9.39999961853027	4424.31689453125\\
9.40499973297119	4635.00244140625\\
9.40999984741211	4835.3115234375\\
9.41499996185303	5018.81005859375\\
9.42000007629395	5178.345703125\\
9.42500019073486	5307.03515625\\
9.43000030517578	5423.2919921875\\
9.4350004196167	5495.5869140625\\
9.4399995803833	5490.76025390625\\
9.44499969482422	5383.4443359375\\
9.44999980926514	5143.04345703125\\
9.45499992370605	4746.44482421875\\
9.46000003814697	4175.76025390625\\
9.46500015258789	3440.32055664063\\
9.47000026702881	2573.86791992188\\
9.47500038146973	1641.27624511719\\
9.47999954223633	730.096313476563\\
9.48499965667725	57.7467193603516\\
9.48999977111816	39.7754173278809\\
9.49499988555908	51.9142608642578\\
9.5	73.9624633789063\\
9.50500011444092	96.1111450195313\\
9.51000022888184	113.349617004395\\
9.51500034332275	123.61491394043\\
9.52000045776367	126.026092529297\\
9.52499961853027	120.73575592041\\
9.52999973297119	107.522903442383\\
9.53499984741211	87.3548431396484\\
9.53999996185303	61.1176643371582\\
9.54500007629395	30.0457611083984\\
9.55000019073486	-4.26555585861206\\
9.55500030517578	-28.9306354522705\\
9.5600004196167	-37.243465423584\\
9.5649995803833	-38.2830467224121\\
9.56999969482422	-36.242130279541\\
9.57499980926514	-33.2177543640137\\
9.57999992370605	-29.8858642578125\\
9.58500003814697	-26.6398448944092\\
9.59000015258789	-23.6589279174805\\
9.59500026702881	-20.9585475921631\\
9.60000038146973	-18.5775890350342\\
9.60499954223633	-16.4479923248291\\
9.60999965667725	-14.5520896911621\\
9.61499977111816	-12.898549079895\\
9.61999988555908	-11.4414443969727\\
9.625	-10.1642389297485\\
9.63000011444092	-9.01381015777588\\
9.63500022888184	-8.01468944549561\\
9.64000034332275	-7.12434339523315\\
9.64500045776367	-6.3423490524292\\
9.64999961853027	-5.68348693847656\\
9.65499973297119	-5.09833812713623\\
9.65999984741211	-4.57870197296143\\
9.66499996185303	-4.12888431549072\\
9.67000007629395	-3.73681020736694\\
9.67500019073486	-3.38611340522766\\
9.68000030517578	-3.07828855514526\\
9.6850004196167	-2.81329202651978\\
9.6899995803833	-2.58115553855896\\
9.69499969482422	-2.3743724822998\\
9.69999980926514	-2.19193267822266\\
9.70499992370605	-2.03091239929199\\
9.71000003814697	-1.888592004776\\
9.71500015258789	-1.76307272911072\\
9.72000026702881	-1.65097641944885\\
9.72500038146973	-1.54959917068481\\
9.72999954223633	-1.45956671237946\\
9.73499965667725	-1.38085579872131\\
9.73999977111816	-1.31443870067596\\
9.74499988555908	-1.26183366775513\\
9.75	-1.21312761306763\\
9.75500011444092	-1.17087614536285\\
9.76000022888184	-1.13333106040955\\
9.76500034332275	-1.10018789768219\\
9.77000045776367	-1.07148969173431\\
9.77499961853027	-1.04549062252045\\
9.77999973297119	-1.02135610580444\\
9.78499984741211	-1.00080406665802\\
9.78999996185303	-0.982936501502991\\
9.79500007629395	-0.96795243024826\\
9.80000019073486	-0.954927802085876\\
9.80500030517578	-0.94278883934021\\
9.8100004196167	-0.931844592094421\\
9.8149995803833	-0.921867728233337\\
9.81999969482422	-0.913391411304474\\
9.82499980926514	-0.906191229820251\\
9.82999992370605	-0.900378167629242\\
9.83500003814697	-0.895473957061768\\
9.84000015258789	-0.890938341617584\\
9.84500026702881	-0.886859834194183\\
9.85000038146973	-0.883102118968964\\
9.85499954223633	-0.880616903305054\\
9.85999965667725	-0.878934323787689\\
9.86499977111816	-0.878066658973694\\
9.86999988555908	-0.876560509204865\\
9.875	-0.873262405395508\\
9.88000011444092	-0.868369877338409\\
9.88500022888184	-0.861863613128662\\
9.89000034332275	-0.85938572883606\\
9.89500045776367	-0.858956277370453\\
9.89999961853027	-0.861206352710724\\
9.90499973297119	-0.863943517208099\\
9.90999984741211	-0.865692913532257\\
9.91499996185303	-0.867404878139496\\
9.92000007629395	-0.86869204044342\\
9.92500019073486	-0.867922425270081\\
9.93000030517578	-0.865157961845398\\
9.9350004196167	-0.85987800359726\\
9.9399995803833	-0.855464398860931\\
9.94499969482422	-0.853158175945282\\
9.94999980926514	-0.8512082695961\\
9.95499992370605	-0.84961473941803\\
9.96000003814697	-0.848377585411072\\
9.96500015258789	-0.847496747970581\\
9.97000026702881	-0.846972227096558\\
9.97500038146973	-0.846731662750244\\
9.97999954223633	-0.846779584884644\\
9.98499965667725	-0.847153425216675\\
9.98999977111816	-0.847853243350983\\
9.99499988555908	-0.848879039287567\\
10	-0.850230872631073\\
};
\addlegendentry{RS}

\addplot [color=red, line width=2.0pt]
  table[row sep=crcr]{%
0.0949999988079071	4.78557252883911\\
0.100000001490116	4.40723943710327\\
0.104999996721745	4.06200933456421\\
0.109999999403954	3.72439980506897\\
0.115000002086163	3.39587664604187\\
0.119999997317791	270.328857421875\\
0.125	627.580505371094\\
0.129999995231628	913.673950195313\\
0.135000005364418	1132.91357421875\\
0.140000000596046	1289.16186523438\\
0.144999995827675	1389.79516601563\\
0.150000005960464	1441.99084472656\\
0.155000001192093	1451.0703125\\
0.159999996423721	1437.72595214844\\
0.165000006556511	1395.22631835938\\
0.170000001788139	1315.64697265625\\
0.174999997019768	1200.56506347656\\
0.180000007152557	1053.36901855469\\
0.185000002384186	880.932495117188\\
0.189999997615814	690.116088867188\\
0.194999992847443	541.318664550781\\
0.200000002980232	356.809448242188\\
0.204999998211861	240.161285400391\\
0.209999993443489	314.721618652344\\
0.215000003576279	565.54443359375\\
0.219999998807907	944.51611328125\\
0.224999994039536	1383.40417480469\\
0.230000004172325	1840.05187988281\\
0.234999999403954	2266.05322265625\\
0.239999994635582	2618.57543945313\\
0.245000004768372	2868.41577148438\\
0.25	3002.69555664063\\
0.254999995231628	3027.12744140625\\
0.259999990463257	2993.72485351563\\
0.264999985694885	2895.72680664063\\
0.270000010728836	2742.75561523438\\
0.275000005960464	2566.53149414063\\
0.280000001192093	2401.52880859375\\
0.284999996423721	2277.26831054688\\
0.28999999165535	2209.66064453125\\
0.294999986886978	2220.77026367188\\
0.300000011920929	2290.05053710938\\
0.305000007152557	2381.80859375\\
0.310000002384186	2474.66967773438\\
0.314999997615814	2551.49536132813\\
0.319999992847443	2597.74926757813\\
0.324999988079071	2604.88793945313\\
0.330000013113022	2588.16479492188\\
0.33500000834465	2537.02172851563\\
0.340000003576279	2446.5625\\
0.344999998807907	2322.91333007813\\
0.349999994039536	2180.6396484375\\
0.354999989271164	2034.50549316406\\
0.360000014305115	1897.44165039063\\
0.365000009536743	1779.0048828125\\
0.370000004768372	1684.31201171875\\
0.375	1613.3466796875\\
0.379999995231628	1563.99682617188\\
0.384999990463257	1529.99682617188\\
0.389999985694885	1503.69384765625\\
0.395000010728836	1476.00805664063\\
0.400000005960464	1439.60559082031\\
0.405000001192093	1388.71948242188\\
0.409999996423721	1320.53845214844\\
0.41499999165535	1235.37426757813\\
0.419999986886978	1136.59741210938\\
0.425000011920929	1029.61071777344\\
0.430000007152557	920.455810546875\\
0.435000002384186	816.669128417969\\
0.439999997615814	724.6015625\\
0.444999992847443	647.606567382813\\
0.449999988079071	587.764404296875\\
0.455000013113022	544.562744140625\\
0.46000000834465	515.308349609375\\
0.465000003576279	496.374908447266\\
0.469999998807907	482.43896484375\\
0.474999994039536	469.243347167969\\
0.479999989271164	452.985229492188\\
0.485000014305115	431.107818603516\\
0.490000009536743	402.816955566406\\
0.495000004768372	369.486236572266\\
0.5	332.676574707031\\
0.504999995231628	296.379486083984\\
0.509999990463257	264.252624511719\\
0.514999985694885	239.832153320313\\
0.519999980926514	225.407836914063\\
0.524999976158142	223.722763061523\\
0.529999971389771	236.225601196289\\
0.535000026226044	256.001678466797\\
0.540000021457672	280.343383789063\\
0.545000016689301	307.033203125\\
0.550000011920929	334.427215576172\\
0.555000007152557	361.540710449219\\
0.560000002384186	387.34814453125\\
0.564999997615814	411.906280517578\\
0.569999992847443	435.916656494141\\
0.574999988079071	460.296783447266\\
0.579999983310699	485.688018798828\\
0.584999978542328	513.135864257813\\
0.589999973773956	543.590270996094\\
0.595000028610229	576.786926269531\\
0.600000023841858	613.108764648438\\
0.605000019073486	651.9228515625\\
0.610000014305115	692.949096679688\\
0.615000009536743	735.590698242188\\
0.620000004768372	778.702941894531\\
0.625	821.804626464844\\
0.629999995231628	864.663696289063\\
0.634999990463257	906.613586425781\\
0.639999985694885	947.405334472656\\
0.644999980926514	986.84423828125\\
0.649999976158142	1024.94030761719\\
0.654999971389771	1061.72802734375\\
0.660000026226044	1097.341796875\\
0.665000021457672	1131.85375976563\\
0.670000016689301	1165.34167480469\\
0.675000011920929	1197.82385253906\\
0.680000007152557	1229.27966308594\\
0.685000002384186	1259.55944824219\\
0.689999997615814	1288.5361328125\\
0.694999992847443	1315.94738769531\\
0.699999988079071	1341.62927246094\\
0.704999983310699	1365.32653808594\\
0.709999978542328	1386.92993164063\\
0.714999973773956	1406.29943847656\\
0.720000028610229	1423.43176269531\\
0.725000023841858	1438.28869628906\\
0.730000019073486	1450.88830566406\\
0.735000014305115	1461.31518554688\\
0.740000009536743	1469.60986328125\\
0.745000004768372	1475.859375\\
0.75	1480.12280273438\\
0.754999995231628	1482.4697265625\\
0.759999990463257	1482.94714355469\\
0.764999985694885	1482.22705078125\\
0.769999980926514	1480.16271972656\\
0.774999976158142	1476.30749511719\\
0.779999971389771	1470.49975585938\\
0.785000026226044	1462.82397460938\\
0.790000021457672	1453.40612792969\\
0.795000016689301	1442.3466796875\\
0.800000011920929	1429.76953125\\
0.805000007152557	1415.75939941406\\
0.810000002384186	1400.47521972656\\
0.814999997615814	1383.99304199219\\
0.819999992847443	1366.43273925781\\
0.824999988079071	1347.98779296875\\
0.829999983310699	1328.72729492188\\
0.834999978542328	1308.78942871094\\
0.839999973773956	1288.35339355469\\
0.845000028610229	1267.48095703125\\
0.850000023841858	1246.33666992188\\
0.855000019073486	1224.97741699219\\
0.860000014305115	1203.49829101563\\
0.865000009536743	1181.96716308594\\
0.870000004768372	1160.48095703125\\
0.875	1139.13305664063\\
0.879999995231628	1118.01428222656\\
0.884999990463257	1097.21142578125\\
0.889999985694885	1076.81567382813\\
0.894999980926514	1056.9169921875\\
0.899999976158142	1037.59094238281\\
0.904999971389771	1018.89538574219\\
0.910000026226044	1000.95886230469\\
0.915000021457672	983.82275390625\\
0.920000016689301	967.523010253906\\
0.925000011920929	952.170166015625\\
0.930000007152557	937.803344726563\\
0.935000002384186	924.453186035156\\
0.939999997615814	912.181335449219\\
0.944999992847443	901.010620117188\\
0.949999988079071	890.870361328125\\
0.954999983310699	881.811645507813\\
0.959999978542328	873.82421875\\
0.964999973773956	866.914733886719\\
0.970000028610229	861.090270996094\\
0.975000023841858	856.34375\\
0.980000019073486	852.665771484375\\
0.985000014305115	850.072387695313\\
0.990000009536743	848.582458496094\\
0.995000004768372	848.210815429688\\
1	849.0048828125\\
1.00499999523163	851.013488769531\\
1.00999999046326	854.236328125\\
1.01499998569489	858.195983886719\\
1.01999998092651	862.857788085938\\
1.02499997615814	868.210571289063\\
1.02999997138977	874.2890625\\
1.0349999666214	881.02001953125\\
1.03999996185303	888.389038085938\\
1.04499995708466	896.336059570313\\
1.04999995231628	904.834167480469\\
1.05499994754791	913.823608398438\\
1.05999994277954	923.259338378906\\
1.06500005722046	933.115966796875\\
1.07000005245209	943.314697265625\\
1.07500004768372	953.816589355469\\
1.08000004291534	964.559448242188\\
1.08500003814697	975.493957519531\\
1.0900000333786	986.565979003906\\
1.09500002861023	997.721740722656\\
1.10000002384186	1008.90783691406\\
1.10500001907349	1020.07366943359\\
1.11000001430511	1031.17309570313\\
1.11500000953674	1042.14770507813\\
1.12000000476837	1052.94616699219\\
1.125	1063.53479003906\\
1.12999999523163	1073.86389160156\\
1.13499999046326	1083.8818359375\\
1.13999998569489	1093.55017089844\\
1.14499998092651	1102.86340332031\\
1.14999997615814	1111.76721191406\\
1.15499997138977	1120.22216796875\\
1.1599999666214	1128.22448730469\\
1.16499996185303	1135.75122070313\\
1.16999995708466	1142.76525878906\\
1.17499995231628	1149.24328613281\\
1.17999994754791	1155.201171875\\
1.18499994277954	1160.60119628906\\
1.19000005722046	1165.4228515625\\
1.19500005245209	1169.67431640625\\
1.20000004768372	1173.365234375\\
1.20500004291534	1176.48852539063\\
1.21000003814697	1179.04724121094\\
1.2150000333786	1181.033203125\\
1.22000002861023	1182.45007324219\\
1.22500002384186	1183.3046875\\
1.23000001907349	1183.63818359375\\
1.23500001430511	1183.5107421875\\
1.24000000953674	1182.95275878906\\
1.24500000476837	1182.01184082031\\
1.25	1180.53796386719\\
1.25499999523163	1178.57360839844\\
1.25999999046326	1176.10217285156\\
1.26499998569489	1173.14978027344\\
1.26999998092651	1169.78442382813\\
1.27499997615814	1165.99377441406\\
1.27999997138977	1161.81079101563\\
1.2849999666214	1157.34338378906\\
1.28999996185303	1152.59484863281\\
1.29499995708466	1147.60070800781\\
1.29999995231628	1142.39855957031\\
1.30499994754791	1137.01611328125\\
1.30999994277954	1131.48901367188\\
1.31500005722046	1125.81799316406\\
1.32000005245209	1120.03283691406\\
1.32500004768372	1114.1611328125\\
1.33000004291534	1108.21752929688\\
1.33500003814697	1102.22338867188\\
1.3400000333786	1096.20446777344\\
1.34500002861023	1090.18774414063\\
1.35000002384186	1084.20495605469\\
1.35500001907349	1078.28381347656\\
1.36000001430511	1072.46411132813\\
1.36500000953674	1066.81335449219\\
1.37000000476837	1061.37951660156\\
1.375	1056.20263671875\\
1.37999999523163	1051.20947265625\\
1.38499999046326	1046.4453125\\
1.38999998569489	1041.93432617188\\
1.39499998092651	1037.62951660156\\
1.39999997615814	1033.55041503906\\
1.40499997138977	1029.70483398438\\
1.4099999666214	1026.115234375\\
1.41499996185303	1022.78680419922\\
1.41999995708466	1019.72991943359\\
1.42499995231628	1016.96282958984\\
1.42999994754791	1014.49981689453\\
1.43499994277954	1012.34527587891\\
1.44000005722046	1010.49383544922\\
1.44500005245209	1008.95758056641\\
1.45000004768372	1007.734375\\
1.45500004291534	1006.8173828125\\
1.46000003814697	1006.21295166016\\
1.4650000333786	1005.94055175781\\
1.47000002861023	1006.02709960938\\
1.47500002384186	1006.49139404297\\
1.48000001907349	1007.21905517578\\
1.48500001430511	1008.1044921875\\
1.49000000953674	1009.06408691406\\
1.49500000476837	1010.34368896484\\
1.5	1011.96868896484\\
1.50499999523163	1013.90936279297\\
1.50999999046326	1016.05767822266\\
1.51499998569489	1018.40399169922\\
1.51999998092651	1020.92590332031\\
1.52499997615814	1023.603515625\\
1.52999997138977	1026.42041015625\\
1.5349999666214	1029.39038085938\\
1.53999996185303	1032.49865722656\\
1.54499995708466	1035.708984375\\
1.54999995231628	1038.97473144531\\
1.55499994754791	1042.27465820313\\
1.55999994277954	1045.5927734375\\
1.56500005722046	1048.90637207031\\
1.57000005245209	1040.85803222656\\
1.57500004768372	1041.69189453125\\
1.58000004291534	1046.27758789063\\
1.58500003814697	1051.88488769531\\
1.5900000333786	1057.26867675781\\
1.59500002861023	1062.15356445313\\
1.60000002384186	1066.42041015625\\
1.60500001907349	1070.02099609375\\
1.61000001430511	1072.93054199219\\
1.61500000953674	1075.27807617188\\
1.62000000476837	1077.21569824219\\
1.625	1078.85632324219\\
1.62999999523163	1080.38452148438\\
1.63499999046326	1081.9267578125\\
1.63999998569489	1083.59313964844\\
1.64499998092651	1085.423828125\\
1.64999997615814	1087.29663085938\\
1.65499997138977	1089.18273925781\\
1.6599999666214	1091.05163574219\\
1.66499996185303	1092.81555175781\\
1.66999995708466	1094.39270019531\\
1.67499995231628	1095.79333496094\\
1.67999994754791	1096.96313476563\\
1.68499994277954	1097.85827636719\\
1.69000005722046	1098.39111328125\\
1.69500005245209	1098.72692871094\\
1.70000004768372	1098.89074707031\\
1.70500004291534	1098.88012695313\\
1.71000003814697	1098.78771972656\\
1.7150000333786	1098.60363769531\\
1.72000002861023	1098.33203125\\
1.72500002384186	1097.94140625\\
1.73000001907349	1097.45153808594\\
1.73500001430511	1096.876953125\\
1.74000000953674	1096.21887207031\\
1.74500000476837	1095.48388671875\\
1.75	1094.68981933594\\
1.75499999523163	1093.82727050781\\
1.75999999046326	1092.904296875\\
1.76499998569489	1091.91735839844\\
1.76999998092651	1090.86572265625\\
1.77499997615814	1089.74877929688\\
1.77999997138977	1088.56262207031\\
1.7849999666214	1087.31018066406\\
1.78999996185303	1085.99328613281\\
1.79499995708466	1084.61938476563\\
1.79999995231628	1083.20874023438\\
1.80499994754791	1081.76806640625\\
1.80999994277954	1080.31433105469\\
1.81500005722046	1078.86083984375\\
1.82000005245209	1077.41906738281\\
1.82500004768372	1075.99865722656\\
1.83000004291534	1074.61083984375\\
1.83500003814697	1073.27294921875\\
1.8400000333786	1071.98120117188\\
1.84500002861023	1070.734375\\
1.85000002384186	1069.5322265625\\
1.85500001907349	1068.3701171875\\
1.86000001430511	1067.251953125\\
1.86500000953674	1066.18383789063\\
1.87000000476837	1065.16271972656\\
1.875	1064.20178222656\\
1.87999999523163	1063.31005859375\\
1.88499999046326	1062.47875976563\\
1.88999998569489	1061.70874023438\\
1.89499998092651	1061.0107421875\\
1.89999997615814	1060.38549804688\\
1.90499997138977	1059.830078125\\
1.9099999666214	1059.353515625\\
1.91499996185303	1058.95361328125\\
1.91999995708466	1058.62939453125\\
1.92499995231628	1058.39184570313\\
1.92999994754791	1058.23913574219\\
1.93499994277954	1058.16870117188\\
1.94000005722046	1058.1708984375\\
1.94500005245209	1058.2412109375\\
1.95000004768372	1058.38354492188\\
1.95500004291534	1058.59790039063\\
1.96000003814697	1058.88256835938\\
1.9650000333786	1059.23425292969\\
1.97000002861023	1059.65173339844\\
1.97500002384186	1060.13513183594\\
1.98000001907349	1060.6962890625\\
1.98500001430511	1061.32946777344\\
1.99000000953674	1062.01831054688\\
1.99500000476837	1062.75622558594\\
2	1063.54138183594\\
2.00500011444092	1064.35791015625\\
2.00999999046326	1065.1923828125\\
2.01500010490417	1066.03637695313\\
2.01999998092651	1066.93835449219\\
2.02500009536743	1067.88745117188\\
2.02999997138977	1068.884765625\\
2.03500008583069	1069.94067382813\\
2.03999996185303	1071.10607910156\\
2.04500007629395	1072.37719726563\\
2.04999995231628	1073.74426269531\\
2.0550000667572	1075.20361328125\\
2.05999994277954	1076.72937011719\\
2.06500005722046	1078.31689453125\\
2.0699999332428	1079.88610839844\\
2.07500004768372	1081.40209960938\\
2.07999992370605	1082.76281738281\\
2.08500003814697	1083.89184570313\\
2.08999991416931	1084.68884277344\\
2.09500002861023	1085.02514648438\\
2.09999990463257	1084.8359375\\
2.10500001907349	1084.03662109375\\
2.10999989509583	1082.72082519531\\
2.11500000953674	1080.91931152344\\
2.11999988555908	1078.3310546875\\
2.125	1074.658203125\\
2.13000011444092	1070.04235839844\\
2.13499999046326	1064.50903320313\\
2.14000010490417	1058.13037109375\\
2.14499998092651	1051.03845214844\\
2.15000009536743	1043.54455566406\\
2.15499997138977	1035.65979003906\\
2.16000008583069	1027.28430175781\\
2.16499996185303	1018.70776367188\\
2.17000007629395	1009.90692138672\\
2.17499995231628	1000.78765869141\\
2.1800000667572	991.428283691406\\
2.18499994277954	981.753601074219\\
2.19000005722046	971.760559082031\\
2.1949999332428	961.443054199219\\
2.20000004768372	950.872802734375\\
2.20499992370605	940.159118652344\\
2.21000003814697	929.445434570313\\
2.21499991416931	918.90771484375\\
2.22000002861023	908.732116699219\\
2.22499990463257	899.114562988281\\
2.23000001907349	890.190063476563\\
2.23499989509583	882.070190429688\\
2.24000000953674	874.921081542969\\
2.24499988555908	868.717956542969\\
2.25	863.460388183594\\
2.25500011444092	859.199829101563\\
2.25999999046326	855.8759765625\\
2.26500010490417	853.385131835938\\
2.26999998092651	851.734313964844\\
2.27500009536743	850.950988769531\\
2.27999997138977	851.156311035156\\
2.28500008583069	852.606750488281\\
2.28999996185303	855.4658203125\\
2.29500007629395	859.530944824219\\
2.29999995231628	863.851135253906\\
2.3050000667572	869.229370117188\\
2.30999994277954	875.648376464844\\
2.31500005722046	882.799377441406\\
2.3199999332428	890.660705566406\\
2.32500004768372	899.399658203125\\
2.32999992370605	909.026489257813\\
2.33500003814697	919.5205078125\\
2.33999991416931	932.408264160156\\
2.34500002861023	947.338256835938\\
2.34999990463257	961.669006347656\\
2.35500001907349	976.5390625\\
2.35999989509583	991.873596191406\\
2.36500000953674	1007.71185302734\\
2.36999988555908	1024.33349609375\\
2.375	1041.53015136719\\
2.38000011444092	1059.37414550781\\
2.38499999046326	1077.98779296875\\
2.39000010490417	1097.33679199219\\
2.39499998092651	1117.38354492188\\
2.40000009536743	1138.19372558594\\
2.40499997138977	1159.70239257813\\
2.41000008583069	1181.85913085938\\
2.41499996185303	1204.57849121094\\
2.42000007629395	1227.64343261719\\
2.42499995231628	1250.85559082031\\
2.4300000667572	1273.92565917969\\
2.43499994277954	1296.63879394531\\
2.44000005722046	1318.71215820313\\
2.4449999332428	1339.80493164063\\
2.45000004768372	1359.83569335938\\
2.45499992370605	1378.35266113281\\
2.46000003814697	1395.28637695313\\
2.46499991416931	1410.5107421875\\
2.47000002861023	1423.76794433594\\
2.47499990463257	1435.16430664063\\
2.48000001907349	1444.67578125\\
2.48499989509583	1452.30786132813\\
2.49000000953674	1458.14916992188\\
2.49499988555908	1462.263671875\\
2.5	1464.69616699219\\
2.50500011444092	1465.54016113281\\
2.50999999046326	1465.22265625\\
2.51500010490417	1464.0498046875\\
2.51999998092651	1461.31848144531\\
2.52500009536743	1456.63500976563\\
2.52999997138977	1450.20764160156\\
2.53500008583069	1441.30737304688\\
2.53999996185303	1430.67639160156\\
2.54500007629395	1418.12255859375\\
2.54999995231628	1403.70336914063\\
2.5550000667572	1387.57336425781\\
2.55999994277954	1369.77307128906\\
2.56500005722046	1350.57495117188\\
2.5699999332428	1330.15649414063\\
2.57500004768372	1308.68334960938\\
2.57999992370605	1286.36291503906\\
2.58500003814697	1263.42724609375\\
2.58999991416931	1240.12683105469\\
2.59500002861023	1216.41577148438\\
2.59999990463257	1192.33337402344\\
2.60500001907349	1167.9150390625\\
2.60999989509583	1143.130859375\\
2.61500000953674	1117.95654296875\\
2.61999988555908	1092.40661621094\\
2.625	1066.53063964844\\
2.63000011444092	1040.51550292969\\
2.63499999046326	1014.49554443359\\
2.64000010490417	988.481628417969\\
2.64499998092651	962.682434082031\\
2.65000009536743	937.4560546875\\
2.65499997138977	912.892761230469\\
2.66000008583069	888.927917480469\\
2.66499996185303	865.69677734375\\
2.67000007629395	843.060241699219\\
2.67499995231628	820.914306640625\\
2.6800000667572	799.356384277344\\
2.68499994277954	778.227233886719\\
2.69000005722046	757.635437011719\\
2.6949999332428	737.473815917969\\
2.70000004768372	717.643676757813\\
2.70499992370605	698.253967285156\\
2.71000003814697	679.373046875\\
2.71499991416931	661.056640625\\
2.72000002861023	643.641174316406\\
2.72499990463257	627.23291015625\\
2.73000001907349	611.935180664063\\
2.73499989509583	597.914611816406\\
2.74000000953674	585.353454589844\\
2.74499988555908	574.533996582031\\
2.75	565.364685058594\\
2.75500011444092	557.773803710938\\
2.75999999046326	552.003723144531\\
2.76500010490417	548.531433105469\\
2.76999998092651	547.539672851563\\
2.77500009536743	549.089965820313\\
2.77999997138977	552.868041992188\\
2.78500008583069	558.959289550781\\
2.78999996185303	567.518432617188\\
2.79500007629395	578.720703125\\
2.79999995231628	592.724914550781\\
2.8050000667572	609.684143066406\\
2.80999994277954	629.827575683594\\
2.81500005722046	653.198852539063\\
2.8199999332428	679.804382324219\\
2.82500004768372	709.600524902344\\
2.82999992370605	742.34375\\
2.83500003814697	778.131896972656\\
2.83999991416931	816.51611328125\\
2.84500002861023	857.2490234375\\
2.84999990463257	900.293579101563\\
2.85500001907349	944.959594726563\\
2.85999989509583	991.354797363281\\
2.86500000953674	1039.22875976563\\
2.86999988555908	1089.30517578125\\
2.875	1140.47985839844\\
2.88000011444092	1193.60791015625\\
2.88499999046326	1248.61633300781\\
2.89000010490417	1305.28283691406\\
2.89499998092651	1362.71435546875\\
2.90000009536743	1418.25463867188\\
2.90499997138977	1470.88452148438\\
2.91000008583069	1519.15161132813\\
2.91499996185303	1561.78015136719\\
2.92000007629395	1598.4384765625\\
2.92499995231628	1629.232421875\\
2.9300000667572	1654.658203125\\
2.93499994277954	1675.39404296875\\
2.94000005722046	1692.43737792969\\
2.9449999332428	1706.35217285156\\
2.95000004768372	1717.25646972656\\
2.95499992370605	1725.55773925781\\
2.96000003814697	1731.97204589844\\
2.96499991416931	1735.08044433594\\
2.97000002861023	1734.73522949219\\
2.97499990463257	1729.75646972656\\
2.98000001907349	1719.10717773438\\
2.98499989509583	1708.49963378906\\
2.99000000953674	1690.66784667969\\
2.99499988555908	1667.09619140625\\
3	1637.94970703125\\
3.00500011444092	1603.669921875\\
3.00999999046326	1564.62097167969\\
3.01500010490417	1521.55688476563\\
3.01999998092651	1474.95129394531\\
3.02500009536743	1425.76501464844\\
3.02999997138977	1375.15649414063\\
3.03500008583069	1323.61242675781\\
3.03999996185303	1271.50085449219\\
3.04500007629395	1219.05053710938\\
3.04999995231628	1166.32470703125\\
3.0550000667572	1113.00805664063\\
3.05999994277954	1058.84851074219\\
3.06500005722046	1003.65618896484\\
3.0699999332428	947.32421875\\
3.07500004768372	890.180114746094\\
3.07999992370605	832.716369628906\\
3.08500003814697	775.509826660156\\
3.08999991416931	719.391235351563\\
3.09500002861023	665.288696289063\\
3.09999990463257	613.742919921875\\
3.10500001907349	565.218627929688\\
3.10999989509583	520.377563476563\\
3.11500000953674	479.937103271484\\
3.11999988555908	443.965454101563\\
3.125	412.562591552734\\
3.13000011444092	386.373596191406\\
3.13499999046326	365.645355224609\\
3.14000010490417	350.243438720703\\
3.14499998092651	341.002960205078\\
3.15000009536743	340.362762451172\\
3.15499997138977	349.817321777344\\
3.16000008583069	367.345764160156\\
3.16499996185303	392.480163574219\\
3.17000007629395	424.914611816406\\
3.17499995231628	464.169738769531\\
3.1800000667572	509.828369140625\\
3.18499994277954	560.381469726563\\
3.19000005722046	615.034790039063\\
3.1949999332428	672.514892578125\\
3.20000004768372	731.575378417969\\
3.20499992370605	791.476135253906\\
3.21000003814697	850.807189941406\\
3.21499991416931	909.609680175781\\
3.22000002861023	966.7587890625\\
3.22499990463257	1022.62957763672\\
3.23000001907349	1076.93017578125\\
3.23499989509583	1129.66052246094\\
3.24000000953674	1180.55505371094\\
3.24499988555908	1228.49011230469\\
3.25	1273.85070800781\\
3.25500011444092	1316.50146484375\\
3.25999999046326	1356.70275878906\\
3.26500010490417	1394.68212890625\\
3.26999998092651	1430.87243652344\\
3.27500009536743	1465.72766113281\\
3.27999997138977	1499.30151367188\\
3.28500008583069	1531.39489746094\\
3.28999996185303	1561.55163574219\\
3.29500007629395	1588.93981933594\\
3.29999995231628	1612.7958984375\\
3.3050000667572	1632.40649414063\\
3.30999994277954	1647.01635742188\\
3.31500005722046	1656.31359863281\\
3.3199999332428	1662.66552734375\\
3.32500004768372	1664.52526855469\\
3.32999992370605	1660.27429199219\\
3.33500003814697	1649.86743164063\\
3.33999991416931	1634.12841796875\\
3.34500002861023	1613.73010253906\\
3.34999990463257	1589.27392578125\\
3.35500001907349	1561.33129882813\\
3.35999989509583	1530.33728027344\\
3.36500000953674	1496.2861328125\\
3.36999988555908	1459.05139160156\\
3.375	1418.43969726563\\
3.38000011444092	1374.35388183594\\
3.38499999046326	1326.53466796875\\
3.39000010490417	1274.95874023438\\
3.39499998092651	1219.83825683594\\
3.40000009536743	1161.36315917969\\
3.40499997138977	1100.22277832031\\
3.41000008583069	1037.216796875\\
3.41499996185303	972.694763183594\\
3.42000007629395	907.804504394531\\
3.42499995231628	843.153015136719\\
3.4300000667572	779.54931640625\\
3.43499994277954	717.501953125\\
3.44000005722046	657.515747070313\\
3.4449999332428	600.267761230469\\
3.45000004768372	546.513366699219\\
3.45499992370605	497.027465820313\\
3.46000003814697	452.551422119141\\
3.46499991416931	413.401092529297\\
3.47000002861023	380.040679931641\\
3.47499990463257	354.415771484375\\
3.48000001907349	336.567291259766\\
3.48499989509583	328.251007080078\\
3.49000000953674	334.313751220703\\
3.49499988555908	350.742736816406\\
3.5	376.177917480469\\
3.50500011444092	410.234130859375\\
3.50999999046326	452.777221679688\\
3.51500010490417	504.360687255859\\
3.51999998092651	565.048278808594\\
3.52500009536743	636.101379394531\\
3.52999997138977	717.5595703125\\
3.53500008583069	810.895324707031\\
3.53999996185303	918.153442382813\\
3.54500007629395	1036.72912597656\\
3.54999995231628	1157.60412597656\\
3.5550000667572	1266.796875\\
3.55999994277954	1354.10864257813\\
3.56500005722046	1414.47644042969\\
3.5699999332428	1452.79028320313\\
3.57500004768372	1472.193359375\\
3.57999992370605	1484.10192871094\\
3.58500003814697	1489.88342285156\\
3.58999991416931	1493.20031738281\\
3.59500002861023	1503.0205078125\\
3.59999990463257	1527.03979492188\\
3.60500001907349	1561.02355957031\\
3.60999989509583	1600.72106933594\\
3.61500000953674	1640.92370605469\\
3.61999988555908	1676.49426269531\\
3.625	1703.16223144531\\
3.63000011444092	1717.28112792969\\
3.63499999046326	1720.38195800781\\
3.64000010490417	1712.98657226563\\
3.64499998092651	1689.69763183594\\
3.65000009536743	1651.40673828125\\
3.65499997138977	1600.50549316406\\
3.66000008583069	1542.20715332031\\
3.66499996185303	1481.779296875\\
3.67000007629395	1422.55590820313\\
3.67499995231628	1366.93798828125\\
3.6800000667572	1315.24475097656\\
3.68499994277954	1266.54736328125\\
3.69000005722046	1218.54992675781\\
3.6949999332428	1168.43151855469\\
3.70000004768372	1113.65881347656\\
3.70499992370605	1052.42687988281\\
3.71000003814697	983.699462890625\\
3.71499991416931	908.519714355469\\
3.72000002861023	828.399230957031\\
3.72499990463257	746.282897949219\\
3.73000001907349	664.538879394531\\
3.73499989509583	587.159362792969\\
3.74000000953674	515.378234863281\\
3.74499988555908	451.745147705078\\
3.75	398.772338867188\\
3.75500011444092	355.365997314453\\
3.75999999046326	322.016571044922\\
3.76500010490417	298.395477294922\\
3.76999998092651	285.424011230469\\
3.77500009536743	288.865417480469\\
3.77999997138977	306.901184082031\\
3.78500008583069	337.638000488281\\
3.78999996185303	379.977478027344\\
3.79500007629395	433.691223144531\\
3.79999995231628	498.020568847656\\
3.8050000667572	572.901123046875\\
3.80999994277954	657.605407714844\\
3.81500005722046	752.163513183594\\
3.8199999332428	856.153991699219\\
3.82500004768372	969.267028808594\\
3.82999992370605	1091.07507324219\\
3.83500003814697	1215.46447753906\\
3.83999991416931	1330.14294433594\\
3.84500002861023	1422.72082519531\\
3.84999990463257	1484.32641601563\\
3.85500001907349	1514.6748046875\\
3.85999989509583	1527.873046875\\
3.86500000953674	1527.77270507813\\
3.86999988555908	1516.51953125\\
3.875	1503.97631835938\\
3.88000011444092	1500.33715820313\\
3.88499999046326	1521.16577148438\\
3.89000010490417	1563.32666015625\\
3.89499998092651	1617.45654296875\\
3.90000009536743	1675.53820800781\\
3.90499997138977	1728.05627441406\\
3.91000008583069	1769.27075195313\\
3.91499996185303	1793.80102539063\\
3.92000007629395	1802.17395019531\\
3.92499995231628	1797.75122070313\\
3.9300000667572	1773.53051757813\\
3.93499994277954	1729.82885742188\\
3.94000005722046	1671.62658691406\\
3.9449999332428	1605.26342773438\\
3.95000004768372	1536.91442871094\\
3.95499992370605	1470.84045410156\\
3.96000003814697	1408.91333007813\\
3.96499991416931	1351.93530273438\\
3.97000002861023	1297.95568847656\\
3.97499990463257	1244.24047851563\\
3.98000001907349	1186.57678222656\\
3.98499989509583	1121.69140625\\
3.99000000953674	1047.65283203125\\
3.99499988555908	964.117797851563\\
4	871.696411132813\\
4.00500011444092	772.261779785156\\
4.01000022888184	669.688659667969\\
4.0149998664856	568.689453125\\
4.01999998092651	471.859375\\
4.02500009536743	382.662078857422\\
4.03000020980835	303.785980224609\\
4.03499984741211	236.345642089844\\
4.03999996185303	180.279647827148\\
4.04500007629395	135.623428344727\\
4.05000019073486	103.063133239746\\
4.05499982833862	82.7012100219727\\
4.05999994277954	75.9889526367188\\
4.06500005722046	88.8186950683594\\
4.07000017166138	116.050834655762\\
4.07499980926514	156.519012451172\\
4.07999992370605	211.37809753418\\
4.08500003814697	283.081878662109\\
4.09000015258789	376.807403564453\\
4.09499979019165	499.308349609375\\
4.09999990463257	663.839111328125\\
4.10500001907349	891.498168945313\\
4.1100001335144	1175.64099121094\\
4.11499977111816	1455.24865722656\\
4.11999988555908	1668.45263671875\\
4.125	1787.79284667969\\
4.13000011444092	1823.24377441406\\
4.13500022888184	1808.69421386719\\
4.1399998664856	1736.09655761719\\
4.14499998092651	1626.84887695313\\
4.15000009536743	1515.57641601563\\
4.15500020980835	1435.79565429688\\
4.15999984741211	1415.0693359375\\
4.16499996185303	1477.28149414063\\
4.17000007629395	1584.52465820313\\
4.17500019073486	1704.4716796875\\
4.17999982833862	1817.5517578125\\
4.18499994277954	1906.82934570313\\
4.19000005722046	1957.08215332031\\
4.19500017166138	1964.56127929688\\
4.19999980926514	1947.76708984375\\
4.20499992370605	1892.09985351563\\
4.21000003814697	1796.20703125\\
4.21500015258789	1676.24499511719\\
4.21999979019165	1550.57250976563\\
4.22499990463257	1433.76000976563\\
4.23000001907349	1336.09631347656\\
4.2350001335144	1259.69299316406\\
4.23999977111816	1201.23120117188\\
4.24499988555908	1154.40893554688\\
4.25	1111.48400878906\\
4.25500011444092	1061.87927246094\\
4.26000022888184	998.899780273438\\
4.2649998664856	917.233581542969\\
4.26999998092651	816.534057617188\\
4.27500009536743	700.587463378906\\
4.28000020980835	575.401550292969\\
4.28499984741211	447.414672851563\\
4.28999996185303	326.722259521484\\
4.29500007629395	220.499221801758\\
4.30000019073486	133.527725219727\\
4.30499982833862	67.4409255981445\\
4.30999994277954	21.6353149414063\\
4.31500005722046	-3.03699493408203\\
4.32000017166138	-6.1732931137085\\
4.32499980926514	11.4653816223145\\
4.32999992370605	64.4574127197266\\
4.33500003814697	132.344573974609\\
4.34000015258789	210.013793945313\\
4.34499979019165	303.828491210938\\
4.34999990463257	418.5625\\
4.35500001907349	561.49853515625\\
4.3600001335144	741.4443359375\\
4.36499977111816	979.681640625\\
4.36999988555908	1279.25305175781\\
4.375	1587.71643066406\\
4.38000011444092	1831.41076660156\\
4.38500022888184	1972.01135253906\\
4.3899998664856	2006.54956054688\\
4.39499998092651	1999.416015625\\
4.40000009536743	1894.33984375\\
4.40500020980835	1724.40356445313\\
4.40999984741211	1540.0791015625\\
4.41499996185303	1392.77307128906\\
4.42000007629395	1319.63806152344\\
4.42500019073486	1374.51049804688\\
4.42999982833862	1512.88317871094\\
4.43499994277954	1681.17919921875\\
4.44000005722046	1847.21240234375\\
4.44500017166138	1982.56970214844\\
4.44999980926514	2065.51293945313\\
4.45499992370605	2085.64697265625\\
4.46000003814697	2072.6015625\\
4.46500015258789	2007.96936035156\\
4.46999979019165	1888.28161621094\\
4.47499990463257	1735.79040527344\\
4.48000001907349	1575.69018554688\\
4.4850001335144	1429.34313964844\\
4.48999977111816	1310.89013671875\\
4.49499988555908	1224.51611328125\\
4.5	1163.75561523438\\
4.50500011444092	1120.06396484375\\
4.51000022888184	1081.47607421875\\
4.5149998664856	1034.884765625\\
4.51999998092651	969.16259765625\\
4.52500009536743	878.4130859375\\
4.53000020980835	762.383544921875\\
4.53499984741211	625.7373046875\\
4.53999996185303	477.241119384766\\
4.54500007629395	324.041137695313\\
4.55000019073486	185.083740234375\\
4.55499982833862	53.6550178527832\\
4.55999994277954	-7.03214359283447\\
4.56500005722046	-14.9503011703491\\
4.57000017166138	-11.3882932662964\\
4.57499980926514	-6.3803391456604\\
4.57999992370605	-2.54972720146179\\
4.58500003814697	-0.606609225273132\\
4.59000015258789	0.297122091054916\\
4.59499979019165	0.331333577632904\\
4.59999990463257	-0.588315367698669\\
4.60500001907349	-2.66110277175903\\
4.6100001335144	-6.43567800521851\\
4.61499977111816	-12.4045381546021\\
4.61999988555908	261.473419189453\\
4.625	871.7578125\\
4.63000011444092	1571.63732910156\\
4.63500022888184	2125.23901367188\\
4.6399998664856	2469.578125\\
4.64499998092651	2596.66284179688\\
4.65000009536743	2586.68090820313\\
4.65500020980835	2449.81396484375\\
4.65999984741211	2170.86401367188\\
4.66499996185303	1847.26977539063\\
4.67000007629395	1416.19299316406\\
4.67500019073486	1081.14184570313\\
4.67999982833862	896.985778808594\\
4.68499994277954	950.250244140625\\
4.69000005722046	1181.87963867188\\
4.69500017166138	1476.85913085938\\
4.69999980926514	1772.54235839844\\
4.70499992370605	2015.16467285156\\
4.71000003814697	2162.47412109375\\
4.71500015258789	2194.71655273438\\
4.71999979019165	2162.84887695313\\
4.72499990463257	2048.06689453125\\
4.73000001907349	1839.96105957031\\
4.7350001335144	1583.26123046875\\
4.73999977111816	1324.13610839844\\
4.74499988555908	1101.48718261719\\
4.75	938.221740722656\\
4.75500011444092	843.140808105469\\
4.76000022888184	798.834777832031\\
4.7649998664856	794.420715332031\\
4.76999998092651	791.816589355469\\
4.77500009536743	779.90576171875\\
4.78000020980835	735.55859375\\
4.78499984741211	655.114074707031\\
4.78999996185303	534.582885742188\\
4.79500007629395	376.463714599609\\
4.80000019073486	213.592819213867\\
4.80499982833862	40.0585823059082\\
4.80999994277954	-6.20859956741333\\
4.81500005722046	-11.0533847808838\\
4.82000017166138	-4.27250671386719\\
4.82499980926514	3.07049298286438\\
4.82999992370605	7.53238248825073\\
4.83500003814697	8.26139354705811\\
4.84000015258789	6.80038976669312\\
4.84499979019165	2.12628078460693\\
4.84999990463257	-6.10065984725952\\
4.85500001907349	-17.9506359100342\\
4.8600001335144	-32.0628547668457\\
4.86499977111816	-41.4311180114746\\
4.86999988555908	450.8701171875\\
4.875	1130.25280761719\\
4.88000011444092	1807.35095214844\\
4.88500022888184	2321.40063476563\\
4.8899998664856	2616.72045898438\\
4.89499998092651	2693.46484375\\
4.90000009536743	2657.19018554688\\
4.90500020980835	2471.15673828125\\
4.90999984741211	2136.986328125\\
4.91499996185303	1693.60888671875\\
4.92000007629395	1238.32934570313\\
4.92500019073486	760.3173828125\\
4.92999982833862	420.460693359375\\
4.93499994277954	334.994842529297\\
4.94000005722046	565.293212890625\\
4.94500017166138	962.895690917969\\
4.94999980926514	1407.66528320313\\
4.95499992370605	1826.25817871094\\
4.96000003814697	2155.47509765625\\
4.96500015258789	2354.07495117188\\
4.96999979019165	2407.69409179688\\
4.97499990463257	2378.42626953125\\
4.98000001907349	2259.6142578125\\
4.9850001335144	2050.40014648438\\
4.98999977111816	1795.16528320313\\
4.99499988555908	1540.83093261719\\
5	1325.68359375\\
5.00500011444092	1171.3994140625\\
5.01000022888184	1085.14624023438\\
5.0149998664856	1060.49365234375\\
5.01999998092651	1085.37414550781\\
5.02500009536743	1113.82531738281\\
5.03000020980835	1125.3271484375\\
5.03499984741211	1107.97595214844\\
5.03999996185303	1073.01184082031\\
5.04500007629395	1009.13391113281\\
5.05000019073486	913.584594726563\\
5.05499982833862	791.155151367188\\
5.05999994277954	647.886474609375\\
5.06500005722046	492.673553466797\\
5.07000017166138	332.543884277344\\
5.07499980926514	174.079940795898\\
5.07999992370605	24.9060668945313\\
5.08500003814697	-17.5825252532959\\
5.09000015258789	-21.2093524932861\\
5.09499979019165	-18.2311344146729\\
5.09999990463257	-15.3321294784546\\
5.10500001907349	-10.2437267303467\\
5.1100001335144	-6.86786270141602\\
5.11499977111816	-5.15514850616455\\
5.11999988555908	-3.84136819839478\\
5.125	-2.85286855697632\\
5.13000011444092	-2.29046034812927\\
5.13500022888184	450.135314941406\\
5.1399998664856	918.387573242188\\
5.14499998092651	1214.54907226563\\
5.15000009536743	1346.1953125\\
5.15500020980835	1350.89965820313\\
5.15999984741211	1333.38073730469\\
5.16499996185303	1221.24731445313\\
5.17000007629395	1085.49072265625\\
5.17500019073486	972.563293457031\\
5.17999982833862	922.467346191406\\
5.18499994277954	980.071350097656\\
5.19000005722046	1100.48889160156\\
5.19500017166138	1250.44018554688\\
5.19999980926514	1401.8212890625\\
5.20499992370605	1538.06103515625\\
5.21000003814697	1644.72399902344\\
5.21500015258789	1713.78479003906\\
5.21999979019165	1742.71301269531\\
5.22499990463257	1736.49047851563\\
5.23000001907349	1712.95849609375\\
5.2350001335144	1671.58483886719\\
5.23999977111816	1613.55395507813\\
5.24499988555908	1548.81103515625\\
5.25	1487.02819824219\\
5.25500011444092	1433.98278808594\\
5.26000022888184	1394.37719726563\\
5.2649998664856	1369.11560058594\\
5.26999998092651	1363.16040039063\\
5.27500009536743	1367.4765625\\
5.28000020980835	1375.21984863281\\
5.28499984741211	1381.34155273438\\
5.28999996185303	1383.09887695313\\
5.29500007629395	1377.95202636719\\
5.30000019073486	1367.72131347656\\
5.30499982833862	1349.95166015625\\
5.30999994277954	1319.61560058594\\
5.31500005722046	1275.48376464844\\
5.32000017166138	1220.27770996094\\
5.32499980926514	1154.78491210938\\
5.32999992370605	1085.38098144531\\
5.33500003814697	1017.69311523438\\
5.34000015258789	955.251647949219\\
5.34499979019165	901.6083984375\\
5.34999990463257	857.34375\\
5.35500001907349	822.560852050781\\
5.3600001335144	795.067138671875\\
5.36499977111816	771.040100097656\\
5.36999988555908	745.361572265625\\
5.375	713.022094726563\\
5.38000011444092	670.156311035156\\
5.38500022888184	614.942260742188\\
5.3899998664856	547.90576171875\\
5.39499998092651	472.627014160156\\
5.40000009536743	392.214599609375\\
5.40500020980835	313.800354003906\\
5.40999984741211	243.06184387207\\
5.41499996185303	182.243148803711\\
5.42000007629395	132.552597045898\\
5.42500019073486	93.0788726806641\\
5.42999982833862	61.5248947143555\\
5.43499994277954	37.1591339111328\\
5.44000005722046	9.1765079498291\\
5.44500017166138	-4.43849515914917\\
5.44999980926514	-6.08297204971313\\
5.45499992370605	-5.00286912918091\\
5.46000003814697	-3.75373578071594\\
5.46500015258789	-2.83543634414673\\
5.46999979019165	-2.26715731620789\\
5.47499990463257	-1.89846074581146\\
5.48000001907349	-1.66629028320313\\
5.4850001335144	-1.45097613334656\\
5.48999977111816	-1.3216849565506\\
5.49499988555908	-1.20256268978119\\
5.5	-1.10608637332916\\
5.50500011444092	-1.0109087228775\\
5.51000022888184	-0.92598819732666\\
5.5149998664856	-0.863430082798004\\
5.51999998092651	35.5636405944824\\
5.52500009536743	115.872619628906\\
5.53000020980835	173.665725708008\\
5.53499984741211	215.450881958008\\
5.53999996185303	244.215957641602\\
5.54500007629395	262.142852783203\\
5.55000019073486	270.674133300781\\
5.55499982833862	271.339904785156\\
5.55999994277954	267.656951904297\\
5.56500005722046	259.4921875\\
5.57000017166138	245.921813964844\\
5.57499980926514	227.732757568359\\
5.57999992370605	205.277206420898\\
5.58500003814697	179.870590209961\\
5.59000015258789	152.536666870117\\
5.59499979019165	124.816970825195\\
5.59999990463257	97.9535217285156\\
5.60500001907349	73.3461532592773\\
5.6100001335144	52.012378692627\\
5.61499977111816	34.3848114013672\\
5.61999988555908	20.6766033172607\\
5.625	10.7825040817261\\
5.63000011444092	4.15441131591797\\
5.63500022888184	-0.0193019434809685\\
5.6399998664856	-1.47837209701538\\
5.64499998092651	-1.40597116947174\\
5.65000009536743	-1.14036011695862\\
5.65500020980835	-0.891873776912689\\
5.65999984741211	-0.684386551380157\\
5.66499996185303	-0.540932655334473\\
5.67000007629395	0.963796734809875\\
5.67500019073486	38.9207420349121\\
5.67999982833862	14.8279886245728\\
5.68499994277954	14.008770942688\\
5.69000005722046	135.913391113281\\
5.69500017166138	255.249969482422\\
5.69999980926514	348.624816894531\\
5.70499992370605	411.234283447266\\
5.71000003814697	438.578491210938\\
5.71500015258789	437.622894287109\\
5.71999979019165	416.105987548828\\
5.72499990463257	368.83154296875\\
5.73000001907349	305.456665039063\\
5.7350001335144	237.591384887695\\
5.73999977111816	176.541748046875\\
5.74499988555908	129.901641845703\\
5.75	100.868339538574\\
5.75500011444092	87.9627456665039\\
5.76000022888184	89.0044479370117\\
5.7649998664856	96.3343887329102\\
5.76999998092651	103.32950592041\\
5.77500009536743	108.141639709473\\
5.78000020980835	110.90650177002\\
5.78499984741211	112.93928527832\\
5.78999996185303	115.03035736084\\
5.79500007629395	118.053199768066\\
5.80000019073486	122.675422668457\\
5.80499982833862	129.307373046875\\
5.80999994277954	137.369369506836\\
5.81500005722046	146.624572753906\\
5.82000017166138	156.443344116211\\
5.82499980926514	166.555725097656\\
5.82999992370605	176.07373046875\\
5.83500003814697	184.415542602539\\
5.84000015258789	191.361907958984\\
5.84499979019165	195.569442749023\\
5.84999990463257	195.799514770508\\
5.85500001907349	191.066040039063\\
5.8600001335144	181.065032958984\\
5.86499977111816	165.297058105469\\
5.86999988555908	140.552124023438\\
5.875	106.890022277832\\
5.88000011444092	66.403938293457\\
5.88500022888184	25.9378108978271\\
5.8899998664856	-1.38335025310516\\
5.89499998092651	42.1777992248535\\
5.90000009536743	152.01335144043\\
5.90500020980835	392.163909912109\\
5.90999984741211	831.653930664063\\
5.91499996185303	1440.20056152344\\
5.92000007629395	2134.2490234375\\
5.92500019073486	2833.20825195313\\
5.92999982833862	3476.01879882813\\
5.93499994277954	4031.47778320313\\
5.94000005722046	4478.0859375\\
5.94500017166138	4809.08056640625\\
5.94999980926514	5020.9072265625\\
5.95499992370605	5106.9306640625\\
5.96000003814697	5125.49951171875\\
5.96500015258789	5023.0673828125\\
5.96999979019165	4758.32958984375\\
5.97499990463257	4316.36279296875\\
5.98000001907349	3643.30810546875\\
5.9850001335144	2907.12475585938\\
5.98999977111816	2201.24096679688\\
5.99499988555908	1685.91906738281\\
6	1486.66674804688\\
6.00500011444092	1819.05859375\\
6.01000022888184	2436.72924804688\\
6.0149998664856	3188.18872070313\\
6.01999998092651	3954.14892578125\\
6.02500009536743	4624.681640625\\
6.03000020980835	5114.43212890625\\
6.03499984741211	5373.98779296875\\
6.03999996185303	5392.9619140625\\
6.04500007629395	5291.29833984375\\
6.05000019073486	5023.3173828125\\
6.05499982833862	4602.53857421875\\
6.05999994277954	4097.02001953125\\
6.06500005722046	3583.76904296875\\
6.07000017166138	3130.27124023438\\
6.07499980926514	2783.318359375\\
6.07999992370605	2563.0576171875\\
6.08500003814697	2467.31030273438\\
6.09000015258789	2512.75512695313\\
6.09499979019165	2617.57885742188\\
6.09999990463257	2719.26977539063\\
6.10500001907349	2779.86303710938\\
6.1100001335144	2777.22900390625\\
6.11499977111816	2734.35791015625\\
6.11999988555908	2620.84594726563\\
6.125	2426.7041015625\\
6.13000011444092	2162.71240234375\\
6.13500022888184	1851.94641113281\\
6.1399998664856	1524.85485839844\\
6.14499998092651	1210.83459472656\\
6.15000009536743	932.953491210938\\
6.15500020980835	708.694641113281\\
6.15999984741211	542.956420898438\\
6.16499996185303	435.207733154297\\
6.17000007629395	376.076171875\\
6.17500019073486	351.646789550781\\
6.17999982833862	344.508148193359\\
6.18499994277954	336.941345214844\\
6.19000005722046	320.3427734375\\
6.19500017166138	276.196899414063\\
6.19999980926514	209.01252746582\\
6.20499992370605	126.313575744629\\
6.21000003814697	37.1085433959961\\
6.21500015258789	-5.6202540397644\\
6.21999979019165	-11.9200210571289\\
6.22499990463257	-10.1263456344604\\
6.23000001907349	-7.66456317901611\\
6.2350001335144	-5.796875\\
6.23999977111816	-4.46222257614136\\
6.24499988555908	-3.6740984916687\\
6.25	-3.18253636360168\\
6.25500011444092	-2.84193181991577\\
6.26000022888184	-2.58439302444458\\
6.2649998664856	-2.3732738494873\\
6.26999998092651	-2.19183230400085\\
6.27500009536743	-2.02835536003113\\
6.28000020980835	-1.87648117542267\\
6.28499984741211	-1.73449206352234\\
6.28999996185303	-1.60058486461639\\
6.29500007629395	-1.47471439838409\\
6.30000019073486	-1.35857224464417\\
6.30499982833862	-1.24992835521698\\
6.30999994277954	-1.14856553077698\\
6.31500005722046	-1.0553103685379\\
6.32000017166138	-0.968223452568054\\
6.32499980926514	-0.888208091259003\\
6.32999992370605	-0.815133154392242\\
6.33500003814697	-0.746273994445801\\
6.34000015258789	-0.68274849653244\\
6.34499979019165	-0.625207662582397\\
6.34999990463257	-0.573007643222809\\
6.35500001907349	-0.525534331798553\\
6.3600001335144	22.6180610656738\\
6.36499977111816	14.4839687347412\\
6.36999988555908	11.7061576843262\\
6.375	9.39849090576172\\
6.38000011444092	5.7526683807373\\
6.38500022888184	50.2188568115234\\
6.3899998664856	450.659881591797\\
6.39499998092651	770.357299804688\\
6.40000009536743	1030.25610351563\\
6.40500020980835	1235.26794433594\\
6.40999984741211	1393.42114257813\\
6.41499996185303	1507.06066894531\\
6.42000007629395	1586.86804199219\\
6.42500019073486	1643.39990234375\\
6.42999982833862	1685.96166992188\\
6.43499994277954	1725.19616699219\\
6.44000005722046	1768.77172851563\\
6.44500017166138	1822.27062988281\\
6.44999980926514	1888.34704589844\\
6.45499992370605	1967.50390625\\
6.46000003814697	2057.37866210938\\
6.46500015258789	2154.4443359375\\
6.46999979019165	2254.0732421875\\
6.47499990463257	2351.68969726563\\
6.48000001907349	2442.998046875\\
6.4850001335144	2524.81909179688\\
6.48999977111816	2594.93408203125\\
6.49499988555908	2652.60009765625\\
6.5	2697.82763671875\\
6.50500011444092	2731.751953125\\
6.51000022888184	2756.07250976563\\
6.5149998664856	2772.64868164063\\
6.51999998092651	2783.51440429688\\
6.52500009536743	2790.67065429688\\
6.53000020980835	2795.28735351563\\
6.53499984741211	2798.4140625\\
6.53999996185303	2800.46044921875\\
6.54500007629395	2801.619140625\\
6.55000019073486	2801.54345703125\\
6.55499982833862	2799.80688476563\\
6.55999994277954	2796.45581054688\\
6.56500005722046	2791.28881835938\\
6.57000017166138	2783.33227539063\\
6.57499980926514	2772.01416015625\\
6.57999992370605	2756.86962890625\\
6.58500003814697	2737.96020507813\\
6.59000015258789	2715.42651367188\\
6.59499979019165	2689.65551757813\\
6.59999990463257	2661.201171875\\
6.60500001907349	2630.76416015625\\
6.6100001335144	2599.12939453125\\
6.61499977111816	2567.0693359375\\
6.61999988555908	2535.2783203125\\
6.625	2504.44091796875\\
6.63000011444092	2475.05297851563\\
6.63500022888184	2447.419921875\\
6.6399998664856	2422.03784179688\\
6.64499998092651	2398.6904296875\\
6.65000009536743	2377.60571289063\\
6.65500020980835	2358.63696289063\\
6.65999984741211	2341.52197265625\\
6.66499996185303	2326.35864257813\\
6.67000007629395	2313.408203125\\
6.67500019073486	2302.03491210938\\
6.67999982833862	2292.53857421875\\
6.68499994277954	2285.32397460938\\
6.69000005722046	2279.73413085938\\
6.69500017166138	2277.1259765625\\
6.69999980926514	2278.29907226563\\
6.70499992370605	2282.5166015625\\
6.71000003814697	2288.97900390625\\
6.71500015258789	2297.7890625\\
6.71999979019165	2308.7197265625\\
6.72499990463257	2321.5673828125\\
6.73000001907349	2336.20629882813\\
6.7350001335144	2352.34716796875\\
6.73999977111816	2368.94946289063\\
6.74499988555908	2386.01513671875\\
6.75	2402.98803710938\\
6.75500011444092	2419.9189453125\\
6.76000022888184	2437.78344726563\\
6.7649998664856	2454.20971679688\\
6.76999998092651	2469.74267578125\\
6.77500009536743	2483.90698242188\\
6.78000020980835	2497.36499023438\\
6.78499984741211	2510.00219726563\\
6.78999996185303	2521.25805664063\\
6.79500007629395	2531.66845703125\\
6.80000019073486	2541.28100585938\\
6.80499982833862	2550.13940429688\\
6.80999994277954	2558.3154296875\\
6.81500005722046	2565.8193359375\\
6.82000017166138	2572.54736328125\\
6.82499980926514	2578.59252929688\\
6.82999992370605	2583.87084960938\\
6.83500003814697	2589.07348632813\\
6.84000015258789	2591.75244140625\\
6.84499979019165	2592.96923828125\\
6.84999990463257	2591.68310546875\\
6.85500001907349	2587.44506835938\\
6.8600001335144	2580.33569335938\\
6.86499977111816	2567.91430664063\\
6.86999988555908	2551.31298828125\\
6.875	2532.18896484375\\
6.88000011444092	2509.03637695313\\
6.88500022888184	2482.6640625\\
6.8899998664856	2453.33081054688\\
6.89499998092651	2421.43701171875\\
6.90000009536743	2387.31372070313\\
6.90500020980835	2351.13671875\\
6.90999984741211	2312.58935546875\\
6.91499996185303	2272.19360351563\\
6.92000007629395	2229.83959960938\\
6.92500019073486	2185.3271484375\\
6.92999982833862	2138.80102539063\\
6.93499994277954	2089.98852539063\\
6.94000005722046	2039.08288574219\\
6.94500017166138	1986.056640625\\
6.94999980926514	1930.87866210938\\
6.95499992370605	1872.47680664063\\
6.96000003814697	1812.40783691406\\
6.96500015258789	1752.17810058594\\
6.96999979019165	1690.28356933594\\
6.97499990463257	1628.07641601563\\
6.98000001907349	1565.73193359375\\
6.9850001335144	1503.96203613281\\
6.98999977111816	1442.5419921875\\
6.99499988555908	1381.490234375\\
7	1321.84948730469\\
7.00500011444092	1263.11206054688\\
7.01000022888184	1205.37854003906\\
7.0149998664856	1148.61474609375\\
7.01999998092651	1092.87463378906\\
7.02500009536743	1038.32153320313\\
7.03000020980835	984.962951660156\\
7.03499984741211	932.83935546875\\
7.03999996185303	882.077087402344\\
7.04500007629395	832.826232910156\\
7.05000019073486	785.280700683594\\
7.05499982833862	739.690185546875\\
7.05999994277954	696.066284179688\\
7.06500005722046	654.664001464844\\
7.07000017166138	615.730041503906\\
7.07499980926514	579.307556152344\\
7.07999992370605	545.517272949219\\
7.08500003814697	514.378784179688\\
7.09000015258789	485.916412353516\\
7.09499979019165	460.111114501953\\
7.09999990463257	436.827270507813\\
7.10500001907349	415.935211181641\\
7.1100001335144	397.317321777344\\
7.11499977111816	380.975280761719\\
7.11999988555908	366.822692871094\\
7.125	354.707489013672\\
7.13000011444092	344.642456054688\\
7.13500022888184	336.623596191406\\
7.1399998664856	330.628112792969\\
7.14499998092651	326.674011230469\\
7.15000009536743	324.871337890625\\
7.15500020980835	325.606903076172\\
7.15999984741211	328.837280273438\\
7.16499996185303	334.373260498047\\
7.17000007629395	341.167846679688\\
7.17500019073486	349.216613769531\\
7.17999982833862	358.583953857422\\
7.18499994277954	369.116363525391\\
7.19000005722046	380.682556152344\\
7.19500017166138	393.193511962891\\
7.19999980926514	406.57080078125\\
7.20499992370605	420.619720458984\\
7.21000003814697	435.421325683594\\
7.21500015258789	450.863555908203\\
7.21999979019165	466.649261474609\\
7.22499990463257	482.934326171875\\
7.23000001907349	499.585479736328\\
7.2350001335144	516.485778808594\\
7.23999977111816	533.594543457031\\
7.24499988555908	550.755126953125\\
7.25	567.854736328125\\
7.25500011444092	584.864685058594\\
7.26000022888184	601.584777832031\\
7.2649998664856	617.977661132813\\
7.26999998092651	634.032836914063\\
7.27500009536743	649.563171386719\\
7.28000020980835	664.439453125\\
7.28499984741211	678.664733886719\\
7.28999996185303	692.175354003906\\
7.29500007629395	704.987121582031\\
7.30000019073486	716.867736816406\\
7.30499982833862	727.872497558594\\
7.30999994277954	738.004516601563\\
7.31500005722046	747.275268554688\\
7.32000017166138	755.618774414063\\
7.32499980926514	763.045776367188\\
7.32999992370605	769.482849121094\\
7.33500003814697	774.915832519531\\
7.34000015258789	779.370361328125\\
7.34499979019165	782.835144042969\\
7.34999990463257	785.304016113281\\
7.35500001907349	786.791198730469\\
7.3600001335144	787.393249511719\\
7.36499977111816	787.139343261719\\
7.36999988555908	786.05126953125\\
7.375	784.127624511719\\
7.38000011444092	781.397277832031\\
7.38500022888184	777.736328125\\
7.3899998664856	773.05224609375\\
7.39499998092651	767.362609863281\\
7.40000009536743	760.88134765625\\
7.40500020980835	753.605590820313\\
7.40999984741211	745.623779296875\\
7.41499996185303	736.993896484375\\
7.42000007629395	727.796203613281\\
7.42500019073486	718.124572753906\\
7.42999982833862	708.03076171875\\
7.43499994277954	697.524841308594\\
7.44000005722046	686.663208007813\\
7.44500017166138	675.490478515625\\
7.44999980926514	664.056091308594\\
7.45499992370605	652.396728515625\\
7.46000003814697	640.539611816406\\
7.46500015258789	628.540588378906\\
7.46999979019165	616.507629394531\\
7.47499990463257	604.524169921875\\
7.48000001907349	592.613891601563\\
7.4850001335144	580.837768554688\\
7.48999977111816	569.233032226563\\
7.49499988555908	557.839965820313\\
7.5	546.68896484375\\
7.50500011444092	535.8046875\\
7.51000022888184	525.220947265625\\
7.5149998664856	515.014709472656\\
7.51999998092651	505.251281738281\\
7.52500009536743	496.215789794922\\
7.53000020980835	487.40380859375\\
7.53499984741211	479.110778808594\\
7.53999996185303	471.297058105469\\
7.54500007629395	463.990844726563\\
7.55000019073486	457.211151123047\\
7.55499982833862	450.984832763672\\
7.55999994277954	445.333068847656\\
7.56500005722046	440.267486572266\\
7.57000017166138	435.798431396484\\
7.57499980926514	431.941009521484\\
7.57999992370605	428.6923828125\\
7.58500003814697	426.0625\\
7.59000015258789	424.035919189453\\
7.59499979019165	422.611663818359\\
7.59999990463257	421.782958984375\\
7.60500001907349	421.55126953125\\
7.6100001335144	421.967224121094\\
7.61499977111816	423.088531494141\\
7.61999988555908	424.751617431641\\
7.625	426.739135742188\\
7.63000011444092	429.081481933594\\
7.63500022888184	431.834838867188\\
7.6399998664856	434.982238769531\\
7.64499998092651	438.510955810547\\
7.65000009536743	442.403167724609\\
7.65500020980835	446.646759033203\\
7.65999984741211	451.216094970703\\
7.66499996185303	456.084442138672\\
7.67000007629395	461.221099853516\\
7.67500019073486	466.598663330078\\
7.67999982833862	472.182098388672\\
7.68499994277954	477.936614990234\\
7.69000005722046	483.829772949219\\
7.69500017166138	489.829284667969\\
7.69999980926514	495.909088134766\\
7.70499992370605	502.039825439453\\
7.71000003814697	508.195404052734\\
7.71500015258789	514.353576660156\\
7.71999979019165	520.485534667969\\
7.72499990463257	526.573120117188\\
7.73000001907349	532.598510742188\\
7.7350001335144	538.537841796875\\
7.73999977111816	544.363464355469\\
7.74499988555908	550.061828613281\\
7.75	555.60986328125\\
7.75500011444092	561.007995605469\\
7.76000022888184	566.229064941406\\
7.7649998664856	571.236694335938\\
7.76999998092651	576.018188476563\\
7.77500009536743	580.582885742188\\
7.78000020980835	584.927551269531\\
7.78499984741211	589.031616210938\\
7.78999996185303	592.902587890625\\
7.79500007629395	596.543579101563\\
7.80000019073486	599.926208496094\\
7.80499982833862	603.060852050781\\
7.80999994277954	605.949462890625\\
7.81500005722046	608.587951660156\\
7.82000017166138	610.976501464844\\
7.82499980926514	613.117004394531\\
7.82999992370605	615.017028808594\\
7.83500003814697	616.685974121094\\
7.84000015258789	618.117614746094\\
7.84499979019165	619.31884765625\\
7.84999990463257	620.326416015625\\
7.85500001907349	621.1376953125\\
7.8600001335144	621.761291503906\\
7.86499977111816	622.1962890625\\
7.86999988555908	622.450073242188\\
7.875	622.529602050781\\
7.88000011444092	622.47509765625\\
7.88500022888184	622.306823730469\\
7.8899998664856	622.036254882813\\
7.89499998092651	621.669738769531\\
7.90000009536743	621.210754394531\\
7.90500020980835	620.676208496094\\
7.90999984741211	620.075012207031\\
7.91499996185303	619.408935546875\\
7.92000007629395	618.6904296875\\
7.92500019073486	617.933288574219\\
7.92999982833862	617.2197265625\\
7.93499994277954	616.510986328125\\
7.94000005722046	615.805847167969\\
7.94500017166138	615.209411621094\\
7.94999980926514	614.701049804688\\
7.95499992370605	614.266052246094\\
7.96000003814697	613.990051269531\\
7.96500015258789	613.907348632813\\
7.96999979019165	613.982055664063\\
7.97499990463257	614.227111816406\\
7.98000001907349	614.663757324219\\
7.9850001335144	615.269714355469\\
7.98999977111816	616.056213378906\\
7.99499988555908	617.088073730469\\
8	618.337463378906\\
8.00500011444092	619.801147460938\\
8.01000022888184	621.558410644531\\
8.01500034332275	623.611572265625\\
8.02000045776367	625.9755859375\\
8.02499961853027	628.631713867188\\
8.02999973297119	631.5732421875\\
8.03499984741211	634.809326171875\\
8.03999996185303	638.351196289063\\
8.04500007629395	642.187438964844\\
8.05000019073486	646.330261230469\\
8.05500030517578	650.78125\\
8.0600004196167	655.536743164063\\
8.0649995803833	660.603088378906\\
8.06999969482422	665.980529785156\\
8.07499980926514	671.736022949219\\
8.07999992370605	677.855773925781\\
8.08500003814697	684.394165039063\\
8.09000015258789	691.257446289063\\
8.09500026702881	698.4306640625\\
8.10000038146973	705.924499511719\\
8.10499954223633	713.7587890625\\
8.10999965667725	721.956604003906\\
8.11499977111816	730.516357421875\\
8.11999988555908	739.432922363281\\
8.125	748.680114746094\\
8.13000011444092	758.262145996094\\
8.13500022888184	768.178955078125\\
8.14000034332275	778.468078613281\\
8.14500045776367	789.120788574219\\
8.14999961853027	800.164123535156\\
8.15499973297119	811.52197265625\\
8.15999984741211	823.201904296875\\
8.16499996185303	835.202026367188\\
8.17000007629395	847.604064941406\\
8.17500019073486	860.484313964844\\
8.18000030517578	873.8486328125\\
8.1850004196167	887.637573242188\\
8.1899995803833	901.693115234375\\
8.19499969482422	916.028686523438\\
8.19999980926514	930.684631347656\\
8.20499992370605	945.734069824219\\
8.21000003814697	961.167602539063\\
8.21500015258789	976.939086914063\\
8.22000026702881	993.010803222656\\
8.22500038146973	1009.3818359375\\
8.22999954223633	1026.46459960938\\
8.23499965667725	1044.57177734375\\
8.23999977111816	1063.23522949219\\
8.24499988555908	1081.95202636719\\
8.25	1100.91809082031\\
8.25500011444092	1120.45581054688\\
8.26000022888184	1140.50805664063\\
8.26500034332275	1161.05163574219\\
8.27000045776367	1182.23791503906\\
8.27499961853027	1204.22204589844\\
8.27999973297119	1226.64660644531\\
8.28499984741211	1249.5927734375\\
8.28999996185303	1273.38757324219\\
8.29500007629395	1297.83251953125\\
8.30000019073486	1322.73828125\\
8.30500030517578	1347.77648925781\\
8.3100004196167	1373.02282714844\\
8.3149995803833	1402.44494628906\\
8.31999969482422	1431.16809082031\\
8.32499980926514	1460.35974121094\\
8.32999992370605	1490.37744140625\\
8.33500003814697	1521.49487304688\\
8.34000015258789	1553.54833984375\\
8.34500026702881	1586.39868164063\\
8.35000038146973	1620.35949707031\\
8.35499954223633	1654.56359863281\\
8.35999965667725	1691.06823730469\\
8.36499977111816	1729.74438476563\\
8.36999988555908	1768.7294921875\\
8.375	1808.78491210938\\
8.38000011444092	1850.42810058594\\
8.38500022888184	1893.05590820313\\
8.39000034332275	1936.99938964844\\
8.39500045776367	1981.21887207031\\
8.39999961853027	2029.19763183594\\
8.40499973297119	2078.01806640625\\
8.40999984741211	2127.87963867188\\
8.41499996185303	2179.43725585938\\
8.42000007629395	2231.90283203125\\
8.42500019073486	2285.17578125\\
8.43000030517578	2338.45751953125\\
8.4350004196167	2391.93920898438\\
8.4399995803833	2446.05444335938\\
8.44499969482422	2498.35180664063\\
8.44999980926514	2548.89892578125\\
8.45499992370605	2597.57543945313\\
8.46000003814697	2642.76733398438\\
8.46500015258789	2685.708984375\\
8.47000026702881	2725.56372070313\\
8.47500038146973	2762.39721679688\\
8.47999954223633	2796.16357421875\\
8.48499965667725	2826.3984375\\
8.48999977111816	2852.40161132813\\
8.49499988555908	2875.46484375\\
8.5	2894.76904296875\\
8.50500011444092	2910.64819335938\\
8.51000022888184	2922.7646484375\\
8.51500034332275	2932.08544921875\\
8.52000045776367	2939.93383789063\\
8.52499961853027	2942.38647460938\\
8.52999973297119	2943.11938476563\\
8.53499984741211	2940.546875\\
8.53999996185303	2935.28881835938\\
8.54500007629395	2927.46337890625\\
8.55000019073486	2917.7978515625\\
8.55500030517578	2906.44921875\\
8.5600004196167	2893.84252929688\\
8.5649995803833	2880.77197265625\\
8.56999969482422	2867.68041992188\\
8.57499980926514	2854.5400390625\\
8.57999992370605	2841.7978515625\\
8.58500003814697	2829.23022460938\\
8.59000015258789	2817.55493164063\\
8.59500026702881	2806.79443359375\\
8.60000038146973	2796.744140625\\
8.60499954223633	2787.1806640625\\
8.60999965667725	2777.7197265625\\
8.61499977111816	2767.80200195313\\
8.61999988555908	2756.79736328125\\
8.625	2744.02612304688\\
8.63000011444092	2728.6064453125\\
8.63500022888184	2710.22192382813\\
8.64000034332275	2688.11645507813\\
8.64500045776367	2662.21508789063\\
8.64999961853027	2632.33081054688\\
8.65499973297119	2598.7431640625\\
8.65999984741211	2561.86059570313\\
8.66499996185303	2522.41796875\\
8.67000007629395	2480.97119140625\\
8.67500019073486	2438.35180664063\\
8.68000030517578	2395.18359375\\
8.6850004196167	2351.86450195313\\
8.6899995803833	2308.83032226563\\
8.69499969482422	2266.236328125\\
8.69999980926514	2223.96264648438\\
8.70499992370605	2181.892578125\\
8.71000003814697	2139.69311523438\\
8.71500015258789	2097.08129882813\\
8.72000026702881	2053.89721679688\\
8.72500038146973	2009.83142089844\\
8.72999954223633	1964.78527832031\\
8.73499965667725	1918.80090332031\\
8.73999977111816	1871.99755859375\\
8.74499988555908	1824.60302734375\\
8.75	1777.12145996094\\
8.75500011444092	1730.03869628906\\
8.76000022888184	1683.85754394531\\
8.76500034332275	1638.91027832031\\
8.77000045776367	1595.6015625\\
8.77499961853027	1554.12390136719\\
8.77999973297119	1514.59606933594\\
8.78499984741211	1477.01379394531\\
8.78999996185303	1441.27612304688\\
8.79500007629395	1407.22521972656\\
8.80000019073486	1374.67749023438\\
8.80500030517578	1343.44079589844\\
8.8100004196167	1313.34643554688\\
8.8149995803833	1284.27783203125\\
8.81999969482422	1256.20324707031\\
8.82499980926514	1229.17456054688\\
8.82999992370605	1203.31860351563\\
8.83500003814697	1178.80676269531\\
8.84000015258789	1155.85131835938\\
8.84500026702881	1134.67456054688\\
8.85000038146973	1115.45703125\\
8.85499954223633	1098.30676269531\\
8.85999965667725	1083.25756835938\\
8.86499977111816	1070.25634765625\\
8.86999988555908	1059.18225097656\\
8.875	1049.85766601563\\
8.88000011444092	1042.06579589844\\
8.88500022888184	1035.55822753906\\
8.89000034332275	1030.11376953125\\
8.89500045776367	1025.58068847656\\
8.89999961853027	1021.85443115234\\
8.90499973297119	1018.88311767578\\
8.90999984741211	1016.68322753906\\
8.91499996185303	1015.32476806641\\
8.92000007629395	1014.94653320313\\
8.92500019073486	1015.61804199219\\
8.93000030517578	1017.7626953125\\
8.9350004196167	1021.17944335938\\
8.9399995803833	1025.27258300781\\
8.94499969482422	1030.28967285156\\
8.94999980926514	1036.11767578125\\
8.95499992370605	1042.63415527344\\
8.96000003814697	1049.72277832031\\
8.96500015258789	1057.25537109375\\
8.97000026702881	1065.12585449219\\
8.97500038146973	1073.22839355469\\
8.97999954223633	1081.51318359375\\
8.98499965667725	1089.97119140625\\
8.98999977111816	1098.51953125\\
8.99499988555908	1107.17504882813\\
9	1115.955078125\\
9.00500011444092	1124.7666015625\\
9.01000022888184	1133.60388183594\\
9.01500034332275	1142.42700195313\\
9.02000045776367	1151.17956542969\\
9.02499961853027	1159.79541015625\\
9.02999973297119	1168.21130371094\\
9.03499984741211	1176.3671875\\
9.03999996185303	1184.19738769531\\
9.04500007629395	1191.65283203125\\
9.05000019073486	1198.64892578125\\
9.05500030517578	1205.15246582031\\
9.0600004196167	1211.14050292969\\
9.0649995803833	1216.55786132813\\
9.06999969482422	1221.40124511719\\
9.07499980926514	1225.63415527344\\
9.07999992370605	1229.24462890625\\
9.08500003814697	1232.22009277344\\
9.09000015258789	1234.55041503906\\
9.09500026702881	1236.26208496094\\
9.10000038146973	1237.35681152344\\
9.10499954223633	1237.89306640625\\
9.10999965667725	1237.83190917969\\
9.11499977111816	1237.24548339844\\
9.11999988555908	1236.17211914063\\
9.125	1234.53515625\\
9.13000011444092	1232.17028808594\\
9.13500022888184	1229.12658691406\\
9.14000034332275	1225.4306640625\\
9.14500045776367	1221.08569335938\\
9.14999961853027	1216.12548828125\\
9.15499973297119	1210.63806152344\\
9.15999984741211	1204.69055175781\\
9.16499996185303	1198.32629394531\\
9.17000007629395	1191.65600585938\\
9.17500019073486	1184.77526855469\\
9.18000030517578	1177.73706054688\\
9.1850004196167	1170.64294433594\\
9.1899995803833	1163.49462890625\\
9.19499969482422	1156.31787109375\\
9.19999980926514	1149.16882324219\\
9.20499992370605	1142.07531738281\\
9.21000003814697	1135.03918457031\\
9.21500015258789	1127.97717285156\\
9.22000026702881	1120.88012695313\\
9.22500038146973	1113.91076660156\\
9.22999954223633	1107.05920410156\\
9.23499965667725	1100.69226074219\\
9.23999977111816	1095.91906738281\\
9.24499988555908	1094.60278320313\\
9.25	1097.62670898438\\
9.25500011444092	1104.568359375\\
9.26000022888184	1115.7216796875\\
9.26500034332275	1131.45556640625\\
9.27000045776367	1151.42919921875\\
9.27499961853027	1175.97961425781\\
9.27999973297119	1205.39221191406\\
9.28499984741211	1239.73913574219\\
9.28999996185303	1279.67431640625\\
9.29500007629395	1325.73327636719\\
9.30000019073486	1377.96069335938\\
9.30500030517578	1440.962890625\\
9.3100004196167	1510.30871582031\\
9.3149995803833	1588.06750488281\\
9.31999969482422	1675.06701660156\\
9.32499980926514	1771.87536621094\\
9.32999992370605	1880.58068847656\\
9.33500003814697	1999.90124511719\\
9.34000015258789	2129.0458984375\\
9.34500026702881	2276.84448242188\\
9.35000038146973	2435.12133789063\\
9.35499954223633	2603.6865234375\\
9.35999965667725	2791.07202148438\\
9.36499977111816	2988.0556640625\\
9.36999988555908	3193.06127929688\\
9.375	3406.236328125\\
9.38000011444092	3624.83984375\\
9.38500022888184	3846.57421875\\
9.39000034332275	4071.23999023438\\
9.39500045776367	4296.8916015625\\
9.39999961853027	4519.365234375\\
9.40499973297119	4735.26806640625\\
9.40999984741211	4940.13427734375\\
9.41499996185303	5127.48583984375\\
9.42000007629395	5289.02490234375\\
9.42500019073486	5418.17138671875\\
9.43000030517578	5533.6728515625\\
9.4350004196167	5602.91748046875\\
9.4399995803833	5593.4775390625\\
9.44499969482422	5479.75439453125\\
9.44999980926514	5231.9482421875\\
9.45499992370605	4828.23583984375\\
9.46000003814697	4252.15380859375\\
9.46500015258789	3510.3759765625\\
9.47000026702881	2631.912109375\\
9.47500038146973	1677.68884277344\\
9.47999954223633	725.449768066406\\
9.48499965667725	23.6717891693115\\
9.48999977111816	-70.0275421142578\\
9.49499988555908	-64.8256530761719\\
9.5	-44.983154296875\\
9.50500011444092	-30.0631046295166\\
9.51000022888184	-21.454029083252\\
9.51500034332275	-17.9297637939453\\
9.52000045776367	-17.9241333007813\\
9.52499961853027	-20.0537300109863\\
9.52999973297119	-22.7549800872803\\
9.53499984741211	-25.6540184020996\\
9.53999996185303	-28.4601631164551\\
9.54500007629395	-30.7065696716309\\
9.55000019073486	-31.9780120849609\\
9.55500030517578	-25.3975830078125\\
9.5600004196167	-20.7279510498047\\
9.5649995803833	-14.8417739868164\\
9.56999969482422	-9.40780544281006\\
9.57499980926514	-5.67299699783325\\
9.57999992370605	-3.46965885162354\\
9.58500003814697	-2.23505425453186\\
9.59000015258789	-1.43530929088593\\
9.59500026702881	-1.15864086151123\\
9.60000038146973	-0.953585267066956\\
9.60499954223633	-0.914074599742889\\
9.60999965667725	-0.795238256454468\\
9.61499977111816	-0.820303201675415\\
9.61999988555908	-0.858123362064362\\
9.625	-0.861610412597656\\
9.63000011444092	-0.834273099899292\\
9.63500022888184	-0.812437117099762\\
9.64000034332275	-0.774315476417542\\
9.64500045776367	-0.73194420337677\\
9.64999961853027	-0.706719040870667\\
9.65499973297119	-0.676128149032593\\
9.65999984741211	-0.632728159427643\\
9.66499996185303	-0.596938073635101\\
9.67000007629395	-0.561952292919159\\
9.67500019073486	-0.524179935455322\\
9.68000030517578	-0.48878201842308\\
9.6850004196167	-0.455332159996033\\
9.6899995803833	-0.423816978931427\\
9.69499969482422	-0.393512040376663\\
9.69999980926514	-0.36445826292038\\
9.70499992370605	-0.336079567670822\\
9.71000003814697	-0.310604363679886\\
9.71500015258789	-0.288008630275726\\
9.72000026702881	-0.265567719936371\\
9.72500038146973	-0.24341581761837\\
9.72999954223633	-0.223287031054497\\
9.73499965667725	-0.204908564686775\\
9.73999977111816	-0.189003065228462\\
9.74499988555908	-0.175792694091797\\
9.75	-0.163625627756119\\
9.75500011444092	-0.152254283428192\\
9.76000022888184	-0.142059504985809\\
9.76500034332275	-0.131670534610748\\
9.77000045776367	-0.120574973523617\\
9.77499961853027	-0.110842019319534\\
9.77999973297119	-0.10144454240799\\
9.78499984741211	-0.0948875397443771\\
9.78999996185303	-0.0880921483039856\\
9.79500007629395	-0.0819545611739159\\
9.80000019073486	-0.0760852992534637\\
9.80500030517578	-0.070154994726181\\
9.8100004196167	-0.064732626080513\\
9.8149995803833	-0.0596457347273827\\
9.81999969482422	-0.0557322800159454\\
9.82499980926514	-0.0527695529162884\\
9.82999992370605	-0.050804927945137\\
9.83500003814697	-0.0488214269280434\\
9.84000015258789	-0.0458218082785606\\
9.84500026702881	-0.0421944037079811\\
9.85000038146973	-0.0380832441151142\\
9.85499954223633	-0.0353442542254925\\
9.85999965667725	-0.0341643579304218\\
9.86499977111816	-0.0335309505462646\\
9.86999988555908	-0.0330065228044987\\
9.875	-0.031803522258997\\
9.88000011444092	-0.0297254845499992\\
9.88500022888184	-0.0274253915995359\\
9.89000034332275	-0.0248399656265974\\
9.89500045776367	-0.0234864037483931\\
9.89999961853027	-0.0223960150033236\\
9.90499973297119	-0.0229532867670059\\
9.90999984741211	-0.0248130094259977\\
9.91499996185303	-0.0263488572090864\\
9.92000007629395	-0.027897622436285\\
9.92500019073486	-0.0236047524958849\\
9.93000030517578	-0.020297983661294\\
9.9350004196167	-0.0163380037993193\\
9.9399995803833	-0.015731917694211\\
9.94499969482422	-0.0171799622476101\\
9.94999980926514	-0.0172229893505573\\
9.95499992370605	-0.0170933902263641\\
9.96000003814697	-0.0170187670737505\\
9.96500015258789	-0.0169514324516058\\
9.97000026702881	-0.0168926008045673\\
9.97500038146973	-0.0162723194807768\\
9.97999954223633	-0.0168103259056807\\
9.98499965667725	-0.0174331143498421\\
9.98999977111816	-0.0186316594481468\\
9.99499988555908	-0.0188995636999607\\
10	-0.0190520100295544\\
};
\addlegendentry{CF}

\end{axis}

\begin{axis}[%
width=4.521in,
height=1.476in,
at={(0.758in,0.498in)},
scale only axis,
xmin=0,
xmax=10,
xlabel style={font=\color{white!15!black}},
xlabel={Time (s)},
ymin=-7046.70263671875,
ymax=5000,
ylabel style={font=\color{white!15!black}},
ylabel={FY (N)},
axis background/.style={fill=white},
xmajorgrids,
ymajorgrids,
legend style={at={(0.85,1)}, anchor=north east, legend cell align=left, align=left, draw=black}
]
\addplot [color=black, dashed, line width=2.0pt]
  table[row sep=crcr]{%
0.0949999988079071	-33.141674041748\\
0.100000001490116	-28.8333625793457\\
0.104999996721745	-25.1284523010254\\
0.109999999403954	-21.9044704437256\\
0.115000002086163	-19.0829772949219\\
0.119999997317791	413.91015625\\
0.125	648.868957519531\\
0.129999995231628	813.440979003906\\
0.135000005364418	930.81494140625\\
0.140000000596046	1013.95928955078\\
0.144999995827675	1066.2412109375\\
0.150000005960464	1094.09399414063\\
0.155000001192093	1096.58569335938\\
0.159999996423721	1097.78039550781\\
0.165000006556511	1070.97485351563\\
0.170000001788139	1010.70843505859\\
0.174999997019768	919.23681640625\\
0.180000007152557	800.708435058594\\
0.185000002384186	658.2333984375\\
0.189999997615814	498.014709472656\\
0.194999992847443	-794.576416015625\\
0.200000002980232	-2141.81494140625\\
0.204999998211861	-2919.80126953125\\
0.209999993443489	-3054.80712890625\\
0.215000003576279	-2728.19702148438\\
0.219999998807907	-2226.65771484375\\
0.224999994039536	-1480.38037109375\\
0.230000004172325	-582.689147949219\\
0.234999999403954	328.354858398438\\
0.239999994635582	1122.1767578125\\
0.245000004768372	1688.185546875\\
0.25	1941.79638671875\\
0.254999995231628	1757.37451171875\\
0.259999990463257	1371.79626464844\\
0.264999985694885	826.735595703125\\
0.270000010728836	216.59162902832\\
0.275000005960464	-354.055908203125\\
0.280000001192093	-800.110595703125\\
0.284999996423721	-1072.79553222656\\
0.28999999165535	-1153.57543945313\\
0.294999986886978	-1058.72106933594\\
0.300000011920929	-859.800720214844\\
0.305000007152557	-566.993896484375\\
0.310000002384186	-228.05451965332\\
0.314999997615814	103.71630859375\\
0.319999992847443	378.949768066406\\
0.324999988079071	560.99951171875\\
0.330000013113022	658.2861328125\\
0.33500000834465	636.775451660156\\
0.340000003576279	472.018737792969\\
0.344999998807907	246.725006103516\\
0.349999994039536	8.07266712188721\\
0.354999989271164	-204.736022949219\\
0.360000014305115	-362.225982666016\\
0.365000009536743	-456.424957275391\\
0.370000004768372	-498.10400390625\\
0.375	-464.468627929688\\
0.379999995231628	-365.150329589844\\
0.384999990463257	-222.15625\\
0.389999985694885	-58.5236282348633\\
0.395000010728836	93.7428436279297\\
0.400000005960464	213.739776611328\\
0.405000001192093	287.490478515625\\
0.409999996423721	301.021942138672\\
0.41499999165535	261.231353759766\\
0.419999986886978	176.65364074707\\
0.425000011920929	58.7056427001953\\
0.430000007152557	-57.8472747802734\\
0.435000002384186	-153.396789550781\\
0.439999997615814	-225.485443115234\\
0.444999992847443	-259.753295898438\\
0.449999988079071	-253.319564819336\\
0.455000013113022	-209.749664306641\\
0.46000000834465	-139.248977661133\\
0.465000003576279	-53.8835220336914\\
0.469999998807907	29.9722938537598\\
0.474999994039536	100.57218170166\\
0.479999989271164	147.297500610352\\
0.485000014305115	164.394287109375\\
0.490000009536743	151.372421264648\\
0.495000004768372	104.967109680176\\
0.5	43.3970794677734\\
0.504999995231628	-17.7123222351074\\
0.509999990463257	-69.4661865234375\\
0.514999985694885	-104.747077941895\\
0.519999980926514	-120.652374267578\\
0.524999976158142	-114.85311126709\\
0.529999971389771	-93.6966323852539\\
0.535000026226044	-64.1277084350586\\
0.540000021457672	-29.5908908843994\\
0.545000016689301	5.0267972946167\\
0.550000011920929	33.702693939209\\
0.555000007152557	48.9900207519531\\
0.560000002384186	52.2229270935059\\
0.564999997615814	45.8059539794922\\
0.569999992847443	32.3493194580078\\
0.574999988079071	15.1764221191406\\
0.579999983310699	-2.69844722747803\\
0.584999978542328	-18.675802230835\\
0.589999973773956	-30.6484813690186\\
0.595000028610229	-37.8094215393066\\
0.600000023841858	-40.024169921875\\
0.605000019073486	-37.7262229919434\\
0.610000014305115	-32.2730255126953\\
0.615000009536743	-25.0109596252441\\
0.620000004768372	-17.5016212463379\\
0.625	-10.6314992904663\\
0.629999995231628	-5.59968328475952\\
0.634999990463257	-2.50679802894592\\
0.639999985694885	-1.91766166687012\\
0.644999980926514	-3.58746433258057\\
0.649999976158142	-6.96837043762207\\
0.654999971389771	-11.4420251846313\\
0.660000026226044	-16.3197765350342\\
0.665000021457672	-20.9557132720947\\
0.670000016689301	-24.8214435577393\\
0.675000011920929	-27.5049953460693\\
0.680000007152557	-28.8688087463379\\
0.685000002384186	-29.0488948822021\\
0.689999997615814	-28.2933292388916\\
0.694999992847443	-26.9226627349854\\
0.699999988079071	-25.3359184265137\\
0.704999983310699	-24.0574111938477\\
0.709999978542328	-23.2693862915039\\
0.714999973773956	-22.8545989990234\\
0.720000028610229	-23.1009120941162\\
0.725000023841858	-23.9595680236816\\
0.730000019073486	-24.9347248077393\\
0.735000014305115	-26.3030471801758\\
0.740000009536743	-27.5801200866699\\
0.745000004768372	-28.8075351715088\\
0.75	-29.7372226715088\\
0.754999995231628	-30.4086856842041\\
0.759999990463257	-30.68581199646\\
0.764999985694885	-30.2201519012451\\
0.769999980926514	-29.8694744110107\\
0.774999976158142	-29.7026233673096\\
0.779999971389771	-29.5053482055664\\
0.785000026226044	-29.2358818054199\\
0.790000021457672	-28.9236602783203\\
0.795000016689301	-28.613260269165\\
0.800000011920929	-28.3433284759521\\
0.805000007152557	-28.1483516693115\\
0.810000002384186	-28.0261573791504\\
0.814999997615814	-27.9965229034424\\
0.819999992847443	-28.0466384887695\\
0.824999988079071	-28.1585388183594\\
0.829999983310699	-28.1843547821045\\
0.834999978542328	-28.1093082427979\\
0.839999973773956	-27.9409313201904\\
0.845000028610229	-27.6245441436768\\
0.850000023841858	-27.0915985107422\\
0.855000019073486	-26.3708019256592\\
0.860000014305115	-25.6846103668213\\
0.865000009536743	-25.0761756896973\\
0.870000004768372	-24.501501083374\\
0.875	-24.0169811248779\\
0.879999995231628	-23.6149730682373\\
0.884999990463257	-23.2798767089844\\
0.889999985694885	-22.9972991943359\\
0.894999980926514	-22.7584075927734\\
0.899999976158142	-22.5560474395752\\
0.904999971389771	-22.3846988677979\\
0.910000026226044	-22.2330055236816\\
0.915000021457672	-22.0713844299316\\
0.920000016689301	-21.8409805297852\\
0.925000011920929	-21.539400100708\\
0.930000007152557	-21.1465282440186\\
0.935000002384186	-20.6547012329102\\
0.939999997615814	-20.0380535125732\\
0.944999992847443	-19.3633308410645\\
0.949999988079071	-18.8064594268799\\
0.954999983310699	-18.3359260559082\\
0.959999978542328	-17.9598255157471\\
0.964999973773956	-17.6825542449951\\
0.970000028610229	-17.5449028015137\\
0.975000023841858	-17.5036067962646\\
0.980000019073486	-17.5528926849365\\
0.985000014305115	-17.6912956237793\\
0.990000009536743	-17.8082485198975\\
0.995000004768372	-17.9074554443359\\
1	-17.923412322998\\
1.00499999523163	-17.7946453094482\\
1.00999999046326	-17.5087890625\\
1.01499998569489	-17.3588924407959\\
1.01999998092651	-17.2937507629395\\
1.02499997615814	-17.2380428314209\\
1.02999997138977	-17.1880931854248\\
1.0349999666214	-17.1754322052002\\
1.03999996185303	-17.2023010253906\\
1.04499995708466	-17.2609272003174\\
1.04999995231628	-17.3499774932861\\
1.05499994754791	-17.468111038208\\
1.05999994277954	-17.6140251159668\\
1.06500005722046	-17.7896385192871\\
1.07000005245209	-17.9963855743408\\
1.07500004768372	-18.2236366271973\\
1.08000004291534	-18.4502658843994\\
1.08500003814697	-18.6730041503906\\
1.0900000333786	-18.8861083984375\\
1.09500002861023	-19.0804977416992\\
1.10000002384186	-19.248083114624\\
1.10500001907349	-19.3943119049072\\
1.11000001430511	-19.5476303100586\\
1.11500000953674	-19.6903095245361\\
1.12000000476837	-19.8227005004883\\
1.125	-19.9655838012695\\
1.12999999523163	-20.1217460632324\\
1.13499999046326	-20.2920303344727\\
1.13999998569489	-20.4762077331543\\
1.14499998092651	-20.6587734222412\\
1.14999997615814	-20.8492259979248\\
1.15499997138977	-21.0479297637939\\
1.1599999666214	-21.2437877655029\\
1.16499996185303	-21.4330062866211\\
1.16999995708466	-21.6161327362061\\
1.17499995231628	-21.7920207977295\\
1.17999994754791	-21.9682769775391\\
1.18499994277954	-22.1397113800049\\
1.19000005722046	-22.3056468963623\\
1.19500005245209	-22.4487018585205\\
1.20000004768372	-22.5522327423096\\
1.20500004291534	-22.610164642334\\
1.21000003814697	-22.6137218475342\\
1.2150000333786	-22.6319980621338\\
1.22000002861023	-22.6472244262695\\
1.22500002384186	-22.6746311187744\\
1.23000001907349	-22.6985855102539\\
1.23500001430511	-22.6951065063477\\
1.24000000953674	-22.6786422729492\\
1.24500000476837	-22.6459369659424\\
1.25	-22.6564159393311\\
1.25499999523163	-22.703088760376\\
1.25999999046326	-22.8038234710693\\
1.26499998569489	-22.9218559265137\\
1.26999998092651	-22.9904651641846\\
1.27499997615814	-23.0434722900391\\
1.27999997138977	-23.0676765441895\\
1.2849999666214	-23.0348949432373\\
1.28999996185303	-22.941162109375\\
1.29499995708466	-22.7792415618896\\
1.29999995231628	-22.6146869659424\\
1.30499994754791	-22.4206638336182\\
1.30999994277954	-22.2007751464844\\
1.31500005722046	-21.999340057373\\
1.32000005245209	-21.8189754486084\\
1.32500004768372	-21.6695938110352\\
1.33000004291534	-21.5498828887939\\
1.33500003814697	-21.4665908813477\\
1.3400000333786	-21.4316158294678\\
1.34500002861023	-21.4438819885254\\
1.35000002384186	-21.4967365264893\\
1.35500001907349	-21.5937042236328\\
1.36000001430511	-21.7000637054443\\
1.36500000953674	-21.7063980102539\\
1.37000000476837	-21.5867576599121\\
1.375	-21.3061904907227\\
1.37999999523163	-21.0724754333496\\
1.38499999046326	-20.783561706543\\
1.38999998569489	-20.4400157928467\\
1.39499998092651	-20.1483535766602\\
1.39999997615814	-19.8937129974365\\
1.40499997138977	-19.6955108642578\\
1.4099999666214	-19.5666275024414\\
1.41499996185303	-19.5300445556641\\
1.41999995708466	-19.5837326049805\\
1.42499995231628	-19.6387348175049\\
1.42999994754791	-19.7245025634766\\
1.43499994277954	-19.8275737762451\\
1.44000005722046	-19.9283123016357\\
1.44500005245209	-20.0176811218262\\
1.45000004768372	-20.0779514312744\\
1.45500004291534	-20.0889701843262\\
1.46000003814697	-20.0380535125732\\
1.4650000333786	-19.8953227996826\\
1.47000002861023	-19.6274948120117\\
1.47500002384186	-19.209171295166\\
1.48000001907349	-18.8739414215088\\
1.48500001430511	-18.7852439880371\\
1.49000000953674	-19.0702934265137\\
1.49500000476837	-19.0900173187256\\
1.5	-18.9236621856689\\
1.50499999523163	-18.6415214538574\\
1.50999999046326	-18.4585227966309\\
1.51499998569489	-18.3069038391113\\
1.51999998092651	-18.1909198760986\\
1.52499997615814	-18.1231746673584\\
1.52999997138977	-18.1144123077393\\
1.5349999666214	-18.0982799530029\\
1.53999996185303	-18.0920085906982\\
1.54499995708466	-18.1240062713623\\
1.54999995231628	-18.2309341430664\\
1.55499994754791	-18.4104442596436\\
1.55999994277954	-18.629467010498\\
1.56500005722046	-18.9016914367676\\
1.57000005245209	-40.4890670776367\\
1.57500004768372	-34.6272125244141\\
1.58000004291534	-29.474925994873\\
1.58500003814697	-24.426441192627\\
1.5900000333786	-19.9652881622314\\
1.59500002861023	-16.391040802002\\
1.60000002384186	-13.8823680877686\\
1.60500001907349	-12.6513662338257\\
1.61000001430511	-12.9375486373901\\
1.61500000953674	-14.2976408004761\\
1.62000000476837	-16.935432434082\\
1.625	-19.9984531402588\\
1.62999999523163	-22.890043258667\\
1.63499999046326	-24.8079490661621\\
1.63999998569489	-25.2948780059814\\
1.64499998092651	-25.4612426757813\\
1.64999997615814	-25.4065322875977\\
1.65499997138977	-24.9802799224854\\
1.6599999666214	-24.3004760742188\\
1.66499996185303	-23.5399150848389\\
1.66999995708466	-22.4605007171631\\
1.67499995231628	-21.5305824279785\\
1.67999994754791	-21.1439056396484\\
1.68499994277954	-21.0616912841797\\
1.69000005722046	-21.3065490722656\\
1.69500005245209	-21.6650314331055\\
1.70000004768372	-22.1302604675293\\
1.70500004291534	-22.6350116729736\\
1.71000003814697	-23.0472564697266\\
1.7150000333786	-23.469181060791\\
1.72000002861023	-23.8992080688477\\
1.72500002384186	-24.2734375\\
1.73000001907349	-24.6009140014648\\
1.73500001430511	-24.8479328155518\\
1.74000000953674	-25.0198822021484\\
1.74500000476837	-25.1265811920166\\
1.75	-25.1800327301025\\
1.75499999523163	-25.1737060546875\\
1.75999999046326	-25.1464653015137\\
1.76499998569489	-25.0909881591797\\
1.76999998092651	-25.0242729187012\\
1.77499997615814	-25.0105667114258\\
1.77999997138977	-25.0532150268555\\
1.7849999666214	-25.1376075744629\\
1.78999996185303	-25.2770690917969\\
1.79499995708466	-25.4696083068848\\
1.79999995231628	-25.682580947876\\
1.80499994754791	-25.9235248565674\\
1.80999994277954	-26.1641540527344\\
1.81500005722046	-26.3937397003174\\
1.82000005245209	-26.6056118011475\\
1.82500004768372	-26.7995796203613\\
1.83000004291534	-26.9631252288818\\
1.83500003814697	-27.0830764770508\\
1.8400000333786	-27.1640224456787\\
1.84500002861023	-27.2131843566895\\
1.85000002384186	-27.2998008728027\\
1.85500001907349	-27.389139175415\\
1.86000001430511	-27.4829978942871\\
1.86500000953674	-27.5858325958252\\
1.87000000476837	-27.6950283050537\\
1.875	-27.8439273834229\\
1.87999999523163	-28.0583343505859\\
1.88499999046326	-28.3172874450684\\
1.88999998569489	-28.6175174713135\\
1.89499998092651	-28.8499565124512\\
1.89999997615814	-29.0762691497803\\
1.90499997138977	-29.2964534759521\\
1.9099999666214	-29.5011501312256\\
1.91499996185303	-29.6937217712402\\
1.91999995708466	-29.8749866485596\\
1.92499995231628	-30.076810836792\\
1.92999994754791	-30.2966861724854\\
1.93499994277954	-30.5269241333008\\
1.94000005722046	-30.7680931091309\\
1.94500005245209	-31.0203800201416\\
1.95000004768372	-31.2835254669189\\
1.95500004291534	-31.5926399230957\\
1.96000003814697	-32.0025939941406\\
1.9650000333786	-32.5075378417969\\
1.97000002861023	-33.0535430908203\\
1.97500002384186	-33.4464988708496\\
1.98000001907349	-33.735538482666\\
1.98500001430511	-33.908073425293\\
1.99000000953674	-34.0894546508789\\
1.99500000476837	-34.2142028808594\\
2	-34.2878379821777\\
2.00500011444092	-34.492603302002\\
2.00999999046326	-34.795654296875\\
2.01500010490417	-35.2336921691895\\
2.01999998092651	-35.8944320678711\\
2.02500009536743	-36.8135986328125\\
2.02999997138977	-37.9980545043945\\
2.03500008583069	-38.8342704772949\\
2.03999996185303	-39.444393157959\\
2.04500007629395	-39.8244132995605\\
2.04999995231628	-39.9974098205566\\
2.0550000667572	-39.8595123291016\\
2.05999994277954	-39.313549041748\\
2.06500005722046	-38.2978515625\\
2.0699999332428	-36.7001914978027\\
2.07500004768372	-34.4828338623047\\
2.07999992370605	-31.7863216400146\\
2.08500003814697	-28.58962059021\\
2.08999991416931	-25.0545616149902\\
2.09500002861023	-21.2291202545166\\
2.09999990463257	-17.2849884033203\\
2.10500001907349	-13.3060846328735\\
2.10999989509583	-9.31949138641357\\
2.11500000953674	-5.41613721847534\\
2.11999988555908	-2.30170392990112\\
2.125	-0.377265483140945\\
2.13000011444092	0.909476161003113\\
2.13499999046326	1.53501784801483\\
2.14000010490417	1.80871272087097\\
2.14499998092651	2.01908278465271\\
2.15000009536743	2.08483099937439\\
2.15499997138977	2.5770959854126\\
2.16000008583069	4.17579936981201\\
2.16499996185303	5.98323774337769\\
2.17000007629395	7.94416666030884\\
2.17499995231628	10.3355197906494\\
2.1800000667572	12.6183881759644\\
2.18499994277954	14.4298505783081\\
2.19000005722046	15.3468828201294\\
2.1949999332428	15.4128217697144\\
2.20000004768372	14.5844078063965\\
2.20499992370605	12.9421348571777\\
2.21000003814697	10.6996965408325\\
2.21499991416931	8.09325504302979\\
2.22000002861023	5.30000114440918\\
2.22499990463257	2.42935371398926\\
2.23000001907349	-0.023333577439189\\
2.23499989509583	-1.30115234851837\\
2.24000000953674	-2.15743327140808\\
2.24499988555908	-2.55512952804565\\
2.25	-2.43719696998596\\
2.25500011444092	-1.90889120101929\\
2.25999999046326	-1.38289248943329\\
2.26500010490417	-1.46095776557922\\
2.26999998092651	-2.36411118507385\\
2.27500009536743	-4.15334558486938\\
2.27999997138977	-6.47822570800781\\
2.28500008583069	-8.24468898773193\\
2.28999996185303	-9.61035060882568\\
2.29500007629395	-10.9573974609375\\
2.29999995231628	-13.6175870895386\\
2.3050000667572	-15.6351222991943\\
2.30999994277954	-17.5805683135986\\
2.31500005722046	-19.8655033111572\\
2.3199999332428	-22.3405380249023\\
2.32500004768372	-24.7082195281982\\
2.32999992370605	-27.0109348297119\\
2.33500003814697	-29.25803565979\\
2.33999991416931	-29.0647010803223\\
2.34500002861023	-28.1282539367676\\
2.34999990463257	-30.5608692169189\\
2.35500001907349	-32.9446563720703\\
2.35999989509583	-35.9476737976074\\
2.36500000953674	-39.5716323852539\\
2.36999988555908	-43.3145294189453\\
2.375	-47.9076538085938\\
2.38000011444092	-53.1821136474609\\
2.38499999046326	-58.594669342041\\
2.39000010490417	-64.0321350097656\\
2.39499998092651	-69.0490798950195\\
2.40000009536743	-73.1768951416016\\
2.40499997138977	-76.0386047363281\\
2.41000008583069	-77.4756240844727\\
2.41499996185303	-77.4371643066406\\
2.42000007629395	-75.9507369995117\\
2.42499995231628	-73.5322799682617\\
2.4300000667572	-69.936897277832\\
2.43499994277954	-65.5728988647461\\
2.44000005722046	-60.6600685119629\\
2.4449999332428	-54.1868019104004\\
2.45000004768372	-48.6368179321289\\
2.45499992370605	-42.96240234375\\
2.46000003814697	-37.8544311523438\\
2.46499991416931	-33.7108917236328\\
2.47000002861023	-30.354076385498\\
2.47499990463257	-27.6328048706055\\
2.48000001907349	-25.4314136505127\\
2.48499989509583	-23.5243244171143\\
2.49000000953674	-22.0270309448242\\
2.49499988555908	-21.302864074707\\
2.5	-20.8408145904541\\
2.50500011444092	-19.2005004882813\\
2.50999999046326	-15.5915060043335\\
2.51500010490417	-10.5922508239746\\
2.51999998092651	-5.27961444854736\\
2.52500009536743	0.145123422145844\\
2.52999997138977	5.68950748443604\\
2.53500008583069	9.82106304168701\\
2.53999996185303	13.9670114517212\\
2.54500007629395	17.0709896087646\\
2.54999995231628	19.3470878601074\\
2.5550000667572	20.936637878418\\
2.55999994277954	21.9347743988037\\
2.56500005722046	22.8377494812012\\
2.5699999332428	23.6973037719727\\
2.57500004768372	24.7274646759033\\
2.57999992370605	26.046516418457\\
2.58500003814697	27.6793308258057\\
2.58999991416931	30.154411315918\\
2.59500002861023	32.1311531066895\\
2.59999990463257	33.5148086547852\\
2.60500001907349	34.5987815856934\\
2.60999989509583	34.7337341308594\\
2.61500000953674	33.7460136413574\\
2.61999988555908	31.3921642303467\\
2.625	27.6472606658936\\
2.63000011444092	22.7821350097656\\
2.63499999046326	17.4830131530762\\
2.64000010490417	10.6661624908447\\
2.64499998092651	3.31387305259705\\
2.65000009536743	-2.70093870162964\\
2.65499997138977	-8.11319160461426\\
2.66000008583069	-12.8612995147705\\
2.66499996185303	-16.0555038452148\\
2.67000007629395	-18.0651397705078\\
2.67499995231628	-19.8868732452393\\
2.6800000667572	-21.7699966430664\\
2.68499994277954	-23.1477394104004\\
2.69000005722046	-24.9287014007568\\
2.6949999332428	-27.4807891845703\\
2.70000004768372	-30.8611106872559\\
2.70499992370605	-35.1323547363281\\
2.71000003814697	-40.0905838012695\\
2.71499991416931	-45.4922561645508\\
2.72000002861023	-50.9767036437988\\
2.72499990463257	-56.2770042419434\\
2.73000001907349	-61.1386413574219\\
2.73499989509583	-65.3892364501953\\
2.74000000953674	-68.8375015258789\\
2.74499988555908	-71.3099136352539\\
2.75	-73.3836364746094\\
2.75500011444092	-75.4741439819336\\
2.75999999046326	-77.9289703369141\\
2.76500010490417	-80.7639007568359\\
2.76999998092651	-83.8568115234375\\
2.77500009536743	-87.2939529418945\\
2.77999997138977	-91.8607635498047\\
2.78500008583069	-97.6353607177734\\
2.78999996185303	-104.340599060059\\
2.79500007629395	-111.39656829834\\
2.79999995231628	-118.082229614258\\
2.8050000667572	-123.761299133301\\
2.80999994277954	-127.556838989258\\
2.81500005722046	-128.849090576172\\
2.8199999332428	-127.025115966797\\
2.82500004768372	-121.89266204834\\
2.82999992370605	-113.463905334473\\
2.83500003814697	-102.494682312012\\
2.83999991416931	-89.6745986938477\\
2.84500002861023	-75.6550750732422\\
2.84999990463257	-61.3633079528809\\
2.85500001907349	-46.5421104431152\\
2.85999989509583	-31.5965175628662\\
2.86500000953674	-16.2539901733398\\
2.86999988555908	0.35601395368576\\
2.875	18.0743179321289\\
2.88000011444092	38.119213104248\\
2.88499999046326	60.281909942627\\
2.89000010490417	84.07666015625\\
2.89499998092651	106.836471557617\\
2.90000009536743	125.66136932373\\
2.90499997138977	139.031356811523\\
2.91000008583069	145.110610961914\\
2.91499996185303	142.573059082031\\
2.92000007629395	132.199081420898\\
2.92499995231628	115.546890258789\\
2.9300000667572	94.7851638793945\\
2.93499994277954	72.7159805297852\\
2.94000005722046	51.6290321350098\\
2.9449999332428	33.8787040710449\\
2.95000004768372	20.1515865325928\\
2.95499992370605	12.4360628128052\\
2.96000003814697	10.6519498825073\\
2.96499991416931	8.88706016540527\\
2.97000002861023	9.7635326385498\\
2.97499990463257	12.6126375198364\\
2.98000001907349	15.340428352356\\
2.98499989509583	29.7028789520264\\
2.99000000953674	33.9543304443359\\
2.99499988555908	37.1934432983398\\
3	36.5041999816895\\
3.00500011444092	31.4883251190186\\
3.00999999046326	22.4980354309082\\
3.01500010490417	10.5140533447266\\
3.01999998092651	-2.83606266975403\\
3.02500009536743	-15.9833459854126\\
3.02999997138977	-26.9380798339844\\
3.03500008583069	-33.8540306091309\\
3.03999996185303	-36.5384140014648\\
3.04500007629395	-35.1694145202637\\
3.04999995231628	-31.3344631195068\\
3.0550000667572	-26.9632835388184\\
3.05999994277954	-23.0277996063232\\
3.06500005722046	-21.8992958068848\\
3.0699999332428	-25.3890590667725\\
3.07500004768372	-31.7815456390381\\
3.07999992370605	-41.6679420471191\\
3.08500003814697	-53.9573554992676\\
3.08999991416931	-66.6377182006836\\
3.09500002861023	-78.7827224731445\\
3.09999990463257	-89.9570236206055\\
3.10500001907349	-99.3666839599609\\
3.10999989509583	-107.137031555176\\
3.11500000953674	-112.922386169434\\
3.11999988555908	-118.044647216797\\
3.125	-124.410774230957\\
3.13000011444092	-134.723815917969\\
3.13499999046326	-154.303955078125\\
3.14000010490417	-179.570755004883\\
3.14499998092651	-207.647750854492\\
3.15000009536743	-231.897994995117\\
3.15499997138977	-248.705810546875\\
3.16000008583069	-258.405090332031\\
3.16499996185303	-257.99609375\\
3.17000007629395	-245.704849243164\\
3.17499995231628	-221.322875976563\\
3.1800000667572	-187.557357788086\\
3.18499994277954	-147.861740112305\\
3.19000005722046	-104.316955566406\\
3.1949999332428	-55.719783782959\\
3.20000004768372	-5.03936862945557\\
3.20499992370605	43.4184265136719\\
3.21000003814697	89.1335525512695\\
3.21499991416931	128.11213684082\\
3.22000002861023	158.218872070313\\
3.22499990463257	181.0810546875\\
3.23000001907349	195.98323059082\\
3.23499989509583	203.282592773438\\
3.24000000953674	202.63801574707\\
3.24499988555908	193.661819458008\\
3.25	178.574066162109\\
3.25500011444092	158.859451293945\\
3.25999999046326	136.424163818359\\
3.26500010490417	113.851379394531\\
3.26999998092651	93.450080871582\\
3.27500009536743	77.2399520874023\\
3.27999997138977	66.0813522338867\\
3.28500008583069	59.8542289733887\\
3.28999996185303	57.8540229797363\\
3.29500007629395	58.3474388122559\\
3.29999995231628	59.7957191467285\\
3.3050000667572	61.1014137268066\\
3.30999994277954	59.3877716064453\\
3.31500005722046	55.5276260375977\\
3.3199999332428	52.9239234924316\\
3.32500004768372	47.2197418212891\\
3.32999992370605	37.311695098877\\
3.33500003814697	24.3227138519287\\
3.33999991416931	8.90079402923584\\
3.34500002861023	-8.15523719787598\\
3.34999990463257	-23.3672161102295\\
3.35500001907349	-35.2366027832031\\
3.35999989509583	-43.3686599731445\\
3.36500000953674	-47.4905090332031\\
3.36999988555908	-48.6283683776855\\
3.375	-47.7166595458984\\
3.38000011444092	-46.3300018310547\\
3.38499999046326	-46.0927085876465\\
3.39000010490417	-48.0130271911621\\
3.39499998092651	-52.9672775268555\\
3.40000009536743	-61.0824165344238\\
3.40499997138977	-71.513801574707\\
3.41000008583069	-83.6916732788086\\
3.41499996185303	-96.5933532714844\\
3.42000007629395	-109.165550231934\\
3.42499995231628	-120.842353820801\\
3.4300000667572	-131.210464477539\\
3.43499994277954	-140.679016113281\\
3.44000005722046	-150.206985473633\\
3.4449999332428	-160.224731445313\\
3.45000004768372	-171.921752929688\\
3.45499992370605	-187.052719116211\\
3.46000003814697	-212.88948059082\\
3.46499991416931	-248.005355834961\\
3.47000002861023	-283.983703613281\\
3.47499990463257	-316.974731445313\\
3.48000001907349	-342.597290039063\\
3.48499989509583	-356.453460693359\\
3.49000000953674	-346.520721435547\\
3.49499988555908	-327.517578125\\
3.5	-300.649353027344\\
3.50500011444092	-263.818176269531\\
3.50999999046326	-214.845016479492\\
3.51500010490417	-157.856491088867\\
3.51999998092651	-86.3227996826172\\
3.52500009536743	-4.09206914901733\\
3.52999997138977	92.5054321289063\\
3.53500008583069	203.294631958008\\
3.53999996185303	329.162963867188\\
3.54500007629395	458.278839111328\\
3.54999995231628	562.744506835938\\
3.5550000667572	615.170532226563\\
3.55999994277954	612.104736328125\\
3.56500005722046	550.241394042969\\
3.5699999332428	445.744140625\\
3.57500004768372	310.290161132813\\
3.57999992370605	171.81575012207\\
3.58500003814697	37.6119003295898\\
3.58999991416931	-76.3511199951172\\
3.59500002861023	-149.484146118164\\
3.59999990463257	-171.264389038086\\
3.60500001907349	-153.79963684082\\
3.60999989509583	-112.039443969727\\
3.61500000953674	-59.9324836730957\\
3.61999988555908	5.42302751541138\\
3.625	71.0191421508789\\
3.63000011444092	124.754364013672\\
3.63499999046326	162.337524414063\\
3.64000010490417	176.998031616211\\
3.64499998092651	157.43864440918\\
3.65000009536743	103.325477600098\\
3.65499997138977	34.3623580932617\\
3.66000008583069	-33.3363990783691\\
3.66499996185303	-91.3255081176758\\
3.67000007629395	-133.902755737305\\
3.67499995231628	-150.789199829102\\
3.6800000667572	-141.723388671875\\
3.68499994277954	-111.592376708984\\
3.69000005722046	-69.2428817749023\\
3.6949999332428	-25.3620853424072\\
3.70000004768372	8.41794776916504\\
3.70499992370605	26.3535690307617\\
3.71000003814697	21.154109954834\\
3.71499991416931	-3.90348362922668\\
3.72000002861023	-45.5405921936035\\
3.72499990463257	-96.7279815673828\\
3.73000001907349	-149.634719848633\\
3.73499989509583	-196.791137695313\\
3.74000000953674	-234.785781860352\\
3.74499988555908	-261.725128173828\\
3.75	-287.450531005859\\
3.75500011444092	-323.162536621094\\
3.75999999046326	-371.766479492188\\
3.76500010490417	-427.077514648438\\
3.76999998092651	-478.856872558594\\
3.77500009536743	-510.710021972656\\
3.77999997138977	-522.270324707031\\
3.78500008583069	-508.031005859375\\
3.78999996185303	-473.703796386719\\
3.79500007629395	-424.883117675781\\
3.79999995231628	-351.212554931641\\
3.8050000667572	-251.414703369141\\
3.80999994277954	-127.774276733398\\
3.81500005722046	19.0004653930664\\
3.8199999332428	185.218826293945\\
3.82500004768372	365.183013916016\\
3.82999992370605	549.332397460938\\
3.83500003814697	716.183715820313\\
3.83999991416931	825.596984863281\\
3.84500002861023	840.51318359375\\
3.84999990463257	775.572265625\\
3.85500001907349	641.067138671875\\
3.85999989509583	470.282836914063\\
3.86500000953674	278.203094482422\\
3.86999988555908	84.5587310791016\\
3.875	-80.6729354858398\\
3.88000011444092	-194.635696411133\\
3.88499999046326	-230.720352172852\\
3.89000010490417	-206.346649169922\\
3.89499998092651	-141.790252685547\\
3.90000009536743	-68.4819183349609\\
3.90499997138977	21.3784236907959\\
3.91000008583069	108.362815856934\\
3.91499996185303	174.734756469727\\
3.92000007629395	215.205505371094\\
3.92499995231628	225.88444519043\\
3.9300000667572	184.179885864258\\
3.93499994277954	104.634269714355\\
3.94000005722046	13.6596355438232\\
3.9449999332428	-70.8179702758789\\
3.95000004768372	-138.436248779297\\
3.95499992370605	-189.486480712891\\
3.96000003814697	-208.115768432617\\
3.96499991416931	-193.676727294922\\
3.97000002861023	-153.365295410156\\
3.97499990463257	-98.4287719726563\\
3.98000001907349	-43.912956237793\\
3.98499989509583	-2.40236282348633\\
3.99000000953674	17.147970199585\\
3.99499988555908	10.6466693878174\\
4	-21.9032726287842\\
4.00500011444092	-75.4550552368164\\
4.01000022888184	-139.7041015625\\
4.0149998664856	-204.579208374023\\
4.01999998092651	-263.432250976563\\
4.02500009536743	-311.358612060547\\
4.03000020980835	-353.781616210938\\
4.03499984741211	-403.95654296875\\
4.03999996185303	-461.662384033203\\
4.04500007629395	-524.284973144531\\
4.05000019073486	-582.977905273438\\
4.05499982833862	-627.617004394531\\
4.05999994277954	-647.488708496094\\
4.06500005722046	-636.535827636719\\
4.07000017166138	-607.077209472656\\
4.07499980926514	-565.828674316406\\
4.07999992370605	-495.41650390625\\
4.08500003814697	-392.043212890625\\
4.09000015258789	-245.555297851563\\
4.09499979019165	-41.7511100769043\\
4.09999990463257	240.112899780273\\
4.10500001907349	627.363403320313\\
4.1100001335144	1071.61364746094\\
4.11499977111816	1421.80529785156\\
4.11999988555908	1616.6943359375\\
4.125	1585.77075195313\\
4.13000011444092	1284.61120605469\\
4.13500022888184	915.01123046875\\
4.1399998664856	445.380187988281\\
4.14499998092651	-35.9352760314941\\
4.15000009536743	-442.988922119141\\
4.15500020980835	-713.335144042969\\
4.15999984741211	-801.93017578125\\
4.16499996185303	-691.313842773438\\
4.17000007629395	-532.7451171875\\
4.17500019073486	-295.437774658203\\
4.17999982833862	-17.0322761535645\\
4.18499994277954	249.385208129883\\
4.19000005722046	453.352203369141\\
4.19500017166138	563.421142578125\\
4.19999980926514	591.068725585938\\
4.20499992370605	455.322998046875\\
4.21000003814697	236.689758300781\\
4.21500015258789	-0.825659334659576\\
4.21999979019165	-205.869873046875\\
4.22499990463257	-342.539886474609\\
4.23000001907349	-410.896331787109\\
4.2350001335144	-419.355529785156\\
4.23999977111816	-349.367950439453\\
4.24499988555908	-219.492538452148\\
4.25	-71.2606811523438\\
4.25500011444092	67.4064254760742\\
4.26000022888184	164.928985595703\\
4.2649998664856	200.005767822266\\
4.26999998092651	167.364593505859\\
4.27500009536743	73.4910125732422\\
4.28000020980835	-77.0947113037109\\
4.28499984741211	-228.044250488281\\
4.28999996185303	-359.0771484375\\
4.29500007629395	-463.818054199219\\
4.30000019073486	-563.450317382813\\
4.30499982833862	-674.035217285156\\
4.30999994277954	-772.929077148438\\
4.31500005722046	-851.036315917969\\
4.32000017166138	-900.441833496094\\
4.32499980926514	-910.576782226563\\
4.32999992370605	-862.219604492188\\
4.33500003814697	-811.379150390625\\
4.34000015258789	-727.17138671875\\
4.34499979019165	-592.445007324219\\
4.34999990463257	-396.468688964844\\
4.35500001907349	-125.92756652832\\
4.3600001335144	234.03288269043\\
4.36499977111816	705.250305175781\\
4.36999988555908	1211.43322753906\\
4.375	1540.75659179688\\
4.38000011444092	1772.951171875\\
4.38500022888184	1889.708984375\\
4.3899998664856	1908.85693359375\\
4.39499998092651	1466.96887207031\\
4.40000009536743	812.668579101563\\
4.40500020980835	110.304107666016\\
4.40999984741211	-513.858581542969\\
4.41499996185303	-955.041076660156\\
4.42000007629395	-1153.25964355469\\
4.42500019073486	-1035.6337890625\\
4.42999982833862	-810.147155761719\\
4.43499994277954	-481.868865966797\\
4.44000005722046	-88.9337768554688\\
4.44500017166138	294.218017578125\\
4.44999980926514	594.231262207031\\
4.45499992370605	755.303955078125\\
4.46000003814697	805.406677246094\\
4.46500015258789	629.657043457031\\
4.46999979019165	326.471160888672\\
4.47499990463257	-1.24739527702332\\
4.48000001907349	-288.15087890625\\
4.4850001335144	-480.361083984375\\
4.48999977111816	-565.841369628906\\
4.49499988555908	-578.409484863281\\
4.5	-484.347961425781\\
4.50500011444092	-312.919738769531\\
4.51000022888184	-109.327186584473\\
4.5149998664856	79.3351974487305\\
4.51999998092651	213.764450073242\\
4.52500009536743	264.197265625\\
4.53000020980835	227.587463378906\\
4.53499984741211	108.597282409668\\
4.53999996185303	-85.0088043212891\\
4.54500007629395	-277.141174316406\\
4.55000019073486	-446.011962890625\\
4.55499982833862	-589.109497070313\\
4.55999994277954	-663.83056640625\\
4.56500005722046	-721.393737792969\\
4.57000017166138	-787.894226074219\\
4.57499980926514	-851.119934082031\\
4.57999992370605	-902.271545410156\\
4.58500003814697	-932.151672363281\\
4.59000015258789	-944.446655273438\\
4.59499979019165	-951.886596679688\\
4.59999990463257	-939.327514648438\\
4.60500001907349	-887.932983398438\\
4.6100001335144	-785.743286132813\\
4.61499977111816	-612.138305664063\\
4.61999988555908	56.2264556884766\\
4.625	1109.52868652344\\
4.63000011444092	1831.92944335938\\
4.63500022888184	2285.16796875\\
4.6399998664856	2515.66772460938\\
4.64499998092651	2550.96044921875\\
4.65000009536743	2522.20825195313\\
4.65500020980835	2363.78051757813\\
4.65999984741211	2050.43774414063\\
4.66499996185303	227.718978881836\\
4.67000007629395	-1028.29553222656\\
4.67500019073486	-1915.87255859375\\
4.67999982833862	-2315.94165039063\\
4.68499994277954	-2107.05004882813\\
4.69000005722046	-1680.51513671875\\
4.69500017166138	-1026.38232421875\\
4.69999980926514	-237.98210144043\\
4.70499992370605	540.440490722656\\
4.71000003814697	1156.95959472656\\
4.71500015258789	1492.41943359375\\
4.71999979019165	1582.85559082031\\
4.72499990463257	1210.09802246094\\
4.73000001907349	614.822021484375\\
4.7350001335144	-23.5629940032959\\
4.73999977111816	-561.811950683594\\
4.74499988555908	-903.937622070313\\
4.75	-1013.17431640625\\
4.75500011444092	-991.833068847656\\
4.76000022888184	-786.2392578125\\
4.7649998664856	-437.982482910156\\
4.76999998092651	-57.9773216247559\\
4.77500009536743	296.376342773438\\
4.78000020980835	534.357971191406\\
4.78499984741211	584.465454101563\\
4.78999996185303	448.84326171875\\
4.79500007629395	288.412445068359\\
4.80000019073486	-86.6345520019531\\
4.80499982833862	-459.701416015625\\
4.80999994277954	-739.880615234375\\
4.81500005722046	-952.188171386719\\
4.82000017166138	-1113.16455078125\\
4.82499980926514	-1242.39819335938\\
4.82999992370605	-1335.55236816406\\
4.83500003814697	-1385.857421875\\
4.84000015258789	-1407.11865234375\\
4.84499979019165	-1353.48120117188\\
4.84999990463257	-1199.404296875\\
4.85500001907349	-916.32470703125\\
4.8600001335144	-497.339233398438\\
4.86499977111816	-52.648307800293\\
4.86999988555908	651.486328125\\
4.875	1415.68823242188\\
4.88000011444092	2050.56494140625\\
4.88500022888184	2458.91577148438\\
4.8899998664856	2651.51831054688\\
4.89499998092651	2653.72680664063\\
4.90000009536743	2623.88305664063\\
4.90500020980835	2408.93383789063\\
4.90999984741211	2034.99450683594\\
4.91499996185303	1548.38244628906\\
4.92000007629395	-307.983215332031\\
4.92500019073486	-1941.94836425781\\
4.92999982833862	-2928.68774414063\\
4.93499994277954	-3131.19702148438\\
4.94000005722046	-2608.86059570313\\
4.94500017166138	-1930.48034667969\\
4.94999980926514	-1027.59436035156\\
4.95499992370605	-70.6939086914063\\
4.96000003814697	768.1982421875\\
4.96500015258789	1348.07141113281\\
4.96999979019165	1587.86889648438\\
4.97499990463257	1517.74426269531\\
4.98000001907349	1085.95947265625\\
4.9850001335144	483.529754638672\\
4.98999977111816	-141.424453735352\\
4.99499988555908	-673.695983886719\\
5	-1036.53393554688\\
5.00500011444092	-1201.41528320313\\
5.01000022888184	-1210.08898925781\\
5.0149998664856	-1099.19995117188\\
5.01999998092651	-860.869018554688\\
5.02500009536743	-590.957092285156\\
5.03000020980835	-326.133544921875\\
5.03499984741211	-98.9573287963867\\
5.03999996185303	92.0698013305664\\
5.04500007629395	209.618057250977\\
5.05000019073486	249.470962524414\\
5.05499982833862	222.70133972168\\
5.05999994277954	143.57373046875\\
5.06500005722046	32.6786613464355\\
5.07000017166138	-89.5090408325195\\
5.07499980926514	-202.967880249023\\
5.07999992370605	-289.317291259766\\
5.08500003814697	-230.239517211914\\
5.09000015258789	-160.107315063477\\
5.09499979019165	-80.929069519043\\
5.09999990463257	6.32083511352539\\
5.10500001907349	37.8068542480469\\
5.1100001335144	38.9463233947754\\
5.11499977111816	36.6970863342285\\
5.11999988555908	33.2285270690918\\
5.125	29.5429763793945\\
5.13000011444092	26.1026611328125\\
5.13500022888184	882.60595703125\\
5.1399998664856	1230.51818847656\\
5.14499998092651	1388.71801757813\\
5.15000009536743	1407.17895507813\\
5.15500020980835	1335.59240722656\\
5.15999984741211	668.872619628906\\
5.16499996185303	0.882819175720215\\
5.17000007629395	-585.616088867188\\
5.17500019073486	-999.646545410156\\
5.17999982833862	-1186.85888671875\\
5.18499994277954	-1112.22094726563\\
5.19000005722046	-928.546203613281\\
5.19500017166138	-701.182861328125\\
5.19999980926514	-426.076507568359\\
5.20499992370605	-151.577484130859\\
5.21000003814697	79.4617767333984\\
5.21500015258789	236.323272705078\\
5.21999979019165	304.891021728516\\
5.22499990463257	277.255554199219\\
5.23000001907349	183.843765258789\\
5.2350001335144	52.4762268066406\\
5.23999977111816	-96.5844345092773\\
5.24499988555908	-233.987594604492\\
5.25	-339.748016357422\\
5.25500011444092	-404.613891601563\\
5.26000022888184	-439.082885742188\\
5.2649998664856	-429.114624023438\\
5.26999998092651	-361.920501708984\\
5.27500009536743	-257.487701416016\\
5.28000020980835	-127.444244384766\\
5.28499984741211	14.6439619064331\\
5.28999996185303	155.753509521484\\
5.29500007629395	281.590148925781\\
5.30000019073486	384.901123046875\\
5.30499982833862	451.825439453125\\
5.30999994277954	469.458923339844\\
5.31500005722046	443.005004882813\\
5.32000017166138	353.442321777344\\
5.32499980926514	249.330932617188\\
5.32999992370605	145.720001220703\\
5.33500003814697	59.5541648864746\\
5.34000015258789	0.966818869113922\\
5.34499979019165	-35.8803482055664\\
5.34999990463257	-34.1110763549805\\
5.35500001907349	6.73245620727539\\
5.3600001335144	78.4868392944336\\
5.36499977111816	165.603286743164\\
5.36999988555908	250.727966308594\\
5.375	311.969909667969\\
5.38000011444092	338.114868164063\\
5.38500022888184	322.920684814453\\
5.3899998664856	264.493225097656\\
5.39499998092651	170.48112487793\\
5.40000009536743	78.5837020874023\\
5.40500020980835	7.41572999954224\\
5.40999984741211	-44.3492164611816\\
5.41499996185303	-60.9630012512207\\
5.42000007629395	-38.0509223937988\\
5.42500019073486	16.4199829101563\\
5.42999982833862	67.7628860473633\\
5.43499994277954	44.6401138305664\\
5.44000005722046	18.5040950775146\\
5.44500017166138	14.9199657440186\\
5.44999980926514	18.071418762207\\
5.45499992370605	18.0870513916016\\
5.46000003814697	16.7958183288574\\
5.46500015258789	15.0375871658325\\
5.46999979019165	13.3714170455933\\
5.47499990463257	11.6215257644653\\
5.48000001907349	10.5329313278198\\
5.4850001335144	9.04507064819336\\
5.48999977111816	8.0418062210083\\
5.49499988555908	6.93598127365112\\
5.5	6.23961782455444\\
5.50500011444092	5.48604822158813\\
5.51000022888184	4.64939451217651\\
5.5149998664856	4.15526390075684\\
5.51999998092651	71.4525375366211\\
5.52500009536743	168.938201904297\\
5.53000020980835	208.622436523438\\
5.53499984741211	238.995727539063\\
5.53999996185303	260.34326171875\\
5.54500007629395	273.172546386719\\
5.55000019073486	278.219451904297\\
5.55499982833862	277.170562744141\\
5.55999994277954	274.434478759766\\
5.56500005722046	266.887115478516\\
5.57000017166138	252.378326416016\\
5.57499980926514	233.037658691406\\
5.57999992370605	209.561996459961\\
5.58500003814697	182.684127807617\\
5.59000015258789	153.998870849609\\
5.59499979019165	124.882499694824\\
5.59999990463257	97.4388198852539\\
5.60500001907349	72.7767868041992\\
5.6100001335144	51.6067695617676\\
5.61499977111816	34.558235168457\\
5.61999988555908	22.0684108734131\\
5.625	13.0891160964966\\
5.63000011444092	7.25729990005493\\
5.63500022888184	3.93386650085449\\
5.6399998664856	3.42851901054382\\
5.64499998092651	3.68942594528198\\
5.65000009536743	3.75910878181458\\
5.65500020980835	3.66688013076782\\
5.65999984741211	3.34655356407166\\
5.66499996185303	2.94104361534119\\
5.67000007629395	-12.8003454208374\\
5.67500019073486	-492.842163085938\\
5.67999982833862	-638.24853515625\\
5.68499994277954	-653.237487792969\\
5.69000005722046	-382.386383056641\\
5.69500017166138	-211.914169311523\\
5.69999980926514	-43.4344329833984\\
5.70499992370605	81.045295715332\\
5.71000003814697	133.792541503906\\
5.71500015258789	113.098731994629\\
5.71999979019165	27.7015857696533\\
5.72499990463257	-102.438537597656\\
5.73000001907349	-239.444595336914\\
5.7350001335144	-358.425903320313\\
5.73999977111816	-443.58251953125\\
5.74499988555908	-488.916351318359\\
5.75	-500.222717285156\\
5.75500011444092	-491.404449462891\\
5.76000022888184	-474.232299804688\\
5.7649998664856	-469.288177490234\\
5.76999998092651	-482.314239501953\\
5.77500009536743	-513.173278808594\\
5.78000020980835	-554.93896484375\\
5.78499984741211	-602.217346191406\\
5.78999996185303	-652.920227050781\\
5.79500007629395	-703.021362304688\\
5.80000019073486	-752.49462890625\\
5.80499982833862	-801.052062988281\\
5.80999994277954	-849.200317382813\\
5.81500005722046	-897.317626953125\\
5.82000017166138	-947.8701171875\\
5.82499980926514	-1005.24591064453\\
5.82999992370605	-1064.294921875\\
5.83500003814697	-1128.32470703125\\
5.84000015258789	-1194.75476074219\\
5.84499979019165	-1261.22741699219\\
5.84999990463257	-1321.64331054688\\
5.85500001907349	-1385.01770019531\\
5.8600001335144	-1425.55627441406\\
5.86499977111816	-1422.21813964844\\
5.86999988555908	-1352.83068847656\\
5.875	-1186.19885253906\\
5.88000011444092	-902.956726074219\\
5.88500022888184	-495.578918457031\\
5.8899998664856	-27.7191734313965\\
5.89499998092651	91.2663192749023\\
5.90000009536743	191.092132568359\\
5.90500020980835	364.160400390625\\
5.90999984741211	694.6171875\\
5.91499996185303	1132.57653808594\\
5.92000007629395	1599.61962890625\\
5.92500019073486	2049.41088867188\\
5.92999982833862	2462.68481445313\\
5.93499994277954	2834.15234375\\
5.94000005722046	3168.17749023438\\
5.94500017166138	3464.94702148438\\
5.94999980926514	3717.74169921875\\
5.95499992370605	3915.05053710938\\
5.96000003814697	4113.81494140625\\
5.96500015258789	4207.5283203125\\
5.96999979019165	4138.6904296875\\
5.97499990463257	3872.19970703125\\
5.98000001907349	1396.38916015625\\
5.9850001335144	-2610.46875\\
5.98999977111816	-5243.07861328125\\
5.99499988555908	-6731.61083984375\\
6	-7046.70263671875\\
6.00500011444092	-6199.4912109375\\
6.01000022888184	-4900.04736328125\\
6.0149998664856	-3103.20190429688\\
6.01999998092651	-1060.46264648438\\
6.02500009536743	936.017028808594\\
6.03000020980835	2622.46606445313\\
6.03499984741211	3774.42504882813\\
6.03999996185303	4253.25\\
6.04500007629395	3982.3115234375\\
6.05000019073486	3035.64721679688\\
6.05499982833862	1725.02392578125\\
6.05999994277954	318.149169921875\\
6.06500005722046	-939.986877441406\\
6.07000017166138	-1873.50354003906\\
6.07499980926514	-2380.369140625\\
6.07999992370605	-2491.89990234375\\
6.08500003814697	-2351.86669921875\\
6.09000015258789	-1818.60131835938\\
6.09499979019165	-1090.84997558594\\
6.09999990463257	-311.375030517578\\
6.10500001907349	409.509735107422\\
6.1100001335144	976.145080566406\\
6.11499977111816	1381.25646972656\\
6.11999988555908	1526.92785644531\\
6.125	1403.12145996094\\
6.13000011444092	991.791931152344\\
6.13500022888184	462.813995361328\\
6.1399998664856	-62.5901222229004\\
6.14499998092651	-502.467681884766\\
6.15000009536743	-801.450561523438\\
6.15500020980835	-964.837463378906\\
6.15999984741211	-999.069458007813\\
6.16499996185303	-876.264099121094\\
6.17000007629395	-633.478576660156\\
6.17500019073486	-317.621765136719\\
6.17999982833862	14.1557369232178\\
6.18499994277954	310.812042236328\\
6.19000005722046	361.098510742188\\
6.19500017166138	308.857727050781\\
6.19999980926514	229.115844726563\\
6.20499992370605	138.328323364258\\
6.21000003814697	48.085132598877\\
6.21500015258789	36.6758880615234\\
6.21999979019165	44.9567718505859\\
6.22499990463257	45.1729736328125\\
6.23000001907349	42.2373390197754\\
6.2350001335144	38.1079254150391\\
6.23999977111816	33.8572654724121\\
6.24499988555908	29.9042625427246\\
6.25	26.3329792022705\\
6.25500011444092	23.1232662200928\\
6.26000022888184	20.2875499725342\\
6.2649998664856	17.7851276397705\\
6.26999998092651	15.5873908996582\\
6.27500009536743	13.6598320007324\\
6.28000020980835	11.9722728729248\\
6.28499984741211	10.491192817688\\
6.28999996185303	9.1872386932373\\
6.29500007629395	8.04332256317139\\
6.30000019073486	7.04401636123657\\
6.30499982833862	6.16938018798828\\
6.30999994277954	5.39841032028198\\
6.31500005722046	4.72041988372803\\
6.32000017166138	4.12508487701416\\
6.32499980926514	3.6030056476593\\
6.32999992370605	3.14739060401917\\
6.33500003814697	2.75370717048645\\
6.34000015258789	2.41954112052917\\
6.34499979019165	2.1199676990509\\
6.34999990463257	1.84590661525726\\
6.35500001907349	1.59554517269135\\
6.3600001335144	-345.840972900391\\
6.36499977111816	-532.567138671875\\
6.36999988555908	-661.254516601563\\
6.375	-731.143859863281\\
6.38000011444092	-746.409973144531\\
6.38500022888184	-647.648254394531\\
6.3899998664856	-88.1728820800781\\
6.39499998092651	190.226989746094\\
6.40000009536743	415.696014404297\\
6.40500020980835	560.547607421875\\
6.40999984741211	585.068481445313\\
6.41499996185303	526.895568847656\\
6.42000007629395	405.308135986328\\
6.42500019073486	244.687484741211\\
6.42999982833862	71.73583984375\\
6.43499994277954	-91.4276885986328\\
6.44000005722046	-226.362640380859\\
6.44500017166138	-321.460693359375\\
6.44999980926514	-371.24658203125\\
6.45499992370605	-378.211090087891\\
6.46000003814697	-348.487976074219\\
6.46500015258789	-292.109954833984\\
6.46999979019165	-221.591613769531\\
6.47499990463257	-147.885665893555\\
6.48000001907349	-80.6985473632813\\
6.4850001335144	-28.1644287109375\\
6.48999977111816	5.6156268119812\\
6.49499988555908	19.1356449127197\\
6.5	13.6146039962769\\
6.50500011444092	-7.75206232070923\\
6.51000022888184	-40.0940856933594\\
6.5149998664856	-78.2261810302734\\
6.51999998092651	-117.056091308594\\
6.52500009536743	-152.059555053711\\
6.53000020980835	-179.733612060547\\
6.53499984741211	-199.17301940918\\
6.53999996185303	-209.744049072266\\
6.54500007629395	-211.885070800781\\
6.55000019073486	-207.282470703125\\
6.55499982833862	-198.776351928711\\
6.55999994277954	-189.175384521484\\
6.56500005722046	-177.489166259766\\
6.57000017166138	-165.384017944336\\
6.57499980926514	-153.422714233398\\
6.57999992370605	-146.969848632813\\
6.58500003814697	-143.73600769043\\
6.59000015258789	-145.021270751953\\
6.59499979019165	-151.072570800781\\
6.59999990463257	-160.485275268555\\
6.60500001907349	-172.136672973633\\
6.6100001335144	-184.934844970703\\
6.61499977111816	-198.237991333008\\
6.61999988555908	-209.83708190918\\
6.625	-219.089324951172\\
6.63000011444092	-225.382522583008\\
6.63500022888184	-228.703842163086\\
6.6399998664856	-229.792709350586\\
6.64499998092651	-228.590698242188\\
6.65000009536743	-225.136123657227\\
6.65500020980835	-221.114303588867\\
6.65999984741211	-216.979553222656\\
6.66499996185303	-212.829574584961\\
6.67000007629395	-210.177764892578\\
6.67500019073486	-208.960113525391\\
6.67999982833862	-208.864227294922\\
6.68499994277954	-207.918792724609\\
6.69000005722046	-205.977676391602\\
6.69500017166138	-202.22216796875\\
6.69999980926514	-199.696594238281\\
6.70499992370605	-195.143493652344\\
6.71000003814697	-188.406402587891\\
6.71500015258789	-177.500869750977\\
6.71999979019165	-163.05305480957\\
6.72499990463257	-145.629898071289\\
6.73000001907349	-125.034942626953\\
6.7350001335144	-101.444602966309\\
6.73999977111816	-77.6052627563477\\
6.74499988555908	-53.7072944641113\\
6.75	-31.3627376556396\\
6.75500011444092	-9.7929744720459\\
6.76000022888184	14.7198009490967\\
6.7649998664856	34.604549407959\\
6.76999998092651	51.6877479553223\\
6.77500009536743	65.1862564086914\\
6.78000020980835	76.4021759033203\\
6.78499984741211	85.4658432006836\\
6.78999996185303	91.4483261108398\\
6.79500007629395	95.8852386474609\\
6.80000019073486	99.123176574707\\
6.80499982833862	101.655784606934\\
6.80999994277954	103.562812805176\\
6.81500005722046	105.713111877441\\
6.82000017166138	107.558670043945\\
6.82499980926514	109.772125244141\\
6.82999992370605	112.389366149902\\
6.83500003814697	116.563034057617\\
6.84000015258789	118.71452331543\\
6.84499979019165	120.154144287109\\
6.84999990463257	119.644142150879\\
6.85500001907349	116.827072143555\\
6.8600001335144	112.05736541748\\
6.86499977111816	102.655731201172\\
6.86999988555908	90.9891128540039\\
6.875	79.809928894043\\
6.88000011444092	66.9432220458984\\
6.88500022888184	54.4870872497559\\
6.8899998664856	43.2932014465332\\
6.89499998092651	34.1719284057617\\
6.90000009536743	27.6084175109863\\
6.90500020980835	23.9889755249023\\
6.90999984741211	22.4560375213623\\
6.91499996185303	23.3951759338379\\
6.92000007629395	26.081506729126\\
6.92500019073486	29.6296443939209\\
6.92999982833862	33.6305694580078\\
6.93499994277954	36.8178215026855\\
6.94000005722046	39.116527557373\\
6.94500017166138	40.0322074890137\\
6.94999980926514	39.2704429626465\\
6.95499992370605	35.0475769042969\\
6.96000003814697	30.1206474304199\\
6.96500015258789	26.3864307403564\\
6.96999979019165	20.6531391143799\\
6.97499990463257	16.0221004486084\\
6.98000001907349	12.0261011123657\\
6.9850001335144	10.0142908096313\\
6.98999977111816	9.14643001556396\\
6.99499988555908	8.97544002532959\\
7	11.1390600204468\\
7.00500011444092	13.8087329864502\\
7.01000022888184	16.9039154052734\\
7.0149998664856	19.8044834136963\\
7.01999998092651	22.1892757415771\\
7.02500009536743	24.0576667785645\\
7.03000020980835	25.1278648376465\\
7.03499984741211	25.2621021270752\\
7.03999996185303	24.5041942596436\\
7.04500007629395	22.9957828521729\\
7.05000019073486	20.9736251831055\\
7.05499982833862	18.8148040771484\\
7.05999994277954	16.4768943786621\\
7.06500005722046	14.479302406311\\
7.07000017166138	13.1599636077881\\
7.07499980926514	12.4903316497803\\
7.07999992370605	12.5126647949219\\
7.08500003814697	13.2257404327393\\
7.09000015258789	14.5206317901611\\
7.09499979019165	16.240327835083\\
7.09999990463257	17.9943809509277\\
7.10500001907349	19.5235691070557\\
7.1100001335144	20.5837306976318\\
7.11499977111816	21.2849578857422\\
7.11999988555908	21.3311157226563\\
7.125	20.4917106628418\\
7.13000011444092	19.2496929168701\\
7.13500022888184	17.6163921356201\\
7.1399998664856	15.6597852706909\\
7.14499998092651	13.7797765731812\\
7.15000009536743	12.1560153961182\\
7.15500020980835	10.8950996398926\\
7.15999984741211	10.7212629318237\\
7.16499996185303	11.7809476852417\\
7.17000007629395	12.1793489456177\\
7.17500019073486	12.5818862915039\\
7.17999982833862	13.1091346740723\\
7.18499994277954	13.4296522140503\\
7.19000005722046	13.4455966949463\\
7.19500017166138	13.1266355514526\\
7.19999980926514	12.5054054260254\\
7.20499992370605	11.5502462387085\\
7.21000003814697	10.4938650131226\\
7.21500015258789	9.41982269287109\\
7.21999979019165	8.20673751831055\\
7.22499990463257	7.19475173950195\\
7.23000001907349	6.29945230484009\\
7.2350001335144	5.52041578292847\\
7.23999977111816	4.91290187835693\\
7.24499988555908	4.41574716567993\\
7.25	4.01081466674805\\
7.25500011444092	3.71489000320435\\
7.26000022888184	3.44122338294983\\
7.2649998664856	3.18549871444702\\
7.26999998092651	2.93518400192261\\
7.27500009536743	2.5772876739502\\
7.28000020980835	2.14700388908386\\
7.28499984741211	1.64481925964355\\
7.28999996185303	1.12185406684875\\
7.29500007629395	0.561009168624878\\
7.30000019073486	-0.0842484086751938\\
7.30499982833862	-0.790284812450409\\
7.30999994277954	-1.41662979125977\\
7.31500005722046	-2.00894069671631\\
7.32000017166138	-2.54687571525574\\
7.32499980926514	-3.03836393356323\\
7.32999992370605	-3.49335646629333\\
7.33500003814697	-3.90855169296265\\
7.34000015258789	-4.24992513656616\\
7.34499979019165	-4.52160310745239\\
7.34999990463257	-4.75955152511597\\
7.35500001907349	-4.95421314239502\\
7.3600001335144	-5.07625198364258\\
7.36499977111816	-5.12549829483032\\
7.36999988555908	-5.15295219421387\\
7.375	-5.25011348724365\\
7.38000011444092	-5.41589403152466\\
7.38500022888184	-5.64813852310181\\
7.3899998664856	-5.95583343505859\\
7.39499998092651	-6.34870195388794\\
7.40000009536743	-6.65633678436279\\
7.40500020980835	-6.90189218521118\\
7.40999984741211	-7.10528087615967\\
7.41499996185303	-7.25021266937256\\
7.42000007629395	-7.32609033584595\\
7.42500019073486	-7.33009672164917\\
7.42999982833862	-7.26345872879028\\
7.43499994277954	-7.16963005065918\\
7.44000005722046	-7.04589986801147\\
7.44500017166138	-6.89150190353394\\
7.44999980926514	-6.7167272567749\\
7.45499992370605	-6.57703304290771\\
7.46000003814697	-6.53744745254517\\
7.46500015258789	-6.58765697479248\\
7.46999979019165	-6.60759449005127\\
7.47499990463257	-6.58458948135376\\
7.48000001907349	-6.58674764633179\\
7.4850001335144	-6.58114957809448\\
7.48999977111816	-6.56473922729492\\
7.49499988555908	-6.53807401657104\\
7.5	-6.49917078018188\\
7.50500011444092	-6.4446849822998\\
7.51000022888184	-6.37052631378174\\
7.5149998664856	-6.25269174575806\\
7.51999998092651	-6.05895948410034\\
7.52500009536743	-5.47326421737671\\
7.53000020980835	-5.41491222381592\\
7.53499984741211	-5.25583124160767\\
7.53999996185303	-5.11094379425049\\
7.54500007629395	-5.08638525009155\\
7.55000019073486	-5.1121506690979\\
7.55499982833862	-5.15868282318115\\
7.55999994277954	-5.23280715942383\\
7.56500005722046	-5.3111777305603\\
7.57000017166138	-5.36146831512451\\
7.57499980926514	-5.37182712554932\\
7.57999992370605	-5.34323644638062\\
7.58500003814697	-5.25691556930542\\
7.59000015258789	-5.14477586746216\\
7.59499979019165	-5.00183534622192\\
7.59999990463257	-4.8235297203064\\
7.60500001907349	-4.60937404632568\\
7.6100001335144	-4.3894305229187\\
7.61499977111816	-4.19581127166748\\
7.61999988555908	-4.10847806930542\\
7.625	-4.21373081207275\\
7.63000011444092	-4.40327024459839\\
7.63500022888184	-4.53294801712036\\
7.6399998664856	-4.6763596534729\\
7.64499998092651	-4.83396816253662\\
7.65000009536743	-4.95516347885132\\
7.65500020980835	-5.02973699569702\\
7.65999984741211	-5.08377981185913\\
7.66499996185303	-5.10031938552856\\
7.67000007629395	-5.05627202987671\\
7.67500019073486	-4.97310543060303\\
7.67999982833862	-4.9008207321167\\
7.68499994277954	-4.86254501342773\\
7.69000005722046	-4.88300228118896\\
7.69500017166138	-4.89157390594482\\
7.69999980926514	-4.89005851745605\\
7.70499992370605	-4.87265682220459\\
7.71000003814697	-4.8677396774292\\
7.71500015258789	-4.96976041793823\\
7.71999979019165	-5.11190271377563\\
7.72499990463257	-5.22351169586182\\
7.73000001907349	-5.28626346588135\\
7.7350001335144	-5.27791023254395\\
7.73999977111816	-5.35471153259277\\
7.74499988555908	-5.43653106689453\\
7.75	-5.41851806640625\\
7.75500011444092	-5.382004737854\\
7.76000022888184	-5.380615234375\\
7.7649998664856	-5.36254072189331\\
7.76999998092651	-5.31805372238159\\
7.77500009536743	-5.2538013458252\\
7.78000020980835	-5.1765718460083\\
7.78499984741211	-5.14034461975098\\
7.78999996185303	-5.11216878890991\\
7.79500007629395	-5.09202098846436\\
7.80000019073486	-5.05981063842773\\
7.80499982833862	-5.02317047119141\\
7.80999994277954	-4.98372030258179\\
7.81500005722046	-4.9310998916626\\
7.82000017166138	-4.86403608322144\\
7.82499980926514	-4.78622913360596\\
7.82999992370605	-4.70170402526855\\
7.83500003814697	-4.61501884460449\\
7.84000015258789	-4.52248096466064\\
7.84499979019165	-4.42237901687622\\
7.84999990463257	-4.30564069747925\\
7.85500001907349	-4.17391443252563\\
7.8600001335144	-4.02514410018921\\
7.86499977111816	-3.87960195541382\\
7.86999988555908	-3.73131847381592\\
7.875	-3.58237552642822\\
7.88000011444092	-3.42610001564026\\
7.88500022888184	-3.26206994056702\\
7.8899998664856	-3.09193849563599\\
7.89499998092651	-2.91934084892273\\
7.90000009536743	-2.74859404563904\\
7.90500020980835	-2.5790331363678\\
7.90999984741211	-2.41009902954102\\
7.91499996185303	-2.23644423484802\\
7.92000007629395	-2.06071329116821\\
7.92500019073486	-1.88241994380951\\
7.92999982833862	-1.6972348690033\\
7.93499994277954	-1.50758302211761\\
7.94000005722046	-1.31346523761749\\
7.94500017166138	-1.10704600811005\\
7.94999980926514	-0.890502989292145\\
7.95499992370605	-0.664933383464813\\
7.96000003814697	-0.40424644947052\\
7.96500015258789	-0.100351698696613\\
7.96999979019165	0.234653696417809\\
7.97499990463257	0.562553226947784\\
7.98000001907349	0.822800695896149\\
7.9850001335144	1.05937302112579\\
7.98999977111816	1.27962338924408\\
7.99499988555908	1.53161489963531\\
8	1.78755855560303\\
8.00500011444092	2.04745030403137\\
8.01000022888184	2.34050798416138\\
8.01500034332275	2.6645495891571\\
8.02000045776367	3.02423524856567\\
8.02499961853027	3.38044285774231\\
8.02999973297119	3.71787548065186\\
8.03499984741211	4.04132127761841\\
8.03999996185303	4.35305643081665\\
8.04500007629395	4.66073703765869\\
8.05000019073486	4.95464420318604\\
8.05500030517578	5.23456048965454\\
8.0600004196167	5.50932884216309\\
8.0649995803833	5.77410984039307\\
8.06999969482422	6.02944183349609\\
8.07499980926514	6.34252691268921\\
8.07999992370605	6.70016098022461\\
8.08500003814697	7.11516714096069\\
8.09000015258789	7.51139354705811\\
8.09500026702881	7.88086318969727\\
8.10000038146973	8.22879219055176\\
8.10499954223633	8.58402633666992\\
8.10999965667725	8.97712421417236\\
8.11499977111816	9.39362525939941\\
8.11999988555908	9.82612419128418\\
8.125	10.2419996261597\\
8.13000011444092	10.6530914306641\\
8.13500022888184	11.0589618682861\\
8.14000034332275	11.5089817047119\\
8.14500045776367	11.9884815216064\\
8.14999961853027	12.5040273666382\\
8.15499973297119	12.9953212738037\\
8.15999984741211	13.4662637710571\\
8.16499996185303	13.914608001709\\
8.17000007629395	14.4421434402466\\
8.17500019073486	15.1148738861084\\
8.18000030517578	15.9221286773682\\
8.1850004196167	16.7773838043213\\
8.1899995803833	17.4781551361084\\
8.19499969482422	18.0882205963135\\
8.19999980926514	18.6498012542725\\
8.20499992370605	19.2395801544189\\
8.21000003814697	19.8141345977783\\
8.21500015258789	20.3453845977783\\
8.22000026702881	20.8064994812012\\
8.22500038146973	21.1999206542969\\
8.22999954223633	22.0741443634033\\
8.23499965667725	23.7526416778564\\
8.23999977111816	25.3780040740967\\
8.24499988555908	26.2755012512207\\
8.25	26.9983959197998\\
8.25500011444092	27.9583644866943\\
8.26000022888184	28.9131774902344\\
8.26500034332275	29.8799076080322\\
8.27000045776367	31.062198638916\\
8.27499961853027	32.4399452209473\\
8.27999973297119	33.6002616882324\\
8.28499984741211	34.6970825195313\\
8.28999996185303	36.1492042541504\\
8.29500007629395	37.6306343078613\\
8.30000019073486	38.9415817260742\\
8.30500030517578	39.6685371398926\\
8.3100004196167	39.9918441772461\\
8.3149995803833	45.2061386108398\\
8.31999969482422	46.7296600341797\\
8.32499980926514	48.2684440612793\\
8.32999992370605	49.9916305541992\\
8.33500003814697	52.084774017334\\
8.34000015258789	54.1794815063477\\
8.34500026702881	56.0128974914551\\
8.35000038146973	58.0579223632813\\
8.35499954223633	59.0440864562988\\
8.35999965667725	62.1193428039551\\
8.36499977111816	65.9235763549805\\
8.36999988555908	68.0491561889648\\
8.375	70.4265823364258\\
8.38000011444092	73.5086135864258\\
8.38500022888184	76.1086044311523\\
8.39000034332275	78.8328323364258\\
8.39500045776367	80.2567138671875\\
8.39999961853027	85.3933181762695\\
8.40499973297119	88.493049621582\\
8.40999984741211	91.3188934326172\\
8.41499996185303	95.0270843505859\\
8.42000007629395	98.3536529541016\\
8.42500019073486	101.564125061035\\
8.43000030517578	103.776725769043\\
8.4350004196167	105.795837402344\\
8.4399995803833	108.31729888916\\
8.44499969482422	108.053565979004\\
8.44999980926514	106.251182556152\\
8.45499992370605	103.097480773926\\
8.46000003814697	96.7653350830078\\
8.46500015258789	89.703857421875\\
8.47000026702881	81.0935897827148\\
8.47500038146973	71.3432464599609\\
8.47999954223633	60.9078941345215\\
8.48499965667725	48.6009178161621\\
8.48999977111816	33.4898796081543\\
8.49499988555908	19.3908309936523\\
8.5	4.15950965881348\\
8.50500011444092	-11.3752956390381\\
8.51000022888184	-28.0359935760498\\
8.51500034332275	-44.0629043579102\\
8.52000045776367	-58.1766624450684\\
8.52499961853027	-79.3586273193359\\
8.52999973297119	-97.0559616088867\\
8.53499984741211	-116.164741516113\\
8.53999996185303	-135.977035522461\\
8.54500007629395	-155.694305419922\\
8.55000019073486	-175.545791625977\\
8.55500030517578	-194.905090332031\\
8.5600004196167	-214.143203735352\\
8.5649995803833	-231.646530151367\\
8.56999969482422	-247.250015258789\\
8.57499980926514	-261.409545898438\\
8.57999992370605	-273.118530273438\\
8.58500003814697	-282.873931884766\\
8.59000015258789	-290.204620361328\\
8.59500026702881	-292.580413818359\\
8.60000038146973	-289.0087890625\\
8.60499954223633	-280.016510009766\\
8.60999965667725	-265.677429199219\\
8.61499977111816	-247.229232788086\\
8.61999988555908	-226.760360717773\\
8.625	-205.769973754883\\
8.63000011444092	-186.412475585938\\
8.63500022888184	-169.631744384766\\
8.64000034332275	-157.523086547852\\
8.64500045776367	-149.776306152344\\
8.64999961853027	-147.754135131836\\
8.65499973297119	-148.332901000977\\
8.65999984741211	-157.219512939453\\
8.66499996185303	-165.115005493164\\
8.67000007629395	-173.396850585938\\
8.67500019073486	-179.758209228516\\
8.68000030517578	-184.554626464844\\
8.6850004196167	-185.89143371582\\
8.6899995803833	-182.955764770508\\
8.69499969482422	-176.231567382813\\
8.69999980926514	-166.661346435547\\
8.70499992370605	-155.292373657227\\
8.71000003814697	-142.510498046875\\
8.71500015258789	-128.996627807617\\
8.72000026702881	-117.513931274414\\
8.72500038146973	-108.247352600098\\
8.72999954223633	-101.614318847656\\
8.73499965667725	-98.2338180541992\\
8.73999977111816	-98.3633270263672\\
8.74499988555908	-101.188346862793\\
8.75	-105.484397888184\\
8.75500011444092	-108.607299804688\\
8.76000022888184	-112.331069946289\\
8.76500034332275	-114.449272155762\\
8.77000045776367	-114.900581359863\\
8.77499961853027	-113.167655944824\\
8.77999973297119	-109.604354858398\\
8.78499984741211	-104.178756713867\\
8.78999996185303	-97.4209899902344\\
8.79500007629395	-90.0854110717773\\
8.80000019073486	-82.9222564697266\\
8.80500030517578	-76.5760116577148\\
8.8100004196167	-71.6017379760742\\
8.8149995803833	-68.4022674560547\\
8.81999969482422	-67.0009841918945\\
8.82499980926514	-67.3009948730469\\
8.82999992370605	-68.9883651733398\\
8.83500003814697	-71.4097213745117\\
8.84000015258789	-73.9347763061523\\
8.84500026702881	-76.0699615478516\\
8.85000038146973	-77.1208038330078\\
8.85499954223633	-77.0195159912109\\
8.85999965667725	-75.6099243164063\\
8.86499977111816	-72.975700378418\\
8.86999988555908	-69.3837509155273\\
8.875	-65.3179779052734\\
8.88000011444092	-61.3104591369629\\
8.88500022888184	-57.7621116638184\\
8.89000034332275	-55.1126022338867\\
8.89500045776367	-53.5396270751953\\
8.89999961853027	-53.1672325134277\\
8.90499973297119	-54.0738792419434\\
8.90999984741211	-55.6533088684082\\
8.91499996185303	-57.0479164123535\\
8.92000007629395	-58.4822845458984\\
8.92500019073486	-59.7148094177246\\
8.93000030517578	-60.1747093200684\\
8.9350004196167	-59.7632598876953\\
8.9399995803833	-58.3766975402832\\
8.94499969482422	-56.3422889709473\\
8.94999980926514	-53.9200325012207\\
8.95499992370605	-51.4035720825195\\
8.96000003814697	-48.9277534484863\\
8.96500015258789	-46.7328643798828\\
8.97000026702881	-45.1579170227051\\
8.97500038146973	-43.9430122375488\\
8.97999954223633	-42.9765548706055\\
8.98499965667725	-42.956714630127\\
8.98999977111816	-42.8509178161621\\
8.99499988555908	-42.2469291687012\\
9	-42.0877304077148\\
9.00500011444092	-41.8787384033203\\
9.01000022888184	-41.523078918457\\
9.01500034332275	-40.9767990112305\\
9.02000045776367	-40.2705039978027\\
9.02499961853027	-39.4039535522461\\
9.02999973297119	-38.4101219177246\\
9.03499984741211	-37.3646354675293\\
9.03999996185303	-36.3136711120605\\
9.04500007629395	-35.2923011779785\\
9.05000019073486	-34.3377113342285\\
9.05500030517578	-33.4807968139648\\
9.0600004196167	-32.7051582336426\\
9.0649995803833	-32.0498161315918\\
9.06999969482422	-31.4634437561035\\
9.07499980926514	-30.9637107849121\\
9.07999992370605	-30.5535736083984\\
9.08500003814697	-30.2356071472168\\
9.09000015258789	-29.9484729766846\\
9.09500026702881	-29.4866561889648\\
9.10000038146973	-28.8350868225098\\
9.10499954223633	-28.3481941223145\\
9.10999965667725	-27.8865089416504\\
9.11499977111816	-27.1930637359619\\
9.11999988555908	-26.1990032196045\\
9.125	-25.0837020874023\\
9.13000011444092	-24.071704864502\\
9.13500022888184	-23.1336841583252\\
9.14000034332275	-22.5247592926025\\
9.14500045776367	-22.1756477355957\\
9.14999961853027	-22.2347164154053\\
9.15499973297119	-22.6451416015625\\
9.15999984741211	-23.3674259185791\\
9.16499996185303	-24.29616355896\\
9.17000007629395	-25.2503662109375\\
9.17500019073486	-26.1560764312744\\
9.18000030517578	-26.8781452178955\\
9.1850004196167	-27.3258609771729\\
9.1899995803833	-27.5033264160156\\
9.19499969482422	-27.3678436279297\\
9.19999980926514	-26.9277114868164\\
9.20499992370605	-26.2350654602051\\
9.21000003814697	-25.4379425048828\\
9.21500015258789	-24.8160991668701\\
9.22000026702881	-24.2966442108154\\
9.22500038146973	-23.9864501953125\\
9.22999954223633	-23.8730487823486\\
9.23499965667725	-23.5496215820313\\
9.23999977111816	-21.8984088897705\\
9.24499988555908	-16.8673496246338\\
9.25	-8.53031539916992\\
9.25500011444092	1.50164425373077\\
9.26000022888184	13.298659324646\\
9.26500034332275	26.823486328125\\
9.27000045776367	39.1177749633789\\
9.27499961853027	50.7631149291992\\
9.27999973297119	62.2328796386719\\
9.28499984741211	72.9421157836914\\
9.28999996185303	83.5782089233398\\
9.29500007629395	94.1737518310547\\
9.30000019073486	104.146522521973\\
9.30500030517578	120.092582702637\\
9.3100004196167	132.039886474609\\
9.3149995803833	144.565673828125\\
9.31999969482422	157.351196289063\\
9.32499980926514	169.757934570313\\
9.32999992370605	183.699249267578\\
9.33500003814697	194.440368652344\\
9.34000015258789	200.536331176758\\
9.34500026702881	213.958221435547\\
9.35000038146973	215.470306396484\\
9.35499954223633	209.104232788086\\
9.35999965667725	209.418670654297\\
9.36499977111816	203.170684814453\\
9.36999988555908	199.914489746094\\
9.375	205.629486083984\\
9.38000011444092	223.375534057617\\
9.38500022888184	257.425354003906\\
9.39000034332275	312.718170166016\\
9.39500045776367	392.450408935547\\
9.39999961853027	493.654205322266\\
9.40499973297119	615.666809082031\\
9.40999984741211	757.244384765625\\
9.41499996185303	913.30126953125\\
9.42000007629395	1079.56872558594\\
9.42500019073486	1251.85607910156\\
9.43000030517578	1447.98522949219\\
9.4350004196167	1637.85888671875\\
9.4399995803833	1787.54626464844\\
9.44499969482422	1865.58251953125\\
9.44999980926514	1820.14929199219\\
9.45499992370605	1608.0927734375\\
9.46000003814697	1172.79943847656\\
9.46500015258789	505.965515136719\\
9.47000026702881	-376.517486572266\\
9.47500038146973	-1395.36511230469\\
9.47999954223633	-2418.36669921875\\
9.48499965667725	-3075.14111328125\\
9.48999977111816	-2762.3369140625\\
9.49499988555908	-2579.17163085938\\
9.5	-2438.3427734375\\
9.50500011444092	-2297.37548828125\\
9.51000022888184	-2139.07006835938\\
9.51500034332275	-1958.47180175781\\
9.52000045776367	-1749.64453125\\
9.52499961853027	-1514.86828613281\\
9.52999973297119	-1245.4951171875\\
9.53499984741211	-954.570739746094\\
9.53999996185303	-645.6572265625\\
9.54500007629395	-327.547790527344\\
9.55000019073486	-12.758508682251\\
9.55500030517578	69.3888473510742\\
9.5600004196167	87.0533447265625\\
9.5649995803833	88.0324401855469\\
9.56999969482422	82.754997253418\\
9.57499980926514	74.5593948364258\\
9.57999992370605	66.6763076782227\\
9.58500003814697	58.9543571472168\\
9.59000015258789	51.7641487121582\\
9.59500026702881	45.5307769775391\\
9.60000038146973	39.9388999938965\\
9.60499954223633	35.0060577392578\\
9.60999965667725	30.6713008880615\\
9.61499977111816	26.8872833251953\\
9.61999988555908	23.5212879180908\\
9.625	20.5683574676514\\
9.63000011444092	18.0244445800781\\
9.63500022888184	15.8172750473022\\
9.64000034332275	13.8757886886597\\
9.64500045776367	12.1123132705688\\
9.64999961853027	10.5800590515137\\
9.65499973297119	9.31364822387695\\
9.65999984741211	8.16388893127441\\
9.66499996185303	7.15140390396118\\
9.67000007629395	6.26389932632446\\
9.67500019073486	5.5036563873291\\
9.68000030517578	4.8401050567627\\
9.6850004196167	4.24382781982422\\
9.6899995803833	3.72343397140503\\
9.69499969482422	3.28065609931946\\
9.69999980926514	2.8904173374176\\
9.70499992370605	2.53599882125854\\
9.71000003814697	2.22513556480408\\
9.71500015258789	1.96603333950043\\
9.72000026702881	1.74434304237366\\
9.72500038146973	1.55444586277008\\
9.72999954223633	1.38699436187744\\
9.73499965667725	1.23088753223419\\
9.73999977111816	1.0726683139801\\
9.74499988555908	0.898936688899994\\
9.75	0.767601013183594\\
9.75500011444092	0.664968430995941\\
9.76000022888184	0.606030106544495\\
9.76500034332275	0.567673325538635\\
9.77000045776367	0.516899347305298\\
9.77499961853027	0.46599206328392\\
9.77999973297119	0.407031238079071\\
9.78499984741211	0.365289330482483\\
9.78999996185303	0.327881783246994\\
9.79500007629395	0.295027524232864\\
9.80000019073486	0.265797197818756\\
9.80500030517578	0.239317655563354\\
9.8100004196167	0.216670617461205\\
9.8149995803833	0.198231741786003\\
9.81999969482422	0.179994121193886\\
9.82499980926514	0.163080155849457\\
9.82999992370605	0.146892800927162\\
9.83500003814697	0.134121775627136\\
9.84000015258789	0.128075003623962\\
9.84500026702881	0.129180163145065\\
9.85000038146973	0.138030216097832\\
9.85499954223633	0.130682855844498\\
9.85999965667725	0.11513164639473\\
9.86499977111816	0.088552325963974\\
9.86999988555908	0.068922370672226\\
9.875	0.0699675902724266\\
9.88000011444092	0.0862336531281471\\
9.88500022888184	0.117835856974125\\
9.89000034332275	0.115879394114017\\
9.89500045776367	0.0974497050046921\\
9.89999961853027	0.0570298954844475\\
9.90499973297119	0.0273030456155539\\
9.90999984741211	0.0312447529286146\\
9.91499996185303	0.0590214654803276\\
9.92000007629395	0.111245468258858\\
9.92500019073486	0.121744856238365\\
9.93000030517578	0.115242473781109\\
9.9350004196167	0.0852055251598358\\
9.9399995803833	0.0608913786709309\\
9.94499969482422	0.0543415993452072\\
9.94999980926514	0.049847174435854\\
9.95499992370605	0.0474081113934517\\
9.96000003814697	0.0470244064927101\\
9.96500015258789	0.0486960597336292\\
9.97000026702881	0.052423071116209\\
9.97500038146973	0.0543879717588425\\
9.97999954223633	0.0548247396945953\\
9.98499965667725	0.0557149387896061\\
9.98999977111816	0.0570585690438747\\
9.99499988555908	0.0588556304574013\\
10	0.0611061230301857\\
};
\addlegendentry{RS}

\addplot [color=red, line width=2.0pt]
  table[row sep=crcr]{%
0.0949999988079071	-16.4773559570313\\
0.100000001490116	-14.7052793502808\\
0.104999996721745	-13.2420291900635\\
0.109999999403954	-11.9119825363159\\
0.115000002086163	-10.7947912216187\\
0.119999997317791	394.257751464844\\
0.125	262.628692626953\\
0.129999995231628	186.987060546875\\
0.135000005364418	108.66089630127\\
0.140000000596046	47.0073547363281\\
0.144999995827675	-2.12534999847412\\
0.150000005960464	-39.1338653564453\\
0.155000001192093	-69.4319686889648\\
0.159999996423721	-75.2312545776367\\
0.165000006556511	-95.6647796630859\\
0.170000001788139	-123.629035949707\\
0.174999997019768	-151.468627929688\\
0.180000007152557	-170.973815917969\\
0.185000002384186	-186.922500610352\\
0.189999997615814	-196.509582519531\\
0.194999992847443	-1185.12927246094\\
0.200000002980232	-1016.77136230469\\
0.204999998211861	-669.493774414063\\
0.209999993443489	-121.950927734375\\
0.215000003576279	304.327484130859\\
0.219999998807907	482.195281982422\\
0.224999994039536	677.180603027344\\
0.230000004172325	782.454772949219\\
0.234999999403954	760.610290527344\\
0.239999994635582	619.679382324219\\
0.245000004768372	386.383941650391\\
0.25	99.3258514404297\\
0.254999995231628	-274.985290527344\\
0.259999990463257	-447.693939208984\\
0.264999985694885	-577.362426757813\\
0.270000010728836	-629.519775390625\\
0.275000005960464	-591.382019042969\\
0.280000001192093	-476.543640136719\\
0.284999996423721	-321.148132324219\\
0.28999999165535	-152.688232421875\\
0.294999986886978	-0.40263232588768\\
0.300000011920929	99.4099960327148\\
0.305000007152557	173.903076171875\\
0.310000002384186	208.314529418945\\
0.314999997615814	195.068222045898\\
0.319999992847443	142.922500610352\\
0.324999988079071	60.193416595459\\
0.330000013113022	-8.46141529083252\\
0.33500000834465	-105.959823608398\\
0.340000003576279	-217.821426391602\\
0.344999998807907	-266.759948730469\\
0.349999994039536	-278.634887695313\\
0.354999989271164	-254.128067016602\\
0.360000014305115	-204.236679077148\\
0.365000009536743	-149.068649291992\\
0.370000004768372	-96.5332870483398\\
0.375	-33.8064651489258\\
0.379999995231628	23.0122127532959\\
0.384999990463257	63.5747261047363\\
0.389999985694885	85.2159271240234\\
0.395000010728836	77.8599853515625\\
0.400000005960464	53.8853225708008\\
0.405000001192093	16.9283714294434\\
0.409999996423721	-32.0322456359863\\
0.41499999165535	-72.550910949707\\
0.419999986886978	-107.871994018555\\
0.425000011920929	-134.725112915039\\
0.430000007152557	-130.854583740234\\
0.435000002384186	-110.907577514648\\
0.439999997615814	-88.383430480957\\
0.444999992847443	-53.7956657409668\\
0.449999988079071	-16.3750381469727\\
0.455000013113022	17.3953304290771\\
0.46000000834465	43.0466384887695\\
0.465000003576279	58.2310943603516\\
0.469999998807907	61.0356750488281\\
0.474999994039536	51.2640075683594\\
0.479999989271164	33.6071701049805\\
0.485000014305115	9.9575777053833\\
0.490000009536743	-14.5978479385376\\
0.495000004768372	-41.8143844604492\\
0.5	-53.8980903625488\\
0.504999995231628	-55.3558044433594\\
0.509999990463257	-46.7157363891602\\
0.514999985694885	-32.0788230895996\\
0.519999980926514	-14.7388000488281\\
0.524999976158142	5.09894514083862\\
0.529999971389771	20.2795352935791\\
0.535000026226044	28.9026927947998\\
0.540000021457672	33.0710906982422\\
0.545000016689301	32.768238067627\\
0.550000011920929	27.1819343566895\\
0.555000007152557	15.0338277816772\\
0.560000002384186	3.81619000434875\\
0.564999997615814	-4.8199782371521\\
0.569999992847443	-11.4728355407715\\
0.574999988079071	-16.0791034698486\\
0.579999983310699	-17.0824851989746\\
0.584999978542328	-15.4094972610474\\
0.589999973773956	-14.6286563873291\\
0.595000028610229	-11.0413608551025\\
0.600000023841858	-8.76599407196045\\
0.605000019073486	-5.94159841537476\\
0.610000014305115	-4.45458555221558\\
0.615000009536743	-5.13076972961426\\
0.620000004768372	-6.19444465637207\\
0.625	-7.65014982223511\\
0.629999995231628	-11.1757030487061\\
0.634999990463257	-14.4430408477783\\
0.639999985694885	-18.0047245025635\\
0.644999980926514	-21.36203956604\\
0.649999976158142	-24.2106075286865\\
0.654999971389771	-26.3310394287109\\
0.660000026226044	-28.1960124969482\\
0.665000021457672	-29.4477634429932\\
0.670000016689301	-30.3058547973633\\
0.675000011920929	-30.7897644042969\\
0.680000007152557	-31.1443099975586\\
0.685000002384186	-31.5184097290039\\
0.689999997615814	-32.0556449890137\\
0.694999992847443	-32.843692779541\\
0.699999988079071	-33.8105926513672\\
0.704999983310699	-35.1465606689453\\
0.709999978542328	-36.4762687683105\\
0.714999973773956	-37.8401260375977\\
0.720000028610229	-39.1132164001465\\
0.725000023841858	-40.4303817749023\\
0.730000019073486	-41.2256927490234\\
0.735000014305115	-42.1105766296387\\
0.740000009536743	-42.5777778625488\\
0.745000004768372	-42.9852485656738\\
0.75	-43.1131286621094\\
0.754999995231628	-43.1938018798828\\
0.759999990463257	-43.0982666015625\\
0.764999985694885	-42.5922546386719\\
0.769999980926514	-42.6005973815918\\
0.774999976158142	-42.7160339355469\\
0.779999971389771	-42.7101097106934\\
0.785000026226044	-42.5732917785645\\
0.790000021457672	-42.4195785522461\\
0.795000016689301	-42.150936126709\\
0.800000011920929	-41.8821296691895\\
0.805000007152557	-41.5216178894043\\
0.810000002384186	-41.1745071411133\\
0.814999997615814	-40.7180023193359\\
0.819999992847443	-40.1621475219727\\
0.824999988079071	-39.6737403869629\\
0.829999983310699	-38.9239540100098\\
0.834999978542328	-38.1181221008301\\
0.839999973773956	-37.3835601806641\\
0.845000028610229	-36.5041084289551\\
0.850000023841858	-35.5996971130371\\
0.855000019073486	-34.6448249816895\\
0.860000014305115	-33.8904609680176\\
0.865000009536743	-33.1615982055664\\
0.870000004768372	-32.369312286377\\
0.875	-31.6147651672363\\
0.879999995231628	-30.8526916503906\\
0.884999990463257	-30.0892200469971\\
0.889999985694885	-29.3301239013672\\
0.894999980926514	-28.5584831237793\\
0.899999976158142	-27.8104228973389\\
0.904999971389771	-27.0177879333496\\
0.910000026226044	-26.3219203948975\\
0.915000021457672	-25.5635280609131\\
0.920000016689301	-24.8451766967773\\
0.925000011920929	-24.0947723388672\\
0.930000007152557	-23.4031620025635\\
0.935000002384186	-22.6977634429932\\
0.939999997615814	-22.0529346466064\\
0.944999992847443	-21.4387226104736\\
0.949999988079071	-21.0380249023438\\
0.954999983310699	-20.6349105834961\\
0.959999978542328	-20.3051948547363\\
0.964999973773956	-19.9738140106201\\
0.970000028610229	-19.7476768493652\\
0.975000023841858	-19.5019454956055\\
0.980000019073486	-19.3319072723389\\
0.985000014305115	-19.1512508392334\\
0.990000009536743	-18.9831771850586\\
0.995000004768372	-18.7920036315918\\
1	-18.6419296264648\\
1.00499999523163	-18.4218521118164\\
1.00999999046326	-18.2808017730713\\
1.01499998569489	-18.4219608306885\\
1.01999998092651	-18.6162796020508\\
1.02499997615814	-18.7782726287842\\
1.02999997138977	-18.9489288330078\\
1.0349999666214	-19.1589317321777\\
1.03999996185303	-19.393705368042\\
1.04499995708466	-19.6503276824951\\
1.04999995231628	-19.9267101287842\\
1.05499994754791	-20.2266540527344\\
1.05999994277954	-20.5472164154053\\
1.06500005722046	-20.8850612640381\\
1.07000005245209	-21.2467422485352\\
1.07500004768372	-21.618766784668\\
1.08000004291534	-21.9941692352295\\
1.08500003814697	-22.3786678314209\\
1.0900000333786	-22.7698669433594\\
1.09500002861023	-23.1647891998291\\
1.10000002384186	-23.5612354278564\\
1.10500001907349	-23.9655361175537\\
1.11000001430511	-24.3911724090576\\
1.11500000953674	-24.8115749359131\\
1.12000000476837	-25.2311878204346\\
1.125	-25.6581859588623\\
1.12999999523163	-26.0847930908203\\
1.13499999046326	-26.507869720459\\
1.13999998569489	-26.9226055145264\\
1.14499998092651	-27.3124084472656\\
1.14999997615814	-27.6972064971924\\
1.15499997138977	-28.0718631744385\\
1.1599999666214	-28.4220962524414\\
1.16499996185303	-28.7523632049561\\
1.16999995708466	-29.0677165985107\\
1.17499995231628	-29.3660678863525\\
1.17999994754791	-29.6456203460693\\
1.18499994277954	-29.9062042236328\\
1.19000005722046	-30.1486854553223\\
1.19500005245209	-30.3569507598877\\
1.20000004768372	-30.5287971496582\\
1.20500004291534	-30.6738624572754\\
1.21000003814697	-30.7909126281738\\
1.2150000333786	-30.9298095703125\\
1.22000002861023	-31.043758392334\\
1.22500002384186	-31.1442165374756\\
1.23000001907349	-31.2093524932861\\
1.23500001430511	-31.2276382446289\\
1.24000000953674	-31.2283153533936\\
1.24500000476837	-31.2046089172363\\
1.25	-31.2288131713867\\
1.25499999523163	-31.2402000427246\\
1.25999999046326	-31.2487258911133\\
1.26499998569489	-31.2174415588379\\
1.26999998092651	-31.1131610870361\\
1.27499997615814	-31.007682800293\\
1.27999997138977	-30.8796310424805\\
1.2849999666214	-30.704496383667\\
1.28999996185303	-30.509521484375\\
1.29499995708466	-30.2965259552002\\
1.29999995231628	-30.1089191436768\\
1.30499994754791	-29.9031219482422\\
1.30999994277954	-29.6883449554443\\
1.31500005722046	-29.497200012207\\
1.32000005245209	-29.3069515228271\\
1.32500004768372	-29.1181163787842\\
1.33000004291534	-28.9267539978027\\
1.33500003814697	-28.7367343902588\\
1.3400000333786	-28.5510139465332\\
1.34500002861023	-28.3629417419434\\
1.35000002384186	-28.1670455932617\\
1.35500001907349	-27.9706134796143\\
1.36000001430511	-27.7511596679688\\
1.36500000953674	-27.4572601318359\\
1.37000000476837	-27.1341991424561\\
1.375	-26.7893657684326\\
1.37999999523163	-26.5919132232666\\
1.38499999046326	-26.3601779937744\\
1.38999998569489	-26.1270694732666\\
1.39499998092651	-25.9703502655029\\
1.39999997615814	-25.8271884918213\\
1.40499997138977	-25.7033023834229\\
1.4099999666214	-25.5914897918701\\
1.41499996185303	-25.5024013519287\\
1.41999995708466	-25.4224491119385\\
1.42499995231628	-25.2976531982422\\
1.42999994754791	-25.1886863708496\\
1.43499994277954	-25.0847797393799\\
1.44000005722046	-24.9826221466064\\
1.44500005245209	-24.8823928833008\\
1.45000004768372	-24.7824172973633\\
1.45500004291534	-24.681468963623\\
1.46000003814697	-24.5818176269531\\
1.4650000333786	-24.4696350097656\\
1.47000002861023	-24.339298248291\\
1.47500002384186	-24.1972732543945\\
1.48000001907349	-24.2274341583252\\
1.48500001430511	-24.4157791137695\\
1.49000000953674	-24.7269325256348\\
1.49500000476837	-24.6110134124756\\
1.5	-24.4685974121094\\
1.50499999523163	-24.3892307281494\\
1.50999999046326	-24.5002460479736\\
1.51499998569489	-24.6037273406982\\
1.51999998092651	-24.7031517028809\\
1.52499997615814	-24.8143844604492\\
1.52999997138977	-24.9431056976318\\
1.5349999666214	-25.0390853881836\\
1.53999996185303	-25.14084815979\\
1.54499995708466	-25.2707195281982\\
1.54999995231628	-25.4386291503906\\
1.55499994754791	-25.6155185699463\\
1.55999994277954	-25.7717742919922\\
1.56500005722046	-25.9355335235596\\
1.57000005245209	-42.7185287475586\\
1.57500004768372	-24.0153217315674\\
1.58000004291534	-20.9141693115234\\
1.58500003814697	-21.4886493682861\\
1.5900000333786	-22.2642402648926\\
1.59500002861023	-23.2745704650879\\
1.60000002384186	-24.3909893035889\\
1.60500001907349	-25.6595516204834\\
1.61000001430511	-27.0682334899902\\
1.61500000953674	-27.948263168335\\
1.62000000476837	-29.0674781799316\\
1.625	-29.3987464904785\\
1.62999999523163	-29.2754936218262\\
1.63499999046326	-28.6311225891113\\
1.63999998569489	-27.4971580505371\\
1.64499998092651	-27.5356521606445\\
1.64999997615814	-27.3257236480713\\
1.65499997138977	-27.0560283660889\\
1.6599999666214	-26.9614772796631\\
1.66499996185303	-26.9919681549072\\
1.66999995708466	-26.8388767242432\\
1.67499995231628	-27.039270401001\\
1.67999994754791	-27.5763130187988\\
1.68499994277954	-27.8351707458496\\
1.69000005722046	-28.1180782318115\\
1.69500005245209	-28.2620906829834\\
1.70000004768372	-28.359094619751\\
1.70500004291534	-28.3602123260498\\
1.71000003814697	-28.3219356536865\\
1.7150000333786	-28.3211498260498\\
1.72000002861023	-28.3178977966309\\
1.72500002384186	-28.2765922546387\\
1.73000001907349	-28.2301712036133\\
1.73500001430511	-28.1606502532959\\
1.74000000953674	-28.0878753662109\\
1.74500000476837	-28.0176467895508\\
1.75	-27.9512481689453\\
1.75499999523163	-27.8812980651855\\
1.75999999046326	-27.8315525054932\\
1.76499998569489	-27.7805309295654\\
1.76999998092651	-27.7385101318359\\
1.77499997615814	-27.7353610992432\\
1.77999997138977	-27.7402038574219\\
1.7849999666214	-27.7332324981689\\
1.78999996185303	-27.7300205230713\\
1.79499995708466	-27.724178314209\\
1.79999995231628	-27.694953918457\\
1.80499994754791	-27.6634330749512\\
1.80999994277954	-27.6130390167236\\
1.81500005722046	-27.5507373809814\\
1.82000005245209	-27.4814338684082\\
1.82500004768372	-27.4106979370117\\
1.83000004291534	-27.3321590423584\\
1.83500003814697	-27.2432308197021\\
1.8400000333786	-27.1584129333496\\
1.84500002861023	-27.0812282562256\\
1.85000002384186	-27.0495681762695\\
1.85500001907349	-27.0076656341553\\
1.86000001430511	-26.9651584625244\\
1.86500000953674	-26.9246578216553\\
1.87000000476837	-26.886682510376\\
1.875	-26.8685855865479\\
1.87999999523163	-26.8696727752686\\
1.88499999046326	-26.8646183013916\\
1.88999998569489	-26.8589401245117\\
1.89499998092651	-26.7870044708252\\
1.89999997615814	-26.7439346313477\\
1.90499997138977	-26.7093696594238\\
1.9099999666214	-26.6731472015381\\
1.91499996185303	-26.6403846740723\\
1.91999995708466	-26.6111717224121\\
1.92499995231628	-26.6018924713135\\
1.92999994754791	-26.597583770752\\
1.93499994277954	-26.5921230316162\\
1.94000005722046	-26.5904941558838\\
1.94500005245209	-26.5925598144531\\
1.95000004768372	-26.5970478057861\\
1.95500004291534	-26.6246910095215\\
1.96000003814697	-26.6917762756348\\
1.9650000333786	-26.7639808654785\\
1.97000002861023	-26.8056411743164\\
1.97500002384186	-26.724552154541\\
1.98000001907349	-26.6531543731689\\
1.98500001430511	-26.5811786651611\\
1.99000000953674	-26.5960750579834\\
1.99500000476837	-26.5867786407471\\
2	-26.577564239502\\
2.00500011444092	-26.6771583557129\\
2.00999999046326	-26.8042221069336\\
2.01500010490417	-26.9293079376221\\
2.01999998092651	-27.1055603027344\\
2.02500009536743	-27.3062648773193\\
2.02999997138977	-27.5176372528076\\
2.03500008583069	-27.2905769348145\\
2.03999996185303	-27.0957279205322\\
2.04500007629395	-26.968240737915\\
2.04999995231628	-26.8846988677979\\
2.0550000667572	-26.720739364624\\
2.05999994277954	-26.4560432434082\\
2.06500005722046	-26.1665935516357\\
2.0699999332428	-25.8291854858398\\
2.07500004768372	-25.4709281921387\\
2.07999992370605	-25.2366504669189\\
2.08500003814697	-25.020393371582\\
2.08999991416931	-25.1572570800781\\
2.09500002861023	-24.9483623504639\\
2.09999990463257	-24.9900894165039\\
2.10500001907349	-25.056188583374\\
2.10999989509583	-25.0662155151367\\
2.11500000953674	-24.9084072113037\\
2.11999988555908	-25.7878036499023\\
2.125	-26.8122291564941\\
2.13000011444092	-27.2883605957031\\
2.13499999046326	-27.6748332977295\\
2.14000010490417	-27.878345489502\\
2.14499998092651	-27.7988834381104\\
2.15000009536743	-27.6712589263916\\
2.15499997138977	-27.1740741729736\\
2.16000008583069	-26.1945095062256\\
2.16499996185303	-25.7400150299072\\
2.17000007629395	-25.3626117706299\\
2.17499995231628	-24.6102771759033\\
2.1800000667572	-24.487190246582\\
2.18499994277954	-24.5837879180908\\
2.19000005722046	-24.9451313018799\\
2.1949999332428	-25.22878074646\\
2.20000004768372	-25.5085124969482\\
2.20499992370605	-25.723560333252\\
2.21000003814697	-25.7591724395752\\
2.21499991416931	-25.5989379882813\\
2.22000002861023	-25.2894897460938\\
2.22499990463257	-24.8563785552979\\
2.23000001907349	-24.0445327758789\\
2.23499989509583	-22.7339267730713\\
2.24000000953674	-22.0150909423828\\
2.24499988555908	-21.3505535125732\\
2.25	-20.5435237884521\\
2.25500011444092	-19.9584350585938\\
2.25999999046326	-19.7193470001221\\
2.26500010490417	-19.8882312774658\\
2.26999998092651	-20.2315635681152\\
2.27500009536743	-20.6692123413086\\
2.27999997138977	-21.0693206787109\\
2.28500008583069	-20.3755378723145\\
2.28999996185303	-19.7889347076416\\
2.29500007629395	-19.8697528839111\\
2.29999995231628	-21.0716133117676\\
2.3050000667572	-20.4544105529785\\
2.30999994277954	-20.508394241333\\
2.31500005722046	-20.9371166229248\\
2.3199999332428	-21.2771816253662\\
2.32500004768372	-21.336462020874\\
2.32999992370605	-21.4875469207764\\
2.33500003814697	-21.7101879119873\\
2.33999991416931	-20.1563377380371\\
2.34500002861023	-19.6606750488281\\
2.34999990463257	-22.4268455505371\\
2.35500001907349	-23.075080871582\\
2.35999989509583	-23.9851741790771\\
2.36500000953674	-24.8951759338379\\
2.36999988555908	-25.3666000366211\\
2.375	-26.5615234375\\
2.38000011444092	-27.5025081634521\\
2.38499999046326	-27.6970043182373\\
2.39000010490417	-28.4546489715576\\
2.39499998092651	-28.8530197143555\\
2.40000009536743	-28.8474807739258\\
2.40499997138977	-28.6785125732422\\
2.41000008583069	-28.0524291992188\\
2.41499996185303	-27.714656829834\\
2.42000007629395	-27.493049621582\\
2.42499995231628	-27.6929931640625\\
2.4300000667572	-27.7567119598389\\
2.43499994277954	-28.157299041748\\
2.44000005722046	-28.8051528930664\\
2.4449999332428	-28.6139545440674\\
2.45000004768372	-30.5015316009521\\
2.45499992370605	-31.3467044830322\\
2.46000003814697	-32.6308135986328\\
2.46499991416931	-34.680549621582\\
2.47000002861023	-35.634220123291\\
2.47499990463257	-36.764820098877\\
2.48000001907349	-37.8186454772949\\
2.48499989509583	-38.625171661377\\
2.49000000953674	-39.4078674316406\\
2.49499988555908	-40.4393424987793\\
2.5	-40.8478012084961\\
2.50500011444092	-40.2933807373047\\
2.50999999046326	-39.1454048156738\\
2.51500010490417	-37.892505645752\\
2.51999998092651	-38.2164268493652\\
2.52500009536743	-38.0273056030273\\
2.52999997138977	-37.8863067626953\\
2.53500008583069	-39.0344390869141\\
2.53999996185303	-38.845703125\\
2.54500007629395	-39.2300682067871\\
2.54999995231628	-39.5251884460449\\
2.5550000667572	-39.690673828125\\
2.55999994277954	-39.8344268798828\\
2.56500005722046	-39.2688217163086\\
2.5699999332428	-38.6714401245117\\
2.57500004768372	-37.9172668457031\\
2.57999992370605	-36.9811058044434\\
2.58500003814697	-35.9887161254883\\
2.58999991416931	-34.5728569030762\\
2.59500002861023	-33.9866714477539\\
2.59999990463257	-33.5428657531738\\
2.60500001907349	-32.9401779174805\\
2.60999989509583	-32.7706642150879\\
2.61500000953674	-32.7475204467773\\
2.61999988555908	-32.8833122253418\\
2.625	-33.0129585266113\\
2.63000011444092	-32.8978462219238\\
2.63499999046326	-32.2822227478027\\
2.64000010490417	-32.479621887207\\
2.64499998092651	-31.9392986297607\\
2.65000009536743	-29.9193592071533\\
2.65499997138977	-28.5557136535645\\
2.66000008583069	-27.2555923461914\\
2.66499996185303	-25.1378040313721\\
2.67000007629395	-23.5309257507324\\
2.67499995231628	-22.7098445892334\\
2.6800000667572	-21.7581462860107\\
2.68499994277954	-20.842191696167\\
2.69000005722046	-20.4477424621582\\
2.6949999332428	-20.507682800293\\
2.70000004768372	-20.3290252685547\\
2.70499992370605	-20.3269729614258\\
2.71000003814697	-20.1938743591309\\
2.71499991416931	-19.40891456604\\
2.72000002861023	-19.1036415100098\\
2.72499990463257	-18.4571018218994\\
2.73000001907349	-17.5522441864014\\
2.73499989509583	-16.5153827667236\\
2.74000000953674	-15.2458848953247\\
2.74499988555908	-13.9878787994385\\
2.75	-13.1566247940063\\
2.75500011444092	-12.6661520004272\\
2.75999999046326	-12.3844585418701\\
2.76500010490417	-12.1241979598999\\
2.76999998092651	-11.8404521942139\\
2.77500009536743	-11.6517753601074\\
2.77999997138977	-12.1284818649292\\
2.78500008583069	-12.700647354126\\
2.78999996185303	-13.2046012878418\\
2.79500007629395	-13.4559392929077\\
2.79999995231628	-13.2305088043213\\
2.8050000667572	-12.3422021865845\\
2.80999994277954	-11.2772455215454\\
2.81500005722046	-9.869948387146\\
2.8199999332428	-8.04409599304199\\
2.82500004768372	-6.19777917861938\\
2.82999992370605	-3.68661618232727\\
2.83500003814697	-2.80989336967468\\
2.83999991416931	-2.41192770004272\\
2.84500002861023	-2.65974950790405\\
2.84999990463257	-4.92108917236328\\
2.85500001907349	-4.97631168365479\\
2.85999989509583	-6.16033267974854\\
2.86500000953674	-6.2611722946167\\
2.86999988555908	-9.22036170959473\\
2.875	-9.24261474609375\\
2.88000011444092	-9.16455745697021\\
2.88499999046326	-9.43332862854004\\
2.89000010490417	-8.71663475036621\\
2.89499998092651	-12.7288074493408\\
2.90000009536743	-18.4345626831055\\
2.90499997138977	-25.2063980102539\\
2.91000008583069	-33.4371681213379\\
2.91499996185303	-42.5513572692871\\
2.92000007629395	-50.8474273681641\\
2.92499995231628	-57.5464248657227\\
2.9300000667572	-62.3023834228516\\
2.93499994277954	-63.0992698669434\\
2.94000005722046	-64.4071731567383\\
2.9449999332428	-63.3791885375977\\
2.95000004768372	-61.9634628295898\\
2.95499992370605	-56.9893035888672\\
2.96000003814697	-53.1122589111328\\
2.96499991416931	-53.7465400695801\\
2.97000002861023	-52.3212242126465\\
2.97499990463257	-51.516716003418\\
2.98000001907349	-51.9014358520508\\
2.98499989509583	-41.7541732788086\\
2.99000000953674	-51.9305381774902\\
2.99499988555908	-52.1129684448242\\
3	-53.513744354248\\
3.00500011444092	-55.7941741943359\\
3.00999999046326	-57.7479820251465\\
3.01500010490417	-59.7717094421387\\
3.01999998092651	-59.0930290222168\\
3.02500009536743	-56.1050796508789\\
3.02999997138977	-53.7822036743164\\
3.03500008583069	-49.8488845825195\\
3.03999996185303	-44.5016021728516\\
3.04500007629395	-39.270133972168\\
3.04999995231628	-35.6879386901855\\
3.0550000667572	-33.5873565673828\\
3.05999994277954	-32.1403350830078\\
3.06500005722046	-32.5028076171875\\
3.0699999332428	-34.1794395446777\\
3.07500004768372	-34.2984161376953\\
3.07999992370605	-35.0020523071289\\
3.08500003814697	-34.6637535095215\\
3.08999991416931	-32.6776084899902\\
3.09500002861023	-30.1047039031982\\
3.09999990463257	-27.2861194610596\\
3.10500001907349	-23.7245788574219\\
3.10999989509583	-20.3794956207275\\
3.11500000953674	-16.7972240447998\\
3.11999988555908	-14.286675453186\\
3.125	-12.865891456604\\
3.13000011444092	-13.9568529129028\\
3.13499999046326	-20.0869560241699\\
3.14000010490417	-22.6149654388428\\
3.14499998092651	-23.8951034545898\\
3.15000009536743	-19.1489219665527\\
3.15499997138977	-11.5560846328735\\
3.16000008583069	-4.9248194694519\\
3.16499996185303	3.1034722328186\\
3.17000007629395	12.3039808273315\\
3.17499995231628	21.9735450744629\\
3.1800000667572	28.0445556640625\\
3.18499994277954	33.1410140991211\\
3.19000005722046	35.1098709106445\\
3.1949999332428	36.6988525390625\\
3.20000004768372	36.8317489624023\\
3.20499992370605	32.2389030456543\\
3.21000003814697	27.6608619689941\\
3.21499991416931	18.5885772705078\\
3.22000002861023	9.72978401184082\\
3.22499990463257	0.965020060539246\\
3.23000001907349	-7.63854026794434\\
3.23499989509583	-15.8680648803711\\
3.24000000953674	-24.5926971435547\\
3.24499988555908	-33.441722869873\\
3.25	-40.3431739807129\\
3.25500011444092	-46.3512535095215\\
3.25999999046326	-50.9140129089355\\
3.26500010490417	-53.1302452087402\\
3.26999998092651	-52.540828704834\\
3.27500009536743	-51.3443870544434\\
3.27999997138977	-48.6027183532715\\
3.28500008583069	-45.7485618591309\\
3.28999996185303	-44.2074851989746\\
3.29500007629395	-43.8813705444336\\
3.29999995231628	-44.4131240844727\\
3.3050000667572	-45.8589057922363\\
3.30999994277954	-49.2724609375\\
3.31500005722046	-51.7798767089844\\
3.3199999332428	-51.6934356689453\\
3.32500004768372	-54.8308563232422\\
3.32999992370605	-58.4962196350098\\
3.33500003814697	-60.3710250854492\\
3.33999991416931	-62.3967895507813\\
3.34500002861023	-63.4659423828125\\
3.34999990463257	-61.6879463195801\\
3.35500001907349	-58.9479331970215\\
3.35999989509583	-55.752799987793\\
3.36500000953674	-52.0725402832031\\
3.36999988555908	-48.8956909179688\\
3.375	-46.0630989074707\\
3.38000011444092	-44.7355575561523\\
3.38499999046326	-44.3205909729004\\
3.39000010490417	-44.3947677612305\\
3.39499998092651	-44.941707611084\\
3.40000009536743	-45.6649589538574\\
3.40499997138977	-45.6602172851563\\
3.41000008583069	-45.9066886901855\\
3.41499996185303	-43.8646011352539\\
3.42000007629395	-41.3521537780762\\
3.42499995231628	-38.4169082641602\\
3.4300000667572	-35.0472640991211\\
3.43499994277954	-31.9723148345947\\
3.44000005722046	-29.6152153015137\\
3.4449999332428	-27.5361328125\\
3.45000004768372	-26.251745223999\\
3.45499992370605	-26.2199287414551\\
3.46000003814697	-31.7823181152344\\
3.46499991416931	-35.3230514526367\\
3.47000002861023	-30.6042633056641\\
3.47499990463257	-31.8305835723877\\
3.48000001907349	-24.032829284668\\
3.48499989509583	-12.2777862548828\\
3.49000000953674	7.32589054107666\\
3.49499988555908	18.1379642486572\\
3.5	24.308988571167\\
3.50500011444092	33.7571754455566\\
3.50999999046326	42.6013832092285\\
3.51500010490417	49.4131202697754\\
3.51999998092651	60.0998344421387\\
3.52500009536743	68.4405746459961\\
3.52999997138977	77.5433120727539\\
3.53500008583069	90.0049667358398\\
3.53999996185303	101.782768249512\\
3.54500007629395	102.018684387207\\
3.54999995231628	77.0717391967773\\
3.5550000667572	30.7338542938232\\
3.55999994277954	-22.7250843048096\\
3.56500005722046	-79.4397888183594\\
3.5699999332428	-121.729751586914\\
3.57500004768372	-152.286956787109\\
3.57999992370605	-159.213775634766\\
3.58500003814697	-155.379470825195\\
3.58999991416931	-139.743118286133\\
3.59500002861023	-105.49259185791\\
3.59999990463257	-63.0412330627441\\
3.60500001907349	-30.6141567230225\\
3.60999989509583	-12.0886754989624\\
3.61500000953674	-1.43712675571442\\
3.61999988555908	5.49920177459717\\
3.625	3.69800066947937\\
3.63000011444092	-9.04775047302246\\
3.63499999046326	-22.988883972168\\
3.64000010490417	-43.4775848388672\\
3.64499998092651	-68.9833908081055\\
3.65000009536743	-98.8760452270508\\
3.65499997138977	-111.223014831543\\
3.66000008583069	-109.68775177002\\
3.66499996185303	-101.199493408203\\
3.67000007629395	-84.9814453125\\
3.67499995231628	-62.9077529907227\\
3.6800000667572	-39.999641418457\\
3.68499994277954	-20.9755954742432\\
3.69000005722046	-8.75592994689941\\
3.6949999332428	-4.36308479309082\\
3.70000004768372	-11.6246700286865\\
3.70499992370605	-22.2716522216797\\
3.71000003814697	-37.8199729919434\\
3.71499991416931	-53.2630844116211\\
3.72000002861023	-64.0873260498047\\
3.72499990463257	-70.2666244506836\\
3.73000001907349	-68.6162719726563\\
3.73499989509583	-62.2708435058594\\
3.74000000953674	-51.0326385498047\\
3.74499988555908	-39.2165794372559\\
3.75	-34.182788848877\\
3.75500011444092	-36.9032325744629\\
3.75999999046326	-43.3618850708008\\
3.76500010490417	-44.9692344665527\\
3.76999998092651	-39.6648635864258\\
3.77500009536743	-22.3899250030518\\
3.77999997138977	-3.40385484695435\\
3.78500008583069	18.1456928253174\\
3.78999996185303	34.3882827758789\\
3.79500007629395	48.0960693359375\\
3.79999995231628	65.659294128418\\
3.8050000667572	86.4572906494141\\
3.80999994277954	105.485748291016\\
3.81500005722046	122.049690246582\\
3.8199999332428	134.60466003418\\
3.82500004768372	142.868896484375\\
3.82999992370605	143.757843017578\\
3.83500003814697	124.239448547363\\
3.83999991416931	71.7889633178711\\
3.84500002861023	-8.16433620452881\\
3.84999990463257	-84.6396942138672\\
3.85500001907349	-148.228118896484\\
3.85999989509583	-185.398544311523\\
3.86500000953674	-205.884048461914\\
3.86999988555908	-207.949890136719\\
3.875	-185.498718261719\\
3.88000011444092	-142.639846801758\\
3.88499999046326	-74.9395065307617\\
3.89000010490417	-21.9056568145752\\
3.89499998092651	11.0731353759766\\
3.90000009536743	18.2450523376465\\
3.90499997138977	27.9109935760498\\
3.91000008583069	21.7685279846191\\
3.91499996185303	0.543567717075348\\
3.92000007629395	-23.0082988739014\\
3.92499995231628	-48.9377708435059\\
3.9300000667572	-90.3586654663086\\
3.93499994277954	-121.678398132324\\
3.94000005722046	-133.7509765625\\
3.9449999332428	-127.65699005127\\
3.95000004768372	-112.690460205078\\
3.95499992370605	-95.4495849609375\\
3.96000003814697	-67.5022659301758\\
3.96499991416931	-38.5432472229004\\
3.97000002861023	-15.1662893295288\\
3.97499990463257	-1.73937201499939\\
3.98000001907349	0.624071776866913\\
3.98499989509583	-6.99707078933716\\
3.99000000953674	-22.7828750610352\\
3.99499988555908	-41.545955657959\\
4	-60.0545387268066\\
4.00500011444092	-76.4256286621094\\
4.01000022888184	-82.3410873413086\\
4.0149998664856	-80.4898223876953\\
4.01999998092651	-71.9059295654297\\
4.02500009536743	-59.4264488220215\\
4.03000020980835	-50.0856666564941\\
4.03499984741211	-50.2240562438965\\
4.03999996185303	-51.4793663024902\\
4.04500007629395	-50.2138061523438\\
4.05000019073486	-44.589656829834\\
4.05499982833862	-30.9448070526123\\
4.05999994277954	-8.26015758514404\\
4.06500005722046	17.9640827178955\\
4.07000017166138	34.8593368530273\\
4.07499980926514	47.8174667358398\\
4.07999992370605	69.7057647705078\\
4.08500003814697	98.9518508911133\\
4.09000015258789	135.005828857422\\
4.09499979019165	183.484527587891\\
4.09999990463257	248.699371337891\\
4.10500001907349	340.966613769531\\
4.1100001335144	391.853424072266\\
4.11499977111816	307.794067382813\\
4.11999988555908	160.158676147461\\
4.125	-77.6216354370117\\
4.13000011444092	-281.109344482422\\
4.13500022888184	-378.767486572266\\
4.1399998664856	-463.724822998047\\
4.14499998092651	-476.510223388672\\
4.15000009536743	-415.598358154297\\
4.15500020980835	-290.623138427734\\
4.15999984741211	-126.665489196777\\
4.16499996185303	47.2221832275391\\
4.17000007629395	103.280059814453\\
4.17500019073486	162.240905761719\\
4.17999982833862	191.293014526367\\
4.18499994277954	170.503173828125\\
4.19000005722046	108.238967895508\\
4.19500017166138	24.1740398406982\\
4.19999980926514	-46.1535263061523\\
4.20499992370605	-174.812393188477\\
4.21000003814697	-245.997283935547\\
4.21500015258789	-263.455108642578\\
4.21999979019165	-235.002914428711\\
4.22499990463257	-176.810684204102\\
4.23000001907349	-115.182014465332\\
4.2350001335144	-56.5103797912598\\
4.23999977111816	7.77544069290161\\
4.24499988555908	60.6030883789063\\
4.25	81.5526962280273\\
4.25500011444092	75.9844055175781\\
4.26000022888184	43.7802467346191\\
4.2649998664856	-5.14458417892456\\
4.26999998092651	-57.8499221801758\\
4.27500009536743	-106.930946350098\\
4.28000020980835	-150.673110961914\\
4.28499984741211	-152.025177001953\\
4.28999996185303	-131.506805419922\\
4.29500007629395	-105.216644287109\\
4.30000019073486	-95.0152740478516\\
4.30499982833862	-92.4687881469727\\
4.30999994277954	-77.2370834350586\\
4.31500005722046	-54.4755210876465\\
4.32000017166138	-26.7450313568115\\
4.32499980926514	7.85210418701172\\
4.32999992370605	59.2006683349609\\
4.33500003814697	65.4519882202148\\
4.34000015258789	89.8332748413086\\
4.34499979019165	129.429122924805\\
4.34999990463257	177.763961791992\\
4.35500001907349	238.15251159668\\
4.3600001335144	311.636016845703\\
4.36499977111816	405.786834716797\\
4.36999988555908	430.92919921875\\
4.375	306.891082763672\\
4.38000011444092	185.994430541992\\
4.38500022888184	51.6766357421875\\
4.3899998664856	-61.4022636413574\\
4.39499998092651	-444.662292480469\\
4.40000009536743	-619.291931152344\\
4.40500020980835	-673.772155761719\\
4.40999984741211	-607.684936523438\\
4.41499996185303	-446.025207519531\\
4.42000007629395	-225.877349853516\\
4.42500019073486	54.182258605957\\
4.42999982833862	165.785125732422\\
4.43499994277954	250.551162719727\\
4.44000005722046	295.184539794922\\
4.44500017166138	274.62353515625\\
4.44999980926514	193.747833251953\\
4.45499992370605	63.1106719970703\\
4.46000003814697	-33.7482986450195\\
4.46500015258789	-212.449676513672\\
4.46999979019165	-322.13134765625\\
4.47499990463257	-344.980712890625\\
4.48000001907349	-308.039794921875\\
4.4850001335144	-225.492752075195\\
4.48999977111816	-133.602111816406\\
4.49499988555908	-60.4618110656738\\
4.5	27.058012008667\\
4.50500011444092	95.721809387207\\
4.51000022888184	126.573310852051\\
4.5149998664856	116.819610595703\\
4.51999998092651	74.9822692871094\\
4.52500009536743	7.43256187438965\\
4.53000020980835	-60.9916572570801\\
4.53499984741211	-127.454765319824\\
4.53999996185303	-185.968063354492\\
4.54500007629395	-185.797210693359\\
4.55000019073486	-76.9229736328125\\
4.55499982833862	-137.916595458984\\
4.55999994277954	-67.4502334594727\\
4.56500005722046	-34.5819435119629\\
4.57000017166138	-33.5239639282227\\
4.57499980926514	-32.4691886901855\\
4.57999992370605	-25.5347709655762\\
4.58500003814697	-11.0998964309692\\
4.59000015258789	0.38956680893898\\
4.59499979019165	2.19761157035828\\
4.59999990463257	19.3560409545898\\
4.60500001907349	45.1311683654785\\
4.6100001335144	83.7130661010742\\
4.61499977111816	138.130355834961\\
4.61999988555908	581.015075683594\\
4.625	951.618591308594\\
4.63000011444092	768.353698730469\\
4.63500022888184	465.117279052734\\
4.6399998664856	182.676391601563\\
4.64499998092651	-67.062629699707\\
4.65000009536743	-168.698043823242\\
4.65500020980835	-292.281555175781\\
4.65999984741211	-424.781372070313\\
4.66499996185303	-1563.60510253906\\
4.67000007629395	-1232.42980957031\\
4.67500019073486	-843.112976074219\\
4.67999982833862	-410.703704833984\\
4.68499994277954	130.662078857422\\
4.69000005722046	361.108581542969\\
4.69500017166138	556.967956542969\\
4.69999980926514	656.151672363281\\
4.70499992370605	622.481994628906\\
4.71000003814697	461.769317626953\\
4.71500015258789	205.935745239258\\
4.71999979019165	-9.66576862335205\\
4.72499990463257	-375.624877929688\\
4.73000001907349	-570.075866699219\\
4.7350001335144	-615.955993652344\\
4.73999977111816	-530.493530273438\\
4.74499988555908	-358.737396240234\\
4.75	-154.540481567383\\
4.75500011444092	-27.1109104156494\\
4.76000022888184	128.742324829102\\
4.7649998664856	254.339752197266\\
4.76999998092651	289.636535644531\\
4.77500009536743	270.805389404297\\
4.78000020980835	176.25407409668\\
4.78499984741211	19.0316162109375\\
4.78999996185303	-125.765014648438\\
4.79500007629395	-160.653015136719\\
4.80000019073486	-330.470794677734\\
4.80499982833862	-335.209838867188\\
4.80999994277954	-229.607421875\\
4.81500005722046	-151.603500366211\\
4.82000017166138	-107.942199707031\\
4.82499980926514	-84.7577819824219\\
4.82999992370605	-60.6765022277832\\
4.83500003814697	-33.4261779785156\\
4.84000015258789	-6.9842963218689\\
4.84499979019165	46.2858428955078\\
4.84999990463257	121.118675231934\\
4.85500001907349	220.703369140625\\
4.8600001335144	325.965057373047\\
4.86499977111816	328.858795166016\\
4.86999988555908	671.771728515625\\
4.875	795.963806152344\\
4.88000011444092	666.44091796875\\
4.88500022888184	399.039276123047\\
4.8899998664856	118.607452392578\\
4.89499998092651	-119.647148132324\\
4.90000009536743	-188.464813232422\\
4.90500020980835	-345.53369140625\\
4.90999984741211	-484.039642333984\\
4.91499996185303	-571.661376953125\\
4.92000007629395	-1644.92016601563\\
4.92500019073486	-1435.18566894531\\
4.92999982833862	-907.728515625\\
4.93499994277954	-221.183044433594\\
4.94000005722046	445.023773193359\\
4.94500017166138	631.130920410156\\
4.94999980926514	804.221435546875\\
4.95499992370605	827.114318847656\\
4.96000003814697	690.484680175781\\
4.96500015258789	434.514099121094\\
4.96999979019165	116.544982910156\\
4.97499990463257	-157.967971801758\\
4.98000001907349	-441.885070800781\\
4.9850001335144	-598.732421875\\
4.98999977111816	-615.53759765625\\
4.99499988555908	-530.545776367188\\
5	-381.1171875\\
5.00500011444092	-201.684051513672\\
5.01000022888184	-60.5940742492676\\
5.0149998664856	56.1062049865723\\
5.01999998092651	163.785720825195\\
5.02500009536743	195.055801391602\\
5.03000020980835	185.674743652344\\
5.03499984741211	150.662216186523\\
5.03999996185303	118.275505065918\\
5.04500007629395	59.0451545715332\\
5.05000019073486	-6.05179405212402\\
5.05499982833862	-60.7524108886719\\
5.05999994277954	-101.020164489746\\
5.06500005722046	-124.313751220703\\
5.07000017166138	-131.979583740234\\
5.07499980926514	-120.507011413574\\
5.07999992370605	-92.8657150268555\\
5.08500003814697	39.0706481933594\\
5.09000015258789	70.7804870605469\\
5.09499979019165	80.0646438598633\\
5.09999990463257	84.6129455566406\\
5.10500001907349	41.0854530334473\\
5.1100001335144	17.5698394775391\\
5.11499977111816	11.7627754211426\\
5.11999988555908	9.20272636413574\\
5.125	7.3898401260376\\
5.13000011444092	6.44980430603027\\
5.13500022888184	748.470947265625\\
5.1399998664856	441.30615234375\\
5.14499998092651	191.84423828125\\
5.15000009536743	7.42612648010254\\
5.15500020980835	-110.586364746094\\
5.15999984741211	-584.902587890625\\
5.16499996185303	-625.113342285156\\
5.17000007629395	-555.882446289063\\
5.17500019073486	-407.662231445313\\
5.17999982833862	-203.331207275391\\
5.18499994277954	31.953031539917\\
5.19000005722046	141.638778686523\\
5.19500017166138	184.993621826172\\
5.19999980926514	215.496307373047\\
5.20499992370605	204.254989624023\\
5.21000003814697	156.393157958984\\
5.21500015258789	82.8946228027344\\
5.21999979019165	2.68057036399841\\
5.22499990463257	-80.9041976928711\\
5.23000001907349	-131.971237182617\\
5.2350001335144	-166.151947021484\\
5.23999977111816	-181.25634765625\\
5.24499988555908	-173.102127075195\\
5.25	-142.563339233398\\
5.25500011444092	-106.277626037598\\
5.26000022888184	-77.0334625244141\\
5.2649998664856	-37.3304901123047\\
5.26999998092651	11.2813301086426\\
5.27500009536743	44.6792869567871\\
5.28000020980835	65.8787841796875\\
5.28499984741211	75.8940124511719\\
5.28999996185303	74.9989166259766\\
5.29500007629395	61.6468849182129\\
5.30000019073486	43.5833969116211\\
5.30499982833862	13.9148168563843\\
5.30999994277954	-25.0657825469971\\
5.31500005722046	-59.2501258850098\\
5.32000017166138	-107.573112487793\\
5.32499980926514	-125.243743896484\\
5.32999992370605	-121.705505371094\\
5.33500003814697	-106.973602294922\\
5.34000015258789	-82.5293197631836\\
5.34499979019165	-60.9695129394531\\
5.34999990463257	-27.7831134796143\\
5.35500001907349	5.665940284729\\
5.3600001335144	33.6964225769043\\
5.36499977111816	48.2119560241699\\
5.36999988555908	47.9822463989258\\
5.375	30.6706600189209\\
5.38000011444092	2.77146887779236\\
5.38500022888184	-30.3063831329346\\
5.3899998664856	-63.4832611083984\\
5.39499998092651	-93.5070190429688\\
5.40000009536743	-92.784912109375\\
5.40500020980835	-76.3569107055664\\
5.40999984741211	-56.1524620056152\\
5.41499996185303	-25.2175025939941\\
5.42000007629395	8.67980289459229\\
5.42500019073486	37.6033248901367\\
5.42999982833862	40.2908325195313\\
5.43499994277954	-8.2971248626709\\
5.44000005722046	-16.0979251861572\\
5.44500017166138	2.30508017539978\\
5.44999980926514	10.9622707366943\\
5.45499992370605	9.97532844543457\\
5.46000003814697	8.02516746520996\\
5.46500015258789	6.58876085281372\\
5.46999979019165	5.71935749053955\\
5.47499990463257	5.01303482055664\\
5.48000001907349	4.61994743347168\\
5.4850001335144	4.05342817306519\\
5.48999977111816	3.77002692222595\\
5.49499988555908	3.39646553993225\\
5.5	3.17238688468933\\
5.50500011444092	2.87662482261658\\
5.51000022888184	2.59146618843079\\
5.5149998664856	2.4270920753479\\
5.51999998092651	60.906795501709\\
5.52500009536743	92.0366668701172\\
5.53000020980835	52.7229118347168\\
5.53499984741211	31.6120204925537\\
5.53999996185303	17.6270618438721\\
5.54500007629395	7.39584827423096\\
5.55000019073486	-2.40246510505676\\
5.55499982833862	-9.23948192596436\\
5.55999994277954	-11.7948541641235\\
5.56500005722046	-15.4166202545166\\
5.57000017166138	-21.1391983032227\\
5.57499980926514	-25.5018215179443\\
5.57999992370605	-28.565990447998\\
5.58500003814697	-30.2435836791992\\
5.59000015258789	-30.8941478729248\\
5.59499979019165	-29.9755325317383\\
5.59999990463257	-27.4028339385986\\
5.60500001907349	-23.7292327880859\\
5.6100001335144	-19.3899040222168\\
5.61499977111816	-14.7171649932861\\
5.61999988555908	-10.0264539718628\\
5.625	-6.11586284637451\\
5.63000011444092	-2.81905341148376\\
5.63500022888184	-0.537835419178009\\
5.6399998664856	2.05950736999512\\
5.64499998092651	2.84848928451538\\
5.65000009536743	2.47568726539612\\
5.65500020980835	2.04721236228943\\
5.65999984741211	1.63944888114929\\
5.66499996185303	1.39680206775665\\
5.67000007629395	-13.8791913986206\\
5.67500019073486	-340.531616210938\\
5.67999982833862	-148.129852294922\\
5.68499994277954	-15.5909261703491\\
5.69000005722046	220.664169311523\\
5.69500017166138	178.802825927734\\
5.69999980926514	151.339004516602\\
5.70499992370605	105.909240722656\\
5.71000003814697	37.8231735229492\\
5.71500015258789	-31.8729228973389\\
5.71999979019165	-85.6692276000977\\
5.72499990463257	-126.647567749023\\
5.73000001907349	-138.451507568359\\
5.7350001335144	-122.689903259277\\
5.73999977111816	-91.971076965332\\
5.74499988555908	-53.762020111084\\
5.75	-20.2753372192383\\
5.75500011444092	1.64878833293915\\
5.76000022888184	13.3413457870483\\
5.7649998664856	7.02879810333252\\
5.76999998092651	-5.77943563461304\\
5.77500009536743	-20.8818607330322\\
5.78000020980835	-29.588586807251\\
5.78499984741211	-36.3721199035645\\
5.78999996185303	-38.9720268249512\\
5.79500007629395	-39.4348793029785\\
5.80000019073486	-36.4678573608398\\
5.80499982833862	-36.4052886962891\\
5.80999994277954	-35.3007469177246\\
5.81500005722046	-35.4036521911621\\
5.82000017166138	-36.5016937255859\\
5.82499980926514	-41.5310249328613\\
5.82999992370605	-43.4133834838867\\
5.83500003814697	-46.6579246520996\\
5.84000015258789	-50.4043273925781\\
5.84499979019165	-50.7922019958496\\
5.84999990463257	-48.9157409667969\\
5.85500001907349	-50.9209594726563\\
5.8600001335144	-35.788646697998\\
5.86499977111816	-5.35012340545654\\
5.86999988555908	42.7052192687988\\
5.875	113.765487670898\\
5.88000011444092	201.803451538086\\
5.88500022888184	301.543823242188\\
5.8899998664856	330.532775878906\\
5.89499998092651	136.204772949219\\
5.90000009536743	115.034156799316\\
5.90500020980835	199.801162719727\\
5.90999984741211	359.977691650391\\
5.91499996185303	475.528930664063\\
5.92000007629395	492.048614501953\\
5.92500019073486	429.827941894531\\
5.92999982833862	323.266967773438\\
5.93499994277954	203.167663574219\\
5.94000005722046	88.2342376708984\\
5.94500017166138	-13.935188293457\\
5.94999980926514	-104.215667724609\\
5.95499992370605	-179.824905395508\\
5.96000003814697	-183.621932983398\\
5.96500015258789	-256.713409423828\\
5.96999979019165	-363.269927978516\\
5.97499990463257	-481.529998779297\\
5.98000001907349	-2907.19091796875\\
5.9850001335144	-2658.66479492188\\
5.98999977111816	-2240.22241210938\\
5.99499988555908	-1337.58996582031\\
6	-370.604858398438\\
6.00500011444092	584.669250488281\\
6.01000022888184	1059.32299804688\\
6.0149998664856	1441.328125\\
6.01999998092651	1607.94641113281\\
6.02500009536743	1497.6826171875\\
6.03000020980835	1165.986328125\\
6.03499984741211	666.494262695313\\
6.03999996185303	94.0979843139648\\
6.04500007629395	-485.711578369141\\
6.05000019073486	-974.115234375\\
6.05499982833862	-1222.73303222656\\
6.05999994277954	-1254.69604492188\\
6.06500005722046	-1107.10021972656\\
6.07000017166138	-838.098754882813\\
6.07499980926514	-501.897521972656\\
6.07999992370605	-203.465026855469\\
6.08500003814697	2.56846570968628\\
6.09000015258789	307.797576904297\\
6.09499979019165	475.259643554688\\
6.09999990463257	521.30322265625\\
6.10500001907349	476.227996826172\\
6.1100001335144	354.482116699219\\
6.11499977111816	229.21208190918\\
6.11999988555908	26.5346260070801\\
6.125	-181.268569946289\\
6.13000011444092	-404.747680664063\\
6.13500022888184	-497.442047119141\\
6.1399998664856	-501.139709472656\\
6.14499998092651	-433.604644775391\\
6.15000009536743	-314.776824951172\\
6.15500020980835	-200.099395751953\\
6.15999984741211	-80.3473587036133\\
6.16499996185303	51.128345489502\\
6.17000007629395	161.022918701172\\
6.17500019073486	237.068481445313\\
6.17999982833862	259.373413085938\\
6.18499994277954	239.099807739258\\
6.19000005722046	52.0889739990234\\
6.19500017166138	-31.4697933197021\\
6.19999980926514	-58.7176361083984\\
6.20499992370605	-75.869499206543\\
6.21000003814697	-78.2522277832031\\
6.21500015258789	-9.96695518493652\\
6.21999979019165	17.3176651000977\\
6.22499990463257	17.7446136474609\\
6.23000001907349	14.200927734375\\
6.2350001335144	11.7567720413208\\
6.23999977111816	10.0371017456055\\
6.24499988555908	9.0802640914917\\
6.25	8.31701469421387\\
6.25500011444092	7.66044950485229\\
6.26000022888184	7.08915615081787\\
6.2649998664856	6.57299566268921\\
6.26999998092651	6.09237670898438\\
6.27500009536743	5.63673734664917\\
6.28000020980835	5.2023458480835\\
6.28499984741211	4.7940993309021\\
6.28999996185303	4.40888118743896\\
6.29500007629395	4.04901885986328\\
6.30000019073486	3.71347308158875\\
6.30499982833862	3.4002628326416\\
6.30999994277954	3.11009240150452\\
6.31500005722046	2.84359550476074\\
6.32000017166138	2.59424161911011\\
6.32499980926514	2.36848640441895\\
6.32999992370605	2.15840172767639\\
6.33500003814697	1.96527457237244\\
6.34000015258789	1.78999674320221\\
6.34499979019165	1.63126146793365\\
6.34999990463257	1.48545145988464\\
6.35500001907349	1.35241174697876\\
6.3600001335144	-250.175155639648\\
6.36499977111816	-165.823822021484\\
6.36999988555908	-102.576705932617\\
6.375	-57.4641189575195\\
6.38000011444092	-16.3203926086426\\
6.38500022888184	98.7949295043945\\
6.3899998664856	482.505645751953\\
6.39499998092651	338.057250976563\\
6.40000009536743	225.081115722656\\
6.40500020980835	126.35147857666\\
6.40999984741211	8.58591842651367\\
6.41499996185303	-76.6278991699219\\
6.42000007629395	-140.016784667969\\
6.42500019073486	-182.834350585938\\
6.42999982833862	-196.992645263672\\
6.43499994277954	-191.41389465332\\
6.44000005722046	-165.937942504883\\
6.44500017166138	-132.367126464844\\
6.44999980926514	-92.9573211669922\\
6.45499992370605	-59.1136436462402\\
6.46000003814697	-30.9091396331787\\
6.46500015258789	-11.258692741394\\
6.46999979019165	-2.84001064300537\\
6.47499990463257	-4.9290657043457\\
6.48000001907349	-14.7512702941895\\
6.4850001335144	-31.6175746917725\\
6.48999977111816	-53.4755973815918\\
6.49499988555908	-74.0279006958008\\
6.5	-91.6268157958984\\
6.50500011444092	-106.078163146973\\
6.51000022888184	-117.232856750488\\
6.5149998664856	-123.3037109375\\
6.51999998092651	-124.067863464355\\
6.52500009536743	-121.854988098145\\
6.53000020980835	-115.566902160645\\
6.53499984741211	-108.391777038574\\
6.53999996185303	-101.246376037598\\
6.54500007629395	-94.1808242797852\\
6.55000019073486	-89.1364212036133\\
6.55499982833862	-85.5359497070313\\
6.55999994277954	-85.3120193481445\\
6.56500005722046	-83.0149383544922\\
6.57000017166138	-82.5510330200195\\
6.57499980926514	-82.2707138061523\\
6.57999992370605	-85.4965209960938\\
6.58500003814697	-87.4103088378906\\
6.59000015258789	-89.8245849609375\\
6.59499979019165	-92.2504425048828\\
6.59999990463257	-93.8864822387695\\
6.60500001907349	-94.2859954833984\\
6.6100001335144	-94.047004699707\\
6.61499977111816	-93.1148300170898\\
6.61999988555908	-90.6109466552734\\
6.625	-87.9543228149414\\
6.63000011444092	-84.4718856811523\\
6.63500022888184	-81.071907043457\\
6.6399998664856	-78.1970748901367\\
6.64499998092651	-75.2683868408203\\
6.65000009536743	-72.3857192993164\\
6.65500020980835	-71.2767028808594\\
6.65999984741211	-69.9436264038086\\
6.66499996185303	-69.0633239746094\\
6.67000007629395	-69.2986679077148\\
6.67500019073486	-69.3954772949219\\
6.67999982833862	-69.5675659179688\\
6.68499994277954	-68.2391662597656\\
6.69000005722046	-67.4737701416016\\
6.69500017166138	-65.6576538085938\\
6.69999980926514	-65.7463607788086\\
6.70499992370605	-64.1753311157227\\
6.71000003814697	-62.7102165222168\\
6.71500015258789	-59.9565582275391\\
6.71999979019165	-57.8708953857422\\
6.72499990463257	-56.0607757568359\\
6.73000001907349	-54.1276359558105\\
6.7350001335144	-52.6232833862305\\
6.73999977111816	-53.1980819702148\\
6.74499988555908	-53.6018257141113\\
6.75	-55.3937301635742\\
6.75500011444092	-56.7442283630371\\
6.76000022888184	-55.413646697998\\
6.7649998664856	-59.8268051147461\\
6.76999998092651	-61.9290199279785\\
6.77500009536743	-64.8375396728516\\
6.78000020980835	-68.3451309204102\\
6.78499984741211	-70.3053665161133\\
6.78999996185303	-73.2327117919922\\
6.79500007629395	-74.8271026611328\\
6.80000019073486	-76.0106811523438\\
6.80499982833862	-77.0954284667969\\
6.80999994277954	-77.7495422363281\\
6.81500005722046	-77.8669281005859\\
6.82000017166138	-78.370361328125\\
6.82499980926514	-78.5121612548828\\
6.82999992370605	-78.5357437133789\\
6.83500003814697	-77.6890716552734\\
6.84000015258789	-79.1863250732422\\
6.84499979019165	-80.1343383789063\\
6.84999990463257	-81.6513519287109\\
6.85500001907349	-83.4194717407227\\
6.8600001335144	-84.8801879882813\\
6.86499977111816	-88.2360610961914\\
6.86999988555908	-89.6398773193359\\
6.875	-89.2545318603516\\
6.88000011444092	-89.8138046264648\\
6.88500022888184	-89.0442733764648\\
6.8899998664856	-87.530143737793\\
6.89499998092651	-85.150634765625\\
6.90000009536743	-82.8896636962891\\
6.90500020980835	-78.9899520874023\\
6.90999984741211	-76.1637268066406\\
6.91499996185303	-73.565673828125\\
6.92000007629395	-71.0704727172852\\
6.92500019073486	-69.0712051391602\\
6.92999982833862	-67.2712020874023\\
6.93499994277954	-66.2253265380859\\
6.94000005722046	-65.2654037475586\\
6.94500017166138	-64.5704498291016\\
6.94999980926514	-64.0214767456055\\
6.95499992370605	-64.7174987792969\\
6.96000003814697	-63.4573135375977\\
6.96500015258789	-60.6885452270508\\
6.96999979019165	-59.9624786376953\\
6.97499990463257	-57.1593360900879\\
6.98000001907349	-54.5569801330566\\
6.9850001335144	-51.0115737915039\\
6.98999977111816	-47.9908218383789\\
6.99499988555908	-45.2037506103516\\
7	-41.3852081298828\\
7.00500011444092	-38.6814765930176\\
7.01000022888184	-36.1723022460938\\
7.0149998664856	-34.1169853210449\\
7.01999998092651	-32.3555221557617\\
7.02500009536743	-30.6504802703857\\
7.03000020980835	-29.1709861755371\\
7.03499984741211	-27.8571300506592\\
7.03999996185303	-26.568696975708\\
7.04500007629395	-25.2334690093994\\
7.05000019073486	-23.7822799682617\\
7.05499982833862	-22.1150054931641\\
7.05999994277954	-20.5214672088623\\
7.06500005722046	-18.6462001800537\\
7.07000017166138	-16.4475193023682\\
7.07499980926514	-14.5174760818481\\
7.07999992370605	-12.607611656189\\
7.08500003814697	-10.746955871582\\
7.09000015258789	-9.04902267456055\\
7.09499979019165	-7.56588363647461\\
7.09999990463257	-6.45881986618042\\
7.10500001907349	-5.65747833251953\\
7.1100001335144	-5.11534738540649\\
7.11499977111816	-4.69054508209229\\
7.11999988555908	-4.48982667922974\\
7.125	-4.54343843460083\\
7.13000011444092	-4.41485404968262\\
7.13500022888184	-4.31278228759766\\
7.1399998664856	-4.21164417266846\\
7.14499998092651	-3.93857479095459\\
7.15000009536743	-3.49181222915649\\
7.15500020980835	-3.03806710243225\\
7.15999984741211	-2.15489792823792\\
7.16499996185303	-1.19422781467438\\
7.17000007629395	-1.78254330158234\\
7.17500019073486	-2.13619804382324\\
7.17999982833862	-2.32416534423828\\
7.18499994277954	-2.8091254234314\\
7.19000005722046	-3.39181160926819\\
7.19500017166138	-4.04292964935303\\
7.19999980926514	-4.7252368927002\\
7.20499992370605	-5.46477508544922\\
7.21000003814697	-6.06003189086914\\
7.21500015258789	-6.60831165313721\\
7.21999979019165	-7.29661321640015\\
7.22499990463257	-7.75509357452393\\
7.23000001907349	-8.25128078460693\\
7.2350001335144	-8.77478885650635\\
7.23999977111816	-9.26737499237061\\
7.24499988555908	-9.81850719451904\\
7.25	-10.4029960632324\\
7.25500011444092	-10.964316368103\\
7.26000022888184	-11.5981769561768\\
7.2649998664856	-12.2391471862793\\
7.26999998092651	-12.8603506088257\\
7.27500009536743	-13.5621519088745\\
7.28000020980835	-14.2512140274048\\
7.28499984741211	-14.9091606140137\\
7.28999996185303	-15.5152902603149\\
7.29500007629395	-16.0887355804443\\
7.30000019073486	-16.7013206481934\\
7.30499982833862	-17.2641792297363\\
7.30999994277954	-17.6931896209717\\
7.31500005722046	-18.0921974182129\\
7.32000017166138	-18.4610767364502\\
7.32499980926514	-18.7956027984619\\
7.32999992370605	-19.1133060455322\\
7.33500003814697	-19.4001598358154\\
7.34000015258789	-19.6229095458984\\
7.34499979019165	-19.8066902160645\\
7.34999990463257	-19.9816856384277\\
7.35500001907349	-20.1170215606689\\
7.3600001335144	-20.1798458099365\\
7.36499977111816	-20.1990852355957\\
7.36999988555908	-20.2160034179688\\
7.375	-20.2695865631104\\
7.38000011444092	-20.3027515411377\\
7.38500022888184	-20.3291206359863\\
7.3899998664856	-20.3563022613525\\
7.39499998092651	-20.3653182983398\\
7.40000009536743	-20.2081050872803\\
7.40500020980835	-20.0102195739746\\
7.40999984741211	-19.78515625\\
7.41499996185303	-19.5192813873291\\
7.42000007629395	-19.2107028961182\\
7.42500019073486	-18.8641223907471\\
7.42999982833862	-18.4978847503662\\
7.43499994277954	-18.1528358459473\\
7.44000005722046	-17.7997493743896\\
7.44500017166138	-17.4311351776123\\
7.44999980926514	-17.0551700592041\\
7.45499992370605	-16.7058734893799\\
7.46000003814697	-16.3979663848877\\
7.46500015258789	-16.0798492431641\\
7.46999979019165	-15.6650886535645\\
7.47499990463257	-15.2127723693848\\
7.48000001907349	-14.8121576309204\\
7.4850001335144	-14.4021406173706\\
7.48999977111816	-13.9844417572021\\
7.49499988555908	-13.5712070465088\\
7.5	-13.1654024124146\\
7.50500011444092	-12.7680540084839\\
7.51000022888184	-12.3775777816772\\
7.5149998664856	-11.9740600585938\\
7.51999998092651	-11.5521240234375\\
7.52500009536743	-10.8862438201904\\
7.53000020980835	-10.8876352310181\\
7.53499984741211	-10.5413618087769\\
7.53999996185303	-10.2417325973511\\
7.54500007629395	-10.0415830612183\\
7.55000019073486	-9.8089771270752\\
7.55499982833862	-9.5618724822998\\
7.55999994277954	-9.34471035003662\\
7.56500005722046	-9.13197422027588\\
7.57000017166138	-8.90838623046875\\
7.57499980926514	-8.6939001083374\\
7.57999992370605	-8.50143337249756\\
7.58500003814697	-8.31596565246582\\
7.59000015258789	-8.16996765136719\\
7.59499979019165	-8.04574489593506\\
7.59999990463257	-7.93646621704102\\
7.60500001907349	-7.84551668167114\\
7.6100001335144	-7.78087282180786\\
7.61499977111816	-7.74306678771973\\
7.61999988555908	-7.80382394790649\\
7.625	-7.98944997787476\\
7.63000011444092	-8.13584232330322\\
7.63500022888184	-8.18032169342041\\
7.6399998664856	-8.26635932922363\\
7.64499998092651	-8.3705587387085\\
7.65000009536743	-8.44377040863037\\
7.65500020980835	-8.51553344726563\\
7.65999984741211	-8.62055969238281\\
7.66499996185303	-8.73126983642578\\
7.67000007629395	-8.84105682373047\\
7.67500019073486	-8.9715518951416\\
7.67999982833862	-9.14244365692139\\
7.68499994277954	-9.34283447265625\\
7.69000005722046	-9.56765270233154\\
7.69500017166138	-9.75711345672607\\
7.69999980926514	-9.94559478759766\\
7.70499992370605	-10.1405801773071\\
7.71000003814697	-10.3570108413696\\
7.71500015258789	-10.6388645172119\\
7.71999979019165	-10.8833141326904\\
7.72499990463257	-11.0641412734985\\
7.73000001907349	-11.2357521057129\\
7.7350001335144	-11.3951787948608\\
7.73999977111816	-11.6593999862671\\
7.74499988555908	-11.8806552886963\\
7.75	-12.0069484710693\\
7.75500011444092	-12.1741962432861\\
7.76000022888184	-12.3890047073364\\
7.7649998664856	-12.5723323822021\\
7.76999998092651	-12.7355346679688\\
7.77500009536743	-12.8901290893555\\
7.78000020980835	-13.0409679412842\\
7.78499984741211	-13.2222242355347\\
7.78999996185303	-13.3791170120239\\
7.79500007629395	-13.5214462280273\\
7.80000019073486	-13.6451797485352\\
7.80499982833862	-13.7617435455322\\
7.80999994277954	-13.8705654144287\\
7.81500005722046	-13.9668140411377\\
7.82000017166138	-14.053129196167\\
7.82499980926514	-14.1318054199219\\
7.82999992370605	-14.2035160064697\\
7.83500003814697	-14.2689428329468\\
7.84000015258789	-14.3257637023926\\
7.84499979019165	-14.3724584579468\\
7.84999990463257	-14.3992490768433\\
7.85500001907349	-14.4193849563599\\
7.8600001335144	-14.4316148757935\\
7.86499977111816	-14.4515361785889\\
7.86999988555908	-14.4635286331177\\
7.875	-14.4707584381104\\
7.88000011444092	-14.4622745513916\\
7.88500022888184	-14.4443988800049\\
7.8899998664856	-14.4219856262207\\
7.89499998092651	-14.3989219665527\\
7.90000009536743	-14.3769083023071\\
7.90500020980835	-14.3505439758301\\
7.90999984741211	-14.321063041687\\
7.91499996185303	-14.2874374389648\\
7.92000007629395	-14.2522621154785\\
7.92500019073486	-14.2144317626953\\
7.92999982833862	-14.160418510437\\
7.93499994277954	-14.1130275726318\\
7.94000005722046	-14.0700645446777\\
7.94500017166138	-14.0056438446045\\
7.94999980926514	-13.9448099136353\\
7.95499992370605	-13.8938226699829\\
7.96000003814697	-13.8177652359009\\
7.96500015258789	-13.7307376861572\\
7.96999979019165	-13.6632633209229\\
7.97499990463257	-13.6273384094238\\
7.98000001907349	-13.6367340087891\\
7.9850001335144	-13.6373357772827\\
7.98999977111816	-13.6358947753906\\
7.99499988555908	-13.5976123809814\\
8	-13.5823774337769\\
8.00500011444092	-13.5805282592773\\
8.01000022888184	-13.5565185546875\\
8.01500034332275	-13.53648853302\\
8.02000045776367	-13.5218725204468\\
8.02499961853027	-13.5439834594727\\
8.02999973297119	-13.5924425125122\\
8.03499984741211	-13.6491842269897\\
8.03999996185303	-13.7161407470703\\
8.04500007629395	-13.7831392288208\\
8.05000019073486	-13.865138053894\\
8.05500030517578	-13.9585447311401\\
8.0600004196167	-14.0568418502808\\
8.0649995803833	-14.1662521362305\\
8.06999969482422	-14.2845544815063\\
8.07499980926514	-14.3614854812622\\
8.07999992370605	-14.4116315841675\\
8.08500003814697	-14.5499134063721\\
8.09000015258789	-14.7002868652344\\
8.09500026702881	-14.8826532363892\\
8.10000038146973	-15.0795984268188\\
8.10499954223633	-15.263503074646\\
8.10999965667725	-15.430775642395\\
8.11499977111816	-15.6129264831543\\
8.11999988555908	-15.8134613037109\\
8.125	-16.0530014038086\\
8.13000011444092	-16.2991523742676\\
8.13500022888184	-16.5562744140625\\
8.14000034332275	-16.7842864990234\\
8.14500045776367	-17.0399379730225\\
8.14999961853027	-17.2852897644043\\
8.15499973297119	-17.5898017883301\\
8.15999984741211	-17.9295406341553\\
8.16499996185303	-18.2592544555664\\
8.17000007629395	-18.5309810638428\\
8.17500019073486	-18.7466697692871\\
8.18000030517578	-18.9731769561768\\
8.1850004196167	-19.2813835144043\\
8.1899995803833	-19.7753009796143\\
8.19499969482422	-20.2699947357178\\
8.19999980926514	-20.7308139801025\\
8.20499992370605	-21.1372356414795\\
8.21000003814697	-21.5426330566406\\
8.21500015258789	-22.0086898803711\\
8.22000026702881	-22.5278091430664\\
8.22500038146973	-23.0665130615234\\
8.22999954223633	-23.1964244842529\\
8.23499965667725	-23.007495880127\\
8.23999977111816	-23.4367771148682\\
8.24499988555908	-24.5353298187256\\
8.25	-25.2799663543701\\
8.25500011444092	-25.7587471008301\\
8.26000022888184	-26.285774230957\\
8.26500034332275	-26.8591842651367\\
8.27000045776367	-27.2122764587402\\
8.27499961853027	-27.8408641815186\\
8.27999973297119	-28.5735664367676\\
8.28499984741211	-29.3363990783691\\
8.28999996185303	-29.7437000274658\\
8.29500007629395	-30.3936710357666\\
8.30000019073486	-31.280345916748\\
8.30500030517578	-32.518238067627\\
8.3100004196167	-33.6948127746582\\
8.3149995803833	-30.8896942138672\\
8.31999969482422	-33.9257507324219\\
8.32499980926514	-35.0950202941895\\
8.32999992370605	-36.0108413696289\\
8.33500003814697	-36.6950187683105\\
8.34000015258789	-37.6475219726563\\
8.34500026702881	-38.8607368469238\\
8.35000038146973	-39.8007583618164\\
8.35499954223633	-41.6637954711914\\
8.35999965667725	-41.3576202392578\\
8.36499977111816	-41.951602935791\\
8.36999988555908	-44.1320991516113\\
8.375	-45.5329551696777\\
8.38000011444092	-46.4396171569824\\
8.38500022888184	-48.1774673461914\\
8.39000034332275	-49.6286010742188\\
8.39500045776367	-52.0827369689941\\
8.39999961853027	-50.6286087036133\\
8.40499973297119	-54.0938034057617\\
8.40999984741211	-56.1442260742188\\
8.41499996185303	-57.6547241210938\\
8.42000007629395	-59.7113952636719\\
8.42500019073486	-61.7994117736816\\
8.43000030517578	-64.6729125976563\\
8.4350004196167	-67.0723648071289\\
8.4399995803833	-68.9904174804688\\
8.44499969482422	-73.2056732177734\\
8.44999980926514	-76.7226028442383\\
8.45499992370605	-79.9718475341797\\
8.46000003814697	-84.6045837402344\\
8.46500015258789	-87.1878662109375\\
8.47000026702881	-90.3312149047852\\
8.47500038146973	-92.8887710571289\\
8.47999954223633	-95.0513458251953\\
8.48499965667725	-97.8896408081055\\
8.48999977111816	-101.013763427734\\
8.49499988555908	-101.554092407227\\
8.5	-103.489349365234\\
8.50500011444092	-104.263977050781\\
8.51000022888184	-105.761581420898\\
8.51500034332275	-105.841255187988\\
8.52000045776367	-105.13501739502\\
8.52499961853027	-110.273330688477\\
8.52999973297119	-107.836570739746\\
8.53499984741211	-107.197219848633\\
8.53999996185303	-108.201728820801\\
8.54500007629395	-108.720718383789\\
8.55000019073486	-107.95467376709\\
8.55500030517578	-107.337310791016\\
8.5600004196167	-106.741142272949\\
8.5649995803833	-105.121795654297\\
8.56999969482422	-103.175605773926\\
8.57499980926514	-101.633003234863\\
8.57999992370605	-99.7385635375977\\
8.58500003814697	-97.6778793334961\\
8.59000015258789	-94.7087936401367\\
8.59500026702881	-90.6217193603516\\
8.60000038146973	-86.0642929077148\\
8.60499954223633	-82.0311431884766\\
8.60999965667725	-78.016960144043\\
8.61499977111816	-75.063591003418\\
8.61999988555908	-73.6740264892578\\
8.625	-74.5128707885742\\
8.63000011444092	-74.4255447387695\\
8.63500022888184	-75.4203033447266\\
8.64000034332275	-78.1567306518555\\
8.64500045776367	-80.5382461547852\\
8.64999961853027	-84.3108062744141\\
8.65499973297119	-84.9388885498047\\
8.65999984741211	-89.6879348754883\\
8.66499996185303	-87.7810134887695\\
8.67000007629395	-87.2417068481445\\
8.67500019073486	-84.5578994750977\\
8.68000030517578	-82.0516128540039\\
8.6850004196167	-78.3447418212891\\
8.6899995803833	-73.818115234375\\
8.69499969482422	-68.9547729492188\\
8.69999980926514	-65.5996246337891\\
8.70499992370605	-62.7362213134766\\
8.71000003814697	-60.2813453674316\\
8.71500015258789	-58.3472061157227\\
8.72000026702881	-58.4279327392578\\
8.72500038146973	-58.510799407959\\
8.72999954223633	-58.8168525695801\\
8.73499965667725	-59.6776161193848\\
8.73999977111816	-60.5244522094727\\
8.74499988555908	-60.6555557250977\\
8.75	-59.937671661377\\
8.75500011444092	-57.2514419555664\\
8.76000022888184	-55.9737358093262\\
8.76500034332275	-52.7933197021484\\
8.77000045776367	-49.7405128479004\\
8.77499961853027	-46.4350166320801\\
8.77999973297119	-43.4934234619141\\
8.78499984741211	-40.7035636901855\\
8.78999996185303	-38.1531791687012\\
8.79500007629395	-36.1837501525879\\
8.80000019073486	-34.8062286376953\\
8.80500030517578	-33.9750785827637\\
8.8100004196167	-33.611213684082\\
8.8149995803833	-33.6474227905273\\
8.81999969482422	-33.7586669921875\\
8.82499980926514	-33.8596420288086\\
8.82999992370605	-33.7957763671875\\
8.83500003814697	-33.3197937011719\\
8.84000015258789	-32.4399566650391\\
8.84500026702881	-31.2264041900635\\
8.85000038146973	-29.5492095947266\\
8.85499954223633	-27.8737506866455\\
8.85999965667725	-26.1143474578857\\
8.86499977111816	-24.499641418457\\
8.86999988555908	-23.1590538024902\\
8.875	-22.2251586914063\\
8.88000011444092	-21.7416114807129\\
8.88500022888184	-21.6998176574707\\
8.89000034332275	-22.0221748352051\\
8.89500045776367	-22.5407276153564\\
8.89999961853027	-23.3888206481934\\
8.90499973297119	-24.2076988220215\\
8.90999984741211	-24.5988025665283\\
8.91499996185303	-24.2389354705811\\
8.92000007629395	-24.2933235168457\\
8.92500019073486	-24.2031555175781\\
8.93000030517578	-23.5627689361572\\
8.9350004196167	-22.8697166442871\\
8.9399995803833	-22.2938117980957\\
8.94499969482422	-21.9860305786133\\
8.94999980926514	-21.9293041229248\\
8.95499992370605	-21.9541702270508\\
8.96000003814697	-22.1977062225342\\
8.96500015258789	-22.6798324584961\\
8.97000026702881	-23.3719291687012\\
8.97500038146973	-23.9822177886963\\
8.97999954223633	-24.5164222717285\\
8.98499965667725	-25.5063076019287\\
8.98999977111816	-25.867280960083\\
8.99499988555908	-25.8519611358643\\
9	-26.4973964691162\\
9.00500011444092	-26.8773250579834\\
9.01000022888184	-27.0890941619873\\
9.01500034332275	-27.2921524047852\\
9.02000045776367	-27.5780296325684\\
9.02499961853027	-27.8538303375244\\
9.02999973297119	-28.1319427490234\\
9.03499984741211	-28.4599151611328\\
9.03999996185303	-28.8200397491455\\
9.04500007629395	-29.2859001159668\\
9.05000019073486	-29.6128482818604\\
9.05500030517578	-29.9920387268066\\
9.0600004196167	-30.364143371582\\
9.0649995803833	-30.7460899353027\\
9.06999969482422	-31.0781021118164\\
9.07499980926514	-31.3939208984375\\
9.07999992370605	-31.6912364959717\\
9.08500003814697	-31.9697723388672\\
9.09000015258789	-32.1848258972168\\
9.09500026702881	-32.2345733642578\\
9.10000038146973	-32.2243537902832\\
9.10499954223633	-32.4123382568359\\
9.10999965667725	-32.5252990722656\\
9.11499977111816	-32.4302787780762\\
9.11999988555908	-32.2335739135742\\
9.125	-32.1366157531738\\
9.13000011444092	-32.2194404602051\\
9.13500022888184	-32.2752265930176\\
9.14000034332275	-32.4481811523438\\
9.14500045776367	-32.612247467041\\
9.14999961853027	-32.7661628723145\\
9.15499973297119	-32.8733367919922\\
9.15999984741211	-33.011344909668\\
9.16499996185303	-32.9518127441406\\
9.17000007629395	-32.6488761901855\\
9.17500019073486	-32.4307022094727\\
9.18000030517578	-32.0625076293945\\
9.1850004196167	-31.6085147857666\\
9.1899995803833	-31.1405143737793\\
9.19499969482422	-30.6523704528809\\
9.19999980926514	-30.1570949554443\\
9.20499992370605	-29.681676864624\\
9.21000003814697	-29.3031730651855\\
9.21500015258789	-29.1813621520996\\
9.22000026702881	-29.0230751037598\\
9.22500038146973	-28.8929290771484\\
9.22999954223633	-28.7570838928223\\
9.23499965667725	-28.2659015655518\\
9.23999977111816	-26.7460346221924\\
9.24499988555908	-23.4218311309814\\
9.25	-20.1067066192627\\
9.25500011444092	-18.256196975708\\
9.26000022888184	-17.3741912841797\\
9.26500034332275	-15.5252504348755\\
9.27000045776367	-16.1432647705078\\
9.27499961853027	-16.4008941650391\\
9.27999973297119	-16.5886650085449\\
9.28499984741211	-17.321605682373\\
9.28999996185303	-17.8704605102539\\
9.29500007629395	-18.503719329834\\
9.30000019073486	-19.7352771759033\\
9.30500030517578	-14.9105014801025\\
9.3100004196167	-18.2308330535889\\
9.3149995803833	-20.4579181671143\\
9.31999969482422	-20.8517303466797\\
9.32499980926514	-24.6870155334473\\
9.32999992370605	-25.4606590270996\\
9.33500003814697	-31.0602416992188\\
9.34000015258789	-38.5276184082031\\
9.34500026702881	-37.3390197753906\\
9.35000038146973	-50.8148727416992\\
9.35499954223633	-64.3407745361328\\
9.35999965667725	-67.0304870605469\\
9.36499977111816	-78.1961898803711\\
9.36999988555908	-88.4529876708984\\
9.375	-93.7301635742188\\
9.38000011444092	-97.0962219238281\\
9.38500022888184	-97.4111022949219\\
9.39000034332275	-94.9663391113281\\
9.39500045776367	-90.457160949707\\
9.39999961853027	-89.2077331542969\\
9.40499973297119	-86.6408462524414\\
9.40999984741211	-85.5136413574219\\
9.41499996185303	-87.0093383789063\\
9.42000007629395	-90.9815826416016\\
9.42500019073486	-95.0530014038086\\
9.43000030517578	-79.5319213867188\\
9.4350004196167	-87.7497253417969\\
9.4399995803833	-111.577285766602\\
9.44499969482422	-149.582382202148\\
9.44999980926514	-213.280685424805\\
9.45499992370605	-301.946136474609\\
9.46000003814697	-429.222595214844\\
9.46500015258789	-576.716186523438\\
9.47000026702881	-743.229064941406\\
9.47500038146973	-862.354370117188\\
9.47999954223633	-924.174987792969\\
9.48499965667725	-636.008850097656\\
9.48999977111816	15.6056232452393\\
9.49499988555908	48.8626136779785\\
9.5	37.1080513000488\\
9.50500011444092	55.7896919250488\\
9.51000022888184	83.6374740600586\\
9.51500034332275	112.709098815918\\
9.52000045776367	144.341125488281\\
9.52499961853027	166.971084594727\\
9.52999973297119	206.287780761719\\
9.53499984741211	230.978576660156\\
9.53999996185303	248.172241210938\\
9.54500007629395	260.558013916016\\
9.55000019073486	259.253387451172\\
9.55500030517578	84.4689865112305\\
9.5600004196167	38.1589851379395\\
9.5649995803833	23.858850479126\\
9.56999969482422	15.6441555023193\\
9.57499980926514	10.4192447662354\\
9.57999992370605	7.57255029678345\\
9.58500003814697	5.9530930519104\\
9.59000015258789	4.71750640869141\\
9.59500026702881	4.37753200531006\\
9.60000038146973	4.02047681808472\\
9.60499954223633	4.05087280273438\\
9.60999965667725	3.55894899368286\\
9.61499977111816	3.59658360481262\\
9.61999988555908	3.45499539375305\\
9.625	3.26063060760498\\
9.63000011444092	3.0604350566864\\
9.63500022888184	2.88341951370239\\
9.64000034332275	2.66923141479492\\
9.64500045776367	2.46668744087219\\
9.64999961853027	2.3266875743866\\
9.65499973297119	2.15743446350098\\
9.65999984741211	1.97153997421265\\
9.66499996185303	1.82898128032684\\
9.67000007629395	1.68805658817291\\
9.67500019073486	1.55027091503143\\
9.68000030517578	1.42689251899719\\
9.6850004196167	1.30935478210449\\
9.6899995803833	1.20004296302795\\
9.69499969482422	1.09780025482178\\
9.69999980926514	1.00173020362854\\
9.70499992370605	0.911303400993347\\
9.71000003814697	0.83192378282547\\
9.71500015258789	0.760906279087067\\
9.72000026702881	0.693352580070496\\
9.72500038146973	0.634108185768127\\
9.72999954223633	0.578986048698425\\
9.73499965667725	0.52499794960022\\
9.73999977111816	0.471761971712112\\
9.74499988555908	0.417894870042801\\
9.75	0.387889117002487\\
9.75500011444092	0.3592269718647\\
9.76000022888184	0.336023420095444\\
9.76500034332275	0.308292031288147\\
9.77000045776367	0.27248477935791\\
9.77499961853027	0.244513645768166\\
9.77999973297119	0.217883467674255\\
9.78499984741211	0.203294664621353\\
9.78999996185303	0.18369697034359\\
9.79500007629395	0.165025860071182\\
9.80000019073486	0.148112624883652\\
9.80500030517578	0.133823186159134\\
9.8100004196167	0.122983172535896\\
9.8149995803833	0.11418940871954\\
9.81999969482422	0.101894319057465\\
9.82499980926514	0.0902877151966095\\
9.82999992370605	0.079167403280735\\
9.83500003814697	0.0717973262071609\\
9.84000015258789	0.0691528618335724\\
9.84500026702881	0.0671179890632629\\
9.85000038146973	0.0660061091184616\\
9.85499954223633	0.0500053912401199\\
9.85999965667725	0.0401887781918049\\
9.86499977111816	0.0314257219433784\\
9.86999988555908	0.0299698673188686\\
9.875	0.0338802710175514\\
9.88000011444092	0.034393522888422\\
9.88500022888184	0.0346009582281113\\
9.89000034332275	0.0225945189595222\\
9.89500045776367	0.0169859174638987\\
9.89999961853027	0.0120021039620042\\
9.90499973297119	0.0146207334473729\\
9.90999984741211	0.0217965003103018\\
9.91499996185303	0.02450031042099\\
9.92000007629395	0.0266738776117563\\
9.92500019073486	0.00957426149398088\\
9.93000030517578	0.0035283078905195\\
9.9350004196167	-0.00360343256033957\\
9.9399995803833	0.000791096070315689\\
9.94499969482422	0.00736136129125953\\
9.94999980926514	0.00566989509388804\\
9.95499992370605	0.00419089570641518\\
9.96000003814697	0.00347472750581801\\
9.96500015258789	0.00293309474363923\\
9.97000026702881	0.00245896331034601\\
9.97500038146973	0.0014169542118907\\
9.97999954223633	0.000858244544360787\\
9.98499965667725	0.00169340672437102\\
9.98999977111816	0.00195145548786968\\
9.99499988555908	0.0032641536090523\\
10	0.00388786126859486\\
};
\addlegendentry{CF}

\end{axis}
\end{tikzpicture}%
    \end{tikzpicture}}
    \caption{Loads at point A of CF under PD Feedback Control}
    \label{fig:pureFeedbkPDA}
\end{figure}

\begin{figure}[h!]
    \centering
    \scalebox{0.8}{
    \begin{tikzpicture}
        % This file was created by matlab2tikz.
%
%The latest updates can be retrieved from
%  http://www.mathworks.com/matlabcentral/fileexchange/22022-matlab2tikz-matlab2tikz
%where you can also make suggestions and rate matlab2tikz.
%
\begin{tikzpicture}

\begin{axis}[%
width=4.521in,
height=1.476in,
at={(0.758in,2.571in)},
scale only axis,
xmin=0,
xmax=10,
xlabel style={font=\color{white!15!black}},
xlabel={Time (s)},
ymin=-500,
ymax=1017.21508789063,
ylabel style={font=\color{white!15!black}},
ylabel={FX (N)},
axis background/.style={fill=white},
xmajorgrids,
ymajorgrids,
legend style={legend cell align=left, align=left, draw=white!15!black}
]
\addplot [color=black, dashed, line width=2.0pt]
  table[row sep=crcr]{%
0.0949999988079071	2.00570869445801\\
0.100000001490116	1.89491093158722\\
0.104999996721745	1.80077695846558\\
0.109999999403954	1.72017312049866\\
0.115000002086163	1.65045356750488\\
0.119999997317791	3.31705713272095\\
0.125	9.30066204071045\\
0.129999995231628	12.6379013061523\\
0.135000005364418	15.031044960022\\
0.140000000596046	16.648458480835\\
0.144999995827675	17.653865814209\\
0.150000005960464	18.2128238677979\\
0.155000001192093	18.4613170623779\\
0.159999996423721	18.6960277557373\\
0.165000006556511	18.8919277191162\\
0.170000001788139	18.8828105926514\\
0.174999997019768	18.6068382263184\\
0.180000007152557	18.0122547149658\\
0.185000002384186	17.12184715271\\
0.189999997615814	15.8765964508057\\
0.194999992847443	219.090148925781\\
0.200000002980232	344.076721191406\\
0.204999998211861	414.820587158203\\
0.209999993443489	441.413787841797\\
0.215000003576279	436.748596191406\\
0.219999998807907	431.400604248047\\
0.224999994039536	399.495666503906\\
0.230000004172325	346.075988769531\\
0.234999999403954	281.704620361328\\
0.239999994635582	219.60368347168\\
0.245000004768372	172.671371459961\\
0.25	155.471939086914\\
0.254999995231628	187.228012084961\\
0.259999990463257	240.139022827148\\
0.264999985694885	300.491485595703\\
0.270000010728836	357.91162109375\\
0.275000005960464	404.962158203125\\
0.280000001192093	436.891387939453\\
0.284999996423721	453.309051513672\\
0.28999999165535	454.831726074219\\
0.294999986886978	448.932006835938\\
0.300000011920929	436.77587890625\\
0.305000007152557	414.071258544922\\
0.310000002384186	384.729400634766\\
0.314999997615814	353.256378173828\\
0.319999992847443	324.247436523438\\
0.324999988079071	301.532501220703\\
0.330000013113022	287.603790283203\\
0.33500000834465	283.388427734375\\
0.340000003576279	291.562408447266\\
0.344999998807907	301.522674560547\\
0.349999994039536	310.160217285156\\
0.354999989271164	315.057922363281\\
0.360000014305115	314.628509521484\\
0.365000009536743	309.861785888672\\
0.370000004768372	302.006530761719\\
0.375	288.123168945313\\
0.379999995231628	269.010589599609\\
0.384999990463257	246.183563232422\\
0.389999985694885	221.722351074219\\
0.395000010728836	197.795837402344\\
0.400000005960464	176.876647949219\\
0.405000001192093	159.266265869141\\
0.409999996423721	147.165252685547\\
0.41499999165535	139.638900756836\\
0.419999986886978	136.132537841797\\
0.425000011920929	136.124496459961\\
0.430000007152557	135.614196777344\\
0.435000002384186	133.431045532227\\
0.439999997615814	130.358978271484\\
0.444999992847443	124.595504760742\\
0.449999988079071	115.962257385254\\
0.455000013113022	104.813758850098\\
0.46000000834465	91.9112014770508\\
0.465000003576279	78.3416442871094\\
0.469999998807907	65.2714233398438\\
0.474999994039536	53.9079856872559\\
0.479999989271164	45.1242752075195\\
0.485000014305115	39.4535484313965\\
0.490000009536743	36.9612197875977\\
0.495000004768372	38.5592346191406\\
0.5	41.5151519775391\\
0.504999995231628	44.6676292419434\\
0.509999990463257	47.298412322998\\
0.514999985694885	48.9513702392578\\
0.519999980926514	49.4721794128418\\
0.524999976158142	49.0688781738281\\
0.529999971389771	48.5936164855957\\
0.535000026226044	47.3401718139648\\
0.540000021457672	45.8112945556641\\
0.545000016689301	44.5954322814941\\
0.550000011920929	44.1814727783203\\
0.555000007152557	45.6632995605469\\
0.560000002384186	48.5185279846191\\
0.564999997615814	52.5384635925293\\
0.569999992847443	57.499942779541\\
0.574999988079071	63.0450630187988\\
0.579999983310699	68.898551940918\\
0.584999978542328	74.821044921875\\
0.589999973773956	80.583381652832\\
0.595000028610229	86.1193542480469\\
0.600000023841858	91.3947601318359\\
0.605000019073486	96.4056015014648\\
0.610000014305115	101.28466796875\\
0.615000009536743	106.092948913574\\
0.620000004768372	110.959754943848\\
0.625	115.920127868652\\
0.629999995231628	121.076271057129\\
0.634999990463257	126.39249420166\\
0.639999985694885	131.905288696289\\
0.644999980926514	137.56364440918\\
0.649999976158142	143.291564941406\\
0.654999971389771	149.016189575195\\
0.660000026226044	154.655792236328\\
0.665000021457672	160.132339477539\\
0.670000016689301	165.384033203125\\
0.675000011920929	170.356903076172\\
0.680000007152557	175.0224609375\\
0.685000002384186	179.378692626953\\
0.689999997615814	183.433135986328\\
0.694999992847443	187.201950073242\\
0.699999988079071	190.703399658203\\
0.704999983310699	193.976486206055\\
0.709999978542328	197.024200439453\\
0.714999973773956	199.815460205078\\
0.720000028610229	202.375762939453\\
0.725000023841858	204.742172241211\\
0.730000019073486	206.754989624023\\
0.735000014305115	208.55110168457\\
0.740000009536743	210.014190673828\\
0.745000004768372	211.199264526367\\
0.75	212.053085327148\\
0.754999995231628	212.609558105469\\
0.759999990463257	212.830596923828\\
0.764999985694885	212.833297729492\\
0.769999980926514	212.652404785156\\
0.774999976158142	212.186752319336\\
0.779999971389771	211.404907226563\\
0.785000026226044	210.340942382813\\
0.790000021457672	209.024978637695\\
0.795000016689301	207.478149414063\\
0.800000011920929	205.718704223633\\
0.805000007152557	203.764343261719\\
0.810000002384186	201.632858276367\\
0.814999997615814	199.341842651367\\
0.819999992847443	196.907501220703\\
0.824999988079071	194.344421386719\\
0.829999983310699	191.654678344727\\
0.834999978542328	188.854125976563\\
0.839999973773956	185.962921142578\\
0.845000028610229	182.991775512695\\
0.850000023841858	179.946563720703\\
0.855000019073486	176.84602355957\\
0.860000014305115	173.730514526367\\
0.865000009536743	170.618316650391\\
0.870000004768372	167.517166137695\\
0.875	164.446578979492\\
0.879999995231628	161.419296264648\\
0.884999990463257	158.44596862793\\
0.889999985694885	155.53776550293\\
0.894999980926514	152.706619262695\\
0.899999976158142	149.962982177734\\
0.904999971389771	147.316909790039\\
0.910000026226044	144.780075073242\\
0.915000021457672	142.357696533203\\
0.920000016689301	140.046691894531\\
0.925000011920929	137.860107421875\\
0.930000007152557	135.80322265625\\
0.935000002384186	133.874160766602\\
0.939999997615814	132.081512451172\\
0.944999992847443	130.434814453125\\
0.949999988079071	128.94938659668\\
0.954999983310699	127.629615783691\\
0.959999978542328	126.474990844727\\
0.964999973773956	125.484062194824\\
0.970000028610229	124.664642333984\\
0.975000023841858	124.009674072266\\
0.980000019073486	123.518074035645\\
0.985000014305115	123.191101074219\\
0.990000009536743	123.021697998047\\
0.995000004768372	123.010063171387\\
1	123.15705871582\\
1.00499999523163	123.463218688965\\
1.00999999046326	123.922584533691\\
1.01499998569489	124.475692749023\\
1.01999998092651	125.122398376465\\
1.02499997615814	125.876655578613\\
1.02999997138977	126.740592956543\\
1.0349999666214	127.704521179199\\
1.03999996185303	128.767562866211\\
1.04499995708466	129.920043945313\\
1.04999995231628	131.158554077148\\
1.05499994754791	132.473648071289\\
1.05999994277954	133.858856201172\\
1.06500005722046	135.310699462891\\
1.07000005245209	136.816772460938\\
1.07500004768372	138.370498657227\\
1.08000004291534	139.959823608398\\
1.08500003814697	141.57763671875\\
1.0900000333786	143.217056274414\\
1.09500002861023	144.86930847168\\
1.10000002384186	146.526245117188\\
1.10500001907349	148.181045532227\\
1.11000001430511	149.829284667969\\
1.11500000953674	151.461807250977\\
1.12000000476837	153.071334838867\\
1.125	154.654769897461\\
1.12999999523163	156.205368041992\\
1.13499999046326	157.715972900391\\
1.13999998569489	159.181198120117\\
1.14499998092651	160.598907470703\\
1.14999997615814	161.961685180664\\
1.15499997138977	163.264038085938\\
1.1599999666214	164.503723144531\\
1.16499996185303	165.676452636719\\
1.16999995708466	166.77668762207\\
1.17499995231628	167.800857543945\\
1.17999994754791	168.751983642578\\
1.18499994277954	169.623489379883\\
1.19000005722046	170.412155151367\\
1.19500005245209	171.117294311523\\
1.20000004768372	171.738067626953\\
1.20500004291534	172.272079467773\\
1.21000003814697	172.718475341797\\
1.2150000333786	173.082717895508\\
1.22000002861023	173.364059448242\\
1.22500002384186	173.56462097168\\
1.23000001907349	173.690567016602\\
1.23500001430511	173.749771118164\\
1.24000000953674	173.747329711914\\
1.24500000476837	173.690887451172\\
1.25	173.553314208984\\
1.25499999523163	173.345260620117\\
1.25999999046326	173.063842773438\\
1.26499998569489	172.711944580078\\
1.26999998092651	172.29704284668\\
1.27499997615814	171.815002441406\\
1.27999997138977	171.270858764648\\
1.2849999666214	170.682144165039\\
1.28999996185303	170.046203613281\\
1.29499995708466	169.36784362793\\
1.29999995231628	168.656616210938\\
1.30499994754791	167.916122436523\\
1.30999994277954	167.151870727539\\
1.31500005722046	166.366317749023\\
1.32000005245209	165.565078735352\\
1.32500004768372	164.753204345703\\
1.33000004291534	163.932846069336\\
1.33500003814697	163.107727050781\\
1.3400000333786	162.282562255859\\
1.34500002861023	161.461639404297\\
1.35000002384186	160.649383544922\\
1.35500001907349	159.849975585938\\
1.36000001430511	159.066772460938\\
1.36500000953674	158.300933837891\\
1.37000000476837	157.556289672852\\
1.375	156.834777832031\\
1.37999999523163	156.134307861328\\
1.38499999046326	155.458648681641\\
1.38999998569489	154.8095703125\\
1.39499998092651	154.193481445313\\
1.39999997615814	153.611465454102\\
1.40499997138977	153.067001342773\\
1.4099999666214	152.574111938477\\
1.41499996185303	152.134140014648\\
1.41999995708466	151.748840332031\\
1.42499995231628	151.405776977539\\
1.42999994754791	151.111892700195\\
1.43499994277954	150.866088867188\\
1.44000005722046	150.664657592773\\
1.44500005245209	150.508407592773\\
1.45000004768372	150.393188476563\\
1.45500004291534	150.313232421875\\
1.46000003814697	150.267822265625\\
1.4650000333786	150.258651733398\\
1.47000002861023	150.288482666016\\
1.47500002384186	150.357620239258\\
1.48000001907349	150.468887329102\\
1.48500001430511	150.625778198242\\
1.49000000953674	150.832565307617\\
1.49500000476837	151.071243286133\\
1.5	151.343734741211\\
1.50499999523163	151.650650024414\\
1.50999999046326	151.994094848633\\
1.51499998569489	152.371887207031\\
1.51999998092651	152.781005859375\\
1.52499997615814	153.219360351563\\
1.52999997138977	153.685668945313\\
1.5349999666214	154.178085327148\\
1.53999996185303	154.694900512695\\
1.54499995708466	155.231994628906\\
1.54999995231628	155.784713745117\\
1.55499994754791	156.350967407227\\
1.55999994277954	156.924835205078\\
1.56500005722046	157.503295898438\\
1.57000005245209	157.774826049805\\
1.57500004768372	158.106552124023\\
1.58000004291534	158.3349609375\\
1.58500003814697	158.455032348633\\
1.5900000333786	158.562149047852\\
1.59500002861023	158.699279785156\\
1.60000002384186	158.898147583008\\
1.60500001907349	159.194869995117\\
1.61000001430511	159.606918334961\\
1.61500000953674	160.094284057617\\
1.62000000476837	160.673751831055\\
1.625	161.24382019043\\
1.62999999523163	161.758865356445\\
1.63499999046326	162.158660888672\\
1.63999998569489	162.412780761719\\
1.64499998092651	162.656784057617\\
1.64999997615814	162.831069946289\\
1.65499997138977	162.940475463867\\
1.6599999666214	162.999328613281\\
1.66499996185303	163.016067504883\\
1.66999995708466	162.98420715332\\
1.67499995231628	162.943710327148\\
1.67999994754791	162.927841186523\\
1.68499994277954	162.903411865234\\
1.69000005722046	162.856079101563\\
1.69500005245209	162.7939453125\\
1.70000004768372	162.716583251953\\
1.70500004291534	162.610885620117\\
1.71000003814697	162.475875854492\\
1.7150000333786	162.318954467773\\
1.72000002861023	162.143142700195\\
1.72500002384186	161.93359375\\
1.73000001907349	161.695770263672\\
1.73500001430511	161.426498413086\\
1.74000000953674	161.126281738281\\
1.74500000476837	160.798095703125\\
1.75	160.44758605957\\
1.75499999523163	160.071670532227\\
1.75999999046326	159.675231933594\\
1.76499998569489	159.257369995117\\
1.76999998092651	158.820266723633\\
1.77499997615814	158.372222900391\\
1.77999997138977	157.913665771484\\
1.7849999666214	157.442138671875\\
1.78999996185303	156.959197998047\\
1.79499995708466	156.46598815918\\
1.79999995231628	155.962387084961\\
1.80499994754791	155.449401855469\\
1.80999994277954	154.926483154297\\
1.81500005722046	154.394012451172\\
1.82000005245209	153.852462768555\\
1.82500004768372	153.302459716797\\
1.83000004291534	152.744293212891\\
1.83500003814697	152.177612304688\\
1.8400000333786	151.602447509766\\
1.84500002861023	151.019653320313\\
1.85000002384186	150.437591552734\\
1.85500001907349	149.85205078125\\
1.86000001430511	149.26286315918\\
1.86500000953674	148.669616699219\\
1.87000000476837	148.072555541992\\
1.875	147.479461669922\\
1.87999999523163	146.896377563477\\
1.88499999046326	146.318405151367\\
1.88999998569489	145.744964599609\\
1.89499998092651	145.157241821289\\
1.89999997615814	144.565887451172\\
1.90499997138977	143.970886230469\\
1.9099999666214	143.372360229492\\
1.91499996185303	142.770263671875\\
1.91999995708466	142.164611816406\\
1.92499995231628	141.560745239258\\
1.92999994754791	140.958267211914\\
1.93499994277954	140.355850219727\\
1.94000005722046	139.750228881836\\
1.94500005245209	139.14030456543\\
1.95000004768372	138.527603149414\\
1.95500004291534	137.917007446289\\
1.96000003814697	137.31608581543\\
1.9650000333786	136.724044799805\\
1.97000002861023	136.132263183594\\
1.97500002384186	135.510513305664\\
1.98000001907349	134.866073608398\\
1.98500001430511	134.196853637695\\
1.99000000953674	133.522857666016\\
1.99500000476837	132.833511352539\\
2	132.129302978516\\
2.00500011444092	131.42951965332\\
2.00999999046326	130.730575561523\\
2.01500010490417	130.036224365234\\
2.01999998092651	129.383316040039\\
2.02500009536743	128.774993896484\\
2.02999997138977	128.2177734375\\
2.03500008583069	127.676963806152\\
2.03999996185303	127.168975830078\\
2.04500007629395	126.719451904297\\
2.04999995231628	126.360618591309\\
2.0550000667572	126.091903686523\\
2.05999994277954	125.908462524414\\
2.06500005722046	125.818687438965\\
2.0699999332428	125.80638885498\\
2.07500004768372	125.87580871582\\
2.07999992370605	126.029640197754\\
2.08500003814697	126.265434265137\\
2.08999991416931	126.572875976563\\
2.09500002861023	126.948692321777\\
2.09999990463257	127.376487731934\\
2.10500001907349	127.850875854492\\
2.10999989509583	128.378768920898\\
2.11500000953674	128.961013793945\\
2.11999988555908	129.599960327148\\
2.125	130.301727294922\\
2.13000011444092	131.032516479492\\
2.13499999046326	131.775741577148\\
2.14000010490417	132.493850708008\\
2.14499998092651	133.170043945313\\
2.15000009536743	133.862121582031\\
2.15499997138977	134.45329284668\\
2.16000008583069	134.768341064453\\
2.16499996185303	135.033630371094\\
2.17000007629395	135.239944458008\\
2.17499995231628	135.273025512695\\
2.1800000667572	135.223388671875\\
2.18499994277954	135.129776000977\\
2.19000005722046	135.054779052734\\
2.1949999332428	134.975708007813\\
2.20000004768372	134.901565551758\\
2.20499992370605	134.839263916016\\
2.21000003814697	134.788131713867\\
2.21499991416931	134.749771118164\\
2.22000002861023	134.729904174805\\
2.22499990463257	134.736328125\\
2.23000001907349	134.707595825195\\
2.23499989509583	134.540588378906\\
2.24000000953674	134.401718139648\\
2.24499988555908	134.282104492188\\
2.25	134.147979736328\\
2.25500011444092	134.042846679688\\
2.25999999046326	134.017303466797\\
2.26500010490417	134.150909423828\\
2.26999998092651	134.494918823242\\
2.27500009536743	135.060546875\\
2.27999997138977	135.807434082031\\
2.28500008583069	136.623138427734\\
2.28999996185303	137.544418334961\\
2.29500007629395	138.577682495117\\
2.29999995231628	139.680999755859\\
2.3050000667572	140.865921020508\\
2.30999994277954	142.122497558594\\
2.31500005722046	143.438583374023\\
2.3199999332428	144.79997253418\\
2.32500004768372	146.195892333984\\
2.32999992370605	147.614242553711\\
2.33500003814697	149.040710449219\\
2.33999991416931	150.500869750977\\
2.34500002861023	152.005706787109\\
2.34999990463257	153.503601074219\\
2.35500001907349	155.006195068359\\
2.35999989509583	156.482315063477\\
2.36500000953674	157.882125854492\\
2.36999988555908	159.16975402832\\
2.375	160.356262207031\\
2.38000011444092	161.427124023438\\
2.38499999046326	162.318450927734\\
2.39000010490417	163.002487182617\\
2.39499998092651	163.45654296875\\
2.40000009536743	163.692352294922\\
2.40499997138977	163.707122802734\\
2.41000008583069	163.545150756836\\
2.41499996185303	163.263473510742\\
2.42000007629395	162.912399291992\\
2.42499995231628	162.618621826172\\
2.4300000667572	162.350814819336\\
2.43499994277954	162.190048217773\\
2.44000005722046	162.169509887695\\
2.4449999332428	162.082275390625\\
2.45000004768372	162.342514038086\\
2.45499992370605	162.674743652344\\
2.46000003814697	163.177856445313\\
2.46499991416931	163.894439697266\\
2.47000002861023	164.759796142578\\
2.47499990463257	165.755233764648\\
2.48000001907349	166.864761352539\\
2.48499989509583	168.053649902344\\
2.49000000953674	169.347946166992\\
2.49499988555908	170.779647827148\\
2.5	172.273483276367\\
2.50500011444092	173.633651733398\\
2.50999999046326	174.84635925293\\
2.51500010490417	176.045303344727\\
2.51999998092651	177.115234375\\
2.52500009536743	178.057846069336\\
2.52999997138977	178.902954101563\\
2.53500008583069	179.663146972656\\
2.53999996185303	180.325653076172\\
2.54500007629395	180.914276123047\\
2.54999995231628	181.396469116211\\
2.5550000667572	181.761993408203\\
2.55999994277954	181.985565185547\\
2.56500005722046	182.030792236328\\
2.5699999332428	181.876556396484\\
2.57500004768372	181.501770019531\\
2.57999992370605	180.904220581055\\
2.58500003814697	180.101623535156\\
2.58999991416931	179.022338867188\\
2.59500002861023	177.815765380859\\
2.59999990463257	176.477096557617\\
2.60500001907349	174.960906982422\\
2.60999989509583	173.335037231445\\
2.61500000953674	171.605209350586\\
2.61999988555908	169.810455322266\\
2.625	167.958755493164\\
2.63000011444092	166.053192138672\\
2.63499999046326	163.998596191406\\
2.64000010490417	161.952224731445\\
2.64499998092651	159.843841552734\\
2.65000009536743	157.453796386719\\
2.65499997138977	154.894332885742\\
2.66000008583069	152.149215698242\\
2.66499996185303	149.137466430664\\
2.67000007629395	145.870101928711\\
2.67499995231628	142.509796142578\\
2.6800000667572	139.177490234375\\
2.68499994277954	135.663070678711\\
2.69000005722046	132.217254638672\\
2.6949999332428	128.853073120117\\
2.70000004768372	125.592864990234\\
2.70499992370605	122.461578369141\\
2.71000003814697	119.448249816895\\
2.71499991416931	116.53092956543\\
2.72000002861023	113.716567993164\\
2.72499990463257	110.989570617676\\
2.73000001907349	108.326904296875\\
2.73499989509583	105.729782104492\\
2.74000000953674	103.199882507324\\
2.74499988555908	100.759735107422\\
2.75	98.4458999633789\\
2.75500011444092	96.3092727661133\\
2.75999999046326	94.4518356323242\\
2.76500010490417	92.9323806762695\\
2.76999998092651	91.7209167480469\\
2.77500009536743	90.7720260620117\\
2.77999997138977	90.0856094360352\\
2.78500008583069	89.6837921142578\\
2.78999996185303	89.5269012451172\\
2.79500007629395	89.5503997802734\\
2.79999995231628	89.6921920776367\\
2.8050000667572	89.9189300537109\\
2.80999994277954	90.1763534545898\\
2.81500005722046	90.4523773193359\\
2.8199999332428	90.7192611694336\\
2.82500004768372	90.9946594238281\\
2.82999992370605	91.2961044311523\\
2.83500003814697	91.7384490966797\\
2.83999991416931	92.3672103881836\\
2.84500002861023	93.2755889892578\\
2.84999990463257	94.610725402832\\
2.85500001907349	96.3661956787109\\
2.85999989509583	98.6754913330078\\
2.86500000953674	101.631713867188\\
2.86999988555908	105.15941619873\\
2.875	109.419845581055\\
2.88000011444092	114.379676818848\\
2.88499999046326	120.082290649414\\
2.89000010490417	126.568962097168\\
2.89499998092651	133.837432861328\\
2.90000009536743	141.877075195313\\
2.90499997138977	150.7158203125\\
2.91000008583069	160.311569213867\\
2.91499996185303	170.60563659668\\
2.92000007629395	181.464080810547\\
2.92499995231628	192.677062988281\\
2.9300000667572	203.995498657227\\
2.93499994277954	215.127868652344\\
2.94000005722046	225.801071166992\\
2.9449999332428	235.7353515625\\
2.95000004768372	244.759841918945\\
2.95499992370605	252.642822265625\\
2.96000003814697	259.387023925781\\
2.96499991416931	265.459930419922\\
2.97000002861023	270.384338378906\\
2.97499990463257	273.899475097656\\
2.98000001907349	276.094268798828\\
2.98499989509583	277.056640625\\
2.99000000953674	276.963836669922\\
2.99499988555908	275.943084716797\\
3	274.181549072266\\
3.00500011444092	271.777557373047\\
3.00999999046326	268.761413574219\\
3.01500010490417	265.107604980469\\
3.01999998092651	260.725341796875\\
3.02500009536743	255.550659179688\\
3.02999997138977	249.405075073242\\
3.03500008583069	242.138290405273\\
3.03999996185303	233.812759399414\\
3.04500007629395	224.477249145508\\
3.04999995231628	214.40510559082\\
3.0550000667572	203.880615234375\\
3.05999994277954	193.030822753906\\
3.06500005722046	182.28759765625\\
3.0699999332428	171.993270874023\\
3.07500004768372	161.910705566406\\
3.07999992370605	152.33479309082\\
3.08500003814697	143.202926635742\\
3.08999991416931	134.318817138672\\
3.09500002861023	125.753257751465\\
3.09999990463257	117.525001525879\\
3.10500001907349	109.51879119873\\
3.10999989509583	101.875938415527\\
3.11500000953674	94.6744842529297\\
3.11999988555908	88.0386276245117\\
3.125	82.2196731567383\\
3.13000011444092	77.6504669189453\\
3.13499999046326	74.8792190551758\\
3.14000010490417	72.9592514038086\\
3.14499998092651	71.5766830444336\\
3.15000009536743	70.3588027954102\\
3.15499997138977	69.1042404174805\\
3.16000008583069	67.7724761962891\\
3.16499996185303	66.4133911132813\\
3.17000007629395	65.0668563842773\\
3.17499995231628	63.7933540344238\\
3.1800000667572	62.9157867431641\\
3.18499994277954	62.6241683959961\\
3.19000005722046	62.8889999389648\\
3.1949999332428	63.4821281433105\\
3.20000004768372	64.567626953125\\
3.20499992370605	66.4948272705078\\
3.21000003814697	69.0147094726563\\
3.21499991416931	72.6471786499023\\
3.22000002861023	77.4419250488281\\
3.22499990463257	83.3674621582031\\
3.23000001907349	90.4710464477539\\
3.23499989509583	98.7725143432617\\
3.24000000953674	108.256622314453\\
3.24499988555908	118.826103210449\\
3.25	130.334289550781\\
3.25500011444092	142.568481445313\\
3.25999999046326	155.304809570313\\
3.26500010490417	168.24089050293\\
3.26999998092651	181.113037109375\\
3.27500009536743	193.607467651367\\
3.27999997138977	205.565841674805\\
3.28500008583069	216.851181030273\\
3.28999996185303	227.350967407227\\
3.29500007629395	237.104934692383\\
3.29999995231628	246.082183837891\\
3.3050000667572	254.222457885742\\
3.30999994277954	261.817626953125\\
3.31500005722046	268.543701171875\\
3.3199999332428	274.487762451172\\
3.32500004768372	279.643280029297\\
3.32999992370605	283.839813232422\\
3.33500003814697	287.012878417969\\
3.33999991416931	289.255279541016\\
3.34500002861023	290.494812011719\\
3.34999990463257	290.290252685547\\
3.35500001907349	288.497985839844\\
3.35999989509583	285.100952148438\\
3.36500000953674	280.066864013672\\
3.36999988555908	273.527740478516\\
3.375	265.655334472656\\
3.38000011444092	256.694122314453\\
3.38499999046326	246.91015625\\
3.39000010490417	236.523071289063\\
3.39499998092651	225.746704101563\\
3.40000009536743	214.725479125977\\
3.40499997138977	203.505508422852\\
3.41000008583069	192.150650024414\\
3.41499996185303	180.676467895508\\
3.42000007629395	169.108734130859\\
3.42499995231628	157.479522705078\\
3.4300000667572	145.906616210938\\
3.43499994277954	134.534759521484\\
3.44000005722046	123.591552734375\\
3.4449999332428	113.228996276855\\
3.45000004768372	103.662063598633\\
3.45499992370605	95.1396331787109\\
3.46000003814697	88.7467498779297\\
3.46499991416931	83.8145751953125\\
3.47000002861023	79.2780914306641\\
3.47499990463257	75.0530395507813\\
3.48000001907349	70.9610137939453\\
3.48499989509583	67.0062789916992\\
3.49000000953674	62.7481002807617\\
3.49499988555908	58.9881362915039\\
3.5	56.1394424438477\\
3.50500011444092	53.7524185180664\\
3.50999999046326	51.8958854675293\\
3.51500010490417	51.0255317687988\\
3.51999998092651	50.5438766479492\\
3.52500009536743	51.0989151000977\\
3.52999997138977	52.1108779907227\\
3.53500008583069	53.7985610961914\\
3.53999996185303	56.3729019165039\\
3.54500007629395	60.3594512939453\\
3.54999995231628	67.6429824829102\\
3.5550000667572	78.8632965087891\\
3.55999994277954	94.018928527832\\
3.56500005722046	113.096382141113\\
3.5699999332428	135.532867431641\\
3.57500004768372	160.10368347168\\
3.57999992370605	185.335830688477\\
3.58500003814697	209.535339355469\\
3.58999991416931	231.335327148438\\
3.59500002861023	249.646957397461\\
3.59999990463257	263.978271484375\\
3.60500001907349	274.106384277344\\
3.60999989509583	281.633026123047\\
3.61500000953674	286.941253662109\\
3.61999988555908	288.723541259766\\
3.625	288.13525390625\\
3.63000011444092	286.387359619141\\
3.63499999046326	284.921539306641\\
3.64000010490417	284.653900146484\\
3.64499998092651	285.968872070313\\
3.65000009536743	289.674987792969\\
3.65499997138977	293.124053955078\\
3.66000008583069	294.872528076172\\
3.66499996185303	294.624969482422\\
3.67000007629395	291.767303466797\\
3.67499995231628	284.890594482422\\
3.6800000667572	273.902374267578\\
3.68499994277954	259.308929443359\\
3.69000005722046	242.195648193359\\
3.6949999332428	223.844055175781\\
3.70000004768372	205.886276245117\\
3.70499992370605	189.238952636719\\
3.71000003814697	175.049942016602\\
3.71499991416931	163.192321777344\\
3.72000002861023	153.426208496094\\
3.72499990463257	145.046417236328\\
3.73000001907349	137.199890136719\\
3.73499989509583	129.143264770508\\
3.74000000953674	120.389381408691\\
3.74499988555908	110.862342834473\\
3.75	102.324325561523\\
3.75500011444092	95.0264587402344\\
3.75999999046326	88.9060974121094\\
3.76500010490417	82.74609375\\
3.76999998092651	76.2732467651367\\
3.77500009536743	70.1796493530273\\
3.77999997138977	64.7241516113281\\
3.78500008583069	60.0224761962891\\
3.78999996185303	57.0250205993652\\
3.79500007629395	55.8133926391602\\
3.79999995231628	55.4323043823242\\
3.8050000667572	55.5197601318359\\
3.80999994277954	55.5367774963379\\
3.81500005722046	55.1324195861816\\
3.8199999332428	54.0803298950195\\
3.82500004768372	52.6568832397461\\
3.82999992370605	51.7557411193848\\
3.83500003814697	52.7793197631836\\
3.83999991416931	58.9007301330566\\
3.84500002861023	72.8449478149414\\
3.84999990463257	92.6566543579102\\
3.85500001907349	117.798820495605\\
3.85999989509583	146.856430053711\\
3.86500000953674	177.524154663086\\
3.86999988555908	207.449554443359\\
3.875	234.53076171875\\
3.88000011444092	257.220581054688\\
3.88499999046326	274.7412109375\\
3.89000010490417	286.929901123047\\
3.89499998092651	294.727844238281\\
3.90000009536743	301.165893554688\\
3.90499997138977	302.965728759766\\
3.91000008583069	302.093566894531\\
3.91499996185303	300.4208984375\\
3.92000007629395	299.456939697266\\
3.92499995231628	300.475738525391\\
3.9300000667572	305.021667480469\\
3.93499994277954	311.170074462891\\
3.94000005722046	316.144866943359\\
3.9449999332428	318.410125732422\\
3.95000004768372	317.401702880859\\
3.95499992370605	313.976898193359\\
3.96000003814697	305.472839355469\\
3.96499991416931	291.767944335938\\
3.97000002861023	273.617645263672\\
3.97499990463257	252.383743286133\\
3.98000001907349	229.831359863281\\
3.98499989509583	207.732009887695\\
3.99000000953674	187.496734619141\\
3.99499988555908	170.03190612793\\
4	155.590805053711\\
4.00500011444092	143.724563598633\\
4.01000022888184	133.593475341797\\
4.0149998664856	124.2041015625\\
4.01999998092651	114.61979675293\\
4.02500009536743	104.521850585938\\
4.03000020980835	94.8766403198242\\
4.03499984741211	86.8752059936523\\
4.03999996185303	79.5986557006836\\
4.04500007629395	72.4842147827148\\
4.05000019073486	65.2733383178711\\
4.05499982833862	58.016227722168\\
4.05999994277954	50.9921989440918\\
4.06500005722046	45.2315673828125\\
4.07000017166138	41.1755561828613\\
4.07499980926514	39.3788604736328\\
4.07999992370605	38.8784790039063\\
4.08500003814697	39.4177703857422\\
4.09000015258789	40.0131225585938\\
4.09499979019165	39.1825675964355\\
4.09999990463257	35.6866264343262\\
4.10500001907349	29.6143989562988\\
4.1100001335144	22.5737609863281\\
4.11499977111816	21.2202205657959\\
4.11999988555908	24.6610298156738\\
4.125	47.4759635925293\\
4.13000011444092	90.599250793457\\
4.13500022888184	141.384918212891\\
4.1399998664856	196.112731933594\\
4.14499998092651	248.35514831543\\
4.15000009536743	292.986602783203\\
4.15500020980835	326.586303710938\\
4.15999984741211	347.723785400391\\
4.16499996185303	356.948486328125\\
4.17000007629395	364.107238769531\\
4.17500019073486	358.292907714844\\
4.17999982833862	342.119445800781\\
4.18499994277954	320.284057617188\\
4.19000005722046	298.914093017578\\
4.19500017166138	283.652709960938\\
4.19999980926514	278.986938476563\\
4.20499992370605	292.368560791016\\
4.21000003814697	311.120269775391\\
4.21500015258789	328.684509277344\\
4.21999979019165	340.260589599609\\
4.22499990463257	342.410919189453\\
4.23000001907349	337.328125\\
4.2350001335144	324.913970947266\\
4.23999977111816	301.547912597656\\
4.24499988555908	268.183715820313\\
4.25	231.28450012207\\
4.25500011444092	193.524826049805\\
4.26000022888184	159.991180419922\\
4.2649998664856	134.11393737793\\
4.26999998092651	117.400039672852\\
4.27500009536743	109.679107666016\\
4.28000020980835	110.867164611816\\
4.28499984741211	111.867980957031\\
4.28999996185303	110.231956481934\\
4.29500007629395	105.295600891113\\
4.30000019073486	98.8982925415039\\
4.30499982833862	90.5534820556641\\
4.30999994277954	77.5890045166016\\
4.31500005722046	62.8579216003418\\
4.32000017166138	49.737735748291\\
4.32499980926514	40.0819511413574\\
4.32999992370605	34.9606246948242\\
4.33500003814697	34.8547325134277\\
4.34000015258789	37.7982406616211\\
4.34499979019165	42.2776069641113\\
4.34999990463257	45.8519897460938\\
4.35500001907349	45.7499847412109\\
4.3600001335144	39.7116508483887\\
4.36499977111816	27.5075912475586\\
4.36999988555908	17.9860610961914\\
4.375	22.6482486724854\\
4.38000011444092	26.5124740600586\\
4.38500022888184	29.0138893127441\\
4.3899998664856	30.1168537139893\\
4.39499998092651	102.447448730469\\
4.40000009536743	176.54817199707\\
4.40500020980835	248.600723266602\\
4.40999984741211	310.420013427734\\
4.41499996185303	356.309356689453\\
4.42000007629395	384.414764404297\\
4.42500019073486	395.693664550781\\
4.42999982833862	402.540496826172\\
4.43499994277954	394.277160644531\\
4.44000005722046	371.044677734375\\
4.44500017166138	339.197174072266\\
4.44999980926514	307.311553955078\\
4.45499992370605	283.979431152344\\
4.46000003814697	275.610595703125\\
4.46500015258789	292.746887207031\\
4.46999979019165	319.512786865234\\
4.47499990463257	344.446960449219\\
4.48000001907349	361.791870117188\\
4.4850001335144	366.702301025391\\
4.48999977111816	359.838134765625\\
4.49499988555908	346.457946777344\\
4.5	318.3408203125\\
4.50500011444092	278.359558105469\\
4.51000022888184	231.810302734375\\
4.5149998664856	184.896041870117\\
4.51999998092651	143.415618896484\\
4.52500009536743	112.345893859863\\
4.53000020980835	93.1472702026367\\
4.53499984741211	86.0232772827148\\
4.53999996185303	90.7100601196289\\
4.54500007629395	94.569450378418\\
4.55000019073486	95.3938674926758\\
4.55499982833862	92.4172210693359\\
4.55999994277954	87.1980667114258\\
4.56500005722046	77.1299896240234\\
4.57000017166138	62.7049903869629\\
4.57499980926514	47.4546775817871\\
4.57999992370605	34.1531753540039\\
4.58500003814697	24.0710716247559\\
4.59000015258789	18.74196434021\\
4.59499979019165	18.9871234893799\\
4.59999990463257	23.6764621734619\\
4.60500001907349	31.283411026001\\
4.6100001335144	38.2868194580078\\
4.61499977111816	39.0984039306641\\
4.61999988555908	29.2156238555908\\
4.625	15.9044342041016\\
4.63000011444092	27.0273742675781\\
4.63500022888184	35.284969329834\\
4.6399998664856	39.8284797668457\\
4.64499998092651	41.1792030334473\\
4.65000009536743	41.0564918518066\\
4.65500020980835	39.9752388000488\\
4.65999984741211	37.368953704834\\
4.66499996185303	266.228546142578\\
4.67000007629395	380.182861328125\\
4.67500019073486	456.426208496094\\
4.67999982833862	494.932342529297\\
4.68499994277954	500.949737548828\\
4.69000005722046	504.085021972656\\
4.69500017166138	472.275421142578\\
4.69999980926514	408.271057128906\\
4.70499992370605	324.433044433594\\
4.71000003814697	241.131729125977\\
4.71500015258789	180.085815429688\\
4.71999979019165	156.458602905273\\
4.72499990463257	198.494277954102\\
4.73000001907349	259.729064941406\\
4.7350001335144	320.764251708984\\
4.73999977111816	365.496459960938\\
4.74499988555908	384.703552246094\\
4.75	375.578674316406\\
4.75500011444092	357.944122314453\\
4.76000022888184	314.171203613281\\
4.7649998664856	249.163040161133\\
4.76999998092651	173.0859375\\
4.77500009536743	98.9974136352539\\
4.78000020980835	38.891414642334\\
4.78499984741211	9.73081588745117\\
4.78999996185303	8.25672245025635\\
4.79500007629395	6.3941798210144\\
4.80000019073486	48.1665725708008\\
4.80499982833862	75.172119140625\\
4.80999994277954	82.1851348876953\\
4.81500005722046	67.2914733886719\\
4.82000017166138	46.776683807373\\
4.82499980926514	30.6417007446289\\
4.82999992370605	24.5009136199951\\
4.83500003814697	28.8527412414551\\
4.84000015258789	41.7755393981934\\
4.84499979019165	58.3023910522461\\
4.84999990463257	70.6645889282227\\
4.85500001907349	67.8956756591797\\
4.8600001335144	41.6819038391113\\
4.86499977111816	3.50963258743286\\
4.86999988555908	8.72592735290527\\
4.875	19.2996692657471\\
4.88000011444092	30.2236099243164\\
4.88500022888184	38.0127334594727\\
4.8899998664856	42.115837097168\\
4.89499998092651	42.9221725463867\\
4.90000009536743	42.9892883300781\\
4.90500020980835	41.4789276123047\\
4.90999984741211	38.4977531433105\\
4.91499996185303	34.3838844299316\\
4.92000007629395	264.081512451172\\
4.92500019073486	404.044219970703\\
4.92999982833862	479.460296630859\\
4.93499994277954	502.226226806641\\
4.94000005722046	490.668395996094\\
4.94500017166138	477.807983398438\\
4.94999980926514	430.759033203125\\
4.95499992370605	361.606994628906\\
4.96000003814697	286.333221435547\\
4.96500015258789	222.673965454102\\
4.96999979019165	184.696807861328\\
4.97499990463257	187.635223388672\\
4.98000001907349	223.591598510742\\
4.9850001335144	264.617095947266\\
4.98999977111816	300.175506591797\\
4.99499988555908	323.075500488281\\
5	330.795684814453\\
5.00500011444092	323.479064941406\\
5.01000022888184	308.6064453125\\
5.0149998664856	287.122528076172\\
5.01999998092651	256.564300537109\\
5.02500009536743	220.436553955078\\
5.03000020980835	182.640975952148\\
5.03499984741211	146.462509155273\\
5.03999996185303	114.509689331055\\
5.04500007629395	88.2855529785156\\
5.05000019073486	68.5273742675781\\
5.05499982833862	55.0981407165527\\
5.05999994277954	47.1034240722656\\
5.06500005722046	43.0321388244629\\
5.07000017166138	41.0981369018555\\
5.07499980926514	39.5217552185059\\
5.07999992370605	36.739818572998\\
5.08500003814697	32.1769943237305\\
5.09000015258789	24.8106555938721\\
5.09499979019165	14.6214590072632\\
5.09999990463257	2.60986280441284\\
5.10500001907349	-0.315004736185074\\
5.1100001335144	-0.288657367229462\\
5.11499977111816	-0.250287264585495\\
5.11999988555908	-0.20632740855217\\
5.125	-0.160908922553062\\
5.13000011444092	-0.121495857834816\\
5.13500022888184	9.23101329803467\\
5.1399998664856	17.3196296691895\\
5.14499998092651	21.9304885864258\\
5.15000009536743	23.9719581604004\\
5.15500020980835	24.098819732666\\
5.15999984741211	125.03409576416\\
5.16499996185303	194.907958984375\\
5.17000007629395	247.96614074707\\
5.17500019073486	281.864715576172\\
5.17999982833862	297.305389404297\\
5.18499994277954	298.226196289063\\
5.19000005722046	291.115905761719\\
5.19500017166138	279.9345703125\\
5.19999980926514	261.224975585938\\
5.20499992370605	239.421752929688\\
5.21000003814697	218.659240722656\\
5.21500015258789	202.149063110352\\
5.21999979019165	191.712387084961\\
5.22499990463257	189.400054931641\\
5.23000001907349	193.55290222168\\
5.2350001335144	199.318664550781\\
5.23999977111816	204.998870849609\\
5.24499988555908	209.472686767578\\
5.25	211.905899047852\\
5.25500011444092	211.951263427734\\
5.26000022888184	211.859481811523\\
5.2649998664856	209.177429199219\\
5.26999998092651	203.336807250977\\
5.27500009536743	194.377593994141\\
5.28000020980835	182.718063354492\\
5.28499984741211	169.220138549805\\
5.28999996185303	155.050323486328\\
5.29500007629395	141.555068969727\\
5.30000019073486	130.2998046875\\
5.30499982833862	122.449928283691\\
5.30999994277954	119.247909545898\\
5.31500005722046	119.846626281738\\
5.32000017166138	128.466751098633\\
5.32499980926514	137.951446533203\\
5.32999992370605	146.794815063477\\
5.33500003814697	153.419952392578\\
5.34000015258789	156.765350341797\\
5.34499979019165	158.286605834961\\
5.34999990463257	154.658996582031\\
5.35500001907349	145.503143310547\\
5.3600001335144	131.455459594727\\
5.36499977111816	114.205307006836\\
5.36999988555908	95.7335891723633\\
5.375	79.3859710693359\\
5.38000011444092	66.5707702636719\\
5.38500022888184	58.9736595153809\\
5.3899998664856	57.0937881469727\\
5.39499998092651	60.8976898193359\\
5.40000009536743	63.6553192138672\\
5.40500020980835	63.0057640075684\\
5.40999984741211	60.5737113952637\\
5.41499996185303	52.6996765136719\\
5.42000007629395	38.7300262451172\\
5.42500019073486	19.509464263916\\
5.42999982833862	2.40218997001648\\
5.43499994277954	1.91859352588654\\
5.44000005722046	1.41803848743439\\
5.44500017166138	1.14787900447845\\
5.44999980926514	1.08561456203461\\
5.45499992370605	1.03774189949036\\
5.46000003814697	1.00113129615784\\
5.46500015258789	0.974171578884125\\
5.46999979019165	0.948079347610474\\
5.47499990463257	0.940666973590851\\
5.48000001907349	0.903235018253326\\
5.4850001335144	0.904013156890869\\
5.48999977111816	0.883900582790375\\
5.49499988555908	0.876433253288269\\
5.5	0.864398181438446\\
5.50500011444092	0.860158145427704\\
5.51000022888184	0.861392676830292\\
5.5149998664856	0.850037753582001\\
5.51999998092651	1.44086730480194\\
5.52500009536743	3.18753671646118\\
5.53000020980835	4.17417812347412\\
5.53499984741211	5.02417802810669\\
5.53999996185303	5.71588897705078\\
5.54500007629395	6.26162147521973\\
5.55000019073486	6.66233682632446\\
5.55499982833862	6.9447135925293\\
5.55999994277954	7.16957569122314\\
5.56500005722046	7.31390285491943\\
5.57000017166138	7.33972501754761\\
5.57499980926514	7.27773284912109\\
5.57999992370605	7.1020622253418\\
5.58500003814697	6.85220193862915\\
5.59000015258789	6.5023627281189\\
5.59499979019165	6.15174722671509\\
5.59999990463257	5.73155689239502\\
5.60500001907349	5.32458162307739\\
5.6100001335144	4.97348022460938\\
5.61499977111816	4.65623188018799\\
5.61999988555908	4.37753009796143\\
5.625	4.18023109436035\\
5.63000011444092	4.05389213562012\\
5.63500022888184	3.92036390304565\\
5.6399998664856	3.87912154197693\\
5.64499998092651	3.86516547203064\\
5.65000009536743	3.848313331604\\
5.65500020980835	3.83613610267639\\
5.65999984741211	3.82295155525208\\
5.66499996185303	3.81722831726074\\
5.67000007629395	8.42273807525635\\
5.67500019073486	108.884826660156\\
5.67999982833862	131.818023681641\\
5.68499994277954	133.046981811523\\
5.69000005722046	127.265655517578\\
5.69500017166138	116.336921691895\\
5.69999980926514	100.558929443359\\
5.70499992370605	85.6963043212891\\
5.71000003814697	75.7878646850586\\
5.71500015258789	74.6244964599609\\
5.71999979019165	80.0309143066406\\
5.72499990463257	86.1121826171875\\
5.73000001907349	90.2329635620117\\
5.7350001335144	91.1660919189453\\
5.73999977111816	88.7475204467773\\
5.74499988555908	83.4331130981445\\
5.75	76.2499694824219\\
5.75500011444092	68.3889694213867\\
5.76000022888184	60.6371078491211\\
5.7649998664856	53.1919822692871\\
5.76999998092651	45.3028602600098\\
5.77500009536743	36.6014556884766\\
5.78000020980835	25.9125194549561\\
5.78499984741211	12.8845205307007\\
5.78999996185303	-2.59577393531799\\
5.79500007629395	-20.7190589904785\\
5.80000019073486	-41.1919364929199\\
5.80499982833862	-63.3027305603027\\
5.80999994277954	-86.3926086425781\\
5.81500005722046	-109.437438964844\\
5.82000017166138	-131.168075561523\\
5.82499980926514	-150.029663085938\\
5.82999992370605	-164.598190307617\\
5.83500003814697	-173.135314941406\\
5.84000015258789	-174.089019775391\\
5.84499979019165	-166.25\\
5.84999990463257	-149.206817626953\\
5.85500001907349	-124.481506347656\\
5.8600001335144	-90.577033996582\\
5.86499977111816	-50.7368431091309\\
5.86999988555908	-9.68663597106934\\
5.875	24.5480117797852\\
5.88000011444092	43.5589561462402\\
5.88500022888184	40.639533996582\\
5.8899998664856	19.8940353393555\\
5.89499998092651	19.9154376983643\\
5.90000009536743	20.5759658813477\\
5.90500020980835	22.1011772155762\\
5.90999984741211	25.5714302062988\\
5.91499996185303	31.0854110717773\\
5.92000007629395	37.2313995361328\\
5.92500019073486	42.5552635192871\\
5.92999982833862	46.2691535949707\\
5.93499994277954	48.0176239013672\\
5.94000005722046	48.1363792419434\\
5.94500017166138	47.2665061950684\\
5.94999980926514	46.1516227722168\\
5.95499992370605	45.607063293457\\
5.96000003814697	46.889835357666\\
5.96500015258789	49.8283920288086\\
5.96999979019165	54.0823745727539\\
5.97499990463257	59.2559280395508\\
5.98000001907349	476.367401123047\\
5.9850001335144	760.791809082031\\
5.98999977111816	913.1806640625\\
5.99499988555908	956.175048828125\\
6	940.124389648438\\
6.00500011444092	917.226013183594\\
6.01000022888184	886.286376953125\\
6.0149998664856	814.450378417969\\
6.01999998092651	701.931213378906\\
6.02500009536743	563.838562011719\\
6.03000020980835	425.777801513672\\
6.03499984741211	317.043365478516\\
6.03999996185303	264.754547119141\\
6.04500007629395	307.075836181641\\
6.05000019073486	399.521820068359\\
6.05499982833862	500.212249755859\\
6.05999994277954	587.75390625\\
6.06500005722046	647.788757324219\\
6.07000017166138	676.229248046875\\
6.07499980926514	673.713989257813\\
6.07999992370605	651.992309570313\\
6.08500003814697	624.95068359375\\
6.09000015258789	575.867370605469\\
6.09499979019165	509.698333740234\\
6.09999990463257	433.058624267578\\
6.10500001907349	354.100555419922\\
6.1100001335144	283.235076904297\\
6.11499977111816	227.258651733398\\
6.11999988555908	191.658111572266\\
6.125	177.2529296875\\
6.13000011444092	189.721908569336\\
6.13500022888184	207.776718139648\\
6.1399998664856	222.440628051758\\
6.14499998092651	228.95295715332\\
6.15000009536743	224.937133789063\\
6.15500020980835	215.006927490234\\
6.15999984741211	197.365707397461\\
6.16499996185303	169.445205688477\\
6.17000007629395	133.205490112305\\
6.17500019073486	91.8606033325195\\
6.17999982833862	49.3978958129883\\
6.18499994277954	10.1494178771973\\
6.19000005722046	4.19211673736572\\
6.19500017166138	3.67328977584839\\
6.19999980926514	2.83171725273132\\
6.20499992370605	1.75673162937164\\
6.21000003814697	0.561404407024384\\
6.21500015258789	-0.0483319871127605\\
6.21999979019165	-0.105352811515331\\
6.22499990463257	-0.140487089753151\\
6.23000001907349	-0.158906400203705\\
6.2350001335144	-0.165811866521835\\
6.23999977111816	-0.164734423160553\\
6.24499988555908	-0.159442394971848\\
6.25	-0.152413591742516\\
6.25500011444092	-0.143329650163651\\
6.26000022888184	-0.133984833955765\\
6.2649998664856	-0.123582690954208\\
6.26999998092651	-0.113170750439167\\
6.27500009536743	-0.103666186332703\\
6.28000020980835	-0.0947950407862663\\
6.28499984741211	-0.0862003266811371\\
6.28999996185303	-0.0777106881141663\\
6.29500007629395	-0.0697755739092827\\
6.30000019073486	-0.0628884434700012\\
6.30499982833862	-0.0568633340299129\\
6.30999994277954	-0.0511793978512287\\
6.31500005722046	-0.0459954217076302\\
6.32000017166138	-0.0416448377072811\\
6.32499980926514	-0.0380459688603878\\
6.32999992370605	-0.0348630547523499\\
6.33500003814697	-0.0313087441027164\\
6.34000015258789	-0.0270224492996931\\
6.34499979019165	-0.0232679452747107\\
6.34999990463257	-0.0201354827731848\\
6.35500001907349	-0.0179150626063347\\
6.3600001335144	48.4339599609375\\
6.36499977111816	68.6913223266602\\
6.36999988555908	81.7798690795898\\
6.375	88.1875381469727\\
6.38000011444092	88.7521362304688\\
6.38500022888184	86.964469909668\\
6.3899998664856	88.8522415161133\\
6.39499998092651	85.0227813720703\\
6.40000009536743	82.0942230224609\\
6.40500020980835	84.4209671020508\\
6.40999984741211	97.1643447875977\\
6.41499996185303	115.381713867188\\
6.42000007629395	137.724838256836\\
6.42500019073486	162.122451782227\\
6.42999982833862	186.366592407227\\
6.43499994277954	208.980133056641\\
6.44000005722046	228.608474731445\\
6.44500017166138	244.714813232422\\
6.44999980926514	257.045227050781\\
6.45499992370605	266.140899658203\\
6.46000003814697	272.421630859375\\
6.46500015258789	276.593505859375\\
6.46999979019165	279.541717529297\\
6.47499990463257	281.872375488281\\
6.48000001907349	284.118286132813\\
6.4850001335144	286.774810791016\\
6.48999977111816	290.025787353516\\
6.49499988555908	293.872283935547\\
6.5	298.181274414063\\
6.50500011444092	302.699859619141\\
6.51000022888184	307.065093994141\\
6.5149998664856	310.968109130859\\
6.51999998092651	314.120422363281\\
6.52500009536743	316.2841796875\\
6.53000020980835	317.253082275391\\
6.53499984741211	317.182312011719\\
6.53999996185303	316.005920410156\\
6.54500007629395	313.820953369141\\
6.55000019073486	310.788269042969\\
6.55499982833862	307.163940429688\\
6.55999994277954	303.238128662109\\
6.56500005722046	298.819305419922\\
6.57000017166138	293.930969238281\\
6.57499980926514	288.641784667969\\
6.57999992370605	283.374694824219\\
6.58500003814697	277.939392089844\\
6.59000015258789	272.459411621094\\
6.59499979019165	267.006896972656\\
6.59999990463257	261.508178710938\\
6.60500001907349	255.93391418457\\
6.6100001335144	250.311462402344\\
6.61499977111816	244.669967651367\\
6.61999988555908	238.833129882813\\
6.625	232.944961547852\\
6.63000011444092	226.951553344727\\
6.63500022888184	220.898223876953\\
6.6399998664856	214.932922363281\\
6.64499998092651	208.951416015625\\
6.65000009536743	203.077926635742\\
6.65500020980835	197.387054443359\\
6.65999984741211	191.829879760742\\
6.66499996185303	186.505142211914\\
6.67000007629395	181.579193115234\\
6.67500019073486	176.939514160156\\
6.67999982833862	172.757705688477\\
6.68499994277954	169.017562866211\\
6.69000005722046	165.611221313477\\
6.69500017166138	162.902770996094\\
6.69999980926514	161.014556884766\\
6.70499992370605	159.584686279297\\
6.71000003814697	158.6025390625\\
6.71500015258789	157.935546875\\
6.71999979019165	157.648742675781\\
6.72499990463257	157.725997924805\\
6.73000001907349	158.051467895508\\
6.7350001335144	158.677917480469\\
6.73999977111816	159.603485107422\\
6.74499988555908	160.79606628418\\
6.75	162.323974609375\\
6.75500011444092	164.097915649414\\
6.76000022888184	165.936172485352\\
6.7649998664856	168.182052612305\\
6.76999998092651	170.61149597168\\
6.77500009536743	173.31298828125\\
6.78000020980835	176.27099609375\\
6.78499984741211	179.402114868164\\
6.78999996185303	182.728713989258\\
6.79500007629395	186.200775146484\\
6.80000019073486	189.766845703125\\
6.80499982833862	193.390380859375\\
6.80999994277954	197.093643188477\\
6.81500005722046	200.750640869141\\
6.82000017166138	204.40657043457\\
6.82499980926514	208.008605957031\\
6.82999992370605	211.517364501953\\
6.83500003814697	214.899688720703\\
6.84000015258789	218.126602172852\\
6.84499979019165	221.275985717773\\
6.84999990463257	224.265594482422\\
6.85500001907349	227.079833984375\\
6.8600001335144	229.703964233398\\
6.86499977111816	232.029678344727\\
6.86999988555908	234.080795288086\\
6.875	235.812622070313\\
6.88000011444092	237.126571655273\\
6.88500022888184	237.98063659668\\
6.8899998664856	238.308471679688\\
6.89499998092651	238.066970825195\\
6.90000009536743	237.236968994141\\
6.90500020980835	235.817535400391\\
6.90999984741211	233.806335449219\\
6.91499996185303	231.247619628906\\
6.92000007629395	228.190444946289\\
6.92500019073486	224.691040039063\\
6.92999982833862	220.819915771484\\
6.93499994277954	216.649887084961\\
6.94000005722046	212.233688354492\\
6.94500017166138	207.608871459961\\
6.94999980926514	202.804092407227\\
6.95499992370605	197.81982421875\\
6.96000003814697	192.685317993164\\
6.96500015258789	187.414794921875\\
6.96999979019165	181.980728149414\\
6.97499990463257	176.347961425781\\
6.98000001907349	170.563629150391\\
6.9850001335144	164.597213745117\\
6.98999977111816	158.457641601563\\
6.99499988555908	152.194030761719\\
7	145.849761962891\\
7.00500011444092	139.46418762207\\
7.01000022888184	133.091720581055\\
7.0149998664856	126.79679107666\\
7.01999998092651	120.628196716309\\
7.02500009536743	114.616676330566\\
7.03000020980835	108.788452148438\\
7.03499984741211	103.167106628418\\
7.03999996185303	97.7684555053711\\
7.04500007629395	92.5998153686523\\
7.05000019073486	87.6615829467773\\
7.05499982833862	82.9475555419922\\
7.05999994277954	78.4457855224609\\
7.06500005722046	74.1484298706055\\
7.07000017166138	70.0484848022461\\
7.07499980926514	66.144775390625\\
7.07999992370605	62.4408493041992\\
7.08500003814697	58.947338104248\\
7.09000015258789	55.6816825866699\\
7.09499979019165	52.6595840454102\\
7.09999990463257	49.9011154174805\\
7.10500001907349	47.4253082275391\\
7.1100001335144	45.2497711181641\\
7.11499977111816	43.3811683654785\\
7.11999988555908	41.8315200805664\\
7.125	40.6060943603516\\
7.13000011444092	39.6675834655762\\
7.13500022888184	39.0082855224609\\
7.1399998664856	38.6156845092773\\
7.14499998092651	38.4542770385742\\
7.15000009536743	38.5246963500977\\
7.15500020980835	38.8852005004883\\
7.15999984741211	39.3950347900391\\
7.16499996185303	39.983211517334\\
7.17000007629395	40.757381439209\\
7.17500019073486	41.6884155273438\\
7.17999982833862	42.7767562866211\\
7.18499994277954	44.0389976501465\\
7.19000005722046	45.4634971618652\\
7.19500017166138	47.0481643676758\\
7.19999980926514	48.7838859558105\\
7.20499992370605	50.6364288330078\\
7.21000003814697	52.6104698181152\\
7.21500015258789	54.6754760742188\\
7.21999979019165	56.7843704223633\\
7.22499990463257	58.9538154602051\\
7.23000001907349	61.1586418151855\\
7.2350001335144	63.3831024169922\\
7.23999977111816	65.6245269775391\\
7.24499988555908	67.8619079589844\\
7.25	70.0848922729492\\
7.25500011444092	72.2975311279297\\
7.26000022888184	74.4739456176758\\
7.2649998664856	76.6178512573242\\
7.26999998092651	78.7344436645508\\
7.27500009536743	80.7988357543945\\
7.28000020980835	82.797119140625\\
7.28499984741211	84.7305603027344\\
7.28999996185303	86.5858993530273\\
7.29500007629395	88.3665237426758\\
7.30000019073486	90.037841796875\\
7.30499982833862	91.6108093261719\\
7.30999994277954	93.0749893188477\\
7.31500005722046	94.4325408935547\\
7.32000017166138	95.6681518554688\\
7.32499980926514	96.7859573364258\\
7.32999992370605	97.773078918457\\
7.33500003814697	98.6277618408203\\
7.34000015258789	99.3520202636719\\
7.34499979019165	99.9427871704102\\
7.34999990463257	100.401672363281\\
7.35500001907349	100.729957580566\\
7.3600001335144	100.939086914063\\
7.36499977111816	101.031341552734\\
7.36999988555908	101.013664245605\\
7.375	100.894508361816\\
7.38000011444092	100.678894042969\\
7.38500022888184	100.339729309082\\
7.3899998664856	99.8604583740234\\
7.39499998092651	99.2475814819336\\
7.40000009536743	98.5346069335938\\
7.40500020980835	97.7135391235352\\
7.40999984741211	96.7948532104492\\
7.41499996185303	95.7850799560547\\
7.42000007629395	94.692985534668\\
7.42500019073486	93.5287551879883\\
7.42999982833862	92.2984313964844\\
7.43499994277954	91.0040893554688\\
7.44000005722046	89.6535797119141\\
7.44500017166138	88.2558135986328\\
7.44999980926514	86.8171005249023\\
7.45499992370605	85.3490829467773\\
7.46000003814697	83.8643264770508\\
7.46500015258789	82.3708038330078\\
7.46999979019165	80.870849609375\\
7.47499990463257	79.3727188110352\\
7.48000001907349	77.8824157714844\\
7.4850001335144	76.4047317504883\\
7.48999977111816	74.9452743530273\\
7.49499988555908	73.5085296630859\\
7.5	72.0976715087891\\
7.50500011444092	70.7149047851563\\
7.51000022888184	69.3647613525391\\
7.5149998664856	68.0590972900391\\
7.51999998092651	66.8051147460938\\
7.52500009536743	65.6245651245117\\
7.53000020980835	64.4814987182617\\
7.53499984741211	63.4151077270508\\
7.53999996185303	62.4089813232422\\
7.54500007629395	61.4832382202148\\
7.55000019073486	60.6312484741211\\
7.55499982833862	59.8519554138184\\
7.55999994277954	59.149242401123\\
7.56500005722046	58.5219383239746\\
7.57000017166138	57.9667358398438\\
7.57499980926514	57.4840202331543\\
7.57999992370605	57.0736083984375\\
7.58500003814697	56.7346839904785\\
7.59000015258789	56.4684944152832\\
7.59499979019165	56.2747383117676\\
7.59999990463257	56.1517944335938\\
7.60500001907349	56.1019287109375\\
7.6100001335144	56.138744354248\\
7.61499977111816	56.2765884399414\\
7.61999988555908	56.4892616271973\\
7.625	56.7458534240723\\
7.63000011444092	57.0548477172852\\
7.63500022888184	57.4193115234375\\
7.6399998664856	57.836841583252\\
7.64499998092651	58.3069915771484\\
7.65000009536743	58.818431854248\\
7.65500020980835	59.3679351806641\\
7.65999984741211	59.9545021057129\\
7.66499996185303	60.5734596252441\\
7.67000007629395	61.2199859619141\\
7.67500019073486	61.892391204834\\
7.67999982833862	62.5911521911621\\
7.68499994277954	63.3135604858398\\
7.69000005722046	64.0574111938477\\
7.69500017166138	64.8125\\
7.69999980926514	65.5752868652344\\
7.70499992370605	66.3443603515625\\
7.71000003814697	67.1189193725586\\
7.71500015258789	67.9044342041016\\
7.71999979019165	68.689338684082\\
7.72499990463257	69.4609603881836\\
7.73000001907349	70.2180023193359\\
7.7350001335144	70.9559173583984\\
7.73999977111816	71.6854476928711\\
7.74499988555908	72.3982238769531\\
7.75	73.078010559082\\
7.75500011444092	73.7332992553711\\
7.76000022888184	74.3673553466797\\
7.7649998664856	74.9720077514648\\
7.76999998092651	75.5449523925781\\
7.77500009536743	76.0877838134766\\
7.78000020980835	76.6001892089844\\
7.78499984741211	77.0795745849609\\
7.78999996185303	77.5275421142578\\
7.79500007629395	77.9440841674805\\
7.80000019073486	78.3256378173828\\
7.80499982833862	78.6735763549805\\
7.80999994277954	78.9881973266602\\
7.81500005722046	79.2679214477539\\
7.82000017166138	79.5125579833984\\
7.82499980926514	79.7226867675781\\
7.82999992370605	79.8997268676758\\
7.83500003814697	80.0452728271484\\
7.84000015258789	80.1580505371094\\
7.84499979019165	80.2388153076172\\
7.84999990463257	80.2914886474609\\
7.85500001907349	80.3153457641602\\
7.8600001335144	80.3112945556641\\
7.86499977111816	80.2817840576172\\
7.86999988555908	80.2274398803711\\
7.875	80.1494293212891\\
7.88000011444092	80.052734375\\
7.88500022888184	79.9396896362305\\
7.8899998664856	79.8117523193359\\
7.89499998092651	79.6695098876953\\
7.90000009536743	79.5132751464844\\
7.90500020980835	79.3454895019531\\
7.90999984741211	79.1670913696289\\
7.91499996185303	78.9776916503906\\
7.92000007629395	78.7791290283203\\
7.92500019073486	78.5732803344727\\
7.92999982833862	78.3736877441406\\
7.93499994277954	78.1726837158203\\
7.94000005722046	77.9702682495117\\
7.94500017166138	77.7808074951172\\
7.94999980926514	77.6003036499023\\
7.95499992370605	77.4267501831055\\
7.96000003814697	77.2685470581055\\
7.96500015258789	77.1283111572266\\
7.96999979019165	77.0021286010742\\
7.97499990463257	76.8956832885742\\
7.98000001907349	76.8179321289063\\
7.9850001335144	76.7623748779297\\
7.98999977111816	76.7293014526367\\
7.99499988555908	76.7206497192383\\
8	76.7353134155273\\
8.00500011444092	76.7732772827148\\
8.01000022888184	76.8431701660156\\
8.01500034332275	76.9443588256836\\
8.02000045776367	77.0782089233398\\
8.02499961853027	77.2438430786133\\
8.02999973297119	77.4418029785156\\
8.03499984741211	77.6738204956055\\
8.03999996185303	77.9399337768555\\
8.04500007629395	78.2396545410156\\
8.05000019073486	78.5741729736328\\
8.05500030517578	78.9435501098633\\
8.0600004196167	79.3462066650391\\
8.0649995803833	79.7830352783203\\
8.06999969482422	80.2540817260742\\
8.07499980926514	80.7620697021484\\
8.07999992370605	81.3065032958984\\
8.08500003814697	81.8879089355469\\
8.09000015258789	82.5024185180664\\
8.09500026702881	83.149528503418\\
8.10000038146973	83.8293914794922\\
8.10499954223633	84.5427017211914\\
8.10999965667725	85.2901992797852\\
8.11499977111816	86.0711288452148\\
8.11999988555908	86.8848724365234\\
8.125	87.7298431396484\\
8.13000011444092	88.6061859130859\\
8.13500022888184	89.5134506225586\\
8.14000034332275	90.4516830444336\\
8.14500045776367	91.420295715332\\
8.14999961853027	92.4189071655273\\
8.15499973297119	93.4452209472656\\
8.15999984741211	94.498779296875\\
8.16499996185303	95.5789489746094\\
8.17000007629395	96.6879043579102\\
8.17500019073486	97.8268356323242\\
8.18000030517578	98.9948806762695\\
8.1850004196167	100.190361022949\\
8.1899995803833	101.409736633301\\
8.19499969482422	102.653755187988\\
8.19999980926514	103.923377990723\\
8.20499992370605	105.220375061035\\
8.21000003814697	106.543724060059\\
8.21500015258789	107.889045715332\\
8.22000026702881	109.252143859863\\
8.22500038146973	110.632888793945\\
8.22999954223633	112.037300109863\\
8.23499965667725	113.468223571777\\
8.23999977111816	114.921173095703\\
8.24499988555908	116.393936157227\\
8.25	117.889633178711\\
8.25500011444092	119.410972595215\\
8.26000022888184	120.95671081543\\
8.26500034332275	122.527114868164\\
8.27000045776367	124.121681213379\\
8.27499961853027	125.739906311035\\
8.27999973297119	127.378311157227\\
8.28499984741211	129.037673950195\\
8.28999996185303	130.721420288086\\
8.29500007629395	132.418533325195\\
8.30000019073486	134.122787475586\\
8.30500030517578	135.831848144531\\
8.3100004196167	137.548583984375\\
8.3149995803833	139.365051269531\\
8.31999969482422	141.169631958008\\
8.32499980926514	142.995376586914\\
8.32999992370605	144.845840454102\\
8.33500003814697	146.715362548828\\
8.34000015258789	148.602874755859\\
8.34500026702881	150.507843017578\\
8.35000038146973	152.432098388672\\
8.35499954223633	154.352935791016\\
8.35999965667725	156.309326171875\\
8.36499977111816	158.29768371582\\
8.36999988555908	160.275268554688\\
8.375	162.255920410156\\
8.38000011444092	164.23762512207\\
8.38500022888184	166.211944580078\\
8.39000034332275	168.183486938477\\
8.39500045776367	170.130249023438\\
8.39999961853027	172.104690551758\\
8.40499973297119	174.052734375\\
8.40999984741211	175.975967407227\\
8.41499996185303	177.876556396484\\
8.42000007629395	179.747756958008\\
8.42500019073486	181.596710205078\\
8.43000030517578	183.411209106445\\
8.4350004196167	185.193099975586\\
8.4399995803833	186.945068359375\\
8.44499969482422	188.632476806641\\
8.44999980926514	190.252151489258\\
8.45499992370605	191.79035949707\\
8.46000003814697	193.214920043945\\
8.46500015258789	194.530975341797\\
8.47000026702881	195.685623168945\\
8.47500038146973	196.694595336914\\
8.47999954223633	197.561630249023\\
8.48499965667725	198.298126220703\\
8.48999977111816	198.866775512695\\
8.49499988555908	199.313385009766\\
8.5	199.650634765625\\
8.50500011444092	199.863037109375\\
8.51000022888184	200.064758300781\\
8.51500034332275	200.267044067383\\
8.52000045776367	200.464691162109\\
8.52499961853027	200.830429077148\\
8.52999973297119	201.236862182617\\
8.53499984741211	201.830032348633\\
8.53999996185303	202.632583618164\\
8.54500007629395	203.682952880859\\
8.55000019073486	205.028106689453\\
8.55500030517578	206.687194824219\\
8.5600004196167	208.766311645508\\
8.5649995803833	211.224136352539\\
8.56999969482422	214.074996948242\\
8.57499980926514	217.32585144043\\
8.57999992370605	220.838302612305\\
8.58500003814697	224.585021972656\\
8.59000015258789	228.466125488281\\
8.59500026702881	232.161163330078\\
8.60000038146973	235.5341796875\\
8.60499954223633	238.564682006836\\
8.60999965667725	241.185485839844\\
8.61499977111816	243.419403076172\\
8.61999988555908	245.325408935547\\
8.625	246.910263061523\\
8.63000011444092	248.250686645508\\
8.63500022888184	249.353927612305\\
8.64000034332275	250.301376342773\\
8.64500045776367	251.020172119141\\
8.64999961853027	251.721374511719\\
8.65499973297119	251.899810791016\\
8.65999984741211	252.618469238281\\
8.66499996185303	252.539657592773\\
8.67000007629395	252.237243652344\\
8.67500019073486	251.439041137695\\
8.68000030517578	250.374954223633\\
8.6850004196167	248.856658935547\\
8.6899995803833	246.801055908203\\
8.69499969482422	244.298477172852\\
8.69999980926514	241.476608276367\\
8.70499992370605	238.348785400391\\
8.71000003814697	234.895126342773\\
8.71500015258789	231.14289855957\\
8.72000026702881	227.428405761719\\
8.72500038146973	223.654739379883\\
8.72999954223633	219.866149902344\\
8.73499965667725	216.169357299805\\
8.73999977111816	212.617904663086\\
8.74499988555908	209.146301269531\\
8.75	205.699356079102\\
8.75500011444092	202.008712768555\\
8.76000022888184	198.505859375\\
8.76500034332275	194.834884643555\\
8.77000045776367	191.119445800781\\
8.77499961853027	187.312881469727\\
8.77999973297119	183.48503112793\\
8.78499984741211	179.599548339844\\
8.78999996185303	175.720230102539\\
8.79500007629395	171.910125732422\\
8.80000019073486	168.224502563477\\
8.80500030517578	164.712432861328\\
8.8100004196167	161.422866821289\\
8.8149995803833	158.397552490234\\
8.81999969482422	155.631698608398\\
8.82499980926514	153.13606262207\\
8.82999992370605	150.900665283203\\
8.83500003814697	148.871932983398\\
8.84000015258789	147.012573242188\\
8.84500026702881	145.304901123047\\
8.85000038146973	143.677978515625\\
8.85499954223633	142.160217285156\\
8.85999965667725	140.72265625\\
8.86499977111816	139.373672485352\\
8.86999988555908	138.128921508789\\
8.875	137.018707275391\\
8.88000011444092	136.077789306641\\
8.88500022888184	135.335113525391\\
8.89000034332275	134.815979003906\\
8.89500045776367	134.514083862305\\
8.89999961853027	134.439071655273\\
8.90499973297119	134.621459960938\\
8.90999984741211	134.973007202148\\
8.91499996185303	135.383651733398\\
8.92000007629395	135.928756713867\\
8.92500019073486	136.618835449219\\
8.93000030517578	137.423736572266\\
8.9350004196167	138.261947631836\\
8.9399995803833	138.987060546875\\
8.94499969482422	139.775299072266\\
8.94999980926514	140.646057128906\\
8.95499992370605	141.580047607422\\
8.96000003814697	142.592041015625\\
8.96500015258789	143.689315795898\\
8.97000026702881	144.894790649414\\
8.97500038146973	146.164886474609\\
8.97999954223633	147.483016967773\\
8.98499965667725	148.962219238281\\
8.98999977111816	150.426544189453\\
8.99499988555908	151.808898925781\\
9	153.294448852539\\
9.00500011444092	154.773101806641\\
9.01000022888184	156.221405029297\\
9.01500034332275	157.630950927734\\
9.02000045776367	159.008209228516\\
9.02499961853027	160.341781616211\\
9.02999973297119	161.624130249023\\
9.03499984741211	162.855270385742\\
9.03999996185303	164.03108215332\\
9.04500007629395	165.148803710938\\
9.05000019073486	166.205657958984\\
9.05500030517578	167.197448730469\\
9.0600004196167	168.117797851563\\
9.0649995803833	168.963439941406\\
9.06999969482422	169.728652954102\\
9.07499980926514	170.409118652344\\
9.07999992370605	171.00163269043\\
9.08500003814697	171.502853393555\\
9.09000015258789	171.903503417969\\
9.09500026702881	172.18278503418\\
9.10000038146973	172.3359375\\
9.10499954223633	172.423934936523\\
9.10999965667725	172.422653198242\\
9.11499977111816	172.303558349609\\
9.11999988555908	172.058990478516\\
9.125	171.701522827148\\
9.13000011444092	171.239730834961\\
9.13500022888184	170.686233520508\\
9.14000034332275	170.091827392578\\
9.14500045776367	169.442886352539\\
9.14999961853027	168.769500732422\\
9.15499973297119	168.073028564453\\
9.15999984741211	167.355026245117\\
9.16499996185303	166.611297607422\\
9.17000007629395	165.833084106445\\
9.17500019073486	165.017013549805\\
9.18000030517578	164.157165527344\\
9.1850004196167	163.25569152832\\
9.1899995803833	162.311111450195\\
9.19499969482422	161.321487426758\\
9.19999980926514	160.296768188477\\
9.20499992370605	159.248321533203\\
9.21000003814697	158.194229125977\\
9.21500015258789	157.148056030273\\
9.22000026702881	156.106292724609\\
9.22500038146973	155.082626342773\\
9.22999954223633	154.075454711914\\
9.23499965667725	153.08837890625\\
9.23999977111816	152.137588500977\\
9.24499988555908	151.257080078125\\
9.25	150.477676391602\\
9.25500011444092	149.834396362305\\
9.26000022888184	149.39143371582\\
9.26500034332275	149.24382019043\\
9.27000045776367	149.658584594727\\
9.27499961853027	150.544219970703\\
9.27999973297119	151.891632080078\\
9.28499984741211	153.710189819336\\
9.28999996185303	155.994857788086\\
9.29500007629395	158.727722167969\\
9.30000019073486	161.871688842773\\
9.30500030517578	165.465072631836\\
9.3100004196167	169.371459960938\\
9.3149995803833	173.579071044922\\
9.31999969482422	177.999740600586\\
9.32499980926514	182.532455444336\\
9.32999992370605	187.069793701172\\
9.33500003814697	191.425994873047\\
9.34000015258789	195.385162353516\\
9.34500026702881	198.848510742188\\
9.35000038146973	201.411911010742\\
9.35499954223633	202.815811157227\\
9.35999965667725	202.925277709961\\
9.36499977111816	201.458038330078\\
9.36999988555908	198.356155395508\\
9.375	193.827117919922\\
9.38000011444092	188.066635131836\\
9.38500022888184	181.404067993164\\
9.39000034332275	174.384033203125\\
9.39500045776367	167.248519897461\\
9.39999961853027	160.71842956543\\
9.40499973297119	155.056213378906\\
9.40999984741211	150.416946411133\\
9.41499996185303	146.72705078125\\
9.42000007629395	143.524566650391\\
9.42500019073486	140.067276000977\\
9.43000030517578	135.22932434082\\
9.4350004196167	127.676567077637\\
9.4399995803833	115.750984191895\\
9.44499969482422	97.9403991699219\\
9.44999980926514	72.702522277832\\
9.45499992370605	39.1442031860352\\
9.46000003814697	-2.43955445289612\\
9.46500015258789	-50.9143600463867\\
9.47000026702881	-102.740707397461\\
9.47500038146973	-153.678863525391\\
9.47999954223633	-198.554412841797\\
9.48499965667725	-230.644836425781\\
9.48999977111816	-241.606262207031\\
9.49499988555908	-241.685852050781\\
9.5	-227.452987670898\\
9.50500011444092	-201.974456787109\\
9.51000022888184	-169.214492797852\\
9.51500034332275	-132.908157348633\\
9.52000045776367	-96.8308029174805\\
9.52499961853027	-63.4038543701172\\
9.52999973297119	-36.3394660949707\\
9.53499984741211	-15.8066263198853\\
9.53999996185303	-3.1326916217804\\
9.54500007629395	1.35464763641357\\
9.55000019073486	-1.79051518440247\\
9.55500030517578	-1.3484719991684\\
9.5600004196167	-1.23700499534607\\
9.5649995803833	-1.12290644645691\\
9.56999969482422	-0.987666964530945\\
9.57499980926514	-0.841245412826538\\
9.57999992370605	-0.74242490530014\\
9.58500003814697	-0.597982466220856\\
9.59000015258789	-0.47397917509079\\
9.59500026702881	-0.374574065208435\\
9.60000038146973	-0.290358990430832\\
9.60499954223633	-0.206613704562187\\
9.60999965667725	-0.124602638185024\\
9.61499977111816	-0.0598373413085938\\
9.61999988555908	-0.000270512886345387\\
9.625	0.0480994060635567\\
9.63000011444092	0.0988239198923111\\
9.63500022888184	0.135431706905365\\
9.64000034332275	0.171691790223122\\
9.64500045776367	0.211730092763901\\
9.64999961853027	0.239277243614197\\
9.65499973297119	0.258926600217819\\
9.65999984741211	0.284768015146255\\
9.66499996185303	0.305459469556808\\
9.67000007629395	0.322387456893921\\
9.67500019073486	0.338313847780228\\
9.68000030517578	0.352400988340378\\
9.6850004196167	0.365059226751328\\
9.6899995803833	0.37596008181572\\
9.69499969482422	0.384878844022751\\
9.69999980926514	0.393013656139374\\
9.70499992370605	0.401162087917328\\
9.71000003814697	0.408085584640503\\
9.71500015258789	0.412478506565094\\
9.72000026702881	0.416557461023331\\
9.72500038146973	0.421744048595428\\
9.72999954223633	0.426762372255325\\
9.73499965667725	0.431613117456436\\
9.73999977111816	0.435913681983948\\
9.74499988555908	0.439018070697784\\
9.75	0.442071706056595\\
9.75500011444092	0.444036871194839\\
9.76000022888184	0.444481939077377\\
9.76500034332275	0.444396734237671\\
9.77000045776367	0.445190399885178\\
9.77499961853027	0.446731120347977\\
9.77999973297119	0.449579060077667\\
9.78499984741211	0.450949370861053\\
9.78999996185303	0.451823174953461\\
9.79500007629395	0.451838672161102\\
9.80000019073486	0.451670050621033\\
9.80500030517578	0.452192574739456\\
9.8100004196167	0.453167170286179\\
9.8149995803833	0.454828709363937\\
9.81999969482422	0.455675005912781\\
9.82499980926514	0.456137001514435\\
9.82999992370605	0.455936193466187\\
9.83500003814697	0.455537468194962\\
9.84000015258789	0.455504804849625\\
9.84500026702881	0.455656439065933\\
9.85000038146973	0.456152558326721\\
9.85499954223633	0.456454038619995\\
9.85999965667725	0.456872791051865\\
9.86499977111816	0.457439035177231\\
9.86999988555908	0.457760006189346\\
9.875	0.45752939581871\\
9.88000011444092	0.45683091878891\\
9.88500022888184	0.455704689025879\\
9.89000034332275	0.456065326929092\\
9.89500045776367	0.45730397105217\\
9.89999961853027	0.459676086902618\\
9.90499973297119	0.461231142282486\\
9.90999984741211	0.460579186677933\\
9.91499996185303	0.458225071430206\\
9.92000007629395	0.454185426235199\\
9.92500019073486	0.453592270612717\\
9.93000030517578	0.454615563154221\\
9.9350004196167	0.457819283008575\\
9.9399995803833	0.460287302732468\\
9.94499969482422	0.460875481367111\\
9.94999980926514	0.461125493049622\\
9.95499992370605	0.461037367582321\\
9.96000003814697	0.460611045360565\\
9.96500015258789	0.459846585988998\\
9.97000026702881	0.458743959665298\\
9.97500038146973	0.458205312490463\\
9.97999954223633	0.458175390958786\\
9.98499965667725	0.458185851573944\\
9.98999977111816	0.45823672413826\\
9.99499988555908	0.458328008651733\\
10	0.458459734916687\\
};
\addlegendentry{RS}

\addplot [color=red, line width=2.0pt]
  table[row sep=crcr]{%
0.0949999988079071	-0.285330444574356\\
0.100000001490116	-0.244563743472099\\
0.104999996721745	-0.207787245512009\\
0.109999999403954	-0.176767811179161\\
0.115000002086163	-0.147762641310692\\
0.119999997317791	-0.846249222755432\\
0.125	6.97986030578613\\
0.129999995231628	6.74411153793335\\
0.135000005364418	5.52446413040161\\
0.140000000596046	4.16163396835327\\
0.144999995827675	2.87618398666382\\
0.150000005960464	1.68552851676941\\
0.155000001192093	0.743249595165253\\
0.159999996423721	0.0860154554247856\\
0.165000006556511	-0.332204431295395\\
0.170000001788139	-0.790101110935211\\
0.174999997019768	-1.32945120334625\\
0.180000007152557	-1.85020017623901\\
0.185000002384186	-2.3284547328949\\
0.189999997615814	-2.76695108413696\\
0.194999992847443	204.584014892578\\
0.200000002980232	351.699768066406\\
0.204999998211861	443.915954589844\\
0.209999993443489	485.322540283203\\
0.215000003576279	486.662322998047\\
0.219999998807907	479.205413818359\\
0.224999994039536	440.432220458984\\
0.230000004172325	376.854156494141\\
0.234999999403954	300.551116943359\\
0.239999994635582	226.245376586914\\
0.245000004768372	168.107391357422\\
0.25	141.896545410156\\
0.254999995231628	168.107498168945\\
0.259999990463257	220.39485168457\\
0.264999985694885	284.304656982422\\
0.270000010728836	347.904235839844\\
0.275000005960464	402.529663085938\\
0.280000001192093	442.084716796875\\
0.284999996423721	464.818359375\\
0.28999999165535	470.699768066406\\
0.294999986886978	466.806304931641\\
0.300000011920929	454.908538818359\\
0.305000007152557	430.774963378906\\
0.310000002384186	398.852264404297\\
0.314999997615814	363.996307373047\\
0.319999992847443	331.328369140625\\
0.324999988079071	305.176116943359\\
0.330000013113022	288.382019042969\\
0.33500000834465	282.330169677734\\
0.340000003576279	289.816955566406\\
0.344999998807907	300.417633056641\\
0.349999994039536	310.734741210938\\
0.354999989271164	317.853088378906\\
0.360000014305115	319.698059082031\\
0.365000009536743	316.878479003906\\
0.370000004768372	310.492889404297\\
0.375	297.457946777344\\
0.379999995231628	278.42138671875\\
0.384999990463257	254.887939453125\\
0.389999985694885	229.078002929688\\
0.395000010728836	203.332366943359\\
0.400000005960464	180.424987792969\\
0.405000001192093	160.905654907227\\
0.409999996423721	147.275268554688\\
0.41499999165535	138.782684326172\\
0.419999986886978	135.012954711914\\
0.425000011920929	135.399948120117\\
0.430000007152557	135.874282836914\\
0.435000002384186	134.96516418457\\
0.439999997615814	133.174057006836\\
0.444999992847443	128.536209106445\\
0.449999988079071	120.660873413086\\
0.455000013113022	109.773094177246\\
0.46000000834465	96.6171035766602\\
0.465000003576279	82.3120727539063\\
0.469999998807907	68.1865310668945\\
0.474999994039536	55.6000900268555\\
0.479999989271164	45.6125602722168\\
0.485000014305115	38.9556617736816\\
0.490000009536743	35.8148193359375\\
0.495000004768372	37.2482032775879\\
0.5	40.5421600341797\\
0.504999995231628	44.3712043762207\\
0.509999990463257	47.8637542724609\\
0.514999985694885	50.3753242492676\\
0.519999980926514	51.6042823791504\\
0.524999976158142	51.6667709350586\\
0.529999971389771	51.4119071960449\\
0.535000026226044	50.0701789855957\\
0.540000021457672	48.2494468688965\\
0.545000016689301	46.5823745727539\\
0.550000011920929	45.6470794677734\\
0.555000007152557	46.6369590759277\\
0.560000002384186	49.1573791503906\\
0.564999997615814	53.0378761291504\\
0.569999992847443	58.0287704467773\\
0.574999988079071	63.7244911193848\\
0.579999983310699	69.8516387939453\\
0.584999978542328	76.1360168457031\\
0.589999973773956	82.2080688476563\\
0.595000028610229	88.0391235351563\\
0.600000023841858	93.5283813476563\\
0.605000019073486	98.6865539550781\\
0.610000014305115	103.649131774902\\
0.615000009536743	108.464668273926\\
0.620000004768372	113.295082092285\\
0.625	118.206886291504\\
0.629999995231628	123.310173034668\\
0.634999990463257	128.589004516602\\
0.639999985694885	134.093963623047\\
0.644999980926514	139.78239440918\\
0.649999976158142	145.577011108398\\
0.654999971389771	151.412261962891\\
0.660000026226044	157.172988891602\\
0.665000021457672	162.777893066406\\
0.670000016689301	168.167953491211\\
0.675000011920929	173.262451171875\\
0.680000007152557	178.040664672852\\
0.685000002384186	182.485931396484\\
0.689999997615814	186.610992431641\\
0.694999992847443	190.431213378906\\
0.699999988079071	193.970153808594\\
0.704999983310699	197.273483276367\\
0.709999978542328	200.346862792969\\
0.714999973773956	203.165222167969\\
0.720000028610229	205.75537109375\\
0.725000023841858	208.156372070313\\
0.730000019073486	210.208877563477\\
0.735000014305115	212.044723510742\\
0.740000009536743	213.549865722656\\
0.745000004768372	214.773208618164\\
0.75	215.662292480469\\
0.754999995231628	216.246688842773\\
0.759999990463257	216.488555908203\\
0.764999985694885	216.50178527832\\
0.769999980926514	216.32568359375\\
0.774999976158142	215.859786987305\\
0.779999971389771	215.072189331055\\
0.785000026226044	213.994354248047\\
0.790000021457672	212.660552978516\\
0.795000016689301	211.090087890625\\
0.800000011920929	209.305023193359\\
0.805000007152557	207.320846557617\\
0.810000002384186	205.161422729492\\
0.814999997615814	202.836929321289\\
0.819999992847443	200.365829467773\\
0.824999988079071	197.771896362305\\
0.829999983310699	195.045837402344\\
0.834999978542328	192.204299926758\\
0.839999973773956	189.275756835938\\
0.845000028610229	186.2607421875\\
0.850000023841858	183.17448425293\\
0.855000019073486	180.025192260742\\
0.860000014305115	176.858413696289\\
0.865000009536743	173.694046020508\\
0.870000004768372	170.540740966797\\
0.875	167.418762207031\\
0.879999995231628	164.341278076172\\
0.884999990463257	161.320266723633\\
0.889999985694885	158.36637878418\\
0.894999980926514	155.491836547852\\
0.899999976158142	152.706741333008\\
0.904999971389771	150.01774597168\\
0.910000026226044	147.44548034668\\
0.915000021457672	144.988830566406\\
0.920000016689301	142.64421081543\\
0.925000011920929	140.426300048828\\
0.930000007152557	138.338150024414\\
0.935000002384186	136.379318237305\\
0.939999997615814	134.556503295898\\
0.944999992847443	132.881011962891\\
0.949999988079071	131.366653442383\\
0.954999983310699	130.021484375\\
0.959999978542328	128.843627929688\\
0.964999973773956	127.833084106445\\
0.970000028610229	126.997627258301\\
0.975000023841858	126.330497741699\\
0.980000019073486	125.830673217773\\
0.985000014305115	125.498733520508\\
0.990000009536743	125.328094482422\\
0.995000004768372	125.317077636719\\
1	125.467506408691\\
1.00499999523163	125.777702331543\\
1.00999999046326	126.24227142334\\
1.01499998569489	126.80078125\\
1.01999998092651	127.455070495605\\
1.02499997615814	128.215682983398\\
1.02999997138977	129.086608886719\\
1.0349999666214	130.05908203125\\
1.03999996185303	131.132781982422\\
1.04499995708466	132.297729492188\\
1.04999995231628	133.550537109375\\
1.05499994754791	134.881393432617\\
1.05999994277954	136.283752441406\\
1.06500005722046	137.754119873047\\
1.07000005245209	139.279754638672\\
1.07500004768372	140.854187011719\\
1.08000004291534	142.465042114258\\
1.08500003814697	144.104934692383\\
1.0900000333786	145.766693115234\\
1.09500002861023	147.441390991211\\
1.10000002384186	149.120635986328\\
1.10500001907349	150.797454833984\\
1.11000001430511	152.467529296875\\
1.11500000953674	154.121612548828\\
1.12000000476837	155.752334594727\\
1.125	157.356658935547\\
1.12999999523163	158.927795410156\\
1.13499999046326	160.458618164063\\
1.13999998569489	161.943786621094\\
1.14499998092651	163.38117980957\\
1.14999997615814	164.762985229492\\
1.15499997138977	166.083801269531\\
1.1599999666214	167.341445922852\\
1.16499996185303	168.531356811523\\
1.16999995708466	169.64778137207\\
1.17499995231628	170.687149047852\\
1.17999994754791	171.652694702148\\
1.18499994277954	172.537445068359\\
1.19000005722046	173.338363647461\\
1.19500005245209	174.054748535156\\
1.20000004768372	174.685516357422\\
1.20500004291534	175.227966308594\\
1.21000003814697	175.681121826172\\
1.2150000333786	176.050552368164\\
1.22000002861023	176.335830688477\\
1.22500002384186	176.539169311523\\
1.23000001907349	176.666854858398\\
1.23500001430511	176.726791381836\\
1.24000000953674	176.723983764648\\
1.24500000476837	176.666397094727\\
1.25	176.526947021484\\
1.25499999523163	176.316925048828\\
1.25999999046326	176.032943725586\\
1.26499998569489	175.677474975586\\
1.26999998092651	175.257293701172\\
1.27499997615814	174.768692016602\\
1.27999997138977	174.216979980469\\
1.2849999666214	173.620178222656\\
1.28999996185303	172.974700927734\\
1.29499995708466	172.285720825195\\
1.29999995231628	171.563186645508\\
1.30499994754791	170.810592651367\\
1.30999994277954	170.033554077148\\
1.31500005722046	169.234497070313\\
1.32000005245209	168.419509887695\\
1.32500004768372	167.593856811523\\
1.33000004291534	166.759735107422\\
1.33500003814697	165.921096801758\\
1.3400000333786	165.082809448242\\
1.34500002861023	164.249298095703\\
1.35000002384186	163.425155639648\\
1.35500001907349	162.614639282227\\
1.36000001430511	161.821212768555\\
1.36500000953674	161.04573059082\\
1.37000000476837	160.291076660156\\
1.375	159.558654785156\\
1.37999999523163	158.846252441406\\
1.38499999046326	158.158599853516\\
1.38999998569489	157.496948242188\\
1.39499998092651	156.867370605469\\
1.39999997615814	156.272491455078\\
1.40499997138977	155.716049194336\\
1.4099999666214	155.212310791016\\
1.41499996185303	154.762893676758\\
1.41999995708466	154.369964599609\\
1.42499995231628	154.02116394043\\
1.42999994754791	153.722854614258\\
1.43499994277954	153.47380065918\\
1.44000005722046	153.270065307617\\
1.44500005245209	153.112289428711\\
1.45000004768372	152.996002197266\\
1.45500004291534	152.915130615234\\
1.46000003814697	152.868759155273\\
1.4650000333786	152.85823059082\\
1.47000002861023	152.885894775391\\
1.47500002384186	152.951675415039\\
1.48000001907349	153.058700561523\\
1.48500001430511	153.212371826172\\
1.49000000953674	153.418838500977\\
1.49500000476837	153.65934753418\\
1.5	153.932876586914\\
1.50499999523163	154.240173339844\\
1.50999999046326	154.584732055664\\
1.51499998569489	154.964309692383\\
1.51999998092651	155.374862670898\\
1.52499997615814	155.814666748047\\
1.52999997138977	156.283157348633\\
1.5349999666214	156.77880859375\\
1.53999996185303	157.299072265625\\
1.54499995708466	157.839889526367\\
1.54999995231628	158.396957397461\\
1.55499994754791	158.968475341797\\
1.55999994277954	159.548126220703\\
1.56500005722046	160.132873535156\\
1.57000005245209	160.399230957031\\
1.57500004768372	160.917358398438\\
1.58000004291534	161.196441650391\\
1.58500003814697	161.295394897461\\
1.5900000333786	161.354476928711\\
1.59500002861023	161.433868408203\\
1.60000002384186	161.576965332031\\
1.60500001907349	161.827835083008\\
1.61000001430511	162.210174560547\\
1.61500000953674	162.688934326172\\
1.62000000476837	163.279724121094\\
1.625	163.881332397461\\
1.62999999523163	164.439926147461\\
1.63499999046326	164.885604858398\\
1.63999998569489	165.177108764648\\
1.64499998092651	165.441116333008\\
1.64999997615814	165.627822875977\\
1.65499997138977	165.743545532227\\
1.6599999666214	165.80207824707\\
1.66499996185303	165.813293457031\\
1.66999995708466	165.771667480469\\
1.67499995231628	165.717712402344\\
1.67999994754791	165.687118530273\\
1.68499994277954	165.652969360352\\
1.69000005722046	165.600311279297\\
1.69500005245209	165.538070678711\\
1.70000004768372	165.463500976563\\
1.70500004291534	165.362411499023\\
1.71000003814697	165.232955932617\\
1.7150000333786	165.081192016602\\
1.72000002861023	164.910537719727\\
1.72500002384186	164.705551147461\\
1.73000001907349	164.471664428711\\
1.73500001430511	164.205322265625\\
1.74000000953674	163.906799316406\\
1.74500000476837	163.579055786133\\
1.75	163.227874755859\\
1.75499999523163	162.850128173828\\
1.75999999046326	162.450973510742\\
1.76499998569489	162.029602050781\\
1.76999998092651	161.588363647461\\
1.77499997615814	161.13591003418\\
1.77999997138977	160.673110961914\\
1.7849999666214	160.197860717773\\
1.78999996185303	159.711715698242\\
1.79499995708466	159.215896606445\\
1.79999995231628	158.71028137207\\
1.80499994754791	158.195755004883\\
1.80999994277954	157.671737670898\\
1.81500005722046	157.138412475586\\
1.82000005245209	156.596038818359\\
1.82500004768372	156.045211791992\\
1.83000004291534	155.485992431641\\
1.83500003814697	154.917938232422\\
1.8400000333786	154.3408203125\\
1.84500002861023	153.755508422852\\
1.85000002384186	153.170623779297\\
1.85500001907349	152.582305908203\\
1.86000001430511	151.990463256836\\
1.86500000953674	151.394729614258\\
1.87000000476837	150.795288085938\\
1.875	150.200164794922\\
1.87999999523163	149.615661621094\\
1.88499999046326	149.037048339844\\
1.88999998569489	148.463851928711\\
1.89499998092651	147.877014160156\\
1.89999997615814	147.286392211914\\
1.90499997138977	146.692001342773\\
1.9099999666214	146.094055175781\\
1.91499996185303	145.492401123047\\
1.91999995708466	144.887084960938\\
1.92499995231628	144.283630371094\\
1.92999994754791	143.681793212891\\
1.93499994277954	143.080337524414\\
1.94000005722046	142.47607421875\\
1.94500005245209	141.867797851563\\
1.95000004768372	141.257049560547\\
1.95500004291534	140.648742675781\\
1.96000003814697	140.050857543945\\
1.9650000333786	139.463150024414\\
1.97000002861023	138.877105712891\\
1.97500002384186	138.261642456055\\
1.98000001907349	137.622665405273\\
1.98500001430511	136.957412719727\\
1.99000000953674	136.285888671875\\
1.99500000476837	135.59831237793\\
2	134.895141601563\\
2.00500011444092	134.196182250977\\
2.00999999046326	133.498672485352\\
2.01500010490417	132.80729675293\\
2.01999998092651	132.159439086914\\
2.02500009536743	131.559112548828\\
2.02999997138977	131.014144897461\\
2.03500008583069	130.488922119141\\
2.03999996185303	129.996185302734\\
2.04500007629395	129.558563232422\\
2.04999995231628	129.208740234375\\
2.0550000667572	128.946441650391\\
2.05999994277954	128.765426635742\\
2.06500005722046	128.673599243164\\
2.0699999332428	128.652740478516\\
2.07500004768372	128.707885742188\\
2.07999992370605	128.837890625\\
2.08500003814697	129.042144775391\\
2.08999991416931	129.309616088867\\
2.09500002861023	129.6396484375\\
2.09999990463257	130.021713256836\\
2.10500001907349	130.442138671875\\
2.10999989509583	130.917785644531\\
2.11500000953674	131.451171875\\
2.11999988555908	132.037109375\\
2.125	132.695175170898\\
2.13000011444092	133.391464233398\\
2.13499999046326	134.109024047852\\
2.14000010490417	134.813278198242\\
2.14499998092651	135.482894897461\\
2.15000009536743	136.172637939453\\
2.15499997138977	136.759460449219\\
2.16000008583069	137.068008422852\\
2.16499996185303	137.317779541016\\
2.17000007629395	137.494720458984\\
2.17499995231628	137.498962402344\\
2.1800000667572	137.413391113281\\
2.18499994277954	137.280303955078\\
2.19000005722046	137.16487121582\\
2.1949999332428	137.053100585938\\
2.20000004768372	136.959823608398\\
2.20499992370605	136.894271850586\\
2.21000003814697	136.851455688477\\
2.21499991416931	136.828826904297\\
2.22000002861023	136.834930419922\\
2.22499990463257	136.875640869141\\
2.23000001907349	136.884384155273\\
2.23499989509583	136.751754760742\\
2.24000000953674	136.635864257813\\
2.24499988555908	136.531066894531\\
2.25	136.406539916992\\
2.25500011444092	136.302536010742\\
2.25999999046326	136.272216796875\\
2.26500010490417	136.401428222656\\
2.26999998092651	136.747467041016\\
2.27500009536743	137.327377319336\\
2.27999997138977	138.099075317383\\
2.28500008583069	138.953033447266\\
2.28999996185303	139.91877746582\\
2.29500007629395	140.993499755859\\
2.29999995231628	142.135925292969\\
2.3050000667572	143.367904663086\\
2.30999994277954	144.673294067383\\
2.31500005722046	146.042098999023\\
2.3199999332428	147.458541870117\\
2.32500004768372	148.907531738281\\
2.32999992370605	150.386199951172\\
2.33500003814697	151.872375488281\\
2.33999991416931	153.385177612305\\
2.34500002861023	154.940551757813\\
2.34999990463257	156.470108032227\\
2.35500001907349	158.012130737305\\
2.35999989509583	159.552810668945\\
2.36500000953674	161.019470214844\\
2.36999988555908	162.379196166992\\
2.375	163.65559387207\\
2.38000011444092	164.818405151367\\
2.38499999046326	165.813568115234\\
2.39000010490417	166.604461669922\\
2.39499998092651	167.161315917969\\
2.40000009536743	167.491470336914\\
2.40499997138977	167.581405639648\\
2.41000008583069	167.486190795898\\
2.41499996185303	167.245483398438\\
2.42000007629395	166.908706665039\\
2.42499995231628	166.605010986328\\
2.4300000667572	166.306701660156\\
2.43499994277954	166.098495483398\\
2.44000005722046	166.014511108398\\
2.4449999332428	165.853164672852\\
2.45000004768372	166.024673461914\\
2.45499992370605	166.26985168457\\
2.46000003814697	166.688293457031\\
2.46499991416931	167.321762084961\\
2.47000002861023	168.125564575195\\
2.47499990463257	169.071563720703\\
2.48000001907349	170.140701293945\\
2.48499989509583	171.297622680664\\
2.49000000953674	172.565856933594\\
2.49499988555908	173.980041503906\\
2.5	175.463806152344\\
2.50500011444092	176.817291259766\\
2.50999999046326	178.010482788086\\
2.51500010490417	179.1748046875\\
2.51999998092651	180.190475463867\\
2.52500009536743	181.058822631836\\
2.52999997138977	181.822463989258\\
2.53500008583069	182.503646850586\\
2.53999996185303	183.076309204102\\
2.54500007629395	183.588195800781\\
2.54999995231628	183.997802734375\\
2.5550000667572	184.302200317383\\
2.55999994277954	184.458023071289\\
2.56500005722046	184.446304321289\\
2.5699999332428	184.2373046875\\
2.57500004768372	183.805511474609\\
2.57999992370605	183.149230957031\\
2.58500003814697	182.275802612305\\
2.58999991416931	181.120986938477\\
2.59500002861023	179.831451416016\\
2.59999990463257	178.415603637695\\
2.60500001907349	176.814041137695\\
2.60999989509583	175.108657836914\\
2.61500000953674	173.307144165039\\
2.61999988555908	171.456481933594\\
2.625	169.567596435547\\
2.63000011444092	167.646286010742\\
2.63499999046326	165.594741821289\\
2.64000010490417	163.569381713867\\
2.64499998092651	161.502151489258\\
2.65000009536743	159.169708251953\\
2.65499997138977	156.665542602539\\
2.66000008583069	153.971817016602\\
2.66499996185303	151.002136230469\\
2.67000007629395	147.764785766602\\
2.67499995231628	144.414566040039\\
2.6800000667572	141.084030151367\\
2.68499994277954	137.575332641602\\
2.69000005722046	134.128555297852\\
2.6949999332428	130.768203735352\\
2.70000004768372	127.522880554199\\
2.70499992370605	124.421264648438\\
2.71000003814697	121.454902648926\\
2.71499991416931	118.601615905762\\
2.72000002861023	115.864295959473\\
2.72499990463257	113.216720581055\\
2.73000001907349	110.63883972168\\
2.73499989509583	108.125617980957\\
2.74000000953674	105.672210693359\\
2.74499988555908	103.30640411377\\
2.75	101.054351806641\\
2.75500011444092	98.9748382568359\\
2.75999999046326	97.1699600219727\\
2.76500010490417	95.714729309082\\
2.76999998092651	94.5739288330078\\
2.77500009536743	93.7072525024414\\
2.77999997138977	93.1146392822266\\
2.78500008583069	92.8161392211914\\
2.78999996185303	92.7904891967773\\
2.79500007629395	92.9550628662109\\
2.79999995231628	93.2467269897461\\
2.8050000667572	93.6169357299805\\
2.80999994277954	94.0146560668945\\
2.81500005722046	94.3957824707031\\
2.8199999332428	94.7298278808594\\
2.82500004768372	95.0249786376953\\
2.82999992370605	95.2983856201172\\
2.83500003814697	95.6617813110352\\
2.83999991416931	96.1654205322266\\
2.84500002861023	96.9145736694336\\
2.84999990463257	98.0563201904297\\
2.85500001907349	99.6260299682617\\
2.85999989509583	101.749099731445\\
2.86500000953674	104.531608581543\\
2.86999988555908	107.846824645996\\
2.875	111.908264160156\\
2.88000011444092	116.659767150879\\
2.88499999046326	122.135192871094\\
2.89000010490417	128.396713256836\\
2.89499998092651	135.395767211914\\
2.90000009536743	143.174026489258\\
2.90499997138977	151.779754638672\\
2.91000008583069	161.209625244141\\
2.91499996185303	171.424652099609\\
2.92000007629395	182.310592651367\\
2.92499995231628	193.662628173828\\
2.9300000667572	205.215698242188\\
2.93499994277954	216.667892456055\\
2.94000005722046	227.690093994141\\
2.9449999332428	237.948974609375\\
2.95000004768372	247.256378173828\\
2.95499992370605	255.40364074707\\
2.96000003814697	262.337707519531\\
2.96499991416931	268.522796630859\\
2.97000002861023	273.498748779297\\
2.97499990463257	277.033813476563\\
2.98000001907349	279.192413330078\\
2.98499989509583	279.966400146484\\
2.99000000953674	279.941009521484\\
2.99499988555908	278.749969482422\\
3	276.850891113281\\
3.00500011444092	274.357971191406\\
3.00999999046326	271.302001953125\\
3.01500010490417	267.655334472656\\
3.01999998092651	263.323272705078\\
3.02500009536743	258.236450195313\\
3.02999997138977	252.181213378906\\
3.03500008583069	244.983535766602\\
3.03999996185303	236.68977355957\\
3.04500007629395	227.345031738281\\
3.04999995231628	217.193908691406\\
3.0550000667572	206.546463012695\\
3.05999994277954	195.544998168945\\
3.06500005722046	184.643844604492\\
3.0699999332428	174.211761474609\\
3.07500004768372	164.044418334961\\
3.07999992370605	154.437118530273\\
3.08500003814697	145.333984375\\
3.08999991416931	136.533020019531\\
3.09500002861023	128.082321166992\\
3.09999990463257	119.987525939941\\
3.10500001907349	112.12109375\\
3.10999989509583	104.608879089355\\
3.11500000953674	97.5290908813477\\
3.11999988555908	91.0020294189453\\
3.125	85.2942810058594\\
3.13000011444092	80.8589324951172\\
3.13499999046326	78.2868728637695\\
3.14000010490417	76.6860656738281\\
3.14499998092651	75.7263031005859\\
3.15000009536743	74.9971923828125\\
3.15499997138977	74.2349548339844\\
3.16000008583069	73.2943420410156\\
3.16499996185303	72.188591003418\\
3.17000007629395	70.9182510375977\\
3.17499995231628	69.5318832397461\\
3.1800000667572	68.3356323242188\\
3.18499994277954	67.5769195556641\\
3.19000005722046	67.2618408203125\\
3.1949999332428	67.1784133911133\\
3.20000004768372	67.5397415161133\\
3.20499992370605	68.6983871459961\\
3.21000003814697	70.4833068847656\\
3.21499991416931	73.4116363525391\\
3.22000002861023	77.6218643188477\\
3.22499990463257	83.074104309082\\
3.23000001907349	89.843620300293\\
3.23499989509583	97.9232482910156\\
3.24000000953674	107.331504821777\\
3.24499988555908	117.943801879883\\
3.25	129.619659423828\\
3.25500011444092	142.107513427734\\
3.25999999046326	155.182189941406\\
3.26500010490417	168.508255004883\\
3.26999998092651	181.797393798828\\
3.27500009536743	194.671142578125\\
3.27999997138977	206.966842651367\\
3.28500008583069	218.530212402344\\
3.28999996185303	229.216400146484\\
3.29500007629395	239.084732055664\\
3.29999995231628	248.123046875\\
3.3050000667572	256.291229248047\\
3.30999994277954	263.899688720703\\
3.31500005722046	270.649993896484\\
3.3199999332428	276.622772216797\\
3.32500004768372	281.833526611328\\
3.32999992370605	286.097869873047\\
3.33500003814697	289.371398925781\\
3.33999991416931	291.754028320313\\
3.34500002861023	293.162750244141\\
3.34999990463257	293.140014648438\\
3.35500001907349	291.538635253906\\
3.35999989509583	288.281951904297\\
3.36500000953674	283.340850830078\\
3.36999988555908	276.844665527344\\
3.375	268.963836669922\\
3.38000011444092	259.947967529297\\
3.38499999046326	250.071746826172\\
3.39000010490417	239.601150512695\\
3.39499998092651	228.756927490234\\
3.40000009536743	217.700531005859\\
3.40499997138977	206.486724853516\\
3.41000008583069	195.169540405273\\
3.41499996185303	183.769104003906\\
3.42000007629395	172.303207397461\\
3.42499995231628	160.789016723633\\
3.4300000667572	149.331130981445\\
3.43499994277954	138.069854736328\\
3.44000005722046	127.23371887207\\
3.4449999332428	116.980506896973\\
3.45000004768372	107.543273925781\\
3.45499992370605	99.1843719482422\\
3.46000003814697	93.019775390625\\
3.46499991416931	88.4798889160156\\
3.47000002861023	84.4788208007813\\
3.47499990463257	80.8217697143555\\
3.48000001907349	77.2641220092773\\
3.48499989509583	73.7551651000977\\
3.49000000953674	69.7729568481445\\
3.49499988555908	66.0254440307617\\
3.5	62.9802856445313\\
3.50500011444092	60.2089157104492\\
3.50999999046326	57.8223342895508\\
3.51500010490417	56.2460746765137\\
3.51999998092651	54.9399108886719\\
3.52500009536743	54.5028381347656\\
3.52999997138977	54.3963050842285\\
3.53500008583069	54.7559814453125\\
3.53999996185303	55.8251571655273\\
3.54500007629395	58.1302871704102\\
3.54999995231628	63.6807098388672\\
3.5550000667572	73.3230056762695\\
3.55999994277954	87.4226913452148\\
3.56500005722046	106.148788452148\\
3.5699999332428	129.002487182617\\
3.57500004768372	154.761703491211\\
3.57999992370605	181.779098510742\\
3.58500003814697	208.081390380859\\
3.58999991416931	231.993316650391\\
3.59500002861023	252.17903137207\\
3.59999990463257	267.979797363281\\
3.60500001907349	278.927886962891\\
3.60999989509583	286.638244628906\\
3.61500000953674	291.719665527344\\
3.61999988555908	292.954193115234\\
3.625	291.577789306641\\
3.63000011444092	288.921630859375\\
3.63499999046326	286.562927246094\\
3.64000010490417	285.642425537109\\
3.64499998092651	286.526885986328\\
3.65000009536743	290.224517822266\\
3.65499997138977	294.070159912109\\
3.66000008583069	296.526824951172\\
3.66499996185303	297.088134765625\\
3.67000007629395	295.065948486328\\
3.67499995231628	288.841125488281\\
3.6800000667572	278.239074707031\\
3.68499994277954	263.682769775391\\
3.69000005722046	246.281799316406\\
3.6949999332428	227.399063110352\\
3.70000004768372	208.75700378418\\
3.70499992370605	191.450149536133\\
3.71000003814697	176.729919433594\\
3.71499991416931	164.655654907227\\
3.72000002861023	154.932678222656\\
3.72499990463257	146.870132446289\\
3.73000001907349	139.590362548828\\
3.73499989509583	132.201309204102\\
3.74000000953674	124.176261901855\\
3.74499988555908	115.29899597168\\
3.75	107.336578369141\\
3.75500011444092	100.617965698242\\
3.75999999046326	95.1863555908203\\
3.76500010490417	89.864143371582\\
3.76999998092651	84.3101196289063\\
3.77500009536743	79.0241928100586\\
3.77999997138977	74.1977996826172\\
3.78500008583069	69.7430572509766\\
3.78999996185303	66.5578384399414\\
3.79500007629395	64.8462600708008\\
3.79999995231628	63.6805534362793\\
3.8050000667572	62.6710205078125\\
3.80999994277954	61.309757232666\\
3.81500005722046	59.2266693115234\\
3.8199999332428	56.2201499938965\\
3.82500004768372	52.5755767822266\\
3.82999992370605	49.2330360412598\\
3.83500003814697	47.7718963623047\\
3.83999991416931	51.4501762390137\\
3.84500002861023	63.5541000366211\\
3.84999990463257	82.5095901489258\\
3.85500001907349	107.937721252441\\
3.85999989509583	138.336654663086\\
3.86500000953674	171.285263061523\\
3.86999988555908	203.968032836914\\
3.875	233.975509643555\\
3.88000011444092	259.360290527344\\
3.88499999046326	278.963134765625\\
3.89000010490417	292.409790039063\\
3.89499998092651	300.554534912109\\
3.90000009536743	306.592346191406\\
3.90499997138977	307.573669433594\\
3.91000008583069	305.582366943359\\
3.91499996185303	302.646148681641\\
3.92000007629395	300.510955810547\\
3.92499995231628	300.732818603516\\
3.9300000667572	304.840850830078\\
3.93499994277954	311.138275146484\\
3.94000005722046	316.778350830078\\
3.9449999332428	320.045196533203\\
3.95000004768372	320.102264404297\\
3.95499992370605	317.681274414063\\
3.96000003814697	309.992004394531\\
3.96499991416931	296.743865966797\\
3.97000002861023	278.624084472656\\
3.97499990463257	257.006408691406\\
3.98000001907349	233.770812988281\\
3.98499989509583	210.852722167969\\
3.99000000953674	189.830108642578\\
3.99499988555908	171.751403808594\\
4	157.023864746094\\
4.00500011444092	145.235702514648\\
4.01000022888184	135.531356811523\\
4.0149998664856	126.853187561035\\
4.01999998092651	118.122543334961\\
4.02500009536743	108.920692443848\\
4.03000020980835	100.12947845459\\
4.03499984741211	92.9689559936523\\
4.03999996185303	86.6254348754883\\
4.04500007629395	80.5478897094727\\
4.05000019073486	74.3773727416992\\
4.05499982833862	68.0647735595703\\
4.05999994277954	61.7352066040039\\
4.06500005722046	56.2980041503906\\
4.07000017166138	52.0793952941895\\
4.07499980926514	49.7863807678223\\
4.07999992370605	48.478443145752\\
4.08500003814697	47.9225997924805\\
4.09000015258789	47.0289344787598\\
4.09499979019165	44.2714958190918\\
4.09999990463257	38.1822700500488\\
4.10500001907349	28.723424911499\\
4.1100001335144	17.3017120361328\\
4.11499977111816	10.3831672668457\\
4.11999988555908	8.35290718078613\\
4.125	26.5122375488281\\
4.13000011444092	68.2845153808594\\
4.13500022888184	121.333435058594\\
4.1399998664856	180.751800537109\\
4.14499998092651	239.454467773438\\
4.15000009536743	291.296813964844\\
4.15500020980835	331.545166015625\\
4.15999984741211	357.669982910156\\
4.16499996185303	369.812835693359\\
4.17000007629395	377.331665039063\\
4.17500019073486	370.275329589844\\
4.17999982833862	351.555694580078\\
4.18499994277954	326.204193115234\\
4.19000005722046	301.061950683594\\
4.19500017166138	282.321899414063\\
4.19999980926514	275.183013916016\\
4.20499992370605	287.345336914063\\
4.21000003814697	306.647064208984\\
4.21500015258789	326.215911865234\\
4.21999979019165	340.530303955078\\
4.22499990463257	345.507873535156\\
4.23000001907349	342.717712402344\\
4.2350001335144	331.943756103516\\
4.23999977111816	309.248413085938\\
4.24499988555908	275.595184326172\\
4.25	237.405303955078\\
4.25500011444092	197.760269165039\\
4.26000022888184	162.092803955078\\
4.2649998664856	134.279693603516\\
4.26999998092651	116.276802062988\\
4.27500009536743	108.131988525391\\
4.28000020980835	109.913345336914\\
4.28499984741211	112.421630859375\\
4.28999996185303	112.829750061035\\
4.29500007629395	110.018348693848\\
4.30000019073486	105.64111328125\\
4.30499982833862	99.3375854492188\\
4.30999994277954	88.450439453125\\
4.31500005722046	75.5527420043945\\
4.32000017166138	63.8257293701172\\
4.32499980926514	54.8510589599609\\
4.32999992370605	49.7283325195313\\
4.33500003814697	49.0185928344727\\
4.34000015258789	50.7991256713867\\
4.34499979019165	53.718318939209\\
4.34999990463257	55.2601013183594\\
4.35500001907349	52.5444641113281\\
4.3600001335144	43.2112121582031\\
4.36499977111816	26.7409076690674\\
4.36999988555908	11.8472156524658\\
4.375	10.5327367782593\\
4.38000011444092	8.83798503875732\\
4.38500022888184	6.59936428070068\\
4.3899998664856	3.7416570186615\\
4.39499998092651	74.3863296508789\\
4.40000009536743	152.601959228516\\
4.40500020980835	232.742172241211\\
4.40999984741211	304.586578369141\\
4.41499996185303	360.282653808594\\
4.42000007629395	396.272735595703\\
4.42500019073486	412.249938964844\\
4.42999982833862	420.295166015625\\
4.43499994277954	410.630157470703\\
4.44000005722046	383.946441650391\\
4.44500017166138	347.281127929688\\
4.44999980926514	310.096099853516\\
4.45499992370605	281.828063964844\\
4.46000003814697	269.692443847656\\
4.46500015258789	284.970886230469\\
4.46999979019165	312.300720214844\\
4.47499990463257	339.899139404297\\
4.48000001907349	361.029541015625\\
4.4850001335144	369.855621337891\\
4.48999977111816	366.2060546875\\
4.49499988555908	355.039855957031\\
4.5	327.896606445313\\
4.50500011444092	287.578460693359\\
4.51000022888184	239.439208984375\\
4.5149998664856	190.003936767578\\
4.51999998092651	145.736068725586\\
4.52500009536743	112.067924499512\\
4.53000020980835	91.1099319458008\\
4.53499984741211	83.3525848388672\\
4.53999996185303	88.7527923583984\\
4.54500007629395	94.5861663818359\\
4.55000019073486	97.2273178100586\\
4.55499982833862	97.7492065429688\\
4.55999994277954	95.0408172607422\\
4.56500005722046	87.5561904907227\\
4.57000017166138	74.9581832885742\\
4.57499980926514	61.0519790649414\\
4.57999992370605	48.6276931762695\\
4.58500003814697	38.9167823791504\\
4.59000015258789	33.4228744506836\\
4.59499979019165	33.096981048584\\
4.59999990463257	37.0156059265137\\
4.60500001907349	43.5773963928223\\
4.6100001335144	49.2120895385742\\
4.61499977111816	48.2493362426758\\
4.61999988555908	34.8023719787598\\
4.625	17.4079189300537\\
4.63000011444092	19.99440574646\\
4.63500022888184	18.2449703216553\\
4.6399998664856	13.5081214904785\\
4.64499998092651	7.72701692581177\\
4.65000009536743	2.6946165561676\\
4.65500020980835	-0.721945285797119\\
4.65999984741211	-4.22778463363647\\
4.66499996185303	231.006103515625\\
4.67000007629395	363.925415039063\\
4.67500019073486	460.86962890625\\
4.67999982833862	515.613037109375\\
4.68499994277954	531.051391601563\\
4.69000005722046	536.863220214844\\
4.69500017166138	502.411346435547\\
4.69999980926514	431.769012451172\\
4.70499992370605	338.518859863281\\
4.71000003814697	244.671127319336\\
4.71500015258789	173.630783081055\\
4.71999979019165	142.128234863281\\
4.72499990463257	180.08056640625\\
4.73000001907349	242.444366455078\\
4.7350001335144	308.828857421875\\
4.73999977111816	361.206298828125\\
4.74499988555908	388.212249755859\\
4.75	385.302307128906\\
4.75500011444092	371.505218505859\\
4.76000022888184	329.253082275391\\
4.7649998664856	263.193267822266\\
4.76999998092651	183.733520507813\\
4.77500009536743	104.742835998535\\
4.78000020980835	39.2497787475586\\
4.78499984741211	5.21631383895874\\
4.78999996185303	1.06372725963593\\
4.79500007629395	-1.31027770042419\\
4.80000019073486	41.7638359069824\\
4.80499982833862	73.1132354736328\\
4.80999994277954	85.9270706176758\\
4.81500005722046	77.2372970581055\\
4.82000017166138	61.6290283203125\\
4.82499980926514	48.7550582885742\\
4.82999992370605	44.461841583252\\
4.83500003814697	49.4506988525391\\
4.84000015258789	62.1093063354492\\
4.84499979019165	77.7100219726563\\
4.84999990463257	88.4118423461914\\
4.85500001907349	83.1076812744141\\
4.8600001335144	53.3334579467773\\
4.86499977111816	10.0840282440186\\
4.86999988555908	9.46257972717285\\
4.875	14.3643865585327\\
4.88000011444092	18.1502113342285\\
4.88500022888184	16.9339065551758\\
4.8899998664856	12.4089250564575\\
4.89499998092651	6.44041204452515\\
4.90000009536743	2.14644026756287\\
4.90500020980835	-1.75881016254425\\
4.90999984741211	-5.31541109085083\\
4.91499996185303	-7.98875713348389\\
4.92000007629395	229.558151245117\\
4.92500019073486	393.661254882813\\
4.92999982833862	494.790954589844\\
4.93499994277954	536.223327636719\\
4.94000005722046	534.381652832031\\
4.94500017166138	521.144775390625\\
4.94999980926514	467.586120605469\\
4.95499992370605	387.967041015625\\
4.96000003814697	300.217803955078\\
4.96500015258789	224.101547241211\\
4.96999979019165	175.714691162109\\
4.97499990463257	171.494277954102\\
4.98000001907349	205.572128295898\\
4.9850001335144	249.165023803711\\
4.98999977111816	290.768157958984\\
4.99499988555908	321.379730224609\\
5	336.625823974609\\
5.00500011444092	335.352874755859\\
5.01000022888184	324.340545654297\\
5.0149998664856	304.655548095703\\
5.01999998092651	274.07958984375\\
5.02500009536743	236.195877075195\\
5.03000020980835	195.563003540039\\
5.03499984741211	156.093566894531\\
5.03999996185303	120.824432373047\\
5.04500007629395	91.7072906494141\\
5.05000019073486	69.6219482421875\\
5.05499982833862	54.6986351013184\\
5.05999994277954	46.0554275512695\\
5.06500005722046	42.1958999633789\\
5.07000017166138	41.0970916748047\\
5.07499980926514	40.7672996520996\\
5.07999992370605	39.2729187011719\\
5.08500003814697	35.7205123901367\\
5.09000015258789	28.8737812042236\\
5.09499979019165	18.3690013885498\\
5.09999990463257	5.47538661956787\\
5.10500001907349	1.38258075714111\\
5.1100001335144	0.646182537078857\\
5.11499977111816	0.327682673931122\\
5.11999988555908	0.180409803986549\\
5.125	0.119469009339809\\
5.13000011444092	0.0792234241962433\\
5.13500022888184	7.37457656860352\\
5.1399998664856	11.5953931808472\\
5.14499998092651	9.63610744476318\\
5.15000009536743	6.11971521377563\\
5.15500020980835	2.43299508094788\\
5.15999984741211	103.403717041016\\
5.16499996185303	180.469390869141\\
5.17000007629395	243.643692016602\\
5.17500019073486	287.526489257813\\
5.17999982833862	310.844268798828\\
5.18499994277954	316.582885742188\\
5.19000005722046	310.743713378906\\
5.19500017166138	298.269866943359\\
5.19999980926514	276.7470703125\\
5.20499992370605	251.161407470703\\
5.21000003814697	226.313583374023\\
5.21500015258789	206.020629882813\\
5.21999979019165	192.710098266602\\
5.22499990463257	188.657196044922\\
5.23000001907349	192.425537109375\\
5.2350001335144	198.963027954102\\
5.23999977111816	206.101379394531\\
5.24499988555908	212.446807861328\\
5.25	216.717361450195\\
5.25500011444092	218.3876953125\\
5.26000022888184	219.513305664063\\
5.2649998664856	217.620727539063\\
5.26999998092651	211.98649597168\\
5.27500009536743	202.641143798828\\
5.28000020980835	190.005233764648\\
5.28499984741211	175.103073120117\\
5.28999996185303	159.196212768555\\
5.29500007629395	143.80224609375\\
5.30000019073486	130.687194824219\\
5.30499982833862	121.173286437988\\
5.30999994277954	116.61678314209\\
5.31500005722046	116.38508605957\\
5.32000017166138	124.813659667969\\
5.32499980926514	134.757888793945\\
5.32999992370605	144.61833190918\\
5.33500003814697	152.443099975586\\
5.34000015258789	156.958236694336\\
5.34499979019165	159.439071655273\\
5.34999990463257	156.46760559082\\
5.35500001907349	147.483047485352\\
5.3600001335144	133.047607421875\\
5.36499977111816	114.866767883301\\
5.36999988555908	95.0677261352539\\
5.375	77.2236938476563\\
5.38000011444092	62.9988327026367\\
5.38500022888184	54.3643951416016\\
5.3899998664856	52.0207748413086\\
5.39499998092651	56.0372009277344\\
5.40000009536743	59.6281547546387\\
5.40500020980835	60.1121139526367\\
5.40999984741211	58.793888092041\\
5.41499996185303	51.8308448791504\\
5.42000007629395	38.3576316833496\\
5.42500019073486	19.0240840911865\\
5.42999982833862	1.26339757442474\\
5.43499994277954	0.202779948711395\\
5.44000005722046	-0.352453082799911\\
5.44500017166138	-0.440565973520279\\
5.44999980926514	-0.238210543990135\\
5.45499992370605	-0.150644838809967\\
5.46000003814697	-0.102490410208702\\
5.46500015258789	-0.0726656690239906\\
5.46999979019165	-0.0562714189291\\
5.47499990463257	-0.0352586172521114\\
5.48000001907349	-0.0432450957596302\\
5.4850001335144	-0.0230432506650686\\
5.48999977111816	-0.0257694683969021\\
5.49499988555908	-0.0204029139131308\\
5.5	-0.0173917468637228\\
5.50500011444092	-0.0119404466822743\\
5.51000022888184	-0.00719419308006763\\
5.5149998664856	-0.0115561364218593\\
5.51999998092651	0.25766858458519\\
5.52500009536743	1.86748147010803\\
5.53000020980835	1.63731622695923\\
5.53499984741211	1.39801371097565\\
5.53999996185303	1.14155721664429\\
5.54500007629395	0.885827422142029\\
5.55000019073486	0.610037088394165\\
5.55499982833862	0.366156578063965\\
5.55999994277954	0.192922472953796\\
5.56500005722046	0.0424385741353035\\
5.57000017166138	-0.131946116685867\\
5.57499980926514	-0.316670089960098\\
5.57999992370605	-0.467283397912979\\
5.58500003814697	-0.602273404598236\\
5.59000015258789	-0.728840708732605\\
5.59499979019165	-0.770244061946869\\
5.59999990463257	-0.81935715675354\\
5.60500001907349	-0.81314879655838\\
5.6100001335144	-0.7408686876297\\
5.61499977111816	-0.6567103266716\\
5.61999988555908	-0.567274451255798\\
5.625	-0.448319345712662\\
5.63000011444092	-0.324272245168686\\
5.63500022888184	-0.262323558330536\\
5.6399998664856	-0.160028785467148\\
5.64499998092651	-0.0824192017316818\\
5.65000009536743	-0.053125761449337\\
5.65500020980835	-0.0378858931362629\\
5.65999984741211	-0.0313375517725945\\
5.66499996185303	-0.0234033279120922\\
5.67000007629395	4.50296306610107\\
5.67500019073486	107.529136657715\\
5.67999982833862	136.016860961914\\
5.68499994277954	141.548187255859\\
5.69000005722046	137.703582763672\\
5.69500017166138	125.174873352051\\
5.69999980926514	106.685836791992\\
5.70499992370605	88.6159210205078\\
5.71000003814697	75.7392578125\\
5.71500015258789	72.4720764160156\\
5.71999979019165	77.3437805175781\\
5.72499990463257	84.0922470092773\\
5.73000001907349	89.8185119628906\\
5.7350001335144	92.9443283081055\\
5.73999977111816	92.7660903930664\\
5.74499988555908	89.3938446044922\\
5.75	83.6345977783203\\
5.75500011444092	76.6511459350586\\
5.76000022888184	69.3228454589844\\
5.7649998664856	62.0850715637207\\
5.76999998092651	54.3420372009277\\
5.77500009536743	45.9048118591309\\
5.78000020980835	35.630485534668\\
5.78499984741211	23.0889568328857\\
5.78999996185303	8.12199878692627\\
5.79500007629395	-9.55254554748535\\
5.80000019073486	-29.7092685699463\\
5.80499982833862	-51.7317008972168\\
5.80999994277954	-74.9580535888672\\
5.81500005722046	-98.3821105957031\\
5.82000017166138	-120.738143920898\\
5.82499980926514	-140.450729370117\\
5.82999992370605	-156.048858642578\\
5.83500003814697	-165.802886962891\\
5.84000015258789	-168.129364013672\\
5.84499979019165	-161.810317993164\\
5.84999990463257	-146.395980834961\\
5.85500001907349	-123.446815490723\\
5.8600001335144	-91.3052444458008\\
5.86499977111816	-53.1916847229004\\
5.86999988555908	-13.9079732894897\\
5.875	18.5939598083496\\
5.88000011444092	35.8492813110352\\
5.88500022888184	31.0439720153809\\
5.8899998664856	7.78453826904297\\
5.89499998092651	3.23829102516174\\
5.90000009536743	1.41619277000427\\
5.90500020980835	1.68902921676636\\
5.90999984741211	4.29813671112061\\
5.91499996185303	8.47150993347168\\
5.92000007629395	12.0004301071167\\
5.92500019073486	13.5640268325806\\
5.92999982833862	13.1887454986572\\
5.93499994277954	11.1351413726807\\
5.94000005722046	8.19376754760742\\
5.94500017166138	5.10470294952393\\
5.94999980926514	2.35250186920166\\
5.95499992370605	0.261585086584091\\
5.96000003814697	-0.678991436958313\\
5.96500015258789	-1.0094428062439\\
5.96999979019165	-1.45634722709656\\
5.97499990463257	-1.90075242519379\\
5.98000001907349	406.586730957031\\
5.9850001335144	739.771240234375\\
5.98999977111816	937.207946777344\\
5.99499988555908	1015.83233642578\\
6	1017.21508789063\\
6.00500011444092	995.35986328125\\
6.01000022888184	954.243041992188\\
6.0149998664856	867.439270019531\\
6.01999998092651	739.4326171875\\
6.02500009536743	586.934936523438\\
6.03000020980835	435.773345947266\\
6.03499984741211	314.949676513672\\
6.03999996185303	251.866424560547\\
6.04500007629395	285.714416503906\\
6.05000019073486	375.48095703125\\
6.05499982833862	479.413238525391\\
6.05999994277954	575.190734863281\\
6.06500005722046	646.093994140625\\
6.07000017166138	685.345153808594\\
6.07499980926514	691.630676269531\\
6.07999992370605	675.501953125\\
6.08500003814697	651.078674316406\\
6.09000015258789	601.9990234375\\
6.09499979019165	533.780700683594\\
6.09999990463257	453.117370605469\\
6.10500001907349	368.883026123047\\
6.1100001335144	292.056671142578\\
6.11499977111816	229.977111816406\\
6.11999988555908	188.768829345703\\
6.125	169.962875366211\\
6.13000011444092	180.234786987305\\
6.13500022888184	198.925598144531\\
6.1399998664856	216.593643188477\\
6.14499998092651	227.411483764648\\
6.15000009536743	228.072219848633\\
6.15500020980835	222.179641723633\\
6.15999984741211	207.669036865234\\
6.16499996185303	181.511520385742\\
6.17000007629395	145.406112670898\\
6.17500019073486	102.557228088379\\
6.17999982833862	57.221996307373\\
6.18499994277954	14.1331596374512\\
6.19000005722046	4.20898723602295\\
6.19500017166138	1.48130309581757\\
6.19999980926514	-0.105705790221691\\
6.20499992370605	-1.01163470745087\\
6.21000003814697	-1.55070579051971\\
6.21500015258789	-1.43309462070465\\
6.21999979019165	-0.642651498317719\\
6.22499990463257	-0.346492767333984\\
6.23000001907349	-0.187227576971054\\
6.2350001335144	-0.0992926731705666\\
6.23999977111816	-0.0563771426677704\\
6.24499988555908	-0.025412168353796\\
6.25	-0.00742929987609386\\
6.25500011444092	0.00268018385395408\\
6.26000022888184	0.00794372800737619\\
6.2649998664856	0.0113126123324037\\
6.26999998092651	0.0129280351102352\\
6.27500009536743	0.012860756367445\\
6.28000020980835	0.0123977214097977\\
6.28499984741211	0.0119275422766805\\
6.28999996185303	0.0115547887980938\\
6.29500007629395	0.0109056103974581\\
6.30000019073486	0.00993513502180576\\
6.30499982833862	0.00874836929142475\\
6.30999994277954	0.00777932535856962\\
6.31500005722046	0.00678244838491082\\
6.32000017166138	0.00561484834179282\\
6.32499980926514	0.00436674524098635\\
6.32999992370605	0.00346604618243873\\
6.33500003814697	0.0035664951428771\\
6.34000015258789	0.0045971991494298\\
6.34499979019165	0.00440966570749879\\
6.34999990463257	0.00332269258797169\\
6.35500001907349	0.00165680760983378\\
6.3600001335144	49.4294929504395\\
6.36499977111816	73.4990310668945\\
6.36999988555908	90.1909255981445\\
6.375	99.0782699584961\\
6.38000011444092	100.926155090332\\
6.38500022888184	97.8449859619141\\
6.3899998664856	101.500144958496\\
6.39499998092651	92.4517822265625\\
6.40000009536743	84.6384353637695\\
6.40500020980835	82.5735931396484\\
6.40999984741211	92.0907363891602\\
6.41499996185303	108.813194274902\\
6.42000007629395	131.298980712891\\
6.42500019073486	157.061096191406\\
6.42999982833862	183.595718383789\\
6.43499994277954	208.908798217773\\
6.44000005722046	231.223785400391\\
6.44500017166138	249.67121887207\\
6.44999980926514	263.802032470703\\
6.45499992370605	274.015960693359\\
6.46000003814697	280.744049072266\\
6.46500015258789	284.831604003906\\
6.46999979019165	287.304016113281\\
6.47499990463257	288.916473388672\\
6.48000001907349	290.374053955078\\
6.4850001335144	292.300659179688\\
6.48999977111816	294.982269287109\\
6.49499988555908	298.468353271484\\
6.5	302.668060302734\\
6.50500011444092	307.305725097656\\
6.51000022888184	311.967346191406\\
6.5149998664856	316.304626464844\\
6.51999998092651	319.971435546875\\
6.52500009536743	322.640502929688\\
6.53000020980835	324.095794677734\\
6.53499984741211	324.431915283203\\
6.53999996185303	323.563995361328\\
6.54500007629395	321.587158203125\\
6.55000019073486	318.653259277344\\
6.55499982833862	315.052276611328\\
6.55999994277954	311.083557128906\\
6.56500005722046	306.586334228516\\
6.57000017166138	301.600921630859\\
6.57499980926514	296.196319580078\\
6.57999992370605	290.813171386719\\
6.58500003814697	285.280639648438\\
6.59000015258789	279.730163574219\\
6.59499979019165	274.237152099609\\
6.59999990463257	268.729217529297\\
6.60500001907349	263.171295166016\\
6.6100001335144	257.581604003906\\
6.61499977111816	251.98258972168\\
6.61999988555908	246.188858032227\\
6.625	240.328277587891\\
6.63000011444092	234.340393066406\\
6.63500022888184	228.264709472656\\
6.6399998664856	222.245544433594\\
6.64499998092651	216.19091796875\\
6.65000009536743	210.216262817383\\
6.65500020980835	204.39909362793\\
6.65999984741211	198.703338623047\\
6.66499996185303	193.232894897461\\
6.67000007629395	188.155075073242\\
6.67500019073486	183.378173828125\\
6.67999982833862	179.06201171875\\
6.68499994277954	175.198120117188\\
6.69000005722046	171.66242980957\\
6.69500017166138	168.804489135742\\
6.69999980926514	166.80615234375\\
6.70499992370605	165.26481628418\\
6.71000003814697	164.16520690918\\
6.71500015258789	163.369979858398\\
6.71999979019165	162.940048217773\\
6.72499990463257	162.85905456543\\
6.73000001907349	163.013076782227\\
6.7350001335144	163.459869384766\\
6.73999977111816	164.201293945313\\
6.74499988555908	165.21125793457\\
6.75	166.578735351563\\
6.75500011444092	168.209487915039\\
6.76000022888184	169.921524047852\\
6.7649998664856	172.058929443359\\
6.76999998092651	174.386657714844\\
6.77500009536743	177.044296264648\\
6.78000020980835	179.968811035156\\
6.78499984741211	183.094711303711\\
6.78999996185303	186.452987670898\\
6.79500007629395	189.981826782227\\
6.80000019073486	193.622528076172\\
6.80499982833862	197.330764770508\\
6.80999994277954	201.125061035156\\
6.81500005722046	204.877578735352\\
6.82000017166138	208.619659423828\\
6.82499980926514	212.309982299805\\
6.82999992370605	215.898376464844\\
6.83500003814697	219.347625732422\\
6.84000015258789	222.619064331055\\
6.84499979019165	225.834197998047\\
6.84999990463257	228.88102722168\\
6.85500001907349	231.752944946289\\
6.8600001335144	234.444122314453\\
6.86499977111816	236.863784790039\\
6.86999988555908	238.999237060547\\
6.875	240.852203369141\\
6.88000011444092	242.286041259766\\
6.88500022888184	243.25749206543\\
6.8899998664856	243.696685791016\\
6.89499998092651	243.538436889648\\
6.90000009536743	242.774932861328\\
6.90500020980835	241.388931274414\\
6.90999984741211	239.375305175781\\
6.91499996185303	236.775253295898\\
6.92000007629395	233.645782470703\\
6.92500019073486	230.046600341797\\
6.92999982833862	226.050872802734\\
6.93499994277954	221.739059448242\\
6.94000005722046	217.174255371094\\
6.94500017166138	212.400939941406\\
6.94999980926514	207.450927734375\\
6.95499992370605	202.337265014648\\
6.96000003814697	197.077484130859\\
6.96500015258789	191.697036743164\\
6.96999979019165	186.170791625977\\
6.97499990463257	180.431411743164\\
6.98000001907349	174.551666259766\\
6.9850001335144	168.476959228516\\
6.98999977111816	162.218795776367\\
6.99499988555908	155.820465087891\\
7	149.325622558594\\
7.00500011444092	142.776123046875\\
7.01000022888184	136.226806640625\\
7.0149998664856	129.748962402344\\
7.01999998092651	123.398010253906\\
7.02500009536743	117.209381103516\\
7.03000020980835	111.212493896484\\
7.03499984741211	105.43448638916\\
7.03999996185303	99.8944320678711\\
7.04500007629395	94.6007995605469\\
7.05000019073486	89.5540771484375\\
7.05499982833862	84.7457504272461\\
7.05999994277954	80.1625061035156\\
7.06500005722046	75.7923126220703\\
7.07000017166138	71.6237258911133\\
7.07499980926514	67.6530380249023\\
7.07999992370605	63.8813400268555\\
7.08500003814697	60.3155937194824\\
7.09000015258789	56.9733657836914\\
7.09499979019165	53.8716125488281\\
7.09999990463257	51.0322570800781\\
7.10500001907349	48.4759902954102\\
7.1100001335144	46.224723815918\\
7.11499977111816	44.2883796691895\\
7.11999988555908	42.6868057250977\\
7.125	41.4203071594238\\
7.13000011444092	40.4524383544922\\
7.13500022888184	39.7799606323242\\
7.1399998664856	39.3869438171387\\
7.14499998092651	39.2345809936523\\
7.15000009536743	39.3195724487305\\
7.15500020980835	39.7001190185547\\
7.15999984741211	40.2307090759277\\
7.16499996185303	40.834358215332\\
7.17000007629395	41.6104125976563\\
7.17500019073486	42.5360908508301\\
7.17999982833862	43.6165008544922\\
7.18499994277954	44.8791770935059\\
7.19000005722046	46.3070755004883\\
7.19500017166138	47.9007759094238\\
7.19999980926514	49.653564453125\\
7.20499992370605	51.5303611755371\\
7.21000003814697	53.5344505310059\\
7.21500015258789	55.6343765258789\\
7.21999979019165	57.783504486084\\
7.22499990463257	59.9934692382813\\
7.23000001907349	62.2396583557129\\
7.2350001335144	64.5054473876953\\
7.23999977111816	66.78662109375\\
7.24499988555908	69.062126159668\\
7.25	71.3215942382813\\
7.25500011444092	73.5683135986328\\
7.26000022888184	75.7772445678711\\
7.2649998664856	77.9532241821289\\
7.26999998092651	80.1011352539063\\
7.27500009536743	82.1971130371094\\
7.28000020980835	84.2263717651367\\
7.28499984741211	86.191276550293\\
7.28999996185303	88.077880859375\\
7.29500007629395	89.8901901245117\\
7.30000019073486	91.5927276611328\\
7.30499982833862	93.1976013183594\\
7.30999994277954	94.692024230957\\
7.31500005722046	96.0783157348633\\
7.32000017166138	97.3407897949219\\
7.32499980926514	98.4838027954102\\
7.32999992370605	99.493408203125\\
7.33500003814697	100.368293762207\\
7.34000015258789	101.110046386719\\
7.34499979019165	101.715354919434\\
7.34999990463257	102.186393737793\\
7.35500001907349	102.524253845215\\
7.3600001335144	102.740394592285\\
7.36499977111816	102.836853027344\\
7.36999988555908	102.820915222168\\
7.375	102.702026367188\\
7.38000011444092	102.485763549805\\
7.38500022888184	102.14485168457\\
7.3899998664856	101.66283416748\\
7.39499998092651	101.045837402344\\
7.40000009536743	100.325798034668\\
7.40500020980835	99.4946823120117\\
7.40999984741211	98.5650482177734\\
7.41499996185303	97.5433349609375\\
7.42000007629395	96.437614440918\\
7.42500019073486	95.2575454711914\\
7.42999982833862	94.009147644043\\
7.43499994277954	92.6952209472656\\
7.44000005722046	91.323974609375\\
7.44500017166138	89.9038391113281\\
7.44999980926514	88.4411773681641\\
7.45499992370605	86.9483489990234\\
7.46000003814697	85.4387588500977\\
7.46500015258789	83.9211349487305\\
7.46999979019165	82.3976593017578\\
7.47499990463257	80.8762817382813\\
7.48000001907349	79.3633193969727\\
7.4850001335144	77.8637771606445\\
7.48999977111816	76.3823471069336\\
7.49499988555908	74.923957824707\\
7.5	73.4915771484375\\
7.50500011444092	72.0873107910156\\
7.51000022888184	70.7158279418945\\
7.5149998664856	69.3889770507813\\
7.51999998092651	68.1140747070313\\
7.52500009536743	66.9115982055664\\
7.53000020980835	65.7507019042969\\
7.53499984741211	64.6630477905273\\
7.53999996185303	63.6390037536621\\
7.54500007629395	62.6967315673828\\
7.55000019073486	61.8305549621582\\
7.55499982833862	61.0391540527344\\
7.55999994277954	60.3260955810547\\
7.56500005722046	59.6900329589844\\
7.57000017166138	59.1274757385254\\
7.57499980926514	58.6383171081543\\
7.57999992370605	58.2221527099609\\
7.58500003814697	57.8779029846191\\
7.59000015258789	57.6065979003906\\
7.59499979019165	57.4082450866699\\
7.59999990463257	57.2812194824219\\
7.60500001907349	57.2275352478027\\
7.6100001335144	57.2611885070801\\
7.61499977111816	57.3972625732422\\
7.61999988555908	57.6101951599121\\
7.625	57.8696212768555\\
7.63000011444092	58.1832084655762\\
7.63500022888184	58.5522689819336\\
7.6399998664856	58.9757537841797\\
7.64499998092651	59.4534912109375\\
7.65000009536743	59.9735107421875\\
7.65500020980835	60.5322113037109\\
7.65999984741211	61.1281509399414\\
7.66499996185303	61.7564506530762\\
7.67000007629395	62.4121284484863\\
7.67500019073486	63.0933456420898\\
7.67999982833862	63.8008041381836\\
7.68499994277954	64.5320739746094\\
7.69000005722046	65.2854156494141\\
7.69500017166138	66.0507354736328\\
7.69999980926514	66.8242034912109\\
7.70499992370605	67.6038513183594\\
7.71000003814697	68.3888854980469\\
7.71500015258789	69.1853103637695\\
7.71999979019165	69.9823989868164\\
7.72499990463257	70.7670822143555\\
7.73000001907349	71.536979675293\\
7.7350001335144	72.2869033813477\\
7.73999977111816	73.0276947021484\\
7.74499988555908	73.751953125\\
7.75	74.4431915283203\\
7.75500011444092	75.1089935302734\\
7.76000022888184	75.7527313232422\\
7.7649998664856	76.366813659668\\
7.76999998092651	76.9486694335938\\
7.77500009536743	77.4997940063477\\
7.78000020980835	78.019775390625\\
7.78499984741211	78.5061187744141\\
7.78999996185303	78.9610137939453\\
7.79500007629395	79.3844299316406\\
7.80000019073486	79.7724304199219\\
7.80499982833862	80.1265258789063\\
7.80999994277954	80.4468154907227\\
7.81500005722046	80.7315979003906\\
7.82000017166138	80.9806823730469\\
7.82499980926514	81.1946868896484\\
7.82999992370605	81.3749465942383\\
7.83500003814697	81.5232009887695\\
7.84000015258789	81.6381912231445\\
7.84499979019165	81.7206802368164\\
7.84999990463257	81.7746429443359\\
7.85500001907349	81.7991714477539\\
7.8600001335144	81.7952575683594\\
7.86499977111816	81.7654571533203\\
7.86999988555908	81.7103729248047\\
7.875	81.631217956543\\
7.88000011444092	81.5330581665039\\
7.88500022888184	81.4181137084961\\
7.8899998664856	81.2879409790039\\
7.89499998092651	81.1432647705078\\
7.90000009536743	80.9844284057617\\
7.90500020980835	80.8139038085938\\
7.90999984741211	80.6326446533203\\
7.91499996185303	80.4402847290039\\
7.92000007629395	80.2385864257813\\
7.92500019073486	80.0293350219727\\
7.92999982833862	79.8262405395508\\
7.93499994277954	79.6214218139648\\
7.94000005722046	79.4152145385742\\
7.94500017166138	79.2224273681641\\
7.94999980926514	79.0383682250977\\
7.95499992370605	78.861328125\\
7.96000003814697	78.7001647949219\\
7.96500015258789	78.5568771362305\\
7.96999979019165	78.4272613525391\\
7.97499990463257	78.3175201416016\\
7.98000001907349	78.2371597290039\\
7.9850001335144	78.1796035766602\\
7.98999977111816	78.1453170776367\\
7.99499988555908	78.1362686157227\\
8	78.1505889892578\\
8.00500011444092	78.1885299682617\\
8.01000022888184	78.2589797973633\\
8.01500034332275	78.3608779907227\\
8.02000045776367	78.4957427978516\\
8.02499961853027	78.6628646850586\\
8.02999973297119	78.8629302978516\\
8.03499984741211	79.0977325439453\\
8.03999996185303	79.3672027587891\\
8.04500007629395	79.670783996582\\
8.05000019073486	80.0099716186523\\
8.05500030517578	80.3848266601563\\
8.0600004196167	80.7938079833984\\
8.0649995803833	81.2378997802734\\
8.06999969482422	81.7168960571289\\
8.07499980926514	82.2337341308594\\
8.07999992370605	82.7862396240234\\
8.08500003814697	83.378532409668\\
8.09000015258789	84.0027618408203\\
8.09500026702881	84.6603698730469\\
8.10000038146973	85.3517227172852\\
8.10499954223633	86.0773849487305\\
8.10999965667725	86.8375701904297\\
8.11499977111816	87.6315460205078\\
8.11999988555908	88.4589385986328\\
8.125	89.3183517456055\\
8.13000011444092	90.2101211547852\\
8.13500022888184	91.1336898803711\\
8.14000034332275	92.0888748168945\\
8.14500045776367	93.0750732421875\\
8.14999961853027	94.0918426513672\\
8.15499973297119	95.1364669799805\\
8.15999984741211	96.2096099853516\\
8.16499996185303	97.3101654052734\\
8.17000007629395	98.4403610229492\\
8.17500019073486	99.6005096435547\\
8.18000030517578	100.789085388184\\
8.1850004196167	102.00479888916\\
8.1899995803833	103.245826721191\\
8.19499969482422	104.513832092285\\
8.19999980926514	105.809288024902\\
8.20499992370605	107.132614135742\\
8.21000003814697	108.481254577637\\
8.21500015258789	109.852119445801\\
8.22000026702881	111.24292755127\\
8.22500038146973	112.654747009277\\
8.22999954223633	114.092353820801\\
8.23499965667725	115.552047729492\\
8.23999977111816	117.029518127441\\
8.24499988555908	118.530319213867\\
8.25	120.055366516113\\
8.25500011444092	121.608673095703\\
8.26000022888184	123.186332702637\\
8.26500034332275	124.790664672852\\
8.27000045776367	126.416641235352\\
8.27499961853027	128.076065063477\\
8.27999973297119	129.751083374023\\
8.28499984741211	131.449295043945\\
8.28999996185303	133.171356201172\\
8.29500007629395	134.906707763672\\
8.30000019073486	136.651382446289\\
8.30500030517578	138.402252197266\\
8.3100004196167	140.166564941406\\
8.3149995803833	142.022003173828\\
8.31999969482422	143.877044677734\\
8.32499980926514	145.737075805664\\
8.32999992370605	147.631576538086\\
8.33500003814697	149.550491333008\\
8.34000015258789	151.488021850586\\
8.34500026702881	153.445358276367\\
8.35000038146973	155.424850463867\\
8.35499954223633	157.406661987305\\
8.35999965667725	159.421417236328\\
8.36499977111816	161.488204956055\\
8.36999988555908	163.517883300781\\
8.375	165.562591552734\\
8.38000011444092	167.612823486328\\
8.38500022888184	169.6572265625\\
8.39000034332275	171.702117919922\\
8.39500045776367	173.71891784668\\
8.39999961853027	175.761108398438\\
8.40499973297119	177.825576782227\\
8.40999984741211	179.827224731445\\
8.41499996185303	181.82633972168\\
8.42000007629395	183.786270141602\\
8.42500019073486	185.720184326172\\
8.43000030517578	187.626922607422\\
8.4350004196167	189.502746582031\\
8.4399995803833	191.352035522461\\
8.44499969482422	193.147369384766\\
8.44999980926514	194.867797851563\\
8.45499992370605	196.514068603516\\
8.46000003814697	198.074066162109\\
8.46500015258789	199.518249511719\\
8.47000026702881	200.820129394531\\
8.47500038146973	201.974639892578\\
8.47999954223633	202.987213134766\\
8.48499965667725	203.86962890625\\
8.48999977111816	204.58625793457\\
8.49499988555908	205.177474975586\\
8.5	205.663558959961\\
8.50500011444092	206.010101318359\\
8.51000022888184	206.345336914063\\
8.51500034332275	206.676162719727\\
8.52000045776367	206.969192504883\\
8.52499961853027	207.466552734375\\
8.52999973297119	207.984893798828\\
8.53499984741211	208.710067749023\\
8.53999996185303	209.625411987305\\
8.54500007629395	210.804641723633\\
8.55000019073486	212.275329589844\\
8.55500030517578	214.066558837891\\
8.5600004196167	216.281356811523\\
8.5649995803833	218.883819580078\\
8.56999969482422	221.877670288086\\
8.57499980926514	225.27018737793\\
8.57999992370605	228.910278320313\\
8.58500003814697	232.779022216797\\
8.59000015258789	236.77587890625\\
8.59500026702881	240.567794799805\\
8.60000038146973	244.000366210938\\
8.60499954223633	247.042465209961\\
8.60999965667725	249.611206054688\\
8.61499977111816	251.735717773438\\
8.61999988555908	253.474884033203\\
8.625	254.836547851563\\
8.63000011444092	255.940734863281\\
8.63500022888184	256.810089111328\\
8.64000034332275	257.530731201172\\
8.64500045776367	258.061645507813\\
8.64999961853027	258.565521240234\\
8.65499973297119	258.614685058594\\
8.65999984741211	259.247528076172\\
8.66499996185303	259.135070800781\\
8.67000007629395	258.82177734375\\
8.67500019073486	258.027130126953\\
8.68000030517578	256.961761474609\\
8.6850004196167	255.431091308594\\
8.6899995803833	253.341979980469\\
8.69499969482422	250.770645141602\\
8.69999980926514	247.829544067383\\
8.70499992370605	244.548873901367\\
8.71000003814697	240.912399291992\\
8.71500015258789	236.949966430664\\
8.72000026702881	233.009155273438\\
8.72500038146973	229.011581420898\\
8.72999954223633	225.014068603516\\
8.73499965667725	221.132171630859\\
8.73999977111816	217.426605224609\\
8.74499988555908	213.845428466797\\
8.75	210.326217651367\\
8.75500011444092	206.592391967773\\
8.76000022888184	203.055770874023\\
8.76500034332275	199.361511230469\\
8.77000045776367	195.622573852539\\
8.77499961853027	191.779983520508\\
8.77999973297119	187.89045715332\\
8.78499984741211	183.93034362793\\
8.78999996185303	179.955368041992\\
8.79500007629395	176.03141784668\\
8.80000019073486	172.221282958984\\
8.80500030517578	168.581756591797\\
8.8100004196167	165.169540405273\\
8.8149995803833	162.034774780273\\
8.81999969482422	159.180694580078\\
8.82499980926514	156.621810913086\\
8.82999992370605	154.348968505859\\
8.83500003814697	152.306823730469\\
8.84000015258789	150.452133178711\\
8.84500026702881	148.760559082031\\
8.85000038146973	147.153076171875\\
8.85499954223633	145.648178100586\\
8.85999965667725	144.213821411133\\
8.86499977111816	142.853469848633\\
8.86999988555908	141.582107543945\\
8.875	140.431274414063\\
8.88000011444092	139.439682006836\\
8.88500022888184	138.643447875977\\
8.89000034332275	138.073974609375\\
8.89500045776367	137.734771728516\\
8.89999961853027	137.626037597656\\
8.90499973297119	137.792495727539\\
8.90999984741211	138.151473999023\\
8.91499996185303	138.585861206055\\
8.92000007629395	139.157943725586\\
8.92500019073486	139.875854492188\\
8.93000030517578	140.709136962891\\
8.9350004196167	141.573699951172\\
8.9399995803833	142.317321777344\\
8.94499969482422	143.099533081055\\
8.94999980926514	143.951812744141\\
8.95499992370605	144.869277954102\\
8.96000003814697	145.858673095703\\
8.96500015258789	146.933166503906\\
8.97000026702881	148.116928100586\\
8.97500038146973	149.371765136719\\
8.97999954223633	150.682556152344\\
8.98499965667725	152.160369873047\\
8.98999977111816	153.631774902344\\
8.99499988555908	155.022994995117\\
9	156.514389038086\\
9.00500011444092	158.00276184082\\
9.01000022888184	159.463195800781\\
9.01500034332275	160.883758544922\\
9.02000045776367	162.268417358398\\
9.02499961853027	163.606979370117\\
9.02999973297119	164.8916015625\\
9.03499984741211	166.121994018555\\
9.03999996185303	167.295562744141\\
9.04500007629395	168.409530639648\\
9.05000019073486	169.461486816406\\
9.05500030517578	170.449096679688\\
9.0600004196167	171.365478515625\\
9.0649995803833	172.20768737793\\
9.06999969482422	172.970001220703\\
9.07499980926514	173.647598266602\\
9.07999992370605	174.23747253418\\
9.08500003814697	174.736358642578\\
9.09000015258789	175.135040283203\\
9.09500026702881	175.411651611328\\
9.10000038146973	175.559661865234\\
9.10499954223633	175.640350341797\\
9.10999965667725	175.630676269531\\
9.11499977111816	175.502243041992\\
9.11999988555908	175.245193481445\\
9.125	174.870941162109\\
9.13000011444092	174.389083862305\\
9.13500022888184	173.813919067383\\
9.14000034332275	173.196701049805\\
9.14500045776367	172.526626586914\\
9.14999961853027	171.834609985352\\
9.15499973297119	171.124938964844\\
9.15999984741211	170.398681640625\\
9.16499996185303	169.649826049805\\
9.17000007629395	168.870376586914\\
9.17500019073486	168.055038452148\\
9.18000030517578	167.195495605469\\
9.1850004196167	166.292861938477\\
9.1899995803833	165.343444824219\\
9.19499969482422	164.34553527832\\
9.19999980926514	163.307968139648\\
9.20499992370605	162.242630004883\\
9.21000003814697	161.168197631836\\
9.21500015258789	160.098510742188\\
9.22000026702881	159.033126831055\\
9.22500038146973	157.986953735352\\
9.22999954223633	156.95964050293\\
9.23499965667725	155.95573425293\\
9.23999977111816	154.988433837891\\
9.24499988555908	154.08576965332\\
9.25	153.2724609375\\
9.25500011444092	152.564987182617\\
9.26000022888184	152.028625488281\\
9.26500034332275	151.766235351563\\
9.27000045776367	152.073532104492\\
9.27499961853027	152.846450805664\\
9.27999973297119	154.112945556641\\
9.28499984741211	155.860610961914\\
9.28999996185303	158.101318359375\\
9.29500007629395	160.817367553711\\
9.30000019073486	163.973190307617\\
9.30500030517578	167.586166381836\\
9.3100004196167	171.509872436523\\
9.3149995803833	175.801559448242\\
9.31999969482422	180.324676513672\\
9.32499980926514	185.030975341797\\
9.32999992370605	189.764526367188\\
9.33500003814697	194.400283813477\\
9.34000015258789	198.714660644531\\
9.34500026702881	202.65950012207\\
9.35000038146973	205.809097290039\\
9.35499954223633	207.997177124023\\
9.35999965667725	209.051452636719\\
9.36499977111816	208.616851806641\\
9.36999988555908	206.769821166992\\
9.375	203.556411743164\\
9.38000011444092	199.098251342773\\
9.38500022888184	193.575454711914\\
9.39000034332275	187.553207397461\\
9.39500045776367	181.090347290039\\
9.39999961853027	174.903060913086\\
9.40499973297119	169.133697509766\\
9.40999984741211	163.940002441406\\
9.41499996185303	159.190963745117\\
9.42000007629395	154.467422485352\\
9.42500019073486	148.901077270508\\
9.43000030517578	141.319671630859\\
9.4350004196167	130.600723266602\\
9.4399995803833	114.927978515625\\
9.44499969482422	93.0072326660156\\
9.44999980926514	63.6666641235352\\
9.45499992370605	26.3638648986816\\
9.46000003814697	-18.1258354187012\\
9.46500015258789	-68.1303634643555\\
9.47000026702881	-119.587356567383\\
9.47500038146973	-167.864028930664\\
9.47999954223633	-207.926467895508\\
9.48499965667725	-236.527084350586\\
9.48999977111816	-237.881591796875\\
9.49499988555908	-234.400848388672\\
9.5	-217.213684082031\\
9.50500011444092	-189.036346435547\\
9.51000022888184	-154.084243774414\\
9.51500034332275	-116.256187438965\\
9.52000045776367	-79.3711853027344\\
9.52499961853027	-45.8551750183105\\
9.52999973297119	-19.3138828277588\\
9.53499984741211	0.00261668697930872\\
9.53999996185303	10.8316841125488\\
9.54500007629395	12.8314447402954\\
9.55000019073486	6.60004806518555\\
9.55500030517578	3.51906871795654\\
9.5600004196167	1.56925797462463\\
9.5649995803833	0.725603401660919\\
9.56999969482422	0.373838037252426\\
9.57499980926514	0.214956998825073\\
9.57999992370605	0.126066267490387\\
9.58500003814697	0.129277035593987\\
9.59000015258789	0.125172689557076\\
9.59500026702881	0.109616331756115\\
9.60000038146973	0.0898883417248726\\
9.60499954223633	0.0783872455358505\\
9.60999965667725	0.0864658057689667\\
9.61499977111816	0.071468897163868\\
9.61999988555908	0.0627377778291702\\
9.625	0.0511555448174477\\
9.63000011444092	0.0513419695198536\\
9.63500022888184	0.0414287447929382\\
9.64000034332275	0.038018811494112\\
9.64500045776367	0.0401882715523243\\
9.64999961853027	0.0306919869035482\\
9.65499973297119	0.0217575654387474\\
9.65999984741211	0.0232138969004154\\
9.66499996185303	0.0205714255571365\\
9.67000007629395	0.0164742227643728\\
9.67500019073486	0.014886561781168\\
9.68000030517578	0.0131833050400019\\
9.6850004196167	0.0111427269876003\\
9.6899995803833	0.00908126309514046\\
9.69499969482422	0.0072542573325336\\
9.69999980926514	0.00607231492176652\\
9.70499992370605	0.00573521805927157\\
9.71000003814697	0.00490828137844801\\
9.71500015258789	0.00277294032275677\\
9.72000026702881	0.00177936709951609\\
9.72500038146973	0.00257974537089467\\
9.72999954223633	0.00307516241446137\\
9.73499965667725	0.00311918021179736\\
9.73999977111816	0.00238236179575324\\
9.74499988555908	0.000480715185403824\\
9.75	-4.75644765174366e-06\\
9.75500011444092	-0.000823928567115217\\
9.76000022888184	-0.0020389452110976\\
9.76500034332275	-0.00289857387542725\\
9.77000045776367	-0.00273654912598431\\
9.77499961853027	-0.00191087462007999\\
9.77999973297119	-0.000374942377675325\\
9.78499984741211	-0.0006717155338265\\
9.78999996185303	-0.00134623469784856\\
9.79500007629395	-0.0024210384581238\\
9.80000019073486	-0.00311669544316828\\
9.80500030517578	-0.00281216809526086\\
9.8100004196167	-0.00219056103378534\\
9.8149995803833	-0.00126474327407777\\
9.81999969482422	-0.00143059424590319\\
9.82499980926514	-0.0018618272151798\\
9.82999992370605	-0.00257505755871534\\
9.83500003814697	-0.00307656405493617\\
9.84000015258789	-0.00301482784561813\\
9.84500026702881	-0.00280760019086301\\
9.85000038146973	-0.00244818511418998\\
9.85499954223633	-0.00247594434767962\\
9.85999965667725	-0.00246165390126407\\
9.86499977111816	-0.00238533411175013\\
9.86999988555908	-0.00241267075762153\\
9.875	-0.00261556566692889\\
9.88000011444092	-0.00289320782758296\\
9.88500022888184	-0.0031734227668494\\
9.89000034332275	-0.00265719345770776\\
9.89500045776367	-0.0019118448253721\\
9.89999961853027	-0.000984760350547731\\
9.90499973297119	-0.0010613618651405\\
9.90999984741211	-0.0025361564476043\\
9.91499996185303	-0.00441586086526513\\
9.92000007629395	-0.00642585754394531\\
9.92500019073486	-0.00494864722713828\\
9.93000030517578	-0.00278792390599847\\
9.9350004196167	1.77930869540432e-05\\
9.9399995803833	0.000804245704784989\\
9.94499969482422	-0.0004019454063382\\
9.94999980926514	-0.00142221804708242\\
9.95499992370605	-0.00219182320870459\\
9.96000003814697	-0.00283701904118061\\
9.96500015258789	-0.00343083217740059\\
9.97000026702881	-0.00400426518172026\\
9.97500038146973	-0.00361379375681281\\
9.97999954223633	-0.00316179590299726\\
9.98499965667725	-0.00289820157922804\\
9.98999977111816	-0.00285381474532187\\
9.99499988555908	-0.00267079682089388\\
10	-0.00253408495336771\\
};
\addlegendentry{CF}

\end{axis}

\begin{axis}[%
width=4.521in,
height=1.476in,
at={(0.758in,0.498in)},
scale only axis,
xmin=0,
xmax=10,
xlabel style={font=\color{white!15!black}},
xlabel={Time (s)},
ymin=-50,
ymax=106.913612365723,
ylabel style={font=\color{white!15!black}},
ylabel={FY (N)},
axis background/.style={fill=white},
xmajorgrids,
ymajorgrids,
legend style={legend cell align=left, align=left, draw=white!15!black}
]
\addplot [color=black, dashed, line width=2.0pt]
  table[row sep=crcr]{%
0.0949999988079071	0.210808470845222\\
0.100000001490116	0.199163168668747\\
0.104999996721745	0.18926927447319\\
0.109999999403954	0.180797472596169\\
0.115000002086163	0.173469662666321\\
0.119999997317791	0.348636746406555\\
0.125	0.97753894329071\\
0.129999995231628	1.32829701900482\\
0.135000005364418	1.57982647418976\\
0.140000000596046	1.74982357025146\\
0.144999995827675	1.85549604892731\\
0.150000005960464	1.91424489021301\\
0.155000001192093	1.94036269187927\\
0.159999996423721	1.96503162384033\\
0.165000006556511	1.98562169075012\\
0.170000001788139	1.98466336727142\\
0.174999997019768	1.95565736293793\\
0.180000007152557	1.89316427707672\\
0.185000002384186	1.79957854747772\\
0.189999997615814	1.66869747638702\\
0.194999992847443	23.0273017883301\\
0.200000002980232	36.1639213562012\\
0.204999998211861	43.5993995666504\\
0.209999993443489	46.394458770752\\
0.215000003576279	45.9041290283203\\
0.219999998807907	45.3420333862305\\
0.224999994039536	41.9886894226074\\
0.230000004172325	36.3740539550781\\
0.234999999403954	29.6083507537842\\
0.239999994635582	23.0812759399414\\
0.245000004768372	18.1484928131104\\
0.25	16.3407592773438\\
0.254999995231628	19.6784572601318\\
0.259999990463257	25.2396278381348\\
0.264999985694885	31.5829277038574\\
0.270000010728836	37.6180267333984\\
0.275000005960464	42.5632362365723\\
0.280000001192093	45.9191360473633\\
0.284999996423721	47.6446990966797\\
0.28999999165535	47.8047409057617\\
0.294999986886978	47.1846542358398\\
0.300000011920929	45.9069938659668\\
0.305000007152557	43.5206413269043\\
0.310000002384186	40.4366912841797\\
0.314999997615814	37.1287422180176\\
0.319999992847443	34.0797805786133\\
0.324999988079071	31.6923427581787\\
0.330000013113022	30.2283763885498\\
0.33500000834465	29.7853240966797\\
0.340000003576279	30.6444435119629\\
0.344999998807907	31.6913108825684\\
0.349999994039536	32.5991516113281\\
0.354999989271164	33.1139221191406\\
0.360000014305115	33.068790435791\\
0.365000009536743	32.5677871704102\\
0.370000004768372	31.742166519165\\
0.375	30.2829647064209\\
0.379999995231628	28.2741546630859\\
0.384999990463257	25.8749351501465\\
0.389999985694885	23.3039588928223\\
0.395000010728836	20.7891807556152\\
0.400000005960464	18.5904846191406\\
0.405000001192093	16.739559173584\\
0.409999996423721	15.4676923751831\\
0.41499999165535	14.6766405105591\\
0.419999986886978	14.3081064224243\\
0.425000011920929	14.3072605133057\\
0.430000007152557	14.2536268234253\\
0.435000002384186	14.0241680145264\\
0.439999997615814	13.7012805938721\\
0.444999992847443	13.0955152511597\\
0.449999988079071	12.1881246566772\\
0.455000013113022	11.0163707733154\\
0.46000000834465	9.66025638580322\\
0.465000003576279	8.23403835296631\\
0.469999998807907	6.86030292510986\\
0.474999994039536	5.66595792770386\\
0.479999989271164	4.74275255203247\\
0.485000014305115	4.14673519134521\\
0.490000009536743	3.88478088378906\\
0.495000004768372	4.05273866653442\\
0.5	4.3634181022644\\
0.504999995231628	4.69475698471069\\
0.509999990463257	4.97126340866089\\
0.514999985694885	5.14499616622925\\
0.519999980926514	5.19973564147949\\
0.524999976158142	5.15734720230103\\
0.529999971389771	5.10739469528198\\
0.535000026226044	4.97565269470215\\
0.540000021457672	4.81496095657349\\
0.545000016689301	4.68716907501221\\
0.550000011920929	4.64366006851196\\
0.555000007152557	4.79940605163574\\
0.560000002384186	5.09950256347656\\
0.564999997615814	5.52201509475708\\
0.569999992847443	6.04348754882813\\
0.574999988079071	6.62630319595337\\
0.579999983310699	7.24152994155884\\
0.584999978542328	7.86400890350342\\
0.589999973773956	8.46965503692627\\
0.595000028610229	9.05150890350342\\
0.600000023841858	9.60597610473633\\
0.605000019073486	10.1326370239258\\
0.610000014305115	10.6454477310181\\
0.615000009536743	11.1508188247681\\
0.620000004768372	11.6623401641846\\
0.625	12.1836967468262\\
0.629999995231628	12.7256288528442\\
0.634999990463257	13.2843866348267\\
0.639999985694885	13.8638038635254\\
0.644999980926514	14.4585218429565\\
0.649999976158142	15.0605497360229\\
0.654999971389771	15.6622333526611\\
0.660000026226044	16.2549800872803\\
0.665000021457672	16.830587387085\\
0.670000016689301	17.3825626373291\\
0.675000011920929	17.9052314758301\\
0.680000007152557	18.395601272583\\
0.685000002384186	18.8534603118896\\
0.689999997615814	19.2796001434326\\
0.694999992847443	19.6757183074951\\
0.699999988079071	20.0437355041504\\
0.704999983310699	20.3877506256104\\
0.709999978542328	20.7080783843994\\
0.714999973773956	21.0014514923096\\
0.720000028610229	21.2705497741699\\
0.725000023841858	21.5192699432373\\
0.730000019073486	21.7308254241943\\
0.735000014305115	21.9196033477783\\
0.740000009536743	22.0733814239502\\
0.745000004768372	22.1979370117188\\
0.75	22.2876758575439\\
0.754999995231628	22.3461647033691\\
0.759999990463257	22.3693981170654\\
0.764999985694885	22.3696823120117\\
0.769999980926514	22.3506698608398\\
0.774999976158142	22.3017272949219\\
0.779999971389771	22.2195510864258\\
0.785000026226044	22.107723236084\\
0.790000021457672	21.969409942627\\
0.795000016689301	21.8068313598633\\
0.800000011920929	21.6219062805176\\
0.805000007152557	21.4164943695068\\
0.810000002384186	21.1924667358398\\
0.814999997615814	20.9516716003418\\
0.819999992847443	20.6958122253418\\
0.824999988079071	20.4264221191406\\
0.829999983310699	20.1437187194824\\
0.834999978542328	19.8493690490723\\
0.839999973773956	19.5454902648926\\
0.845000028610229	19.2332096099854\\
0.850000023841858	18.9131469726563\\
0.855000019073486	18.5872669219971\\
0.860000014305115	18.2598133087158\\
0.865000009536743	17.9327087402344\\
0.870000004768372	17.6067638397217\\
0.875	17.2840328216553\\
0.879999995231628	16.9658527374268\\
0.884999990463257	16.6533432006836\\
0.889999985694885	16.3476791381836\\
0.894999980926514	16.0501117706299\\
0.899999976158142	15.7617454528809\\
0.904999971389771	15.4836311340332\\
0.910000026226044	15.2169990539551\\
0.915000021457672	14.9623966217041\\
0.920000016689301	14.7195014953613\\
0.925000011920929	14.4896812438965\\
0.930000007152557	14.2734937667847\\
0.935000002384186	14.0707416534424\\
0.939999997615814	13.8823261260986\\
0.944999992847443	13.7092514038086\\
0.949999988079071	13.553126335144\\
0.954999983310699	13.4144124984741\\
0.959999978542328	13.2930574417114\\
0.964999973773956	13.1889066696167\\
0.970000028610229	13.1027822494507\\
0.975000023841858	13.0339412689209\\
0.980000019073486	12.9822731018066\\
0.985000014305115	12.9479064941406\\
0.990000009536743	12.9301013946533\\
0.995000004768372	12.9288778305054\\
1	12.9443292617798\\
1.00499999523163	12.9765071868896\\
1.00999999046326	13.024787902832\\
1.01499998569489	13.0829219818115\\
1.01999998092651	13.1508941650391\\
1.02499997615814	13.2301692962646\\
1.02999997138977	13.3209733963013\\
1.0349999666214	13.4222860336304\\
1.03999996185303	13.5340166091919\\
1.04499995708466	13.6551475524902\\
1.04999995231628	13.7853193283081\\
1.05499994754791	13.9235410690308\\
1.05999994277954	14.0691318511963\\
1.06500005722046	14.2217273712158\\
1.07000005245209	14.3800220489502\\
1.07500004768372	14.5433254241943\\
1.08000004291534	14.7103700637817\\
1.08500003814697	14.8804092407227\\
1.0900000333786	15.0527181625366\\
1.09500002861023	15.2263784408569\\
1.10000002384186	15.4005289077759\\
1.10500001907349	15.5744562149048\\
1.11000001430511	15.7476921081543\\
1.11500000953674	15.9192771911621\\
1.12000000476837	16.0884456634521\\
1.125	16.2548713684082\\
1.12999999523163	16.4178466796875\\
1.13499999046326	16.5766162872314\\
1.13999998569489	16.7306175231934\\
1.14499998092651	16.8796253204346\\
1.14999997615814	17.0228595733643\\
1.15499997138977	17.1597423553467\\
1.1599999666214	17.2900371551514\\
1.16499996185303	17.4132957458496\\
1.16999995708466	17.5289363861084\\
1.17499995231628	17.6365814208984\\
1.17999994754791	17.7365493774414\\
1.18499994277954	17.828145980835\\
1.19000005722046	17.911039352417\\
1.19500005245209	17.9851512908936\\
1.20000004768372	18.0503978729248\\
1.20500004291534	18.1065254211426\\
1.21000003814697	18.1534442901611\\
1.2150000333786	18.1917266845703\\
1.22000002861023	18.2212963104248\\
1.22500002384186	18.2423782348633\\
1.23000001907349	18.2556133270264\\
1.23500001430511	18.2618370056152\\
1.24000000953674	18.2615795135498\\
1.24500000476837	18.2556476593018\\
1.25	18.2411880493164\\
1.25499999523163	18.2193202972412\\
1.25999999046326	18.1897430419922\\
1.26499998569489	18.1527576446533\\
1.26999998092651	18.1091480255127\\
1.27499997615814	18.0584850311279\\
1.27999997138977	18.001293182373\\
1.2849999666214	17.939416885376\\
1.28999996185303	17.8725757598877\\
1.29499995708466	17.8012771606445\\
1.29999995231628	17.7265243530273\\
1.30499994754791	17.6486949920654\\
1.30999994277954	17.5683708190918\\
1.31500005722046	17.4858055114746\\
1.32000005245209	17.40159034729\\
1.32500004768372	17.3162593841553\\
1.33000004291534	17.2300357818604\\
1.33500003814697	17.1433124542236\\
1.3400000333786	17.0565853118896\\
1.34500002861023	16.9703025817871\\
1.35000002384186	16.8849296569824\\
1.35500001907349	16.8009090423584\\
1.36000001430511	16.7185916900635\\
1.36500000953674	16.6380977630615\\
1.37000000476837	16.5598335266113\\
1.375	16.4839992523193\\
1.37999999523163	16.4103755950928\\
1.38499999046326	16.3393611907959\\
1.38999998569489	16.2711429595947\\
1.39499998092651	16.2063884735107\\
1.39999997615814	16.1452159881592\\
1.40499997138977	16.0879898071289\\
1.4099999666214	16.0361862182617\\
1.41499996185303	15.9899415969849\\
1.41999995708466	15.9494457244873\\
1.42499995231628	15.9133882522583\\
1.42999994754791	15.8824996948242\\
1.43499994277954	15.8566646575928\\
1.44000005722046	15.8354940414429\\
1.44500005245209	15.81907081604\\
1.45000004768372	15.806960105896\\
1.45500004291534	15.7985582351685\\
1.46000003814697	15.7937850952148\\
1.4650000333786	15.792820930481\\
1.47000002861023	15.795955657959\\
1.47500002384186	15.8032236099243\\
1.48000001907349	15.8149175643921\\
1.48500001430511	15.8314065933228\\
1.49000000953674	15.8531408309937\\
1.49500000476837	15.878228187561\\
1.5	15.906867980957\\
1.50499999523163	15.9391260147095\\
1.50999999046326	15.9752225875854\\
1.51499998569489	16.0149307250977\\
1.51999998092651	16.0579319000244\\
1.52499997615814	16.10400390625\\
1.52999997138977	16.1530151367188\\
1.5349999666214	16.2047710418701\\
1.53999996185303	16.2590885162354\\
1.54499995708466	16.3155403137207\\
1.54999995231628	16.3736343383789\\
1.55499994754791	16.4331474304199\\
1.55999994277954	16.493465423584\\
1.56500005722046	16.5542640686035\\
1.57000005245209	16.5828037261963\\
1.57500004768372	16.6176681518555\\
1.58000004291534	16.641674041748\\
1.58500003814697	16.6542949676514\\
1.5900000333786	16.6655540466309\\
1.59500002861023	16.6799659729004\\
1.60000002384186	16.7008686065674\\
1.60500001907349	16.7320537567139\\
1.61000001430511	16.7753620147705\\
1.61500000953674	16.826587677002\\
1.62000000476837	16.8874912261963\\
1.625	16.9474086761475\\
1.62999999523163	17.0015430450439\\
1.63499999046326	17.0435619354248\\
1.63999998569489	17.0702705383301\\
1.64499998092651	17.0959167480469\\
1.64999997615814	17.1142349243164\\
1.65499997138977	17.1257343292236\\
1.6599999666214	17.1319198608398\\
1.66499996185303	17.1336784362793\\
1.66999995708466	17.1303310394287\\
1.67499995231628	17.1260738372803\\
1.67999994754791	17.1244068145752\\
1.68499994277954	17.1218395233154\\
1.69000005722046	17.1168632507324\\
1.69500005245209	17.1103324890137\\
1.70000004768372	17.102201461792\\
1.70500004291534	17.0910930633545\\
1.71000003814697	17.0769023895264\\
1.7150000333786	17.0604095458984\\
1.72000002861023	17.0419311523438\\
1.72500002384186	17.0199069976807\\
1.73000001907349	16.994909286499\\
1.73500001430511	16.9666080474854\\
1.74000000953674	16.9350547790527\\
1.74500000476837	16.9005603790283\\
1.75	16.8637199401855\\
1.75499999523163	16.8242092132568\\
1.75999999046326	16.782543182373\\
1.76499998569489	16.7386245727539\\
1.76999998092651	16.6926822662354\\
1.77499997615814	16.6455917358398\\
1.77999997138977	16.5973949432373\\
1.7849999666214	16.5478343963623\\
1.78999996185303	16.4970760345459\\
1.79499995708466	16.4452381134033\\
1.79999995231628	16.3923072814941\\
1.80499994754791	16.3383903503418\\
1.80999994277954	16.2834300994873\\
1.81500005722046	16.2274646759033\\
1.82000005245209	16.1705455780029\\
1.82500004768372	16.1127376556396\\
1.83000004291534	16.0540714263916\\
1.83500003814697	15.9945106506348\\
1.8400000333786	15.9340600967407\\
1.84500002861023	15.8728055953979\\
1.85000002384186	15.8116273880005\\
1.85500001907349	15.7500848770142\\
1.86000001430511	15.6881589889526\\
1.86500000953674	15.6258058547974\\
1.87000000476837	15.5630521774292\\
1.875	15.5007162094116\\
1.87999999523163	15.4394311904907\\
1.88499999046326	15.37868309021\\
1.88999998569489	15.318413734436\\
1.89499998092651	15.2566413879395\\
1.89999997615814	15.194486618042\\
1.90499997138977	15.1319494247437\\
1.9099999666214	15.0690422058105\\
1.91499996185303	15.0057601928711\\
1.91999995708466	14.9421033859253\\
1.92499995231628	14.8786344528198\\
1.92999994754791	14.8153104782104\\
1.93499994277954	14.7519941329956\\
1.94000005722046	14.6883411407471\\
1.94500005245209	14.6242351531982\\
1.95000004768372	14.5598382949829\\
1.95500004291534	14.4956607818604\\
1.96000003814697	14.4325017929077\\
1.9650000333786	14.3702754974365\\
1.97000002861023	14.3080778121948\\
1.97500002384186	14.2427291870117\\
1.98000001907349	14.1749954223633\\
1.98500001430511	14.1046581268311\\
1.99000000953674	14.0338172912598\\
1.99500000476837	13.9613647460938\\
2	13.8873491287231\\
2.00500011444092	13.8137998580933\\
2.00999999046326	13.7403364181519\\
2.01500010490417	13.6673583984375\\
2.01999998092651	13.5987339019775\\
2.02500009536743	13.5347967147827\\
2.02999997138977	13.4762306213379\\
2.03500008583069	13.4193897247314\\
2.03999996185303	13.3659982681274\\
2.04500007629395	13.318751335144\\
2.04999995231628	13.2810363769531\\
2.0550000667572	13.2527933120728\\
2.05999994277954	13.233512878418\\
2.06500005722046	13.2240772247314\\
2.0699999332428	13.2227840423584\\
2.07500004768372	13.2300806045532\\
2.07999992370605	13.2462482452393\\
2.08500003814697	13.271032333374\\
2.08999991416931	13.3033447265625\\
2.09500002861023	13.3428449630737\\
2.09999990463257	13.3878087997437\\
2.10500001907349	13.437668800354\\
2.10999989509583	13.4931526184082\\
2.11500000953674	13.5543479919434\\
2.11999988555908	13.6215047836304\\
2.125	13.6952638626099\\
2.13000011444092	13.7720727920532\\
2.13499999046326	13.8501882553101\\
2.14000010490417	13.9256639480591\\
2.14499998092651	13.9967365264893\\
2.15000009536743	14.0694761276245\\
2.15499997138977	14.1316108703613\\
2.16000008583069	14.1647233963013\\
2.16499996185303	14.19260597229\\
2.17000007629395	14.2142915725708\\
2.17499995231628	14.2177677154541\\
2.1800000667572	14.2125511169434\\
2.18499994277954	14.2027111053467\\
2.19000005722046	14.1948299407959\\
2.1949999332428	14.1865186691284\\
2.20000004768372	14.1787261962891\\
2.20499992370605	14.1721782684326\\
2.21000003814697	14.1668033599854\\
2.21499991416931	14.1627712249756\\
2.22000002861023	14.1606845855713\\
2.22499990463257	14.1613578796387\\
2.23000001907349	14.1583385467529\\
2.23499989509583	14.1407852172852\\
2.24000000953674	14.1261901855469\\
2.24499988555908	14.1136178970337\\
2.25	14.0995206832886\\
2.25500011444092	14.0884704589844\\
2.25999999046326	14.0857858657837\\
2.26500010490417	14.0998287200928\\
2.26999998092651	14.135986328125\\
2.27500009536743	14.1954364776611\\
2.27999997138977	14.2739362716675\\
2.28500008583069	14.3596696853638\\
2.28999996185303	14.4565010070801\\
2.29500007629395	14.5651016235352\\
2.29999995231628	14.6810655593872\\
2.3050000667572	14.8056058883667\\
2.30999994277954	14.9376773834229\\
2.31500005722046	15.076003074646\\
2.3199999332428	15.2190895080566\\
2.32500004768372	15.3658065795898\\
2.32999992370605	15.5148820877075\\
2.33500003814697	15.6648101806641\\
2.33999991416931	15.8182783126831\\
2.34500002861023	15.9764423370361\\
2.34999990463257	16.1338787078857\\
2.35500001907349	16.2918071746826\\
2.35999989509583	16.4469528198242\\
2.36500000953674	16.5940799713135\\
2.36999988555908	16.7294158935547\\
2.375	16.8541221618652\\
2.38000011444092	16.9666748046875\\
2.38499999046326	17.0603561401367\\
2.39000010490417	17.132251739502\\
2.39499998092651	17.1799755096436\\
2.40000009536743	17.2047595977783\\
2.40499997138977	17.2063121795654\\
2.41000008583069	17.1892890930176\\
2.41499996185303	17.1596813201904\\
2.42000007629395	17.1227836608887\\
2.42499995231628	17.0919055938721\\
2.4300000667572	17.063756942749\\
2.43499994277954	17.0468616485596\\
2.44000005722046	17.0447025299072\\
2.4449999332428	17.0355339050293\\
2.45000004768372	17.0628852844238\\
2.45499992370605	17.0978050231934\\
2.46000003814697	17.1506824493408\\
2.46499991416931	17.2259998321533\\
2.47000002861023	17.316951751709\\
2.47499990463257	17.4215774536133\\
2.48000001907349	17.5381927490234\\
2.48499989509583	17.6631507873535\\
2.49000000953674	17.7991847991943\\
2.49499988555908	17.9496650695801\\
2.5	18.1066722869873\\
2.50500011444092	18.2496318817139\\
2.50999999046326	18.3770923614502\\
2.51500010490417	18.5031070709229\\
2.51999998092651	18.6155624389648\\
2.52500009536743	18.7146339416504\\
2.52999997138977	18.8034591674805\\
2.53500008583069	18.8833560943604\\
2.53999996185303	18.9529895782471\\
2.54500007629395	19.014856338501\\
2.54999995231628	19.0655384063721\\
2.5550000667572	19.1039543151855\\
2.55999994277954	19.1274547576904\\
2.56500005722046	19.1322078704834\\
2.5699999332428	19.1159954071045\\
2.57500004768372	19.0766048431396\\
2.57999992370605	19.0137996673584\\
2.58500003814697	18.929443359375\\
2.58999991416931	18.8160057067871\\
2.59500002861023	18.6891899108887\\
2.59999990463257	18.548490524292\\
2.60500001907349	18.3891334533691\\
2.60999989509583	18.2182464599609\\
2.61500000953674	18.036434173584\\
2.61999988555908	17.8477973937988\\
2.625	17.6531772613525\\
2.63000011444092	17.4528942108154\\
2.63499999046326	17.2369480133057\\
2.64000010490417	17.0218658447266\\
2.64499998092651	16.8002643585205\\
2.65000009536743	16.5490608215332\\
2.65499997138977	16.28005027771\\
2.66000008583069	15.9915266036987\\
2.66499996185303	15.6749801635742\\
2.67000007629395	15.3315658569336\\
2.67499995231628	14.97838306427\\
2.6800000667572	14.6281442642212\\
2.68499994277954	14.2587642669678\\
2.69000005722046	13.8965940475464\\
2.6949999332428	13.5430040359497\\
2.70000004768372	13.2003421783447\\
2.70499992370605	12.8712310791016\\
2.71000003814697	12.5545167922974\\
2.71499991416931	12.2478942871094\\
2.72000002861023	11.9520931243896\\
2.72499990463257	11.6654739379883\\
2.73000001907349	11.3856163024902\\
2.73499989509583	11.1126480102539\\
2.74000000953674	10.8467445373535\\
2.74499988555908	10.590274810791\\
2.75	10.3470811843872\\
2.75500011444092	10.1225128173828\\
2.75999999046326	9.92728805541992\\
2.76500010490417	9.76758670806885\\
2.76999998092651	9.64025688171387\\
2.77500009536743	9.54052448272705\\
2.77999997138977	9.46837902069092\\
2.78500008583069	9.42614650726318\\
2.78999996185303	9.4096565246582\\
2.79500007629395	9.41212558746338\\
2.79999995231628	9.42702865600586\\
2.8050000667572	9.45086002349854\\
2.80999994277954	9.4779167175293\\
2.81500005722046	9.50692749023438\\
2.8199999332428	9.53497886657715\\
2.82500004768372	9.56392383575439\\
2.82999992370605	9.59560775756836\\
2.83500003814697	9.64209938049316\\
2.83999991416931	9.70818519592285\\
2.84500002861023	9.80365943908691\\
2.84999990463257	9.94398784637451\\
2.85500001907349	10.1284952163696\\
2.85999989509583	10.3712120056152\\
2.86500000953674	10.681923866272\\
2.86999988555908	11.0527000427246\\
2.875	11.5004892349243\\
2.88000011444092	12.0217885971069\\
2.88499999046326	12.6211576461792\\
2.89000010490417	13.3029336929321\\
2.89499998092651	14.0668802261353\\
2.90000009536743	14.9118814468384\\
2.90499997138977	15.8408718109131\\
2.91000008583069	16.8494262695313\\
2.91499996185303	17.93137550354\\
2.92000007629395	19.0726432800293\\
2.92499995231628	20.2511749267578\\
2.9300000667572	21.4407901763916\\
2.93499994277954	22.6108493804932\\
2.94000005722046	23.7326488494873\\
2.9449999332428	24.776782989502\\
2.95000004768372	25.7252960205078\\
2.95499992370605	26.5538291931152\\
2.96000003814697	27.262674331665\\
2.96499991416931	27.9009628295898\\
2.97000002861023	28.4185390472412\\
2.97499990463257	28.7879943847656\\
2.98000001907349	29.0186767578125\\
2.98499989509583	29.1198272705078\\
2.99000000953674	29.1100730895996\\
2.99499988555908	29.0027866363525\\
3	28.8176422119141\\
3.00500011444092	28.5649719238281\\
3.00999999046326	28.2479629516602\\
3.01500010490417	27.8639316558838\\
3.01999998092651	27.4033374786377\\
3.02500009536743	26.8594570159912\\
3.02999997138977	26.213529586792\\
3.03500008583069	25.4497585296631\\
3.03999996185303	24.5747108459473\\
3.04500007629395	23.5935096740723\\
3.04999995231628	22.5348854064941\\
3.0550000667572	21.4287147521973\\
3.05999994277954	20.2883567810059\\
3.06500005722046	19.1591987609863\\
3.0699999332428	18.077220916748\\
3.07500004768372	17.0174999237061\\
3.07999992370605	16.0110321044922\\
3.08500003814697	15.0512342453003\\
3.08999991416931	14.1174764633179\\
3.09500002861023	13.2172002792358\\
3.09999990463257	12.3523759841919\\
3.10500001907349	11.5108880996704\\
3.10999989509583	10.7075929641724\\
3.11500000953674	9.9506893157959\\
3.11999988555908	9.25323295593262\\
3.125	8.64163589477539\\
3.13000011444092	8.16139316558838\\
3.13499999046326	7.8701229095459\\
3.14000010490417	7.66832590103149\\
3.14499998092651	7.52301216125488\\
3.15000009536743	7.3950080871582\\
3.15499997138977	7.26314783096313\\
3.16000008583069	7.1231746673584\\
3.16499996185303	6.98032855987549\\
3.17000007629395	6.83880233764648\\
3.17499995231628	6.70495176315308\\
3.1800000667572	6.61271572113037\\
3.18499994277954	6.58206558227539\\
3.19000005722046	6.60989999771118\\
3.1949999332428	6.67224073410034\\
3.20000004768372	6.78633117675781\\
3.20499992370605	6.98888826370239\\
3.21000003814697	7.25373792648315\\
3.21499991416931	7.63552618026733\\
3.22000002861023	8.1394739151001\\
3.22499990463257	8.76227378845215\\
3.23000001907349	9.50889015197754\\
3.23499989509583	10.3814096450806\\
3.24000000953674	11.3782291412354\\
3.24499988555908	12.4891271591187\\
3.25	13.6986856460571\\
3.25500011444092	14.9845514297485\\
3.25999999046326	16.3231945037842\\
3.26500010490417	17.6828289031982\\
3.26999998092651	19.0357475280762\\
3.27500009536743	20.3489646911621\\
3.27999997138977	21.6058406829834\\
3.28500008583069	22.7919769287109\\
3.28999996185303	23.8955497741699\\
3.29500007629395	24.9207324981689\\
3.29999995231628	25.8642807006836\\
3.3050000667572	26.719856262207\\
3.30999994277954	27.5181407928467\\
3.31500005722046	28.2250804901123\\
3.3199999332428	28.8498268127441\\
3.32500004768372	29.3916912078857\\
3.32999992370605	29.8327655792236\\
3.33500003814697	30.1662693023682\\
3.33999991416931	30.4019546508789\\
3.34500002861023	30.5322341918945\\
3.34999990463257	30.5107345581055\\
3.35500001907349	30.322359085083\\
3.35999989509583	29.9653186798096\\
3.36500000953674	29.4362144470215\\
3.36999988555908	28.7489242553711\\
3.375	27.921501159668\\
3.38000011444092	26.9796390533447\\
3.38499999046326	25.9513034820557\\
3.39000010490417	24.8595771789551\\
3.39499998092651	23.7269344329834\\
3.40000009536743	22.5685577392578\\
3.40499997138977	21.3892917633057\\
3.41000008583069	20.1958465576172\\
3.41499996185303	18.9898624420166\\
3.42000007629395	17.7740440368652\\
3.42499995231628	16.5517654418945\\
3.4300000667572	15.3354034423828\\
3.43499994277954	14.140172958374\\
3.44000005722046	12.9899959564209\\
3.4449999332428	11.9008474349976\\
3.45000004768372	10.8953218460083\\
3.45499992370605	9.99957847595215\\
3.46000003814697	9.32765865325928\\
3.46499991416931	8.80926609039307\\
3.47000002861023	8.33246326446533\\
3.47499990463257	7.88839244842529\\
3.48000001907349	7.45830297470093\\
3.48499989509583	7.04264354705811\\
3.49000000953674	6.59509134292603\\
3.49499988555908	6.19990301132202\\
3.5	5.90049314498901\\
3.50500011444092	5.64960670471191\\
3.50999999046326	5.45447731018066\\
3.51500010490417	5.3629994392395\\
3.51999998092651	5.31237554550171\\
3.52500009536743	5.37071228027344\\
3.52999997138977	5.47707414627075\\
3.53500008583069	5.654456615448\\
3.53999996185303	5.92503070831299\\
3.54500007629395	6.34403419494629\\
3.54999995231628	7.10956382751465\\
3.5550000667572	8.28886699676514\\
3.55999994277954	9.88178730010986\\
3.56500005722046	11.886908531189\\
3.5699999332428	14.245078086853\\
3.57500004768372	16.8275756835938\\
3.57999992370605	19.4795799255371\\
3.58500003814697	22.0230522155762\\
3.58999991416931	24.314323425293\\
3.59500002861023	26.2389526367188\\
3.59999990463257	27.7452354431152\\
3.60500001907349	28.809741973877\\
3.60999989509583	29.6008243560791\\
3.61500000953674	30.1587390899658\\
3.61999988555908	30.3460655212402\\
3.625	30.284236907959\\
3.63000011444092	30.1005268096924\\
3.63499999046326	29.9464626312256\\
3.64000010490417	29.9183292388916\\
3.64499998092651	30.0565395355225\\
3.65000009536743	30.4460678100586\\
3.65499997138977	30.8085784912109\\
3.66000008583069	30.9923496246338\\
3.66499996185303	30.9663314819336\\
3.67000007629395	30.665979385376\\
3.67499995231628	29.943208694458\\
3.6800000667572	28.7883014678955\\
3.68499994277954	27.254467010498\\
3.69000005722046	25.45578956604\\
3.6949999332428	23.5269584655762\\
3.70000004768372	21.6395206451416\\
3.70499992370605	19.8898162841797\\
3.71000003814697	18.3984909057617\\
3.71499991416931	17.1522045135498\\
3.72000002861023	16.1257457733154\\
3.72499990463257	15.2449932098389\\
3.73000001907349	14.4202899932861\\
3.73499989509583	13.573504447937\\
3.74000000953674	12.6534337997437\\
3.74499988555908	11.6521024703979\\
3.75	10.7547206878662\\
3.75500011444092	9.98768329620361\\
3.75999999046326	9.34440803527832\\
3.76500010490417	8.69696521759033\\
3.76999998092651	8.01664161682129\\
3.77500009536743	7.37617874145508\\
3.77999997138977	6.80278205871582\\
3.78500008583069	6.30861616134644\\
3.78999996185303	5.99357128143311\\
3.79500007629395	5.86622381210327\\
3.79999995231628	5.82616996765137\\
3.8050000667572	5.83536195755005\\
3.80999994277954	5.83715057373047\\
3.81500005722046	5.79465055465698\\
3.8199999332428	5.68407154083252\\
3.82500004768372	5.5344614982605\\
3.82999992370605	5.43974733352661\\
3.83500003814697	5.54732990264893\\
3.83999991416931	6.19071626663208\\
3.84500002861023	7.65631198883057\\
3.84999990463257	9.73860645294189\\
3.85500001907349	12.3811550140381\\
3.85999989509583	15.4352321624756\\
3.86500000953674	18.658540725708\\
3.86999988555908	21.8038272857666\\
3.875	24.6501750946045\\
3.88000011444092	27.0349731445313\\
3.88499999046326	28.87646484375\\
3.89000010490417	30.1575469970703\\
3.89499998092651	30.9771461486816\\
3.90000009536743	31.65380859375\\
3.90499997138977	31.842981338501\\
3.91000008583069	31.7513122558594\\
3.91499996185303	31.5755081176758\\
3.92000007629395	31.4741916656494\\
3.92499995231628	31.5812721252441\\
3.9300000667572	32.0590705871582\\
3.93499994277954	32.7052955627441\\
3.94000005722046	33.2281646728516\\
3.9449999332428	33.466251373291\\
3.95000004768372	33.3602638244629\\
3.95499992370605	33.000301361084\\
3.96000003814697	32.1064910888672\\
3.96499991416931	30.6660480499268\\
3.97000002861023	28.7583713531494\\
3.97499990463257	26.5265998840332\\
3.98000001907349	24.1562480926514\\
3.98499989509583	21.8335132598877\\
3.99000000953674	19.7067012786865\\
3.99499988555908	17.871072769165\\
4	16.3532524108887\\
4.00500011444092	15.1060600280762\\
4.01000022888184	14.0412397384644\\
4.0149998664856	13.0543766021729\\
4.01999998092651	12.047025680542\\
4.02500009536743	10.985689163208\\
4.03000020980835	9.97193717956543\\
4.03499984741211	9.13095188140869\\
4.03999996185303	8.36615562438965\\
4.04500007629395	7.61839818954468\\
4.05000019073486	6.86050415039063\\
4.05499982833862	6.09775114059448\\
4.05999994277954	5.35949611663818\\
4.06500005722046	4.75402927398682\\
4.07000017166138	4.32772541046143\\
4.07499980926514	4.13888502120972\\
4.07999992370605	4.08629274368286\\
4.08500003814697	4.14297485351563\\
4.09000015258789	4.20554876327515\\
4.09499979019165	4.11825370788574\\
4.09999990463257	3.75081562995911\\
4.10500001907349	3.11259889602661\\
4.1100001335144	2.37259793281555\\
4.11499977111816	2.2303352355957\\
4.11999988555908	2.59197854995728\\
4.125	4.98992490768433\\
4.13000011444092	9.52236557006836\\
4.13500022888184	14.8601531982422\\
4.1399998664856	20.6122779846191\\
4.14499998092651	26.103178024292\\
4.15000009536743	30.794132232666\\
4.15500020980835	34.3256034851074\\
4.15999984741211	36.5472412109375\\
4.16499996185303	37.5167999267578\\
4.17000007629395	38.2692108154297\\
4.17500019073486	37.6581039428711\\
4.17999982833862	35.9582023620605\\
4.18499994277954	33.6632118225098\\
4.19000005722046	31.4171371459961\\
4.19500017166138	29.8131008148193\\
4.19999980926514	29.3227100372314\\
4.20499992370605	30.729175567627\\
4.21000003814697	32.7000579833984\\
4.21500015258789	34.5461349487305\\
4.21999979019165	35.7628288269043\\
4.22499990463257	35.9888381958008\\
4.23000001907349	35.4546127319336\\
4.2350001335144	34.1498336791992\\
4.23999977111816	31.693962097168\\
4.24499988555908	28.1872425079346\\
4.25	24.3089809417725\\
4.25500011444092	20.3402786254883\\
4.26000022888184	16.8157501220703\\
4.2649998664856	14.0959424972534\\
4.26999998092651	12.3392419815063\\
4.27500009536743	11.527738571167\\
4.28000020980835	11.65260887146\\
4.28499984741211	11.7577981948853\\
4.28999996185303	11.5858459472656\\
4.29500007629395	11.0670137405396\\
4.30000019073486	10.3946294784546\\
4.30499982833862	9.51755428314209\\
4.30999994277954	8.15493297576904\\
4.31500005722046	6.60663414001465\\
4.32000017166138	5.2276463508606\\
4.32499980926514	4.21278285980225\\
4.32999992370605	3.67450976371765\\
4.33500003814697	3.66338014602661\\
4.34000015258789	3.97275519371033\\
4.34499979019165	4.44355583190918\\
4.34999990463257	4.81923866271973\\
4.35500001907349	4.80851697921753\\
4.3600001335144	4.17386293411255\\
4.36499977111816	2.89116430282593\\
4.36999988555908	1.89041125774384\\
4.375	2.38042688369751\\
4.38000011444092	2.78657341003418\\
4.38500022888184	3.04948282241821\\
4.3899998664856	3.16540884971619\\
4.39499998092651	10.7676610946655\\
4.40000009536743	18.5559616088867\\
4.40500020980835	26.1289901733398\\
4.40999984741211	32.6264572143555\\
4.41499996185303	37.4496231079102\\
4.42000007629395	40.403621673584\\
4.42500019073486	41.5890808105469\\
4.42999982833862	42.3087120056152\\
4.43499994277954	41.4402008056641\\
4.44000005722046	38.9983673095703\\
4.44500017166138	35.6510581970215\\
4.44999980926514	32.2997436523438\\
4.45499992370605	29.8474407196045\\
4.46000003814697	28.9678401947021\\
4.46500015258789	30.7689361572266\\
4.46999979019165	33.582145690918\\
4.47499990463257	36.2028350830078\\
4.48000001907349	38.0258560180664\\
4.4850001335144	38.5419654846191\\
4.48999977111816	37.8205108642578\\
4.49499988555908	36.4141998291016\\
4.5	33.4589691162109\\
4.50500011444092	29.2567691802979\\
4.51000022888184	24.3642444610596\\
4.5149998664856	19.4333572387695\\
4.51999998092651	15.0735893249512\\
4.52500009536743	11.8080291748047\\
4.53000020980835	9.7901725769043\\
4.53499984741211	9.04141139984131\\
4.53999996185303	9.534010887146\\
4.54500007629395	9.93964958190918\\
4.55000019073486	10.0262994766235\\
4.55499982833862	9.71344184875488\\
4.55999994277954	9.16488647460938\\
4.56500005722046	8.10668849945068\\
4.57000017166138	6.59055995941162\\
4.57499980926514	4.98768758773804\\
4.57999992370605	3.58964323997498\\
4.58500003814697	2.52997159957886\\
4.59000015258789	1.96985983848572\\
4.59499979019165	1.9956271648407\\
4.59999990463257	2.48849630355835\\
4.60500001907349	3.28801894187927\\
4.6100001335144	4.02410697937012\\
4.61499977111816	4.10940790176392\\
4.61999988555908	3.07068586349487\\
4.625	1.67162334918976\\
4.63000011444092	2.84069156646729\\
4.63500022888184	3.70859956741333\\
4.6399998664856	4.18614196777344\\
4.64499998092651	4.32810878753662\\
4.65000009536743	4.31521129608154\\
4.65500020980835	4.20156717300415\\
4.65999984741211	3.92763543128967\\
4.66499996185303	27.9817485809326\\
4.67000007629395	39.9588279724121\\
4.67500019073486	47.9723281860352\\
4.67999982833862	52.0194854736328\\
4.68499994277954	52.6519393920898\\
4.69000005722046	52.9814682006836\\
4.69500017166138	49.6381492614746\\
4.69999980926514	42.911018371582\\
4.70499992370605	34.0992851257324\\
4.71000003814697	25.3439655303955\\
4.71500015258789	18.9277820587158\\
4.71999979019165	16.4444618225098\\
4.72499990463257	20.8625888824463\\
4.73000001907349	27.2986259460449\\
4.7350001335144	33.7136840820313\\
4.73999977111816	38.415225982666\\
4.74499988555908	40.4339714050293\\
4.75	39.474910736084\\
4.75500011444092	37.6214447021484\\
4.76000022888184	33.0207214355469\\
4.7649998664856	26.1880912780762\\
4.76999998092651	18.192066192627\\
4.77500009536743	10.405047416687\\
4.78000020980835	4.0876522064209\\
4.78499984741211	1.02275002002716\\
4.78999996185303	0.867816507816315\\
4.79500007629395	0.672055423259735\\
4.80000019073486	5.06251096725464\\
4.80499982833862	7.90090799331665\\
4.80999994277954	8.63800525665283\\
4.81500005722046	7.07261896133423\\
4.82000017166138	4.91642761230469\\
4.82499980926514	3.22057247161865\\
4.82999992370605	2.57514977455139\\
4.83500003814697	3.03254532814026\\
4.84000015258789	4.39078617095947\\
4.84499979019165	6.1278281211853\\
4.84999990463257	7.42714786529541\\
4.85500001907349	7.13612270355225\\
4.8600001335144	4.38094472885132\\
4.86499977111816	0.36887726187706\\
4.86999988555908	0.917131900787354\\
4.875	2.02847695350647\\
4.88000011444092	3.17662954330444\\
4.88500022888184	3.99529933929443\\
4.8899998664856	4.42655277252197\\
4.89499998092651	4.51130199432373\\
4.90000009536743	4.51835632324219\\
4.90500020980835	4.35961103439331\\
4.90999984741211	4.04627656936646\\
4.91499996185303	3.61389183998108\\
4.92000007629395	27.7560844421387\\
4.92500019073486	42.4667587280273\\
4.92999982833862	50.3933067321777\\
4.93499994277954	52.7861022949219\\
4.94000005722046	51.5713272094727\\
4.94500017166138	50.2196426391602\\
4.94999980926514	45.274600982666\\
4.95499992370605	38.0064277648926\\
4.96000003814697	30.0948333740234\\
4.96500015258789	23.4039764404297\\
4.96999979019165	19.4124164581299\\
4.97499990463257	19.7212562561035\\
4.98000001907349	23.5004234313965\\
4.9850001335144	27.8123779296875\\
4.98999977111816	31.5497169494629\\
4.99499988555908	33.956600189209\\
5	34.7680244445801\\
5.00500011444092	33.9990196228027\\
5.01000022888184	32.4358444213867\\
5.0149998664856	30.1777935028076\\
5.01999998092651	26.9659957885742\\
5.02500009536743	23.168815612793\\
5.03000020980835	19.1963405609131\\
5.03499984741211	15.3938302993774\\
5.03999996185303	12.0354537963867\\
5.04500007629395	9.27918529510498\\
5.05000019073486	7.20251750946045\\
5.05499982833862	5.7910475730896\\
5.05999994277954	4.95076942443848\\
5.06500005722046	4.52286005020142\\
5.07000017166138	4.31958818435669\\
5.07499980926514	4.15390396118164\\
5.07999992370605	3.86151051521301\\
5.08500003814697	3.38193845748901\\
5.09000015258789	2.60770511627197\\
5.09499979019165	1.53677725791931\\
5.09999990463257	0.274307608604431\\
5.10500001907349	-0.0331083312630653\\
5.1100001335144	-0.0303391125053167\\
5.11499977111816	-0.0263062529265881\\
5.11999988555908	-0.0216858852654696\\
5.125	-0.0169122088700533\\
5.13000011444092	-0.0127697288990021\\
5.13500022888184	0.970218598842621\\
5.1399998664856	1.82036638259888\\
5.14499998092651	2.30498719215393\\
5.15000009536743	2.51955437660217\\
5.15500020980835	2.53288793563843\\
5.15999984741211	13.1416130065918\\
5.16499996185303	20.4856510162354\\
5.17000007629395	26.0622901916504\\
5.17500019073486	29.6251773834229\\
5.17999982833862	31.2480545043945\\
5.18499994277954	31.3448352813721\\
5.19000005722046	30.5975151062012\\
5.19500017166138	29.4223098754883\\
5.19999980926514	27.4558506011963\\
5.20499992370605	25.1642398834229\\
5.21000003814697	22.9820117950439\\
5.21500015258789	21.2467231750488\\
5.21999979019165	20.1497840881348\\
5.22499990463257	19.9067478179932\\
5.23000001907349	20.3432292938232\\
5.2350001335144	20.9492359161377\\
5.23999977111816	21.5462493896484\\
5.24499988555908	22.0164661407471\\
5.25	22.2722072601318\\
5.25500011444092	22.2769756317139\\
5.26000022888184	22.2673282623291\\
5.2649998664856	21.9854335784912\\
5.26999998092651	21.3715591430664\\
5.27500009536743	20.4299087524414\\
5.28000020980835	19.2044429779053\\
5.28499984741211	17.7857532501221\\
5.28999996185303	16.2964458465576\\
5.29500007629395	14.8780374526978\\
5.30000019073486	13.6950616836548\\
5.30499982833862	12.870005607605\\
5.30999994277954	12.5334606170654\\
5.31500005722046	12.5963878631592\\
5.32000017166138	13.5024003982544\\
5.32499980926514	14.4992809295654\\
5.32999992370605	15.4287567138672\\
5.33500003814697	16.1250877380371\\
5.34000015258789	16.4767017364502\\
5.34499979019165	16.6365928649902\\
5.34999990463257	16.2553157806396\\
5.35500001907349	15.2929964065552\\
5.3600001335144	13.8165245056152\\
5.36499977111816	12.0034618377686\\
5.36999988555908	10.0620059967041\\
5.375	8.34380149841309\\
5.38000011444092	6.99687004089355\\
5.38500022888184	6.1983814239502\\
5.3899998664856	6.00079917907715\\
5.39499998092651	6.40060520172119\\
5.40000009536743	6.69044351577759\\
5.40500020980835	6.62217283248901\\
5.40999984741211	6.36655378341675\\
5.41499996185303	5.53895902633667\\
5.42000007629395	4.07068967819214\\
5.42500019073486	2.05052733421326\\
5.42999982833862	0.252480328083038\\
5.43499994277954	0.201652303338051\\
5.44000005722046	0.149041846394539\\
5.44500017166138	0.120646946132183\\
5.44999980926514	0.114102683961391\\
5.45499992370605	0.10907106846571\\
5.46000003814697	0.105223141610622\\
5.46500015258789	0.102389559149742\\
5.46999979019165	0.099647156894207\\
5.47499990463257	0.0988680794835091\\
5.48000001907349	0.0949338227510452\\
5.4850001335144	0.0950156152248383\\
5.48999977111816	0.0929016917943954\\
5.49499988555908	0.0921168476343155\\
5.5	0.0908519104123116\\
5.50500011444092	0.0904062688350677\\
5.51000022888184	0.0905360206961632\\
5.5149998664856	0.0893425643444061\\
5.51999998092651	0.151441261172295\\
5.52500009536743	0.335023611783981\\
5.53000020980835	0.438723802566528\\
5.53499984741211	0.528062403202057\\
5.53999996185303	0.600764155387878\\
5.54500007629395	0.658122956752777\\
5.55000019073486	0.700239777565002\\
5.55499982833862	0.729918777942657\\
5.55999994277954	0.753552794456482\\
5.56500005722046	0.768722116947174\\
5.57000017166138	0.771436214447021\\
5.57499980926514	0.7649205327034\\
5.57999992370605	0.746456801891327\\
5.58500003814697	0.720195412635803\\
5.59000015258789	0.683425843715668\\
5.59499979019165	0.646574676036835\\
5.59999990463257	0.602410912513733\\
5.60500001907349	0.559636116027832\\
5.6100001335144	0.522733807563782\\
5.61499977111816	0.489389687776566\\
5.61999988555908	0.460096955299377\\
5.625	0.439359992742538\\
5.63000011444092	0.426081210374832\\
5.63500022888184	0.412046849727631\\
5.6399998664856	0.40771210193634\\
5.64499998092651	0.40624526143074\\
5.65000009536743	0.404474020004272\\
5.65500020980835	0.40319412946701\\
5.65999984741211	0.401808381080627\\
5.66499996185303	0.401206851005554\\
5.67000007629395	0.885265469551086\\
5.67500019073486	11.4442567825317\\
5.67999982833862	13.8546323776245\\
5.68499994277954	13.9838008880615\\
5.69000005722046	13.3761596679688\\
5.69500017166138	12.227502822876\\
5.69999980926514	10.5691699981689\\
5.70499992370605	9.00704479217529\\
5.71000003814697	7.96562576293945\\
5.71500015258789	7.84335088729858\\
5.71999979019165	8.41158866882324\\
5.72499990463257	9.05075550079346\\
5.73000001907349	9.48386669158936\\
5.7350001335144	9.58194255828857\\
5.73999977111816	9.32773971557617\\
5.74499988555908	8.76917362213135\\
5.75	8.01419448852539\\
5.75500011444092	7.18797016143799\\
5.76000022888184	6.37321710586548\\
5.7649998664856	5.59070253372192\\
5.76999998092651	4.76152229309082\\
5.77500009536743	3.84696793556213\\
5.78000020980835	2.72351551055908\\
5.78499984741211	1.35421764850616\\
5.78999996185303	-0.272826820611954\\
5.79500007629395	-2.17766094207764\\
5.80000019073486	-4.3294472694397\\
5.80499982833862	-6.65338516235352\\
5.80999994277954	-9.08022975921631\\
5.81500005722046	-11.5023384094238\\
5.82000017166138	-13.786319732666\\
5.82499980926514	-15.7687530517578\\
5.82999992370605	-17.2999668121338\\
5.83500003814697	-18.1972541809082\\
5.84000015258789	-18.297492980957\\
5.84499979019165	-17.4735794067383\\
5.84999990463257	-15.6822690963745\\
5.85500001907349	-13.0835332870483\\
5.8600001335144	-9.52003002166748\\
5.86499977111816	-5.33265733718872\\
5.86999988555908	-1.01810646057129\\
5.875	2.58010005950928\\
5.88000011444092	4.57823085784912\\
5.88500022888184	4.27138710021973\\
5.8899998664856	2.09094738960266\\
5.89499998092651	2.09319686889648\\
5.90000009536743	2.16262125968933\\
5.90500020980835	2.32292747497559\\
5.90999984741211	2.6876654624939\\
5.91499996185303	3.26720833778381\\
5.92000007629395	3.91317772865295\\
5.92500019073486	4.47273874282837\\
5.92999982833862	4.86308431625366\\
5.93499994277954	5.04685544967651\\
5.94000005722046	5.05933713912964\\
5.94500017166138	4.9679102897644\\
5.94999980926514	4.85073089599609\\
5.95499992370605	4.79349517822266\\
5.96000003814697	4.92832040786743\\
5.96500015258789	5.23717498779297\\
5.96999979019165	5.68428659439087\\
5.97499990463257	6.22804880142212\\
5.98000001907349	50.0682334899902\\
5.9850001335144	79.9624404907227\\
5.98999977111816	95.9791564941406\\
5.99499988555908	100.498046875\\
6	98.8110504150391\\
6.00500011444092	96.4043426513672\\
6.01000022888184	93.1524505615234\\
6.0149998664856	85.6021881103516\\
6.01999998092651	73.7759399414063\\
6.02500009536743	59.2618217468262\\
6.03000020980835	44.751049041748\\
6.03499984741211	33.3226013183594\\
6.03999996185303	27.8268222808838\\
6.04500007629395	32.2749710083008\\
6.05000019073486	41.9914360046387\\
6.05499982833862	52.5744247436523\\
6.05999994277954	61.7754249572754\\
6.06500005722046	68.0853424072266\\
6.07000017166138	71.0745620727539\\
6.07499980926514	70.8101959228516\\
6.07999992370605	68.5271530151367\\
6.08500003814697	65.6849594116211\\
6.09000015258789	60.5261039733887\\
6.09499979019165	53.5714530944824\\
6.09999990463257	45.5162963867188\\
6.10500001907349	37.2174682617188\\
6.1100001335144	29.7692070007324\\
6.11499977111816	23.8858470916748\\
6.11999988555908	20.144079208374\\
6.125	18.630033493042\\
6.13000011444092	19.9405765533447\\
6.13500022888184	21.8382129669189\\
6.1399998664856	23.379451751709\\
6.14499998092651	24.0639247894287\\
6.15000009536743	23.641845703125\\
6.15500020980835	22.5981388092041\\
6.15999984741211	20.7439708709717\\
6.16499996185303	17.8094081878662\\
6.17000007629395	14.0004615783691\\
6.17500019073486	9.65493869781494\\
6.17999982833862	5.19192790985107\\
6.18499994277954	1.06674683094025\\
6.19000005722046	0.440609216690063\\
6.19500017166138	0.386078298091888\\
6.19999980926514	0.297625482082367\\
6.20499992370605	0.184639945626259\\
6.21000003814697	0.0590059794485569\\
6.21500015258789	-0.0050798961892724\\
6.21999979019165	-0.0110730268061161\\
6.22499990463257	-0.0147657878696918\\
6.23000001907349	-0.0167017355561256\\
6.2350001335144	-0.0174275301396847\\
6.23999977111816	-0.0173142850399017\\
6.24499988555908	-0.0167580712586641\\
6.25	-0.0160193145275116\\
6.25500011444092	-0.0150645533576608\\
6.26000022888184	-0.0140823740512133\\
6.2649998664856	-0.0129890646785498\\
6.26999998092651	-0.0118947252631187\\
6.27500009536743	-0.0108957551419735\\
6.28000020980835	-0.00996336061507463\\
6.28499984741211	-0.00906001962721348\\
6.28999996185303	-0.00816772226244211\\
6.29500007629395	-0.00733370799571276\\
6.30000019073486	-0.00660984171554446\\
6.30499982833862	-0.00597657728940248\\
6.30999994277954	-0.00537917157635093\\
6.31500005722046	-0.0048343138769269\\
6.32000017166138	-0.00437704892829061\\
6.32499980926514	-0.00399879273027182\\
6.32999992370605	-0.00366425467655063\\
6.33500003814697	-0.00329068140126765\\
6.34000015258789	-0.00284017389640212\\
6.34499979019165	-0.0024455594830215\\
6.34999990463257	-0.00211632461287081\\
6.35500001907349	-0.00188294902909547\\
6.3600001335144	5.09061431884766\\
6.36499977111816	7.21974897384644\\
6.36999988555908	8.59541034698486\\
6.375	9.26888370513916\\
6.38000011444092	9.32822513580322\\
6.38500022888184	9.1403341293335\\
6.3899998664856	9.33874702453613\\
6.39499998092651	8.93625450134277\\
6.40000009536743	8.62845039367676\\
6.40500020980835	8.87300109863281\\
6.40999984741211	10.212384223938\\
6.41499996185303	12.1271076202393\\
6.42000007629395	14.4754638671875\\
6.42500019073486	17.0397567749023\\
6.42999982833862	19.5879173278809\\
6.43499994277954	21.9646968841553\\
6.44000005722046	24.027717590332\\
6.44500017166138	25.7205638885498\\
6.44999980926514	27.0165424346924\\
6.45499992370605	27.9725360870361\\
6.46000003814697	28.6326694488525\\
6.46500015258789	29.0711479187012\\
6.46999979019165	29.3810176849365\\
6.47499990463257	29.6259822845459\\
6.48000001907349	29.8620338439941\\
6.4850001335144	30.1412487030029\\
6.48999977111816	30.4829406738281\\
6.49499988555908	30.8872203826904\\
6.5	31.3401145935059\\
6.50500011444092	31.815034866333\\
6.51000022888184	32.2738418579102\\
6.5149998664856	32.6840629577637\\
6.51999998092651	33.0153846740723\\
6.52500009536743	33.242805480957\\
6.53000020980835	33.3446426391602\\
6.53499984741211	33.3372039794922\\
6.53999996185303	33.2135581970215\\
6.54500007629395	32.9839096069336\\
6.55000019073486	32.6651649475098\\
6.55499982833862	32.2842292785645\\
6.55999994277954	31.8716106414795\\
6.56500005722046	31.4071750640869\\
6.57000017166138	30.8933887481689\\
6.57499980926514	30.337474822998\\
6.57999992370605	29.7838802337646\\
6.58500003814697	29.2126064300537\\
6.59000015258789	28.6366367340088\\
6.59499979019165	28.0635566711426\\
6.59999990463257	27.48561668396\\
6.60500001907349	26.8997383117676\\
6.6100001335144	26.3087940216064\\
6.61499977111816	25.7158508300781\\
6.61999988555908	25.1023750305176\\
6.625	24.4835014343262\\
6.63000011444092	23.8535690307617\\
6.63500022888184	23.2173385620117\\
6.6399998664856	22.5903606414795\\
6.64499998092651	21.9616794586182\\
6.65000009536743	21.3443508148193\\
6.65500020980835	20.7462139129639\\
6.65999984741211	20.1621341705322\\
6.66499996185303	19.6024799346924\\
6.67000007629395	19.0847434997559\\
6.67500019073486	18.5970916748047\\
6.67999982833862	18.1575660705566\\
6.68499994277954	17.764461517334\\
6.69000005722046	17.4064407348633\\
6.69500017166138	17.1217727661133\\
6.69999980926514	16.9233112335205\\
6.70499992370605	16.7730274200439\\
6.71000003814697	16.6697978973389\\
6.71500015258789	16.5996952056885\\
6.71999979019165	16.5695495605469\\
6.72499990463257	16.5776691436768\\
6.73000001907349	16.6118793487549\\
6.7350001335144	16.6777210235596\\
6.73999977111816	16.7750015258789\\
6.74499988555908	16.9003467559814\\
6.75	17.0609378814697\\
6.75500011444092	17.247386932373\\
6.76000022888184	17.4405956268311\\
6.7649998664856	17.6766471862793\\
6.76999998092651	17.9319915771484\\
6.77500009536743	18.2159290313721\\
6.78000020980835	18.5268287658691\\
6.78499984741211	18.8559226989746\\
6.78999996185303	19.2055625915527\\
6.79500007629395	19.5704898834229\\
6.80000019073486	19.9452991485596\\
6.80499982833862	20.3261470794678\\
6.80999994277954	20.7153778076172\\
6.81500005722046	21.0997428894043\\
6.82000017166138	21.4839954376221\\
6.82499980926514	21.862585067749\\
6.82999992370605	22.2313709259033\\
6.83500003814697	22.5868682861328\\
6.84000015258789	22.9260292053223\\
6.84499979019165	23.2570419311523\\
6.84999990463257	23.5712642669678\\
6.85500001907349	23.8670520782471\\
6.8600001335144	24.1428604125977\\
6.86499977111816	24.3873023986816\\
6.86999988555908	24.6028823852539\\
6.875	24.7849063873291\\
6.88000011444092	24.9230079650879\\
6.88500022888184	25.0127735137939\\
6.8899998664856	25.0472297668457\\
6.89499998092651	25.0218467712402\\
6.90000009536743	24.9346103668213\\
6.90500020980835	24.78542137146\\
6.90999984741211	24.5740356445313\\
6.91499996185303	24.305103302002\\
6.92000007629395	23.9837818145752\\
6.92500019073486	23.6159801483154\\
6.92999982833862	23.2091083526611\\
6.93499994277954	22.7708206176758\\
6.94000005722046	22.3066596984863\\
6.94500017166138	21.8205718994141\\
6.94999980926514	21.3155689239502\\
6.95499992370605	20.7917003631592\\
6.96000003814697	20.2520427703857\\
6.96500015258789	19.6980876922607\\
6.96999979019165	19.1269454956055\\
6.97499990463257	18.5349178314209\\
6.98000001907349	17.9269599914551\\
6.9850001335144	17.2998638153076\\
6.98999977111816	16.6545677185059\\
6.99499988555908	15.9962368011475\\
7	15.3294277191162\\
7.00500011444092	14.6582775115967\\
7.01000022888184	13.9885034561157\\
7.0149998664856	13.3268804550171\\
7.01999998092651	12.6785345077515\\
7.02500009536743	12.0466985702515\\
7.03000020980835	11.4341268539429\\
7.03499984741211	10.8432998657227\\
7.03999996185303	10.27587890625\\
7.04500007629395	9.73263263702393\\
7.05000019073486	9.21360397338867\\
7.05499982833862	8.7181396484375\\
7.05999994277954	8.24498462677002\\
7.06500005722046	7.79331398010254\\
7.07000017166138	7.36239242553711\\
7.07499980926514	6.9520959854126\\
7.07999992370605	6.56279754638672\\
7.08500003814697	6.1956148147583\\
7.09000015258789	5.85238075256348\\
7.09499979019165	5.53474521636963\\
7.09999990463257	5.24481868743896\\
7.10500001907349	4.98460054397583\\
7.1100001335144	4.75594282150269\\
7.11499977111816	4.55954456329346\\
7.11999988555908	4.39666986465454\\
7.125	4.26787281036377\\
7.13000011444092	4.16923093795776\\
7.13500022888184	4.09993600845337\\
7.1399998664856	4.05867195129395\\
7.14499998092651	4.04170751571655\\
7.15000009536743	4.04910850524902\\
7.15500020980835	4.08699893951416\\
7.15999984741211	4.14058494567871\\
7.16499996185303	4.20240449905396\\
7.17000007629395	4.28377342224121\\
7.17500019073486	4.38162899017334\\
7.17999982833862	4.496018409729\\
7.18499994277954	4.62868499755859\\
7.19000005722046	4.77840614318848\\
7.19500017166138	4.94496154785156\\
7.19999980926514	5.12739276885986\\
7.20499992370605	5.32210350036621\\
7.21000003814697	5.52958345413208\\
7.21500015258789	5.74662399291992\\
7.21999979019165	5.96827793121338\\
7.22499990463257	6.19629573822021\\
7.23000001907349	6.42803239822388\\
7.2350001335144	6.66183233261108\\
7.23999977111816	6.89741563796997\\
7.24499988555908	7.1325740814209\\
7.25	7.36621904373169\\
7.25500011444092	7.59877681732178\\
7.26000022888184	7.82752704620361\\
7.2649998664856	8.05286026000977\\
7.26999998092651	8.27532291412354\\
7.27500009536743	8.49230003356934\\
7.28000020980835	8.70232772827148\\
7.28499984741211	8.90554046630859\\
7.28999996185303	9.10054492950439\\
7.29500007629395	9.28769588470459\\
7.30000019073486	9.46335792541504\\
7.30499982833862	9.62868404388428\\
7.30999994277954	9.7825756072998\\
7.31500005722046	9.92526054382324\\
7.32000017166138	10.0551280975342\\
7.32499980926514	10.1726140975952\\
7.32999992370605	10.2763643264771\\
7.33500003814697	10.3661956787109\\
7.34000015258789	10.4423179626465\\
7.34499979019165	10.5044097900391\\
7.34999990463257	10.552640914917\\
7.35500001907349	10.5871458053589\\
7.3600001335144	10.6091260910034\\
7.36499977111816	10.618821144104\\
7.36999988555908	10.6169633865356\\
7.375	10.6044406890869\\
7.38000011444092	10.5817775726318\\
7.38500022888184	10.5461301803589\\
7.3899998664856	10.4957571029663\\
7.39499998092651	10.4313411712646\\
7.40000009536743	10.3564052581787\\
7.40500020980835	10.2701072692871\\
7.40999984741211	10.1735486984253\\
7.41499996185303	10.0674180984497\\
7.42000007629395	9.95263385772705\\
7.42500019073486	9.83026790618896\\
7.42999982833862	9.70095634460449\\
7.43499994277954	9.56491470336914\\
7.44000005722046	9.42297172546387\\
7.44500017166138	9.27606010437012\\
7.44999980926514	9.12484455108643\\
7.45499992370605	8.97055053710938\\
7.46000003814697	8.81449604034424\\
7.46500015258789	8.65752029418945\\
7.46999979019165	8.49986934661865\\
7.47499990463257	8.34240913391113\\
7.48000001907349	8.18577194213867\\
7.4850001335144	8.03046131134033\\
7.48999977111816	7.87706613540649\\
7.49499988555908	7.72605752944946\\
7.5	7.57777070999146\\
7.50500011444092	7.43243598937988\\
7.51000022888184	7.29053068161011\\
7.5149998664856	7.1532998085022\\
7.51999998092651	7.02150058746338\\
7.52500009536743	6.89741945266724\\
7.53000020980835	6.77727842330933\\
7.53499984741211	6.66519641876221\\
7.53999996185303	6.5594482421875\\
7.54500007629395	6.46214866638184\\
7.55000019073486	6.37260103225708\\
7.55499982833862	6.29069423675537\\
7.55999994277954	6.21683597564697\\
7.56500005722046	6.15090370178223\\
7.57000017166138	6.09254932403564\\
7.57499980926514	6.04181432723999\\
7.57999992370605	5.99867820739746\\
7.58500003814697	5.96305561065674\\
7.59000015258789	5.93507766723633\\
7.59499979019165	5.91471338272095\\
7.59999990463257	5.9017915725708\\
7.60500001907349	5.89655017852783\\
7.6100001335144	5.90041971206665\\
7.61499977111816	5.91490793228149\\
7.61999988555908	5.93726062774658\\
7.625	5.96422958374023\\
7.63000011444092	5.99670648574829\\
7.63500022888184	6.03501272201538\\
7.6399998664856	6.07889699935913\\
7.64499998092651	6.12831163406372\\
7.65000009536743	6.18206644058228\\
7.65500020980835	6.239821434021\\
7.65999984741211	6.30147218704224\\
7.66499996185303	6.36652755737305\\
7.67000007629395	6.43447971343994\\
7.67500019073486	6.50515270233154\\
7.67999982833862	6.57859516143799\\
7.68499994277954	6.65452337265015\\
7.69000005722046	6.73270559310913\\
7.69500017166138	6.81206798553467\\
7.69999980926514	6.89224004745483\\
7.70499992370605	6.97307348251343\\
7.71000003814697	7.05448293685913\\
7.71500015258789	7.13704347610474\\
7.71999979019165	7.2195405960083\\
7.72499990463257	7.30064153671265\\
7.73000001907349	7.38020944595337\\
7.7350001335144	7.45776748657227\\
7.73999977111816	7.53444385528564\\
7.74499988555908	7.60935974121094\\
7.75	7.68080806732178\\
7.75500011444092	7.74968194961548\\
7.76000022888184	7.81632423400879\\
7.7649998664856	7.87987565994263\\
7.76999998092651	7.94009447097778\\
7.77500009536743	7.99714851379395\\
7.78000020980835	8.05100440979004\\
7.78499984741211	8.10138988494873\\
7.78999996185303	8.14847278594971\\
7.79500007629395	8.19225311279297\\
7.80000019073486	8.23235607147217\\
7.80499982833862	8.26892566680908\\
7.80999994277954	8.30199432373047\\
7.81500005722046	8.33139419555664\\
7.82000017166138	8.35710620880127\\
7.82499980926514	8.37919235229492\\
7.82999992370605	8.39779949188232\\
7.83500003814697	8.4130973815918\\
7.84000015258789	8.42495059967041\\
7.84499979019165	8.43343925476074\\
7.84999990463257	8.43897533416748\\
7.85500001907349	8.44148349761963\\
7.8600001335144	8.4410572052002\\
7.86499977111816	8.43795585632324\\
7.86999988555908	8.43224334716797\\
7.875	8.42404460906982\\
7.88000011444092	8.41388130187988\\
7.88500022888184	8.40200042724609\\
7.8899998664856	8.38855266571045\\
7.89499998092651	8.37360286712646\\
7.90000009536743	8.35718154907227\\
7.90500020980835	8.3395471572876\\
7.90999984741211	8.32079696655273\\
7.91499996185303	8.30088996887207\\
7.92000007629395	8.28002071380615\\
7.92500019073486	8.25838470458984\\
7.92999982833862	8.23740673065186\\
7.93499994277954	8.21627998352051\\
7.94000005722046	8.19500541687012\\
7.94500017166138	8.17509174346924\\
7.94999980926514	8.15612030029297\\
7.95499992370605	8.13787937164307\\
7.96000003814697	8.12125205993652\\
7.96500015258789	8.10651206970215\\
7.96999979019165	8.0932502746582\\
7.97499990463257	8.08206176757813\\
7.98000001907349	8.07389068603516\\
7.9850001335144	8.06805038452148\\
7.98999977111816	8.06457424163818\\
7.99499988555908	8.06366539001465\\
8	8.06520652770996\\
8.00500011444092	8.0691967010498\\
8.01000022888184	8.07654285430908\\
8.01500034332275	8.08717823028564\\
8.02000045776367	8.10124588012695\\
8.02499961853027	8.11865520477295\\
8.02999973297119	8.13946151733398\\
8.03499984741211	8.16384792327881\\
8.03999996185303	8.19181728363037\\
8.04500007629395	8.2233190536499\\
8.05000019073486	8.25847911834717\\
8.05500030517578	8.29730129241943\\
8.0600004196167	8.33962249755859\\
8.0649995803833	8.38553524017334\\
8.06999969482422	8.43504333496094\\
8.07499980926514	8.48843574523926\\
8.07999992370605	8.54565715789795\\
8.08500003814697	8.60676670074463\\
8.09000015258789	8.67135334014893\\
8.09500026702881	8.73936748504639\\
8.10000038146973	8.81082439422607\\
8.10499954223633	8.88579654693604\\
8.10999965667725	8.9643611907959\\
8.11499977111816	9.04644012451172\\
8.11999988555908	9.13196849822998\\
8.125	9.220778465271\\
8.13000011444092	9.31288528442383\\
8.13500022888184	9.40824222564697\\
8.14000034332275	9.50685501098633\\
8.14500045776367	9.60866069793701\\
8.14999961853027	9.71361827850342\\
8.15499973297119	9.82148838043213\\
8.15999984741211	9.93222236633301\\
8.16499996185303	10.0457525253296\\
8.17000007629395	10.1623077392578\\
8.17500019073486	10.2820148468018\\
8.18000030517578	10.4047813415527\\
8.1850004196167	10.5304317474365\\
8.1899995803833	10.6585931777954\\
8.19499969482422	10.7893447875977\\
8.19999980926514	10.9227876663208\\
8.20499992370605	11.0591068267822\\
8.21000003814697	11.1981964111328\\
8.21500015258789	11.3395957946777\\
8.22000026702881	11.4828624725342\\
8.22500038146973	11.6279850006104\\
8.22999954223633	11.7755947113037\\
8.23499965667725	11.9259910583496\\
8.23999977111816	12.0787019729614\\
8.24499988555908	12.2334957122803\\
8.25	12.3906993865967\\
8.25500011444092	12.5505990982056\\
8.26000022888184	12.713062286377\\
8.26500034332275	12.8781185150146\\
8.27000045776367	13.0457143783569\\
8.27499961853027	13.2157964706421\\
8.27999973297119	13.3879995346069\\
8.28499984741211	13.562406539917\\
8.28999996185303	13.7393760681152\\
8.29500007629395	13.9177484512329\\
8.30000019073486	14.0968742370605\\
8.30500030517578	14.2765026092529\\
8.3100004196167	14.4569387435913\\
8.3149995803833	14.6478576660156\\
8.31999969482422	14.8375263214111\\
8.32499980926514	15.0294189453125\\
8.32999992370605	15.2239112854004\\
8.33500003814697	15.4204063415527\\
8.34000015258789	15.6187915802002\\
8.34500026702881	15.8190116882324\\
8.35000038146973	16.0212593078613\\
8.35499954223633	16.2231464385986\\
8.35999965667725	16.4287719726563\\
8.36499977111816	16.6377563476563\\
8.36999988555908	16.845609664917\\
8.375	17.0537853240967\\
8.38000011444092	17.2620697021484\\
8.38500022888184	17.4695796966553\\
8.39000034332275	17.6767959594727\\
8.39500045776367	17.8814105987549\\
8.39999961853027	18.0889320373535\\
8.40499973297119	18.29368019104\\
8.40999984741211	18.4958190917969\\
8.41499996185303	18.6955795288086\\
8.42000007629395	18.8922500610352\\
8.42500019073486	19.0865840911865\\
8.43000030517578	19.2772941589355\\
8.4350004196167	19.46457862854\\
8.4399995803833	19.648717880249\\
8.44499969482422	19.8260726928711\\
8.44999980926514	19.9963054656982\\
8.45499992370605	20.15797996521\\
8.46000003814697	20.3077068328857\\
8.46500015258789	20.4460296630859\\
8.47000026702881	20.5673866271973\\
8.47500038146973	20.6734352111816\\
8.47999954223633	20.7645645141602\\
8.48499965667725	20.8419723510742\\
8.48999977111816	20.9017391204834\\
8.49499988555908	20.9486808776855\\
8.5	20.9841270446777\\
8.50500011444092	21.0064506530762\\
8.51000022888184	21.0276527404785\\
8.51500034332275	21.0489139556885\\
8.52000045776367	21.0696887969971\\
8.52499961853027	21.1081275939941\\
8.52999973297119	21.1508464813232\\
8.53499984741211	21.213191986084\\
8.53999996185303	21.2975425720215\\
8.54500007629395	21.4079418182373\\
8.55000019073486	21.5493240356445\\
8.55500030517578	21.7236995697021\\
8.5600004196167	21.9422245025635\\
8.5649995803833	22.2005500793457\\
8.56999969482422	22.5001888275146\\
8.57499980926514	22.8418674468994\\
8.57999992370605	23.2110404968262\\
8.58500003814697	23.6048374176025\\
8.59000015258789	24.0127563476563\\
8.59500026702881	24.4011211395264\\
8.60000038146973	24.7556381225586\\
8.60499954223633	25.0741577148438\\
8.60999965667725	25.3496150970459\\
8.61499977111816	25.5844097137451\\
8.61999988555908	25.784740447998\\
8.625	25.9513149261475\\
8.63000011444092	26.0921993255615\\
8.63500022888184	26.2081546783447\\
8.64000034332275	26.3077335357666\\
8.64500045776367	26.3832836151123\\
8.64999961853027	26.4569835662842\\
8.65499973297119	26.475736618042\\
8.65999984741211	26.5512714385986\\
8.66499996185303	26.5429878234863\\
8.67000007629395	26.5112037658691\\
8.67500019073486	26.4273071289063\\
8.68000030517578	26.3154678344727\\
8.6850004196167	26.1558876037598\\
8.6899995803833	25.9398365020752\\
8.69499969482422	25.6768035888672\\
8.69999980926514	25.3802146911621\\
8.70499992370605	25.0514659881592\\
8.71000003814697	24.6884727478027\\
8.71500015258789	24.2940979003906\\
8.72000026702881	23.9036884307861\\
8.72500038146973	23.5070610046387\\
8.72999954223633	23.1088638305664\\
8.73499965667725	22.7203140258789\\
8.73999977111816	22.3470420837402\\
8.74499988555908	21.9821624755859\\
8.75	21.619873046875\\
8.75500011444092	21.2319717407227\\
8.76000022888184	20.8638076782227\\
8.76500034332275	20.477970123291\\
8.77000045776367	20.0874633789063\\
8.77499961853027	19.6873779296875\\
8.77999973297119	19.2850532531738\\
8.78499984741211	18.876672744751\\
8.78999996185303	18.4689407348633\\
8.79500007629395	18.0684833526611\\
8.80000019073486	17.6811065673828\\
8.80500030517578	17.311975479126\\
8.8100004196167	16.9662265777588\\
8.8149995803833	16.6482543945313\\
8.81999969482422	16.357551574707\\
8.82499980926514	16.0952491760254\\
8.82999992370605	15.8602981567383\\
8.83500003814697	15.6470699310303\\
8.84000015258789	15.4516439437866\\
8.84500026702881	15.2721614837646\\
8.85000038146973	15.1011638641357\\
8.85499954223633	14.9416408538818\\
8.85999965667725	14.7905473709106\\
8.86499977111816	14.6487636566162\\
8.86999988555908	14.5179347991943\\
8.875	14.4012460708618\\
8.88000011444092	14.3023519515991\\
8.88500022888184	14.2242937088013\\
8.89000034332275	14.1697301864624\\
8.89500045776367	14.1379995346069\\
8.89999961853027	14.1301164627075\\
8.90499973297119	14.1492862701416\\
8.90999984741211	14.1862344741821\\
8.91499996185303	14.2293949127197\\
8.92000007629395	14.2866888046265\\
8.92500019073486	14.3592185974121\\
8.93000030517578	14.4438171386719\\
8.9350004196167	14.5319156646729\\
8.9399995803833	14.6081285476685\\
8.94499969482422	14.6909761428833\\
8.94999980926514	14.7824964523315\\
8.95499992370605	14.8806629180908\\
8.96000003814697	14.9870281219482\\
8.96500015258789	15.1023569107056\\
8.97000026702881	15.2290573120117\\
8.97500038146973	15.3625497817993\\
8.97999954223633	15.5010890960693\\
8.98499965667725	15.6565608978271\\
8.98999977111816	15.8104677200317\\
8.99499988555908	15.9557571411133\\
9	16.1118946075439\\
9.00500011444092	16.2673072814941\\
9.01000022888184	16.4195308685303\\
9.01500034332275	16.5676803588867\\
9.02000045776367	16.7124366760254\\
9.02499961853027	16.8526000976563\\
9.02999973297119	16.9873809814453\\
9.03499984741211	17.1167793273926\\
9.03999996185303	17.2403621673584\\
9.04500007629395	17.3578395843506\\
9.05000019073486	17.4689178466797\\
9.05500030517578	17.5731601715088\\
9.0600004196167	17.6698913574219\\
9.0649995803833	17.7587738037109\\
9.06999969482422	17.8391990661621\\
9.07499980926514	17.9107208251953\\
9.07999992370605	17.9729957580566\\
9.08500003814697	18.0256767272949\\
9.09000015258789	18.0677852630615\\
9.09500026702881	18.0971393585205\\
9.10000038146973	18.1132373809814\\
9.10499954223633	18.122486114502\\
9.10999965667725	18.122350692749\\
9.11499977111816	18.1098327636719\\
9.11999988555908	18.0841293334961\\
9.125	18.046558380127\\
9.13000011444092	17.9980201721191\\
9.13500022888184	17.9398460388184\\
9.14000034332275	17.8773727416992\\
9.14500045776367	17.8091659545898\\
9.14999961853027	17.7383899688721\\
9.15499973297119	17.6651859283447\\
9.15999984741211	17.5897216796875\\
9.16499996185303	17.5115528106689\\
9.17000007629395	17.429759979248\\
9.17500019073486	17.3439865112305\\
9.18000030517578	17.2536144256592\\
9.1850004196167	17.1588649749756\\
9.1899995803833	17.0595855712891\\
9.19499969482422	16.9555721282959\\
9.19999980926514	16.8478698730469\\
9.20499992370605	16.7376728057861\\
9.21000003814697	16.6268844604492\\
9.21500015258789	16.5169258117676\\
9.22000026702881	16.4074325561523\\
9.22500038146973	16.299840927124\\
9.22999954223633	16.1939830780029\\
9.23499965667725	16.0902366638184\\
9.23999977111816	15.9903049468994\\
9.24499988555908	15.8977603912354\\
9.25	15.8158416748047\\
9.25500011444092	15.7482299804688\\
9.26000022888184	15.7016725540161\\
9.26500034332275	15.6861581802368\\
9.27000045776367	15.7297506332397\\
9.27499961853027	15.8228349685669\\
9.27999973297119	15.9644546508789\\
9.28499984741211	16.1555919647217\\
9.28999996185303	16.3957214355469\\
9.29500007629395	16.6829566955566\\
9.30000019073486	17.0133991241455\\
9.30500030517578	17.3910808563232\\
9.3100004196167	17.8016586303711\\
9.3149995803833	18.2438945770264\\
9.31999969482422	18.7085266113281\\
9.32499980926514	19.1849327087402\\
9.32999992370605	19.6618270874023\\
9.33500003814697	20.1196842193604\\
9.34000015258789	20.5358085632324\\
9.34500026702881	20.8998203277588\\
9.35000038146973	21.1692447662354\\
9.35499954223633	21.316801071167\\
9.35999965667725	21.3283061981201\\
9.36499977111816	21.17409324646\\
9.36999988555908	20.848072052002\\
9.375	20.3720512390137\\
9.38000011444092	19.7665996551514\\
9.38500022888184	19.0663356781006\\
9.39000034332275	18.3285007476807\\
9.39500045776367	17.5785274505615\\
9.39999961853027	16.8921871185303\\
9.40499973297119	16.2970638275146\\
9.40999984741211	15.8094577789307\\
9.41499996185303	15.4216346740723\\
9.42000007629395	15.0850400924683\\
9.42500019073486	14.7216634750366\\
9.43000030517578	14.213173866272\\
9.4350004196167	13.4193477630615\\
9.4399995803833	12.1659183502197\\
9.44499969482422	10.2939510345459\\
9.44999980926514	7.64134311676025\\
9.45499992370605	4.11422157287598\\
9.46000003814697	-0.256407499313354\\
9.46500015258789	-5.35131502151489\\
9.47000026702881	-10.7984838485718\\
9.47500038146973	-16.1522998809814\\
9.47999954223633	-20.8689098358154\\
9.48499965667725	-24.2417488098145\\
9.48999977111816	-25.3938426971436\\
9.49499988555908	-25.4022064208984\\
9.5	-23.9062728881836\\
9.50500011444092	-21.2283706665039\\
9.51000022888184	-17.7851600646973\\
9.51500034332275	-13.9692106246948\\
9.52000045776367	-10.1773271560669\\
9.52499961853027	-6.66401338577271\\
9.52999973297119	-3.8194317817688\\
9.53499984741211	-1.66134345531464\\
9.53999996185303	-0.329259157180786\\
9.54500007629395	0.142379194498062\\
9.55000019073486	-0.18819072842598\\
9.55500030517578	-0.141730114817619\\
9.5600004196167	-0.130014464259148\\
9.5649995803833	-0.118022225797176\\
9.56999969482422	-0.103807978332043\\
9.57499980926514	-0.0884184539318085\\
9.57999992370605	-0.0780320018529892\\
9.58500003814697	-0.0628504902124405\\
9.59000015258789	-0.049817219376564\\
9.59500026702881	-0.0393693223595619\\
9.60000038146973	-0.0305179599672556\\
9.60499954223633	-0.0217159762978554\\
9.60999965667725	-0.0130962645635009\\
9.61499977111816	-0.00628915801644325\\
9.61999988555908	-2.84320503851632e-05\\
9.625	0.00505545129999518\\
9.63000011444092	0.0103868125006557\\
9.63500022888184	0.0142344459891319\\
9.64000034332275	0.0180455334484577\\
9.64500045776367	0.022253729403019\\
9.64999961853027	0.0251490511000156\\
9.65499973297119	0.0272142831236124\\
9.65999984741211	0.0299303233623505\\
9.66499996185303	0.0321050845086575\\
9.67000007629395	0.0338842868804932\\
9.67500019073486	0.0355582199990749\\
9.68000030517578	0.0370388366281986\\
9.6850004196167	0.0383692719042301\\
9.6899995803833	0.039514996111393\\
9.69499969482422	0.0404523946344852\\
9.69999980926514	0.0413074009120464\\
9.70499992370605	0.0421638339757919\\
9.71000003814697	0.0428915247321129\\
9.71500015258789	0.0433532372117043\\
9.72000026702881	0.0437819510698318\\
9.72500038146973	0.0443270839750767\\
9.72999954223633	0.0448545329272747\\
9.73499965667725	0.0453643687069416\\
9.73999977111816	0.0458163730800152\\
9.74499988555908	0.0461426563560963\\
9.75	0.0464636087417603\\
9.75500011444092	0.0466701574623585\\
9.76000022888184	0.046716932207346\\
9.76500034332275	0.0467079766094685\\
9.77000045776367	0.0467913970351219\\
9.77499961853027	0.0469533316791058\\
9.77999973297119	0.0472526662051678\\
9.78499984741211	0.0473966896533966\\
9.78999996185303	0.0474885292351246\\
9.79500007629395	0.0474901609122753\\
9.80000019073486	0.0474724359810352\\
9.80500030517578	0.0475273542106152\\
9.8100004196167	0.047629788517952\\
9.8149995803833	0.0478044264018536\\
9.81999969482422	0.0478933714330196\\
9.82499980926514	0.0479419305920601\\
9.82999992370605	0.047920823097229\\
9.83500003814697	0.0478789173066616\\
9.84000015258789	0.0478754825890064\\
9.84500026702881	0.0478914231061935\\
9.85000038146973	0.0479435659945011\\
9.85499954223633	0.0479752533137798\\
9.85999965667725	0.0480192638933659\\
9.86499977111816	0.0480787791311741\\
9.86999988555908	0.048112515360117\\
9.875	0.0480882786214352\\
9.88000011444092	0.0480148643255234\\
9.88500022888184	0.0478964932262897\\
9.89000034332275	0.0479343980550766\\
9.89500045776367	0.0480645820498466\\
9.89999961853027	0.0483139045536518\\
9.90499973297119	0.0484773479402065\\
9.90999984741211	0.0484088249504566\\
9.91499996185303	0.0481613948941231\\
9.92000007629395	0.0477368123829365\\
9.92500019073486	0.0476744696497917\\
9.93000030517578	0.0477820187807083\\
9.9350004196167	0.048118744045496\\
9.9399995803833	0.0483781471848488\\
9.94499969482422	0.0484399646520615\\
9.94999980926514	0.0484662428498268\\
9.95499992370605	0.0484569780528545\\
9.96000003814697	0.0484121739864349\\
9.96500015258789	0.0483318231999874\\
9.97000026702881	0.0482159331440926\\
9.97500038146973	0.0481593199074268\\
9.97999954223633	0.0481561720371246\\
9.98499965667725	0.048157274723053\\
9.98999977111816	0.0481626205146313\\
9.99499988555908	0.0481722168624401\\
10	0.0481860600411892\\
};
\addlegendentry{RS}

\addplot [color=red, line width=2.0pt]
  table[row sep=crcr]{%
0.0949999988079071	-0.0299894381314516\\
0.100000001490116	-0.0257046855986118\\
0.104999996721745	-0.0218393206596375\\
0.109999999403954	-0.0185790471732616\\
0.115000002086163	-0.0155304791405797\\
0.119999997317791	-0.0889443755149841\\
0.125	0.733612835407257\\
0.129999995231628	0.708834707736969\\
0.135000005364418	0.580644547939301\\
0.140000000596046	0.437405377626419\\
0.144999995827675	0.302299112081528\\
0.150000005960464	0.177156195044518\\
0.155000001192093	0.0781186819076538\\
0.159999996423721	0.0090405885130167\\
0.165000006556511	-0.0349160917103291\\
0.170000001788139	-0.0830429717898369\\
0.174999997019768	-0.139730960130692\\
0.180000007152557	-0.19446387887001\\
0.185000002384186	-0.244730442762375\\
0.189999997615814	-0.290818274021149\\
0.194999992847443	21.5026473999023\\
0.200000002980232	36.9651336669922\\
0.204999998211861	46.6574478149414\\
0.209999993443489	51.0094566345215\\
0.215000003576279	51.1502723693848\\
0.219999998807907	50.3665199279785\\
0.224999994039536	46.2912902832031\\
0.230000004172325	39.6089668273926\\
0.234999999403954	31.5891933441162\\
0.239999994635582	23.7793464660645\\
0.245000004768372	17.6687984466553\\
0.25	14.9139270782471\\
0.254999995231628	17.6688098907471\\
0.259999990463257	23.1644325256348\\
0.264999985694885	29.8816242218018\\
0.270000010728836	36.5662078857422\\
0.275000005960464	42.3075714111328\\
0.280000001192093	46.4649772644043\\
0.284999996423721	48.854377746582\\
0.28999999165535	49.4725379943848\\
0.294999986886978	49.0633201599121\\
0.300000011920929	47.8128128051758\\
0.305000007152557	45.2762718200684\\
0.310000002384186	41.9210624694824\\
0.314999997615814	38.2575531005859\\
0.319999992847443	34.8240165710449\\
0.324999988079071	32.0753021240234\\
0.330000013113022	30.3101711273193\\
0.33500000834465	29.6740970611572\\
0.340000003576279	30.4609870910645\\
0.344999998807907	31.5751647949219\\
0.349999994039536	32.659538269043\\
0.354999989271164	33.4077033996582\\
0.360000014305115	33.601619720459\\
0.365000009536743	33.3052711486816\\
0.370000004768372	32.6341171264648\\
0.375	31.2640914916992\\
0.379999995231628	29.2632675170898\\
0.384999990463257	26.7898025512695\\
0.389999985694885	24.0770683288574\\
0.395000010728836	21.37109375\\
0.400000005960464	18.9634304046631\\
0.405000001192093	16.911865234375\\
0.409999996423721	15.4792547225952\\
0.41499999165535	14.5866479873657\\
0.419999986886978	14.1904325485229\\
0.425000011920929	14.2311086654663\\
0.430000007152557	14.2809619903564\\
0.435000002384186	14.1854095458984\\
0.439999997615814	13.9971570968628\\
0.444999992847443	13.5096998214722\\
0.449999988079071	12.6819686889648\\
0.455000013113022	11.5376176834106\\
0.46000000834465	10.1548671722412\\
0.465000003576279	8.65134811401367\\
0.469999998807907	7.16669273376465\\
0.474999994039536	5.84380483627319\\
0.479999989271164	4.79407358169556\\
0.485000014305115	4.09440469741821\\
0.490000009536743	3.76428914070129\\
0.495000004768372	3.91494417190552\\
0.5	4.26115274429321\\
0.504999995231628	4.66360139846802\\
0.509999990463257	5.0306830406189\\
0.514999985694885	5.29465961456299\\
0.519999980926514	5.42382860183716\\
0.524999976158142	5.43039655685425\\
0.529999971389771	5.40360927581787\\
0.535000026226044	5.26258754730225\\
0.540000021457672	5.07122135162354\\
0.545000016689301	4.89600515365601\\
0.550000011920929	4.79770135879517\\
0.555000007152557	4.90174198150635\\
0.560000002384186	5.16664838790894\\
0.564999997615814	5.57450532913208\\
0.569999992847443	6.09906959533691\\
0.574999988079071	6.69771385192871\\
0.579999983310699	7.34170341491699\\
0.584999978542328	8.00221729278564\\
0.589999973773956	8.64041614532471\\
0.595000028610229	9.2532844543457\\
0.600000023841858	9.83022880554199\\
0.605000019073486	10.3723745346069\\
0.610000014305115	10.8939628601074\\
0.615000009536743	11.4000959396362\\
0.620000004768372	11.9077930450439\\
0.625	12.4240446090698\\
0.629999995231628	12.9604215621948\\
0.634999990463257	13.5152492523193\\
0.639999985694885	14.093843460083\\
0.644999980926514	14.6917219161987\\
0.649999976158142	15.300760269165\\
0.654999971389771	15.9140701293945\\
0.660000026226044	16.5195465087891\\
0.665000021457672	17.1086444854736\\
0.670000016689301	17.675163269043\\
0.675000011920929	18.2106170654297\\
0.680000007152557	18.7128276824951\\
0.685000002384186	19.1800441741943\\
0.689999997615814	19.6136054992676\\
0.694999992847443	20.0151271820068\\
0.699999988079071	20.3870849609375\\
0.704999983310699	20.7342796325684\\
0.709999978542328	21.0573024749756\\
0.714999973773956	21.3535251617432\\
0.720000028610229	21.6257610321045\\
0.725000023841858	21.8781147003174\\
0.730000019073486	22.093843460083\\
0.735000014305115	22.2867984771729\\
0.740000009536743	22.444995880127\\
0.745000004768372	22.5735740661621\\
0.75	22.6670207977295\\
0.754999995231628	22.728443145752\\
0.759999990463257	22.7538642883301\\
0.764999985694885	22.7552547454834\\
0.769999980926514	22.7367458343506\\
0.774999976158142	22.6877784729004\\
0.779999971389771	22.6049976348877\\
0.785000026226044	22.4917125701904\\
0.790000021457672	22.3515243530273\\
0.795000016689301	22.1864624023438\\
0.800000011920929	21.9988441467285\\
0.805000007152557	21.7902984619141\\
0.810000002384186	21.5633354187012\\
0.814999997615814	21.319019317627\\
0.819999992847443	21.0592975616455\\
0.824999988079071	20.7866649627686\\
0.829999983310699	20.5001430511475\\
0.834999978542328	20.2014865875244\\
0.839999973773956	19.8936824798584\\
0.845000028610229	19.5767917633057\\
0.850000023841858	19.2524147033691\\
0.855000019073486	18.9214115142822\\
0.860000014305115	18.5885677337646\\
0.865000009536743	18.2559795379639\\
0.870000004768372	17.9245548248291\\
0.875	17.5964202880859\\
0.879999995231628	17.2729644775391\\
0.884999990463257	16.9554443359375\\
0.889999985694885	16.6449775695801\\
0.894999980926514	16.3428516387939\\
0.899999976158142	16.0501251220703\\
0.904999971389771	15.7675008773804\\
0.910000026226044	15.4971446990967\\
0.915000021457672	15.2389402389526\\
0.920000016689301	14.9925107955933\\
0.925000011920929	14.7593984603882\\
0.930000007152557	14.5399255752563\\
0.935000002384186	14.3340435028076\\
0.939999997615814	14.1424589157104\\
0.944999992847443	13.9663572311401\\
0.949999988079071	13.8071918487549\\
0.954999983310699	13.6658086776733\\
0.959999978542328	13.5420112609863\\
0.964999973773956	13.4357986450195\\
0.970000028610229	13.3479881286621\\
0.975000023841858	13.2778701782227\\
0.980000019073486	13.2253360748291\\
0.985000014305115	13.1904487609863\\
0.990000009536743	13.1725130081177\\
0.995000004768372	13.1713552474976\\
1	13.1871662139893\\
1.00499999523163	13.2197685241699\\
1.00999999046326	13.2685976028442\\
1.01499998569489	13.327299118042\\
1.01999998092651	13.3960676193237\\
1.02499997615814	13.4760112762451\\
1.02999997138977	13.5675487518311\\
1.0349999666214	13.6697597503662\\
1.03999996185303	13.7826108932495\\
1.04499995708466	13.9050512313843\\
1.04999995231628	14.0367269515991\\
1.05499994754791	14.1766052246094\\
1.05999994277954	14.3239994049072\\
1.06500005722046	14.4785413742065\\
1.07000005245209	14.6388912200928\\
1.07500004768372	14.8043718338013\\
1.08000004291534	14.9736785888672\\
1.08500003814697	15.1460390090942\\
1.0900000333786	15.3206968307495\\
1.09500002861023	15.49671459198\\
1.10000002384186	15.6732091903687\\
1.10500001907349	15.8494520187378\\
1.11000001430511	16.0249824523926\\
1.11500000953674	16.1988334655762\\
1.12000000476837	16.3702297210693\\
1.125	16.5388507843018\\
1.12999999523163	16.7039833068848\\
1.13499999046326	16.8648815155029\\
1.13999998569489	17.0209770202637\\
1.14499998092651	17.1720542907715\\
1.14999997615814	17.3172874450684\\
1.15499997138977	17.4561100006104\\
1.1599999666214	17.5882949829102\\
1.16499996185303	17.7133598327637\\
1.16999995708466	17.8306999206543\\
1.17499995231628	17.9399433135986\\
1.17999994754791	18.0414237976074\\
1.18499994277954	18.1344165802002\\
1.19000005722046	18.2185955047607\\
1.19500005245209	18.2938919067383\\
1.20000004768372	18.3601875305176\\
1.20500004291534	18.4172019958496\\
1.21000003814697	18.4648303985596\\
1.2150000333786	18.5036582946777\\
1.22000002861023	18.5336418151855\\
1.22500002384186	18.5550136566162\\
1.23000001907349	18.5684356689453\\
1.23500001430511	18.5747337341309\\
1.24000000953674	18.5744380950928\\
1.24500000476837	18.5683860778809\\
1.25	18.5537300109863\\
1.25499999523163	18.5316543579102\\
1.25999999046326	18.5018081665039\\
1.26499998569489	18.4644451141357\\
1.26999998092651	18.4202842712402\\
1.27499997615814	18.3689289093018\\
1.27999997138977	18.3109436035156\\
1.2849999666214	18.2482147216797\\
1.28999996185303	18.1803722381592\\
1.29499995708466	18.1079597473145\\
1.29999995231628	18.032018661499\\
1.30499994754791	17.952917098999\\
1.30999994277954	17.8712463378906\\
1.31500005722046	17.7872619628906\\
1.32000005245209	17.701602935791\\
1.32500004768372	17.6148242950439\\
1.33000004291534	17.5271549224854\\
1.33500003814697	17.4390087127686\\
1.3400000333786	17.350902557373\\
1.34500002861023	17.263298034668\\
1.35000002384186	17.1766757965088\\
1.35500001907349	17.0914878845215\\
1.36000001430511	17.0080947875977\\
1.36500000953674	16.9265899658203\\
1.37000000476837	16.8472709655762\\
1.375	16.7702903747559\\
1.37999999523163	16.6954154968262\\
1.38499999046326	16.6231384277344\\
1.38999998569489	16.5535945892334\\
1.39499998092651	16.4874248504639\\
1.39999997615814	16.4249000549316\\
1.40499997138977	16.3664169311523\\
1.4099999666214	16.3134708404541\\
1.41499996185303	16.2662353515625\\
1.41999995708466	16.2249374389648\\
1.42499995231628	16.1882762908936\\
1.42999994754791	16.1569232940674\\
1.43499994277954	16.1307468414307\\
1.44000005722046	16.1093330383301\\
1.44500005245209	16.0927505493164\\
1.45000004768372	16.0805282592773\\
1.45500004291534	16.0720291137695\\
1.46000003814697	16.0671539306641\\
1.4650000333786	16.066047668457\\
1.47000002861023	16.0689544677734\\
1.47500002384186	16.075870513916\\
1.48000001907349	16.0871181488037\\
1.48500001430511	16.1032695770264\\
1.49000000953674	16.1249694824219\\
1.49500000476837	16.1502494812012\\
1.5	16.1789970397949\\
1.50499999523163	16.211296081543\\
1.50999999046326	16.2475109100342\\
1.51499998569489	16.2874050140381\\
1.51999998092651	16.3305568695068\\
1.52499997615814	16.376781463623\\
1.52999997138977	16.4260215759277\\
1.5349999666214	16.4781169891357\\
1.53999996185303	16.5327987670898\\
1.54499995708466	16.5896415710449\\
1.54999995231628	16.6481914520264\\
1.55499994754791	16.7082595825195\\
1.55999994277954	16.7691841125488\\
1.56500005722046	16.8306427001953\\
1.57000005245209	16.8586387634277\\
1.57500004768372	16.9130973815918\\
1.58000004291534	16.9424285888672\\
1.58500003814697	16.9528293609619\\
1.5900000333786	16.9590377807617\\
1.59500002861023	16.9673843383789\\
1.60000002384186	16.9824237823486\\
1.60500001907349	17.0087909698486\\
1.61000001430511	17.0489768981934\\
1.61500000953674	17.0992965698242\\
1.62000000476837	17.1613903045654\\
1.625	17.2246227264404\\
1.62999999523163	17.283332824707\\
1.63499999046326	17.3301753997803\\
1.63999998569489	17.3608131408691\\
1.64499998092651	17.3885631561279\\
1.64999997615814	17.4081859588623\\
1.65499997138977	17.4203491210938\\
1.6599999666214	17.4265003204346\\
1.66499996185303	17.4276790618896\\
1.66999995708466	17.4233055114746\\
1.67499995231628	17.4176330566406\\
1.67999994754791	17.4144191741943\\
1.68499994277954	17.4108276367188\\
1.69000005722046	17.405294418335\\
1.69500005245209	17.3987522125244\\
1.70000004768372	17.3909149169922\\
1.70500004291534	17.3802890777588\\
1.71000003814697	17.3666839599609\\
1.7150000333786	17.3507328033447\\
1.72000002861023	17.3327960968018\\
1.72500002384186	17.3112506866455\\
1.73000001907349	17.2866668701172\\
1.73500001430511	17.258674621582\\
1.74000000953674	17.2272987365723\\
1.74500000476837	17.1928520202637\\
1.75	17.1559410095215\\
1.75499999523163	17.1162376403809\\
1.75999999046326	17.0742855072021\\
1.76499998569489	17.0299987792969\\
1.76999998092651	16.98362159729\\
1.77499997615814	16.9360656738281\\
1.77999997138977	16.8874244689941\\
1.7849999666214	16.8374729156494\\
1.78999996185303	16.786376953125\\
1.79499995708466	16.7342643737793\\
1.79999995231628	16.6811237335205\\
1.80499994754791	16.6270427703857\\
1.80999994277954	16.5719680786133\\
1.81500005722046	16.5159130096436\\
1.82000005245209	16.4589080810547\\
1.82500004768372	16.4010124206543\\
1.83000004291534	16.3422355651855\\
1.83500003814697	16.2825317382813\\
1.8400000333786	16.2218742370605\\
1.84500002861023	16.1603546142578\\
1.85000002384186	16.0988807678223\\
1.85500001907349	16.0370464324951\\
1.86000001430511	15.9748420715332\\
1.86500000953674	15.9122266769409\\
1.87000000476837	15.8492231369019\\
1.875	15.7866735458374\\
1.87999999523163	15.7252397537231\\
1.88499999046326	15.6644248962402\\
1.88999998569489	15.6041784286499\\
1.89499998092651	15.5425004959106\\
1.89999997615814	15.4804229736328\\
1.90499997138977	15.417950630188\\
1.9099999666214	15.3551034927368\\
1.91499996185303	15.2918672561646\\
1.91999995708466	15.2282466888428\\
1.92499995231628	15.1648206710815\\
1.92999994754791	15.1015644073486\\
1.93499994277954	15.0383501052856\\
1.94000005722046	14.9748382568359\\
1.94500005245209	14.910906791687\\
1.95000004768372	14.8467149734497\\
1.95500004291534	14.7827777862549\\
1.96000003814697	14.7199382781982\\
1.9650000333786	14.6581678390503\\
1.97000002861023	14.5965719223022\\
1.97500002384186	14.5318851470947\\
1.98000001907349	14.4647245407104\\
1.98500001430511	14.3948040008545\\
1.99000000953674	14.3242244720459\\
1.99500000476837	14.2519578933716\\
2	14.178050994873\\
2.00500011444092	14.1045875549316\\
2.00999999046326	14.0312767028809\\
2.01500010490417	13.9586095809937\\
2.01999998092651	13.8905172348022\\
2.02500009536743	13.8274192810059\\
2.02999997138977	13.7701425552368\\
2.03500008583069	13.7149381637573\\
2.03999996185303	13.6631498336792\\
2.04500007629395	13.6171531677246\\
2.04999995231628	13.5803852081299\\
2.0550000667572	13.5528173446655\\
2.05999994277954	13.5337915420532\\
2.06500005722046	13.5241403579712\\
2.0699999332428	13.5219478607178\\
2.07500004768372	13.5277433395386\\
2.07999992370605	13.541407585144\\
2.08500003814697	13.5628757476807\\
2.08999991416931	13.590989112854\\
2.09500002861023	13.6256761550903\\
2.09999990463257	13.6658334732056\\
2.10500001907349	13.7100210189819\\
2.10999989509583	13.7600126266479\\
2.11500000953674	13.8160753250122\\
2.11999988555908	13.8776597976685\\
2.125	13.9468250274658\\
2.13000011444092	14.0200080871582\\
2.13499999046326	14.0954265594482\\
2.14000010490417	14.1694459915161\\
2.14499998092651	14.2398262023926\\
2.15000009536743	14.3123207092285\\
2.15499997138977	14.3739986419678\\
2.16000008583069	14.4064283370972\\
2.16499996185303	14.4326801300049\\
2.17000007629395	14.4512777328491\\
2.17499995231628	14.4517221450806\\
2.1800000667572	14.4427299499512\\
2.18499994277954	14.4287414550781\\
2.19000005722046	14.4166088104248\\
2.1949999332428	14.4048614501953\\
2.20000004768372	14.3950576782227\\
2.20499992370605	14.3881673812866\\
2.21000003814697	14.3836679458618\\
2.21499991416931	14.3812885284424\\
2.22000002861023	14.3819313049316\\
2.22499990463257	14.386209487915\\
2.23000001907349	14.3871278762817\\
2.23499989509583	14.3731889724731\\
2.24000000953674	14.3610076904297\\
2.24499988555908	14.3499937057495\\
2.25	14.3369045257568\\
2.25500011444092	14.3259744644165\\
2.25999999046326	14.3227872848511\\
2.26500010490417	14.336368560791\\
2.26999998092651	14.3727378845215\\
2.27500009536743	14.433690071106\\
2.27999997138977	14.5147981643677\\
2.28500008583069	14.6045522689819\\
2.28999996185303	14.7060565948486\\
2.29500007629395	14.8190145492554\\
2.29999995231628	14.9390878677368\\
2.3050000667572	15.0685729980469\\
2.30999994277954	15.2057762145996\\
2.31500005722046	15.3496427536011\\
2.3199999332428	15.498517036438\\
2.32500004768372	15.6508121490479\\
2.32999992370605	15.8062257766724\\
2.33500003814697	15.9624290466309\\
2.33999991416931	16.121431350708\\
2.34500002861023	16.2849082946777\\
2.34999990463257	16.445671081543\\
2.35500001907349	16.6077442169189\\
2.35999989509583	16.7696762084961\\
2.36500000953674	16.923828125\\
2.36999988555908	17.0667419433594\\
2.375	17.2008953094482\\
2.38000011444092	17.323112487793\\
2.38499999046326	17.4277076721191\\
2.39000010490417	17.5108337402344\\
2.39499998092651	17.5693626403809\\
2.40000009536743	17.6040630340576\\
2.40499997138977	17.6135158538818\\
2.41000008583069	17.6035079956055\\
2.41499996185303	17.5782089233398\\
2.42000007629395	17.5428123474121\\
2.42499995231628	17.510892868042\\
2.4300000667572	17.4795379638672\\
2.43499994277954	17.4576568603516\\
2.44000005722046	17.4488277435303\\
2.4449999332428	17.4318714141846\\
2.45000004768372	17.4498958587646\\
2.45499992370605	17.4756641387939\\
2.46000003814697	17.519645690918\\
2.46499991416931	17.5862255096436\\
2.47000002861023	17.6707096099854\\
2.47499990463257	17.7701358795166\\
2.48000001907349	17.8825092315674\\
2.48499989509583	18.0041065216064\\
2.49000000953674	18.1374015808105\\
2.49499988555908	18.286039352417\\
2.5	18.441987991333\\
2.50500011444092	18.5842456817627\\
2.50999999046326	18.7096557617188\\
2.51500010490417	18.83203125\\
2.51999998092651	18.9387817382813\\
2.52500009536743	19.0300483703613\\
2.52999997138977	19.1103115081787\\
2.53500008583069	19.1819076538086\\
2.53999996185303	19.2420959472656\\
2.54500007629395	19.2958965301514\\
2.54999995231628	19.3389492034912\\
2.5550000667572	19.3709411621094\\
2.55999994277954	19.3873195648193\\
2.56500005722046	19.3860893249512\\
2.5699999332428	19.3641204833984\\
2.57500004768372	19.3187370300293\\
2.57999992370605	19.2497596740723\\
2.58500003814697	19.157958984375\\
2.58999991416931	19.0365829467773\\
2.59500002861023	18.9010467529297\\
2.59999990463257	18.7522354125977\\
2.60500001907349	18.5839042663574\\
2.60999989509583	18.4046611785889\\
2.61500000953674	18.2153148651123\\
2.61999988555908	18.0208015441895\\
2.625	17.8222713470459\\
2.63000011444092	17.6203346252441\\
2.63499999046326	17.4047088623047\\
2.64000010490417	17.1918354034424\\
2.64499998092651	16.9745597839355\\
2.65000009536743	16.7294101715088\\
2.65499997138977	16.4662132263184\\
2.66000008583069	16.1830902099609\\
2.66499996185303	15.8709630966187\\
2.67000007629395	15.530704498291\\
2.67499995231628	15.1785821914673\\
2.6800000667572	14.8285293579102\\
2.68499994277954	14.4597501754761\\
2.69000005722046	14.0974798202515\\
2.6949999332428	13.7442922592163\\
2.70000004768372	13.4031944274902\\
2.70499992370605	13.0772018432617\\
2.71000003814697	12.7654247283936\\
2.71499991416931	12.4655323028564\\
2.72000002861023	12.1778287887573\\
2.72499990463257	11.8995571136475\\
2.73000001907349	11.6286106109619\\
2.73499989509583	11.3644599914551\\
2.74000000953674	11.1065969467163\\
2.74499988555908	10.8579406738281\\
2.75	10.6212396621704\\
2.75500011444092	10.4026746749878\\
2.75999999046326	10.2129745483398\\
2.76500010490417	10.0600233078003\\
2.76999998092651	9.94012069702148\\
2.77500009536743	9.84902858734131\\
2.77999997138977	9.7867431640625\\
2.78500008583069	9.75536918640137\\
2.78999996185303	9.75267314910889\\
2.79500007629395	9.76997089385986\\
2.79999995231628	9.80062675476074\\
2.8050000667572	9.83953666687012\\
2.80999994277954	9.88133907318115\\
2.81500005722046	9.92139625549316\\
2.8199999332428	9.95650577545166\\
2.82500004768372	9.98752689361572\\
2.82999992370605	10.016263961792\\
2.83500003814697	10.0544586181641\\
2.83999991416931	10.1073932647705\\
2.84500002861023	10.1861324310303\\
2.84999990463257	10.306134223938\\
2.85500001907349	10.4711179733276\\
2.85999989509583	10.694260597229\\
2.86500000953674	10.9867153167725\\
2.86999988555908	11.3351583480835\\
2.875	11.7620325088501\\
2.88000011444092	12.261435508728\\
2.88499999046326	12.8369264602661\\
2.89000010490417	13.4950380325317\\
2.89499998092651	14.2306690216064\\
2.90000009536743	15.0481967926025\\
2.90499997138977	15.9526948928833\\
2.91000008583069	16.9438152313232\\
2.91499996185303	18.0174560546875\\
2.92000007629395	19.1616153717041\\
2.92499995231628	20.3547630310059\\
2.9300000667572	21.5690402984619\\
2.93499994277954	22.7727146148682\\
2.94000005722046	23.9311923980713\\
2.9449999332428	25.0094451904297\\
2.95000004768372	25.9876918792725\\
2.95499992370605	26.8440036773682\\
2.96000003814697	27.5728054046631\\
2.96499991416931	28.2228832244873\\
2.97000002861023	28.7458763122559\\
2.97499990463257	29.1174278259277\\
2.98000001907349	29.3443031311035\\
2.98499989509583	29.4256534576416\\
2.99000000953674	29.4229850769043\\
2.99499988555908	29.2978019714355\\
3	29.0981998443604\\
3.00500011444092	28.8361854553223\\
3.00999999046326	28.5149917602539\\
3.01500010490417	28.1317081451416\\
3.01999998092651	27.6763916015625\\
3.02500009536743	27.1417446136475\\
3.02999997138977	26.505313873291\\
3.03500008583069	25.7488079071045\\
3.03999996185303	24.8770961761475\\
3.04500007629395	23.894926071167\\
3.04999995231628	22.8280010223389\\
3.0550000667572	21.7089080810547\\
3.05999994277954	20.5526065826416\\
3.06500005722046	19.4068489074707\\
3.0699999332428	18.3103942871094\\
3.07500004768372	17.2417640686035\\
3.07999992370605	16.2319965362549\\
3.08500003814697	15.2752170562744\\
3.08999991416931	14.3501987457275\\
3.09500002861023	13.4619951248169\\
3.09999990463257	12.6111965179443\\
3.10500001907349	11.7844018936157\\
3.10999989509583	10.9948358535767\\
3.11500000953674	10.2507209777832\\
3.11999988555908	9.56469917297363\\
3.125	8.96479034423828\\
3.13000011444092	8.49861621856689\\
3.13499999046326	8.22828197479248\\
3.14000010490417	8.06003093719482\\
3.14499998092651	7.95915508270264\\
3.15000009536743	7.88252305984497\\
3.15499997138977	7.80240869522095\\
3.16000008583069	7.70354557037354\\
3.16499996185303	7.587327003479\\
3.17000007629395	7.45380878448486\\
3.17499995231628	7.30809545516968\\
3.1800000667572	7.18236446380615\\
3.18499994277954	7.10262012481689\\
3.19000005722046	7.06950378417969\\
3.1949999332428	7.06073570251465\\
3.20000004768372	7.09871339797974\\
3.20499992370605	7.2204909324646\\
3.21000003814697	7.40809440612793\\
3.21499991416931	7.71587419509888\\
3.22000002861023	8.15838718414307\\
3.22499990463257	8.7314395904541\\
3.23000001907349	9.44294452667236\\
3.23499989509583	10.2921485900879\\
3.24000000953674	11.2809953689575\\
3.24499988555908	12.3963937759399\\
3.25	13.6235752105713\\
3.25500011444092	14.9361009597778\\
3.25999999046326	16.3103046417236\\
3.26500010490417	17.7109298706055\\
3.26999998092651	19.1076755523682\\
3.27500009536743	20.4607601165771\\
3.27999997138977	21.7530918121338\\
3.28500008583069	22.9684524536133\\
3.28999996185303	24.0916137695313\\
3.29500007629395	25.1288185119629\\
3.29999995231628	26.0787830352783\\
3.3050000667572	26.9372959136963\\
3.30999994277954	27.7369747161865\\
3.31500005722046	28.4464588165283\\
3.3199999332428	29.0742244720459\\
3.32500004768372	29.621898651123\\
3.32999992370605	30.0700969696045\\
3.33500003814697	30.4141597747803\\
3.33999991416931	30.6645851135254\\
3.34500002861023	30.8126468658447\\
3.34999990463257	30.8102588653564\\
3.35500001907349	30.6419448852539\\
3.35999989509583	30.299654006958\\
3.36500000953674	29.7803230285645\\
3.36999988555908	29.0975475311279\\
3.375	28.2692375183105\\
3.38000011444092	27.3216323852539\\
3.38499999046326	26.2835998535156\\
3.39000010490417	25.1830959320068\\
3.39499998092651	24.0433216094971\\
3.40000009536743	22.8812484741211\\
3.40499997138977	21.7026290893555\\
3.41000008583069	20.5131454467773\\
3.41499996185303	19.3149108886719\\
3.42000007629395	18.1097965240479\\
3.42499995231628	16.8996067047119\\
3.4300000667572	15.695333480835\\
3.43499994277954	14.5117263793945\\
3.44000005722046	13.372802734375\\
3.4449999332428	12.2951469421387\\
3.45000004768372	11.3032531738281\\
3.45499992370605	10.4246969223022\\
3.46000003814697	9.77677249908447\\
3.46499991416931	9.29961109161377\\
3.47000002861023	8.87908172607422\\
3.47499990463257	8.49471092224121\\
3.48000001907349	8.12078666687012\\
3.48499989509583	7.75198030471802\\
3.49000000953674	7.33343315124512\\
3.49499988555908	6.93955373764038\\
3.5	6.61949491500854\\
3.50500011444092	6.32821178436279\\
3.50999999046326	6.0773720741272\\
3.51500010490417	5.91170072555542\\
3.51999998092651	5.77441740036011\\
3.52500009536743	5.72847890853882\\
3.52999997138977	5.71728181838989\\
3.53500008583069	5.75508546829224\\
3.53999996185303	5.86746025085449\\
3.54500007629395	6.10973930358887\\
3.54999995231628	6.69311189651489\\
3.5550000667572	7.70655870437622\\
3.55999994277954	9.18849468231201\\
3.56500005722046	11.1566867828369\\
3.5699999332428	13.558708190918\\
3.57500004768372	16.2661113739014\\
3.57999992370605	19.1057529449463\\
3.58500003814697	21.8702354431152\\
3.58999991416931	24.3834800720215\\
3.59500002861023	26.5050830841064\\
3.59999990463257	28.1658096313477\\
3.60500001907349	29.3165035247803\\
3.60999989509583	30.1268939971924\\
3.61500000953674	30.6609725952148\\
3.61999988555908	30.7907276153564\\
3.625	30.6460609436035\\
3.63000011444092	30.366886138916\\
3.63499999046326	30.1189765930176\\
3.64000010490417	30.0222301483154\\
3.64499998092651	30.1151885986328\\
3.65000009536743	30.5038261413574\\
3.65499997138977	30.9080200195313\\
3.66000008583069	31.1662254333496\\
3.66499996185303	31.2252216339111\\
3.67000007629395	31.0126819610596\\
3.67499995231628	30.3584270477295\\
3.6800000667572	29.2441062927246\\
3.68499994277954	27.7141761779785\\
3.69000005722046	25.8852596282959\\
3.6949999332428	23.9006042480469\\
3.70000004768372	21.9412460327148\\
3.70499992370605	20.1222229003906\\
3.71000003814697	18.5750617980957\\
3.71499991416931	17.3060054779053\\
3.72000002861023	16.2840805053711\\
3.72499990463257	15.4366731643677\\
3.73000001907349	14.6715383529663\\
3.73499989509583	13.8949165344238\\
3.74000000953674	13.0514507293701\\
3.74499988555908	12.1184129714966\\
3.75	11.2815284729004\\
3.75500011444092	10.5753746032715\\
3.75999999046326	10.0044889450073\\
3.76500010490417	9.44510173797607\\
3.76999998092651	8.86135101318359\\
3.77500009536743	8.30577754974365\\
3.77999997138977	7.79850339889526\\
3.78500008583069	7.3302903175354\\
3.78999996185303	6.99551105499268\\
3.79500007629395	6.81561660766602\\
3.79999995231628	6.69309568405151\\
3.8050000667572	6.58698987960815\\
3.80999994277954	6.44391536712646\\
3.81500005722046	6.22497415542603\\
3.8199999332428	5.90897607803345\\
3.82500004768372	5.52591609954834\\
3.82999992370605	5.17460060119629\\
3.83500003814697	5.02102851867676\\
3.83999991416931	5.40763139724731\\
3.84500002861023	6.67980527877808\\
3.84999990463257	8.6721076965332\\
3.85500001907349	11.3447113037109\\
3.85999989509583	14.5397691726685\\
3.86500000953674	18.0028057098389\\
3.86999988555908	21.4379043579102\\
3.875	24.5918159484863\\
3.88000011444092	27.2598648071289\\
3.88499999046326	29.3202075958252\\
3.89000010490417	30.7335090637207\\
3.89499998092651	31.5895557403564\\
3.90000009536743	32.2241554260254\\
3.90499997138977	32.3272972106934\\
3.91000008583069	32.1180000305176\\
3.91499996185303	31.8093910217285\\
3.92000007629395	31.5849742889404\\
3.92499995231628	31.6082935333252\\
3.9300000667572	32.0400657653809\\
3.93499994277954	32.7019500732422\\
3.94000005722046	33.2947463989258\\
3.9449999332428	33.6381072998047\\
3.95000004768372	33.6441040039063\\
3.95499992370605	33.3896484375\\
3.96000003814697	32.5814743041992\\
3.96499991416931	31.189037322998\\
3.97000002861023	29.2845726013184\\
3.97499990463257	27.0124626159668\\
3.98000001907349	24.5703029632568\\
3.98499989509583	22.1615142822266\\
3.99000000953674	19.9519481658936\\
3.99499988555908	18.0517997741699\\
4	16.5038738250732\\
4.00500011444092	15.2648868560791\\
4.01000022888184	14.2449207305908\\
4.0149998664856	13.3328075408936\\
4.01999998092651	12.4151792526245\\
4.02500009536743	11.4480257034302\\
4.03000020980835	10.5240325927734\\
4.03499984741211	9.77143096923828\\
4.03999996185303	9.10470008850098\\
4.04500007629395	8.46592426300049\\
4.05000019073486	7.8173770904541\\
4.05499982833862	7.15389633178711\\
4.05999994277954	6.48863172531128\\
4.06500005722046	5.91715860366821\\
4.07000017166138	5.47376537322998\\
4.07499980926514	5.23275947570801\\
4.07999992370605	5.09528970718384\\
4.08500003814697	5.03686857223511\\
4.09000015258789	4.94294023513794\\
4.09499979019165	4.65312147140503\\
4.09999990463257	4.01311826705933\\
4.10500001907349	3.01895356178284\\
4.1100001335144	1.81848335266113\\
4.11499977111816	1.09131479263306\\
4.11999988555908	0.877925932407379\\
4.125	2.78654837608337\\
4.13000011444092	7.17699193954468\\
4.13500022888184	12.7526578903198\\
4.1399998664856	18.9977798461914\\
4.14499998092651	25.1676788330078\\
4.15000009536743	30.616527557373\\
4.15500020980835	34.8467979431152\\
4.15999984741211	37.5926322937012\\
4.16499996185303	38.8688926696777\\
4.17000007629395	39.6591567993164\\
4.17500019073486	38.9175033569336\\
4.17999982833862	36.9499931335449\\
4.18499994277954	34.2854423522949\\
4.19000005722046	31.6428852081299\\
4.19500017166138	29.6732273101807\\
4.19999980926514	28.9229011535645\\
4.20499992370605	30.2012119293213\\
4.21000003814697	32.2299041748047\\
4.21500015258789	34.2866744995117\\
4.21999979019165	35.7911758422852\\
4.22499990463257	36.314338684082\\
4.23000001907349	36.0210838317871\\
4.2350001335144	34.8886947631836\\
4.23999977111816	32.5033149719238\\
4.24499988555908	28.9662227630615\\
4.25	24.9523029327393\\
4.25500011444092	20.7854404449463\\
4.26000022888184	17.0366401672363\\
4.2649998664856	14.1133642196655\\
4.26999998092651	12.2211847305298\\
4.27500009536743	11.3651294708252\\
4.28000020980835	11.5523586273193\\
4.28499984741211	11.8159894943237\\
4.28999996185303	11.8588848114014\\
4.29500007629395	11.5633945465088\\
4.30000019073486	11.103328704834\\
4.30499982833862	10.4408006668091\\
4.30999994277954	9.29651546478271\\
4.31500005722046	7.94091320037842\\
4.32000017166138	6.70835447311401\\
4.32499980926514	5.7650785446167\\
4.32999992370605	5.2266583442688\\
4.33500003814697	5.1520619392395\\
4.34000015258789	5.33920335769653\\
4.34499979019165	5.64602279663086\\
4.34999990463257	5.80807065963745\\
4.35500001907349	5.52264547348022\\
4.3600001335144	4.54168128967285\\
4.36499977111816	2.81058263778687\\
4.36999988555908	1.245192527771\\
4.375	1.10703527927399\\
4.38000011444092	0.928909599781036\\
4.38500022888184	0.693621158599854\\
4.3899998664856	0.39326399564743\\
4.39499998092651	7.81831836700439\\
4.40000009536743	16.0391120910645\\
4.40500020980835	24.4621868133545\\
4.40999984741211	32.0133399963379\\
4.41499996185303	37.8672332763672\\
4.42000007629395	41.6499443054199\\
4.42500019073486	43.329216003418\\
4.42999982833862	44.1748046875\\
4.43499994277954	43.1589698791504\\
4.44000005722046	40.3543968200684\\
4.44500017166138	36.5007171630859\\
4.44999980926514	32.5924110412598\\
4.45499992370605	29.6213245391846\\
4.46000003814697	28.345817565918\\
4.46500015258789	29.9516448974609\\
4.46999979019165	32.8241271972656\\
4.47499990463257	35.7248382568359\\
4.48000001907349	37.9457359313965\\
4.4850001335144	38.8733940124512\\
4.48999977111816	38.4898071289063\\
4.49499988555908	37.3161926269531\\
4.5	34.463321685791\\
4.50500011444092	30.2257137298584\\
4.51000022888184	25.1660747528076\\
4.5149998664856	19.9702186584473\\
4.51999998092651	15.3174781799316\\
4.52500009536743	11.7788133621216\\
4.53000020980835	9.57603931427002\\
4.53499984741211	8.76070976257324\\
4.53999996185303	9.32829475402832\\
4.54500007629395	9.94140625\\
4.55000019073486	10.2190027236938\\
4.55499982833862	10.2738561630249\\
4.55999994277954	9.98919296264648\\
4.56500005722046	9.20252704620361\\
4.57000017166138	7.87842273712158\\
4.57499980926514	6.41682147979736\\
4.57999992370605	5.1109766960144\\
4.58500003814697	4.09031867980957\\
4.59000015258789	3.5128858089447\\
4.59499979019165	3.47863268852234\\
4.59999990463257	3.89049696922302\\
4.60500001907349	4.58016872406006\\
4.6100001335144	5.17239904403687\\
4.61499977111816	5.07120943069458\\
4.61999988555908	3.65787672996521\\
4.625	1.82964587211609\\
4.63000011444092	2.10149669647217\\
4.63500022888184	1.91762363910675\\
4.6399998664856	1.41976082324982\\
4.64499998092651	0.812142193317413\\
4.65000009536743	0.28321561217308\\
4.65500020980835	-0.0758795067667961\\
4.65999984741211	-0.444358050823212\\
4.66499996185303	24.2797203063965\\
4.67000007629395	38.2501029968262\\
4.67500019073486	48.4393501281738\\
4.67999982833862	54.193115234375\\
4.68499994277954	55.8157501220703\\
4.69000005722046	56.4265975952148\\
4.69500017166138	52.8055610656738\\
4.69999980926514	45.3807525634766\\
4.70499992370605	35.5797653198242\\
4.71000003814697	25.715970993042\\
4.71500015258789	18.2493305206299\\
4.71999979019165	14.9382791519165\\
4.72499990463257	18.9272308349609\\
4.73000001907349	25.4819297790527\\
4.7350001335144	32.4592208862305\\
4.73999977111816	37.9643135070801\\
4.74499988555908	40.8027496337891\\
4.75	40.4969062805176\\
4.75500011444092	39.0467720031738\\
4.76000022888184	34.6058921813965\\
4.7649998664856	27.662727355957\\
4.76999998092651	19.3111705780029\\
4.77500009536743	11.0089159011841\\
4.78000020980835	4.12531805038452\\
4.78499984741211	0.548256695270538\\
4.78999996185303	0.111802242696285\\
4.79500007629395	-0.137715727090836\\
4.80000019073486	4.38955640792847\\
4.80499982833862	7.68451070785522\\
4.80999994277954	9.03129863739014\\
4.81500005722046	8.1179666519165\\
4.82000017166138	6.47747230529785\\
4.82499980926514	5.12436294555664\\
4.82999992370605	4.6731276512146\\
4.83500003814697	5.1974778175354\\
4.84000015258789	6.52795124053955\\
4.84499979019165	8.16765213012695\\
4.84999990463257	9.29245948791504\\
4.85500001907349	8.73496913909912\\
4.8600001335144	5.60557222366333\\
4.86499977111816	1.05987405776978\\
4.86999988555908	0.994557201862335\\
4.875	1.50975775718689\\
4.88000011444092	1.90766417980194\\
4.88500022888184	1.77982521057129\\
4.8899998664856	1.30423057079315\\
4.89499998092651	0.676914632320404\\
4.90000009536743	0.225599959492683\\
4.90500020980835	-0.184858396649361\\
4.90999984741211	-0.558672189712524\\
4.91499996185303	-0.839652180671692\\
4.92000007629395	24.127534866333\\
4.92500019073486	41.3754653930664\\
4.92999982833862	52.0046234130859\\
4.93499994277954	56.3593444824219\\
4.94000005722046	56.1657752990723\\
4.94500017166138	54.7745246887207\\
4.94999980926514	49.1452827453613\\
4.95499992370605	40.7769813537598\\
4.96000003814697	31.5541610717773\\
4.96500015258789	23.5540218353271\\
4.96999979019165	18.4683589935303\\
4.97499990463257	18.0247745513916\\
4.98000001907349	21.606502532959\\
4.9850001335144	26.1882991790771\\
4.98999977111816	30.5609645843506\\
4.99499988555908	33.7783699035645\\
5	35.380802154541\\
5.00500011444092	35.2470092773438\\
5.01000022888184	34.0895652770996\\
5.0149998664856	32.0205879211426\\
5.01999998092651	28.8069248199463\\
5.02500009536743	24.8251876831055\\
5.03000020980835	20.554500579834\\
5.03499984741211	16.4060955047607\\
5.03999996185303	12.6991596221924\\
5.04500007629395	9.63882446289063\\
5.05000019073486	7.31756162643433\\
5.05499982833862	5.74905824661255\\
5.05999994277954	4.84062051773071\\
5.06500005722046	4.43496751785278\\
5.07000017166138	4.3194785118103\\
5.07499980926514	4.28481578826904\\
5.07999992370605	4.12775039672852\\
5.08500003814697	3.7543773651123\\
5.09000015258789	3.03475666046143\\
5.09499979019165	1.93065989017487\\
5.09999990463257	0.575486361980438\\
5.10500001907349	0.14531509578228\\
5.1100001335144	0.06791652739048\\
5.11499977111816	0.0344408340752125\\
5.11999988555908	0.0189618337899446\\
5.125	0.0125566981732845\\
5.13000011444092	0.00832671765238047\\
5.13500022888184	0.775099217891693\\
5.1399998664856	1.21872496604919\\
5.14499998092651	1.0127956867218\\
5.15000009536743	0.643207967281342\\
5.15500020980835	0.25571808218956\\
5.15999984741211	10.8681688308716\\
5.16499996185303	18.9680976867676\\
5.17000007629395	25.6079845428467\\
5.17500019073486	30.220251083374\\
5.17999982833862	32.6710472106934\\
5.18499994277954	33.2742042541504\\
5.19000005722046	32.6604804992676\\
5.19500017166138	31.3494262695313\\
5.19999980926514	29.087287902832\\
5.20499992370605	26.3981285095215\\
5.21000003814697	23.7865161895752\\
5.21500015258789	21.6536407470703\\
5.21999979019165	20.2546463012695\\
5.22499990463257	19.828670501709\\
5.23000001907349	20.224739074707\\
5.2350001335144	20.9118576049805\\
5.23999977111816	21.6621284484863\\
5.24499988555908	22.3290596008301\\
5.25	22.7779140472412\\
5.25500011444092	22.9534721374512\\
5.26000022888184	23.0717792510986\\
5.2649998664856	22.872859954834\\
5.26999998092651	22.2806777954102\\
5.27500009536743	21.2984428405762\\
5.28000020980835	19.9703559875488\\
5.28499984741211	18.40407371521\\
5.28999996185303	16.7321968078613\\
5.29500007629395	15.1142253875732\\
5.30000019073486	13.7357778549194\\
5.30499982833862	12.7358255386353\\
5.30999994277954	12.2569179534912\\
5.31500005722046	12.2325658798218\\
5.32000017166138	13.1184434890747\\
5.32499980926514	14.1636247634888\\
5.32999992370605	15.1999998092651\\
5.33500003814697	16.0224151611328\\
5.34000015258789	16.4969749450684\\
5.34499979019165	16.7577228546143\\
5.34999990463257	16.4454078674316\\
5.35500001907349	15.5010938644409\\
5.3600001335144	13.9838676452637\\
5.36499977111816	12.0729837417603\\
5.36999988555908	9.99202060699463\\
5.375	8.11653709411621\\
5.38000011444092	6.62144422531128\\
5.38500022888184	5.71392822265625\\
5.3899998664856	5.46760368347168\\
5.39499998092651	5.88974714279175\\
5.40000009536743	6.26717138290405\\
5.40500020980835	6.31803750991821\\
5.40999984741211	6.1794867515564\\
5.41499996185303	5.44764137268066\\
5.42000007629395	4.03154945373535\\
5.42500019073486	1.99951171875\\
5.42999982833862	0.132788434624672\\
5.43499994277954	0.0213130302727222\\
5.44000005722046	-0.0370443128049374\\
5.44500017166138	-0.0463053472340107\\
5.44999980926514	-0.0250369366258383\\
5.45499992370605	-0.0158334113657475\\
5.46000003814697	-0.0107721760869026\\
5.46500015258789	-0.0076374695636332\\
5.46999979019165	-0.00591436447575688\\
5.47499990463257	-0.00370582984760404\\
5.48000001907349	-0.00454524299129844\\
5.4850001335144	-0.00242194323800504\\
5.48999977111816	-0.00270848022773862\\
5.49499988555908	-0.00214443262666464\\
5.5	-0.00182794616557658\\
5.50500011444092	-0.0012549915118143\\
5.51000022888184	-0.000756140158046037\\
5.5149998664856	-0.0012145988876\\
5.51999998092651	0.0270820595324039\\
5.52500009536743	0.196280211210251\\
5.53000020980835	0.172088876366615\\
5.53499984741211	0.146937161684036\\
5.53999996185303	0.119982495903969\\
5.54500007629395	0.093104213476181\\
5.55000019073486	0.0641174763441086\\
5.55499982833862	0.0384846068918705\\
5.55999994277954	0.0202769692987204\\
5.56500005722046	0.00446047401055694\\
5.57000017166138	-0.0138680962845683\\
5.57499980926514	-0.0332833677530289\\
5.57999992370605	-0.0491134636104107\\
5.58500003814697	-0.0633014887571335\\
5.59000015258789	-0.0766042396426201\\
5.59499979019165	-0.0809559151530266\\
5.59999990463257	-0.0861179083585739\\
5.60500001907349	-0.0854653865098953\\
5.6100001335144	-0.0778684392571449\\
5.61499977111816	-0.069023035466671\\
5.61999988555908	-0.059622947126627\\
5.625	-0.0471202619373798\\
5.63000011444092	-0.0340823866426945\\
5.63500022888184	-0.0275713168084621\\
5.6399998664856	-0.0168197024613619\\
5.64499998092651	-0.00866260658949614\\
5.65000009536743	-0.00558374263346195\\
5.65500020980835	-0.0039819679223001\\
5.65999984741211	-0.00329370913095772\\
5.66499996185303	-0.0024597889278084\\
5.67000007629395	0.473280519247055\\
5.67500019073486	11.3017673492432\\
5.67999982833862	14.2959480285645\\
5.68499994277954	14.8773145675659\\
5.69000005722046	14.4732294082642\\
5.69500017166138	13.1564092636108\\
5.69999980926514	11.2131328582764\\
5.70499992370605	9.31390857696533\\
5.71000003814697	7.96051692962646\\
5.71500015258789	7.61712265014648\\
5.71999979019165	8.12915897369385\\
5.72499990463257	8.83845138549805\\
5.73000001907349	9.44030570983887\\
5.7350001335144	9.76884269714355\\
5.73999977111816	9.75010967254639\\
5.74499988555908	9.39567184448242\\
5.75	8.79035091400146\\
5.75500011444092	8.05636024475098\\
5.76000022888184	7.28612470626831\\
5.7649998664856	6.52540397644043\\
5.76999998092651	5.71157836914063\\
5.77500009536743	4.82479000091553\\
5.78000020980835	3.74491477012634\\
5.78499984741211	2.42674708366394\\
5.78999996185303	0.853656470775604\\
5.79500007629395	-1.00401306152344\\
5.80000019073486	-3.1225700378418\\
5.80499982833862	-5.43722057342529\\
5.80999994277954	-7.87840890884399\\
5.81500005722046	-10.3403759002686\\
5.82000017166138	-12.6900901794434\\
5.82499980926514	-14.7619667053223\\
5.82999992370605	-16.4013957977295\\
5.83500003814697	-17.4265842437744\\
5.84000015258789	-17.6711082458496\\
5.84499979019165	-17.0069484710693\\
5.84999990463257	-15.3868370056152\\
5.85500001907349	-12.9747829437256\\
5.8600001335144	-9.59656810760498\\
5.86499977111816	-5.59067153930664\\
5.86999988555908	-1.46178686618805\\
5.875	1.95430386066437\\
5.88000011444092	3.76791143417358\\
5.88500022888184	3.26285314559937\\
5.8899998664856	0.818187952041626\\
5.89499998092651	0.340358078479767\\
5.90000009536743	0.148847863078117\\
5.90500020980835	0.177524119615555\\
5.90999984741211	0.451752364635468\\
5.91499996185303	0.890391528606415\\
5.92000007629395	1.26129603385925\\
5.92500019073486	1.42563664913177\\
5.92999982833862	1.38619303703308\\
5.93499994277954	1.17035043239594\\
5.94000005722046	0.861199676990509\\
5.94500017166138	0.536525905132294\\
5.94999980926514	0.247257918119431\\
5.95499992370605	0.0274937003850937\\
5.96000003814697	-0.0713648796081543\\
5.96500015258789	-0.106096714735031\\
5.96999979019165	-0.153068259358406\\
5.97499990463257	-0.199777126312256\\
5.98000001907349	42.7339859008789\\
5.9850001335144	77.7530899047852\\
5.98999977111816	98.504524230957\\
5.99499988555908	106.768280029297\\
6	106.913612365723\\
6.00500011444092	104.616539001465\\
6.01000022888184	100.294982910156\\
6.0149998664856	91.1715393066406\\
6.01999998092651	77.7174987792969\\
6.02500009536743	61.6893463134766\\
6.03000020980835	45.8016242980957\\
6.03499984741211	33.1025466918945\\
6.03999996185303	26.4722270965576\\
6.04500007629395	30.0297946929932\\
6.05000019073486	39.4646377563477\\
6.05499982833862	50.3883590698242\\
6.05999994277954	60.4549827575684\\
6.06500005722046	67.9072189331055\\
6.07000017166138	72.0326766967773\\
6.07499980926514	72.6933135986328\\
6.07999992370605	70.9981155395508\\
6.08500003814697	68.4311294555664\\
6.09000015258789	63.272647857666\\
6.09499979019165	56.1026153564453\\
6.09999990463257	47.6245536804199\\
6.10500001907349	38.771167755127\\
6.1100001335144	30.6963920593262\\
6.11499977111816	24.1715679168701\\
6.11999988555908	19.8404026031494\\
6.125	17.8638172149658\\
6.13000011444092	18.9434394836426\\
6.13500022888184	20.907922744751\\
6.1399998664856	22.7649097442627\\
6.14499998092651	23.9019107818604\\
6.15000009536743	23.9713573455811\\
6.15500020980835	23.3520221710205\\
6.15999984741211	21.8268947601318\\
6.16499996185303	19.0776309967041\\
6.17000007629395	15.2827978134155\\
6.17500019073486	10.7791996002197\\
6.17999982833862	6.01427412033081\\
6.18499994277954	1.48545491695404\\
6.19000005722046	0.442382395267487\\
6.19500017166138	0.155691236257553\\
6.19999980926514	-0.0111101269721985\\
6.20499992370605	-0.106327094137669\\
6.21000003814697	-0.162985742092133\\
6.21500015258789	-0.150624319911003\\
6.21999979019165	-0.0675453990697861\\
6.22499990463257	-0.0364178568124771\\
6.23000001907349	-0.0196784120053053\\
6.2350001335144	-0.0104360803961754\\
6.23999977111816	-0.00592547655105591\\
6.24499988555908	-0.0026709265075624\\
6.25	-0.000780850881710649\\
6.25500011444092	0.000281698652543128\\
6.26000022888184	0.00083491945406422\\
6.2649998664856	0.00118900346569717\\
6.26999998092651	0.00135879125446081\\
6.27500009536743	0.0013517199549824\\
6.28000020980835	0.00130305299535394\\
6.28499984741211	0.00125363527331501\\
6.28999996185303	0.00121445721015334\\
6.29500007629395	0.00114622584078461\\
6.30000019073486	0.00104422471486032\\
6.30499982833862	0.000919490645173937\\
6.30999994277954	0.00081764004426077\\
6.31500005722046	0.000712864042725414\\
6.32000017166138	0.000590144365560263\\
6.32499980926514	0.000458963419077918\\
6.32999992370605	0.000364296138286591\\
6.33500003814697	0.000374853756511584\\
6.34000015258789	0.000483185081975535\\
6.34499979019165	0.000463474512798712\\
6.34999990463257	0.000349229056155309\\
6.35500001907349	0.000174137501744553\\
6.3600001335144	5.19524908065796\\
6.36499977111816	7.7250599861145\\
6.36999988555908	9.47944831848145\\
6.375	10.4135456085205\\
6.38000011444092	10.6077661514282\\
6.38500022888184	10.2839221954346\\
6.3899998664856	10.6680946350098\\
6.39499998092651	9.71707344055176\\
6.40000009536743	8.89585781097412\\
6.40500020980835	8.67883491516113\\
6.40999984741211	9.67912578582764\\
6.41499996185303	11.4367275238037\\
6.42000007629395	13.8000783920288\\
6.42500019073486	16.5077857971191\\
6.42999982833862	19.296688079834\\
6.43499994277954	21.9571990966797\\
6.44000005722046	24.3025989532471\\
6.44500017166138	26.2415027618408\\
6.44999980926514	27.7267093658447\\
6.45499992370605	28.8002376556396\\
6.46000003814697	29.5073890686035\\
6.46500015258789	29.9370079040527\\
6.46999979019165	30.196870803833\\
6.47499990463257	30.3663463592529\\
6.48000001907349	30.5195426940918\\
6.4850001335144	30.722038269043\\
6.48999977111816	31.003885269165\\
6.49499988555908	31.370288848877\\
6.5	31.8116970062256\\
6.50500011444092	32.2991333007813\\
6.51000022888184	32.7890892028809\\
6.5149998664856	33.2449531555176\\
6.51999998092651	33.6303520202637\\
6.52500009536743	33.9108848571777\\
6.53000020980835	34.0638389587402\\
6.53499984741211	34.0991706848145\\
6.53999996185303	34.0079460144043\\
6.54500007629395	33.8001708984375\\
6.55000019073486	33.4918060302734\\
6.55499982833862	33.1133308410645\\
6.55999994277954	32.6961975097656\\
6.56500005722046	32.2235221862793\\
6.57000017166138	31.699535369873\\
6.57499980926514	31.1314868927002\\
6.57999992370605	30.5656967163086\\
6.58500003814697	29.9842014312744\\
6.59000015258789	29.4008274078369\\
6.59499979019165	28.8234882354736\\
6.59999990463257	28.2445793151855\\
6.60500001907349	27.6604194641113\\
6.6100001335144	27.0729179382324\\
6.61499977111816	26.4844379425049\\
6.61999988555908	25.8754920959473\\
6.625	25.2595195770264\\
6.63000011444092	24.6301670074463\\
6.63500022888184	23.9915885925293\\
6.6399998664856	23.3589477539063\\
6.64499998092651	22.7225818634033\\
6.65000009536743	22.0946197509766\\
6.65500020980835	21.4832096099854\\
6.65999984741211	20.8845634460449\\
6.66499996185303	20.3095951080322\\
6.67000007629395	19.7758941650391\\
6.67500019073486	19.2738227844238\\
6.67999982833862	18.8201751708984\\
6.68499994277954	18.4140644073486\\
6.69000005722046	18.0424480438232\\
6.69500017166138	17.7420673370361\\
6.69999980926514	17.5320320129395\\
6.70499992370605	17.3700332641602\\
6.71000003814697	17.2544593811035\\
6.71500015258789	17.170877456665\\
6.71999979019165	17.1256885528564\\
6.72499990463257	17.1171779632568\\
6.73000001907349	17.1333656311035\\
6.7350001335144	17.1803245544434\\
6.73999977111816	17.2582511901855\\
6.74499988555908	17.3644027709961\\
6.75	17.5081310272217\\
6.75500011444092	17.6795291900635\\
6.76000022888184	17.8594703674316\\
6.7649998664856	18.0841217041016\\
6.76999998092651	18.3287754058838\\
6.77500009536743	18.6081066131592\\
6.78000020980835	18.9154853820801\\
6.78499984741211	19.2440299987793\\
6.78999996185303	19.5969982147217\\
6.79500007629395	19.9678955078125\\
6.80000019073486	20.3505477905273\\
6.80499982833862	20.7402992248535\\
6.80999994277954	21.1390953063965\\
6.81500005722046	21.5335025787354\\
6.82000017166138	21.9268093109131\\
6.82499980926514	22.3146781921387\\
6.82999992370605	22.6918334960938\\
6.83500003814697	23.0543632507324\\
6.84000015258789	23.3982067108154\\
6.84499979019165	23.7361316680908\\
6.84999990463257	24.0563640594482\\
6.85500001907349	24.3582153320313\\
6.8600001335144	24.6410694122314\\
6.86499977111816	24.8953876495361\\
6.86999988555908	25.1198329925537\\
6.875	25.3145866394043\\
6.88000011444092	25.4652900695801\\
6.88500022888184	25.5673923492432\\
6.8899998664856	25.6135540008545\\
6.89499998092651	25.5969219207764\\
6.90000009536743	25.5166721343994\\
6.90500020980835	25.3709983825684\\
6.90999984741211	25.1593589782715\\
6.91499996185303	24.8860836029053\\
6.92000007629395	24.5571613311768\\
6.92500019073486	24.1788711547852\\
6.92999982833862	23.758903503418\\
6.93499994277954	23.3057136535645\\
6.94000005722046	22.8259334564209\\
6.94500017166138	22.3242378234863\\
6.94999980926514	21.8039703369141\\
6.95499992370605	21.2665042877197\\
6.96000003814697	20.7136783599854\\
6.96500015258789	20.1481704711914\\
6.96999979019165	19.5673389434814\\
6.97499990463257	18.9641056060791\\
6.98000001907349	18.346118927002\\
6.9850001335144	17.7076416015625\\
6.98999977111816	17.0498809814453\\
6.99499988555908	16.3773899078369\\
7	15.6947555541992\\
7.00500011444092	15.0063753128052\\
7.01000022888184	14.3180141448975\\
7.0149998664856	13.6371650695801\\
7.01999998092651	12.9696531295776\\
7.02500009536743	12.3192024230957\\
7.03000020980835	11.6889038085938\\
7.03499984741211	11.0816116333008\\
7.03999996185303	10.4993276596069\\
7.04500007629395	9.94294452667236\\
7.05000019073486	9.41251277923584\\
7.05499982833862	8.90713787078857\\
7.05999994277954	8.42541885375977\\
7.06500005722046	7.96609258651733\\
7.07000017166138	7.52795696258545\\
7.07499980926514	7.11062049865723\\
7.07999992370605	6.71419954299927\\
7.08500003814697	6.33942461013794\\
7.09000015258789	5.9881420135498\\
7.09499979019165	5.66213464736938\\
7.09999990463257	5.36370611190796\\
7.10500001907349	5.09503173828125\\
7.1100001335144	4.85841417312622\\
7.11499977111816	4.65489625930786\\
7.11999988555908	4.48656368255615\\
7.125	4.35344982147217\\
7.13000011444092	4.25172281265259\\
7.13500022888184	4.18104267120361\\
7.1399998664856	4.13973474502563\\
7.14499998092651	4.12372064590454\\
7.15000009536743	4.13265371322632\\
7.15500020980835	4.1726508140564\\
7.15999984741211	4.22841787338257\\
7.16499996185303	4.29186391830444\\
7.17000007629395	4.37343072891235\\
7.17500019073486	4.4707236289978\\
7.17999982833862	4.58427858352661\\
7.18499994277954	4.71699142456055\\
7.19000005722046	4.86706972122192\\
7.19500017166138	5.03457450866699\\
7.19999980926514	5.21880006790161\\
7.20499992370605	5.41605949401855\\
7.21000003814697	5.6266975402832\\
7.21500015258789	5.84740829467773\\
7.21999979019165	6.07329082489014\\
7.22499990463257	6.30556774139404\\
7.23000001907349	6.54165172576904\\
7.2350001335144	6.77979612350464\\
7.23999977111816	7.01955699920654\\
7.24499988555908	7.25872230529785\\
7.25	7.49620199203491\\
7.25500011444092	7.73234128952026\\
7.26000022888184	7.9645094871521\\
7.2649998664856	8.19321441650391\\
7.26999998092651	8.41896820068359\\
7.27500009536743	8.6392650604248\\
7.28000020980835	8.85254859924316\\
7.28499984741211	9.05906867980957\\
7.28999996185303	9.25735855102539\\
7.29500007629395	9.44783973693848\\
7.30000019073486	9.62678337097168\\
7.30499982833862	9.7954626083374\\
7.30999994277954	9.95253276824951\\
7.31500005722046	10.098237991333\\
7.32000017166138	10.2309293746948\\
7.32499980926514	10.3510646820068\\
7.32999992370605	10.4571781158447\\
7.33500003814697	10.5491333007813\\
7.34000015258789	10.6270942687988\\
7.34499979019165	10.6907148361206\\
7.34999990463257	10.7402229309082\\
7.35500001907349	10.7757329940796\\
7.3600001335144	10.7984504699707\\
7.36499977111816	10.8085880279541\\
7.36999988555908	10.8069133758545\\
7.375	10.7944173812866\\
7.38000011444092	10.7716884613037\\
7.38500022888184	10.7358560562134\\
7.3899998664856	10.6851940155029\\
7.39499998092651	10.6203460693359\\
7.40000009536743	10.5446662902832\\
7.40500020980835	10.4573125839233\\
7.40999984741211	10.3596038818359\\
7.41499996185303	10.2522172927856\\
7.42000007629395	10.1360015869141\\
7.42500019073486	10.0119714736938\\
7.42999982833862	9.88076019287109\\
7.43499994277954	9.74266052246094\\
7.44000005722046	9.59853649139404\\
7.44500017166138	9.44927406311035\\
7.44999980926514	9.29554176330566\\
7.45499992370605	9.13864040374756\\
7.46000003814697	8.97997570037842\\
7.46500015258789	8.82046604156494\\
7.46999979019165	8.66034317016602\\
7.47499990463257	8.50043964385986\\
7.48000001907349	8.34142112731934\\
7.4850001335144	8.18381309509277\\
7.48999977111816	8.02810859680176\\
7.49499988555908	7.87482500076294\\
7.5	7.72427558898926\\
7.50500011444092	7.57668209075928\\
7.51000022888184	7.432532787323\\
7.5149998664856	7.29307508468628\\
7.51999998092651	7.15907764434814\\
7.52500009536743	7.03269243240356\\
7.53000020980835	6.91067695617676\\
7.53499984741211	6.79636001586914\\
7.53999996185303	6.68872880935669\\
7.54500007629395	6.58969163894653\\
7.55000019073486	6.49865341186523\\
7.55499982833862	6.41547346115112\\
7.55999994277954	6.34052801132202\\
7.56500005722046	6.27367496490479\\
7.57000017166138	6.21454811096191\\
7.57499980926514	6.16313552856445\\
7.57999992370605	6.11939477920532\\
7.58500003814697	6.08321285247803\\
7.59000015258789	6.05469751358032\\
7.59499979019165	6.03384971618652\\
7.59999990463257	6.02049875259399\\
7.60500001907349	6.01485633850098\\
7.6100001335144	6.01839351654053\\
7.61499977111816	6.03269529342651\\
7.61999988555908	6.05507516860962\\
7.625	6.08234214782715\\
7.63000011444092	6.11530160903931\\
7.63500022888184	6.15409135818481\\
7.6399998664856	6.19860172271729\\
7.64499998092651	6.24881362915039\\
7.65000009536743	6.30346965789795\\
7.65500020980835	6.36219167709351\\
7.65999984741211	6.42482757568359\\
7.66499996185303	6.49086475372314\\
7.67000007629395	6.55977916717529\\
7.67500019073486	6.63137817382813\\
7.67999982833862	6.70573472976685\\
7.68499994277954	6.78259420394897\\
7.69000005722046	6.86177349090576\\
7.69500017166138	6.94221210479736\\
7.69999980926514	7.02350664138794\\
7.70499992370605	7.10545110702515\\
7.71000003814697	7.18796157836914\\
7.71500015258789	7.27166938781738\\
7.71999979019165	7.35544633865356\\
7.72499990463257	7.43791961669922\\
7.73000001907349	7.51883983612061\\
7.7350001335144	7.59765911102295\\
7.73999977111816	7.6755199432373\\
7.74499988555908	7.75164270401001\\
7.75	7.82429456710815\\
7.75500011444092	7.89427328109741\\
7.76000022888184	7.96193265914917\\
7.7649998664856	8.02647590637207\\
7.76999998092651	8.08763122558594\\
7.77500009536743	8.14555644989014\\
7.78000020980835	8.20020961761475\\
7.78499984741211	8.2513256072998\\
7.78999996185303	8.29913711547852\\
7.79500007629395	8.3436393737793\\
7.80000019073486	8.38442039489746\\
7.80499982833862	8.42163753509521\\
7.80999994277954	8.45530033111572\\
7.81500005722046	8.48523235321045\\
7.82000017166138	8.51141262054443\\
7.82499980926514	8.53390502929688\\
7.82999992370605	8.55285167694092\\
7.83500003814697	8.56843376159668\\
7.84000015258789	8.5805196762085\\
7.84499979019165	8.58918952941895\\
7.84999990463257	8.59486103057861\\
7.85500001907349	8.59743976593018\\
7.8600001335144	8.5970287322998\\
7.86499977111816	8.59389591217041\\
7.86999988555908	8.58810615539551\\
7.875	8.5797872543335\\
7.88000011444092	8.5694694519043\\
7.88500022888184	8.55738830566406\\
7.8899998664856	8.5437068939209\\
7.89499998092651	8.5285005569458\\
7.90000009536743	8.51180648803711\\
7.90500020980835	8.49388408660889\\
7.90999984741211	8.47483253479004\\
7.91499996185303	8.45461463928223\\
7.92000007629395	8.43341445922852\\
7.92500019073486	8.41142177581787\\
7.92999982833862	8.39007663726807\\
7.93499994277954	8.36854839324951\\
7.94000005722046	8.34687519073486\\
7.94500017166138	8.32661247253418\\
7.94999980926514	8.30726718902588\\
7.95499992370605	8.28866004943848\\
7.96000003814697	8.27171993255615\\
7.96500015258789	8.25666046142578\\
7.96999979019165	8.24303722381592\\
7.97499990463257	8.23150253295898\\
7.98000001907349	8.22305679321289\\
7.9850001335144	8.21700763702393\\
7.98999977111816	8.21340370178223\\
7.99499988555908	8.21245288848877\\
8	8.21395778656006\\
8.00500011444092	8.21794605255127\\
8.01000022888184	8.22535037994385\\
8.01500034332275	8.23606014251709\\
8.02000045776367	8.25023555755615\\
8.02499961853027	8.26780033111572\\
8.02999973297119	8.28882789611816\\
8.03499984741211	8.31350708007813\\
8.03999996185303	8.34182834625244\\
8.04500007629395	8.37373638153076\\
8.05000019073486	8.40938663482666\\
8.05500030517578	8.44878578186035\\
8.0600004196167	8.49177169799805\\
8.0649995803833	8.53844738006592\\
8.06999969482422	8.588791847229\\
8.07499980926514	8.64311408996582\\
8.07999992370605	8.70118427276611\\
8.08500003814697	8.76343727111816\\
8.09000015258789	8.82904624938965\\
8.09500026702881	8.89816379547119\\
8.10000038146973	8.97082710266113\\
8.10499954223633	9.04709720611572\\
8.10999965667725	9.12699699401855\\
8.11499977111816	9.21044635772705\\
8.11999988555908	9.29740905761719\\
8.125	9.38773727416992\\
8.13000011444092	9.48146533966064\\
8.13500022888184	9.57853698730469\\
8.14000034332275	9.67893123626709\\
8.14500045776367	9.78258419036865\\
8.14999961853027	9.8894510269165\\
8.15499973297119	9.99924564361572\\
8.15999984741211	10.1120376586914\\
8.16499996185303	10.227710723877\\
8.17000007629395	10.3464984893799\\
8.17500019073486	10.4684352874756\\
8.18000030517578	10.5933599472046\\
8.1850004196167	10.7211360931396\\
8.1899995803833	10.8515739440918\\
8.19499969482422	10.9848461151123\\
8.19999980926514	11.1210041046143\\
8.20499992370605	11.2600908279419\\
8.21000003814697	11.4018392562866\\
8.21500015258789	11.5459222793579\\
8.22000026702881	11.6921033859253\\
8.22500038146973	11.8404912948608\\
8.22999954223633	11.9915895462036\\
8.23499965667725	12.1450090408325\\
8.23999977111816	12.3002977371216\\
8.24499988555908	12.4580383300781\\
8.25	12.6183280944824\\
8.25500011444092	12.7815866470337\\
8.26000022888184	12.9474048614502\\
8.26500034332275	13.1160268783569\\
8.27000045776367	13.2869243621826\\
8.27499961853027	13.4613361358643\\
8.27999973297119	13.6373882293701\\
8.28499984741211	13.8158779144287\\
8.28999996185303	13.9968729019165\\
8.29500007629395	14.1792669296265\\
8.30000019073486	14.3626394271851\\
8.30500030517578	14.5466623306274\\
8.3100004196167	14.7321004867554\\
8.3149995803833	14.92711353302\\
8.31999969482422	15.1220874786377\\
8.32499980926514	15.3175830841064\\
8.32999992370605	15.5167036056519\\
8.33500003814697	15.7183904647827\\
8.34000015258789	15.9220323562622\\
8.34500026702881	16.127758026123\\
8.35000038146973	16.3358097076416\\
8.35499954223633	16.5441074371338\\
8.35999965667725	16.7558670043945\\
8.36499977111816	16.9730949401855\\
8.36999988555908	17.1864223480225\\
8.375	17.401330947876\\
8.38000011444092	17.6168174743652\\
8.38500022888184	17.8316917419434\\
8.39000034332275	18.0466194152832\\
8.39500045776367	18.2585945129395\\
8.39999961853027	18.473237991333\\
8.40499973297119	18.690221786499\\
8.40999984741211	18.9006042480469\\
8.41499996185303	19.1107196807861\\
8.42000007629395	19.3167152404785\\
8.42500019073486	19.5199775695801\\
8.43000030517578	19.7203845977783\\
8.4350004196167	19.9175415039063\\
8.4399995803833	20.111909866333\\
8.44499969482422	20.3006057739258\\
8.44999980926514	20.4814319610596\\
8.45499992370605	20.6544609069824\\
8.46000003814697	20.8184223175049\\
8.46500015258789	20.9702129364014\\
8.47000026702881	21.1070461273193\\
8.47500038146973	21.2283897399902\\
8.47999954223633	21.3348159790039\\
8.48499965667725	21.4275608062744\\
8.48999977111816	21.5028820037842\\
8.49499988555908	21.5650215148926\\
8.5	21.6161117553711\\
8.50500011444092	21.6525344848633\\
8.51000022888184	21.6877689361572\\
8.51500034332275	21.7225399017334\\
8.52000045776367	21.7533397674561\\
8.52499961853027	21.8056144714355\\
8.52999973297119	21.8600921630859\\
8.53499984741211	21.9363117218018\\
8.53999996185303	22.0325183868408\\
8.54500007629395	22.1564598083496\\
8.55000019073486	22.3110370635986\\
8.55500030517578	22.4993019104004\\
8.5600004196167	22.7320861816406\\
8.5649995803833	23.005615234375\\
8.56999969482422	23.3202838897705\\
8.57499980926514	23.6768493652344\\
8.57999992370605	24.0594387054443\\
8.58500003814697	24.4660606384277\\
8.59000015258789	24.8861484527588\\
8.59500026702881	25.2846946716309\\
8.60000038146973	25.645471572876\\
8.60499954223633	25.9652099609375\\
8.60999965667725	26.2351951599121\\
8.61499977111816	26.4584884643555\\
8.61999988555908	26.6412830352783\\
8.625	26.7843990325928\\
8.63000011444092	26.9004554748535\\
8.63500022888184	26.991828918457\\
8.64000034332275	27.067569732666\\
8.64500045776367	27.1233730316162\\
8.64999961853027	27.1763305664063\\
8.65499973297119	27.1814994812012\\
8.65999984741211	27.2480144500732\\
8.66499996185303	27.2361927032471\\
8.67000007629395	27.2032642364502\\
8.67500019073486	27.1197452545166\\
8.68000030517578	27.0077686309814\\
8.6850004196167	26.8468894958496\\
8.6899995803833	26.6273155212402\\
8.69499969482422	26.3570556640625\\
8.69999980926514	26.0479354858398\\
8.70499992370605	25.7031211853027\\
8.71000003814697	25.3209133148193\\
8.71500015258789	24.9044437408447\\
8.72000026702881	24.4902496337891\\
8.72500038146973	24.0700855255127\\
8.72999954223633	23.6499309539795\\
8.73499965667725	23.2419281005859\\
8.73999977111816	22.8524570465088\\
8.74499988555908	22.4760589599609\\
8.75	22.1061763763428\\
8.75500011444092	21.7137355804443\\
8.76000022888184	21.3420219421387\\
8.76500034332275	20.9537391662598\\
8.77000045776367	20.5607604980469\\
8.77499961853027	20.156888961792\\
8.77999973297119	19.748083114624\\
8.78499984741211	19.3318576812744\\
8.78999996185303	18.9140720367432\\
8.79500007629395	18.5016479492188\\
8.80000019073486	18.1011848449707\\
8.80500030517578	17.718656539917\\
8.8100004196167	17.3600196838379\\
8.8149995803833	17.0305423736572\\
8.81999969482422	16.7305660247803\\
8.82499980926514	16.4616146087646\\
8.82999992370605	16.2227306365967\\
8.83500003814697	16.0080909729004\\
8.84000015258789	15.813157081604\\
8.84500026702881	15.6353635787964\\
8.85000038146973	15.4664115905762\\
8.85499954223633	15.3082408905029\\
8.85999965667725	15.1574831008911\\
8.86499977111816	15.0145044326782\\
8.86999988555908	14.8808784484863\\
8.875	14.7599210739136\\
8.88000011444092	14.6557006835938\\
8.88500022888184	14.5720138549805\\
8.89000034332275	14.5121603012085\\
8.89500045776367	14.4765071868896\\
8.89999961853027	14.4650783538818\\
8.90499973297119	14.4825754165649\\
8.90999984741211	14.5203046798706\\
8.91499996185303	14.5659608840942\\
8.92000007629395	14.6260890960693\\
8.92500019073486	14.7015447616577\\
8.93000030517578	14.7891254425049\\
8.9350004196167	14.8799953460693\\
8.9399995803833	14.9581527709961\\
8.94499969482422	15.0403671264648\\
8.94999980926514	15.1299448013306\\
8.95499992370605	15.2263736724854\\
8.96000003814697	15.3303642272949\\
8.96500015258789	15.4432983398438\\
8.97000026702881	15.5677175521851\\
8.97500038146973	15.6996049880981\\
8.97999954223633	15.8373756408691\\
8.98499965667725	15.9926986694336\\
8.98999977111816	16.1473503112793\\
8.99499988555908	16.2935733795166\\
9	16.450325012207\\
9.00500011444092	16.6067600250244\\
9.01000022888184	16.7602577209473\\
9.01500034332275	16.9095630645752\\
9.02000045776367	17.0550975799561\\
9.02499961853027	17.1957855224609\\
9.02999973297119	17.3308048248291\\
9.03499984741211	17.4601249694824\\
9.03999996185303	17.5834732055664\\
9.04500007629395	17.7005558013916\\
9.05000019073486	17.8111190795898\\
9.05500030517578	17.9149208068848\\
9.0600004196167	18.0112361907959\\
9.0649995803833	18.0997581481934\\
9.06999969482422	18.1798801422119\\
9.07499980926514	18.2510986328125\\
9.07999992370605	18.3130970001221\\
9.08500003814697	18.3655319213867\\
9.09000015258789	18.407434463501\\
9.09500026702881	18.4365081787109\\
9.10000038146973	18.4520645141602\\
9.10499954223633	18.4605445861816\\
9.10999965667725	18.4595279693604\\
9.11499977111816	18.4460296630859\\
9.11999988555908	18.4190120697021\\
9.125	18.3796768188477\\
9.13000011444092	18.3290309906006\\
9.13500022888184	18.2685794830322\\
9.14000034332275	18.203706741333\\
9.14500045776367	18.133279800415\\
9.14999961853027	18.0605449676514\\
9.15499973297119	17.9859561920166\\
9.15999984741211	17.9096240997314\\
9.16499996185303	17.8309154510498\\
9.17000007629395	17.7489910125732\\
9.17500019073486	17.6632957458496\\
9.18000030517578	17.5729560852051\\
9.1850004196167	17.478084564209\\
9.1899995803833	17.3782958984375\\
9.19499969482422	17.2734127044678\\
9.19999980926514	17.1643600463867\\
9.20499992370605	17.0523872375488\\
9.21000003814697	16.9394607543945\\
9.21500015258789	16.8270320892334\\
9.22000026702881	16.7150554656982\\
9.22500038146973	16.6050987243652\\
9.22999954223633	16.4971237182617\\
9.23499965667725	16.3916072845459\\
9.23999977111816	16.2899417877197\\
9.24499988555908	16.1950664520264\\
9.25	16.1095848083496\\
9.25500011444092	16.0352249145508\\
9.26000022888184	15.9788522720337\\
9.26500034332275	15.9512739181519\\
9.27000045776367	15.9835720062256\\
9.27499961853027	16.0648097991943\\
9.27999973297119	16.1979236602783\\
9.28499984741211	16.3816108703613\\
9.28999996185303	16.6171188354492\\
9.29500007629395	16.9025859832764\\
9.30000019073486	17.2342777252197\\
9.30500030517578	17.6140155792236\\
9.3100004196167	18.0264129638672\\
9.3149995803833	18.4774875640869\\
9.31999969482422	18.9528865814209\\
9.32499980926514	19.4475383758545\\
9.32999992370605	19.9450550079346\\
9.33500003814697	20.4322929382324\\
9.34000015258789	20.8857536315918\\
9.34500026702881	21.3003711700439\\
9.35000038146973	21.6314067840576\\
9.35499954223633	21.861385345459\\
9.35999965667725	21.9721927642822\\
9.36499977111816	21.9265155792236\\
9.36999988555908	21.7323837280273\\
9.375	21.3946418762207\\
9.38000011444092	20.9260692596436\\
9.38500022888184	20.3456001281738\\
9.39000034332275	19.7126369476318\\
9.39500045776367	19.0333633422852\\
9.39999961853027	18.3830528259277\\
9.40499973297119	17.7766666412354\\
9.40999984741211	17.2307891845703\\
9.41499996185303	16.7316436767578\\
9.42000007629395	16.235179901123\\
9.42500019073486	15.6501340866089\\
9.43000030517578	14.8532962799072\\
9.4350004196167	13.7266893386841\\
9.4399995803833	12.0794172286987\\
9.44499969482422	9.77545356750488\\
9.44999980926514	6.69163608551025\\
9.45499992370605	2.7709538936615\\
9.46000003814697	-1.90510213375092\\
9.46500015258789	-7.16078948974609\\
9.47000026702881	-12.5691375732422\\
9.47500038146973	-17.6432209014893\\
9.47999954223633	-21.8539524078369\\
9.48499965667725	-24.8599987030029\\
9.48999977111816	-25.0023612976074\\
9.49499988555908	-24.6365203857422\\
9.5	-22.830078125\\
9.50500011444092	-19.8685207366943\\
9.51000022888184	-16.1949062347412\\
9.51500034332275	-12.2190179824829\\
9.52000045776367	-8.34224796295166\\
9.52499961853027	-4.81957340240479\\
9.52999973297119	-2.02997088432312\\
9.53499984741211	0.000275024882284924\\
9.53999996185303	1.13845586776733\\
9.54500007629395	1.34863913059235\\
9.55000019073486	0.693692982196808\\
9.55500030517578	0.369869023561478\\
9.5600004196167	0.164935663342476\\
9.5649995803833	0.0762639939785004\\
9.56999969482422	0.0392919629812241\\
9.57499980926514	0.0225928910076618\\
9.57999992370605	0.0132500994950533\\
9.58500003814697	0.0135875632986426\\
9.59000015258789	0.0131561802700162\\
9.59500026702881	0.0115211410447955\\
9.60000038146973	0.00944764539599419\\
9.60499954223633	0.00823883153498173\\
9.60999965667725	0.00908792298287153\\
9.61499977111816	0.00751168373972178\\
9.61999988555908	0.00659400643780828\\
9.625	0.00537666445598006\\
9.63000011444092	0.00539625808596611\\
9.63500022888184	0.00435433676466346\\
9.64000034332275	0.00399593822658062\\
9.64500045776367	0.00422395765781403\\
9.64999961853027	0.00322585785761476\\
9.65499973297119	0.00228681229054928\\
9.65999984741211	0.00243987888097763\\
9.66499996185303	0.00216214382089674\\
9.67000007629395	0.00173151050694287\\
9.67500019073486	0.00156464066822082\\
9.68000030517578	0.00138562126085162\\
9.6850004196167	0.00117114780005068\\
9.6899995803833	0.000954479211941361\\
9.69499969482422	0.000762453186325729\\
9.69999980926514	0.00063822598895058\\
9.70499992370605	0.000602795684244484\\
9.71000003814697	0.000515881169121712\\
9.71500015258789	0.000291447766358033\\
9.72000026702881	0.000187019017175771\\
9.72500038146973	0.000271142169367522\\
9.72999954223633	0.000323212589137256\\
9.73499965667725	0.00032783905044198\\
9.73999977111816	0.000250396318733692\\
9.74499988555908	5.05252028233372e-05\\
9.75	-4.99922805374808e-07\\
9.75500011444092	-8.65983820403926e-05\\
9.76000022888184	-0.000214301777305081\\
9.76500034332275	-0.000304652377963066\\
9.77000045776367	-0.000287622911855578\\
9.77499961853027	-0.000200841008336283\\
9.77999973297119	-3.94080343539827e-05\\
9.78499984741211	-7.06001446815208e-05\\
9.78999996185303	-0.000141494965646416\\
9.79500007629395	-0.000254461396252736\\
9.80000019073486	-0.000327577901771292\\
9.80500030517578	-0.000295570760499686\\
9.8100004196167	-0.000230237244977616\\
9.8149995803833	-0.000132929868414067\\
9.81999969482422	-0.000150361505802721\\
9.82499980926514	-0.000195685919607058\\
9.82999992370605	-0.000270649441517889\\
9.83500003814697	-0.000323359912727028\\
9.84000015258789	-0.000316871184622869\\
9.84500026702881	-0.000295090692816302\\
9.85000038146973	-0.000257314619375393\\
9.85499954223633	-0.00026023224927485\\
9.85999965667725	-0.000258730258792639\\
9.86499977111816	-0.000250708719249815\\
9.86999988555908	-0.000253581907600164\\
9.875	-0.000274907040875405\\
9.88000011444092	-0.000304088403936476\\
9.88500022888184	-0.000333540170686319\\
9.89000034332275	-0.000279282277915627\\
9.89500045776367	-0.000200942988158204\\
9.89999961853027	-0.000103502483398188\\
9.90499973297119	-0.000111553621536586\\
9.90999984741211	-0.000266560789896175\\
9.91499996185303	-0.000464125681901351\\
9.92000007629395	-0.000675384828355163\\
9.92500019073486	-0.000520123809110373\\
9.93000030517578	-0.000293022603727877\\
9.9350004196167	1.87012869901082e-06\\
9.9399995803833	8.45296308398247e-05\\
9.94499969482422	-4.22461671405472e-05\\
9.94999980926514	-0.000149481144035235\\
9.95499992370605	-0.000230369914788753\\
9.96000003814697	-0.000298182712867856\\
9.96500015258789	-0.000360594975063577\\
9.97000026702881	-0.000420865224441513\\
9.97500038146973	-0.00037982503999956\\
9.97999954223633	-0.000332318129949272\\
9.98499965667725	-0.000304613262414932\\
9.98999977111816	-0.00029994803480804\\
9.99499988555908	-0.000280712061794475\\
10	-0.000266343064140528\\
};
\addlegendentry{CF}

\end{axis}
\end{tikzpicture}%
    \end{tikzpicture}}
    \caption{Loads at point B of CF under PD Feedback Control}
    \label{fig:pureFeedbkPDB}
\end{figure}

\begin{figure}[h!]
    \centering
    \scalebox{0.8}{
    \begin{tikzpicture}
            % This file was created by matlab2tikz.
%
%The latest updates can be retrieved from
%  http://www.mathworks.com/matlabcentral/fileexchange/22022-matlab2tikz-matlab2tikz
%where you can also make suggestions and rate matlab2tikz.
%
\begin{tikzpicture}

\begin{axis}[%
width=4.521in,
height=1.476in,
at={(0.758in,2.571in)},
scale only axis,
xmin=0,
xmax=10,
xlabel style={font=\color{white!15!black}},
xlabel={Time (s)},
ymin=-72.4036254882813,
ymax=4308.158203125,
ylabel style={font=\color{white!15!black}},
ylabel={FX (N)},
axis background/.style={fill=white},
xmajorgrids,
ymajorgrids,
legend style={at={(0.85,1)}, anchor=north east, legend cell align=left, align=left, draw=black}
]
\addplot [color=black, dashed, line width=2.0pt]
  table[row sep=crcr]{%
0.0949999988079071	15.1623277664185\\
0.100000001490116	13.2157039642334\\
0.104999996721745	11.5256834030151\\
0.109999999403954	10.0556793212891\\
0.115000002086163	8.77785968780518\\
0.119999997317791	158.447402954102\\
0.125	358.020935058594\\
0.129999995231628	521.468811035156\\
0.135000005364418	652.920227050781\\
0.140000000596046	753.271728515625\\
0.144999995827675	823.419006347656\\
0.150000005960464	867.019470214844\\
0.155000001192093	885.403076171875\\
0.159999996423721	890.36376953125\\
0.165000006556511	876.576110839844\\
0.170000001788139	839.247741699219\\
0.174999997019768	777.953063964844\\
0.180000007152557	694.513488769531\\
0.185000002384186	591.138732910156\\
0.189999997615814	471.741668701172\\
0.194999992847443	493.009094238281\\
0.200000002980232	570.360046386719\\
0.204999998211861	638.454040527344\\
0.209999993443489	781.938659667969\\
0.215000003576279	979.735656738281\\
0.219999998807907	1224.19946289063\\
0.224999994039536	1472.34008789063\\
0.230000004172325	1696.94323730469\\
0.234999999403954	1882.64575195313\\
0.239999994635582	2022.6640625\\
0.245000004768372	2117.45263671875\\
0.25	2171.58032226563\\
0.254999995231628	2201.37768554688\\
0.259999990463257	2230.07397460938\\
0.264999985694885	2228.59057617188\\
0.270000010728836	2191.095703125\\
0.275000005960464	2130.05126953125\\
0.280000001192093	2062.79296875\\
0.284999996423721	2006.10034179688\\
0.28999999165535	1971.14624023438\\
0.294999986886978	1977.01586914063\\
0.300000011920929	2013.06433105469\\
0.305000007152557	2053.74438476563\\
0.310000002384186	2088.15893554688\\
0.314999997615814	2108.2607421875\\
0.319999992847443	2109.01611328125\\
0.324999988079071	2088.35522460938\\
0.330000013113022	2059.52514648438\\
0.33500000834465	2015.52941894531\\
0.340000003576279	1954.208984375\\
0.344999998807907	1876.16955566406\\
0.349999994039536	1786.75439453125\\
0.354999989271164	1693.07141113281\\
0.360000014305115	1601.96484375\\
0.365000009536743	1519.39196777344\\
0.370000004768372	1449.43029785156\\
0.375	1390.95129394531\\
0.379999995231628	1342.04382324219\\
0.384999990463257	1298.375\\
0.389999985694885	1257.76037597656\\
0.395000010728836	1213.79870605469\\
0.400000005960464	1165.91394042969\\
0.405000001192093	1110.99633789063\\
0.409999996423721	1048.81799316406\\
0.41499999165535	980.244506835938\\
0.419999986886978	907.133911132813\\
0.425000011920929	832.867492675781\\
0.430000007152557	759.200622558594\\
0.435000002384186	688.823486328125\\
0.439999997615814	625.377563476563\\
0.444999992847443	570.013610839844\\
0.449999988079071	522.979248046875\\
0.455000013113022	483.813232421875\\
0.46000000834465	450.929260253906\\
0.465000003576279	423.217926025391\\
0.469999998807907	397.779144287109\\
0.474999994039536	374.042541503906\\
0.479999989271164	350.572540283203\\
0.485000014305115	326.85986328125\\
0.490000009536743	302.944793701172\\
0.495000004768372	280.463775634766\\
0.5	259.315734863281\\
0.504999995231628	239.975814819336\\
0.509999990463257	223.675079345703\\
0.514999985694885	211.615051269531\\
0.519999980926514	204.657836914063\\
0.524999976158142	204.553817749023\\
0.529999971389771	212.891464233398\\
0.535000026226044	224.554046630859\\
0.540000021457672	238.645462036133\\
0.545000016689301	254.294036865234\\
0.550000011920929	271.092926025391\\
0.555000007152557	289.506011962891\\
0.560000002384186	309.199584960938\\
0.564999997615814	329.985412597656\\
0.569999992847443	351.911193847656\\
0.574999988079071	375.092620849609\\
0.579999983310699	399.710418701172\\
0.584999978542328	425.932861328125\\
0.589999973773956	453.812591552734\\
0.595000028610229	483.300048828125\\
0.600000023841858	514.275268554688\\
0.605000019073486	546.546447753906\\
0.610000014305115	579.823974609375\\
0.615000009536743	613.80078125\\
0.620000004768372	648.13427734375\\
0.625	682.603149414063\\
0.629999995231628	716.882690429688\\
0.634999990463257	750.875732421875\\
0.639999985694885	784.325134277344\\
0.644999980926514	817.096252441406\\
0.649999976158142	849.106506347656\\
0.654999971389771	880.29833984375\\
0.660000026226044	910.618896484375\\
0.665000021457672	940.023254394531\\
0.670000016689301	968.464721679688\\
0.675000011920929	995.891479492188\\
0.680000007152557	1022.23004150391\\
0.685000002384186	1047.37585449219\\
0.689999997615814	1071.2255859375\\
0.694999992847443	1093.69372558594\\
0.699999988079071	1114.65368652344\\
0.704999983310699	1133.97998046875\\
0.709999978542328	1151.61291503906\\
0.714999973773956	1167.53173828125\\
0.720000028610229	1181.65979003906\\
0.725000023841858	1194.0263671875\\
0.730000019073486	1204.57470703125\\
0.735000014305115	1213.38488769531\\
0.740000009536743	1220.44506835938\\
0.745000004768372	1225.81066894531\\
0.75	1229.50256347656\\
0.754999995231628	1231.57019042969\\
0.759999990463257	1232.03796386719\\
0.764999985694885	1231.44494628906\\
0.769999980926514	1229.67810058594\\
0.774999976158142	1226.43212890625\\
0.779999971389771	1221.59191894531\\
0.785000026226044	1215.22424316406\\
0.790000021457672	1207.42395019531\\
0.795000016689301	1198.27917480469\\
0.800000011920929	1187.87756347656\\
0.805000007152557	1176.30639648438\\
0.810000002384186	1163.67272949219\\
0.814999997615814	1150.0712890625\\
0.819999992847443	1135.6083984375\\
0.824999988079071	1120.38415527344\\
0.829999983310699	1104.50402832031\\
0.834999978542328	1088.06726074219\\
0.839999973773956	1071.16638183594\\
0.845000028610229	1053.90087890625\\
0.850000023841858	1036.36584472656\\
0.855000019073486	1018.64770507813\\
0.860000014305115	1000.80950927734\\
0.865000009536743	982.931579589844\\
0.870000004768372	965.101379394531\\
0.875	947.394348144531\\
0.879999995231628	929.890930175781\\
0.884999990463257	912.661865234375\\
0.889999985694885	895.776184082031\\
0.894999980926514	879.309326171875\\
0.899999976158142	863.324340820313\\
0.904999971389771	847.880187988281\\
0.910000026226044	833.050354003906\\
0.915000021457672	818.885559082031\\
0.920000016689301	805.411682128906\\
0.925000011920929	792.702575683594\\
0.930000007152557	780.804870605469\\
0.935000002384186	769.727783203125\\
0.939999997615814	759.535217285156\\
0.944999992847443	750.239868164063\\
0.949999988079071	741.813293457031\\
0.954999983310699	734.305419921875\\
0.959999978542328	727.707885742188\\
0.964999973773956	722.00927734375\\
0.970000028610229	717.229736328125\\
0.975000023841858	713.355163574219\\
0.980000019073486	710.382873535156\\
0.985000014305115	708.31982421875\\
0.990000009536743	707.185546875\\
0.995000004768372	706.979125976563\\
1	707.736877441406\\
1.00499999523163	709.491577148438\\
1.00999999046326	712.224182128906\\
1.01499998569489	715.553405761719\\
1.01999998092651	719.503234863281\\
1.02499997615814	724.042724609375\\
1.02999997138977	729.191162109375\\
1.0349999666214	734.879455566406\\
1.03999996185303	741.098693847656\\
1.04499995708466	747.80322265625\\
1.04999995231628	754.974792480469\\
1.05499994754791	762.560852050781\\
1.05999994277954	770.525634765625\\
1.06500005722046	778.846801757813\\
1.07000005245209	787.450866699219\\
1.07500004768372	796.3076171875\\
1.08000004291534	805.363159179688\\
1.08500003814697	814.575927734375\\
1.0900000333786	823.903564453125\\
1.09500002861023	833.300659179688\\
1.10000002384186	842.722595214844\\
1.10500001907349	852.126403808594\\
1.11000001430511	861.471862792969\\
1.11500000953674	870.71337890625\\
1.12000000476837	879.8076171875\\
1.125	888.726806640625\\
1.12999999523163	897.429565429688\\
1.13499999046326	905.872924804688\\
1.13999998569489	914.026123046875\\
1.14499998092651	921.888427734375\\
1.14999997615814	929.409973144531\\
1.15499997138977	936.558959960938\\
1.1599999666214	943.334167480469\\
1.16499996185303	949.714721679688\\
1.16999995708466	955.66845703125\\
1.17499995231628	961.176025390625\\
1.17999994754791	966.247619628906\\
1.18499994277954	970.850524902344\\
1.19000005722046	974.96728515625\\
1.19500005245209	978.608764648438\\
1.20000004768372	981.784729003906\\
1.20500004291534	984.486694335938\\
1.21000003814697	986.71826171875\\
1.2150000333786	988.464538574219\\
1.22000002861023	989.736389160156\\
1.22500002384186	990.538513183594\\
1.23000001907349	990.907287597656\\
1.23500001430511	990.891296386719\\
1.24000000953674	990.509216308594\\
1.24500000476837	989.799743652344\\
1.25	988.642333984375\\
1.25499999523163	987.090576171875\\
1.25999999046326	985.136535644531\\
1.26499998569489	982.796142578125\\
1.26999998092651	980.10400390625\\
1.27499997615814	977.046936035156\\
1.27999997138977	973.6552734375\\
1.2849999666214	970.026977539063\\
1.28999996185303	966.150695800781\\
1.29499995708466	962.055053710938\\
1.29999995231628	957.783447265625\\
1.30499994754791	953.3603515625\\
1.30999994277954	948.816589355469\\
1.31500005722046	944.148864746094\\
1.32000005245209	939.388793945313\\
1.32500004768372	934.561889648438\\
1.33000004291534	929.679870605469\\
1.33500003814697	924.761047363281\\
1.3400000333786	919.828186035156\\
1.34500002861023	914.901306152344\\
1.35000002384186	910.000793457031\\
1.35500001907349	905.146728515625\\
1.36000001430511	900.373901367188\\
1.36500000953674	895.750793457031\\
1.37000000476837	891.306823730469\\
1.375	887.073852539063\\
1.37999999523163	882.965209960938\\
1.38499999046326	879.047119140625\\
1.38999998569489	875.335083007813\\
1.39499998092651	871.790283203125\\
1.39999997615814	868.436218261719\\
1.40499997138977	865.280395507813\\
1.4099999666214	862.355224609375\\
1.41499996185303	859.661926269531\\
1.41999995708466	857.2109375\\
1.42499995231628	855.010131835938\\
1.42999994754791	853.070861816406\\
1.43499994277954	851.395690917969\\
1.44000005722046	849.974365234375\\
1.44500005245209	848.818359375\\
1.45000004768372	847.920166015625\\
1.45500004291534	847.266723632813\\
1.46000003814697	846.861999511719\\
1.4650000333786	846.722778320313\\
1.47000002861023	846.870361328125\\
1.47500002384186	847.315490722656\\
1.48000001907349	847.964538574219\\
1.48500001430511	848.756469726563\\
1.49000000953674	849.648010253906\\
1.49500000476837	850.854187011719\\
1.5	852.344848632813\\
1.50499999523163	854.092529296875\\
1.50999999046326	856.019409179688\\
1.51499998569489	858.1455078125\\
1.51999998092651	860.449890136719\\
1.52499997615814	862.915283203125\\
1.52999997138977	865.531494140625\\
1.5349999666214	868.31298828125\\
1.53999996185303	871.24365234375\\
1.54499995708466	874.291687011719\\
1.54999995231628	877.419555664063\\
1.55499994754791	880.616394042969\\
1.55999994277954	883.859069824219\\
1.56500005722046	887.126342773438\\
1.57000005245209	879.732360839844\\
1.57500004768372	882.239562988281\\
1.58000004291534	885.683654785156\\
1.58500003814697	889.37890625\\
1.5900000333786	893.0234375\\
1.59500002861023	896.453430175781\\
1.60000002384186	899.593078613281\\
1.60500001907349	902.412536621094\\
1.61000001430511	904.897888183594\\
1.61500000953674	907.141662597656\\
1.62000000476837	909.226867675781\\
1.625	911.19091796875\\
1.62999999523163	913.07177734375\\
1.63499999046326	914.855102539063\\
1.63999998569489	916.559936523438\\
1.64499998092651	918.255249023438\\
1.64999997615814	919.923461914063\\
1.65499997138977	921.543640136719\\
1.6599999666214	923.089477539063\\
1.66499996185303	924.517639160156\\
1.66999995708466	925.758361816406\\
1.67499995231628	926.845397949219\\
1.67999994754791	927.794128417969\\
1.68499994277954	928.57763671875\\
1.69000005722046	929.092956542969\\
1.69500005245209	929.48046875\\
1.70000004768372	929.744384765625\\
1.70500004291534	929.869262695313\\
1.71000003814697	929.913635253906\\
1.7150000333786	929.870971679688\\
1.72000002861023	929.752868652344\\
1.72500002384186	929.521850585938\\
1.73000001907349	929.196350097656\\
1.73500001430511	928.783874511719\\
1.74000000953674	928.283264160156\\
1.74500000476837	927.701232910156\\
1.75	927.054565429688\\
1.75499999523163	926.334106445313\\
1.75999999046326	925.55126953125\\
1.76499998569489	924.703857421875\\
1.76999998092651	923.792907714844\\
1.77499997615814	922.825317382813\\
1.77999997138977	921.801574707031\\
1.7849999666214	920.723205566406\\
1.78999996185303	919.592529296875\\
1.79499995708466	918.416198730469\\
1.79999995231628	917.208740234375\\
1.80499994754791	915.97314453125\\
1.80999994277954	914.721313476563\\
1.81500005722046	913.461120605469\\
1.82000005245209	912.200317382813\\
1.82500004768372	910.946228027344\\
1.83000004291534	909.707275390625\\
1.83500003814697	908.495727539063\\
1.8400000333786	907.307312011719\\
1.84500002861023	906.141906738281\\
1.85000002384186	905.006469726563\\
1.85500001907349	903.897583007813\\
1.86000001430511	902.817260742188\\
1.86500000953674	901.770935058594\\
1.87000000476837	900.75537109375\\
1.875	899.784973144531\\
1.87999999523163	898.870727539063\\
1.88499999046326	898.003723144531\\
1.88999998569489	897.183837890625\\
1.89499998092651	896.408081054688\\
1.89999997615814	895.678100585938\\
1.90499997138977	894.993957519531\\
1.9099999666214	894.361389160156\\
1.91499996185303	893.778381347656\\
1.91999995708466	893.244384765625\\
1.92499995231628	892.770751953125\\
1.92999994754791	892.356506347656\\
1.93499994277954	891.9990234375\\
1.94000005722046	891.687866210938\\
1.94500005245209	891.419677734375\\
1.95000004768372	891.19921875\\
1.95500004291534	891.029052734375\\
1.96000003814697	890.913208007813\\
1.9650000333786	890.851135253906\\
1.97000002861023	890.835998535156\\
1.97500002384186	890.844482421875\\
1.98000001907349	890.881469726563\\
1.98500001430511	890.942138671875\\
1.99000000953674	891.023376464844\\
1.99500000476837	891.119567871094\\
2	891.226318359375\\
2.00500011444092	891.347106933594\\
2.00999999046326	891.475708007813\\
2.01500010490417	891.609191894531\\
2.01999998092651	891.799133300781\\
2.02500009536743	892.041320800781\\
2.02999997138977	892.338806152344\\
2.03500008583069	892.650085449219\\
2.03999996185303	892.990173339844\\
2.04500007629395	893.358337402344\\
2.04999995231628	893.754333496094\\
2.0550000667572	894.166442871094\\
2.05999994277954	894.555603027344\\
2.06500005722046	894.903930664063\\
2.0699999332428	895.131530761719\\
2.07500004768372	895.206176757813\\
2.07999992370605	895.052856445313\\
2.08500003814697	894.621215820313\\
2.08999991416931	893.837158203125\\
2.09500002861023	892.642761230469\\
2.09999990463257	890.984680175781\\
2.10500001907349	888.81298828125\\
2.10999989509583	886.236145019531\\
2.11500000953674	883.28466796875\\
2.11999988555908	879.722412109375\\
2.125	875.48388671875\\
2.13000011444092	870.677001953125\\
2.13499999046326	865.269226074219\\
2.14000010490417	859.274047851563\\
2.14499998092651	852.768859863281\\
2.15000009536743	845.97509765625\\
2.15499997138977	838.856323242188\\
2.16000008583069	831.28662109375\\
2.16499996185303	823.466552734375\\
2.17000007629395	815.429321289063\\
2.17499995231628	807.184326171875\\
2.1800000667572	798.741455078125\\
2.18499994277954	790.121398925781\\
2.19000005722046	781.377197265625\\
2.1949999332428	772.577087402344\\
2.20000004768372	763.784912109375\\
2.20499992370605	755.05615234375\\
2.21000003814697	746.466491699219\\
2.21499991416931	738.12548828125\\
2.22000002861023	730.202026367188\\
2.22499990463257	722.768371582031\\
2.23000001907349	715.876953125\\
2.23499989509583	709.539855957031\\
2.24000000953674	703.89501953125\\
2.24499988555908	698.966918945313\\
2.25	694.752563476563\\
2.25500011444092	691.295471191406\\
2.25999999046326	688.598022460938\\
2.26500010490417	686.666809082031\\
2.26999998092651	685.577453613281\\
2.27500009536743	685.363098144531\\
2.27999997138977	686.077392578125\\
2.28500008583069	687.857604980469\\
2.28999996185303	690.76123046875\\
2.29500007629395	694.648132324219\\
2.29999995231628	698.796325683594\\
2.3050000667572	703.957763671875\\
2.30999994277954	709.900573730469\\
2.31500005722046	716.451232910156\\
2.3199999332428	723.623657226563\\
2.32500004768372	731.507690429688\\
2.32999992370605	740.083374023438\\
2.33500003814697	749.34423828125\\
2.33999991416931	760.491821289063\\
2.34500002861023	773.189575195313\\
2.34999990463257	785.441345214844\\
2.35500001907349	798.4189453125\\
2.35999989509583	811.927001953125\\
2.36500000953674	825.942138671875\\
2.36999988555908	840.654113769531\\
2.375	855.931091308594\\
2.38000011444092	871.850769042969\\
2.38499999046326	888.458374023438\\
2.39000010490417	905.62255859375\\
2.39499998092651	923.327697753906\\
2.40000009536743	941.561096191406\\
2.40499997138977	960.197937011719\\
2.41000008583069	979.157043457031\\
2.41499996185303	998.325805664063\\
2.42000007629395	1017.55944824219\\
2.42499995231628	1036.75463867188\\
2.4300000667572	1055.6904296875\\
2.43499994277954	1074.22277832031\\
2.44000005722046	1092.17199707031\\
2.4449999332428	1109.19458007813\\
2.45000004768372	1125.38220214844\\
2.45499992370605	1140.43957519531\\
2.46000003814697	1154.3046875\\
2.46499991416931	1166.92065429688\\
2.47000002861023	1178.20190429688\\
2.47499990463257	1188.10217285156\\
2.48000001907349	1196.60705566406\\
2.48499989509583	1203.70715332031\\
2.49000000953674	1209.46459960938\\
2.49499988555908	1213.96362304688\\
2.5	1217.21557617188\\
2.50500011444092	1219.14013671875\\
2.50999999046326	1219.97436523438\\
2.51500010490417	1219.98522949219\\
2.51999998092651	1218.57971191406\\
2.52500009536743	1215.67077636719\\
2.52999997138977	1211.38208007813\\
2.53500008583069	1205.25280761719\\
2.53999996185303	1197.85693359375\\
2.54500007629395	1189.01953125\\
2.54999995231628	1178.80590820313\\
2.5550000667572	1167.31433105469\\
2.55999994277954	1154.58569335938\\
2.56500005722046	1140.79211425781\\
2.5699999332428	1125.98852539063\\
2.57500004768372	1110.30041503906\\
2.57999992370605	1093.87817382813\\
2.58500003814697	1076.89916992188\\
2.58999991416931	1059.49389648438\\
2.59500002861023	1041.64221191406\\
2.59999990463257	1023.43243408203\\
2.60500001907349	1004.90802001953\\
2.60999989509583	986.01318359375\\
2.61500000953674	966.737365722656\\
2.61999988555908	947.102844238281\\
2.625	927.138610839844\\
2.63000011444092	906.968200683594\\
2.63499999046326	886.61669921875\\
2.64000010490417	866.063171386719\\
2.64499998092651	845.518371582031\\
2.65000009536743	825.128234863281\\
2.65499997138977	804.82080078125\\
2.66000008583069	784.586730957031\\
2.66499996185303	764.579345703125\\
2.67000007629395	744.595153808594\\
2.67499995231628	724.653686523438\\
2.6800000667572	705.108093261719\\
2.68499994277954	685.628601074219\\
2.69000005722046	666.477355957031\\
2.6949999332428	647.627746582031\\
2.70000004768372	629.138000488281\\
2.70499992370605	611.069580078125\\
2.71000003814697	593.490966796875\\
2.71499991416931	576.464233398438\\
2.72000002861023	560.167175292969\\
2.72499990463257	544.691040039063\\
2.73000001907349	530.116821289063\\
2.73499989509583	516.546081542969\\
2.74000000953674	504.1044921875\\
2.74499988555908	493.002563476563\\
2.75	483.180847167969\\
2.75500011444092	474.697845458984\\
2.75999999046326	467.813812255859\\
2.76500010490417	462.946472167969\\
2.76999998092651	460.178344726563\\
2.77500009536743	459.548156738281\\
2.77999997138977	460.861846923828\\
2.78500008583069	464.25146484375\\
2.78999996185303	469.791534423828\\
2.79500007629395	477.517852783203\\
2.79999995231628	487.440032958984\\
2.8050000667572	499.562255859375\\
2.80999994277954	513.866088867188\\
2.81500005722046	530.291687011719\\
2.8199999332428	548.756774902344\\
2.82500004768372	569.18017578125\\
2.82999992370605	591.483581542969\\
2.83500003814697	615.663635253906\\
2.83999991416931	641.693420410156\\
2.84500002861023	669.601196289063\\
2.84999990463257	699.483581542969\\
2.85500001907349	731.337097167969\\
2.85999989509583	765.180541992188\\
2.86500000953674	801.23291015625\\
2.86999988555908	839.428649902344\\
2.875	880.011047363281\\
2.88000011444092	923.284790039063\\
2.88499999046326	969.336181640625\\
2.89000010490417	1018.12872314453\\
2.89499998092651	1068.63830566406\\
2.90000009536743	1119.54919433594\\
2.90499997138977	1169.8935546875\\
2.91000008583069	1218.5244140625\\
2.91499996185303	1264.27087402344\\
2.92000007629395	1306.55834960938\\
2.92499995231628	1344.97436523438\\
2.9300000667572	1379.33605957031\\
2.93499994277954	1409.69702148438\\
2.94000005722046	1436.09130859375\\
2.9449999332428	1458.65222167969\\
2.95000004768372	1477.25012207031\\
2.95499992370605	1492.18627929688\\
2.96000003814697	1503.74841308594\\
2.96499991416931	1511.25512695313\\
2.97000002861023	1514.86730957031\\
2.97499990463257	1513.31127929688\\
2.98000001907349	1505.78088378906\\
2.98499989509583	1497.517578125\\
2.99000000953674	1481.93786621094\\
2.99499988555908	1461.85461425781\\
3	1436.89758300781\\
3.00500011444092	1407.4443359375\\
3.00999999046326	1373.94055175781\\
3.01500010490417	1336.88720703125\\
3.01999998092651	1296.83959960938\\
3.02500009536743	1254.38635253906\\
3.02999997138977	1210.00671386719\\
3.03500008583069	1164.11108398438\\
3.03999996185303	1117.03564453125\\
3.04500007629395	1068.94909667969\\
3.04999995231628	1019.99810791016\\
3.0550000667572	970.26025390625\\
3.05999994277954	919.766357421875\\
3.06500005722046	868.71240234375\\
3.0699999332428	817.394287109375\\
3.07500004768372	766.073608398438\\
3.07999992370605	715.233764648438\\
3.08500003814697	665.359191894531\\
3.08999991416931	616.916809082031\\
3.09500002861023	570.485412597656\\
3.09999990463257	526.471740722656\\
3.10500001907349	485.155822753906\\
3.10999989509583	447.003265380859\\
3.11500000953674	412.548553466797\\
3.11999988555908	381.916351318359\\
3.125	355.453826904297\\
3.13000011444092	333.936157226563\\
3.13499999046326	318.208526611328\\
3.14000010490417	308.161834716797\\
3.14499998092651	303.885070800781\\
3.15000009536743	306.517944335938\\
3.15499997138977	316.194763183594\\
3.16000008583069	330.890899658203\\
3.16499996185303	349.973419189453\\
3.17000007629395	372.820678710938\\
3.17499995231628	398.850280761719\\
3.1800000667572	427.744354248047\\
3.18499994277954	459.256683349609\\
3.19000005722046	493.025451660156\\
3.1949999332428	528.266052246094\\
3.20000004768372	564.669921875\\
3.20499992370605	602.565673828125\\
3.21000003814697	641.393493652344\\
3.21499991416931	681.706604003906\\
3.22000002861023	723.753784179688\\
3.22499990463257	767.709228515625\\
3.23000001907349	813.529235839844\\
3.23499989509583	861.003601074219\\
3.24000000953674	909.732666015625\\
3.24499988555908	958.718566894531\\
3.25	1007.7626953125\\
3.25500011444092	1056.10095214844\\
3.25999999046326	1103.45935058594\\
3.26500010490417	1149.47375488281\\
3.26999998092651	1193.96618652344\\
3.27500009536743	1236.65856933594\\
3.27999997138977	1277.41931152344\\
3.28500008583069	1315.89880371094\\
3.28999996185303	1351.60461425781\\
3.29500007629395	1384.01330566406\\
3.29999995231628	1412.57629394531\\
3.3050000667572	1436.74072265625\\
3.30999994277954	1456.14208984375\\
3.31500005722046	1470.43249511719\\
3.3199999332428	1481.34887695313\\
3.32500004768372	1487.54516601563\\
3.32999992370605	1487.85961914063\\
3.33500003814697	1482.37707519531\\
3.33999991416931	1471.59765625\\
3.34500002861023	1455.931640625\\
3.34999990463257	1435.65490722656\\
3.35500001907349	1411.03662109375\\
3.35999989509583	1382.29675292969\\
3.36500000953674	1349.51306152344\\
3.36999988555908	1312.74401855469\\
3.375	1272.16638183594\\
3.38000011444092	1227.88940429688\\
3.38499999046326	1180.11279296875\\
3.39000010490417	1129.13464355469\\
3.39499998092651	1075.33703613281\\
3.40000009536743	1019.19287109375\\
3.40499997138977	961.361083984375\\
3.41000008583069	902.414855957031\\
3.41499996185303	842.985778808594\\
3.42000007629395	783.751831054688\\
3.42499995231628	725.266784667969\\
3.4300000667572	668.170349121094\\
3.43499994277954	612.969482421875\\
3.44000005722046	560.2177734375\\
3.4449999332428	510.596343994141\\
3.45000004768372	464.821655273438\\
3.45499992370605	423.680419921875\\
3.46000003814697	388.501342773438\\
3.46499991416931	359.73974609375\\
3.47000002861023	336.981597900391\\
3.47499990463257	320.203704833984\\
3.48000001907349	309.266387939453\\
3.48499989509583	304.309600830078\\
3.49000000953674	307.230224609375\\
3.49499988555908	315.271881103516\\
3.5	328.363800048828\\
3.50500011444092	346.247772216797\\
3.50999999046326	368.592864990234\\
3.51500010490417	396.373596191406\\
3.51999998092651	429.720886230469\\
3.52500009536743	470.257507324219\\
3.52999997138977	519.095886230469\\
3.53500008583069	578.344482421875\\
3.53999996185303	651.167236328125\\
3.54500007629395	736.7451171875\\
3.54999995231628	830.906372070313\\
3.5550000667572	925.279724121094\\
3.55999994277954	1012.68743896484\\
3.56500005722046	1088.07678222656\\
3.5699999332428	1152.77416992188\\
3.57500004768372	1206.07824707031\\
3.57999992370605	1253.80688476563\\
3.58500003814697	1293.77575683594\\
3.58999991416931	1327.181640625\\
3.59500002861023	1359.34387207031\\
3.59999990463257	1395.1943359375\\
3.60500001907349	1430.98852539063\\
3.60999989509583	1465.51818847656\\
3.61500000953674	1496.75231933594\\
3.61999988555908	1520.76721191406\\
3.625	1535.25024414063\\
3.63000011444092	1538.83959960938\\
3.63499999046326	1534.18139648438\\
3.64000010490417	1522.78063964844\\
3.64499998092651	1500.69970703125\\
3.65000009536743	1469.44030761719\\
3.65499997138977	1430.10009765625\\
3.66000008583069	1384.79138183594\\
3.66499996185303	1336.22436523438\\
3.67000007629395	1286.67224121094\\
3.67499995231628	1236.54150390625\\
3.6800000667572	1185.89904785156\\
3.68499994277954	1134.17370605469\\
3.69000005722046	1080.57336425781\\
3.6949999332428	1024.22387695313\\
3.70000004768372	964.541687011719\\
3.70499992370605	901.714050292969\\
3.71000003814697	835.887268066406\\
3.71499991416931	768.63818359375\\
3.72000002861023	700.925170898438\\
3.72499990463257	634.257202148438\\
3.73000001907349	570.050659179688\\
3.73499989509583	509.84765625\\
3.74000000953674	454.374359130859\\
3.74499988555908	404.736358642578\\
3.75	363.633972167969\\
3.75500011444092	331.534759521484\\
3.75999999046326	309.669158935547\\
3.76500010490417	297.362121582031\\
3.76999998092651	293.921752929688\\
3.77500009536743	301.395843505859\\
3.77999997138977	316.966186523438\\
3.78500008583069	338.070648193359\\
3.78999996185303	363.919738769531\\
3.79500007629395	395.034973144531\\
3.79999995231628	430.048980712891\\
3.8050000667572	468.843292236328\\
3.80999994277954	511.001831054688\\
3.81500005722046	558.144470214844\\
3.8199999332428	611.75341796875\\
3.82500004768372	674.3115234375\\
3.82999992370605	748.501647949219\\
3.83500003814697	832.689880371094\\
3.83999991416931	921.069213867188\\
3.84500002861023	1008.45166015625\\
3.84999990463257	1086.32434082031\\
3.85500001907349	1151.5986328125\\
3.85999989509583	1209.87219238281\\
3.86500000953674	1259.87048339844\\
3.86999988555908	1299.35485839844\\
3.875	1332.77758789063\\
3.88000011444092	1365.39208984375\\
3.88499999046326	1407.81762695313\\
3.89000010490417	1456.54064941406\\
3.89499998092651	1506.10375976563\\
3.90000009536743	1554.10437011719\\
3.90499997138977	1593.63415527344\\
3.91000008583069	1620.99682617188\\
3.91499996185303	1633.95043945313\\
3.92000007629395	1634.77893066406\\
3.92499995231628	1627.45471191406\\
3.9300000667572	1606.63684082031\\
3.93499994277954	1573.17431640625\\
3.94000005722046	1528.87048339844\\
3.9449999332428	1476.94213867188\\
3.95000004768372	1420.48352050781\\
3.95499992370605	1363.10595703125\\
3.96000003814697	1305.06201171875\\
3.96499991416931	1246.19909667969\\
3.97000002861023	1185.51708984375\\
3.97499990463257	1121.92687988281\\
3.98000001907349	1054.07592773438\\
3.98499989509583	981.325805664063\\
3.99000000953674	903.88330078125\\
3.99499988555908	822.79296875\\
4	739.305480957031\\
4.00500011444092	654.850952148438\\
4.01000022888184	572.085083007813\\
4.0149998664856	493.281616210938\\
4.01999998092651	419.332977294922\\
4.02500009536743	352.071136474609\\
4.03000020980835	293.568939208984\\
4.03499984741211	245.80485534668\\
4.03999996185303	209.336563110352\\
4.04500007629395	183.766708374023\\
4.05000019073486	168.306945800781\\
4.05499982833862	161.368103027344\\
4.05999994277954	161.894332885742\\
4.06500005722046	172.004150390625\\
4.07000017166138	187.840209960938\\
4.07499980926514	209.889419555664\\
4.07999992370605	237.154937744141\\
4.08500003814697	270.808563232422\\
4.09000015258789	313.445434570313\\
4.09499979019165	370.648620605469\\
4.09999990463257	454.524963378906\\
4.10500001907349	588.256225585938\\
4.1100001335144	770.647155761719\\
4.11499977111816	962.587158203125\\
4.11999988555908	1126.20629882813\\
4.125	1246.02429199219\\
4.13000011444092	1338.88879394531\\
4.13500022888184	1410.99365234375\\
4.1399998664856	1441.31628417969\\
4.14499998092651	1439.99450683594\\
4.15000009536743	1425.89196777344\\
4.15500020980835	1418.15673828125\\
4.15999984741211	1435.15551757813\\
4.16499996185303	1496.46044921875\\
4.17000007629395	1574.83190917969\\
4.17500019073486	1651.04235839844\\
4.17999982833862	1711.45544433594\\
4.18499994277954	1747.01196289063\\
4.19000005722046	1753.38586425781\\
4.19500017166138	1732.97387695313\\
4.19999980926514	1705.33630371094\\
4.20499992370605	1659.06726074219\\
4.21000003814697	1592.72741699219\\
4.21500015258789	1511.68090820313\\
4.21999979019165	1424.05627441406\\
4.22499990463257	1337.07019042969\\
4.23000001907349	1257.27990722656\\
4.2350001335144	1186.83154296875\\
4.23999977111816	1121.32373046875\\
4.24499988555908	1056.25256347656\\
4.25	988.864624023438\\
4.25500011444092	915.188415527344\\
4.26000022888184	834.513244628906\\
4.2649998664856	747.024597167969\\
4.26999998092651	655.028442382813\\
4.27500009536743	561.979370117188\\
4.28000020980835	472.856872558594\\
4.28499984741211	387.607757568359\\
4.28999996185303	309.570343017578\\
4.29500007629395	241.698959350586\\
4.30000019073486	188.588638305664\\
4.30499982833862	153.698608398438\\
4.30999994277954	135.047103881836\\
4.31500005722046	131.008270263672\\
4.32000017166138	140.34147644043\\
4.32499980926514	159.694686889648\\
4.32999992370605	196.174530029297\\
4.33500003814697	237.22673034668\\
4.34000015258789	279.681915283203\\
4.34499979019165	325.107116699219\\
4.34999990463257	374.982208251953\\
4.35500001907349	434.293090820313\\
4.3600001335144	513.724426269531\\
4.36499977111816	636.161437988281\\
4.36999988555908	816.233276367188\\
4.375	1032.2548828125\\
4.38000011444092	1222.71643066406\\
4.38500022888184	1352.43359375\\
4.3899998664856	1412.1767578125\\
4.39499998092651	1486.71484375\\
4.40000009536743	1515.32702636719\\
4.40500020980835	1495.533203125\\
4.40999984741211	1451.8974609375\\
4.41499996185303	1414.62731933594\\
4.42000007629395	1408.27087402344\\
4.42500019073486	1476.33410644531\\
4.42999982833862	1583.28356933594\\
4.43499994277954	1696.46325683594\\
4.44000005722046	1792.17346191406\\
4.44500017166138	1854.68103027344\\
4.44999980926514	1876.29028320313\\
4.45499992370605	1856.45874023438\\
4.46000003814697	1826.79858398438\\
4.46500015258789	1772.61669921875\\
4.46999979019165	1690.626953125\\
4.47499990463257	1588.44323730469\\
4.48000001907349	1477.80639648438\\
4.4850001335144	1369.94152832031\\
4.48999977111816	1273.19653320313\\
4.49499988555908	1192.86694335938\\
4.5	1121.11328125\\
4.50500011444092	1051.70678710938\\
4.51000022888184	978.390930175781\\
4.5149998664856	896.764953613281\\
4.51999998092651	805.203552246094\\
4.52500009536743	704.071533203125\\
4.53000020980835	597.122314453125\\
4.53499984741211	488.744415283203\\
4.53999996185303	386.266967773438\\
4.54500007629395	288.916107177734\\
4.55000019073486	200.764984130859\\
4.55499982833862	126.081741333008\\
4.55999994277954	97.0110244750977\\
4.56500005722046	102.197052001953\\
4.57000017166138	113.846527099609\\
4.57499980926514	127.935905456543\\
4.57999992370605	141.749542236328\\
4.58500003814697	152.849685668945\\
4.59000015258789	160.817932128906\\
4.59499979019165	166.874114990234\\
4.59999990463257	169.958038330078\\
4.60500001907349	166.570205688477\\
4.6100001335144	153.80924987793\\
4.61499977111816	127.341033935547\\
4.61999988555908	237.676467895508\\
4.625	607.538452148438\\
4.63000011444092	1074.50720214844\\
4.63500022888184	1465.181640625\\
4.6399998664856	1731.15063476563\\
4.64499998092651	1858.06433105469\\
4.65000009536743	1896.65600585938\\
4.65500020980835	1838.3681640625\\
4.65999984741211	1664.5234375\\
4.66499996185303	1576.95434570313\\
4.67000007629395	1413.45446777344\\
4.67500019073486	1269.87121582031\\
4.67999982833862	1196.44262695313\\
4.68499994277954	1269.73583984375\\
4.69000005722046	1441.66052246094\\
4.69500017166138	1625.96179199219\\
4.69999980926514	1776.53198242188\\
4.70499992370605	1865.2109375\\
4.71000003814697	1881.19189453125\\
4.71500015258789	1829.15307617188\\
4.71999979019165	1764.74768066406\\
4.72499990463257	1679.44543457031\\
4.73000001907349	1554.60668945313\\
4.7350001335144	1404.326171875\\
4.73999977111816	1247.85693359375\\
4.74499988555908	1103.93786621094\\
4.75	983.807312011719\\
4.75500011444092	899.834411621094\\
4.76000022888184	835.232727050781\\
4.7649998664856	780.034423828125\\
4.76999998092651	711.475158691406\\
4.77500009536743	634.00048828125\\
4.78000020980835	541.520751953125\\
4.78499984741211	444.810089111328\\
4.78999996185303	348.1240234375\\
4.79500007629395	239.236831665039\\
4.80000019073486	154.040618896484\\
4.80499982833862	77.7313766479492\\
4.80999994277954	95.5549621582031\\
4.81500005722046	128.674392700195\\
4.82000017166138	161.624114990234\\
4.82499980926514	192.264617919922\\
4.82999992370605	218.724182128906\\
4.83500003814697	237.773559570313\\
4.84000015258789	250.951812744141\\
4.84499979019165	251.64143371582\\
4.84999990463257	233.778335571289\\
4.85500001907349	190.950958251953\\
4.8600001335144	119.808647155762\\
4.86499977111816	34.2377090454102\\
4.86999988555908	291.652923583984\\
4.875	740.9296875\\
4.88000011444092	1217.72546386719\\
4.88500022888184	1601.74169921875\\
4.8899998664856	1850.32287597656\\
4.89499998092651	1953.73095703125\\
4.90000009536743	1984.30969238281\\
4.90500020980835	1890.04431152344\\
4.90999984741211	1673.4990234375\\
4.91499996185303	1358.60473632813\\
4.92000007629395	1133.74206542969\\
4.92500019073486	934.336120605469\\
4.92999982833862	798.070678710938\\
4.93499994277954	815.016723632813\\
4.94000005722046	1036.02075195313\\
4.94500017166138	1336.93139648438\\
4.94999980926514	1632.54602050781\\
4.95499992370605	1870.82238769531\\
4.96000003814697	2023.634765625\\
4.96500015258789	2083.10961914063\\
4.96999979019165	2057.13598632813\\
4.97499990463257	2009.67578125\\
4.98000001907349	1929.171875\\
4.9850001335144	1798.064453125\\
4.98999977111816	1638.68591308594\\
4.99499988555908	1474.80578613281\\
5	1329.5634765625\\
5.00500011444092	1215.43151855469\\
5.01000022888184	1138.060546875\\
5.0149998664856	1094.41235351563\\
5.01999998092651	1073.19506835938\\
5.02500009536743	1040.54064941406\\
5.03000020980835	989.502197265625\\
5.03499984741211	917.959045410156\\
5.03999996185303	838.342163085938\\
5.04500007629395	746.190124511719\\
5.05000019073486	643.874572753906\\
5.05499982833862	537.057739257813\\
5.05999994277954	430.056732177734\\
5.06500005722046	326.882751464844\\
5.07000017166138	229.807464599609\\
5.07499980926514	140.30256652832\\
5.07999992370605	59.5021324157715\\
5.08500003814697	31.1821479797363\\
5.09000015258789	17.004581451416\\
5.09499979019165	3.92091274261475\\
5.09999990463257	-10.265251159668\\
5.10500001907349	-16.9192485809326\\
5.1100001335144	-17.5985774993896\\
5.11499977111816	-16.6732330322266\\
5.11999988555908	-15.1461458206177\\
5.125	-13.5032548904419\\
5.13000011444092	-11.9508676528931\\
5.13500022888184	373.347625732422\\
5.1399998664856	686.819091796875\\
5.14499998092651	891.010009765625\\
5.15000009536743	991.037048339844\\
5.15500020980835	1005.21490478516\\
5.15999984741211	1073.04431152344\\
5.16499996185303	1089.09680175781\\
5.17000007629395	1074.2841796875\\
5.17500019073486	1056.32666015625\\
5.17999982833862	1061.14587402344\\
5.18499994277954	1123.62072753906\\
5.19000005722046	1204.74304199219\\
5.19500017166138	1293.60424804688\\
5.19999980926514	1372.2021484375\\
5.20499992370605	1431.62878417969\\
5.21000003814697	1468.00744628906\\
5.21500015258789	1481.08850097656\\
5.21999979019165	1473.72924804688\\
5.22499990463257	1451.72082519531\\
5.23000001907349	1428.41674804688\\
5.2350001335144	1397.82983398438\\
5.23999977111816	1358.13635253906\\
5.24499988555908	1314.60107421875\\
5.25	1271.19750976563\\
5.25500011444092	1230.34411621094\\
5.26000022888184	1196.13879394531\\
5.2649998664856	1168.33825683594\\
5.26999998092651	1149.12524414063\\
5.27500009536743	1132.01647949219\\
5.28000020980835	1113.94445800781\\
5.28499984741211	1093.50439453125\\
5.28999996185303	1070.7158203125\\
5.29500007629395	1045.67346191406\\
5.30000019073486	1021.32739257813\\
5.30499982833862	997.419555664063\\
5.30999994277954	971.382141113281\\
5.31500005722046	942.40673828125\\
5.32000017166138	913.285583496094\\
5.32499980926514	881.181518554688\\
5.32999992370605	847.048278808594\\
5.33500003814697	811.930419921875\\
5.34000015258789	776.972229003906\\
5.34499979019165	744.83203125\\
5.34999990463257	714.618347167969\\
5.35500001907349	685.497375488281\\
5.3600001335144	656.35009765625\\
5.36499977111816	625.83154296875\\
5.36999988555908	592.346008300781\\
5.375	555.19677734375\\
5.38000011444092	513.479614257813\\
5.38500022888184	467.437896728516\\
5.3899998664856	418.137176513672\\
5.39499998092651	367.898345947266\\
5.40000009536743	315.650421142578\\
5.40500020980835	262.896484375\\
5.40999984741211	213.460815429688\\
5.41499996185303	166.653259277344\\
5.42000007629395	122.921508789063\\
5.42500019073486	81.1771774291992\\
5.42999982833862	43.8336219787598\\
5.43499994277954	20.5535259246826\\
5.44000005722046	-0.594296395778656\\
5.44500017166138	-9.85066699981689\\
5.44999980926514	-11.1231870651245\\
5.45499992370605	-10.8680429458618\\
5.46000003814697	-10.0037574768066\\
5.46500015258789	-8.95816707611084\\
5.46999979019165	-7.94244003295898\\
5.47499990463257	-7.04852724075317\\
5.48000001907349	-6.16009998321533\\
5.4850001335144	-5.43950366973877\\
5.48999977111816	-4.75594711303711\\
5.49499988555908	-4.187912940979\\
5.5	-3.65646982192993\\
5.50500011444092	-3.20947074890137\\
5.51000022888184	-2.83590698242188\\
5.5149998664856	-2.4737401008606\\
5.51999998092651	26.818078994751\\
5.52500009536743	89.2800521850586\\
5.53000020980835	128.353485107422\\
5.53499984741211	159.950714111328\\
5.53999996185303	183.827606201172\\
5.54500007629395	200.173736572266\\
5.55000019073486	209.490905761719\\
5.55499982833862	212.81999206543\\
5.55999994277954	212.67073059082\\
5.56500005722046	208.621566772461\\
5.57000017166138	199.57698059082\\
5.57499980926514	186.447784423828\\
5.57999992370605	169.66389465332\\
5.58500003814697	149.889205932617\\
5.59000015258789	128.091384887695\\
5.59499979019165	105.489067077637\\
5.59999990463257	83.3498153686523\\
5.60500001907349	62.7643966674805\\
5.6100001335144	44.6070518493652\\
5.61499977111816	29.4445114135742\\
5.61999988555908	17.5396900177002\\
5.625	8.72833633422852\\
5.63000011444092	2.68395066261292\\
5.63500022888184	-1.0991142988205\\
5.6399998664856	-2.43565964698792\\
5.64499998092651	-2.47264647483826\\
5.65000009536743	-2.3966748714447\\
5.65500020980835	-2.24327397346497\\
5.65999984741211	-2.00757551193237\\
5.66499996185303	-1.78036606311798\\
5.67000007629395	1.25866401195526\\
5.67500019073486	92.8151473999023\\
5.67999982833862	126.02473449707\\
5.68499994277954	146.795135498047\\
5.69000005722046	276.904418945313\\
5.69500017166138	360.000366210938\\
5.69999980926514	420.709655761719\\
5.70499992370605	454.570953369141\\
5.71000003814697	461.218353271484\\
5.71500015258789	452.475219726563\\
5.71999979019165	436.931610107422\\
5.72499990463257	402.991149902344\\
5.73000001907349	358.00146484375\\
5.7350001335144	309.3330078125\\
5.73999977111816	264.308258056641\\
5.74499988555908	228.310333251953\\
5.75	203.608215332031\\
5.75500011444092	189.891128540039\\
5.76000022888184	187.077407836914\\
5.7649998664856	189.86164855957\\
5.76999998092651	194.517227172852\\
5.77500009536743	200.216796875\\
5.78000020980835	206.748672485352\\
5.78499984741211	213.991409301758\\
5.78999996185303	222.428756713867\\
5.79500007629395	231.848739624023\\
5.80000019073486	242.46875\\
5.80499982833862	254.119995117188\\
5.80999994277954	266.716522216797\\
5.81500005722046	279.838317871094\\
5.82000017166138	293.409484863281\\
5.82499980926514	307.728332519531\\
5.82999992370605	321.932891845703\\
5.83500003814697	336.069000244141\\
5.84000015258789	349.610778808594\\
5.84499979019165	361.664764404297\\
5.84999990463257	370.723022460938\\
5.85500001907349	376.534545898438\\
5.8600001335144	377.927795410156\\
5.86499977111816	370.61474609375\\
5.86999988555908	349.465362548828\\
5.875	309.602478027344\\
5.88000011444092	247.715774536133\\
5.88500022888184	162.97526550293\\
5.8899998664856	64.9621200561523\\
5.89499998092651	28.0643711090088\\
5.90000009536743	35.9130592346191\\
5.90500020980835	104.890907287598\\
5.90999984741211	282.585723876953\\
5.91499996185303	567.259521484375\\
5.92000007629395	916.498718261719\\
5.92500019073486	1287.00805664063\\
5.92999982833862	1648.04821777344\\
5.93499994277954	1978.12255859375\\
5.94000005722046	2269.88427734375\\
5.94500017166138	2518.09326171875\\
5.94999980926514	2719.32568359375\\
5.95499992370605	2869.72241210938\\
5.96000003814697	2994.81909179688\\
5.96500015258789	3061.29467773438\\
5.96999979019165	3032.33740234375\\
5.97499990463257	2881.71166992188\\
5.98000001907349	2684.47705078125\\
5.9850001335144	2420.75561523438\\
5.98999977111816	2109.71240234375\\
5.99499988555908	1863.26513671875\\
6	1789.53918457031\\
6.00500011444092	2068.93237304688\\
6.01000022888184	2495.89794921875\\
6.0149998664856	2966.29907226563\\
6.01999998092651	3402.28051757813\\
6.02500009536743	3749.92456054688\\
6.03000020980835	3983.90234375\\
6.03499984741211	4092.56713867188\\
6.03999996185303	4084.43139648438\\
6.04500007629395	4039.8623046875\\
6.05000019073486	3904.7529296875\\
6.05499982833862	3659.34765625\\
6.05999994277954	3337.49267578125\\
6.06500005722046	2991.32885742188\\
6.07000017166138	2673.6875\\
6.07499980926514	2426.041015625\\
6.07999992370605	2264.22119140625\\
6.08500003814697	2188.31396484375\\
6.09000015258789	2206.47583007813\\
6.09499979019165	2252.490234375\\
6.09999990463257	2282.33203125\\
6.10500001907349	2275.45483398438\\
6.1100001335144	2222.9453125\\
6.11499977111816	2148.55786132813\\
6.11999988555908	2033.26416015625\\
6.125	1873.45385742188\\
6.13000011444092	1680.001953125\\
6.13500022888184	1464.70947265625\\
6.1399998664856	1242.29235839844\\
6.14499998092651	1028.53674316406\\
6.15000009536743	837.560119628906\\
6.15500020980835	680.6923828125\\
6.15999984741211	557.414794921875\\
6.16499996185303	467.105346679688\\
6.17000007629395	399.632232666016\\
6.17500019073486	346.494964599609\\
6.17999982833862	297.834869384766\\
6.18499994277954	247.133651733398\\
6.19000005722046	208.254089355469\\
6.19500017166138	168.816665649414\\
6.19999980926514	120.60523223877\\
6.20499992370605	65.1201248168945\\
6.21000003814697	6.64842176437378\\
6.21500015258789	-19.1262683868408\\
6.21999979019165	-22.727632522583\\
6.22499990463257	-22.6688365936279\\
6.23000001907349	-21.1603851318359\\
6.2350001335144	-19.1070365905762\\
6.23999977111816	-17.00315284729\\
6.24499988555908	-15.0480880737305\\
6.25	-13.2739400863647\\
6.25500011444092	-11.6873617172241\\
6.26000022888184	-10.2742757797241\\
6.2649998664856	-9.02722454071045\\
6.26999998092651	-7.93045234680176\\
6.27500009536743	-6.96630620956421\\
6.28000020980835	-6.11773729324341\\
6.28499984741211	-5.37273979187012\\
6.28999996185303	-4.7184624671936\\
6.29500007629395	-4.14330673217773\\
6.30000019073486	-3.63606071472168\\
6.30499982833862	-3.1890127658844\\
6.30999994277954	-2.79681873321533\\
6.31500005722046	-2.45377135276794\\
6.32000017166138	-2.15119767189026\\
6.32499980926514	-1.88578772544861\\
6.32999992370605	-1.65311050415039\\
6.33500003814697	-1.44871640205383\\
6.34000015258789	-1.27002918720245\\
6.34499979019165	-1.11476719379425\\
6.34999990463257	-0.981365561485291\\
6.35500001907349	-0.866477489471436\\
6.3600001335144	57.4189186096191\\
6.36499977111816	97.9559936523438\\
6.36999988555908	127.756080627441\\
6.375	145.683242797852\\
6.38000011444092	152.10514831543\\
6.38500022888184	177.213638305664\\
6.3899998664856	450.963012695313\\
6.39499998092651	617.627380371094\\
6.40000009536743	761.492980957031\\
6.40500020980835	884.860473632813\\
6.40999984741211	994.91259765625\\
6.41499996185303	1090.46569824219\\
6.42000007629395	1173.64978027344\\
6.42500019073486	1246.759765625\\
6.42999982833862	1312.80200195313\\
6.43499994277954	1374.80090332031\\
6.44000005722046	1435.17736816406\\
6.44500017166138	1496.05310058594\\
6.44999980926514	1558.59521484375\\
6.45499992370605	1623.26989746094\\
6.46000003814697	1689.45764160156\\
6.46500015258789	1755.95007324219\\
6.46999979019165	1821.27563476563\\
6.47499990463257	1883.69714355469\\
6.48000001907349	1941.62255859375\\
6.4850001335144	1993.92248535156\\
6.48999977111816	2039.7265625\\
6.49499988555908	2078.65087890625\\
6.5	2110.70947265625\\
6.50500011444092	2136.23217773438\\
6.51000022888184	2155.82690429688\\
6.5149998664856	2170.29223632813\\
6.51999998092651	2180.4912109375\\
6.52500009536743	2187.23217773438\\
6.53000020980835	2191.1552734375\\
6.53499984741211	2192.78491210938\\
6.53999996185303	2192.4013671875\\
6.54500007629395	2190.15283203125\\
6.55000019073486	2185.966796875\\
6.55499982833862	2179.82739257813\\
6.55999994277954	2171.953125\\
6.56500005722046	2162.44116210938\\
6.57000017166138	2150.64331054688\\
6.57499980926514	2136.29345703125\\
6.57999992370605	2119.32397460938\\
6.58500003814697	2099.85717773438\\
6.59000015258789	2078.09521484375\\
6.59499979019165	2054.30029296875\\
6.59999990463257	2028.86059570313\\
6.60500001907349	2002.18444824219\\
6.6100001335144	1974.75634765625\\
6.61499977111816	1947.06713867188\\
6.61999988555908	1919.45703125\\
6.625	1892.35217285156\\
6.63000011444092	1866.02734375\\
6.63500022888184	1840.66198730469\\
6.6399998664856	1816.66772460938\\
6.64499998092651	1793.86938476563\\
6.65000009536743	1772.47875976563\\
6.65500020980835	1752.44396972656\\
6.65999984741211	1733.65625\\
6.66499996185303	1716.2470703125\\
6.67000007629395	1700.47692871094\\
6.67500019073486	1685.93688964844\\
6.67999982833862	1672.8671875\\
6.68499994277954	1661.42309570313\\
6.69000005722046	1650.96813964844\\
6.69500017166138	1642.66357421875\\
6.69999980926514	1636.84313964844\\
6.70499992370605	1633.06005859375\\
6.71000003814697	1630.70458984375\\
6.71500015258789	1629.67004394531\\
6.71999979019165	1629.830078125\\
6.72499990463257	1631.06457519531\\
6.73000001907349	1633.236328125\\
6.7350001335144	1636.33862304688\\
6.73999977111816	1639.84448242188\\
6.74499988555908	1643.96960449219\\
6.75	1648.41809082031\\
6.75500011444092	1653.35754394531\\
6.76000022888184	1659.15100097656\\
6.7649998664856	1664.90539550781\\
6.76999998092651	1670.95324707031\\
6.77500009536743	1677.10864257813\\
6.78000020980835	1683.78149414063\\
6.78499984741211	1690.90148925781\\
6.78999996185303	1698.15051269531\\
6.79500007629395	1705.83166503906\\
6.80000019073486	1713.91223144531\\
6.80499982833862	1722.33764648438\\
6.80999994277954	1731.13928222656\\
6.81500005722046	1740.25354003906\\
6.82000017166138	1749.54602050781\\
6.82499980926514	1759.05004882813\\
6.82999992370605	1768.61962890625\\
6.83500003814697	1778.60632324219\\
6.84000015258789	1787.26904296875\\
6.84499979019165	1795.33349609375\\
6.84999990463257	1802.00012207031\\
6.85500001907349	1806.85852050781\\
6.8600001335144	1809.85754394531\\
6.86499977111816	1809.41442871094\\
6.86999988555908	1805.98974609375\\
6.875	1800.66723632813\\
6.88000011444092	1792.27160644531\\
6.88500022888184	1781.26904296875\\
6.8899998664856	1767.744140625\\
6.89499998092651	1751.82238769531\\
6.90000009536743	1733.63293457031\\
6.90500020980835	1713.39074707031\\
6.90999984741211	1690.7958984375\\
6.91499996185303	1666.19885253906\\
6.92000007629395	1639.587890625\\
6.92500019073486	1610.92150878906\\
6.92999982833862	1580.39526367188\\
6.93499994277954	1547.85375976563\\
6.94000005722046	1513.57116699219\\
6.94500017166138	1477.55773925781\\
6.94999980926514	1439.8359375\\
6.95499992370605	1399.57141113281\\
6.96000003814697	1357.98010253906\\
6.96500015258789	1316.12231445313\\
6.96999979019165	1272.68298339844\\
6.97499990463257	1228.82250976563\\
6.98000001907349	1184.43310546875\\
6.9850001335144	1140.15258789063\\
6.98999977111816	1095.70678710938\\
6.99499988555908	1051.07214355469\\
7	1007.19976806641\\
7.00500011444092	963.599243164063\\
7.01000022888184	920.492004394531\\
7.0149998664856	877.914428710938\\
7.01999998092651	835.965270996094\\
7.02500009536743	794.851806640625\\
7.03000020980835	754.592712402344\\
7.03499984741211	715.263793945313\\
7.03999996185303	676.981628417969\\
7.04500007629395	639.866455078125\\
7.05000019073486	604.050598144531\\
7.05499982833862	569.706970214844\\
7.05999994277954	536.786315917969\\
7.06500005722046	505.493316650391\\
7.07000017166138	475.973571777344\\
7.07499980926514	448.242065429688\\
7.07999992370605	422.350555419922\\
7.08500003814697	398.355712890625\\
7.09000015258789	376.302551269531\\
7.09499979019165	356.203704833984\\
7.09999990463257	337.982940673828\\
7.10500001907349	321.602691650391\\
7.1100001335144	307.027038574219\\
7.11499977111816	294.325866699219\\
7.11999988555908	283.401702880859\\
7.125	274.154998779297\\
7.13000011444092	266.660675048828\\
7.13500022888184	260.86083984375\\
7.1399998664856	256.685180664063\\
7.14499998092651	254.146057128906\\
7.15000009536743	253.294952392578\\
7.15500020980835	254.386260986328\\
7.15999984741211	257.297119140625\\
7.16499996185303	261.846343994141\\
7.17000007629395	267.339874267578\\
7.17500019073486	273.905639648438\\
7.17999982833862	281.532928466797\\
7.18499994277954	290.112335205078\\
7.19000005722046	299.560119628906\\
7.19500017166138	309.821380615234\\
7.19999980926514	320.839965820313\\
7.20499992370605	332.466979980469\\
7.21000003814697	344.737213134766\\
7.21500015258789	357.528717041016\\
7.21999979019165	370.633575439453\\
7.22499990463257	384.133728027344\\
7.23000001907349	397.906463623047\\
7.2350001335144	411.865081787109\\
7.23999977111816	425.957763671875\\
7.24499988555908	440.079559326172\\
7.25	454.149108886719\\
7.25500011444092	468.120758056641\\
7.26000022888184	481.865661621094\\
7.2649998664856	495.347839355469\\
7.26999998092651	508.539916992188\\
7.27500009536743	521.326538085938\\
7.28000020980835	533.607360839844\\
7.28499984741211	545.390380859375\\
7.28999996185303	556.597473144531\\
7.29500007629395	567.249145507813\\
7.30000019073486	577.1708984375\\
7.30499982833862	586.418762207031\\
7.30999994277954	594.955261230469\\
7.31500005722046	602.787536621094\\
7.32000017166138	609.851440429688\\
7.32499980926514	616.164123535156\\
7.32999992370605	621.66845703125\\
7.33500003814697	626.356994628906\\
7.34000015258789	630.240234375\\
7.34499979019165	633.30419921875\\
7.34999990463257	635.551879882813\\
7.35500001907349	636.994750976563\\
7.3600001335144	637.709655761719\\
7.36499977111816	637.711853027344\\
7.36999988555908	637.021850585938\\
7.375	635.652099609375\\
7.38000011444092	633.633239746094\\
7.38500022888184	630.859924316406\\
7.3899998664856	627.267639160156\\
7.39499998092651	622.883911132813\\
7.40000009536743	617.864868164063\\
7.40500020980835	612.178588867188\\
7.40999984741211	605.903991699219\\
7.41499996185303	599.08447265625\\
7.42000007629395	591.781799316406\\
7.42500019073486	584.0703125\\
7.42999982833862	575.991027832031\\
7.43499994277954	567.556640625\\
7.44000005722046	558.822204589844\\
7.44500017166138	549.827026367188\\
7.44999980926514	540.613220214844\\
7.45499992370605	531.211059570313\\
7.46000003814697	521.648620605469\\
7.46500015258789	511.976654052734\\
7.46999979019165	502.28125\\
7.47499990463257	492.615356445313\\
7.48000001907349	482.998626708984\\
7.4850001335144	473.488006591797\\
7.48999977111816	464.111419677734\\
7.49499988555908	454.900024414063\\
7.5	445.877075195313\\
7.50500011444092	437.057861328125\\
7.51000022888184	428.46923828125\\
7.5149998664856	420.183319091797\\
7.51999998092651	412.247192382813\\
7.52500009536743	404.891174316406\\
7.53000020980835	397.683746337891\\
7.53499984741211	390.9423828125\\
7.53999996185303	384.587890625\\
7.54500007629395	378.656127929688\\
7.55000019073486	373.165374755859\\
7.55499982833862	368.130889892578\\
7.55999994277954	363.56640625\\
7.56500005722046	359.481109619141\\
7.57000017166138	355.878021240234\\
7.57499980926514	352.766143798828\\
7.57999992370605	350.141510009766\\
7.58500003814697	348.011383056641\\
7.59000015258789	346.364715576172\\
7.59499979019165	345.202789306641\\
7.59999990463257	344.519500732422\\
7.60500001907349	344.314422607422\\
7.6100001335144	344.634613037109\\
7.61499977111816	345.53173828125\\
7.61999988555908	346.871215820313\\
7.625	348.496765136719\\
7.63000011444092	350.447052001953\\
7.63500022888184	352.739715576172\\
7.6399998664856	355.351715087891\\
7.64499998092651	358.274963378906\\
7.65000009536743	361.487335205078\\
7.65500020980835	364.974609375\\
7.65999984741211	368.716979980469\\
7.66499996185303	372.694152832031\\
7.67000007629395	376.877777099609\\
7.67500019073486	381.246856689453\\
7.67999982833862	385.778381347656\\
7.68499994277954	390.450469970703\\
7.69000005722046	395.240478515625\\
7.69500017166138	400.115875244141\\
7.69999980926514	405.051818847656\\
7.70499992370605	410.0244140625\\
7.71000003814697	415.014801025391\\
7.71500015258789	420.015869140625\\
7.71999979019165	425.001190185547\\
7.72499990463257	429.945892333984\\
7.73000001907349	434.828125\\
7.7350001335144	439.62548828125\\
7.73999977111816	444.329925537109\\
7.74499988555908	448.932830810547\\
7.75	453.398590087891\\
7.75500011444092	457.72900390625\\
7.76000022888184	461.910430908203\\
7.7649998664856	465.913055419922\\
7.76999998092651	469.724670410156\\
7.77500009536743	473.353698730469\\
7.78000020980835	476.798370361328\\
7.78499984741211	480.043914794922\\
7.78999996185303	483.099365234375\\
7.79500007629395	485.964752197266\\
7.80000019073486	488.612915039063\\
7.80499982833862	491.054260253906\\
7.80999994277954	493.291107177734\\
7.81500005722046	495.318817138672\\
7.82000017166138	497.136901855469\\
7.82499980926514	498.747039794922\\
7.82999992370605	500.156799316406\\
7.83500003814697	501.374603271484\\
7.84000015258789	502.393707275391\\
7.84499979019165	503.220123291016\\
7.84999990463257	503.885498046875\\
7.85500001907349	504.384063720703\\
7.8600001335144	504.722991943359\\
7.86499977111816	504.902587890625\\
7.86999988555908	504.931213378906\\
7.875	504.814758300781\\
7.88000011444092	504.588317871094\\
7.88500022888184	504.266479492188\\
7.8899998664856	503.857116699219\\
7.89499998092651	503.365264892578\\
7.90000009536743	502.794403076172\\
7.90500020980835	502.159454345703\\
7.90999984741211	501.467559814453\\
7.91499996185303	500.718627929688\\
7.92000007629395	499.924499511719\\
7.92500019073486	499.096954345703\\
7.92999982833862	498.306518554688\\
7.93499994277954	497.512847900391\\
7.94000005722046	496.715911865234\\
7.94500017166138	496.006469726563\\
7.94999980926514	495.359283447266\\
7.95499992370605	494.761688232422\\
7.96000003814697	494.286743164063\\
7.96500015258789	493.957122802734\\
7.96999979019165	493.738922119141\\
7.97499990463257	493.650451660156\\
7.98000001907349	493.720550537109\\
7.9850001335144	493.928283691406\\
7.98999977111816	494.281005859375\\
7.99499988555908	494.826751708984\\
8	495.537780761719\\
8.00500011444092	496.4140625\\
8.01000022888184	497.519134521484\\
8.01500034332275	498.848205566406\\
8.02000045776367	500.411468505859\\
8.02499961853027	502.195953369141\\
8.02999973297119	504.20263671875\\
8.03499984741211	506.444061279297\\
8.03999996185303	508.919982910156\\
8.04500007629395	511.627197265625\\
8.05000019073486	514.572204589844\\
8.05500030517578	517.755432128906\\
8.0600004196167	521.170349121094\\
8.0649995803833	524.821044921875\\
8.06999969482422	528.708374023438\\
8.07499980926514	532.883728027344\\
8.07999992370605	537.337524414063\\
8.08500003814697	542.079833984375\\
8.09000015258789	547.062438964844\\
8.09500026702881	552.28173828125\\
8.10000038146973	557.742858886719\\
8.10499954223633	563.457763671875\\
8.10999965667725	569.439086914063\\
8.11499977111816	575.679809570313\\
8.11999988555908	582.174987792969\\
8.125	588.906311035156\\
8.13000011444092	595.878723144531\\
8.13500022888184	603.090942382813\\
8.14000034332275	610.566772460938\\
8.14500045776367	618.297485351563\\
8.14999961853027	626.285278320313\\
8.15499973297119	634.494323730469\\
8.15999984741211	642.925109863281\\
8.16499996185303	651.574768066406\\
8.17000007629395	660.504150390625\\
8.17500019073486	669.752258300781\\
8.18000030517578	679.311462402344\\
8.1850004196167	689.144592285156\\
8.1899995803833	699.160888671875\\
8.19499969482422	709.392700195313\\
8.19999980926514	719.843566894531\\
8.20499992370605	730.524719238281\\
8.21000003814697	741.420227050781\\
8.21500015258789	752.515197753906\\
8.22000026702881	763.795227050781\\
8.22500038146973	775.254638671875\\
8.22999954223633	787.182678222656\\
8.23499965667725	799.74853515625\\
8.23999977111816	812.587890625\\
8.24499988555908	825.429931640625\\
8.25	838.473815917969\\
8.25500011444092	851.853088378906\\
8.26000022888184	865.501586914063\\
8.26500034332275	879.433288574219\\
8.27000045776367	893.769836425781\\
8.27499961853027	908.513244628906\\
8.27999973297119	923.484680175781\\
8.28499984741211	938.760314941406\\
8.28999996185303	954.521057128906\\
8.29500007629395	970.610534667969\\
8.30000019073486	986.930114746094\\
8.30500030517578	1003.17724609375\\
8.3100004196167	1019.44439697266\\
8.3149995803833	1038.56811523438\\
8.31999969482422	1056.80639648438\\
8.32499980926514	1075.35229492188\\
8.32999992370605	1094.34375\\
8.33500003814697	1113.93701171875\\
8.34000015258789	1133.9775390625\\
8.34500026702881	1154.357421875\\
8.35000038146973	1175.29772949219\\
8.35499954223633	1196.10021972656\\
8.35999965667725	1218.27099609375\\
8.36499977111816	1241.71520996094\\
8.36999988555908	1264.94421386719\\
8.375	1288.70446777344\\
8.38000011444092	1313.28625488281\\
8.38500022888184	1338.1708984375\\
8.39000034332275	1363.62756347656\\
8.39500045776367	1388.89892578125\\
8.39999961853027	1416.27624511719\\
8.40499973297119	1443.84106445313\\
8.40999984741211	1471.76953125\\
8.41499996185303	1500.64501953125\\
8.42000007629395	1530.01647949219\\
8.42500019073486	1559.98181152344\\
8.43000030517578	1590.10620117188\\
8.4350004196167	1620.76208496094\\
8.4399995803833	1652.42321777344\\
8.44499969482422	1683.50927734375\\
8.44999980926514	1714.21643066406\\
8.45499992370605	1744.54296875\\
8.46000003814697	1773.50463867188\\
8.46500015258789	1801.85498046875\\
8.47000026702881	1829.15612792969\\
8.47500038146973	1855.36389160156\\
8.47999954223633	1880.47473144531\\
8.48499965667725	1904.20178222656\\
8.48999977111816	1926.07348632813\\
8.49499988555908	1946.85314941406\\
8.5	1965.986328125\\
8.50500011444092	1983.44458007813\\
8.51000022888184	1999.19482421875\\
8.51500034332275	2013.61511230469\\
8.52000045776367	2027.26611328125\\
8.52499961853027	2038.20092773438\\
8.52999973297119	2048.33862304688\\
8.53499984741211	2056.57788085938\\
8.53999996185303	2063.15795898438\\
8.54500007629395	2068.19384765625\\
8.55000019073486	2071.9384765625\\
8.55500030517578	2074.4765625\\
8.5600004196167	2076.03100585938\\
8.5649995803833	2077.009765625\\
8.56999969482422	2077.59741210938\\
8.57499980926514	2077.7822265625\\
8.57999992370605	2077.70849609375\\
8.58500003814697	2077.3955078125\\
8.59000015258789	2077.11547851563\\
8.59500026702881	2076.80395507813\\
8.60000038146973	2076.21704101563\\
8.60499954223633	2075.18530273438\\
8.60999965667725	2073.45483398438\\
8.61499977111816	2070.74072265625\\
8.61999988555908	2066.7314453125\\
8.625	2061.044921875\\
8.63000011444092	2053.3076171875\\
8.63500022888184	2043.32043457031\\
8.64000034332275	2030.77075195313\\
8.64500045776367	2015.50170898438\\
8.64999961853027	1997.64831542969\\
8.65499973297119	1977.08215332031\\
8.65999984741211	1954.48852539063\\
8.66499996185303	1929.78662109375\\
8.67000007629395	1903.50939941406\\
8.67500019073486	1875.91662597656\\
8.68000030517578	1847.50085449219\\
8.6850004196167	1818.48022460938\\
8.6899995803833	1788.998046875\\
8.69499969482422	1759.21899414063\\
8.69999980926514	1729.16687011719\\
8.70499992370605	1698.78234863281\\
8.71000003814697	1667.9306640625\\
8.71500015258789	1636.46655273438\\
8.72000026702881	1604.45178222656\\
8.72500038146973	1571.79956054688\\
8.72999954223633	1538.50122070313\\
8.73499965667725	1504.64025878906\\
8.73999977111816	1470.376953125\\
8.74499988555908	1435.89599609375\\
8.75	1401.416015625\\
8.75500011444092	1367.10900878906\\
8.76000022888184	1333.39709472656\\
8.76500034332275	1300.43176269531\\
8.77000045776367	1268.43090820313\\
8.77499961853027	1237.50549316406\\
8.77999973297119	1207.74536132813\\
8.78499984741211	1179.14660644531\\
8.78999996185303	1151.69409179688\\
8.79500007629395	1125.36291503906\\
8.80000019073486	1100.10083007813\\
8.80500030517578	1075.85095214844\\
8.8100004196167	1052.5712890625\\
8.8149995803833	1030.2568359375\\
8.81999969482422	1008.90814208984\\
8.82499980926514	988.573364257813\\
8.82999992370605	969.338317871094\\
8.83500003814697	951.267150878906\\
8.84000015258789	934.454772949219\\
8.84500026702881	919.005004882813\\
8.85000038146973	904.947570800781\\
8.85499954223633	892.364013671875\\
8.85999965667725	881.227233886719\\
8.86499977111816	871.509643554688\\
8.86999988555908	863.148071289063\\
8.875	856.055725097656\\
8.88000011444092	850.130249023438\\
8.88500022888184	845.263610839844\\
8.89000034332275	841.346923828125\\
8.89500045776367	838.295349121094\\
8.89999961853027	836.0556640625\\
8.90499973297119	834.644958496094\\
8.90999984741211	833.997680664063\\
8.91499996185303	834.066650390625\\
8.92000007629395	834.893981933594\\
8.92500019073486	836.554443359375\\
8.93000030517578	839.306213378906\\
8.9350004196167	842.885009765625\\
8.9399995803833	846.79833984375\\
8.94499969482422	851.362976074219\\
8.94999980926514	856.484191894531\\
8.95499992370605	862.056335449219\\
8.96000003814697	868.052856445313\\
8.96500015258789	874.420593261719\\
8.97000026702881	881.081848144531\\
8.97500038146973	887.978088378906\\
8.97999954223633	895.063293457031\\
8.98499965667725	902.373168945313\\
8.98999977111816	909.764526367188\\
8.99499988555908	917.165344238281\\
9	924.679504394531\\
9.00500011444092	932.208374023438\\
9.01000022888184	939.709594726563\\
9.01500034332275	947.145385742188\\
9.02000045776367	954.473937988281\\
9.02499961853027	961.644287109375\\
9.02999973297119	968.60693359375\\
9.03499984741211	975.322631835938\\
9.03999996185303	981.746704101563\\
9.04500007629395	987.838989257813\\
9.05000019073486	993.55908203125\\
9.05500030517578	998.870666503906\\
9.0600004196167	1003.75421142578\\
9.0649995803833	1008.16961669922\\
9.06999969482422	1012.1171875\\
9.07499980926514	1015.56512451172\\
9.07999992370605	1018.50360107422\\
9.08500003814697	1020.92150878906\\
9.09000015258789	1022.80651855469\\
9.09500026702881	1024.16564941406\\
9.10000038146973	1024.98657226563\\
9.10499954223633	1025.34875488281\\
9.10999965667725	1025.22045898438\\
9.11499977111816	1024.63842773438\\
9.11999988555908	1023.61212158203\\
9.125	1022.08654785156\\
9.13000011444092	1019.96368408203\\
9.13500022888184	1017.30737304688\\
9.14000034332275	1014.15911865234\\
9.14500045776367	1010.51635742188\\
9.14999961853027	1006.43676757813\\
9.15499973297119	1001.97766113281\\
9.15999984741211	997.177917480469\\
9.16499996185303	992.088928222656\\
9.17000007629395	986.775756835938\\
9.17500019073486	981.265258789063\\
9.18000030517578	975.608215332031\\
9.1850004196167	969.872985839844\\
9.1899995803833	964.050598144531\\
9.19499969482422	958.159362792969\\
9.19999980926514	952.241394042969\\
9.20499992370605	946.325561523438\\
9.21000003814697	940.429321289063\\
9.21500015258789	934.500183105469\\
9.22000026702881	928.55224609375\\
9.22500038146973	922.575744628906\\
9.22999954223633	916.571838378906\\
9.23499965667725	910.734680175781\\
9.23999977111816	905.634582519531\\
9.24499988555908	902.507446289063\\
9.25	901.837585449219\\
9.25500011444092	903.30029296875\\
9.26000022888184	907.203125\\
9.26500034332275	913.962646484375\\
9.27000045776367	923.420166015625\\
9.27499961853027	935.743591308594\\
9.27999973297119	951.133728027344\\
9.28499984741211	969.315124511719\\
9.28999996185303	990.619689941406\\
9.29500007629395	1015.10522460938\\
9.30000019073486	1042.46252441406\\
9.30500030517578	1075.85705566406\\
9.3100004196167	1110.67333984375\\
9.3149995803833	1148.47583007813\\
9.31999969482422	1188.86645507813\\
9.32499980926514	1231.14611816406\\
9.32999992370605	1275.54602050781\\
9.33500003814697	1319.37939453125\\
9.34000015258789	1360.46130371094\\
9.34500026702881	1402.95007324219\\
9.35000038146973	1437.65222167969\\
9.35499954223633	1463.29809570313\\
9.35999965667725	1485.07897949219\\
9.36499977111816	1496.92590332031\\
9.36999988555908	1502.11743164063\\
9.375	1503.75622558594\\
9.38000011444092	1504.74694824219\\
9.38500022888184	1508.88232421875\\
9.39000034332275	1521.81628417969\\
9.39500045776367	1547.66674804688\\
9.39999961853027	1590.29577636719\\
9.40499973297119	1653.06945800781\\
9.40999984741211	1738.88061523438\\
9.41499996185303	1848.05847167969\\
9.42000007629395	1980.33349609375\\
9.42500019073486	2133.85083007813\\
9.43000030517578	2312.13598632813\\
9.4350004196167	2504.724609375\\
9.4399995803833	2689.28125\\
9.44499969482422	2841.22094726563\\
9.44999980926514	2923.9765625\\
9.45499992370605	2903.42456054688\\
9.46000003814697	2739.51123046875\\
9.46500015258789	2415.779296875\\
9.47000026702881	1933.2392578125\\
9.47500038146973	1329.0439453125\\
9.47999954223633	674.556701660156\\
9.48499965667725	152.675994873047\\
9.48999977111816	121.226570129395\\
9.49499988555908	129.014389038086\\
9.5	150.286468505859\\
9.50500011444092	171.912689208984\\
9.51000022888184	187.518829345703\\
9.51500034332275	195.044479370117\\
9.52000045776367	193.183013916016\\
9.52499961853027	182.2861328125\\
9.52999973297119	161.49250793457\\
9.53499984741211	131.986282348633\\
9.53999996185303	94.5654754638672\\
9.54500007629395	50.6394882202148\\
9.55000019073486	2.31766510009766\\
9.55500030517578	-25.472261428833\\
9.5600004196167	-34.5400009155273\\
9.5649995803833	-35.8118438720703\\
9.56999969482422	-33.899730682373\\
9.57499980926514	-30.9543743133545\\
9.57999992370605	-27.6238174438477\\
9.58500003814697	-24.5062274932861\\
9.59000015258789	-21.6296787261963\\
9.59500026702881	-19.0063667297363\\
9.60000038146973	-16.6855201721191\\
9.60499954223633	-14.63942527771\\
9.60999965667725	-12.8436012268066\\
9.61499977111816	-11.2587652206421\\
9.61999988555908	-9.87175750732422\\
9.625	-8.64572238922119\\
9.63000011444092	-7.56859159469604\\
9.63500022888184	-6.61389017105103\\
9.64000034332275	-5.77581596374512\\
9.64500045776367	-5.06001758575439\\
9.64999961853027	-4.43587970733643\\
9.65499973297119	-3.87515354156494\\
9.65999984741211	-3.3982880115509\\
9.66499996185303	-2.98085236549377\\
9.67000007629395	-2.61452198028564\\
9.67500019073486	-2.29132127761841\\
9.68000030517578	-2.0083339214325\\
9.6850004196167	-1.76448142528534\\
9.6899995803833	-1.55051100254059\\
9.69499969482422	-1.35920894145966\\
9.69999980926514	-1.19150447845459\\
9.70499992370605	-1.04603028297424\\
9.71000003814697	-0.916996359825134\\
9.71500015258789	-0.799171090126038\\
9.72000026702881	-0.694994688034058\\
9.72500038146973	-0.605665266513824\\
9.72999954223633	-0.527653753757477\\
9.73499965667725	-0.460395455360413\\
9.73999977111816	-0.403071790933609\\
9.74499988555908	-0.354637235403061\\
9.75	-0.311594933271408\\
9.75500011444092	-0.27276200056076\\
9.76000022888184	-0.235925078392029\\
9.76500034332275	-0.202673867344856\\
9.77000045776367	-0.17571547627449\\
9.77499961853027	-0.153629019856453\\
9.77999973297119	-0.136958330869675\\
9.78499984741211	-0.120036870241165\\
9.78999996185303	-0.104396246373653\\
9.79500007629395	-0.0891327932476997\\
9.80000019073486	-0.0752721801400185\\
9.80500030517578	-0.0642856508493423\\
9.8100004196167	-0.0557514429092407\\
9.8149995803833	-0.0501282215118408\\
9.81999969482422	-0.043894037604332\\
9.82499980926514	-0.038077674806118\\
9.82999992370605	-0.0321159362792969\\
9.83500003814697	-0.0265634004026651\\
9.84000015258789	-0.0221318323165178\\
9.84500026702881	-0.0184308178722858\\
9.85000038146973	-0.0156457889825106\\
9.85499954223633	-0.0137485126033425\\
9.85999965667725	-0.0128561146557331\\
9.86499977111816	-0.0130512472242117\\
9.86999988555908	-0.0122885210439563\\
9.875	-0.00895760115236044\\
9.88000011444092	-0.00341031071729958\\
9.88500022888184	0.00429784413427114\\
9.89000034332275	0.00565160997211933\\
9.89500045776367	0.00352487061172724\\
9.89999961853027	-0.00310997245833278\\
9.90499973297119	-0.00839851517230272\\
9.90999984741211	-0.00823531579226255\\
9.91499996185303	-0.00442573381587863\\
9.92000007629395	0.00328513444401324\\
9.92500019073486	0.00507023325189948\\
9.93000030517578	0.00482437154278159\\
9.9350004196167	0.00172387750353664\\
9.9399995803833	-0.000258364743785933\\
9.94499969482422	0.000472034531412646\\
9.94999980926514	0.00179940578527749\\
9.95499992370605	0.00372374686412513\\
9.96000003814697	0.00624505523592234\\
9.96500015258789	0.00936332996934652\\
9.97000026702881	0.0130785694345832\\
9.97500038146973	0.0148602537810802\\
9.97999954223633	0.014863483607769\\
9.98499965667725	0.014401794411242\\
9.98999977111816	0.0134751861914992\\
9.99499988555908	0.0120836589485407\\
10	0.0102272126823664\\
};
\addlegendentry{RS}

\addplot [color=red, line width=2.0pt]
  table[row sep=crcr]{%
0.0949999988079071	4.24330282211304\\
0.100000001490116	3.87630486488342\\
0.104999996721745	3.54322338104248\\
0.109999999403954	3.22589516639709\\
0.115000002086163	2.921541929245\\
0.119999997317791	183.522583007813\\
0.125	447.012481689453\\
0.129999995231628	672.644897460938\\
0.135000005364418	846.649169921875\\
0.140000000596046	971.472961425781\\
0.144999995827675	1052.78247070313\\
0.150000005960464	1098.53198242188\\
0.155000001192093	1112.39892578125\\
0.159999996423721	1109.06286621094\\
0.165000006556511	1085.85302734375\\
0.170000001788139	1035.57067871094\\
0.174999997019768	957.668029785156\\
0.180000007152557	852.879577636719\\
0.185000002384186	725.447143554688\\
0.189999997615814	579.310791015625\\
0.194999992847443	535.323974609375\\
0.200000002980232	409.723571777344\\
0.204999998211861	296.665985107422\\
0.209999993443489	318.947235107422\\
0.215000003576279	489.249572753906\\
0.219999998807907	783.744018554688\\
0.224999994039536	1127.62280273438\\
0.230000004172325	1488.52490234375\\
0.234999999403954	1828.29833984375\\
0.239999994635582	2113.63500976563\\
0.245000004768372	2322.84106445313\\
0.25	2446.23779296875\\
0.254999995231628	2492.03588867188\\
0.259999990463257	2486.05883789063\\
0.264999985694885	2425.74267578125\\
0.270000010728836	2316.13842773438\\
0.275000005960464	2179.89672851563\\
0.280000001192093	2044.76892089844\\
0.284999996423721	1937.16296386719\\
0.28999999165535	1872.01025390625\\
0.294999986886978	1870.66357421875\\
0.300000011920929	1918.33850097656\\
0.305000007152557	1984.69714355469\\
0.310000002384186	2053.07202148438\\
0.314999997615814	2110.369140625\\
0.319999992847443	2145.26538085938\\
0.324999988079071	2151.13842773438\\
0.330000013113022	2139.6240234375\\
0.33500000834465	2102.607421875\\
0.340000003576279	2036.94921875\\
0.344999998807907	1942.64331054688\\
0.349999994039536	1830.54833984375\\
0.354999989271164	1712.54357910156\\
0.360000014305115	1599.06616210938\\
0.365000009536743	1499.13000488281\\
0.370000004768372	1417.876953125\\
0.375	1354.46398925781\\
0.379999995231628	1308.55688476563\\
0.384999990463257	1275.1318359375\\
0.389999985694885	1249.83972167969\\
0.395000010728836	1223.96691894531\\
0.400000005960464	1193.37573242188\\
0.405000001192093	1152.56604003906\\
0.409999996423721	1098.56213378906\\
0.41499999165535	1031.62280273438\\
0.419999986886978	953.6044921875\\
0.425000011920929	868.339660644531\\
0.430000007152557	779.475952148438\\
0.435000002384186	693.18994140625\\
0.439999997615814	615.673767089844\\
0.444999992847443	549.345397949219\\
0.449999988079071	496.440124511719\\
0.455000013113022	457.214416503906\\
0.46000000834465	429.727966308594\\
0.465000003576279	411.911865234375\\
0.469999998807907	398.743835449219\\
0.474999994039536	387.623809814453\\
0.479999989271164	375.091979980469\\
0.485000014305115	358.728942871094\\
0.490000009536743	337.552185058594\\
0.495000004768372	312.594635009766\\
0.5	283.927825927734\\
0.504999995231628	254.622512817383\\
0.509999990463257	227.705062866211\\
0.514999985694885	206.384017944336\\
0.519999980926514	192.890472412109\\
0.524999976158142	189.72607421875\\
0.529999971389771	198.505081176758\\
0.535000026226044	213.961990356445\\
0.540000021457672	233.552505493164\\
0.545000016689301	255.580474853516\\
0.550000011920929	278.721893310547\\
0.555000007152557	302.364166259766\\
0.560000002384186	325.106353759766\\
0.564999997615814	346.709320068359\\
0.569999992847443	367.714141845703\\
0.574999988079071	388.887115478516\\
0.579999983310699	410.548095703125\\
0.584999978542328	433.58056640625\\
0.589999973773956	459.12451171875\\
0.595000028610229	486.681060791016\\
0.600000023841858	516.859497070313\\
0.605000019073486	549.077026367188\\
0.610000014305115	583.135131835938\\
0.615000009536743	618.720581054688\\
0.620000004768372	654.764221191406\\
0.625	690.872253417969\\
0.629999995231628	726.876037597656\\
0.634999990463257	762.207946777344\\
0.639999985694885	796.597900390625\\
0.644999980926514	829.836303710938\\
0.649999976158142	861.928588867188\\
0.654999971389771	892.844116210938\\
0.660000026226044	922.753784179688\\
0.665000021457672	951.691589355469\\
0.670000016689301	979.698303222656\\
0.675000011920929	1006.83093261719\\
0.680000007152557	1033.06042480469\\
0.685000002384186	1058.28527832031\\
0.689999997615814	1082.40551757813\\
0.694999992847443	1105.25268554688\\
0.699999988079071	1126.67724609375\\
0.704999983310699	1146.44104003906\\
0.709999978542328	1164.46630859375\\
0.714999973773956	1180.67333984375\\
0.720000028610229	1195.00317382813\\
0.725000023841858	1207.44421386719\\
0.730000019073486	1217.97766113281\\
0.735000014305115	1226.71240234375\\
0.740000009536743	1233.64624023438\\
0.745000004768372	1238.86755371094\\
0.75	1242.41870117188\\
0.754999995231628	1244.36499023438\\
0.759999990463257	1244.74560546875\\
0.764999985694885	1244.11145019531\\
0.769999980926514	1242.38427734375\\
0.774999976158142	1239.17810058594\\
0.779999971389771	1234.341796875\\
0.785000026226044	1227.95190429688\\
0.790000021457672	1220.1025390625\\
0.795000016689301	1210.88891601563\\
0.800000011920929	1200.40698242188\\
0.805000007152557	1188.73815917969\\
0.810000002384186	1176.00048828125\\
0.814999997615814	1162.27294921875\\
0.819999992847443	1147.64758300781\\
0.824999988079071	1132.27380371094\\
0.829999983310699	1116.22021484375\\
0.834999978542328	1099.60217285156\\
0.839999973773956	1082.55700683594\\
0.845000028610229	1065.14416503906\\
0.850000023841858	1047.49426269531\\
0.855000019073486	1029.67492675781\\
0.860000014305115	1011.75805664063\\
0.865000009536743	993.79638671875\\
0.870000004768372	975.87158203125\\
0.875	958.065246582031\\
0.879999995231628	940.453552246094\\
0.884999990463257	923.105407714844\\
0.889999985694885	906.092468261719\\
0.894999980926514	889.49365234375\\
0.899999976158142	873.370178222656\\
0.904999971389771	857.781311035156\\
0.910000026226044	842.821166992188\\
0.915000021457672	828.527038574219\\
0.920000016689301	814.927917480469\\
0.925000011920929	802.113159179688\\
0.930000007152557	790.122863769531\\
0.935000002384186	778.981201171875\\
0.939999997615814	768.7451171875\\
0.944999992847443	759.434265136719\\
0.949999988079071	751.006164550781\\
0.954999983310699	743.492248535156\\
0.959999978542328	736.880187988281\\
0.964999973773956	731.165344238281\\
0.970000028610229	726.3583984375\\
0.975000023841858	722.450866699219\\
0.980000019073486	719.434814453125\\
0.985000014305115	717.326049804688\\
0.990000009536743	716.142639160156\\
0.995000004768372	715.902954101563\\
1	716.643127441406\\
1.00499999523163	718.408508300781\\
1.00999999046326	721.200500488281\\
1.01499998569489	724.621704101563\\
1.01999998092651	728.635681152344\\
1.02499997615814	733.226989746094\\
1.02999997138977	738.424377441406\\
1.0349999666214	744.165222167969\\
1.03999996185303	750.437255859375\\
1.04499995708466	757.197937011719\\
1.04999995231628	764.427429199219\\
1.05499994754791	772.074829101563\\
1.05999994277954	780.102905273438\\
1.06500005722046	788.486694335938\\
1.07000005245209	797.154846191406\\
1.07500004768372	806.074279785156\\
1.08000004291534	815.192932128906\\
1.08500003814697	824.470336914063\\
1.0900000333786	833.865234375\\
1.09500002861023	843.332946777344\\
1.10000002384186	852.830200195313\\
1.10500001907349	862.313842773438\\
1.11000001430511	871.740234375\\
1.11500000953674	881.062438964844\\
1.12000000476837	890.237548828125\\
1.125	899.235229492188\\
1.12999999523163	908.012878417969\\
1.13499999046326	916.52685546875\\
1.13999998569489	924.745300292969\\
1.14499998092651	932.666198730469\\
1.14999997615814	940.243286132813\\
1.15499997138977	947.443786621094\\
1.1599999666214	954.264404296875\\
1.16499996185303	960.686279296875\\
1.16999995708466	966.678894042969\\
1.17499995231628	972.222229003906\\
1.17999994754791	977.323425292969\\
1.18499994277954	981.9521484375\\
1.19000005722046	986.090270996094\\
1.19500005245209	989.74853515625\\
1.20000004768372	992.939086914063\\
1.20500004291534	995.657470703125\\
1.21000003814697	997.908447265625\\
1.2150000333786	999.674621582031\\
1.22000002861023	1000.9609375\\
1.22500002384186	1001.77136230469\\
1.23000001907349	1002.14111328125\\
1.23500001430511	1002.12255859375\\
1.24000000953674	1001.73999023438\\
1.24500000476837	1001.03295898438\\
1.25	999.882568359375\\
1.25499999523163	998.3232421875\\
1.25999999046326	996.343566894531\\
1.26499998569489	993.958923339844\\
1.26999998092651	991.215454101563\\
1.27499997615814	988.109558105469\\
1.27999997138977	984.668762207031\\
1.2849999666214	980.989013671875\\
1.28999996185303	977.071594238281\\
1.29499995708466	972.946899414063\\
1.29999995231628	968.653625488281\\
1.30499994754791	964.211853027344\\
1.30999994277954	959.652648925781\\
1.31500005722046	954.973083496094\\
1.32000005245209	950.197631835938\\
1.32500004768372	945.349304199219\\
1.33000004291534	940.439758300781\\
1.33500003814697	935.486877441406\\
1.3400000333786	930.511901855469\\
1.34500002861023	925.533935546875\\
1.35000002384186	920.574523925781\\
1.35500001907349	915.655090332031\\
1.36000001430511	910.8125\\
1.36500000953674	906.123229980469\\
1.37000000476837	901.630615234375\\
1.375	897.373962402344\\
1.37999999523163	893.262512207031\\
1.38499999046326	889.342529296875\\
1.38999998569489	885.633972167969\\
1.39499998092651	882.100524902344\\
1.39999997615814	878.752502441406\\
1.40499997138977	875.595764160156\\
1.4099999666214	872.661071777344\\
1.41499996185303	869.9501953125\\
1.41999995708466	867.472045898438\\
1.42499995231628	865.237121582031\\
1.42999994754791	863.263122558594\\
1.43499994277954	861.552734375\\
1.44000005722046	860.097961425781\\
1.44500005245209	858.91064453125\\
1.45000004768372	857.985961914063\\
1.45500004291534	857.313781738281\\
1.46000003814697	856.899841308594\\
1.4650000333786	856.763732910156\\
1.47000002861023	856.933349609375\\
1.47500002384186	857.426940917969\\
1.48000001907349	858.145385742188\\
1.48500001430511	858.9921875\\
1.49000000953674	859.890991210938\\
1.49500000476837	861.072326660156\\
1.5	862.574584960938\\
1.50499999523163	864.374450683594\\
1.50999999046326	866.366821289063\\
1.51499998569489	868.546875\\
1.51999998092651	870.901000976563\\
1.52499997615814	873.414123535156\\
1.52999997138977	876.072509765625\\
1.5349999666214	878.889221191406\\
1.53999996185303	881.856872558594\\
1.54499995708466	884.942504882813\\
1.54999995231628	888.102233886719\\
1.55499994754791	891.318298339844\\
1.55999994277954	894.570617675781\\
1.56500005722046	897.84033203125\\
1.57000005245209	890.160583496094\\
1.57500004768372	889.786804199219\\
1.58000004291534	893.3857421875\\
1.58500003814697	898.15380859375\\
1.5900000333786	902.817626953125\\
1.59500002861023	907.101257324219\\
1.60000002384186	910.888122558594\\
1.60500001907349	914.133850097656\\
1.61000001430511	916.807800292969\\
1.61500000953674	919.004760742188\\
1.62000000476837	920.863464355469\\
1.625	922.430419921875\\
1.62999999523163	923.861206054688\\
1.63499999046326	925.2373046875\\
1.63999998569489	926.677001953125\\
1.64499998092651	928.292358398438\\
1.64999997615814	929.935729980469\\
1.65499997138977	931.599853515625\\
1.6599999666214	933.254455566406\\
1.66499996185303	934.822143554688\\
1.66999995708466	936.220275878906\\
1.67499995231628	937.48681640625\\
1.67999994754791	938.581481933594\\
1.68499994277954	939.437133789063\\
1.69000005722046	939.983947753906\\
1.69500005245209	940.351623535156\\
1.70000004768372	940.566345214844\\
1.70500004291534	940.623168945313\\
1.71000003814697	940.601318359375\\
1.7150000333786	940.499938964844\\
1.72000002861023	940.321655273438\\
1.72500002384186	940.034545898438\\
1.73000001907349	939.658386230469\\
1.73500001430511	939.202697753906\\
1.74000000953674	938.670227050781\\
1.74500000476837	938.067321777344\\
1.75	937.40869140625\\
1.75499999523163	936.686157226563\\
1.75999999046326	935.908203125\\
1.76499998569489	935.0703125\\
1.76999998092651	934.172302246094\\
1.77499997615814	933.216247558594\\
1.77999997138977	932.196655273438\\
1.7849999666214	931.114318847656\\
1.78999996185303	929.971435546875\\
1.79499995708466	928.773742675781\\
1.79999995231628	927.536987304688\\
1.80499994754791	926.267822265625\\
1.80999994277954	924.97900390625\\
1.81500005722046	923.682067871094\\
1.82000005245209	922.386779785156\\
1.82500004768372	921.101196289063\\
1.83000004291534	919.83447265625\\
1.83500003814697	918.6005859375\\
1.8400000333786	917.39697265625\\
1.84500002861023	916.223205566406\\
1.85000002384186	915.081726074219\\
1.85500001907349	913.96484375\\
1.86000001430511	912.8759765625\\
1.86500000953674	911.820495605469\\
1.87000000476837	910.795471191406\\
1.875	909.812866210938\\
1.87999999523163	908.880126953125\\
1.88499999046326	907.988037109375\\
1.88999998569489	907.137268066406\\
1.89499998092651	906.330017089844\\
1.89999997615814	905.574279785156\\
1.90499997138977	904.866760253906\\
1.9099999666214	904.212707519531\\
1.91499996185303	903.611022949219\\
1.91999995708466	903.061340332031\\
1.92499995231628	902.572814941406\\
1.92999994754791	902.142883300781\\
1.93499994277954	901.769165039063\\
1.94000005722046	901.44189453125\\
1.94500005245209	901.156799316406\\
1.95000004768372	900.917907714844\\
1.95500004291534	900.726318359375\\
1.96000003814697	900.580871582031\\
1.9650000333786	900.476440429688\\
1.97000002861023	900.408203125\\
1.97500002384186	900.366821289063\\
1.98000001907349	900.370666503906\\
1.98500001430511	900.414794921875\\
1.99000000953674	900.491577148438\\
1.99500000476837	900.588012695313\\
2	900.703186035156\\
2.00500011444092	900.832153320313\\
2.00999999046326	900.957153320313\\
2.01500010490417	901.070190429688\\
2.01999998092651	901.2158203125\\
2.02500009536743	901.382080078125\\
2.02999997138977	901.567016601563\\
2.03500008583069	901.743591308594\\
2.03999996185303	901.98095703125\\
2.04500007629395	902.278259277344\\
2.04999995231628	902.626281738281\\
2.0550000667572	903.015625\\
2.05999994277954	903.423645019531\\
2.06500005722046	903.847961425781\\
2.0699999332428	904.221557617188\\
2.07500004768372	904.517028808594\\
2.07999992370605	904.663879394531\\
2.08500003814697	904.595703125\\
2.08999991416931	904.255187988281\\
2.09500002861023	903.512573242188\\
2.09999990463257	902.334350585938\\
2.10500001907349	900.663269042969\\
2.10999989509583	898.570495605469\\
2.11500000953674	896.083740234375\\
2.11999988555908	893.023803710938\\
2.125	889.116088867188\\
2.13000011444092	884.454223632813\\
2.13499999046326	879.081481933594\\
2.14000010490417	873.020935058594\\
2.14499998092651	866.373840332031\\
2.15000009536743	859.411071777344\\
2.15499997138977	852.150329589844\\
2.16000008583069	844.532165527344\\
2.16499996185303	836.76708984375\\
2.17000007629395	828.849304199219\\
2.17499995231628	820.742309570313\\
2.1800000667572	812.510070800781\\
2.18499994277954	804.083679199219\\
2.19000005722046	795.468505859375\\
2.1949999332428	786.667236328125\\
2.20000004768372	777.733581542969\\
2.20499992370605	768.725891113281\\
2.21000003814697	759.7431640625\\
2.21499991416931	750.929138183594\\
2.22000002861023	742.466491699219\\
2.22499990463257	734.470764160156\\
2.23000001907349	727.037536621094\\
2.23499989509583	720.255981445313\\
2.24000000953674	714.322814941406\\
2.24499988555908	709.196166992188\\
2.25	704.865600585938\\
2.25500011444092	701.401062011719\\
2.25999999046326	698.762939453125\\
2.26500010490417	696.871398925781\\
2.26999998092651	695.735412597656\\
2.27500009536743	695.359069824219\\
2.27999997138977	695.823608398438\\
2.28500008583069	697.27392578125\\
2.28999996185303	699.930419921875\\
2.29500007629395	703.662353515625\\
2.29999995231628	707.621276855469\\
2.3050000667572	712.433715820313\\
2.30999994277954	718.128601074219\\
2.31500005722046	724.419738769531\\
2.3199999332428	731.290954589844\\
2.32500004768372	738.884643554688\\
2.32999992370605	747.192199707031\\
2.33500003814697	756.21337890625\\
2.33999991416931	767.282470703125\\
2.34500002861023	780.156127929688\\
2.34999990463257	792.679077148438\\
2.35500001907349	805.642211914063\\
2.35999989509583	818.952270507813\\
2.36500000953674	832.724792480469\\
2.36999988555908	847.168395996094\\
2.375	862.1005859375\\
2.38000011444092	877.612976074219\\
2.38499999046326	893.743591308594\\
2.39000010490417	910.484802246094\\
2.39499998092651	927.777221679688\\
2.40000009536743	945.682373046875\\
2.40499997138977	964.139709472656\\
2.41000008583069	983.084167480469\\
2.41499996185303	1002.48937988281\\
2.42000007629395	1022.17919921875\\
2.42499995231628	1042.01684570313\\
2.4300000667572	1061.73205566406\\
2.43499994277954	1081.17736816406\\
2.44000005722046	1100.12268066406\\
2.4449999332428	1118.21655273438\\
2.45000004768372	1135.57751464844\\
2.45499992370605	1151.68591308594\\
2.46000003814697	1166.55419921875\\
2.46499991416931	1180.10754394531\\
2.47000002861023	1192.03723144531\\
2.47499990463257	1202.48608398438\\
2.48000001907349	1211.42956542969\\
2.48499989509583	1218.8701171875\\
2.49000000953674	1224.88977050781\\
2.49499988555908	1229.56079101563\\
2.5	1232.88354492188\\
2.50500011444092	1234.88391113281\\
2.50999999046326	1235.94299316406\\
2.51500010490417	1236.34631347656\\
2.51999998092651	1235.5654296875\\
2.52500009536743	1233.28210449219\\
2.52999997138977	1229.61303710938\\
2.53500008583069	1224.03503417969\\
2.53999996185303	1217.06225585938\\
2.54500007629395	1208.55358886719\\
2.54999995231628	1198.55725097656\\
2.5550000667572	1187.17370605469\\
2.55999994277954	1174.44873046875\\
2.56500005722046	1160.56066894531\\
2.5699999332428	1145.66162109375\\
2.57500004768372	1129.88024902344\\
2.57999992370605	1113.38366699219\\
2.58500003814697	1096.37854003906\\
2.58999991416931	1079.03454589844\\
2.59500002861023	1061.31127929688\\
2.59999990463257	1043.18432617188\\
2.60500001907349	1024.69250488281\\
2.60999989509583	1005.75726318359\\
2.61500000953674	986.304748535156\\
2.61999988555908	966.315979003906\\
2.625	945.787841796875\\
2.63000011444092	924.841247558594\\
2.63499999046326	903.547912597656\\
2.64000010490417	881.925354003906\\
2.64499998092651	860.112426757813\\
2.65000009536743	838.401123046875\\
2.65499997138977	816.903625488281\\
2.66000008583069	795.568725585938\\
2.66499996185303	774.581726074219\\
2.67000007629395	753.825073242188\\
2.67499995231628	733.25341796875\\
2.6800000667572	713.087707519531\\
2.68499994277954	693.046203613281\\
2.69000005722046	673.371887207031\\
2.6949999332428	653.973083496094\\
2.70000004768372	634.744262695313\\
2.70499992370605	615.814819335938\\
2.71000003814697	597.240600585938\\
2.71499991416931	579.050842285156\\
2.72000002861023	561.601928710938\\
2.72499990463257	544.98193359375\\
2.73000001907349	529.284240722656\\
2.73499989509583	514.667785644531\\
2.74000000953674	501.300079345703\\
2.74499988555908	489.414398193359\\
2.75	478.995452880859\\
2.75500011444092	469.985260009766\\
2.75999999046326	462.598876953125\\
2.76500010490417	457.195404052734\\
2.76999998092651	453.911987304688\\
2.77500009536743	452.768676757813\\
2.77999997138977	453.521820068359\\
2.78500008583069	456.216918945313\\
2.78999996185303	460.900360107422\\
2.79500007629395	467.693603515625\\
2.79999995231628	476.672668457031\\
2.8050000667572	487.946716308594\\
2.80999994277954	501.669738769531\\
2.81500005722046	517.89306640625\\
2.8199999332428	536.623474121094\\
2.82500004768372	557.875183105469\\
2.82999992370605	581.467346191406\\
2.83500003814697	607.680969238281\\
2.83999991416931	636.209716796875\\
2.84500002861023	666.966552734375\\
2.84999990463257	700.137817382813\\
2.85500001907349	735.021667480469\\
2.85999989509583	772.035583496094\\
2.86500000953674	811.089965820313\\
2.86999988555908	853.0146484375\\
2.875	896.970275878906\\
2.88000011444092	943.883361816406\\
2.88499999046326	993.905334472656\\
2.89000010490417	1046.65625\\
2.89499998092651	1101.73596191406\\
2.90000009536743	1156.76123046875\\
2.90499997138977	1210.42919921875\\
2.91000008583069	1261.26501464844\\
2.91499996185303	1307.912109375\\
2.92000007629395	1349.646484375\\
2.92499995231628	1386.2109375\\
2.9300000667572	1417.73413085938\\
2.93499994277954	1444.4873046875\\
2.94000005722046	1467.35888671875\\
2.9449999332428	1486.69750976563\\
2.95000004768372	1502.58776855469\\
2.95499992370605	1515.20227050781\\
2.96000003814697	1525.39111328125\\
2.96499991416931	1532.21105957031\\
2.97000002861023	1535.42724609375\\
2.97499990463257	1533.69885253906\\
2.98000001907349	1526.17309570313\\
2.98499989509583	1519.2578125\\
2.99000000953674	1503.99353027344\\
2.99499988555908	1484.08801269531\\
3	1458.880859375\\
3.00500011444092	1428.76281738281\\
3.00999999046326	1394.06616210938\\
3.01500010490417	1355.48156738281\\
3.01999998092651	1313.33569335938\\
3.02500009536743	1268.48107910156\\
3.02999997138977	1222.07861328125\\
3.03500008583069	1174.47900390625\\
3.03999996185303	1126.08654785156\\
3.04500007629395	1077.17822265625\\
3.04999995231628	1027.98168945313\\
3.0550000667572	978.331909179688\\
3.05999994277954	928.004638671875\\
3.06500005722046	876.997131347656\\
3.0699999332428	825.286437988281\\
3.07500004768372	772.968688964844\\
3.07999992370605	720.638305664063\\
3.08500003814697	668.733825683594\\
3.08999991416931	617.9072265625\\
3.09500002861023	569.012817382813\\
3.09999990463257	522.56689453125\\
3.10500001907349	478.959045410156\\
3.10999989509583	438.795928955078\\
3.11500000953674	402.603576660156\\
3.11999988555908	370.529418945313\\
3.125	342.683563232422\\
3.13000011444092	319.632110595703\\
3.13499999046326	301.725494384766\\
3.14000010490417	288.252838134766\\
3.14499998092651	279.787200927734\\
3.15000009536743	277.897186279297\\
3.15499997138977	283.669982910156\\
3.16000008583069	295.801116943359\\
3.16499996185303	313.656829833984\\
3.17000007629395	337.000885009766\\
3.17499995231628	365.480407714844\\
3.1800000667572	399.077117919922\\
3.18499994277954	436.689514160156\\
3.19000005722046	477.9111328125\\
3.1949999332428	521.66552734375\\
3.20000004768372	567.137756347656\\
3.20499992370605	614.523864746094\\
3.21000003814697	662.385009765625\\
3.21499991416931	711.325927734375\\
3.22000002861023	760.475219726563\\
3.22499990463257	810.274169921875\\
3.23000001907349	860.397155761719\\
3.23499989509583	910.835205078125\\
3.24000000953674	961.076721191406\\
3.24499988555908	1010.12384033203\\
3.25	1057.77038574219\\
3.25500011444092	1103.76879882813\\
3.25999999046326	1148.01025390625\\
3.26500010490417	1190.45068359375\\
3.26999998092651	1231.24267578125\\
3.27500009536743	1270.76843261719\\
3.27999997138977	1308.9287109375\\
3.28500008583069	1345.54284667969\\
3.28999996185303	1380.25720214844\\
3.29500007629395	1412.23376464844\\
3.29999995231628	1440.69396972656\\
3.3050000667572	1464.880859375\\
3.30999994277954	1484.11303710938\\
3.31500005722046	1497.86340332031\\
3.3199999332428	1508.12585449219\\
3.32500004768372	1513.58447265625\\
3.32999992370605	1512.74719238281\\
3.33500003814697	1505.41528320313\\
3.33999991416931	1492.38989257813\\
3.34500002861023	1474.19213867188\\
3.34999990463257	1451.279296875\\
3.35500001907349	1424.20593261719\\
3.35999989509583	1393.53063964844\\
3.36500000953674	1359.28527832031\\
3.36999988555908	1321.54528808594\\
3.375	1280.27905273438\\
3.38000011444092	1235.59265136719\\
3.38499999046326	1187.40942382813\\
3.39000010490417	1135.82409667969\\
3.39499998092651	1081.13452148438\\
3.40000009536743	1023.66705322266\\
3.40499997138977	964.054138183594\\
3.41000008583069	903.1435546875\\
3.41499996185303	841.242492675781\\
3.42000007629395	779.465942382813\\
3.42499995231628	718.448059082031\\
3.4300000667572	658.935546875\\
3.43499994277954	601.470031738281\\
3.44000005722046	546.560852050781\\
3.4449999332428	494.777526855469\\
3.45000004768372	446.734222412109\\
3.45499992370605	403.056091308594\\
3.46000003814697	364.585845947266\\
3.46499991416931	331.079345703125\\
3.47000002861023	302.149536132813\\
3.47499990463257	279.768432617188\\
3.48000001907349	263.154998779297\\
3.48499989509583	253.630813598633\\
3.49000000953674	254.281967163086\\
3.49499988555908	263.360778808594\\
3.5	279.547546386719\\
3.50500011444092	302.261993408203\\
3.50999999046326	331.428283691406\\
3.51500010490417	368.192077636719\\
3.51999998092651	412.427154541016\\
3.52500009536743	466.1787109375\\
3.52999997138977	530.476257324219\\
3.53500008583069	608.476318359375\\
3.53999996185303	702.491760253906\\
3.54500007629395	811.812805175781\\
3.54999995231628	929.061706542969\\
3.5550000667572	1041.87756347656\\
3.55999994277954	1138.40173339844\\
3.56500005722046	1212.62158203125\\
3.5699999332428	1266.05432128906\\
3.57500004768372	1300.84289550781\\
3.57999992370605	1325.62609863281\\
3.58500003814697	1341.94104003906\\
3.58999991416931	1353.32092285156\\
3.59500002861023	1367.60083007813\\
3.59999990463257	1392.06604003906\\
3.60500001907349	1424.42895507813\\
3.60999989509583	1461.60803222656\\
3.61500000953674	1498.80651855469\\
3.61999988555908	1530.98278808594\\
3.625	1554.69274902344\\
3.63000011444092	1567.03674316406\\
3.63499999046326	1569.32946777344\\
3.64000010490417	1561.97021484375\\
3.64499998092651	1540.64721679688\\
3.65000009536743	1505.7490234375\\
3.65499997138977	1458.53588867188\\
3.66000008583069	1403.32934570313\\
3.66499996185303	1345.14367675781\\
3.67000007629395	1287.00244140625\\
3.67499995231628	1231.04333496094\\
3.6800000667572	1177.88977050781\\
3.68499994277954	1127.14306640625\\
3.69000005722046	1077.22863769531\\
3.6949999332428	1026.06103515625\\
3.70000004768372	971.927917480469\\
3.70499992370605	913.142272949219\\
3.71000003814697	849.097473144531\\
3.71499991416931	780.400024414063\\
3.72000002861023	708.288879394531\\
3.72499990463257	635.047241210938\\
3.73000001907349	562.463684082031\\
3.73499989509583	493.890502929688\\
3.74000000953674	430.34521484375\\
3.74499988555908	374.146636962891\\
3.75	327.660980224609\\
3.75500011444092	290.320709228516\\
3.75999999046326	262.057952880859\\
3.76500010490417	241.767059326172\\
3.76999998092651	229.559463500977\\
3.77500009536743	229.175018310547\\
3.77999997138977	239.278396606445\\
3.78500008583069	258.466491699219\\
3.78999996185303	286.359375\\
3.79500007629395	322.829895019531\\
3.79999995231628	366.692077636719\\
3.8050000667572	418.826354980469\\
3.80999994277954	478.767700195313\\
3.81500005722046	548.347045898438\\
3.8199999332428	628.789916992188\\
3.82500004768372	721.746643066406\\
3.82999992370605	828.724853515625\\
3.83500003814697	945.444396972656\\
3.83999991416931	1062.33435058594\\
3.84500002861023	1166.92126464844\\
3.84999990463257	1247.34411621094\\
3.85500001907349	1300.71752929688\\
3.85999989509583	1336.5048828125\\
3.86500000953674	1357.36022949219\\
3.86999988555908	1364.98559570313\\
3.875	1367.33801269531\\
3.88000011444092	1373.92956542969\\
3.88499999046326	1399.16796875\\
3.89000010490417	1442.19262695313\\
3.89499998092651	1495.68530273438\\
3.90000009536743	1553.21472167969\\
3.90499997138977	1604.39086914063\\
3.91000008583069	1644.56652832031\\
3.91499996185303	1669.14819335938\\
3.92000007629395	1678.42541503906\\
3.92499995231628	1675.07214355469\\
3.9300000667572	1653.787109375\\
3.93499994277954	1613.60290527344\\
3.94000005722046	1558.19689941406\\
3.9449999332428	1493.27563476563\\
3.95000004768372	1424.88403320313\\
3.95499992370605	1357.52893066406\\
3.96000003814697	1292.47241210938\\
3.96499991416931	1230.94116210938\\
3.97000002861023	1171.79565429688\\
3.97499990463257	1113.14660644531\\
3.98000001907349	1051.94128417969\\
3.98499989509583	985.659118652344\\
3.99000000953674	912.95751953125\\
3.99499988555908	833.645385742188\\
4	748.239135742188\\
4.00500011444092	658.415466308594\\
4.01000022888184	566.999938964844\\
4.0149998664856	477.855499267578\\
4.01999998092651	393.227111816406\\
4.02500009536743	315.94677734375\\
4.03000020980835	248.53239440918\\
4.03499984741211	192.352005004883\\
4.03999996185303	146.582443237305\\
4.04500007629395	110.659103393555\\
4.05000019073486	84.4329605102539\\
4.05499982833862	67.1992568969727\\
4.05999994277954	59.4867134094238\\
4.06500005722046	65.0375213623047\\
4.07000017166138	81.3378448486328\\
4.07499980926514	107.074653625488\\
4.07999992370605	141.929901123047\\
4.08500003814697	187.910186767578\\
4.09000015258789	250.009399414063\\
4.09499979019165	335.343719482422\\
4.09999990463257	459.931762695313\\
4.10500001907349	649.385131835938\\
4.1100001335144	904.818908691406\\
4.11499977111816	1176.91870117188\\
4.11999988555908	1402.42663574219\\
4.125	1553.06896972656\\
4.13000011444092	1630.45141601563\\
4.13500022888184	1651.07263183594\\
4.1399998664856	1613.77673339844\\
4.14499998092651	1533.70129394531\\
4.15000009536743	1441.63671875\\
4.15500020980835	1369.62182617188\\
4.15999984741211	1345.927734375\\
4.16499996185303	1394.56884765625\\
4.17000007629395	1488.61145019531\\
4.17500019073486	1592.86486816406\\
4.17999982833862	1690.15368652344\\
4.18499994277954	1766.509765625\\
4.19000005722046	1808.81469726563\\
4.19500017166138	1814.06896972656\\
4.19999980926514	1798.01025390625\\
4.20499992370605	1750.54223632813\\
4.21000003814697	1664.98132324219\\
4.21500015258789	1554.470703125\\
4.21999979019165	1435.1611328125\\
4.22499990463257	1320.69067382813\\
4.23000001907349	1221.92163085938\\
4.2350001335144	1141.40625\\
4.23999977111816	1075.63342285156\\
4.24499988555908	1019.86376953125\\
4.25	969.322326660156\\
4.25500011444092	915.307067871094\\
4.26000022888184	852.963989257813\\
4.2649998664856	778.060791015625\\
4.26999998092651	689.920166015625\\
4.27500009536743	591.377380371094\\
4.28000020980835	487.326873779297\\
4.28499984741211	380.143157958984\\
4.28999996185303	278.507415771484\\
4.29500007629395	189.033096313477\\
4.30000019073486	117.116233825684\\
4.30499982833862	63.9741630554199\\
4.30999994277954	26.8671607971191\\
4.31500005722046	6.21131801605225\\
4.32000017166138	1.54345905780792\\
4.32499980926514	11.6016340255737\\
4.32999992370605	45.0855522155762\\
4.33500003814697	91.0278549194336\\
4.34000015258789	142.152633666992\\
4.34499979019165	202.464492797852\\
4.34999990463257	276.69873046875\\
4.35500001907349	372.777618408203\\
4.3600001335144	503.735168457031\\
4.36499977111816	695.948608398438\\
4.36999988555908	962.507019042969\\
4.375	1265.0732421875\\
4.38000011444092	1520.56848144531\\
4.38500022888184	1689.45031738281\\
4.3899998664856	1762.45178222656\\
4.39499998092651	1813.27966308594\\
4.40000009536743	1764.67700195313\\
4.40500020980835	1641.78186035156\\
4.40999984741211	1489.52807617188\\
4.41499996185303	1357.49621582031\\
4.42000007629395	1284.70935058594\\
4.42500019073486	1325.55688476563\\
4.42999982833862	1448.66638183594\\
4.43499994277954	1599.71618652344\\
4.44000005722046	1747.6943359375\\
4.44500017166138	1867.60424804688\\
4.44999980926514	1940.60998535156\\
4.45499992370605	1958.07336425781\\
4.46000003814697	1946.06628417969\\
4.46500015258789	1891.07275390625\\
4.46999979019165	1783.94567871094\\
4.47499990463257	1641.822265625\\
4.48000001907349	1488.03210449219\\
4.4850001335144	1342.56457519531\\
4.48999977111816	1220.10949707031\\
4.49499988555908	1126.67492675781\\
4.5	1054.42724609375\\
4.50500011444092	997.334594726563\\
4.51000022888184	946.364440917969\\
4.5149998664856	891.733215332031\\
4.51999998092651	824.632019042969\\
4.52500009536743	741.026306152344\\
4.53000020980835	639.909851074219\\
4.53499984741211	525.250793457031\\
4.53999996185303	404.152008056641\\
4.54500007629395	278.254547119141\\
4.55000019073486	174.261505126953\\
4.55499982833862	57.4531402587891\\
4.55999994277954	5.19539833068848\\
4.56500005722046	-4.75389909744263\\
4.57000017166138	-2.34561800956726\\
4.57499980926514	1.55156219005585\\
4.57999992370605	3.92794561386108\\
4.58500003814697	3.94009590148926\\
4.59000015258789	3.09818577766418\\
4.59499979019165	2.21306896209717\\
4.59999990463257	-0.0706229582428932\\
4.60500001907349	-4.41154003143311\\
4.6100001335144	-11.4452686309814\\
4.61499977111816	-22.2442626953125\\
4.61999988555908	134.761077880859\\
4.625	633.037902832031\\
4.63000011444092	1288.38635253906\\
4.63500022888184	1834.49584960938\\
4.6399998664856	2192.22192382813\\
4.64499998092651	2349.60791015625\\
4.65000009536743	2373.703125\\
4.65500020980835	2276.8515625\\
4.65999984741211	2047.97631835938\\
4.66499996185303	1839.78442382813\\
4.67000007629395	1450.84387207031\\
4.67500019073486	1125.4287109375\\
4.67999982833862	927.637573242188\\
4.68499994277954	947.531921386719\\
4.69000005722046	1144.80065917969\\
4.69500017166138	1400.39245605469\\
4.69999980926514	1654.60986328125\\
4.70499992370605	1860.47985839844\\
4.71000003814697	1983.44860839844\\
4.71500015258789	2008.96118164063\\
4.71999979019165	1982.35473632813\\
4.72499990463257	1893.87573242188\\
4.73000001907349	1717.70129394531\\
4.7350001335144	1490.01245117188\\
4.73999977111816	1250.52685546875\\
4.74499988555908	1035.72595214844\\
4.75	869.228698730469\\
4.75500011444092	765.985534667969\\
4.76000022888184	704.991638183594\\
4.7649998664856	678.603942871094\\
4.76999998092651	658.15185546875\\
4.77500009536743	634.701904296875\\
4.78000020980835	592.067810058594\\
4.78499984741211	528.682373046875\\
4.78999996185303	437.965179443359\\
4.79500007629395	312.806701660156\\
4.80000019073486	196.219421386719\\
4.80499982833862	63.3141937255859\\
4.80999994277954	22.957498550415\\
4.81500005722046	14.1650457382202\\
4.82000017166138	15.6754598617554\\
4.82499980926514	18.5081348419189\\
4.82999992370605	18.8501110076904\\
4.83500003814697	15.5495138168335\\
4.84000015258789	11.3836088180542\\
4.84499979019165	2.07820749282837\\
4.84999990463257	-12.6671190261841\\
4.85500001907349	-33.3148727416992\\
4.8600001335144	-57.8537826538086\\
4.86499977111816	-72.4036254882813\\
4.86999988555908	263.160736083984\\
4.875	860.700927734375\\
4.88000011444092	1504.65051269531\\
4.88500022888184	2023.10827636719\\
4.8899998664856	2345.84594726563\\
4.89499998092651	2464.76806640625\\
4.90000009536743	2468.7666015625\\
4.90500020980835	2332.26806640625\\
4.90999984741211	2052.30053710938\\
4.91499996185303	1658.01379394531\\
4.92000007629395	1302.53308105469\\
4.92500019073486	864.925598144531\\
4.92999982833862	517.543395996094\\
4.93499994277954	401.673797607422\\
4.94000005722046	591.992248535156\\
4.94500017166138	959.237121582031\\
4.94999980926514	1371.18444824219\\
4.95499992370605	1754.94470214844\\
4.96000003814697	2052.95434570313\\
4.96500015258789	2229.98193359375\\
4.96999979019165	2275.53149414063\\
4.97499990463257	2249.76416015625\\
4.98000001907349	2144.80493164063\\
4.9850001335144	1953.94018554688\\
4.98999977111816	1714.76818847656\\
4.99499988555908	1469.28881835938\\
5	1254.64038085938\\
5.00500011444092	1091.71069335938\\
5.01000022888184	989.891357421875\\
5.0149998664856	943.308349609375\\
5.01999998092651	937.023803710938\\
5.02500009536743	935.129455566406\\
5.03000020980835	919.981567382813\\
5.03499984741211	882.770080566406\\
5.03999996185303	832.986267089844\\
5.04500007629395	764.821350097656\\
5.05000019073486	677.786315917969\\
5.05499982833862	575.883728027344\\
5.05999994277954	464.255706787109\\
5.06500005722046	349.043365478516\\
5.07000017166138	235.129669189453\\
5.07499980926514	126.047676086426\\
5.07999992370605	26.8150768280029\\
5.08500003814697	-14.0240497589111\\
5.09000015258789	-21.3663501739502\\
5.09499979019165	-21.0397815704346\\
5.09999990463257	-20.0826816558838\\
5.10500001907349	-13.1734180450439\\
5.1100001335144	-7.69926309585571\\
5.11499977111816	-5.06081914901733\\
5.11999988555908	-3.33988404273987\\
5.125	-2.21407651901245\\
5.13000011444092	-1.60632824897766\\
5.13500022888184	415.447570800781\\
5.1399998664856	859.976196289063\\
5.14499998092651	1151.17663574219\\
5.15000009536743	1285.69262695313\\
5.15500020980835	1298.01013183594\\
5.15999984741211	1318.15405273438\\
5.16499996185303	1224.13415527344\\
5.17000007629395	1094.59814453125\\
5.17500019073486	976.66845703125\\
5.17999982833862	911.444152832031\\
5.18499994277954	942.596435546875\\
5.19000005722046	1034.65734863281\\
5.19500017166138	1156.30676269531\\
5.19999980926514	1277.14611816406\\
5.20499992370605	1383.88940429688\\
5.21000003814697	1464.70239257813\\
5.21500015258789	1513.36315917969\\
5.21999979019165	1527.9765625\\
5.22499990463257	1513.80346679688\\
5.23000001907349	1484.96362304688\\
5.2350001335144	1440.53564453125\\
5.23999977111816	1382.03271484375\\
5.24499988555908	1317.09765625\\
5.25	1254.00024414063\\
5.25500011444092	1197.89428710938\\
5.26000022888184	1153.57214355469\\
5.2649998664856	1120.88403320313\\
5.26999998092651	1103.61083984375\\
5.27500009536743	1095.75720214844\\
5.28000020980835	1092.34851074219\\
5.28499984741211	1089.72741699219\\
5.28999996185303	1086.59118652344\\
5.29500007629395	1081.19006347656\\
5.30000019073486	1074.46655273438\\
5.30499982833862	1064.94580078125\\
5.30999994277954	1048.28454589844\\
5.31500005722046	1021.9638671875\\
5.32000017166138	988.428100585938\\
5.32499980926514	945.026916503906\\
5.32999992370605	895.974487304688\\
5.33500003814697	846.05419921875\\
5.34000015258789	798.2529296875\\
5.34499979019165	756.564453125\\
5.34999990463257	720.830444335938\\
5.35500001907349	691.931091308594\\
5.3600001335144	668.713684082031\\
5.36499977111816	648.8798828125\\
5.36999988555908	628.559814453125\\
5.375	603.956298828125\\
5.38000011444092	571.504272460938\\
5.38500022888184	529.143310546875\\
5.3899998664856	476.706329345703\\
5.39499998092651	416.7021484375\\
5.40000009536743	349.602600097656\\
5.40500020980835	281.850433349609\\
5.40999984741211	219.64778137207\\
5.41499996185303	164.07209777832\\
5.42000007629395	117.549263000488\\
5.42500019073486	79.7161254882813\\
5.42999982833862	50.7523384094238\\
5.43499994277954	32.8662338256836\\
5.44000005722046	9.60488796234131\\
5.44500017166138	-2.67803263664246\\
5.44999980926514	-4.78096199035645\\
5.45499992370605	-3.9048638343811\\
5.46000003814697	-2.77254319190979\\
5.46500015258789	-1.94968497753143\\
5.46999979019165	-1.47163200378418\\
5.47499990463257	-1.21332240104675\\
5.48000001907349	-1.01020228862762\\
5.4850001335144	-0.901321530342102\\
5.48999977111816	-0.812971115112305\\
5.49499988555908	-0.753615617752075\\
5.5	-0.705136597156525\\
5.50500011444092	-0.661663591861725\\
5.51000022888184	-0.623586773872375\\
5.5149998664856	-0.578170418739319\\
5.51999998092651	32.8180160522461\\
5.52500009536743	106.052963256836\\
5.53000020980835	163.835952758789\\
5.53499984741211	206.692947387695\\
5.53999996185303	237.299362182617\\
5.54500007629395	257.535583496094\\
5.55000019073486	268.825714111328\\
5.55499982833862	272.354248046875\\
5.55999994277954	271.427947998047\\
5.56500005722046	265.874847412109\\
5.57000017166138	254.613739013672\\
5.57499980926514	238.371551513672\\
5.57999992370605	217.137481689453\\
5.58500003814697	192.257675170898\\
5.59000015258789	164.796279907227\\
5.59499979019165	136.240447998047\\
5.59999990463257	108.16919708252\\
5.60500001907349	82.0330505371094\\
5.6100001335144	58.9667701721191\\
5.61499977111816	39.6862754821777\\
5.61999988555908	24.5260753631592\\
5.625	13.3602933883667\\
5.63000011444092	5.72600221633911\\
5.63500022888184	0.941133201122284\\
5.6399998664856	-0.902413547039032\\
5.64499998092651	-1.01748430728912\\
5.65000009536743	-0.836591958999634\\
5.65500020980835	-0.634318232536316\\
5.65999984741211	-0.449430227279663\\
5.66499996185303	-0.335526913404465\\
5.67000007629395	2.42653608322144\\
5.67500019073486	66.7674407958984\\
5.67999982833862	35.9223899841309\\
5.68499994277954	27.4871692657471\\
5.69000005722046	157.72526550293\\
5.69500017166138	285.065307617188\\
5.69999980926514	385.318878173828\\
5.70499992370605	452.605041503906\\
5.71000003814697	482.326843261719\\
5.71500015258789	482.860260009766\\
5.71999979019165	460.483551025391\\
5.72499990463257	409.262115478516\\
5.73000001907349	340.0478515625\\
5.7350001335144	264.796325683594\\
5.73999977111816	196.551651000977\\
5.74499988555908	143.615188598633\\
5.75	109.696228027344\\
5.75500011444092	93.3838424682617\\
5.76000022888184	92.3128128051758\\
5.7649998664856	98.4079055786133\\
5.76999998092651	104.845489501953\\
5.77500009536743	109.238090515137\\
5.78000020980835	111.339797973633\\
5.78499984741211	112.347541809082\\
5.78999996185303	112.994834899902\\
5.79500007629395	114.01473236084\\
5.80000019073486	116.069221496582\\
5.80499982833862	119.613861083984\\
5.80999994277954	124.114631652832\\
5.81500005722046	129.264785766602\\
5.82000017166138	134.559188842773\\
5.82499980926514	140.004547119141\\
5.82999992370605	144.237350463867\\
5.83500003814697	147.023147583008\\
5.84000015258789	148.306823730469\\
5.84499979019165	146.924285888672\\
5.84999990463257	141.843643188477\\
5.85500001907349	133.381637573242\\
5.8600001335144	120.347839355469\\
5.86499977111816	102.117980957031\\
5.86999988555908	77.3667831420898\\
5.875	45.6422691345215\\
5.88000011444092	9.40395927429199\\
5.88500022888184	-27.5439968109131\\
5.8899998664856	-54.3561706542969\\
5.89499998092651	-22.245677947998\\
5.90000009536743	24.56227684021\\
5.90500020980835	129.765106201172\\
5.90999984741211	359.020263671875\\
5.91499996185303	724.013061523438\\
5.92000007629395	1174.28662109375\\
5.92500019073486	1646.07934570313\\
5.92999982833862	2090.58154296875\\
5.93499994277954	2477.26782226563\\
5.94000005722046	2797.33666992188\\
5.94500017166138	3047.9765625\\
5.94999980926514	3229.9453125\\
5.95499992370605	3344.9326171875\\
5.96000003814697	3425.77490234375\\
5.96500015258789	3445.52709960938\\
5.96999979019165	3365.71801757813\\
5.97499990463257	3160.75512695313\\
5.98000001907349	2833.75708007813\\
5.9850001335144	2429.58227539063\\
5.98999977111816	1936.37121582031\\
5.99499988555908	1517.43408203125\\
6	1318.00720214844\\
6.00500011444092	1545.49182128906\\
6.01000022888184	2027.01586914063\\
6.0149998664856	2617.83154296875\\
6.01999998092651	3214.39282226563\\
6.02500009536743	3728.15893554688\\
6.03000020980835	4097.5361328125\\
6.03499984741211	4289.09521484375\\
6.03999996185303	4308.158203125\\
6.04500007629395	4249.0869140625\\
6.05000019073486	4074.75708007813\\
6.05499982833862	3781.61865234375\\
6.05999994277954	3411.30029296875\\
6.06500005722046	3018.68798828125\\
6.07000017166138	2657.74438476563\\
6.07499980926514	2371.2314453125\\
6.07999992370605	2180.259765625\\
6.08500003814697	2089.96801757813\\
6.09000015258789	2109.79345703125\\
6.09499979019165	2177.78100585938\\
6.09999990463257	2244.64111328125\\
6.10500001907349	2280.48315429688\\
6.1100001335144	2269.45043945313\\
6.11499977111816	2229.96728515625\\
6.11999988555908	2141.17236328125\\
6.125	1993.36389160156\\
6.13000011444092	1793.6220703125\\
6.13500022888184	1553.1884765625\\
6.1399998664856	1293.62670898438\\
6.14499998092651	1037.76379394531\\
6.15000009536743	805.323303222656\\
6.15500020980835	614.883911132813\\
6.15999984741211	468.333984375\\
6.16499996185303	368.930236816406\\
6.17000007629395	309.241180419922\\
6.17500019073486	279.892181396484\\
6.17999982833862	267.512939453125\\
6.18499994277954	258.815399169922\\
6.19000005722046	255.260025024414\\
6.19500017166138	227.299865722656\\
6.19999980926514	177.142440795898\\
6.20499992370605	111.058578491211\\
6.21000003814697	38.3680000305176\\
6.21500015258789	-0.207256704568863\\
6.21999979019165	-8.53025245666504\\
6.22499990463257	-7.58105945587158\\
6.23000001907349	-5.55664396286011\\
6.2350001335144	-3.97735714912415\\
6.23999977111816	-2.85837292671204\\
6.24499988555908	-2.27604579925537\\
6.25	-1.95991003513336\\
6.25500011444092	-1.76877331733704\\
6.26000022888184	-1.63863170146942\\
6.2649998664856	-1.53941190242767\\
6.26999998092651	-1.45481526851654\\
6.27500009536743	-1.37389671802521\\
6.28000020980835	-1.29430615901947\\
6.28499984741211	-1.21674263477325\\
6.28999996185303	-1.14030230045319\\
6.29500007629395	-1.0646755695343\\
6.30000019073486	-0.992218136787415\\
6.30499982833862	-0.921954810619354\\
6.30999994277954	-0.854799270629883\\
6.31500005722046	-0.791765034198761\\
6.32000017166138	-0.730815529823303\\
6.32499980926514	-0.673609495162964\\
6.32999992370605	-0.621738970279694\\
6.33500003814697	-0.574325203895569\\
6.34000015258789	-0.532747626304626\\
6.34499979019165	-0.491829425096512\\
6.34999990463257	-0.451771318912506\\
6.35500001907349	-0.413074016571045\\
6.3600001335144	42.3081321716309\\
6.36499977111816	33.3051414489746\\
6.36999988555908	26.3421211242676\\
6.375	18.9876232147217\\
6.38000011444092	10.3159017562866\\
6.38500022888184	44.1340408325195\\
6.3899998664856	329.966339111328\\
6.39499998092651	596.893493652344\\
6.40000009536743	814.4013671875\\
6.40500020980835	987.07177734375\\
6.40999984741211	1124.03747558594\\
6.41499996185303	1223.37475585938\\
6.42000007629395	1293.99047851563\\
6.42500019073486	1344.02258300781\\
6.42999982833862	1380.650390625\\
6.43499994277954	1412.19616699219\\
6.44000005722046	1445.91235351563\\
6.44500017166138	1485.82141113281\\
6.44999980926514	1534.66809082031\\
6.45499992370605	1593.5244140625\\
6.46000003814697	1660.72863769531\\
6.46500015258789	1733.615234375\\
6.46999979019165	1808.77087402344\\
6.47499990463257	1882.66479492188\\
6.48000001907349	1951.81884765625\\
6.4850001335144	2013.83996582031\\
6.48999977111816	2066.7802734375\\
6.49499988555908	2110.14770507813\\
6.5	2143.7314453125\\
6.50500011444092	2168.43432617188\\
6.51000022888184	2185.5185546875\\
6.5149998664856	2196.3125\\
6.51999998092651	2202.33374023438\\
6.52500009536743	2205.14501953125\\
6.53000020980835	2205.62866210938\\
6.53499984741211	2204.65112304688\\
6.53999996185303	2202.51440429688\\
6.54500007629395	2199.47143554688\\
6.55000019073486	2195.31127929688\\
6.55499982833862	2189.70971679688\\
6.55999994277954	2182.83276367188\\
6.56500005722046	2174.41650390625\\
6.57000017166138	2163.77001953125\\
6.57499980926514	2150.54638671875\\
6.57999992370605	2134.46606445313\\
6.58500003814697	2115.49877929688\\
6.59000015258789	2093.85424804688\\
6.59499979019165	2069.77270507813\\
6.59999990463257	2043.64685058594\\
6.60500001907349	2016.00512695313\\
6.6100001335144	1987.41467285156\\
6.61499977111816	1958.44750976563\\
6.61999988555908	1929.57092285156\\
6.625	1901.29064941406\\
6.63000011444092	1873.99011230469\\
6.63500022888184	1847.90441894531\\
6.6399998664856	1823.44409179688\\
6.64499998092651	1800.39868164063\\
6.65000009536743	1778.95727539063\\
6.65500020980835	1759.07275390625\\
6.65999984741211	1740.53308105469\\
6.66499996185303	1723.39770507813\\
6.67000007629395	1707.89880371094\\
6.67500019073486	1693.52136230469\\
6.67999982833862	1680.49658203125\\
6.68499994277954	1669.02954101563\\
6.69000005722046	1658.55017089844\\
6.69500017166138	1650.32727050781\\
6.69999980926514	1644.69006347656\\
6.70499992370605	1641.21313476563\\
6.71000003814697	1639.31726074219\\
6.71500015258789	1638.9521484375\\
6.71999979019165	1640.07690429688\\
6.72499990463257	1642.59252929688\\
6.73000001907349	1646.39575195313\\
6.7350001335144	1651.4345703125\\
6.73999977111816	1657.07092285156\\
6.74499988555908	1663.44641113281\\
6.75	1670.13305664063\\
6.75500011444092	1677.22180175781\\
6.76000022888184	1685.22399902344\\
6.7649998664856	1693.06140136719\\
6.76999998092651	1701.02917480469\\
6.77500009536743	1708.69836425781\\
6.78000020980835	1716.71484375\\
6.78499984741211	1724.90100097656\\
6.78999996185303	1732.93090820313\\
6.79500007629395	1741.11755371094\\
6.80000019073486	1749.53515625\\
6.80499982833862	1758.2001953125\\
6.80999994277954	1767.13098144531\\
6.81500005722046	1776.35437011719\\
6.82000017166138	1785.79016113281\\
6.82499980926514	1795.40051269531\\
6.82999992370605	1805.09631347656\\
6.83500003814697	1815.27893066406\\
6.84000015258789	1824.27941894531\\
6.84499979019165	1832.40100097656\\
6.84999990463257	1839.00927734375\\
6.85500001907349	1843.61840820313\\
6.8600001335144	1846.13708496094\\
6.86499977111816	1844.82202148438\\
6.86999988555908	1840.28747558594\\
6.875	1833.53247070313\\
6.88000011444092	1823.70495605469\\
6.88500022888184	1811.16918945313\\
6.8899998664856	1796.08862304688\\
6.89499998092651	1778.78198242188\\
6.90000009536743	1759.44836425781\\
6.90500020980835	1738.17016601563\\
6.90999984741211	1714.79333496094\\
6.91499996185303	1689.73620605469\\
6.92000007629395	1662.88977050781\\
6.92500019073486	1634.13732910156\\
6.92999982833862	1603.62585449219\\
6.93499994277954	1571.14331054688\\
6.94000005722046	1536.86645507813\\
6.94500017166138	1500.76452636719\\
6.94999980926514	1462.81494140625\\
6.95499992370605	1422.05212402344\\
6.96000003814697	1379.736328125\\
6.96500015258789	1337.08264160156\\
6.96999979019165	1292.76049804688\\
6.97499990463257	1248.01220703125\\
6.98000001907349	1202.73266601563\\
6.9850001335144	1157.67004394531\\
6.98999977111816	1112.5869140625\\
6.99499988555908	1067.45739746094\\
7	1023.23864746094\\
7.00500011444092	979.494689941406\\
7.01000022888184	936.353637695313\\
7.0149998664856	893.785278320313\\
7.01999998092651	851.835083007813\\
7.02500009536743	810.672729492188\\
7.03000020980835	770.301086425781\\
7.03499984741211	730.754150390625\\
7.03999996185303	692.136352539063\\
7.04500007629395	654.562744140625\\
7.05000019073486	618.182067871094\\
7.05499982833862	583.200744628906\\
7.05999994277954	549.608520507813\\
7.06500005722046	517.635986328125\\
7.07000017166138	487.491973876953\\
7.07499980926514	459.220275878906\\
7.07999992370605	432.926300048828\\
7.08500003814697	408.650421142578\\
7.09000015258789	386.440582275391\\
7.09499979019165	366.295867919922\\
7.09999990463257	348.113830566406\\
7.10500001907349	331.805938720703\\
7.1100001335144	317.287017822266\\
7.11499977111816	304.592651367188\\
7.11999988555908	293.6181640625\\
7.125	284.226226806641\\
7.13000011444092	276.484161376953\\
7.13500022888184	270.364593505859\\
7.1399998664856	265.816986083984\\
7.14499998092651	262.877777099609\\
7.15000009536743	261.638061523438\\
7.15500020980835	262.413818359375\\
7.15999984741211	265.159027099609\\
7.16499996185303	269.744201660156\\
7.17000007629395	275.409423828125\\
7.17500019073486	282.100128173828\\
7.17999982833862	289.872955322266\\
7.18499994277954	298.585174560547\\
7.19000005722046	308.156372070313\\
7.19500017166138	318.507843017578\\
7.19999980926514	329.567016601563\\
7.20499992370605	341.186096191406\\
7.21000003814697	353.406158447266\\
7.21500015258789	366.140106201172\\
7.21999979019165	379.175323486328\\
7.22499990463257	392.601776123047\\
7.23000001907349	406.330627441406\\
7.2350001335144	420.268920898438\\
7.23999977111816	434.370422363281\\
7.24499988555908	448.528900146484\\
7.25	462.652496337891\\
7.25500011444092	476.696685791016\\
7.26000022888184	490.533508300781\\
7.2649998664856	504.109985351563\\
7.26999998092651	517.394409179688\\
7.27500009536743	530.265991210938\\
7.28000020980835	542.616943359375\\
7.28499984741211	554.448974609375\\
7.28999996185303	565.686096191406\\
7.29500007629395	576.351013183594\\
7.30000019073486	586.27490234375\\
7.30499982833862	595.494506835938\\
7.30999994277954	603.989074707031\\
7.31500005722046	611.781311035156\\
7.32000017166138	618.807312011719\\
7.32499980926514	625.082153320313\\
7.32999992370605	630.55029296875\\
7.33500003814697	635.199645996094\\
7.34000015258789	639.043029785156\\
7.34499979019165	642.0732421875\\
7.34999990463257	644.287841796875\\
7.35500001907349	645.695983886719\\
7.3600001335144	646.375122070313\\
7.36499977111816	646.349609375\\
7.36999988555908	645.639526367188\\
7.375	644.246154785156\\
7.38000011444092	642.188903808594\\
7.38500022888184	639.364868164063\\
7.3899998664856	635.697509765625\\
7.39499998092651	631.200561523438\\
7.40000009536743	626.042053222656\\
7.40500020980835	620.224731445313\\
7.40999984741211	613.819213867188\\
7.41499996185303	606.871215820313\\
7.42000007629395	599.448669433594\\
7.42500019073486	591.630432128906\\
7.42999982833862	583.458068847656\\
7.43499994277954	574.941284179688\\
7.44000005722046	566.123596191406\\
7.44500017166138	557.047058105469\\
7.44999980926514	547.753479003906\\
7.45499992370605	538.270324707031\\
7.46000003814697	528.614807128906\\
7.46500015258789	518.831176757813\\
7.46999979019165	509.012634277344\\
7.47499990463257	499.231201171875\\
7.48000001907349	489.504608154297\\
7.4850001335144	479.881774902344\\
7.48999977111816	470.395446777344\\
7.49499988555908	461.078552246094\\
7.5	451.952911376953\\
7.50500011444092	443.034027099609\\
7.51000022888184	434.34912109375\\
7.5149998664856	425.970611572266\\
7.51999998092651	417.956176757813\\
7.52500009536743	410.566772460938\\
7.53000020980835	403.33056640625\\
7.53499984741211	396.540588378906\\
7.53999996185303	390.133941650391\\
7.54500007629395	384.1455078125\\
7.55000019073486	378.587036132813\\
7.55499982833862	373.482086181641\\
7.55999994277954	368.848968505859\\
7.56500005722046	364.696228027344\\
7.57000017166138	361.031188964844\\
7.57499980926514	357.867065429688\\
7.57999992370605	355.203094482422\\
7.58500003814697	353.046997070313\\
7.59000015258789	351.389068603516\\
7.59499979019165	350.227142333984\\
7.59999990463257	349.555816650391\\
7.60500001907349	349.376129150391\\
7.6100001335144	349.732238769531\\
7.61499977111816	350.673309326172\\
7.61999988555908	352.064849853516\\
7.625	353.727386474609\\
7.63000011444092	355.679992675781\\
7.63500022888184	357.966186523438\\
7.6399998664856	360.574951171875\\
7.64499998092651	363.494323730469\\
7.65000009536743	366.709503173828\\
7.65500020980835	370.211029052734\\
7.65999984741211	373.980438232422\\
7.66499996185303	377.994049072266\\
7.67000007629395	382.225311279297\\
7.67500019073486	386.653228759766\\
7.67999982833862	391.250793457031\\
7.68499994277954	395.989074707031\\
7.69000005722046	400.841430664063\\
7.69500017166138	405.775573730469\\
7.69999980926514	410.771636962891\\
7.70499992370605	415.805755615234\\
7.71000003814697	420.858093261719\\
7.71500015258789	425.914184570313\\
7.71999979019165	430.943023681641\\
7.72499990463257	435.926910400391\\
7.73000001907349	440.853424072266\\
7.7350001335144	445.702575683594\\
7.73999977111816	450.460876464844\\
7.74499988555908	455.108093261719\\
7.75	459.621154785156\\
7.75500011444092	464.008148193359\\
7.76000022888184	468.246612548828\\
7.7649998664856	472.303436279297\\
7.76999998092651	476.169555664063\\
7.77500009536743	479.853576660156\\
7.78000020980835	483.353118896484\\
7.78499984741211	486.651702880859\\
7.78999996185303	489.753295898438\\
7.79500007629395	492.6611328125\\
7.80000019073486	495.350921630859\\
7.80499982833862	497.832489013672\\
7.80999994277954	500.107666015625\\
7.81500005722046	502.172515869141\\
7.82000017166138	504.027404785156\\
7.82499980926514	505.674530029297\\
7.82999992370605	507.120361328125\\
7.83500003814697	508.372650146484\\
7.84000015258789	509.425964355469\\
7.84499979019165	510.286315917969\\
7.84999990463257	510.984344482422\\
7.85500001907349	511.51806640625\\
7.8600001335144	511.894744873047\\
7.86499977111816	512.113098144531\\
7.86999988555908	512.178833007813\\
7.875	512.097351074219\\
7.88000011444092	511.902893066406\\
7.88500022888184	511.612701416016\\
7.8899998664856	511.236022949219\\
7.89499998092651	510.778167724609\\
7.90000009536743	510.242004394531\\
7.90500020980835	509.641448974609\\
7.90999984741211	508.983489990234\\
7.91499996185303	508.26806640625\\
7.92000007629395	507.506195068359\\
7.92500019073486	506.708953857422\\
7.92999982833862	505.945129394531\\
7.93499994277954	505.182159423828\\
7.94000005722046	504.4189453125\\
7.94500017166138	503.740539550781\\
7.94999980926514	503.130706787109\\
7.95499992370605	502.577392578125\\
7.96000003814697	502.148620605469\\
7.96500015258789	501.872955322266\\
7.96999979019165	501.722259521484\\
7.97499990463257	501.708343505859\\
7.98000001907349	501.849822998047\\
7.9850001335144	502.123901367188\\
7.98999977111816	502.539611816406\\
7.99499988555908	503.146514892578\\
8	503.925750732422\\
8.00500011444092	504.874145507813\\
8.01000022888184	506.052459716797\\
8.01500034332275	507.463897705078\\
8.02000045776367	509.1201171875\\
8.02499961853027	511.00732421875\\
8.02999973297119	513.119201660156\\
8.03499984741211	515.461791992188\\
8.03999996185303	518.043640136719\\
8.04500007629395	520.854125976563\\
8.05000019073486	523.903747558594\\
8.05500030517578	527.192077636719\\
8.0600004196167	530.713073730469\\
8.0649995803833	534.471923828125\\
8.06999969482422	538.468444824219\\
8.07499980926514	542.755432128906\\
8.07999992370605	547.323120117188\\
8.08500003814697	552.216552734375\\
8.09000015258789	557.356750488281\\
8.09500026702881	562.729309082031\\
8.10000038146973	568.340698242188\\
8.10499954223633	574.205078125\\
8.10999965667725	580.339782714844\\
8.11499977111816	586.744262695313\\
8.11999988555908	593.412414550781\\
8.125	600.321838378906\\
8.13000011444092	607.473083496094\\
8.13500022888184	614.864685058594\\
8.14000034332275	622.523986816406\\
8.14500045776367	630.444641113281\\
8.14999961853027	638.645324707031\\
8.15499973297119	647.065490722656\\
8.15999984741211	655.708068847656\\
8.16499996185303	664.567504882813\\
8.17000007629395	673.704406738281\\
8.17500019073486	683.178833007813\\
8.18000030517578	692.997253417969\\
8.1850004196167	703.119812011719\\
8.1899995803833	713.433410644531\\
8.19499969482422	723.9345703125\\
8.19999980926514	734.632629394531\\
8.20499992370605	745.557067871094\\
8.21000003814697	756.708312988281\\
8.21500015258789	768.061462402344\\
8.22000026702881	779.593139648438\\
8.22500038146973	791.288208007813\\
8.22999954223633	803.4619140625\\
8.23499965667725	816.377136230469\\
8.23999977111816	829.683898925781\\
8.24499988555908	842.956420898438\\
8.25	856.312683105469\\
8.25500011444092	870.015747070313\\
8.26000022888184	884.01513671875\\
8.26500034332275	898.286560058594\\
8.27000045776367	912.948669433594\\
8.27499961853027	928.123779296875\\
8.27999973297119	943.516418457031\\
8.28499984741211	959.173950195313\\
8.28999996185303	975.331359863281\\
8.29500007629395	991.859924316406\\
8.30000019073486	1008.61553955078\\
8.30500030517578	1025.25708007813\\
8.3100004196167	1041.82556152344\\
8.3149995803833	1061.45520019531\\
8.31999969482422	1080.43737792969\\
8.32499980926514	1099.55725097656\\
8.32999992370605	1119.04528808594\\
8.33500003814697	1139.13012695313\\
8.34000015258789	1159.69140625\\
8.34500026702881	1180.59240722656\\
8.35000038146973	1202.04174804688\\
8.35499954223633	1223.33154296875\\
8.35999965667725	1246.01892089844\\
8.36499977111816	1270.03405761719\\
8.36999988555908	1294.01770019531\\
8.375	1318.36682128906\\
8.38000011444092	1343.52136230469\\
8.38500022888184	1369.00866699219\\
8.39000034332275	1395.03454589844\\
8.39500045776367	1420.77978515625\\
8.39999961853027	1448.81823730469\\
8.40499973297119	1477.03210449219\\
8.40999984741211	1505.63049316406\\
8.41499996185303	1535.10327148438\\
8.42000007629395	1565.04187011719\\
8.42500019073486	1595.52941894531\\
8.43000030517578	1626.10009765625\\
8.4350004196167	1657.1064453125\\
8.4399995803833	1689.080078125\\
8.44499969482422	1720.33776855469\\
8.44999980926514	1751.00073242188\\
8.45499992370605	1781.09594726563\\
8.46000003814697	1809.48278808594\\
8.46500015258789	1837.09741210938\\
8.47000026702881	1863.4921875\\
8.47500038146973	1888.70544433594\\
8.47999954223633	1912.74877929688\\
8.48499965667725	1935.32397460938\\
8.48999977111816	1955.93273925781\\
8.49499988555908	1975.3837890625\\
8.5	1993.15747070313\\
8.50500011444092	2009.31701660156\\
8.51000022888184	2023.74438476563\\
8.51500034332275	2036.83911132813\\
8.52000045776367	2049.41284179688\\
8.52499961853027	2059.16943359375\\
8.52999973297119	2067.93603515625\\
8.53499984741211	2074.8046875\\
8.53999996185303	2080.19262695313\\
8.54500007629395	2083.8720703125\\
8.55000019073486	2086.34375\\
8.55500030517578	2087.62060546875\\
8.5600004196167	2087.9541015625\\
8.5649995803833	2087.7431640625\\
8.56999969482422	2087.2734375\\
8.57499980926514	2086.5126953125\\
8.57999992370605	2085.7041015625\\
8.58500003814697	2084.72509765625\\
8.59000015258789	2083.90063476563\\
8.59500026702881	2083.31127929688\\
8.60000038146973	2082.7724609375\\
8.60499954223633	2082.15869140625\\
8.60999965667725	2081.21020507813\\
8.61499977111816	2079.59765625\\
8.61999988555908	2076.94677734375\\
8.625	2072.82250976563\\
8.63000011444092	2066.50659179688\\
8.63500022888184	2057.7939453125\\
8.64000034332275	2046.27355957031\\
8.64500045776367	2031.67529296875\\
8.64999961853027	2014.11608886719\\
8.65499973297119	1993.4130859375\\
8.65999984741211	1970.24633789063\\
8.66499996185303	1944.62268066406\\
8.67000007629395	1917.25415039063\\
8.67500019073486	1888.58374023438\\
8.68000030517578	1859.18395996094\\
8.6850004196167	1829.31823730469\\
8.6899995803833	1799.28466796875\\
8.69499969482422	1769.27331542969\\
8.69999980926514	1739.30932617188\\
8.70499992370605	1709.28662109375\\
8.71000003814697	1679.001953125\\
8.71500015258789	1648.24584960938\\
8.72000026702881	1616.97998046875\\
8.72500038146973	1584.93444824219\\
8.72999954223633	1552.03784179688\\
8.73499965667725	1518.322265625\\
8.73999977111816	1483.87670898438\\
8.74499988555908	1448.85144042969\\
8.75	1413.56652832031\\
8.75500011444092	1378.30798339844\\
8.76000022888184	1343.65710449219\\
8.76500034332275	1309.75500488281\\
8.77000045776367	1276.97082519531\\
8.77499961853027	1245.46179199219\\
8.77999973297119	1215.37268066406\\
8.78499984741211	1186.69555664063\\
8.78999996185303	1159.39636230469\\
8.79500007629395	1133.38208007813\\
8.80000019073486	1108.52331542969\\
8.80500030517578	1084.6806640625\\
8.8100004196167	1061.73571777344\\
8.8149995803833	1039.5966796875\\
8.81999969482422	1018.23004150391\\
8.82499980926514	997.658569335938\\
8.82999992370605	977.978820800781\\
8.83500003814697	959.303527832031\\
8.84000015258789	941.799499511719\\
8.84500026702881	925.654846191406\\
8.85000038146973	910.979064941406\\
8.85499954223633	897.927551269531\\
8.85999965667725	886.509582519531\\
8.86499977111816	876.714782714844\\
8.86999988555908	868.465881347656\\
8.875	861.638488769531\\
8.88000011444092	856.075073242188\\
8.88500022888184	851.582702636719\\
8.89000034332275	847.989685058594\\
8.89500045776367	845.149169921875\\
8.89999961853027	843.005859375\\
8.90499973297119	841.497375488281\\
8.90999984741211	840.595092773438\\
8.91499996185303	840.30517578125\\
8.92000007629395	840.820129394531\\
8.92500019073486	842.171936035156\\
8.93000030517578	844.688293457031\\
8.9350004196167	848.189819335938\\
8.9399995803833	852.154602050781\\
8.94499969482422	856.908020019531\\
8.94999980926514	862.335815429688\\
8.95499992370605	868.254760742188\\
8.96000003814697	874.647033691406\\
8.96500015258789	881.408813476563\\
8.97000026702881	888.438720703125\\
8.97500038146973	895.634765625\\
8.97999954223633	902.956787109375\\
8.98499965667725	910.427612304688\\
8.98999977111816	917.905639648438\\
8.99499988555908	925.412109375\\
9	933.047424316406\\
9.00500011444092	940.667358398438\\
9.01000022888184	948.268981933594\\
9.01500034332275	955.830627441406\\
9.02000045776367	963.314575195313\\
9.02499961853027	970.661865234375\\
9.02999973297119	977.821166992188\\
9.03499984741211	984.747253417969\\
9.03999996185303	991.387512207031\\
9.04500007629395	997.704528808594\\
9.05000019073486	1003.62060546875\\
9.05500030517578	1009.11407470703\\
9.0600004196167	1014.16522216797\\
9.0649995803833	1018.73071289063\\
9.06999969482422	1022.80578613281\\
9.07499980926514	1026.36413574219\\
9.07999992370605	1029.39404296875\\
9.08500003814697	1031.88366699219\\
9.09000015258789	1033.82373046875\\
9.09500026702881	1035.23608398438\\
9.10000038146973	1036.12915039063\\
9.10499954223633	1036.56420898438\\
9.10999965667725	1036.49694824219\\
9.11499977111816	1035.97875976563\\
9.11999988555908	1035.04699707031\\
9.125	1033.65161132813\\
9.13000011444092	1031.66418457031\\
9.13500022888184	1029.11706542969\\
9.14000034332275	1026.04577636719\\
9.14500045776367	1022.44158935547\\
9.14999961853027	1018.33874511719\\
9.15499973297119	1013.80181884766\\
9.15999984741211	1008.8876953125\\
9.16499996185303	1003.62487792969\\
9.17000007629395	998.095581054688\\
9.17500019073486	992.390441894531\\
9.18000030517578	986.5458984375\\
9.1850004196167	980.646240234375\\
9.1899995803833	974.699096679688\\
9.19499969482422	968.725402832031\\
9.19999980926514	962.769653320313\\
9.20499992370605	956.858764648438\\
9.21000003814697	950.998901367188\\
9.21500015258789	945.121032714844\\
9.22000026702881	939.203369140625\\
9.22500038146973	933.227661132813\\
9.22999954223633	927.197692871094\\
9.23499965667725	921.319885253906\\
9.23999977111816	916.2568359375\\
9.24499988555908	913.451110839844\\
9.25	913.647766113281\\
9.25500011444092	916.578674316406\\
9.26000022888184	922.438903808594\\
9.26500034332275	931.379699707031\\
9.27000045776367	943.12060546875\\
9.27499961853027	957.790222167969\\
9.27999973297119	975.37060546875\\
9.28499984741211	995.774047851563\\
9.28999996185303	1019.22985839844\\
9.29500007629395	1045.84265136719\\
9.30000019073486	1075.27746582031\\
9.30500030517578	1110.80078125\\
9.3100004196167	1148.40405273438\\
9.3149995803833	1188.67346191406\\
9.31999969482422	1231.56896972656\\
9.32499980926514	1276.05078125\\
9.32999992370605	1322.64038085938\\
9.33500003814697	1368.63989257813\\
9.34000015258789	1411.23779296875\\
9.34500026702881	1454.56005859375\\
9.35000038146973	1490.46057128906\\
9.35499954223633	1515.51245117188\\
9.35999965667725	1535.46899414063\\
9.36499977111816	1546.04187011719\\
9.36999988555908	1548.5263671875\\
9.375	1547.25048828125\\
9.38000011444092	1545.94738769531\\
9.38500022888184	1548.68103027344\\
9.39000034332275	1561.06701660156\\
9.39500045776367	1588.02221679688\\
9.39999961853027	1632.23718261719\\
9.40499973297119	1697.45983886719\\
9.40999984741211	1785.5966796875\\
9.41499996185303	1896.8779296875\\
9.42000007629395	2029.94616699219\\
9.42500019073486	2182.84594726563\\
9.43000030517578	2359.46020507813\\
9.4350004196167	2548.48779296875\\
9.4399995803833	2727.82495117188\\
9.44499969482422	2872.50146484375\\
9.44999980926514	2946.60229492188\\
9.45499992370605	2917.48388671875\\
9.46000003814697	2747.333984375\\
9.46500015258789	2416.85229492188\\
9.47000026702881	1921.74523925781\\
9.47500038146973	1297.95593261719\\
9.47999954223633	603.170104980469\\
9.48499965667725	45.9526672363281\\
9.48999977111816	-59.4299201965332\\
9.49499988555908	-53.074146270752\\
9.5	-30.4234962463379\\
9.50500011444092	-15.5797119140625\\
9.51000022888184	-9.53832149505615\\
9.51500034332275	-9.52199172973633\\
9.52000045776367	-13.7433137893677\\
9.52499961853027	-20.6212062835693\\
9.52999973297119	-27.1416168212891\\
9.53499984741211	-34.2036514282227\\
9.53999996185303	-40.8650245666504\\
9.54500007629395	-46.5349388122559\\
9.55000019073486	-50.4669914245605\\
9.55500030517578	-35.7142105102539\\
9.5600004196167	-25.084888458252\\
9.5649995803833	-16.2114944458008\\
9.56999969482422	-9.2534761428833\\
9.57499980926514	-4.83238697052002\\
9.57999992370605	-2.34660148620605\\
9.58500003814697	-1.16579139232635\\
9.59000015258789	-0.442380338907242\\
9.59500026702881	-0.241003111004829\\
9.60000038146973	-0.112100265920162\\
9.60499954223633	-0.185922637581825\\
9.60999965667725	-0.162918463349342\\
9.61499977111816	-0.259052336215973\\
9.61999988555908	-0.359487742185593\\
9.625	-0.406370013952255\\
9.63000011444092	-0.445014357566833\\
9.63500022888184	-0.454397529363632\\
9.64000034332275	-0.457560777664185\\
9.64500045776367	-0.467763096094131\\
9.64999961853027	-0.457084327936172\\
9.65499973297119	-0.437500566244125\\
9.65999984741211	-0.430649608373642\\
9.66499996185303	-0.416501224040985\\
9.67000007629395	-0.395694613456726\\
9.67500019073486	-0.375835806131363\\
9.68000030517578	-0.356049329042435\\
9.6850004196167	-0.334743559360504\\
9.6899995803833	-0.31375589966774\\
9.69499969482422	-0.292506128549576\\
9.69999980926514	-0.272977441549301\\
9.70499992370605	-0.255079388618469\\
9.71000003814697	-0.237612694501877\\
9.71500015258789	-0.218251436948776\\
9.72000026702881	-0.200997546315193\\
9.72500038146973	-0.188348293304443\\
9.72999954223633	-0.176157787442207\\
9.73499965667725	-0.163813143968582\\
9.73999977111816	-0.151319608092308\\
9.74499988555908	-0.137802585959435\\
9.75	-0.128434330224991\\
9.75500011444092	-0.118500925600529\\
9.76000022888184	-0.108127735555172\\
9.76500034332275	-0.0982756987214088\\
9.77000045776367	-0.0904479622840881\\
9.77499961853027	-0.0856268182396889\\
9.77999973297119	-0.0829319357872009\\
9.78499984741211	-0.0775267779827118\\
9.78999996185303	-0.0705621764063835\\
9.79500007629395	-0.0629309788346291\\
9.80000019073486	-0.0564672760665417\\
9.80500030517578	-0.0526410974562168\\
9.8100004196167	-0.0500955991446972\\
9.8149995803833	-0.0486357100307941\\
9.81999969482422	-0.0452500581741333\\
9.82499980926514	-0.0420141518115997\\
9.82999992370605	-0.0389540679752827\\
9.83500003814697	-0.0362909324467182\\
9.84000015258789	-0.0339215397834778\\
9.84500026702881	-0.0311873368918896\\
9.85000038146973	-0.0282676983624697\\
9.85499954223633	-0.0260877814143896\\
9.85999965667725	-0.0255888681858778\\
9.86499977111816	-0.0257884357124567\\
9.86999988555908	-0.0255767516791821\\
9.875	-0.0238961577415466\\
9.88000011444092	-0.0210505165159702\\
9.88500022888184	-0.0179312694817781\\
9.89000034332275	-0.017101040109992\\
9.89500045776367	-0.018181262537837\\
9.89999961853027	-0.0200350191444159\\
9.90499973297119	-0.0206145122647285\\
9.90999984741211	-0.0184380039572716\\
9.91499996185303	-0.0147195272147655\\
9.92000007629395	-0.0106426272541285\\
9.92500019073486	-0.0105568915605545\\
9.93000030517578	-0.0134870819747448\\
9.9350004196167	-0.0175132788717747\\
9.9399995803833	-0.0192282367497683\\
9.94499969482422	-0.0172911342233419\\
9.94999980926514	-0.0144386254251003\\
9.95499992370605	-0.0121273249387741\\
9.96000003814697	-0.0102163217961788\\
9.96500015258789	-0.00845506135374308\\
9.97000026702881	-0.0067576807923615\\
9.97500038146973	-0.00725379446521401\\
9.97999954223633	-0.0092875836417079\\
9.98499965667725	-0.0108047537505627\\
9.98999977111816	-0.012318498454988\\
9.99499988555908	-0.0131670162081718\\
10	-0.0137691432610154\\
};
\addlegendentry{CF}

\end{axis}

\begin{axis}[%
width=4.521in,
height=1.476in,
at={(0.758in,0.498in)},
scale only axis,
xmin=0,
xmax=10,
xlabel style={font=\color{white!15!black}},
xlabel={Time (s)},
ymin=-10000,
ymax=2012.3505859375,
ylabel style={font=\color{white!15!black}},
ylabel={FY (N)},
axis background/.style={fill=white},
xmajorgrids,
ymajorgrids,
legend style={at={(0.85,0.6)}, anchor=north east, legend cell align=left, align=left, draw=black}
]
\addplot [color=black, dashed, line width=2.0pt]
  table[row sep=crcr]{%
0.0949999988079071	-45.0914001464844\\
0.100000001490116	-39.0320129394531\\
0.104999996721745	-33.7944679260254\\
0.109999999403954	-29.2313060760498\\
0.115000002086163	-25.2414703369141\\
0.119999997317791	306.117919921875\\
0.125	382.132293701172\\
0.129999995231628	417.609649658203\\
0.135000005364418	431.298095703125\\
0.140000000596046	437.015625\\
0.144999995827675	436.909027099609\\
0.150000005960464	435.663909912109\\
0.155000001192093	430.265319824219\\
0.159999996423721	436.149597167969\\
0.165000006556511	428.588653564453\\
0.170000001788139	404.961669921875\\
0.174999997019768	366.59375\\
0.180000007152557	315.569519042969\\
0.185000002384186	252.283401489258\\
0.189999997615814	179.566314697266\\
0.194999992847443	-1019.24212646484\\
0.200000002980232	-2328.83227539063\\
0.204999998211861	-3111.01977539063\\
0.209999993443489	-3319.92602539063\\
0.215000003576279	-3124.99243164063\\
0.219999998807907	-2790.08715820313\\
0.224999994039536	-2232.21264648438\\
0.230000004172325	-1521.60729980469\\
0.234999999403954	-776.625427246094\\
0.239999994635582	-114.111953735352\\
0.245000004768372	363.480621337891\\
0.25	572.282043457031\\
0.254999995231628	382.398101806641\\
0.259999990463257	3.63624572753906\\
0.264999985694885	-515.918395996094\\
0.270000010728836	-1080.85534667969\\
0.275000005960464	-1596.22534179688\\
0.280000001192093	-1988.31018066406\\
0.284999996423721	-2218.24487304688\\
0.28999999165535	-2273.7060546875\\
0.294999986886978	-2179.18334960938\\
0.300000011920929	-2002.87744140625\\
0.305000007152557	-1740.86975097656\\
0.310000002384186	-1431.78881835938\\
0.314999997615814	-1122.03552246094\\
0.319999992847443	-856.153747558594\\
0.324999988079071	-668.262084960938\\
0.330000013113022	-555.862976074219\\
0.33500000834465	-551.417602539063\\
0.340000003576279	-676.924987792969\\
0.344999998807907	-852.397766113281\\
0.349999994039536	-1034.25744628906\\
0.354999989271164	-1187.86560058594\\
0.360000014305115	-1288.23266601563\\
0.365000009536743	-1330.90539550781\\
0.370000004768372	-1329.06823730469\\
0.375	-1260.734375\\
0.379999995231628	-1134.17199707031\\
0.384999990463257	-969.066833496094\\
0.389999985694885	-785.730834960938\\
0.395000010728836	-613.063842773438\\
0.400000005960464	-469.520904541016\\
0.405000001192093	-366.957214355469\\
0.409999996423721	-318.871795654297\\
0.41499999165535	-318.431610107422\\
0.419999986886978	-358.279296875\\
0.425000011920929	-429.120208740234\\
0.430000007152557	-498.675354003906\\
0.435000002384186	-549.444580078125\\
0.439999997615814	-580.989990234375\\
0.444999992847443	-580.643859863281\\
0.449999988079071	-546.031127929688\\
0.455000013113022	-480.483459472656\\
0.46000000834465	-393.218292236328\\
0.465000003576279	-294.872955322266\\
0.469999998807907	-200.157211303711\\
0.474999994039536	-119.207229614258\\
0.479999989271164	-61.4049644470215\\
0.485000014305115	-31.737756729126\\
0.490000009536743	-30.5302753448486\\
0.495000004768372	-61.5294418334961\\
0.5	-107.616065979004\\
0.504999995231628	-154.264266967773\\
0.509999990463257	-193.578567504883\\
0.514999985694885	-219.61360168457\\
0.519999980926514	-230.186370849609\\
0.524999976158142	-223.967910766602\\
0.529999971389771	-208.242965698242\\
0.535000026226044	-187.076217651367\\
0.540000021457672	-162.933090209961\\
0.545000016689301	-139.795013427734\\
0.550000011920929	-122.96656036377\\
0.555000007152557	-119.589141845703\\
0.560000002384186	-128.303604125977\\
0.564999997615814	-146.739974975586\\
0.569999992847443	-172.441452026367\\
0.574999988079071	-202.328720092773\\
0.579999983310699	-233.680160522461\\
0.584999978542328	-264.160675048828\\
0.589999973773956	-291.822631835938\\
0.595000028610229	-315.914764404297\\
0.600000023841858	-336.243255615234\\
0.605000019073486	-353.086608886719\\
0.610000014305115	-367.591125488281\\
0.615000009536743	-380.814483642578\\
0.620000004768372	-394.031677246094\\
0.625	-407.910064697266\\
0.629999995231628	-423.449493408203\\
0.634999990463257	-440.588226318359\\
0.639999985694885	-459.769714355469\\
0.644999980926514	-480.670379638672\\
0.649999976158142	-502.717803955078\\
0.654999971389771	-525.29150390625\\
0.660000026226044	-547.715637207031\\
0.665000021457672	-569.354797363281\\
0.670000016689301	-589.6845703125\\
0.675000011920929	-608.282653808594\\
0.680000007152557	-624.976257324219\\
0.685000002384186	-639.84521484375\\
0.689999997615814	-653.064208984375\\
0.694999992847443	-664.864501953125\\
0.699999988079071	-675.562927246094\\
0.704999983310699	-685.611572265625\\
0.709999978542328	-695.1298828125\\
0.714999973773956	-703.964904785156\\
0.720000028610229	-712.383850097656\\
0.725000023841858	-720.356140136719\\
0.730000019073486	-727.353149414063\\
0.735000014305115	-733.708801269531\\
0.740000009536743	-738.943298339844\\
0.745000004768372	-743.141174316406\\
0.75	-746.069396972656\\
0.754999995231628	-747.803649902344\\
0.759999990463257	-748.221374511719\\
0.764999985694885	-747.258972167969\\
0.769999980926514	-745.720886230469\\
0.774999976158142	-743.543518066406\\
0.779999971389771	-740.445617675781\\
0.785000026226044	-736.396545410156\\
0.790000021457672	-731.468078613281\\
0.795000016689301	-725.752807617188\\
0.800000011920929	-719.340393066406\\
0.805000007152557	-712.315979003906\\
0.810000002384186	-704.740356445313\\
0.814999997615814	-696.689025878906\\
0.819999992847443	-688.215026855469\\
0.824999988079071	-679.359985351563\\
0.829999983310699	-670.040893554688\\
0.834999978542328	-660.302978515625\\
0.839999973773956	-650.209655761719\\
0.845000028610229	-639.765319824219\\
0.850000023841858	-628.954040527344\\
0.855000019073486	-617.854370117188\\
0.860000014305115	-606.726318359375\\
0.865000009536743	-595.657287597656\\
0.870000004768372	-584.650573730469\\
0.875	-573.803894042969\\
0.879999995231628	-563.153442382813\\
0.884999990463257	-552.726745605469\\
0.889999985694885	-542.55322265625\\
0.894999980926514	-532.667358398438\\
0.899999976158142	-523.09814453125\\
0.904999971389771	-513.873229980469\\
0.910000026226044	-505.024841308594\\
0.915000021457672	-496.55517578125\\
0.920000016689301	-488.426452636719\\
0.925000011920929	-480.678344726563\\
0.930000007152557	-473.318634033203\\
0.935000002384186	-466.342468261719\\
0.939999997615814	-459.763427734375\\
0.944999992847443	-453.652252197266\\
0.949999988079071	-448.159973144531\\
0.954999983310699	-443.284942626953\\
0.959999978542328	-439.033416748047\\
0.964999973773956	-435.406707763672\\
0.970000028610229	-432.455200195313\\
0.975000023841858	-430.127807617188\\
0.980000019073486	-428.417510986328\\
0.985000014305115	-427.326538085938\\
0.990000009536743	-426.751556396484\\
0.995000004768372	-426.696258544922\\
1	-427.116363525391\\
1.00499999523163	-427.969360351563\\
1.00999999046326	-429.233123779297\\
1.01499998569489	-431.003570556641\\
1.01999998092651	-433.242889404297\\
1.02499997615814	-435.832855224609\\
1.02999997138977	-438.781585693359\\
1.0349999666214	-442.080505371094\\
1.03999996185303	-445.726165771484\\
1.04499995708466	-449.680541992188\\
1.04999995231628	-453.930206298828\\
1.05499994754791	-458.444519042969\\
1.05999994277954	-463.201385498047\\
1.06500005722046	-468.192108154297\\
1.07000005245209	-473.378875732422\\
1.07500004768372	-478.733642578125\\
1.08000004291534	-484.203704833984\\
1.08500003814697	-489.761322021484\\
1.0900000333786	-495.372802734375\\
1.09500002861023	-501.002410888672\\
1.10000002384186	-506.615051269531\\
1.10500001907349	-512.191772460938\\
1.11000001430511	-517.741149902344\\
1.11500000953674	-523.216552734375\\
1.12000000476837	-528.592956542969\\
1.125	-533.876098632813\\
1.12999999523163	-539.044616699219\\
1.13499999046326	-544.074279785156\\
1.13999998569489	-548.946594238281\\
1.14499998092651	-553.643920898438\\
1.14999997615814	-558.147216796875\\
1.15499997138977	-562.438049316406\\
1.1599999666214	-566.505310058594\\
1.16499996185303	-570.333251953125\\
1.16999995708466	-573.903137207031\\
1.17499995231628	-577.203308105469\\
1.17999994754791	-580.252380371094\\
1.18499994277954	-583.024780273438\\
1.19000005722046	-585.510498046875\\
1.19500005245209	-587.695922851563\\
1.20000004768372	-589.567626953125\\
1.20500004291534	-591.114562988281\\
1.21000003814697	-592.32861328125\\
1.2150000333786	-593.278869628906\\
1.22000002861023	-593.951293945313\\
1.22500002384186	-594.365295410156\\
1.23000001907349	-594.524291992188\\
1.23500001430511	-594.42919921875\\
1.24000000953674	-594.106506347656\\
1.24500000476837	-593.574401855469\\
1.25	-592.829284667969\\
1.25499999523163	-591.893676757813\\
1.25999999046326	-590.782348632813\\
1.26499998569489	-589.468017578125\\
1.26999998092651	-587.903381347656\\
1.27499997615814	-586.1162109375\\
1.27999997138977	-584.109802246094\\
1.2849999666214	-581.905700683594\\
1.28999996185303	-579.495666503906\\
1.29499995708466	-576.888244628906\\
1.29999995231628	-574.176208496094\\
1.30499994754791	-571.345397949219\\
1.30999994277954	-568.416870117188\\
1.31500005722046	-565.437194824219\\
1.32000005245209	-562.426818847656\\
1.32500004768372	-559.410766601563\\
1.33000004291534	-556.393310546875\\
1.33500003814697	-553.391784667969\\
1.3400000333786	-550.431701660156\\
1.34500002861023	-547.525024414063\\
1.35000002384186	-544.67919921875\\
1.35500001907349	-541.909973144531\\
1.36000001430511	-539.200439453125\\
1.36500000953674	-536.469421386719\\
1.37000000476837	-533.707580566406\\
1.375	-530.896667480469\\
1.37999999523163	-528.219360351563\\
1.38499999046326	-525.602478027344\\
1.38999998569489	-523.057556152344\\
1.39499998092651	-520.663452148438\\
1.39999997615814	-518.421936035156\\
1.40499997138977	-516.356384277344\\
1.4099999666214	-514.485717773438\\
1.41499996185303	-512.835693359375\\
1.41999995708466	-511.409362792969\\
1.42499995231628	-510.128051757813\\
1.42999994754791	-509.024505615234\\
1.43499994277954	-508.087890625\\
1.44000005722046	-507.296600341797\\
1.44500005245209	-506.647644042969\\
1.45000004768372	-506.121520996094\\
1.45500004291534	-505.693908691406\\
1.46000003814697	-505.35546875\\
1.4650000333786	-505.082946777344\\
1.47000002861023	-504.851104736328\\
1.47500002384186	-504.639282226563\\
1.48000001907349	-504.633453369141\\
1.48500001430511	-504.964660644531\\
1.49000000953674	-505.735778808594\\
1.49500000476837	-506.400054931641\\
1.5	-507.018737792969\\
1.50499999523163	-507.648803710938\\
1.50999999046326	-508.478881835938\\
1.51499998569489	-509.445434570313\\
1.51999998092651	-510.535278320313\\
1.52499997615814	-511.748474121094\\
1.52999997138977	-513.089294433594\\
1.5349999666214	-514.495910644531\\
1.53999996185303	-515.974670410156\\
1.54499995708466	-517.538635253906\\
1.54999995231628	-519.206970214844\\
1.55499994754791	-520.969604492188\\
1.55999994277954	-522.782165527344\\
1.56500005722046	-524.646850585938\\
1.57000005245209	-543.278564453125\\
1.57500004768372	-538.383605957031\\
1.58000004291534	-535.096435546875\\
1.58500003814697	-532.251647949219\\
1.5900000333786	-530.028198242188\\
1.59500002861023	-528.570861816406\\
1.60000002384186	-527.976440429688\\
1.60500001907349	-528.412475585938\\
1.61000001430511	-530.095275878906\\
1.61500000953674	-532.640197753906\\
1.62000000476837	-536.291076660156\\
1.625	-540.272399902344\\
1.62999999523163	-544.044494628906\\
1.63499999046326	-546.846923828125\\
1.63999998569489	-548.238159179688\\
1.64499998092651	-549.292602539063\\
1.64999997615814	-550.1171875\\
1.65499997138977	-550.566528320313\\
1.6599999666214	-550.740295410156\\
1.66499996185303	-550.778747558594\\
1.66999995708466	-550.418579101563\\
1.67499995231628	-550.112670898438\\
1.67999994754791	-550.235473632813\\
1.68499994277954	-550.557067871094\\
1.69000005722046	-551.039733886719\\
1.69500005245209	-551.558837890625\\
1.70000004768372	-552.109985351563\\
1.70500004291534	-552.619689941406\\
1.71000003814697	-552.992980957031\\
1.7150000333786	-553.328430175781\\
1.72000002861023	-553.631591796875\\
1.72500002384186	-553.82080078125\\
1.73000001907349	-553.914672851563\\
1.73500001430511	-553.884216308594\\
1.74000000953674	-553.734130859375\\
1.74500000476837	-553.477722167969\\
1.75	-553.136352539063\\
1.75499999523163	-552.698486328125\\
1.75999999046326	-552.208923339844\\
1.76499998569489	-551.659118652344\\
1.76999998092651	-551.065979003906\\
1.77499997615814	-550.495178222656\\
1.77999997138977	-549.9501953125\\
1.7849999666214	-549.418579101563\\
1.78999996185303	-548.91455078125\\
1.79499995708466	-548.439758300781\\
1.79999995231628	-547.970458984375\\
1.80499994754791	-547.515747070313\\
1.80999994277954	-547.055419921875\\
1.81500005722046	-546.583557128906\\
1.82000005245209	-546.098510742188\\
1.82500004768372	-545.605285644531\\
1.83000004291534	-545.0966796875\\
1.83500003814697	-544.568298339844\\
1.8400000333786	-544.021667480469\\
1.84500002861023	-543.463684082031\\
1.85000002384186	-542.965026855469\\
1.85500001907349	-542.490173339844\\
1.86000001430511	-542.042297363281\\
1.86500000953674	-541.628845214844\\
1.87000000476837	-541.245422363281\\
1.875	-540.932312011719\\
1.87999999523163	-540.720703125\\
1.88499999046326	-540.585266113281\\
1.88999998569489	-540.522888183594\\
1.89499998092651	-540.429748535156\\
1.89999997615814	-540.364624023438\\
1.90499997138977	-540.327514648438\\
1.9099999666214	-540.313415527344\\
1.91499996185303	-540.32421875\\
1.91999995708466	-540.360290527344\\
1.92499995231628	-540.458984375\\
1.92999994754791	-540.617431640625\\
1.93499994277954	-540.826599121094\\
1.94000005722046	-541.082763671875\\
1.94500005245209	-541.384765625\\
1.95000004768372	-541.734252929688\\
1.95500004291534	-542.166442871094\\
1.96000003814697	-542.736267089844\\
1.9650000333786	-543.437866210938\\
1.97000002861023	-544.216674804688\\
1.97500002384186	-544.87646484375\\
1.98000001907349	-545.466796875\\
1.98500001430511	-545.973083496094\\
1.99000000953674	-546.511291503906\\
1.99500000476837	-547.016906738281\\
2	-547.49267578125\\
2.00500011444092	-548.1142578125\\
2.00999999046326	-548.846984863281\\
2.01500010490417	-549.72412109375\\
2.01999998092651	-550.860046386719\\
2.02500009536743	-552.28662109375\\
2.02999997138977	-554.012878417969\\
2.03500008583069	-555.442199707031\\
2.03999996185303	-556.698608398438\\
2.04500007629395	-557.775512695313\\
2.04999995231628	-558.690979003906\\
2.0550000667572	-559.341613769531\\
2.05999994277954	-559.612854003906\\
2.06500005722046	-559.435974121094\\
2.0699999332428	-558.658813476563\\
2.07500004768372	-557.226196289063\\
2.07999992370605	-555.222106933594\\
2.08500003814697	-552.595825195313\\
2.08999991416931	-549.450988769531\\
2.09500002861023	-545.796325683594\\
2.09999990463257	-541.757263183594\\
2.10500001907349	-537.379089355469\\
2.10999989509583	-532.733032226563\\
2.11500000953674	-527.912963867188\\
2.11999988555908	-523.487426757813\\
2.125	-519.810485839844\\
2.13000011444092	-516.372802734375\\
2.13499999046326	-513.172485351563\\
2.14000010490417	-509.933532714844\\
2.14499998092651	-506.418548583984\\
2.15000009536743	-502.832946777344\\
2.15499997138977	-498.612945556641\\
2.16000008583069	-493.041442871094\\
2.16499996185303	-487.101745605469\\
2.17000007629395	-480.861328125\\
2.17499995231628	-474.033905029297\\
2.1800000667572	-467.163208007813\\
2.18499994277954	-460.610565185547\\
2.19000005722046	-454.80810546875\\
2.1949999332428	-449.73876953125\\
2.20000004768372	-445.478942871094\\
2.20499992370605	-441.996856689453\\
2.21000003814697	-439.140594482422\\
2.21499991416931	-436.749267578125\\
2.22000002861023	-434.732147216797\\
2.22499990463257	-433.053680419922\\
2.23000001907349	-431.28369140625\\
2.23499989509583	-428.715545654297\\
2.24000000953674	-426.136535644531\\
2.24499988555908	-423.529235839844\\
2.25	-420.843109130859\\
2.25500011444092	-418.199920654297\\
2.25999999046326	-415.997131347656\\
2.26500010490417	-414.813293457031\\
2.26999998092651	-414.886535644531\\
2.27500009536743	-416.298370361328\\
2.27999997138977	-418.752288818359\\
2.28500008583069	-421.27587890625\\
2.28999996185303	-424.041107177734\\
2.29500007629395	-427.356231689453\\
2.29999995231628	-432.198669433594\\
2.3050000667572	-436.928558349609\\
2.30999994277954	-442.034698486328\\
2.31500005722046	-447.863403320313\\
2.3199999332428	-454.254547119141\\
2.32500004768372	-460.943939208984\\
2.32999992370605	-467.978576660156\\
2.33500003814697	-475.367797851563\\
2.33999991416931	-481.306304931641\\
2.34500002861023	-487.423095703125\\
2.34999990463257	-496.737182617188\\
2.35500001907349	-506.294891357422\\
2.35999989509583	-516.753967285156\\
2.36500000953674	-528.074645996094\\
2.36999988555908	-539.853271484375\\
2.375	-552.787841796875\\
2.38000011444092	-566.711242675781\\
2.38499999046326	-581.143432617188\\
2.39000010490417	-595.945617675781\\
2.39499998092651	-610.688720703125\\
2.40000009536743	-624.902282714844\\
2.40499997138977	-638.171508789063\\
2.41000008583069	-650.296264648438\\
2.41499996185303	-661.159362792969\\
2.42000007629395	-670.69921875\\
2.42499995231628	-679.336486816406\\
2.4300000667572	-686.712097167969\\
2.43499994277954	-693.119384765625\\
2.44000005722046	-698.652954101563\\
2.4449999332428	-702.158569335938\\
2.45000004768372	-705.999755859375\\
2.45499992370605	-709.028015136719\\
2.46000003814697	-711.85595703125\\
2.46499991416931	-714.822937011719\\
2.47000002861023	-717.708740234375\\
2.47499990463257	-720.343933105469\\
2.48000001907349	-722.605712890625\\
2.48499989509583	-724.269165039063\\
2.49000000953674	-725.475280761719\\
2.49499988555908	-726.615112304688\\
2.5	-727.224487304688\\
2.50500011444092	-725.904357910156\\
2.50999999046326	-721.989624023438\\
2.51500010490417	-716.156616210938\\
2.51999998092651	-709.219421386719\\
2.52500009536743	-701.256591796875\\
2.52999997138977	-692.3203125\\
2.53500008583069	-683.653503417969\\
2.53999996185303	-674.134887695313\\
2.54500007629395	-664.751098632813\\
2.54999995231628	-655.326110839844\\
2.5550000667572	-645.781860351563\\
2.55999994277954	-636.066467285156\\
2.56500005722046	-625.786376953125\\
2.5699999332428	-614.942932128906\\
2.57500004768372	-603.406860351563\\
2.57999992370605	-591.152404785156\\
2.58500003814697	-578.255493164063\\
2.58999991416931	-564.28759765625\\
2.59500002861023	-550.596374511719\\
2.59999990463257	-537.338439941406\\
2.60500001907349	-524.251403808594\\
2.60999989509583	-511.976837158203\\
2.61500000953674	-500.697631835938\\
2.61999988555908	-490.677978515625\\
2.625	-481.975830078125\\
2.63000011444092	-474.406768798828\\
2.63499999046326	-467.339324951172\\
2.64000010490417	-461.850067138672\\
2.64499998092651	-457.075134277344\\
2.65000009536743	-451.267333984375\\
2.65499997138977	-445.155975341797\\
2.66000008583069	-438.669891357422\\
2.66499996185303	-430.953735351563\\
2.67000007629395	-422.314971923828\\
2.67499995231628	-413.7001953125\\
2.6800000667572	-405.404632568359\\
2.68499994277954	-396.837158203125\\
2.69000005722046	-388.919799804688\\
2.6949999332428	-381.990173339844\\
2.70000004768372	-376.124114990234\\
2.70499992370605	-371.4033203125\\
2.71000003814697	-367.657135009766\\
2.71499991416931	-364.678131103516\\
2.72000002861023	-362.202362060547\\
2.72499990463257	-360.0166015625\\
2.73000001907349	-357.919860839844\\
2.73499989509583	-355.799957275391\\
2.74000000953674	-353.540161132813\\
2.74499988555908	-351.087097167969\\
2.75	-348.97216796875\\
2.75500011444092	-347.624755859375\\
2.75999999046326	-347.492370605469\\
2.76500010490417	-348.828186035156\\
2.76999998092651	-351.584625244141\\
2.77500009536743	-355.910888671875\\
2.77999997138977	-362.482452392578\\
2.78500008583069	-371.431640625\\
2.78999996185303	-382.557678222656\\
2.79500007629395	-395.360443115234\\
2.79999995231628	-409.189147949219\\
2.8050000667572	-423.463256835938\\
2.80999994277954	-437.356750488281\\
2.81500005722046	-450.262512207031\\
2.8199999332428	-461.541961669922\\
2.82500004768372	-470.940551757813\\
2.82999992370605	-478.383850097656\\
2.83500003814697	-484.515960693359\\
2.83999991416931	-489.906158447266\\
2.84500002861023	-495.109985351563\\
2.84999990463257	-500.967437744141\\
2.85500001907349	-507.160705566406\\
2.85999989509583	-513.970520019531\\
2.86500000953674	-521.093994140625\\
2.86999988555908	-527.68212890625\\
2.875	-533.744323730469\\
2.88000011444092	-538.201904296875\\
2.88499999046326	-541.2001953125\\
2.89000010490417	-543.25048828125\\
2.89499998092651	-546.416748046875\\
2.90000009536743	-552.800109863281\\
2.90499997138977	-563.46826171875\\
2.91000008583069	-579.713317871094\\
2.91499996185303	-602.307861328125\\
2.92000007629395	-630.351379394531\\
2.92499995231628	-662.282592773438\\
2.9300000667572	-696.066711425781\\
2.93499994277954	-729.128784179688\\
2.94000005722046	-759.405456542969\\
2.9449999332428	-784.727600097656\\
2.95000004768372	-804.451965332031\\
2.95499992370605	-816.728210449219\\
2.96000003814697	-821.805847167969\\
2.96499991416931	-825.077270507813\\
2.97000002861023	-823.980346679688\\
2.97499990463257	-818.799011230469\\
2.98000001907349	-811.036682128906\\
2.98499989509583	-790.116271972656\\
2.99000000953674	-777.599548339844\\
2.99499988555908	-763.10107421875\\
3	-749.953552246094\\
3.00500011444092	-738.746154785156\\
3.00999999046326	-729.367492675781\\
3.01500010490417	-721.124633789063\\
3.01999998092651	-712.721862792969\\
3.02500009536743	-702.934692382813\\
3.02999997138977	-690.083374023438\\
3.03500008583069	-672.608703613281\\
3.03999996185303	-650.478515625\\
3.04500007629395	-623.960754394531\\
3.04999995231628	-594.657104492188\\
3.0550000667572	-564.448364257813\\
3.05999994277954	-534.2783203125\\
3.06500005722046	-506.483337402344\\
3.0699999332428	-482.902008056641\\
3.07500004768372	-462.011901855469\\
3.07999992370605	-444.596954345703\\
3.08500003814697	-429.823913574219\\
3.08999991416931	-416.003814697266\\
3.09500002861023	-402.573059082031\\
3.09999990463257	-389.337646484375\\
3.10500001907349	-375.694427490234\\
3.10999989509583	-362.027984619141\\
3.11500000953674	-348.358795166016\\
3.11999988555908	-336.063232421875\\
3.125	-327.14892578125\\
3.13000011444092	-324.611297607422\\
3.13499999046326	-333.995208740234\\
3.14000010490417	-352.011596679688\\
3.14499998092651	-376.097869873047\\
3.15000009536743	-400.615814208984\\
3.15499997138977	-422.619506835938\\
3.16000008583069	-441.41455078125\\
3.16499996185303	-453.77685546875\\
3.17000007629395	-457.606292724609\\
3.17499995231628	-452.316375732422\\
3.1800000667572	-440.169494628906\\
3.18499994277954	-424.124633789063\\
3.19000005722046	-405.859375\\
3.1949999332428	-383.726806640625\\
3.20000004768372	-360.103179931641\\
3.20499992370605	-338.7333984375\\
3.21000003814697	-319.892700195313\\
3.21499991416931	-307.352630615234\\
3.22000002861023	-303.082763671875\\
3.22499990463257	-305.624298095703\\
3.23000001907349	-315.663391113281\\
3.23499989509583	-332.745727539063\\
3.24000000953674	-357.224609375\\
3.24499988555908	-389.02392578125\\
3.25	-426.122802734375\\
3.25500011444092	-466.770599365234\\
3.25999999046326	-509.209899902344\\
3.26500010490417	-550.885314941406\\
3.26999998092651	-589.568725585938\\
3.27500009536743	-623.253723144531\\
3.27999997138977	-651.068115234375\\
3.28500008583069	-673.006652832031\\
3.28999996185303	-689.534240722656\\
3.29500007629395	-702.093872070313\\
3.29999995231628	-711.923706054688\\
3.3050000667572	-719.858947753906\\
3.30999994277954	-728.529724121094\\
3.31500005722046	-736.95458984375\\
3.3199999332428	-742.569885253906\\
3.32500004768372	-749.297546386719\\
3.32999992370605	-757.582092285156\\
3.33500003814697	-766.301086425781\\
3.33999991416931	-775.014465332031\\
3.34500002861023	-783.193786621094\\
3.34999990463257	-787.622314453125\\
3.35500001907349	-786.973266601563\\
3.35999989509583	-780.935607910156\\
3.36500000953674	-769.268005371094\\
3.36999988555908	-752.937683105469\\
3.375	-732.855285644531\\
3.38000011444092	-710.5517578125\\
3.38499999046326	-687.622619628906\\
3.39000010490417	-665.122680664063\\
3.39499998092651	-644.0126953125\\
3.40000009536743	-624.573181152344\\
3.40499997138977	-606.272216796875\\
3.41000008583069	-588.792419433594\\
3.41499996185303	-571.421875\\
3.42000007629395	-553.461181640625\\
3.42499995231628	-534.64501953125\\
3.4300000667572	-514.879089355469\\
3.43499994277954	-494.819702148438\\
3.44000005722046	-475.670837402344\\
3.4449999332428	-458.214080810547\\
3.45000004768372	-444.013092041016\\
3.45499992370605	-435.224975585938\\
3.46000003814697	-439.506683349609\\
3.46499991416931	-455.958679199219\\
3.47000002861023	-476.666290283203\\
3.47499990463257	-498.068359375\\
3.48000001907349	-516.085388183594\\
3.48499989509583	-526.772644042969\\
3.49000000953674	-520.16357421875\\
3.49499988555908	-508.744689941406\\
3.5	-493.472106933594\\
3.50500011444092	-472.116058349609\\
3.50999999046326	-442.463226318359\\
3.51500010490417	-408.519439697266\\
3.51999998092651	-364.123565673828\\
3.52500009536743	-313.126190185547\\
3.52999997138977	-251.992065429688\\
3.53500008583069	-181.135406494141\\
3.53999996185303	-100.445503234863\\
3.54500007629395	-20.6481399536133\\
3.54999995231628	33.4491996765137\\
3.5550000667572	39.5352935791016\\
3.55999994277954	-2.88752388954163\\
3.56500005722046	-94.8046646118164\\
3.5699999332428	-221.79508972168\\
3.57500004768372	-372.906585693359\\
3.57999992370605	-524.53076171875\\
3.58500003814697	-669.41064453125\\
3.58999991416931	-792.380737304688\\
3.59500002861023	-875.762939453125\\
3.59999990463257	-912.186340332031\\
3.60500001907349	-911.662475585938\\
3.60999989509583	-887.691162109375\\
3.61500000953674	-852.677001953125\\
3.61999988555908	-801.993469238281\\
3.625	-746.659729003906\\
3.63000011444092	-697.445434570313\\
3.63499999046326	-659.2578125\\
3.64000010490417	-640.001586914063\\
3.64499998092651	-648.65966796875\\
3.65000009536743	-686.206481933594\\
3.65499997138977	-733.990112304688\\
3.66000008583069	-777.421142578125\\
3.66499996185303	-809.623474121094\\
3.67000007629395	-826.305114746094\\
3.67499995231628	-817.934387207031\\
3.6800000667572	-784.318786621094\\
3.68499994277954	-729.935974121094\\
3.69000005722046	-662.918212890625\\
3.6949999332428	-593.032775878906\\
3.70000004768372	-531.105773925781\\
3.70499992370605	-482.525939941406\\
3.71000003814697	-454.222015380859\\
3.71499991416931	-443.755523681641\\
3.72000002861023	-448.271850585938\\
3.72499990463257	-461.787780761719\\
3.73000001907349	-477.536865234375\\
3.73499989509583	-489.263122558594\\
3.74000000953674	-494.2431640625\\
3.74499988555908	-491.3203125\\
3.75	-491.478546142578\\
3.75500011444092	-505.783325195313\\
3.75999999046326	-537.924377441406\\
3.76500010490417	-582.130065917969\\
3.76999998092651	-628.744262695313\\
3.77500009536743	-663.438598632813\\
3.77999997138977	-685.230834960938\\
3.78500008583069	-687.244812011719\\
3.78999996185303	-673.887817382813\\
3.79500007629395	-650.65673828125\\
3.79999995231628	-607.284362792969\\
3.8050000667572	-541.788208007813\\
3.80999994277954	-455.932556152344\\
3.81500005722046	-350.022644042969\\
3.8199999332428	-227.413787841797\\
3.82500004768372	-93.6096725463867\\
3.82999992370605	41.9209403991699\\
3.83500003814697	159.463317871094\\
3.83999991416931	223.571380615234\\
3.84500002861023	200.646636962891\\
3.84999990463257	107.28515625\\
3.85500001907349	-45.6513175964355\\
3.85999989509583	-229.446502685547\\
3.86500000953674	-430.789642333984\\
3.86999988555908	-629.4833984375\\
3.875	-798.795227050781\\
3.88000011444092	-919.127746582031\\
3.88499999046326	-969.281372070313\\
3.89000010490417	-965.698791503906\\
3.89499998092651	-924.852722167969\\
3.90000009536743	-875.596069335938\\
3.90499997138977	-807.355651855469\\
3.91000008583069	-736.611267089844\\
3.91499996185303	-679.124206542969\\
3.92000007629395	-640.465087890625\\
3.92499995231628	-626.880798339844\\
3.9300000667572	-658.003540039063\\
3.93499994277954	-719.957763671875\\
3.94000005722046	-787.826477050781\\
3.9449999332428	-845.595275878906\\
3.95000004768372	-884.651123046875\\
3.95499992370605	-907.043518066406\\
3.96000003814697	-897.836608886719\\
3.96499991416931	-856.361267089844\\
3.97000002861023	-789.134643554688\\
3.97499990463257	-706.435180664063\\
3.98000001907349	-621.965576171875\\
3.98499989509583	-547.335754394531\\
3.99000000953674	-490.987457275391\\
3.99499988555908	-457.087982177734\\
4	-446.068695068359\\
4.00500011444092	-453.536895751953\\
4.01000022888184	-470.870910644531\\
4.0149998664856	-489.665679931641\\
4.01999998092651	-504.136810302734\\
4.02500009536743	-510.706939697266\\
4.03000020980835	-515.858459472656\\
4.03499984741211	-533.457153320313\\
4.03999996185303	-564.061096191406\\
4.04500007629395	-605.447265625\\
4.05000019073486	-649.292663574219\\
4.05499982833862	-685.589660644531\\
4.05999994277954	-703.908264160156\\
4.06500005722046	-700.122497558594\\
4.07000017166138	-683.921020507813\\
4.07499980926514	-661.636901855469\\
4.07999992370605	-616.981872558594\\
4.08500003814697	-546.636474609375\\
4.09000015258789	-441.947723388672\\
4.09499979019165	-291.114410400391\\
4.09999990463257	-77.5299987792969\\
4.10500001907349	218.15315246582\\
4.1100001335144	550.132751464844\\
4.11499977111816	792.046325683594\\
4.11999988555908	901.708374023438\\
4.125	821.100219726563\\
4.13000011444092	498.527374267578\\
4.13500022888184	117.872848510742\\
4.1399998664856	-341.196472167969\\
4.14499998092651	-797.996704101563\\
4.15000009536743	-1178.26574707031\\
4.15500020980835	-1430.73559570313\\
4.15999984741211	-1519.65600585938\\
4.16499996185303	-1438.49194335938\\
4.17000007629395	-1322.314453125\\
4.17500019073486	-1132.41271972656\\
4.17999982833862	-897.035339355469\\
4.18499994277954	-661.767272949219\\
4.19000005722046	-472.378387451172\\
4.19500017166138	-359.516296386719\\
4.19999980926514	-320.7255859375\\
4.20499992370605	-431.691497802734\\
4.21000003814697	-613.384948730469\\
4.21500015258789	-805.542602539063\\
4.21999979019165	-961.994750976563\\
4.22499990463257	-1051.53564453125\\
4.23000001907349	-1077.75610351563\\
4.2350001335144	-1050.57958984375\\
4.23999977111816	-950.554077148438\\
4.24499988555908	-793.604187011719\\
4.25	-618.063415527344\\
4.25500011444092	-448.831848144531\\
4.26000022888184	-315.402557373047\\
4.2649998664856	-237.753540039063\\
4.26999998092651	-221.499893188477\\
4.27500009536743	-261.773071289063\\
4.28000020980835	-356.463348388672\\
4.28499984741211	-451.950622558594\\
4.28999996185303	-530.627746582031\\
4.29500007629395	-588.2646484375\\
4.30000019073486	-647.541137695313\\
4.30499982833862	-726.291931152344\\
4.30999994277954	-803.778991699219\\
4.31500005722046	-871.445617675781\\
4.32000017166138	-921.827209472656\\
4.32499980926514	-942.994140625\\
4.32999992370605	-921.450012207031\\
4.33500003814697	-903.380493164063\\
4.34000015258789	-857.126770019531\\
4.34499979019165	-767.187255859375\\
4.34999990463257	-623.527282714844\\
4.35500001907349	-414.813781738281\\
4.3600001335144	-128.829635620117\\
4.36499977111816	248.955718994141\\
4.36999988555908	641.400817871094\\
4.375	855.00830078125\\
4.38000011444092	988.441284179688\\
4.38500022888184	1043.83630371094\\
4.3899998664856	1044.16027832031\\
4.39499998092651	607.332580566406\\
4.40000009536743	-23.3214092254639\\
4.40500020980835	-681.424621582031\\
4.40999984741211	-1255.26745605469\\
4.41499996185303	-1656.68017578125\\
4.42000007629395	-1838.59155273438\\
4.42500019073486	-1748.30126953125\\
4.42999982833862	-1576.95483398438\\
4.43499994277954	-1314.28039550781\\
4.44000005722046	-984.203308105469\\
4.44500017166138	-649.361877441406\\
4.44999980926514	-375.548309326172\\
4.45499992370605	-215.606674194336\\
4.46000003814697	-154.644927978516\\
4.46500015258789	-302.232971191406\\
4.46999979019165	-560.574401855469\\
4.47499990463257	-832.243347167969\\
4.48000001907349	-1058.77856445313\\
4.4850001335144	-1193.23901367188\\
4.48999977111816	-1228.47229003906\\
4.49499988555908	-1200.71057128906\\
4.5	-1074.74487304688\\
4.50500011444092	-876.060363769531\\
4.51000022888184	-645.25244140625\\
4.5149998664856	-425.041534423828\\
4.51999998092651	-251.716567993164\\
4.52500009536743	-153.211135864258\\
4.53000020980835	-133.278579711914\\
4.53499984741211	-189.122695922852\\
4.53999996185303	-316.605773925781\\
4.54500007629395	-442.922821044922\\
4.55000019073486	-549.434631347656\\
4.55499982833862	-636.312255859375\\
4.55999994277954	-679.833984375\\
4.56500005722046	-731.16650390625\\
4.57000017166138	-798.256774902344\\
4.57499980926514	-865.726989746094\\
4.57999992370605	-922.8505859375\\
4.58500003814697	-958.930541992188\\
4.59000015258789	-976.45751953125\\
4.59499979019165	-987.370239257813\\
4.59999990463257	-977.98193359375\\
4.60500001907349	-928.49365234375\\
4.6100001335144	-825.692993164063\\
4.61499977111816	-647.778747558594\\
4.61999988555908	-85.4923553466797\\
4.625	726.259826660156\\
4.63000011444092	1184.59851074219\\
4.63500022888184	1413.58337402344\\
4.6399998664856	1491.03283691406\\
4.64499998092651	1456.21472167969\\
4.65000009536743	1416.1650390625\\
4.65500020980835	1301.64807128906\\
4.65999984741211	1096.74487304688\\
4.66499996185303	-591.809631347656\\
4.67000007629395	-1722.58361816406\\
4.67500019073486	-2508.90307617188\\
4.67999982833862	-2854.984375\\
4.68499994277954	-2673.0751953125\\
4.69000005722046	-2334.62158203125\\
4.69500017166138	-1793.16198730469\\
4.69999980926514	-1113.37805175781\\
4.70499992370605	-417.194213867188\\
4.71000003814697	157.601806640625\\
4.71500015258789	496.253143310547\\
4.71999979019165	611.955688476563\\
4.72499990463257	288.768859863281\\
4.73000001907349	-231.090789794922\\
4.7350001335144	-778.393737792969\\
4.73999977111816	-1222.74963378906\\
4.74499988555908	-1480.32336425781\\
4.75	-1522.01916503906\\
4.75500011444092	-1454.50659179688\\
4.76000022888184	-1220.64282226563\\
4.7649998664856	-856.201477050781\\
4.76999998092651	-457.695495605469\\
4.77500009536743	-79.6690521240234\\
4.78000020980835	194.123245239258\\
4.78499984741211	291.989562988281\\
4.78999996185303	214.77783203125\\
4.79500007629395	123.091506958008\\
4.80000019073486	-179.863906860352\\
4.80499982833862	-482.684997558594\\
4.80999994277954	-739.575134277344\\
4.81500005722046	-955.088928222656\\
4.82000017166138	-1127.45373535156\\
4.82499980926514	-1270.35473632813\\
4.82999992370605	-1377.8857421875\\
4.83500003814697	-1440.37609863281\\
4.84000015258789	-1470.49743652344\\
4.84499979019165	-1423.00854492188\\
4.84999990463257	-1269.47338867188\\
4.85500001907349	-979.068359375\\
4.8600001335144	-543.358520507813\\
4.86499977111816	-71.0103530883789\\
4.86999988555908	449.224700927734\\
4.875	951.948852539063\\
4.88000011444092	1316.06005859375\\
4.88500022888184	1508.38073730469\\
4.8899998664856	1564.61645507813\\
4.89499998092651	1516.40234375\\
4.90000009536743	1485.64111328125\\
4.90500020980835	1337.14880371094\\
4.90999984741211	1095.31750488281\\
4.91499996185303	790.574890136719\\
4.92000007629395	-862.891967773438\\
4.92500019073486	-2344.74633789063\\
4.92999982833862	-3234.29223632813\\
4.93499994277954	-3423.55200195313\\
4.94000005722046	-3003.5009765625\\
4.94500017166138	-2478.49438476563\\
4.94999980926514	-1749.24658203125\\
4.95499992370605	-951.81396484375\\
4.96000003814697	-233.729232788086\\
4.96500015258789	277.866180419922\\
4.96999979019165	504.619506835938\\
4.97499990463257	452.371185302734\\
4.98000001907349	64.6684799194336\\
4.9850001335144	-464.458190917969\\
4.98999977111816	-999.425354003906\\
4.99499988555908	-1439.65185546875\\
5	-1721.77026367188\\
5.00500011444092	-1825.43786621094\\
5.01000022888184	-1795.28405761719\\
5.0149998664856	-1667.61291503906\\
5.01999998092651	-1432.76843261719\\
5.02500009536743	-1166.25012207031\\
5.03000020980835	-896.979248046875\\
5.03499984741211	-652.948303222656\\
5.03999996185303	-437.033508300781\\
5.04500007629395	-282.769744873047\\
5.05000019073486	-193.468170166016\\
5.05499982833862	-160.303695678711\\
5.05999994277954	-171.643035888672\\
5.06500005722046	-210.194183349609\\
5.07000017166138	-258.198791503906\\
5.07499980926514	-298.330078125\\
5.07999992370605	-314.489593505859\\
5.08500003814697	-231.151489257813\\
5.09000015258789	-153.372604370117\\
5.09499979019165	-70.3730773925781\\
5.09999990463257	19.8613948822021\\
5.10500001907349	54.1461067199707\\
5.1100001335144	55.8288612365723\\
5.11499977111816	52.6579246520996\\
5.11999988555908	47.7257690429688\\
5.125	42.4817390441895\\
5.13000011444092	37.5847473144531\\
5.13500022888184	722.248596191406\\
5.1399998664856	883.451965332031\\
5.14499998092651	913.475646972656\\
5.15000009536743	863.067626953125\\
5.15500020980835	774.92919921875\\
5.15999984741211	112.713241577148\\
5.16499996185303	-540.559692382813\\
5.17000007629395	-1101.68115234375\\
5.17500019073486	-1494.68579101563\\
5.17999982833862	-1678.92346191406\\
5.18499994277954	-1638.27075195313\\
5.19000005722046	-1506.72509765625\\
5.19500017166138	-1340.35278320313\\
5.19999980926514	-1127.19738769531\\
5.20499992370605	-906.955688476563\\
5.21000003814697	-717.056213378906\\
5.21500015258789	-585.555725097656\\
5.21999979019165	-526.696411132813\\
5.22499990463257	-550.422180175781\\
5.23000001907349	-634.141418457031\\
5.2350001335144	-750.792846679688\\
5.23999977111816	-879.082580566406\\
5.24499988555908	-993.142700195313\\
5.25	-1075.72521972656\\
5.25500011444092	-1119.80651855469\\
5.26000022888184	-1137.55773925781\\
5.2649998664856	-1116.1650390625\\
5.26999998092651	-1044.43151855469\\
5.27500009536743	-938.542846679688\\
5.28000020980835	-807.212158203125\\
5.28499984741211	-662.214782714844\\
5.28999996185303	-515.366821289063\\
5.29500007629395	-380.223419189453\\
5.30000019073486	-265.443908691406\\
5.30499982833862	-184.614486694336\\
5.30999994277954	-148.90412902832\\
5.31500005722046	-153.060363769531\\
5.32000017166138	-217.030181884766\\
5.32499980926514	-293.495422363281\\
5.32999992370605	-368.596801757813\\
5.33500003814697	-426.798034667969\\
5.34000015258789	-459.057739257813\\
5.34499979019165	-472.305541992188\\
5.34999990463257	-450.20166015625\\
5.35500001907349	-391.800476074219\\
5.3600001335144	-304.249481201172\\
5.36499977111816	-201.653533935547\\
5.36999988555908	-99.7173767089844\\
5.375	-18.8914012908936\\
5.38000011444092	30.6410121917725\\
5.38500022888184	42.9427833557129\\
5.3899998664856	15.6696014404297\\
5.39499998092651	-45.0415725708008\\
5.40000009536743	-102.719543457031\\
5.40500020980835	-140.362976074219\\
5.40999984741211	-161.037887573242\\
5.41499996185303	-150.02571105957\\
5.42000007629395	-103.019371032715\\
5.42500019073486	-27.4546089172363\\
5.42999982833862	42.5746955871582\\
5.43499994277954	34.4290008544922\\
5.44000005722046	21.5460166931152\\
5.44500017166138	24.6453018188477\\
5.44999980926514	29.0163841247559\\
5.45499992370605	28.8581676483154\\
5.46000003814697	26.8350467681885\\
5.46500015258789	24.1753540039063\\
5.46999979019165	21.6348705291748\\
5.47499990463257	19.1067123413086\\
5.48000001907349	17.2656059265137\\
5.4850001335144	15.1462535858154\\
5.48999977111816	13.5609540939331\\
5.49499988555908	11.9736404418945\\
5.5	10.813196182251\\
5.50500011444092	9.67001247406006\\
5.51000022888184	8.51384735107422\\
5.5149998664856	7.71457290649414\\
5.51999998092651	61.7477912902832\\
5.52500009536743	124.940254211426\\
5.53000020980835	141.979278564453\\
5.53499984741211	154.403869628906\\
5.53999996185303	162.597778320313\\
5.54500007629395	166.875244140625\\
5.55000019073486	167.602981567383\\
5.55499982833862	165.802444458008\\
5.55999994277954	164.581893920898\\
5.56500005722046	160.618179321289\\
5.57000017166138	152.118026733398\\
5.57499980926514	140.878753662109\\
5.57999992370605	127.096961975098\\
5.58500003814697	111.247741699219\\
5.59000015258789	94.3695678710938\\
5.59499979019165	77.244514465332\\
5.59999990463257	61.4129447937012\\
5.60500001907349	47.4507255554199\\
5.6100001335144	35.6649055480957\\
5.61499977111816	26.4417552947998\\
5.61999988555908	20.0868911743164\\
5.625	15.6651039123535\\
5.63000011444092	12.9601097106934\\
5.63500022888184	11.6289319992065\\
5.6399998664856	11.9240379333496\\
5.64499998092651	12.2405261993408\\
5.65000009536743	12.2744245529175\\
5.65500020980835	12.0666046142578\\
5.65999984741211	11.5696582794189\\
5.66499996185303	10.9824781417847\\
5.67000007629395	-5.34082746505737\\
5.67500019073486	-500.670501708984\\
5.67999982833862	-658.853942871094\\
5.68499994277954	-684.221984863281\\
5.69000005722046	-467.106384277344\\
5.69500017166138	-339.407745361328\\
5.69999980926514	-206.157165527344\\
5.70499992370605	-105.213623046875\\
5.71000003814697	-62.2781105041504\\
5.71500015258789	-81.7753448486328\\
5.71999979019165	-160.353302001953\\
5.72499990463257	-274.903656005859\\
5.73000001907349	-390.27490234375\\
5.7350001335144	-485.591918945313\\
5.73999977111816	-548.491333007813\\
5.74499988555908	-575.852478027344\\
5.75	-574.768249511719\\
5.75500011444092	-558.996337890625\\
5.76000022888184	-540.001037597656\\
5.7649998664856	-535.969787597656\\
5.76999998092651	-550.480163574219\\
5.77500009536743	-582.754150390625\\
5.78000020980835	-626.031616210938\\
5.78499984741211	-675.051513671875\\
5.78999996185303	-727.913818359375\\
5.79500007629395	-780.810974121094\\
5.80000019073486	-833.913513183594\\
5.80499982833862	-886.924621582031\\
5.80999994277954	-940.306457519531\\
5.81500005722046	-994.343444824219\\
5.82000017166138	-1051.2587890625\\
5.82499980926514	-1115.24401855469\\
5.82999992370605	-1181.38623046875\\
5.83500003814697	-1252.39294433594\\
5.84000015258789	-1325.6494140625\\
5.84499979019165	-1398.15515136719\\
5.84999990463257	-1462.95886230469\\
5.85500001907349	-1527.11254882813\\
5.8600001335144	-1566.83544921875\\
5.86499977111816	-1559.26293945313\\
5.86999988555908	-1479.32019042969\\
5.875	-1294.33569335938\\
5.88000011444092	-983.731079101563\\
5.88500022888184	-541.735168457031\\
5.8899998664856	-35.2216415405273\\
5.89499998092651	106.680313110352\\
5.90000009536743	173.817016601563\\
5.90500020980835	243.232452392578\\
5.90999984741211	374.159027099609\\
5.91499996185303	534.771240234375\\
5.92000007629395	684.592407226563\\
5.92500019073486	811.856750488281\\
5.92999982833862	924.332275390625\\
5.93499994277954	1032.81872558594\\
5.94000005722046	1154.3408203125\\
5.94500017166138	1294.46337890625\\
5.94999980926514	1448.7490234375\\
5.95499992370605	1608.65014648438\\
5.96000003814697	1808.42321777344\\
5.96500015258789	1957.07275390625\\
5.96999979019165	2012.3505859375\\
5.97499990463257	1944.04541015625\\
5.98000001907349	-297.992279052734\\
5.9850001335144	-4034.92114257813\\
5.98999977111816	-6455.13525390625\\
5.99499988555908	-7785.66943359375\\
6	-8041.5166015625\\
6.00500011444092	-7321.52734375\\
6.01000022888184	-6255.5732421875\\
6.0149998664856	-4742.4970703125\\
6.01999998092651	-2980.44458007813\\
6.02500009536743	-1219.20227050781\\
6.03000020980835	305.339141845703\\
6.03499984741211	1381.32482910156\\
6.03999996185303	1868.16174316406\\
6.04500007629395	1643.23950195313\\
6.05000019073486	784.439514160156\\
6.05499982833862	-386.462554931641\\
6.05999994277954	-1617.84460449219\\
6.06500005722046	-2688.31323242188\\
6.07000017166138	-3446.11376953125\\
6.07499980926514	-3808.43041992188\\
6.07999992370605	-3817.3916015625\\
6.08500003814697	-3619.7734375\\
6.09000015258789	-3081.1591796875\\
6.09499979019165	-2372.96362304688\\
6.09999990463257	-1611.31469726563\\
6.10500001907349	-892.194030761719\\
6.1100001335144	-303.380676269531\\
6.11499977111816	139.144165039063\\
6.11999988555908	345.069671630859\\
6.125	308.738250732422\\
6.13000011444092	8.18753719329834\\
6.13500022888184	-395.218353271484\\
6.1399998664856	-788.591430664063\\
6.14499998092651	-1099.37231445313\\
6.15000009536743	-1280.93762207031\\
6.15500020980835	-1344.84350585938\\
6.15999984741211	-1301.67456054688\\
6.16499996185303	-1123.43188476563\\
6.17000007629395	-843.338562011719\\
6.17500019073486	-502.808227539063\\
6.17999982833862	-152.684356689453\\
6.18499994277954	162.329345703125\\
6.19000005722046	235.064331054688\\
6.19500017166138	206.901733398438\\
6.19999980926514	156.077377319336\\
6.20499992370605	99.4274139404297\\
6.21000003814697	46.2111206054688\\
6.21500015258789	54.2791938781738\\
6.21999979019165	65.8905944824219\\
6.22499990463257	66.0036239624023\\
6.23000001907349	61.6230964660645\\
6.2350001335144	55.5583724975586\\
6.23999977111816	49.3388023376465\\
6.24499988555908	43.5666694641113\\
6.25	38.3540840148926\\
6.25500011444092	33.6833534240723\\
6.26000022888184	29.5518817901611\\
6.2649998664856	25.908821105957\\
6.26999998092651	22.7109146118164\\
6.27500009536743	19.9077816009521\\
6.28000020980835	17.4519157409668\\
6.28499984741211	15.2976093292236\\
6.28999996185303	13.4030103683472\\
6.29500007629395	11.740743637085\\
6.30000019073486	10.2856578826904\\
6.30499982833862	9.01039505004883\\
6.30999994277954	7.8882884979248\\
6.31500005722046	6.9036865234375\\
6.32000017166138	6.03891658782959\\
6.32499980926514	5.28151798248291\\
6.32999992370605	4.61994028091431\\
6.33500003814697	4.0438346862793\\
6.34000015258789	3.54764318466187\\
6.34499979019165	3.1079409122467\\
6.34999990463257	2.71472811698914\\
6.35500001907349	2.36362528800964\\
6.3600001335144	-346.261779785156\\
6.36499977111816	-546.023986816406\\
6.36999988555908	-686.328857421875\\
6.375	-764.912902832031\\
6.38000011444092	-785.369995117188\\
6.38500022888184	-699.475891113281\\
6.3899998664856	-328.133270263672\\
6.39499998092651	-175.683120727539\\
6.40000009536743	-60.0681457519531\\
6.40500020980835	-5.9901704788208\\
6.40999984741211	-53.7278137207031\\
6.41499996185303	-168.913711547852\\
6.42000007629395	-334.977294921875\\
6.42500019073486	-531.102966308594\\
6.42999982833862	-734.248229980469\\
6.43499994277954	-925.7978515625\\
6.44000005722046	-1089.9619140625\\
6.44500017166138	-1217.06909179688\\
6.44999980926514	-1302.58288574219\\
6.45499992370605	-1349.03820800781\\
6.46000003814697	-1361.79016113281\\
6.46500015258789	-1349.5595703125\\
6.46999979019165	-1323.24938964844\\
6.47499990463257	-1292.205078125\\
6.48000001907349	-1264.67236328125\\
6.4850001335144	-1247.66918945313\\
6.48999977111816	-1244.60632324219\\
6.49499988555908	-1256.69653320313\\
6.5	-1282.82055664063\\
6.50500011444092	-1320.15002441406\\
6.51000022888184	-1364.40893554688\\
6.5149998664856	-1411.11999511719\\
6.51999998092651	-1455.9248046875\\
6.52500009536743	-1494.95922851563\\
6.53000020980835	-1525.23645019531\\
6.53499984741211	-1546.20568847656\\
6.53999996185303	-1557.41967773438\\
6.54500007629395	-1559.40173339844\\
6.55000019073486	-1553.76818847656\\
6.55499982833862	-1543.25329589844\\
6.55999994277954	-1530.68786621094\\
6.56500005722046	-1515.20434570313\\
6.57000017166138	-1498.10510253906\\
6.57499980926514	-1479.6923828125\\
6.57999992370605	-1465.09350585938\\
6.58500003814697	-1452.18151855469\\
6.59000015258789	-1442.29418945313\\
6.59499979019165	-1435.84387207031\\
6.59999990463257	-1431.68859863281\\
6.60500001907349	-1428.970703125\\
6.6100001335144	-1426.90466308594\\
6.61499977111816	-1425.17346191406\\
6.61999988555908	-1421.87219238281\\
6.625	-1416.62927246094\\
6.63000011444092	-1409.017578125\\
6.63500022888184	-1399.16186523438\\
6.6399998664856	-1387.99597167969\\
6.64499998092651	-1375.48168945313\\
6.65000009536743	-1361.73449707031\\
6.65500020980835	-1348.359375\\
6.65999984741211	-1335.79138183594\\
6.66499996185303	-1324.15502929688\\
6.67000007629395	-1315.04418945313\\
6.67500019073486	-1308.28308105469\\
6.67999982833862	-1303.6396484375\\
6.68499994277954	-1299.33898925781\\
6.69000005722046	-1294.9521484375\\
6.69500017166138	-1289.89794921875\\
6.69999980926514	-1287.73461914063\\
6.70499992370605	-1285.11926269531\\
6.71000003814697	-1281.45703125\\
6.71500015258789	-1274.69958496094\\
6.71999979019165	-1265.29272460938\\
6.72499990463257	-1253.67919921875\\
6.73000001907349	-1239.58874511719\\
6.7350001335144	-1223.1162109375\\
6.73999977111816	-1206.55493164063\\
6.74499988555908	-1190.05822753906\\
6.75	-1175.06591796875\\
6.75500011444092	-1160.85229492188\\
6.76000022888184	-1144.09069824219\\
6.7649998664856	-1131.388671875\\
6.76999998092651	-1121.03796386719\\
6.77500009536743	-1113.79467773438\\
6.78000020980835	-1108.55969238281\\
6.78499984741211	-1105.14050292969\\
6.78999996185303	-1104.2822265625\\
6.79500007629395	-1104.64758300781\\
6.80000019073486	-1105.88586425781\\
6.80499982833862	-1107.48986816406\\
6.80999994277954	-1109.41479492188\\
6.81500005722046	-1110.77258300781\\
6.82000017166138	-1112.02770996094\\
6.82499980926514	-1112.60229492188\\
6.82999992370605	-1112.40844726563\\
6.83500003814697	-1110.54699707031\\
6.84000015258789	-1109.5517578125\\
6.84499979019165	-1108.67883300781\\
6.84999990463257	-1108.70190429688\\
6.85500001907349	-1109.76525878906\\
6.8600001335144	-1111.52478027344\\
6.86499977111816	-1115.79772949219\\
6.86999988555908	-1120.33435058594\\
6.875	-1123.19445800781\\
6.88000011444092	-1125.92590332031\\
6.88500022888184	-1126.7216796875\\
6.8899998664856	-1124.86010742188\\
6.89499998092651	-1119.57763671875\\
6.90000009536743	-1110.537109375\\
6.90500020980835	-1097.4853515625\\
6.90999984741211	-1081.15270996094\\
6.91499996185303	-1061.29992675781\\
6.92000007629395	-1038.65979003906\\
6.92500019073486	-1014.08282470703\\
6.92999982833862	-988.050109863281\\
6.93499994277954	-961.737915039063\\
6.94000005722046	-935.325805664063\\
6.94500017166138	-909.327514648438\\
6.94999980926514	-884.042053222656\\
6.95499992370605	-860.896484375\\
6.96000003814697	-837.603759765625\\
6.96500015258789	-812.900329589844\\
6.96999979019165	-789.556945800781\\
6.97499990463257	-764.845031738281\\
6.98000001907349	-739.387573242188\\
6.9850001335144	-712.103515625\\
6.98999977111816	-683.794677734375\\
6.99499988555908	-654.8857421875\\
7	-624.165466308594\\
7.00500011444092	-593.298278808594\\
7.01000022888184	-562.426574707031\\
7.0149998664856	-532.186767578125\\
7.01999998092651	-502.936798095703\\
7.02500009536743	-474.759185791016\\
7.03000020980835	-447.952362060547\\
7.03499984741211	-422.692169189453\\
7.03999996185303	-398.994232177734\\
7.04500007629395	-376.790313720703\\
7.05000019073486	-355.930755615234\\
7.05499982833862	-336.145660400391\\
7.05999994277954	-317.490905761719\\
7.06500005722046	-299.548706054688\\
7.07000017166138	-282.077667236328\\
7.07499980926514	-265.145812988281\\
7.07999992370605	-248.743743896484\\
7.08500003814697	-232.898666381836\\
7.09000015258789	-217.731475830078\\
7.09499979019165	-203.3984375\\
7.09999990463257	-190.243011474609\\
7.10500001907349	-178.48046875\\
7.1100001335144	-168.312194824219\\
7.11499977111816	-159.633712768555\\
7.11999988555908	-152.692642211914\\
7.125	-147.67399597168\\
7.13000011444092	-144.10466003418\\
7.13500022888184	-141.950164794922\\
7.1399998664856	-141.124633789063\\
7.14499998092651	-141.241973876953\\
7.15000009536743	-142.155258178711\\
7.15500020980835	-143.885986328125\\
7.15999984741211	-145.710540771484\\
7.16499996185303	-147.417037963867\\
7.17000007629395	-150.442840576172\\
7.17500019073486	-154.122055053711\\
7.17999982833862	-158.287170410156\\
7.18499994277954	-163.232040405273\\
7.19000005722046	-168.984451293945\\
7.19500017166138	-175.54118347168\\
7.19999980926514	-182.840667724609\\
7.20499992370605	-190.821258544922\\
7.21000003814697	-199.278839111328\\
7.21500015258789	-208.06184387207\\
7.21999979019165	-217.167037963867\\
7.22499990463257	-226.308334350586\\
7.23000001907349	-235.496765136719\\
7.2350001335144	-244.682678222656\\
7.23999977111816	-253.783889770508\\
7.24499988555908	-262.789855957031\\
7.25	-271.667633056641\\
7.25500011444092	-280.376953125\\
7.26000022888184	-288.907165527344\\
7.2649998664856	-297.248992919922\\
7.26999998092651	-305.409454345703\\
7.27500009536743	-313.419769287109\\
7.28000020980835	-321.177276611328\\
7.28499984741211	-328.686981201172\\
7.28999996185303	-335.865631103516\\
7.29500007629395	-342.737945556641\\
7.30000019073486	-349.235687255859\\
7.30499982833862	-355.372741699219\\
7.30999994277954	-360.996276855469\\
7.31500005722046	-366.153472900391\\
7.32000017166138	-370.795257568359\\
7.32499980926514	-374.936096191406\\
7.32999992370605	-378.551605224609\\
7.33500003814697	-381.634155273438\\
7.34000015258789	-384.158447265625\\
7.34499979019165	-386.120330810547\\
7.34999990463257	-387.557373046875\\
7.35500001907349	-388.46728515625\\
7.3600001335144	-388.866180419922\\
7.36499977111816	-388.763244628906\\
7.36999988555908	-388.221130371094\\
7.375	-387.337738037109\\
7.38000011444092	-386.13037109375\\
7.38500022888184	-384.544860839844\\
7.3899998664856	-382.558990478516\\
7.39499998092651	-380.197509765625\\
7.40000009536743	-377.363616943359\\
7.40500020980835	-374.066040039063\\
7.40999984741211	-370.376342773438\\
7.41499996185303	-366.303924560547\\
7.42000007629395	-361.875793457031\\
7.42500019073486	-357.13525390625\\
7.42999982833862	-352.109527587891\\
7.43499994277954	-346.854278564453\\
7.44000005722046	-341.400909423828\\
7.44500017166138	-335.763549804688\\
7.44999980926514	-329.979156494141\\
7.45499992370605	-324.121429443359\\
7.46000003814697	-318.272644042969\\
7.46500015258789	-312.451110839844\\
7.46999979019165	-306.583374023438\\
7.47499990463257	-300.687377929688\\
7.48000001907349	-294.849639892578\\
7.4850001335144	-289.072174072266\\
7.48999977111816	-283.368316650391\\
7.49499988555908	-277.759429931641\\
7.5	-272.259918212891\\
7.50500011444092	-266.878265380859\\
7.51000022888184	-261.626403808594\\
7.5149998664856	-256.513610839844\\
7.51999998092651	-251.530364990234\\
7.52500009536743	-246.468185424805\\
7.53000020980835	-242.088851928711\\
7.53499984741211	-237.857833862305\\
7.53999996185303	-233.88200378418\\
7.54500007629395	-230.274597167969\\
7.55000019073486	-226.979736328125\\
7.55499982833862	-223.978744506836\\
7.55999994277954	-221.28694152832\\
7.56500005722046	-218.888366699219\\
7.57000017166138	-216.755249023438\\
7.57499980926514	-214.882080078125\\
7.57999992370605	-213.266067504883\\
7.58500003814697	-211.894378662109\\
7.59000015258789	-210.790084838867\\
7.59499979019165	-209.949172973633\\
7.59999990463257	-209.364105224609\\
7.60500001907349	-209.032592773438\\
7.6100001335144	-209.006881713867\\
7.61499977111816	-209.34391784668\\
7.61999988555908	-210.053207397461\\
7.625	-211.139526367188\\
7.63000011444092	-212.509521484375\\
7.63500022888184	-214.022338867188\\
7.6399998664856	-215.738189697266\\
7.64499998092651	-217.650588989258\\
7.65000009536743	-219.703262329102\\
7.65500020980835	-221.877731323242\\
7.65999984741211	-224.187133789063\\
7.66499996185303	-226.604125976563\\
7.67000007629395	-229.090774536133\\
7.67500019073486	-231.654449462891\\
7.67999982833862	-234.327072143555\\
7.68499994277954	-237.11701965332\\
7.69000005722046	-240.033584594727\\
7.69500017166138	-242.993438720703\\
7.69999980926514	-245.983413696289\\
7.70499992370605	-248.982666015625\\
7.71000003814697	-252.005599975586\\
7.71500015258789	-255.133361816406\\
7.71999979019165	-258.290069580078\\
7.72499990463257	-261.3974609375\\
7.73000001907349	-264.424682617188\\
7.7350001335144	-267.337738037109\\
7.73999977111816	-270.273345947266\\
7.74499988555908	-273.153564453125\\
7.75	-275.861267089844\\
7.75500011444092	-278.47265625\\
7.76000022888184	-281.029510498047\\
7.7649998664856	-283.465148925781\\
7.76999998092651	-285.762939453125\\
7.77500009536743	-287.934295654297\\
7.78000020980835	-289.984344482422\\
7.78499984741211	-291.953979492188\\
7.78999996185303	-293.818267822266\\
7.79500007629395	-295.577209472656\\
7.80000019073486	-297.196960449219\\
7.80499982833862	-298.690399169922\\
7.80999994277954	-300.060333251953\\
7.81500005722046	-301.294006347656\\
7.82000017166138	-302.389892578125\\
7.82499980926514	-303.352508544922\\
7.82999992370605	-304.190032958984\\
7.83500003814697	-304.911651611328\\
7.84000015258789	-305.509948730469\\
7.84499979019165	-305.986663818359\\
7.84999990463257	-306.351196289063\\
7.85500001907349	-306.601837158203\\
7.8600001335144	-306.740661621094\\
7.86499977111816	-306.790679931641\\
7.86999988555908	-306.750274658203\\
7.875	-306.625457763672\\
7.88000011444092	-306.428436279297\\
7.88500022888184	-306.167144775391\\
7.8899998664856	-305.848175048828\\
7.89499998092651	-305.478393554688\\
7.90000009536743	-305.064605712891\\
7.90500020980835	-304.61474609375\\
7.90999984741211	-304.133148193359\\
7.91499996185303	-303.617095947266\\
7.92000007629395	-303.075775146484\\
7.92500019073486	-302.516082763672\\
7.92999982833862	-301.971588134766\\
7.93499994277954	-301.422729492188\\
7.94000005722046	-300.869567871094\\
7.94500017166138	-300.358306884766\\
7.94999980926514	-299.876098632813\\
7.95499992370605	-299.416442871094\\
7.96000003814697	-298.999450683594\\
7.96500015258789	-298.631225585938\\
7.96999979019165	-298.302551269531\\
7.97499990463257	-298.062225341797\\
7.98000001907349	-297.987518310547\\
7.9850001335144	-298.022338867188\\
7.98999977111816	-298.164245605469\\
7.99499988555908	-298.397735595703\\
8	-298.731781005859\\
8.00500011444092	-299.166412353516\\
8.01000022888184	-299.711669921875\\
8.01500034332275	-300.366790771484\\
8.02000045776367	-301.133453369141\\
8.02499961853027	-302.044189453125\\
8.02999973297119	-303.115570068359\\
8.03499984741211	-304.350646972656\\
8.03999996185303	-305.746673583984\\
8.04500007629395	-307.293273925781\\
8.05000019073486	-309.004730224609\\
8.05500030517578	-310.882110595703\\
8.0600004196167	-312.914245605469\\
8.0649995803833	-315.1083984375\\
8.06999969482422	-317.464996337891\\
8.07499980926514	-319.947052001953\\
8.07999992370605	-322.562774658203\\
8.08500003814697	-325.305725097656\\
8.09000015258789	-328.226806640625\\
8.09500026702881	-331.332946777344\\
8.10000038146973	-334.622650146484\\
8.10499954223633	-338.073455810547\\
8.10999965667725	-341.661529541016\\
8.11499977111816	-345.398559570313\\
8.11999988555908	-349.289886474609\\
8.125	-353.359039306641\\
8.13000011444092	-357.597442626953\\
8.13500022888184	-362.005432128906\\
8.14000034332275	-366.548950195313\\
8.14500045776367	-371.238250732422\\
8.14999961853027	-376.068786621094\\
8.15499973297119	-381.081329345703\\
8.15999984741211	-386.273193359375\\
8.16499996185303	-391.645874023438\\
8.17000007629395	-397.135833740234\\
8.17500019073486	-402.701904296875\\
8.18000030517578	-408.351165771484\\
8.1850004196167	-414.145172119141\\
8.1899995803833	-420.225402832031\\
8.19499969482422	-426.550109863281\\
8.19999980926514	-433.09033203125\\
8.20499992370605	-439.792449951172\\
8.21000003814697	-446.688568115234\\
8.21500015258789	-453.791870117188\\
8.22000026702881	-461.114990234375\\
8.22500038146973	-468.657012939453\\
8.22999954223633	-476.042388916016\\
8.23499965667725	-483.048553466797\\
8.23999977111816	-490.337768554688\\
8.24499988555908	-498.445678710938\\
8.25	-506.912750244141\\
8.25500011444092	-515.383239746094\\
8.26000022888184	-524.084289550781\\
8.26500034332275	-533.01220703125\\
8.27000045776367	-542.039916992188\\
8.27499961853027	-551.19482421875\\
8.27999973297119	-560.797485351563\\
8.28499984741211	-570.742553710938\\
8.28999996185303	-580.7158203125\\
8.29500007629395	-590.955200195313\\
8.30000019073486	-601.602111816406\\
8.30500030517578	-612.929138183594\\
8.3100004196167	-624.804809570313\\
8.3149995803833	-633.447875976563\\
8.31999969482422	-645.633056640625\\
8.32499980926514	-658.061157226563\\
8.32999992370605	-670.696105957031\\
8.33500003814697	-683.467041015625\\
8.34000015258789	-696.666259765625\\
8.34500026702881	-710.508972167969\\
8.35000038146973	-724.649169921875\\
8.35499954223633	-740.002380371094\\
8.35999965667725	-754.200439453125\\
8.36499977111816	-768.73583984375\\
8.36999988555908	-785.08447265625\\
8.375	-801.715270996094\\
8.38000011444092	-818.349243164063\\
8.38500022888184	-835.926208496094\\
8.39000034332275	-853.976318359375\\
8.39500045776367	-873.534851074219\\
8.39999961853027	-890.7490234375\\
8.40499973297119	-910.655456542969\\
8.40999984741211	-931.241333007813\\
8.41499996185303	-951.666931152344\\
8.42000007629395	-972.894409179688\\
8.42500019073486	-994.552185058594\\
8.43000030517578	-1017.19396972656\\
8.4350004196167	-1040.02966308594\\
8.4399995803833	-1062.50952148438\\
8.44499969482422	-1087.03369140625\\
8.44999980926514	-1112.23486328125\\
8.45499992370605	-1137.84399414063\\
8.46000003814697	-1165.13781738281\\
8.46500015258789	-1191.95922851563\\
8.47000026702881	-1218.91967773438\\
8.47500038146973	-1245.55834960938\\
8.47999954223633	-1271.42236328125\\
8.48499965667725	-1297.50769042969\\
8.48999977111816	-1324.47143554688\\
8.49499988555908	-1349.03161621094\\
8.5	-1373.02612304688\\
8.50500011444092	-1395.69262695313\\
8.51000022888184	-1417.79309082031\\
8.51500034332275	-1437.95678710938\\
8.52000045776367	-1455.36804199219\\
8.52499961853027	-1477.60888671875\\
8.52999973297119	-1495.66479492188\\
8.53499984741211	-1513.7275390625\\
8.53999996185303	-1531.2080078125\\
8.54500007629395	-1547.57214355469\\
8.55000019073486	-1563.19555664063\\
8.55500030517578	-1577.6181640625\\
8.5600004196167	-1591.35180664063\\
8.5649995803833	-1603.16174316406\\
8.56999969482422	-1613.02783203125\\
8.57499980926514	-1621.43420410156\\
8.57999992370605	-1627.48706054688\\
8.58500003814697	-1631.67736816406\\
8.59000015258789	-1633.76086425781\\
8.59500026702881	-1631.236328125\\
8.60000038146973	-1622.98376464844\\
8.60499954223633	-1609.3486328125\\
8.60999965667725	-1590.24462890625\\
8.61499977111816	-1566.67724609375\\
8.61999988555908	-1540.45385742188\\
8.625	-1512.82104492188\\
8.63000011444092	-1485.6474609375\\
8.63500022888184	-1459.73522949219\\
8.64000034332275	-1436.90686035156\\
8.64500045776367	-1416.8154296875\\
8.64999961853027	-1400.828125\\
8.65499973297119	-1385.84521484375\\
8.65999984741211	-1377.8056640625\\
8.66499996185303	-1367.70544433594\\
8.67000007629395	-1357.11083984375\\
8.67500019073486	-1343.94787597656\\
8.68000030517578	-1328.85827636719\\
8.6850004196167	-1310.13720703125\\
8.6899995803833	-1287.09216308594\\
8.69499969482422	-1260.2939453125\\
8.69999980926514	-1230.65930175781\\
8.70499992370605	-1199.1962890625\\
8.71000003814697	-1166.19348144531\\
8.71500015258789	-1132.22253417969\\
8.72000026702881	-1099.99230957031\\
8.72500038146973	-1069.63159179688\\
8.72999954223633	-1041.53881835938\\
8.73499965667725	-1016.357421875\\
8.73999977111816	-994.408508300781\\
8.74499988555908	-974.995544433594\\
8.75	-957.081970214844\\
8.75500011444092	-938.211853027344\\
8.76000022888184	-920.301208496094\\
8.76500034332275	-901.3232421875\\
8.77000045776367	-881.37158203125\\
8.77499961853027	-860.016906738281\\
8.77999973297119	-837.651916503906\\
8.78499984741211	-814.252746582031\\
8.78999996185303	-790.318908691406\\
8.79500007629395	-766.559875488281\\
8.80000019073486	-743.664733886719\\
8.80500030517578	-722.219421386719\\
8.8100004196167	-702.729553222656\\
8.8149995803833	-685.568298339844\\
8.81999969482422	-670.74853515625\\
8.82499980926514	-658.201538085938\\
8.82999992370605	-647.665893554688\\
8.83500003814697	-638.554626464844\\
8.84000015258789	-630.313781738281\\
8.84500026702881	-622.524230957031\\
8.85000038146973	-614.559326171875\\
8.85499954223633	-606.383178710938\\
8.85999965667725	-597.844909667969\\
8.86499977111816	-589.008361816406\\
8.86999988555908	-580.091857910156\\
8.875	-571.502746582031\\
8.88000011444092	-563.684387207031\\
8.88500022888184	-556.9541015625\\
8.89000034332275	-551.662902832031\\
8.89500045776367	-547.926513671875\\
8.89999961853027	-545.825256347656\\
8.90499973297119	-545.428039550781\\
8.90999984741211	-546.125915527344\\
8.91499996185303	-547.084533691406\\
8.92000007629395	-548.529235839844\\
8.92500019073486	-550.26416015625\\
8.93000030517578	-551.86767578125\\
8.9350004196167	-553.157104492188\\
8.9399995803833	-553.855773925781\\
8.94499969482422	-554.309692382813\\
8.94999980926514	-554.701843261719\\
8.95499992370605	-555.305358886719\\
8.96000003814697	-556.209899902344\\
8.96500015258789	-557.618041992188\\
8.97000026702881	-559.8095703125\\
8.97500038146973	-562.502136230469\\
8.97999954223633	-565.55908203125\\
8.98499965667725	-569.651489257813\\
8.98999977111816	-573.730163574219\\
8.99499988555908	-577.364562988281\\
9	-581.482971191406\\
9.00500011444092	-585.572937011719\\
9.01000022888184	-589.520141601563\\
9.01500034332275	-593.261352539063\\
9.02000045776367	-596.798522949219\\
9.02499961853027	-600.101379394531\\
9.02999973297119	-603.171447753906\\
9.03499984741211	-606.056640625\\
9.03999996185303	-608.773742675781\\
9.04500007629395	-611.332336425781\\
9.05000019073486	-613.743713378906\\
9.05500030517578	-616.015319824219\\
9.0600004196167	-618.119506835938\\
9.0649995803833	-620.069458007813\\
9.06999969482422	-621.814331054688\\
9.07499980926514	-623.351684570313\\
9.07999992370605	-624.679626464844\\
9.08500003814697	-625.795043945313\\
9.09000015258789	-626.630432128906\\
9.09500026702881	-626.989440917969\\
9.10000038146973	-626.85009765625\\
9.10499954223633	-626.596740722656\\
9.10999965667725	-626.078002929688\\
9.11499977111816	-625.06982421875\\
9.11999988555908	-623.512145996094\\
9.125	-621.548095703125\\
9.13000011444092	-619.343994140625\\
9.13500022888184	-616.900817871094\\
9.14000034332275	-614.475708007813\\
9.14500045776367	-612.004699707031\\
9.14999961853027	-609.659240722656\\
9.15499973297119	-607.420471191406\\
9.15999984741211	-605.273864746094\\
9.16499996185303	-603.149963378906\\
9.17000007629395	-600.916076660156\\
9.17500019073486	-598.517822265625\\
9.18000030517578	-595.856872558594\\
9.1850004196167	-592.887817382813\\
9.1899995803833	-589.609252929688\\
9.19499969482422	-585.991638183594\\
9.19999980926514	-582.069091796875\\
9.20499992370605	-577.908203125\\
9.21000003814697	-573.6630859375\\
9.21500015258789	-569.583312988281\\
9.22000026702881	-565.604064941406\\
9.22500038146973	-561.917602539063\\
9.22999954223633	-558.501586914063\\
9.23499965667725	-555.106262207031\\
9.23999977111816	-551.076843261719\\
9.24499988555908	-545.137573242188\\
9.25	-537.776489257813\\
9.25500011444092	-530.470947265625\\
9.26000022888184	-523.321594238281\\
9.26500034332275	-516.608154296875\\
9.27000045776367	-513.22119140625\\
9.27499961853027	-512.721801757813\\
9.27999973297119	-514.910278320313\\
9.28499984741211	-520.296752929688\\
9.28999996185303	-528.539978027344\\
9.29500007629395	-539.872619628906\\
9.30000019073486	-554.960754394531\\
9.30500030517578	-569.169555664063\\
9.3100004196167	-590.464172363281\\
9.3149995803833	-615.489501953125\\
9.31999969482422	-644.882202148438\\
9.32499980926514	-679.61376953125\\
9.32999992370605	-718.668640136719\\
9.33500003814697	-766.382446289063\\
9.34000015258789	-824.043579101563\\
9.34500026702881	-883.28515625\\
9.35000038146973	-959.911010742188\\
9.35499954223633	-1050.23498535156\\
9.35999965667725	-1142.61962890625\\
9.36499977111816	-1245.64331054688\\
9.36999988555908	-1348.91674804688\\
9.375	-1445.35363769531\\
9.38000011444092	-1529.7236328125\\
9.38500022888184	-1596.0498046875\\
9.39000034332275	-1638.79846191406\\
9.39500045776367	-1653.07116699219\\
9.39999961853027	-1640.72180175781\\
9.40499973297119	-1600.63537597656\\
9.40999984741211	-1532.74438476563\\
9.41499996185303	-1439.87609863281\\
9.42000007629395	-1324.17028808594\\
9.42500019073486	-1187.50744628906\\
9.43000030517578	-1019.41949462891\\
9.4350004196167	-840.914794921875\\
9.4399995803833	-673.315368652344\\
9.44499969482422	-538.884216308594\\
9.44999980926514	-478.982208251953\\
9.45499992370605	-527.72607421875\\
9.46000003814697	-733.987243652344\\
9.46500015258789	-1105.9521484375\\
9.47000026702881	-1635.50463867188\\
9.47500038146973	-2262.716796875\\
9.47999954223633	-2885.21240234375\\
9.48499965667725	-3204.34057617188\\
9.48999977111816	-2823.4912109375\\
9.49499988555908	-2620.943359375\\
9.5	-2481.333984375\\
9.50500011444092	-2347.7978515625\\
9.51000022888184	-2196.66137695313\\
9.51500034332275	-2020.34887695313\\
9.52000045776367	-1811.97583007813\\
9.52499961853027	-1573.92004394531\\
9.52999973297119	-1297.68017578125\\
9.53499984741211	-996.584838867188\\
9.53999996185303	-674.842163085938\\
9.54500007629395	-341.83544921875\\
9.55000019073486	-10.7475099563599\\
9.55500030517578	93.9508895874023\\
9.5600004196167	119.991500854492\\
9.5649995803833	122.045059204102\\
9.56999969482422	114.846755981445\\
9.57499980926514	103.817977905273\\
9.57999992370605	92.8202438354492\\
9.58500003814697	82.1155776977539\\
9.59000015258789	72.2184448242188\\
9.59500026702881	63.5441398620605\\
9.60000038146973	55.8182907104492\\
9.60499954223633	48.9992256164551\\
9.60999965667725	43.0015525817871\\
9.61499977111816	37.7714691162109\\
9.61999988555908	33.1441955566406\\
9.625	29.0893707275391\\
9.63000011444092	25.5616340637207\\
9.63500022888184	22.5001220703125\\
9.64000034332275	19.8040084838867\\
9.64500045776367	17.3870296478271\\
9.64999961853027	15.2999887466431\\
9.65499973297119	13.5409555435181\\
9.65999984741211	11.9604225158691\\
9.66499996185303	10.5760917663574\\
9.67000007629395	9.36545658111572\\
9.67500019073486	8.31750583648682\\
9.68000030517578	7.40217876434326\\
9.6850004196167	6.58992338180542\\
9.6899995803833	5.88073682785034\\
9.69499969482422	5.26991558074951\\
9.69999980926514	4.73231792449951\\
9.70499992370605	4.24934196472168\\
9.71000003814697	3.82514905929565\\
9.71500015258789	3.46498441696167\\
9.72000026702881	3.1530864238739\\
9.72500038146973	2.88226556777954\\
9.72999954223633	2.64334344863892\\
9.73499965667725	2.4255964756012\\
9.73999977111816	2.21676754951477\\
9.74499988555908	2.00502729415894\\
9.75	1.83722734451294\\
9.75500011444092	1.70201897621155\\
9.76000022888184	1.61187434196472\\
9.76500034332275	1.54515314102173\\
9.77000045776367	1.47139513492584\\
9.77499961853027	1.40076506137848\\
9.77999973297119	1.32541227340698\\
9.78499984741211	1.26854526996613\\
9.78999996185303	1.21788036823273\\
9.79500007629395	1.17345523834229\\
9.80000019073486	1.1340012550354\\
9.80500030517578	1.09831643104553\\
9.8100004196167	1.06742811203003\\
9.8149995803833	1.04162073135376\\
9.81999969482422	1.01734709739685\\
9.82499980926514	0.995693147182465\\
9.82999992370605	0.976282954216003\\
9.83500003814697	0.960774779319763\\
9.84000015258789	0.951257944107056\\
9.84500026702881	0.948152124881744\\
9.85000038146973	0.951705753803253\\
9.85499954223633	0.942520380020142\\
9.85999965667725	0.926800608634949\\
9.86499977111816	0.902075052261353\\
9.86999988555908	0.8832927942276\\
9.875	0.881842255592346\\
9.88000011444092	0.893049299716949\\
9.88500022888184	0.916947424411774\\
9.89000034332275	0.913069725036621\\
9.89500045776367	0.895473301410675\\
9.89999961853027	0.859537243843079\\
9.90499973297119	0.833877563476563\\
9.90999984741211	0.838663935661316\\
9.91499996185303	0.865519642829895\\
9.92000007629395	0.914748966693878\\
9.92500019073486	0.924092710018158\\
9.93000030517578	0.916442036628723\\
9.9350004196167	0.885407626628876\\
9.9399995803833	0.859995007514954\\
9.94499969482422	0.852085888385773\\
9.94999980926514	0.846130549907684\\
9.95499992370605	0.842129111289978\\
9.96000003814697	0.840081453323364\\
9.96500015258789	0.839987635612488\\
9.97000026702881	0.841847658157349\\
9.97500038146973	0.842938542366028\\
9.97999954223633	0.843427121639252\\
9.98499965667725	0.844726920127869\\
9.98999977111816	0.846837878227234\\
9.99499988555908	0.849759995937347\\
10	0.853493332862854\\
};
\addlegendentry{RS}

\addplot [color=red, line width=2.0pt]
  table[row sep=crcr]{%
0.0949999988079071	-17.3459358215332\\
0.100000001490116	-15.6278848648071\\
0.104999996721745	-14.1891918182373\\
0.109999999403954	-12.854998588562\\
0.115000002086163	-11.7085342407227\\
0.119999997317791	271.577087402344\\
0.125	-48.06982421875\\
0.129999995231628	-280.982360839844\\
0.135000005364418	-477.655090332031\\
0.140000000596046	-621.624816894531\\
0.144999995827675	-722.193542480469\\
0.150000005960464	-784.289978027344\\
0.155000001192093	-817.027648925781\\
0.159999996423721	-812.428344726563\\
0.165000006556511	-809.000183105469\\
0.170000001788139	-794.881103515625\\
0.174999997019768	-763.088012695313\\
0.180000007152557	-706.690063476563\\
0.185000002384186	-634.278442382813\\
0.189999997615814	-546.306335449219\\
0.194999992847443	-1407.18249511719\\
0.200000002980232	-1170.94714355469\\
0.204999998211861	-782.937805175781\\
0.209999993443489	-277.288604736328\\
0.215000003576279	20.478328704834\\
0.219999998807907	11.6841955184937\\
0.224999994039536	-15.9538745880127\\
0.230000004172325	-142.939407348633\\
0.234999999403954	-379.594055175781\\
0.239999994635582	-696.667907714844\\
0.245000004768372	-1053.28002929688\\
0.25	-1406.18371582031\\
0.254999995231628	-1789.16442871094\\
0.259999990463257	-1941.75915527344\\
0.264999985694885	-2024.47302246094\\
0.270000010728836	-2003.11328125\\
0.275000005960464	-1877.70617675781\\
0.280000001192093	-1678.47766113281\\
0.284999996423721	-1457.29821777344\\
0.28999999165535	-1250.33215332031\\
0.294999986886978	-1096.51733398438\\
0.300000011920929	-1027.8515625\\
0.305000007152557	-998.887634277344\\
0.310000002384186	-1011.68243408203\\
0.314999997615814	-1064.50073242188\\
0.319999992847443	-1141.06469726563\\
0.324999988079071	-1228.75134277344\\
0.330000013113022	-1288.76806640625\\
0.33500000834465	-1362.15344238281\\
0.340000003576279	-1431.48474121094\\
0.344999998807907	-1421.28405761719\\
0.349999994039536	-1363.81530761719\\
0.354999989271164	-1266.97839355469\\
0.360000014305115	-1148.21020507813\\
0.365000009536743	-1032.68774414063\\
0.370000004768372	-931.75390625\\
0.375	-832.731567382813\\
0.379999995231628	-750.874267578125\\
0.384999990463257	-693.462158203125\\
0.389999985694885	-659.105834960938\\
0.395000010728836	-653.587280273438\\
0.400000005960464	-660.557006835938\\
0.405000001192093	-673.321044921875\\
0.409999996423721	-689.348693847656\\
0.41499999165535	-688.584533691406\\
0.419999986886978	-675.552734375\\
0.425000011920929	-649.385192871094\\
0.430000007152557	-591.479736328125\\
0.435000002384186	-519.797912597656\\
0.439999997615814	-450.894836425781\\
0.444999992847443	-377.531829833984\\
0.449999988079071	-310.027374267578\\
0.455000013113022	-254.566375732422\\
0.46000000834465	-214.329528808594\\
0.465000003576279	-189.783660888672\\
0.469999998807907	-180.319030761719\\
0.474999994039536	-183.683685302734\\
0.479999989271164	-193.477432250977\\
0.485000014305115	-206.345138549805\\
0.490000009536743	-216.818466186523\\
0.495000004768372	-227.101104736328\\
0.5	-220.840789794922\\
0.504999995231628	-204.215362548828\\
0.509999990463257	-179.481842041016\\
0.514999985694885	-152.535522460938\\
0.519999980926514	-127.85986328125\\
0.524999976158142	-106.916976928711\\
0.529999971389771	-97.7239608764648\\
0.535000026226044	-99.3392105102539\\
0.540000021457672	-107.743598937988\\
0.545000016689301	-121.694160461426\\
0.550000011920929	-141.139709472656\\
0.555000007152557	-166.761688232422\\
0.560000002384186	-190.907180786133\\
0.564999997615814	-211.896148681641\\
0.569999992847443	-230.617706298828\\
0.574999988079071	-247.476989746094\\
0.579999983310699	-261.245880126953\\
0.584999978542328	-273.395141601563\\
0.589999973773956	-287.891235351563\\
0.595000028610229	-301.006317138672\\
0.600000023841858	-316.974365234375\\
0.605000019073486	-333.642852783203\\
0.610000014305115	-352.762512207031\\
0.615000009536743	-374.873474121094\\
0.620000004768372	-397.566497802734\\
0.625	-420.611480712891\\
0.629999995231628	-445.614074707031\\
0.634999990463257	-469.880493164063\\
0.639999985694885	-493.847686767578\\
0.644999980926514	-516.906982421875\\
0.649999976158142	-538.774291992188\\
0.654999971389771	-559.242248535156\\
0.660000026226044	-578.831176757813\\
0.665000021457672	-597.244934082031\\
0.670000016689301	-614.743041992188\\
0.675000011920929	-631.347045898438\\
0.680000007152557	-647.304016113281\\
0.685000002384186	-662.679809570313\\
0.689999997615814	-677.568908691406\\
0.694999992847443	-691.909790039063\\
0.699999988079071	-705.568420410156\\
0.704999983310699	-718.596069335938\\
0.709999978542328	-730.577026367188\\
0.714999973773956	-741.462158203125\\
0.720000028610229	-751.146789550781\\
0.725000023841858	-759.740417480469\\
0.730000019073486	-766.677612304688\\
0.735000014305115	-772.629699707031\\
0.740000009536743	-777.100036621094\\
0.745000004768372	-780.497009277344\\
0.75	-782.62841796875\\
0.754999995231628	-783.766174316406\\
0.759999990463257	-783.80078125\\
0.764999985694885	-782.793640136719\\
0.769999980926514	-781.650939941406\\
0.774999976158142	-779.786926269531\\
0.779999971389771	-776.864685058594\\
0.785000026226044	-772.878356933594\\
0.790000021457672	-767.998840332031\\
0.795000016689301	-762.181518554688\\
0.800000011920929	-755.613830566406\\
0.805000007152557	-748.244873046875\\
0.810000002384186	-740.271911621094\\
0.814999997615814	-731.59228515625\\
0.819999992847443	-722.274047851563\\
0.824999988079071	-712.612243652344\\
0.829999983310699	-702.279113769531\\
0.834999978542328	-691.546752929688\\
0.839999973773956	-680.661071777344\\
0.845000028610229	-669.406677246094\\
0.850000023841858	-658.008483886719\\
0.855000019073486	-646.443542480469\\
0.860000014305115	-635.022705078125\\
0.865000009536743	-623.604248046875\\
0.870000004768372	-612.146545410156\\
0.875	-600.796630859375\\
0.879999995231628	-589.55322265625\\
0.884999990463257	-578.464477539063\\
0.889999985694885	-567.581848144531\\
0.894999980926514	-556.934326171875\\
0.899999976158142	-546.5927734375\\
0.904999971389771	-536.513732910156\\
0.910000026226044	-526.918701171875\\
0.915000021457672	-517.658020019531\\
0.920000016689301	-508.847625732422\\
0.925000011920929	-500.480438232422\\
0.930000007152557	-492.656921386719\\
0.935000002384186	-485.328369140625\\
0.939999997615814	-478.592346191406\\
0.944999992847443	-472.438232421875\\
0.949999988079071	-467.00830078125\\
0.954999983310699	-462.116729736328\\
0.959999978542328	-457.828948974609\\
0.964999973773956	-454.079071044922\\
0.970000028610229	-450.972961425781\\
0.975000023841858	-448.386169433594\\
0.980000019073486	-446.405029296875\\
0.985000014305115	-444.955444335938\\
0.990000009536743	-444.062896728516\\
0.995000004768372	-443.704284667969\\
1	-443.963226318359\\
1.00499999523163	-444.754028320313\\
1.00999999046326	-446.228912353516\\
1.01499998569489	-448.380126953125\\
1.01999998092651	-450.949096679688\\
1.02499997615814	-453.812530517578\\
1.02999997138977	-457.035430908203\\
1.0349999666214	-460.618103027344\\
1.03999996185303	-464.540313720703\\
1.04499995708466	-468.770935058594\\
1.04999995231628	-473.294555664063\\
1.05499994754791	-478.085723876953\\
1.05999994277954	-483.119049072266\\
1.06500005722046	-488.378204345703\\
1.07000005245209	-493.830963134766\\
1.07500004768372	-499.444030761719\\
1.08000004291534	-505.178894042969\\
1.08500003814697	-511.016448974609\\
1.0900000333786	-516.926513671875\\
1.09500002861023	-522.879455566406\\
1.10000002384186	-528.84619140625\\
1.10500001907349	-534.808227539063\\
1.11000001430511	-540.756958007813\\
1.11500000953674	-546.635803222656\\
1.12000000476837	-552.423767089844\\
1.125	-558.112548828125\\
1.12999999523163	-563.66943359375\\
1.13499999046326	-569.065856933594\\
1.13999998569489	-574.277954101563\\
1.14499998092651	-579.285827636719\\
1.14999997615814	-584.082824707031\\
1.15499997138977	-588.644409179688\\
1.1599999666214	-592.954711914063\\
1.16499996185303	-597.0068359375\\
1.16999995708466	-600.787414550781\\
1.17499995231628	-604.283264160156\\
1.17999994754791	-607.50146484375\\
1.18499994277954	-610.422546386719\\
1.19000005722046	-613.037780761719\\
1.19500005245209	-615.334350585938\\
1.20000004768372	-617.314331054688\\
1.20500004291534	-618.984252929688\\
1.21000003814697	-620.344604492188\\
1.2150000333786	-621.44384765625\\
1.22000002861023	-622.235900878906\\
1.22500002384186	-622.735900878906\\
1.23000001907349	-622.940673828125\\
1.23500001430511	-622.866760253906\\
1.24000000953674	-622.559997558594\\
1.24500000476837	-622.037292480469\\
1.25	-621.308532714844\\
1.25499999523163	-620.328308105469\\
1.25999999046326	-619.098693847656\\
1.26499998569489	-617.589904785156\\
1.26999998092651	-615.795776367188\\
1.27499997615814	-613.790588378906\\
1.27999997138977	-611.568298339844\\
1.2849999666214	-609.153625488281\\
1.28999996185303	-606.577941894531\\
1.29499995708466	-603.861389160156\\
1.29999995231628	-601.072570800781\\
1.30499994754791	-598.1767578125\\
1.30999994277954	-595.200927734375\\
1.31500005722046	-592.180358886719\\
1.32000005245209	-589.106201171875\\
1.32500004768372	-585.992492675781\\
1.33000004291534	-582.840942382813\\
1.33500003814697	-579.666320800781\\
1.3400000333786	-576.484802246094\\
1.34500002861023	-573.302612304688\\
1.35000002384186	-570.129516601563\\
1.35500001907349	-566.986938476563\\
1.36000001430511	-563.871520996094\\
1.36500000953674	-560.763671875\\
1.37000000476837	-557.732116699219\\
1.375	-554.804870605469\\
1.37999999523163	-552.123962402344\\
1.38499999046326	-549.524047851563\\
1.38999998569489	-547.050170898438\\
1.39499998092651	-544.755554199219\\
1.39999997615814	-542.588073730469\\
1.40499997138977	-540.556884765625\\
1.4099999666214	-538.661499023438\\
1.41499996185303	-536.917663574219\\
1.41999995708466	-535.317077636719\\
1.42499995231628	-533.816772460938\\
1.42999994754791	-532.483337402344\\
1.43499994277954	-531.308349609375\\
1.44000005722046	-530.286804199219\\
1.44500005245209	-529.424377441406\\
1.45000004768372	-528.718383789063\\
1.45500004291534	-528.164123535156\\
1.46000003814697	-527.767395019531\\
1.4650000333786	-527.522033691406\\
1.47000002861023	-527.433959960938\\
1.47500002384186	-527.519104003906\\
1.48000001907349	-527.912658691406\\
1.48500001430511	-528.549987792969\\
1.49000000953674	-529.353637695313\\
1.49500000476837	-529.872680664063\\
1.5	-530.528747558594\\
1.50499999523163	-531.402160644531\\
1.50999999046326	-532.576721191406\\
1.51499998569489	-533.842834472656\\
1.51999998092651	-535.190795898438\\
1.52499997615814	-536.62646484375\\
1.52999997138977	-538.147338867188\\
1.5349999666214	-539.706115722656\\
1.53999996185303	-541.335754394531\\
1.54499995708466	-543.042907714844\\
1.54999995231628	-544.816284179688\\
1.55499994754791	-546.614685058594\\
1.55999994277954	-548.3984375\\
1.56500005722046	-550.185852050781\\
1.57000005245209	-564.007385253906\\
1.57500004768372	-544.705444335938\\
1.58000004291534	-543.595031738281\\
1.58500003814697	-547.026611328125\\
1.5900000333786	-550.606750488281\\
1.59500002861023	-554.149597167969\\
1.60000002384186	-557.454406738281\\
1.60500001907349	-560.540161132813\\
1.61000001430511	-563.390258789063\\
1.61500000953674	-565.429138183594\\
1.62000000476837	-567.473876953125\\
1.625	-568.600280761719\\
1.62999999523163	-569.228759765625\\
1.63499999046326	-569.350830078125\\
1.63999998569489	-569.058959960938\\
1.64499998092651	-569.954772949219\\
1.64999997615814	-570.639831542969\\
1.65499997138977	-571.297912597656\\
1.6599999666214	-572.128051757813\\
1.66499996185303	-573.032958984375\\
1.66999995708466	-573.673034667969\\
1.67499995231628	-574.568786621094\\
1.67999994754791	-575.663940429688\\
1.68499994277954	-576.356018066406\\
1.69000005722046	-576.8984375\\
1.69500005245209	-577.202575683594\\
1.70000004768372	-577.371948242188\\
1.70500004291534	-577.35888671875\\
1.71000003814697	-577.262145996094\\
1.7150000333786	-577.159484863281\\
1.72000002861023	-577.012756347656\\
1.72500002384186	-576.770751953125\\
1.73000001907349	-576.475646972656\\
1.73500001430511	-576.1142578125\\
1.74000000953674	-575.708129882813\\
1.74500000476837	-575.266845703125\\
1.75	-574.800109863281\\
1.75499999523163	-574.296325683594\\
1.75999999046326	-573.7841796875\\
1.76499998569489	-573.239685058594\\
1.76999998092651	-572.672607421875\\
1.77499997615814	-572.112854003906\\
1.77999997138977	-571.527038574219\\
1.7849999666214	-570.897583007813\\
1.78999996185303	-570.241088867188\\
1.79499995708466	-569.554138183594\\
1.79999995231628	-568.825134277344\\
1.80499994754791	-568.079223632813\\
1.80999994277954	-567.308044433594\\
1.81500005722046	-566.525329589844\\
1.82000005245209	-565.741821289063\\
1.82500004768372	-564.968566894531\\
1.83000004291534	-564.204406738281\\
1.83500003814697	-563.455871582031\\
1.8400000333786	-562.7353515625\\
1.84500002861023	-562.045593261719\\
1.85000002384186	-561.424499511719\\
1.85500001907349	-560.813415527344\\
1.86000001430511	-560.224426269531\\
1.86500000953674	-559.663330078125\\
1.87000000476837	-559.128723144531\\
1.875	-558.644226074219\\
1.87999999523163	-558.213439941406\\
1.88499999046326	-557.807067871094\\
1.88999998569489	-557.431396484375\\
1.89499998092651	-557.027282714844\\
1.89999997615814	-556.689331054688\\
1.90499997138977	-556.395263671875\\
1.9099999666214	-556.138854980469\\
1.91499996185303	-555.924621582031\\
1.91999995708466	-555.752380371094\\
1.92499995231628	-555.643371582031\\
1.92999994754791	-555.581970214844\\
1.93499994277954	-555.561096191406\\
1.94000005722046	-555.581848144531\\
1.94500005245209	-555.641845703125\\
1.95000004768372	-555.740905761719\\
1.95500004291534	-555.899475097656\\
1.96000003814697	-556.132507324219\\
1.9650000333786	-556.404174804688\\
1.97000002861023	-556.679443359375\\
1.97500002384186	-556.868408203125\\
1.98000001907349	-557.108337402344\\
1.98500001430511	-557.385131835938\\
1.99000000953674	-557.774780273438\\
1.99500000476837	-558.164916992188\\
2	-558.579406738281\\
2.00500011444092	-559.118896484375\\
2.00999999046326	-559.693115234375\\
2.01500010490417	-560.272216796875\\
2.01999998092651	-560.932861328125\\
2.02500009536743	-561.642456054688\\
2.02999997138977	-562.388366699219\\
2.03500008583069	-562.738952636719\\
2.03999996185303	-563.1875\\
2.04500007629395	-563.755310058594\\
2.04999995231628	-564.407348632813\\
2.0550000667572	-565.025329589844\\
2.05999994277954	-565.582641601563\\
2.06500005722046	-566.150573730469\\
2.0699999332428	-566.667724609375\\
2.07500004768372	-567.141235351563\\
2.07999992370605	-567.66064453125\\
2.08500003814697	-568.081604003906\\
2.08999991416931	-568.683044433594\\
2.09500002861023	-568.716186523438\\
2.09999990463257	-568.735900878906\\
2.10500001907349	-568.469482421875\\
2.10999989509583	-567.877563476563\\
2.11500000953674	-566.867797851563\\
2.11999988555908	-566.5107421875\\
2.125	-565.783020019531\\
2.13000011444092	-564.024658203125\\
2.13499999046326	-561.712219238281\\
2.14000010490417	-558.78759765625\\
2.14499998092651	-555.201599121094\\
2.15000009536743	-551.350646972656\\
2.15499997138977	-546.944030761719\\
2.16000008583069	-541.855834960938\\
2.16499996185303	-537.158020019531\\
2.17000007629395	-532.403015136719\\
2.17499995231628	-527.142456054688\\
2.1800000667572	-522.375671386719\\
2.18499994277954	-517.667724609375\\
2.19000005722046	-513.056335449219\\
2.1949999332428	-508.207824707031\\
2.20000004768372	-503.232177734375\\
2.20499992370605	-498.118225097656\\
2.21000003814697	-492.818084716797\\
2.21499991416931	-487.400634765625\\
2.22000002861023	-482.008758544922\\
2.22499990463257	-476.767700195313\\
2.23000001907349	-471.502044677734\\
2.23499989509583	-466.157287597656\\
2.24000000953674	-461.869995117188\\
2.24499988555908	-458.100830078125\\
2.25	-454.67431640625\\
2.25500011444092	-451.966949462891\\
2.25999999046326	-450.073364257813\\
2.26500010490417	-449.000762939453\\
2.26999998092651	-448.515319824219\\
2.27500009536743	-448.552612304688\\
2.27999997138977	-449.040863037109\\
2.28500008583069	-449.057891845703\\
2.28999996185303	-449.885528564453\\
2.29500007629395	-451.985717773438\\
2.29999995231628	-455.395233154297\\
2.3050000667572	-457.461029052734\\
2.30999994277954	-460.693786621094\\
2.31500005722046	-464.684387207031\\
2.3199999332428	-468.945220947266\\
2.32500004768372	-473.347351074219\\
2.32999992370605	-478.281097412109\\
2.33500003814697	-483.716644287109\\
2.33999991416931	-488.462249755859\\
2.34500002861023	-495.31982421875\\
2.34999990463257	-505.319244384766\\
2.35500001907349	-513.378723144531\\
2.35999989509583	-521.886047363281\\
2.36500000953674	-530.644226074219\\
2.36999988555908	-539.325561523438\\
2.375	-549.018859863281\\
2.38000011444092	-558.778015136719\\
2.38499999046326	-568.19384765625\\
2.39000010490417	-578.533447265625\\
2.39499998092651	-588.869079589844\\
2.40000009536743	-599.18701171875\\
2.40499997138977	-609.677551269531\\
2.41000008583069	-620.069641113281\\
2.41499996185303	-631.024963378906\\
2.42000007629395	-642.264770507813\\
2.42499995231628	-653.98583984375\\
2.4300000667572	-665.505432128906\\
2.43499994277954	-677.176940917969\\
2.44000005722046	-688.764038085938\\
2.4449999332428	-699.049011230469\\
2.45000004768372	-710.80615234375\\
2.45499992370605	-720.787292480469\\
2.46000003814697	-730.413513183594\\
2.46499991416931	-739.928894042969\\
2.47000002861023	-747.368774414063\\
2.47499990463257	-754.056274414063\\
2.48000001907349	-759.722106933594\\
2.48499989509583	-764.2021484375\\
2.49000000953674	-767.751953125\\
2.49499988555908	-770.674865722656\\
2.5	-772.152099609375\\
2.50500011444092	-771.911682128906\\
2.50999999046326	-770.478271484375\\
2.51500010490417	-768.466247558594\\
2.51999998092651	-767.314880371094\\
2.52500009536743	-764.711486816406\\
2.52999997138977	-761.265625\\
2.53500008583069	-757.912292480469\\
2.53999996185303	-752.301818847656\\
2.54500007629395	-746.322082519531\\
2.54999995231628	-739.335083007813\\
2.5550000667572	-731.379821777344\\
2.55999994277954	-722.549499511719\\
2.56500005722046	-712.319763183594\\
2.5699999332428	-701.457580566406\\
2.57500004768372	-689.910522460938\\
2.57999992370605	-677.761047363281\\
2.58500003814697	-665.226623535156\\
2.58999991416931	-652.092712402344\\
2.59500002861023	-639.597961425781\\
2.59999990463257	-627.084167480469\\
2.60500001907349	-614.239318847656\\
2.60999989509583	-601.659484863281\\
2.61500000953674	-589.048034667969\\
2.61999988555908	-576.41796875\\
2.625	-563.630310058594\\
2.63000011444092	-550.5341796875\\
2.63499999046326	-536.951904296875\\
2.64000010490417	-524.199829101563\\
2.64499998092651	-510.826812744141\\
2.65000009536743	-496.275512695313\\
2.65499997138977	-482.728729248047\\
2.66000008583069	-469.565155029297\\
2.66499996185303	-455.965179443359\\
2.67000007629395	-443.1953125\\
2.67499995231628	-431.459503173828\\
2.6800000667572	-419.853393554688\\
2.68499994277954	-408.557739257813\\
2.69000005722046	-398.04443359375\\
2.6949999332428	-388.217590332031\\
2.70000004768372	-378.303558349609\\
2.70499992370605	-368.792510986328\\
2.71000003814697	-359.410705566406\\
2.71499991416931	-349.661987304688\\
2.72000002861023	-340.836669921875\\
2.72499990463257	-332.178100585938\\
2.73000001907349	-323.81591796875\\
2.73499989509583	-315.957153320313\\
2.74000000953674	-308.592712402344\\
2.74499988555908	-302.091094970703\\
2.75	-296.8525390625\\
2.75500011444092	-292.74755859375\\
2.75999999046326	-289.735748291016\\
2.76500010490417	-287.868957519531\\
2.76999998092651	-287.213134765625\\
2.77500009536743	-287.942108154297\\
2.77999997138977	-290.483337402344\\
2.78500008583069	-294.278442382813\\
2.78999996185303	-299.253662109375\\
2.79500007629395	-305.309753417969\\
2.79999995231628	-312.315246582031\\
2.8050000667572	-320.164916992188\\
2.80999994277954	-329.464019775391\\
2.81500005722046	-340.047454833984\\
2.8199999332428	-351.850006103516\\
2.82500004768372	-365.229034423828\\
2.82999992370605	-379.3984375\\
2.83500003814697	-396.704833984375\\
2.83999991416931	-415.763732910156\\
2.84500002861023	-436.602722167969\\
2.84999990463257	-460.567504882813\\
2.85500001907349	-483.099334716797\\
2.85999989509583	-507.547271728516\\
2.86500000953674	-531.548278808594\\
2.86999988555908	-559.474426269531\\
2.875	-584.887145996094\\
2.88000011444092	-611.019104003906\\
2.88499999046326	-638.293518066406\\
2.89000010490417	-665.259399414063\\
2.89499998092651	-697.335998535156\\
2.90000009536743	-730.197082519531\\
2.90499997138977	-762.603088378906\\
2.91000008583069	-794.267150878906\\
2.91499996185303	-824.0087890625\\
2.92000007629395	-849.93408203125\\
2.92499995231628	-871.327575683594\\
2.9300000667572	-888.09912109375\\
2.93499994277954	-898.568237304688\\
2.94000005722046	-907.72802734375\\
2.9449999332428	-913.045593261719\\
2.95000004768372	-916.55615234375\\
2.95499992370605	-915.208251953125\\
2.96000003814697	-914.032592773438\\
2.96499991416931	-915.818725585938\\
2.97000002861023	-913.94287109375\\
2.97499990463257	-910.581176757813\\
2.98000001907349	-905.663269042969\\
2.98499989509583	-889.55712890625\\
2.99000000953674	-891.249877929688\\
2.99499988555908	-879.783996582031\\
3	-866.718688964844\\
3.00500011444092	-852.012390136719\\
3.00999999046326	-834.632629394531\\
3.01500010490417	-815.376647949219\\
3.01999998092651	-791.628845214844\\
3.02500009536743	-764.269165039063\\
3.02999997138977	-736.931213378906\\
3.03500008583069	-707.545471191406\\
3.03999996185303	-676.476257324219\\
3.04500007629395	-645.353149414063\\
3.04999995231628	-615.773315429688\\
3.0550000667572	-587.411926269531\\
3.05999994277954	-559.311706542969\\
3.06500005722046	-532.505737304688\\
3.0699999332428	-506.421203613281\\
3.07500004768372	-478.366119384766\\
3.07999992370605	-450.700744628906\\
3.08500003814697	-422.073150634766\\
3.08999991416931	-392.302612304688\\
3.09500002861023	-362.906158447266\\
3.09999990463257	-334.500732421875\\
3.10500001907349	-306.839996337891\\
3.10999989509583	-281.183380126953\\
3.11500000953674	-257.437316894531\\
3.11999988555908	-236.984924316406\\
3.125	-219.86572265625\\
3.13000011444092	-207.749328613281\\
3.13499999046326	-203.231292724609\\
3.14000010490417	-197.931900024414\\
3.14499998092651	-194.5419921875\\
3.15000009536743	-189.475250244141\\
3.15499997138977	-186.827590942383\\
3.16000008583069	-189.489181518555\\
3.16499996185303	-194.687942504883\\
3.17000007629395	-202.399017333984\\
3.17499995231628	-213.039260864258\\
3.1800000667572	-230.356704711914\\
3.18499994277954	-251.036285400391\\
3.19000005722046	-276.914764404297\\
3.1949999332428	-304.637756347656\\
3.20000004768372	-334.515533447266\\
3.20499992370605	-369.223327636719\\
3.21000003814697	-403.632049560547\\
3.21499991416931	-442.011169433594\\
3.22000002861023	-479.221496582031\\
3.22499990463257	-515.606262207031\\
3.23000001907349	-550.936157226563\\
3.23499989509583	-584.985778808594\\
3.24000000953674	-618.555358886719\\
3.24499988555908	-650.749389648438\\
3.25	-679.66357421875\\
3.25500011444092	-706.261840820313\\
3.25999999046326	-730.180358886719\\
3.26500010490417	-750.638305664063\\
3.26999998092651	-767.375610351563\\
3.27500009536743	-782.828735351563\\
3.27999997138977	-796.081970214844\\
3.28500008583069	-808.504821777344\\
3.28999996185303	-821.322692871094\\
3.29500007629395	-834.021240234375\\
3.29999995231628	-845.891967773438\\
3.3050000667572	-856.647766113281\\
3.30999994277954	-866.916015625\\
3.31500005722046	-873.72802734375\\
3.3199999332428	-876.348571777344\\
3.32500004768372	-880.179870605469\\
3.32999992370605	-881.70068359375\\
3.33500003814697	-878.4033203125\\
3.33999991416931	-872.572143554688\\
3.34500002861023	-863.504638671875\\
3.34999990463257	-849.628295898438\\
3.35500001907349	-833.080139160156\\
3.35999989509583	-814.626708984375\\
3.36500000953674	-794.223266601563\\
3.36999988555908	-772.830017089844\\
3.375	-750.1064453125\\
3.38000011444092	-727.202758789063\\
3.38499999046326	-703.370849609375\\
3.39000010490417	-678.204956054688\\
3.39499998092651	-651.77099609375\\
3.40000009536743	-623.83935546875\\
3.40499997138977	-593.832214355469\\
3.41000008583069	-563.151733398438\\
3.41499996185303	-529.348876953125\\
3.42000007629395	-494.853637695313\\
3.42499995231628	-460.035552978516\\
3.4300000667572	-425.248199462891\\
3.43499994277954	-391.481018066406\\
3.44000005722046	-359.393402099609\\
3.4449999332428	-328.867980957031\\
3.45000004768372	-300.785125732422\\
3.45499992370605	-275.980773925781\\
3.46000003814697	-259.02001953125\\
3.46499991416931	-242.823974609375\\
3.47000002861023	-221.590576171875\\
3.47499990463257	-210.071426391602\\
3.48000001907349	-193.624725341797\\
3.48499989509583	-178.038635253906\\
3.49000000953674	-161.747131347656\\
3.49499988555908	-159.714004516602\\
3.5	-166.744140625\\
3.50500011444092	-175.009826660156\\
3.50999999046326	-188.131225585938\\
3.51500010490417	-207.727355957031\\
3.51999998092651	-228.070785522461\\
3.52500009536743	-255.511444091797\\
3.52999997138977	-287.130828857422\\
3.53500008583069	-321.442108154297\\
3.53999996185303	-362.517395019531\\
3.54500007629395	-420.440338134766\\
3.54999995231628	-504.313354492188\\
3.5550000667572	-604.509033203125\\
3.55999994277954	-701.200378417969\\
3.56500005722046	-787.876647949219\\
3.5699999332428	-848.606140136719\\
3.57500004768372	-888.135864257813\\
3.57999992370605	-899.873718261719\\
3.58500003814697	-898.259216308594\\
3.58999991416931	-883.63037109375\\
3.59500002861023	-853.184631347656\\
3.59999990463257	-821.311401367188\\
3.60500001907349	-804.899230957031\\
3.60999989509583	-805.37890625\\
3.61500000953674	-814.438110351563\\
3.61999988555908	-825.30908203125\\
3.625	-840.633361816406\\
3.63000011444092	-860.661254882813\\
3.63499999046326	-876.043273925781\\
3.64000010490417	-892.783630371094\\
3.64499998092651	-907.167785644531\\
3.65000009536743	-918.507080078125\\
3.65499997138977	-906.010559082031\\
3.66000008583069	-875.725769042969\\
3.66499996185303	-837.160339355469\\
3.67000007629395	-791.395629882813\\
3.67499995231628	-741.654541015625\\
3.6800000667572	-693.138061523438\\
3.68499994277954	-650.141479492188\\
3.69000005722046	-614.447631835938\\
3.6949999332428	-585.656982421875\\
3.70000004768372	-566.276977539063\\
3.70499992370605	-547.0361328125\\
3.71000003814697	-528.915405273438\\
3.71499991416931	-507.392974853516\\
3.72000002861023	-478.616973876953\\
3.72499990463257	-444.061920166016\\
3.73000001907349	-401.704742431641\\
3.73499989509583	-356.703765869141\\
3.74000000953674	-309.569580078125\\
3.74499988555908	-265.791778564453\\
3.75	-233.700469970703\\
3.75500011444092	-214.086624145508\\
3.75999999046326	-203.215194702148\\
3.76500010490417	-192.852340698242\\
3.76999998092651	-181.501647949219\\
3.77500009536743	-166.376602172852\\
3.77999997138977	-157.574462890625\\
3.78500008583069	-152.922576904297\\
3.78999996185303	-158.906616210938\\
3.79500007629395	-173.018981933594\\
3.79999995231628	-188.747909545898\\
3.8050000667572	-206.61213684082\\
3.80999994277954	-230.938583374023\\
3.81500005722046	-262.223663330078\\
3.8199999332428	-301.778015136719\\
3.82500004768372	-349.583892822266\\
3.82999992370605	-408.425384521484\\
3.83500003814697	-488.508636474609\\
3.83999991416931	-596.537658691406\\
3.84500002861023	-721.345458984375\\
3.84999990463257	-828.0673828125\\
3.85500001907349	-906.498168945313\\
3.85999989509583	-949.081726074219\\
3.86500000953674	-968.779479980469\\
3.86999988555908	-964.694763183594\\
3.875	-935.093994140625\\
3.88000011444092	-889.076232910156\\
3.88499999046326	-829.382446289063\\
3.89000010490417	-795.710815429688\\
3.89499998092651	-788.72216796875\\
3.90000009536743	-809.825805664063\\
3.90499997138977	-826.48583984375\\
3.91000008583069	-853.529907226563\\
3.91499996185303	-887.380249023438\\
3.92000007629395	-915.117065429688\\
3.92499995231628	-938.709899902344\\
3.9300000667572	-968.612365722656\\
3.93499994277954	-979.021362304688\\
3.94000005722046	-962.803405761719\\
3.9449999332428	-924.021789550781\\
3.95000004768372	-875.083862304688\\
3.95499992370605	-824.786865234375\\
3.96000003814697	-766.016357421875\\
3.96499991416931	-708.902465820313\\
3.97000002861023	-659.08349609375\\
3.97499990463257	-619.51220703125\\
3.98000001907349	-589.245056152344\\
3.98499989509583	-565.44775390625\\
3.99000000953674	-545.216857910156\\
3.99499988555908	-523.098083496094\\
4	-496.168487548828\\
4.00500011444092	-463.424163818359\\
4.01000022888184	-418.355804443359\\
4.0149998664856	-366.06298828125\\
4.01999998092651	-309.035247802734\\
4.02500009536743	-251.717407226563\\
4.03000020980835	-202.337661743164\\
4.03499984741211	-167.850189208984\\
4.03999996185303	-140.207077026367\\
4.04500007629395	-116.158851623535\\
4.05000019073486	-94.2361373901367\\
4.05499982833862	-70.9871063232422\\
4.05999994277954	-45.9273910522461\\
4.06500005722046	-26.6715888977051\\
4.07000017166138	-24.2107067108154\\
4.07499980926514	-32.3629608154297\\
4.07999992370605	-39.3767700195313\\
4.08500003814697	-47.5638961791992\\
4.09000015258789	-59.8730888366699\\
4.09499979019165	-73.7119216918945\\
4.09999990463257	-90.6841354370117\\
4.10500001907349	-109.510055541992\\
4.1100001335144	-197.226638793945\\
4.11499977111816	-418.972717285156\\
4.11999988555908	-675.195190429688\\
4.125	-970.818481445313\\
4.13000011444092	-1192.564453125\\
4.13500022888184	-1282.01171875\\
4.1399998664856	-1331.9970703125\\
4.14499998092651	-1290.96960449219\\
4.15000009536743	-1173.19140625\\
4.15500020980835	-1005.90002441406\\
4.15999984741211	-827.495910644531\\
4.16499996185303	-679.542419433594\\
4.17000007629395	-674.648620605469\\
4.17500019073486	-676.237365722656\\
4.17999982833862	-705.1796875\\
4.18499994277954	-772.241577148438\\
4.19000005722046	-861.096008300781\\
4.19500017166138	-949.982849121094\\
4.19999980926514	-1011.19329833984\\
4.20499992370605	-1113.23571777344\\
4.21000003814697	-1138.63256835938\\
4.21500015258789	-1097.71203613281\\
4.21999979019165	-1006.91156005859\\
4.22499990463257	-889.882141113281\\
4.23000001907349	-778.361755371094\\
4.2350001335144	-680.893859863281\\
4.23999977111816	-587.315307617188\\
4.24499988555908	-511.831146240234\\
4.25	-470.337799072266\\
4.25500011444092	-452.458892822266\\
4.26000022888184	-454.605682373047\\
4.2649998664856	-464.018951416016\\
4.26999998092651	-467.348052978516\\
4.27500009536743	-458.935699462891\\
4.28000020980835	-439.885498046875\\
4.28499984741211	-377.07373046875\\
4.28999996185303	-295.777496337891\\
4.29500007629395	-215.349029541016\\
4.30000019073486	-159.514678955078\\
4.30499982833862	-121.691078186035\\
4.30999994277954	-82.7169418334961\\
4.31500005722046	-48.2979583740234\\
4.32000017166138	-20.6575794219971\\
4.32499980926514	3.15163111686707\\
4.32999992370605	26.8264503479004\\
4.33500003814697	-2.55324769020081\\
4.34000015258789	-19.5775337219238\\
4.34499979019165	-29.4443130493164\\
4.34999990463257	-40.5962600708008\\
4.35500001907349	-53.231575012207\\
4.3600001335144	-69.9322204589844\\
4.36499977111816	-92.4248962402344\\
4.36999988555908	-209.791030883789\\
4.375	-483.746887207031\\
4.38000011444092	-729.598266601563\\
4.38500022888184	-939.467834472656\\
4.3899998664856	-1072.9814453125\\
4.39499998092651	-1437.39111328125\\
4.40000009536743	-1563.29992675781\\
4.40500020980835	-1536.78100585938\\
4.40999984741211	-1379.56103515625\\
4.41499996185303	-1141.73107910156\\
4.42000007629395	-880.349731445313\\
4.42500019073486	-619.076843261719\\
4.42999982833862	-572.035461425781\\
4.43499994277954	-571.576416015625\\
4.44000005722046	-612.103271484375\\
4.44500017166138	-702.834106445313\\
4.44999980926514	-827.548583984375\\
4.45499992370605	-970.423400878906\\
4.46000003814697	-1059.7236328125\\
4.46500015258789	-1207.75390625\\
4.46999979019165	-1260.62426757813\\
4.47499990463257	-1209.40100097656\\
4.48000001907349	-1093.02258300781\\
4.4850001335144	-936.571655273438\\
4.48999977111816	-783.916809082031\\
4.49499988555908	-666.274963378906\\
4.5	-548.120178222656\\
4.50500011444092	-458.349304199219\\
4.51000022888184	-409.631774902344\\
4.5149998664856	-397.93701171875\\
4.51999998092651	-408.738922119141\\
4.52500009536743	-432.644866943359\\
4.53000020980835	-444.141143798828\\
4.53499984741211	-442.669525146484\\
4.53999996185303	-426.514007568359\\
4.54500007629395	-349.233215332031\\
4.55000019073486	-180.959289550781\\
4.55499982833862	-163.331069946289\\
4.55999994277954	-56.5946388244629\\
4.56500005722046	-18.0301342010498\\
4.57000017166138	-21.0443649291992\\
4.57499980926514	-25.7542324066162\\
4.57999992370605	-23.7264976501465\\
4.58500003814697	-12.6230459213257\\
4.59000015258789	-2.87867593765259\\
4.59499979019165	-0.842852532863617\\
4.59999990463257	15.823371887207\\
4.60500001907349	41.6480255126953\\
4.6100001335144	80.9455871582031\\
4.61499977111816	137.113464355469\\
4.61999988555908	456.852813720703\\
4.625	536.277587890625\\
4.63000011444092	8.39792823791504\\
4.63500022888184	-584.860534667969\\
4.6399998664856	-1055.03552246094\\
4.64499998092651	-1379.3291015625\\
4.65000009536743	-1476.34204101563\\
4.65500020980835	-1531.41674804688\\
4.65999984741211	-1525.7509765625\\
4.66499996185303	-2480.79809570313\\
4.67000007629395	-1937.51892089844\\
4.67500019073486	-1383.0458984375\\
4.67999982833862	-854.387329101563\\
4.68499994277954	-326.6298828125\\
4.69000005722046	-203.856872558594\\
4.69500017166138	-157.337585449219\\
4.69999980926514	-210.978515625\\
4.70499992370605	-371.096954345703\\
4.71000003814697	-610.014221191406\\
4.71500015258789	-886.014343261719\\
4.71999979019165	-1084.14587402344\\
4.72499990463257	-1394.62939453125\\
4.73000001907349	-1490.08752441406\\
4.7350001335144	-1410.72998046875\\
4.73999977111816	-1195.70373535156\\
4.74499988555908	-910.310302734375\\
4.75	-621.388549804688\\
4.75500011444092	-442.938873291016\\
4.76000022888184	-264.024993896484\\
4.7649998664856	-136.811141967773\\
4.76999998092651	-103.33211517334\\
4.77500009536743	-118.247222900391\\
4.78000020980835	-193.085174560547\\
4.78499984741211	-310.721160888672\\
4.78999996185303	-396.133605957031\\
4.79500007629395	-353.071685791016\\
4.80000019073486	-436.178253173828\\
4.80499982833862	-351.124816894531\\
4.80999994277954	-207.842788696289\\
4.81500005722046	-131.324462890625\\
4.82000017166138	-99.2278747558594\\
4.82499980926514	-86.0814666748047\\
4.82999992370605	-68.8080139160156\\
4.83500003814697	-44.1467552185059\\
4.84000015258789	-18.0590152740479\\
4.84499979019165	35.9233207702637\\
4.84999990463257	113.423080444336\\
4.85500001907349	217.661087036133\\
4.8600001335144	329.722717285156\\
4.86499977111816	345.606872558594\\
4.86999988555908	474.684326171875\\
4.875	265.120544433594\\
4.88000011444092	-214.273498535156\\
4.88500022888184	-755.435119628906\\
4.8899998664856	-1197.39221191406\\
4.89499998092651	-1482.53979492188\\
4.90000009536743	-1530.19958496094\\
4.90500020980835	-1595.42041015625\\
4.90999984741211	-1568.28527832031\\
4.91499996185303	-1433.73986816406\\
4.92000007629395	-2244.75073242188\\
4.92500019073486	-1805.33959960938\\
4.92999982833862	-1112.58129882813\\
4.93499994277954	-376.469512939453\\
4.94000005722046	183.788116455078\\
4.94500017166138	178.968795776367\\
4.94999980926514	125.785774230957\\
4.95499992370605	-66.7649078369141\\
4.96000003814697	-373.530639648438\\
4.96500015258789	-733.435852050781\\
4.96999979019165	-1081.90759277344\\
4.97499990463257	-1338.08374023438\\
4.98000001907349	-1564.39282226563\\
4.9850001335144	-1621.78979492188\\
4.98999977111816	-1514.18774414063\\
4.99499988555908	-1302.23645019531\\
5	-1043.31823730469\\
5.00500011444092	-784.030822753906\\
5.01000022888184	-596.478149414063\\
5.0149998664856	-466.082824707031\\
5.01999998092651	-371.415710449219\\
5.02500009536743	-358.155975341797\\
5.03000020980835	-377.234405517578\\
5.03499984741211	-406.6923828125\\
5.03999996185303	-422.553161621094\\
5.04500007629395	-451.067535400391\\
5.05000019073486	-469.501953125\\
5.05499982833862	-463.387176513672\\
5.05999994277954	-431.749633789063\\
5.06500005722046	-376.628784179688\\
5.07000017166138	-302.958160400391\\
5.07499980926514	-210.574890136719\\
5.07999992370605	-106.272743225098\\
5.08500003814697	53.5075263977051\\
5.09000015258789	86.508430480957\\
5.09499979019165	91.6926116943359\\
5.09999990463257	92.7016372680664\\
5.10500001907349	47.5984153747559\\
5.1100001335144	22.2714214324951\\
5.11499977111816	14.9031085968018\\
5.11999988555908	11.0591201782227\\
5.125	8.34492683410645\\
5.13000011444092	6.95058107376099\\
5.13500022888184	561.839111328125\\
5.1399998664856	5.22566652297974\\
5.14499998092651	-409.736999511719\\
5.15000009536743	-670.742309570313\\
5.15500020980835	-795.363525390625\\
5.15999984741211	-1239.97705078125\\
5.16499996185303	-1228.26489257813\\
5.17000007629395	-1095.79907226563\\
5.17500019073486	-892.233459472656\\
5.17999982833862	-661.079040527344\\
5.18499994277954	-449.922058105469\\
5.19000005722046	-398.788604736328\\
5.19500017166138	-430.886199951172\\
5.19999980926514	-478.340972900391\\
5.20499992370605	-560.003051757813\\
5.21000003814697	-662.868286132813\\
5.21500015258789	-771.938842773438\\
5.21999979019165	-867.205627441406\\
5.22499990463257	-947.529663085938\\
5.23000001907349	-986.0712890625\\
5.2350001335144	-1000.34313964844\\
5.23999977111816	-987.875\\
5.24499988555908	-948.192504882813\\
5.25	-887.008728027344\\
5.25500011444092	-824.2373046875\\
5.26000022888184	-774.608154296875\\
5.2649998664856	-722.223449707031\\
5.26999998092651	-670.452392578125\\
5.27500009536743	-639.763305664063\\
5.28000020980835	-623.103454589844\\
5.28499984741211	-616.670288085938\\
5.28999996185303	-618.673522949219\\
5.29500007629395	-629.378662109375\\
5.30000019073486	-641.846008300781\\
5.30499982833862	-662.364440917969\\
5.30999994277954	-686.244079589844\\
5.31500005722046	-698.61083984375\\
5.32000017166138	-719.256286621094\\
5.32499980926514	-704.364868164063\\
5.32999992370605	-666.306396484375\\
5.33500003814697	-617.679748535156\\
5.34000015258789	-561.741516113281\\
5.34499979019165	-512.755676269531\\
5.34999990463257	-457.08837890625\\
5.35500001907349	-406.051300048828\\
5.3600001335144	-364.171630859375\\
5.36499977111816	-337.624176025391\\
5.36999988555908	-325.129333496094\\
5.375	-326.371459960938\\
5.38000011444092	-333.045379638672\\
5.38500022888184	-338.633026123047\\
5.3899998664856	-338.348999023438\\
5.39499998092651	-330.571166992188\\
5.40000009536743	-289.648406982422\\
5.40500020980835	-233.942886352539\\
5.40999984741211	-177.953704833984\\
5.41499996185303	-116.411506652832\\
5.42000007629395	-57.5890083312988\\
5.42500019073486	-9.0030460357666\\
5.42999982833862	9.77724742889404\\
5.43499994277954	-25.7634735107422\\
5.44000005722046	-19.7998561859131\\
5.44500017166138	5.8602557182312\\
5.44999980926514	15.3203897476196\\
5.45499992370605	13.2364301681519\\
5.46000003814697	10.1276388168335\\
5.46500015258789	7.87540483474731\\
5.46999979019165	6.55205297470093\\
5.47499990463257	5.60884857177734\\
5.48000001907349	5.07328414916992\\
5.4850001335144	4.42191934585571\\
5.48999977111816	4.09777212142944\\
5.49499988555908	3.70617771148682\\
5.5	3.4621696472168\\
5.50500011444092	3.15422201156616\\
5.51000022888184	2.86256456375122\\
5.5149998664856	2.69012665748596\\
5.51999998092651	45.6561470031738\\
5.52500009536743	38.0589866638184\\
5.53000020980835	-32.171802520752\\
5.53499984741211	-75.2509002685547\\
5.53999996185303	-103.889724731445\\
5.54500007629395	-122.985946655273\\
5.55000019073486	-136.70068359375\\
5.55499982833862	-143.449783325195\\
5.55999994277954	-143.630493164063\\
5.56500005722046	-142.759674072266\\
5.57000017166138	-141.575408935547\\
5.57499980926514	-136.779479980469\\
5.57999992370605	-128.623275756836\\
5.58500003814697	-117.686500549316\\
5.59000015258789	-104.834655761719\\
5.59499979019165	-90.238883972168\\
5.59999990463257	-74.4470062255859\\
5.60500001907349	-58.7007713317871\\
5.6100001335144	-43.9004974365234\\
5.61499977111816	-30.6348266601563\\
5.61999988555908	-19.327579498291\\
5.625	-10.6715841293335\\
5.63000011444092	-4.25975704193115\\
5.63500022888184	-0.0836495533585548\\
5.6399998664856	3.12626528739929\\
5.64499998092651	3.74621367454529\\
5.65000009536743	3.15029430389404\\
5.65500020980835	2.51756381988525\\
5.65999984741211	1.94346582889557\\
5.66499996185303	1.58998608589172\\
5.67000007629395	-13.6420087814331\\
5.67500019073486	-349.100006103516\\
5.67999982833862	-151.82942199707\\
5.68499994277954	-23.677173614502\\
5.69000005722046	159.177734375\\
5.69500017166138	57.2741203308105\\
5.69999980926514	-17.8342514038086\\
5.70499992370605	-95.0855712890625\\
5.71000003814697	-176.888931274414\\
5.71500015258789	-244.876861572266\\
5.71999979019165	-287.181518554688\\
5.72499990463257	-305.562286376953\\
5.73000001907349	-286.984741210938\\
5.7350001335144	-238.38249206543\\
5.73999977111816	-177.725204467773\\
5.74499988555908	-116.374008178711\\
5.75	-68.2587432861328\\
5.75500011444092	-39.6035919189453\\
5.76000022888184	-27.9353160858154\\
5.7649998664856	-37.6784057617188\\
5.76999998092651	-54.0507621765137\\
5.77500009536743	-71.4990921020508\\
5.78000020980835	-81.821403503418\\
5.78499984741211	-89.7557525634766\\
5.78999996185303	-93.4776458740234\\
5.79500007629395	-95.4878234863281\\
5.80000019073486	-95.1641387939453\\
5.80499982833862	-98.556037902832\\
5.80999994277954	-101.789245605469\\
5.81500005722046	-106.928901672363\\
5.82000017166138	-113.402984619141\\
5.82499980926514	-123.937171936035\\
5.82999992370605	-131.463134765625\\
5.83500003814697	-139.678421020508\\
5.84000015258789	-147.949447631836\\
5.84499979019165	-151.683837890625\\
5.84999990463257	-151.026443481445\\
5.85500001907349	-151.161239624023\\
5.8600001335144	-132.670364379883\\
5.86499977111816	-96.4264144897461\\
5.86999988555908	-38.1578674316406\\
5.875	47.9456596374512\\
5.88000011444092	155.401794433594\\
5.88500022888184	275.523468017578\\
5.8899998664856	326.015380859375\\
5.89499998092651	127.013053894043\\
5.90000009536743	50.1968231201172\\
5.90500020980835	6.05967283248901\\
5.90999984741211	-63.6800384521484\\
5.91499996185303	-265.474822998047\\
5.92000007629395	-611.067138671875\\
5.92500019073486	-1036.38244628906\\
5.92999982833862	-1473.12133789063\\
5.93499994277954	-1872.67224121094\\
5.94000005722046	-2205.63549804688\\
5.94500017166138	-2461.93603515625\\
5.94999980926514	-2642.83178710938\\
5.95499992370605	-2745.32836914063\\
5.96000003814697	-2737.32470703125\\
5.96500015258789	-2748.39794921875\\
5.96999979019165	-2721.43530273438\\
5.97499990463257	-2624.85131835938\\
5.98000001907349	-4707.02294921875\\
5.9850001335144	-4149.228515625\\
5.98999977111816	-3394.70336914063\\
5.99499988555908	-2220.5546875\\
6	-1123.6123046875\\
6.00500011444092	-277.175323486328\\
6.01000022888184	-88.8940811157227\\
6.0149998664856	-72.8045806884766\\
6.01999998092651	-284.119720458984\\
6.02500009536743	-728.195373535156\\
6.03000020980835	-1307.33935546875\\
6.03499984741211	-1944.53955078125\\
6.03999996185303	-2540.14038085938\\
6.04500007629395	-3074.66528320313\\
6.05000019073486	-3451.05322265625\\
6.05499982833862	-3512.13793945313\\
6.05999994277954	-3305.71166992188\\
6.06500005722046	-2904.4921875\\
6.07000017166138	-2402.509765625\\
6.07499980926514	-1880.58178710938\\
6.07999992370605	-1458.62585449219\\
6.08500003814697	-1194.22326660156\\
6.09000015258789	-897.858642578125\\
6.09499979019165	-779.944274902344\\
6.09999990463257	-787.904296875\\
6.10500001907349	-868.865539550781\\
6.1100001335144	-996.334045410156\\
6.11499977111816	-1103.57434082031\\
6.11999988555908	-1258.03259277344\\
6.125	-1379.00012207031\\
6.13000011444092	-1479.79772949219\\
6.13500022888184	-1424.07739257813\\
6.1399998664856	-1268.201171875\\
6.14499998092651	-1044.52404785156\\
6.15000009536743	-785.037048339844\\
6.15500020980835	-554.827880859375\\
6.15999984741211	-349.715637207031\\
6.16499996185303	-162.517883300781\\
6.17000007629395	-22.3001136779785\\
6.17500019073486	65.8780212402344\\
6.17999982833862	90.8482284545898\\
6.18499994277954	73.067253112793\\
6.19000005722046	-103.036041259766\\
6.19500017166138	-166.552795410156\\
6.19999980926514	-163.818267822266\\
6.20499992370605	-141.121429443359\\
6.21000003814697	-99.3492050170898\\
6.21500015258789	-5.82116365432739\\
6.21999979019165	26.0324153900146\\
6.22499990463257	24.6441497802734\\
6.23000001907349	18.7739105224609\\
6.2350001335144	14.6000270843506\\
6.23999977111816	11.728009223938\\
6.24499988555908	10.1748600006104\\
6.25	9.12943744659424\\
6.25500011444092	8.33899974822998\\
6.26000022888184	7.70386838912964\\
6.2649998664856	7.15631532669067\\
6.26999998092651	6.65941286087036\\
6.27500009536743	6.19133567810059\\
6.28000020980835	5.74293184280396\\
6.28499984741211	5.31856441497803\\
6.28999996185303	4.9141263961792\\
6.29500007629395	4.53239154815674\\
6.30000019073486	4.17494201660156\\
6.30499982833862	3.83894658088684\\
6.30999994277954	3.5250551700592\\
6.31500005722046	3.23559331893921\\
6.32000017166138	2.96287322044373\\
6.32499980926514	2.7145369052887\\
6.32999992370605	2.48394417762756\\
6.33500003814697	2.2700879573822\\
6.34000015258789	2.07542729377747\\
6.34499979019165	1.89824914932251\\
6.34999990463257	1.73492455482483\\
6.35500001907349	1.58519268035889\\
6.3600001335144	-248.529418945313\\
6.36499977111816	-166.050598144531\\
6.36999988555908	-107.216819763184\\
6.375	-64.1005172729492\\
6.38000011444092	-22.7099685668945\\
6.38500022888184	72.7747192382813\\
6.3899998664856	267.901458740234\\
6.39499998092651	-46.2287063598633\\
6.40000009536743	-296.904693603516\\
6.40500020980835	-501.564483642578\\
6.40999984741211	-698.372497558594\\
6.41499996185303	-840.103393554688\\
6.42000007629395	-942.81396484375\\
6.42500019073486	-1013.02917480469\\
6.42999982833862	-1047.54467773438\\
6.43499994277954	-1060.3037109375\\
6.44000005722046	-1055.86499023438\\
6.44500017166138	-1047.98706054688\\
6.44999980926514	-1040.61938476563\\
6.45499992370605	-1045.29113769531\\
6.46000003814697	-1060.97473144531\\
6.46500015258789	-1088.86926269531\\
6.46999979019165	-1129.26586914063\\
6.47499990463257	-1179.23132324219\\
6.48000001907349	-1233.79455566406\\
6.4850001335144	-1290.68408203125\\
6.48999977111816	-1346.60559082031\\
6.49499988555908	-1395.09228515625\\
6.5	-1434.36303710938\\
6.50500011444092	-1464.81787109375\\
6.51000022888184	-1487.13977050781\\
6.5149998664856	-1500.47314453125\\
6.51999998092651	-1505.63073730469\\
6.52500009536743	-1505.9365234375\\
6.53000020980835	-1500.95837402344\\
6.53499984741211	-1494.36499023438\\
6.53999996185303	-1487.29418945313\\
6.54500007629395	-1479.97644042969\\
6.55000019073486	-1474.16381835938\\
6.55499982833862	-1469.048828125\\
6.55999994277954	-1466.60241699219\\
6.56500005722046	-1461.2822265625\\
6.57000017166138	-1456.57214355469\\
6.57499980926514	-1450.52526855469\\
6.57999992370605	-1446.09655761719\\
6.58500003814697	-1438.568359375\\
6.59000015258789	-1429.81750488281\\
6.59499979019165	-1419.52014160156\\
6.59999990463257	-1407.12854003906\\
6.60500001907349	-1392.57434082031\\
6.6100001335144	-1376.78930664063\\
6.61499977111816	-1360.119140625\\
6.61999988555908	-1342.06579589844\\
6.625	-1324.31286621094\\
6.63000011444092	-1306.48034667969\\
6.63500022888184	-1289.62731933594\\
6.6399998664856	-1274.38647460938\\
6.64499998092651	-1260.19567871094\\
6.65000009536743	-1247.16442871094\\
6.65500020980835	-1236.95056152344\\
6.65999984741211	-1227.52966308594\\
6.66499996185303	-1219.50463867188\\
6.67000007629395	-1213.62133789063\\
6.67500019073486	-1208.50256347656\\
6.67999982833862	-1204.37768554688\\
6.68499994277954	-1199.91857910156\\
6.69000005722046	-1196.83837890625\\
6.69500017166138	-1193.88464355469\\
6.69999980926514	-1194.69616699219\\
6.70499992370605	-1195.54040527344\\
6.71000003814697	-1197.65600585938\\
6.71500015258789	-1199.60302734375\\
6.71999979019165	-1203.17749023438\\
6.72499990463257	-1207.91748046875\\
6.73000001907349	-1213.38500976563\\
6.7350001335144	-1220.00231933594\\
6.73999977111816	-1228.89782714844\\
6.74499988555908	-1237.80981445313\\
6.75	-1248.04370117188\\
6.75500011444092	-1257.77429199219\\
6.76000022888184	-1265.24890136719\\
6.7649998664856	-1277.73461914063\\
6.76999998092651	-1287.45166015625\\
6.77500009536743	-1297.265625\\
6.78000020980835	-1307.30114746094\\
6.78499984741211	-1315.33068847656\\
6.78999996185303	-1323.6689453125\\
6.79500007629395	-1330.20849609375\\
6.80000019073486	-1335.92541503906\\
6.80499982833862	-1341.14575195313\\
6.80999994277954	-1345.56811523438\\
6.81500005722046	-1349.1298828125\\
6.82000017166138	-1352.68395996094\\
6.82499980926514	-1355.53503417969\\
6.82999992370605	-1357.90002441406\\
6.83500003814697	-1359.30688476563\\
6.84000015258789	-1361.98657226563\\
6.84499979019165	-1363.32800292969\\
6.84999990463257	-1364.08642578125\\
6.85500001907349	-1363.6845703125\\
6.8600001335144	-1361.55578613281\\
6.86499977111816	-1358.93640136719\\
6.86999988555908	-1352.17053222656\\
6.875	-1342.22570800781\\
6.88000011444092	-1331.43310546875\\
6.88500022888184	-1317.70092773438\\
6.8899998664856	-1301.78723144531\\
6.89499998092651	-1283.74194335938\\
6.90000009536743	-1264.77136230469\\
6.90500020980835	-1243.10302734375\\
6.90999984741211	-1221.400390625\\
6.91499996185303	-1199.05395507813\\
6.92000007629395	-1175.87963867188\\
6.92500019073486	-1152.17553710938\\
6.92999982833862	-1127.68957519531\\
6.93499994277954	-1102.86511230469\\
6.94000005722046	-1077.08752441406\\
6.94500017166138	-1050.54272460938\\
6.94999980926514	-1023.09527587891\\
6.95499992370605	-995.408020019531\\
6.96000003814697	-964.777404785156\\
6.96500015258789	-932.416809082031\\
6.96999979019165	-901.447875976563\\
6.97499990463257	-868.141357421875\\
6.98000001907349	-834.942443847656\\
6.9850001335144	-801.01806640625\\
6.98999977111816	-767.816955566406\\
6.99499988555908	-735.018371582031\\
7	-701.776428222656\\
7.00500011444092	-670.118469238281\\
7.01000022888184	-639.129577636719\\
7.0149998664856	-609.052856445313\\
7.01999998092651	-579.7451171875\\
7.02500009536743	-551.041320800781\\
7.03000020980835	-523.128723144531\\
7.03499984741211	-495.969177246094\\
7.03999996185303	-469.481384277344\\
7.04500007629395	-443.667022705078\\
7.05000019073486	-418.5546875\\
7.05499982833862	-394.167419433594\\
7.05999994277954	-370.825469970703\\
7.06500005722046	-348.281372070313\\
7.07000017166138	-326.622131347656\\
7.07499980926514	-306.463287353516\\
7.07999992370605	-287.643859863281\\
7.08500003814697	-270.188537597656\\
7.09000015258789	-254.222946166992\\
7.09499979019165	-239.793914794922\\
7.09999990463257	-227.006042480469\\
7.10500001907349	-215.723220825195\\
7.1100001335144	-205.839691162109\\
7.11499977111816	-197.198364257813\\
7.11999988555908	-189.88346862793\\
7.125	-183.849472045898\\
7.13000011444092	-178.662536621094\\
7.13500022888184	-174.530044555664\\
7.1399998664856	-171.416168212891\\
7.14499998092651	-169.160522460938\\
7.15000009536743	-167.808334350586\\
7.15500020980835	-167.69075012207\\
7.15999984741211	-168.426834106445\\
7.16499996185303	-170.295364379883\\
7.17000007629395	-174.392013549805\\
7.17500019073486	-178.862991333008\\
7.17999982833862	-183.792343139648\\
7.18499994277954	-189.597640991211\\
7.19000005722046	-196.021865844727\\
7.19500017166138	-202.9912109375\\
7.19999980926514	-210.429946899414\\
7.20499992370605	-218.269989013672\\
7.21000003814697	-226.33708190918\\
7.21500015258789	-234.676574707031\\
7.21999979019165	-243.341720581055\\
7.22499990463257	-252.014129638672\\
7.23000001907349	-260.906280517578\\
7.2350001335144	-269.950775146484\\
7.23999977111816	-279.063079833984\\
7.24499988555908	-288.259307861328\\
7.25	-297.455413818359\\
7.25500011444092	-306.573547363281\\
7.26000022888184	-315.614685058594\\
7.2649998664856	-324.491149902344\\
7.26999998092651	-333.170928955078\\
7.27500009536743	-341.670074462891\\
7.28000020980835	-349.822662353516\\
7.28499984741211	-357.612243652344\\
7.28999996185303	-364.985900878906\\
7.29500007629395	-371.972198486328\\
7.30000019073486	-378.529479980469\\
7.30499982833862	-384.594360351563\\
7.30999994277954	-390.077728271484\\
7.31500005722046	-395.094970703125\\
7.32000017166138	-399.616058349609\\
7.32499980926514	-403.6416015625\\
7.32999992370605	-407.154205322266\\
7.33500003814697	-410.132110595703\\
7.34000015258789	-412.551940917969\\
7.34499979019165	-414.434906005859\\
7.34999990463257	-415.810211181641\\
7.35500001907349	-416.653076171875\\
7.3600001335144	-416.976959228516\\
7.36499977111816	-416.826293945313\\
7.36999988555908	-416.255737304688\\
7.375	-415.306213378906\\
7.38000011444092	-413.933685302734\\
7.38500022888184	-412.099517822266\\
7.3899998664856	-409.768249511719\\
7.39499998092651	-406.922393798828\\
7.40000009536743	-403.492645263672\\
7.40500020980835	-399.617492675781\\
7.40999984741211	-395.362121582031\\
7.41499996185303	-390.740295410156\\
7.42000007629395	-385.791687011719\\
7.42500019073486	-380.569702148438\\
7.42999982833862	-375.118377685547\\
7.43499994277954	-369.48876953125\\
7.44000005722046	-363.677612304688\\
7.44500017166138	-357.693695068359\\
7.44999980926514	-351.572509765625\\
7.45499992370605	-345.366027832031\\
7.46000003814697	-339.102355957031\\
7.46500015258789	-332.757171630859\\
7.46999979019165	-326.295501708984\\
7.47499990463257	-319.818237304688\\
7.48000001907349	-313.431549072266\\
7.4850001335144	-307.104064941406\\
7.48999977111816	-300.8564453125\\
7.49499988555908	-294.722015380859\\
7.5	-288.718231201172\\
7.50500011444092	-282.857543945313\\
7.51000022888184	-277.154876708984\\
7.5149998664856	-271.624298095703\\
7.51999998092651	-266.290771484375\\
7.52500009536743	-261.057067871094\\
7.53000020980835	-256.653778076172\\
7.53499984741211	-252.141983032227\\
7.53999996185303	-247.920059204102\\
7.54500007629395	-244.046630859375\\
7.55000019073486	-240.405899047852\\
7.55499982833862	-237.030471801758\\
7.55999994277954	-233.972564697266\\
7.56500005722046	-231.213790893555\\
7.57000017166138	-228.745986938477\\
7.57499980926514	-226.595733642578\\
7.57999992370605	-224.773040771484\\
7.58500003814697	-223.268753051758\\
7.59000015258789	-222.107223510742\\
7.59499979019165	-221.270309448242\\
7.59999990463257	-220.748825073242\\
7.60500001907349	-220.544769287109\\
7.6100001335144	-220.688735961914\\
7.61499977111816	-221.208633422852\\
7.61999988555908	-222.108032226563\\
7.625	-223.314987182617\\
7.63000011444092	-224.662475585938\\
7.63500022888184	-226.103637695313\\
7.6399998664856	-227.779190063477\\
7.64499998092651	-229.659149169922\\
7.65000009536743	-231.694351196289\\
7.65500020980835	-233.906295776367\\
7.65999984741211	-236.315521240234\\
7.66499996185303	-238.881591796875\\
7.67000007629395	-241.582565307617\\
7.67500019073486	-244.425674438477\\
7.67999982833862	-247.411544799805\\
7.68499994277954	-250.511993408203\\
7.69000005722046	-253.705902099609\\
7.69500017166138	-256.920166015625\\
7.69999980926514	-260.175170898438\\
7.70499992370605	-263.461212158203\\
7.71000003814697	-266.779205322266\\
7.71500015258789	-270.158386230469\\
7.71999979019165	-273.487609863281\\
7.72499990463257	-276.736267089844\\
7.73000001907349	-279.945251464844\\
7.7350001335144	-283.098785400391\\
7.73999977111816	-286.292358398438\\
7.74499988555908	-289.378204345703\\
7.75	-292.299468994141\\
7.75500011444092	-295.185577392578\\
7.76000022888184	-298.026580810547\\
7.7649998664856	-300.729278564453\\
7.76999998092651	-303.299346923828\\
7.77500009536743	-305.751983642578\\
7.78000020980835	-308.089630126953\\
7.78499984741211	-310.333770751953\\
7.78999996185303	-312.435974121094\\
7.79500007629395	-314.408020019531\\
7.80000019073486	-316.233306884766\\
7.80499982833862	-317.927459716797\\
7.80999994277954	-319.490173339844\\
7.81500005722046	-320.914306640625\\
7.82000017166138	-322.202819824219\\
7.82499980926514	-323.359222412109\\
7.82999992370605	-324.387756347656\\
7.83500003814697	-325.293792724609\\
7.84000015258789	-326.072143554688\\
7.84499979019165	-326.724853515625\\
7.84999990463257	-327.260101318359\\
7.85500001907349	-327.689910888672\\
7.8600001335144	-328.017608642578\\
7.86499977111816	-328.260406494141\\
7.86999988555908	-328.4052734375\\
7.875	-328.458801269531\\
7.88000011444092	-328.428527832031\\
7.88500022888184	-328.330993652344\\
7.8899998664856	-328.177429199219\\
7.89499998092651	-327.975555419922\\
7.90000009536743	-327.729309082031\\
7.90500020980835	-327.441223144531\\
7.90999984741211	-327.117431640625\\
7.91499996185303	-326.758972167969\\
7.92000007629395	-326.373931884766\\
7.92500019073486	-325.967803955078\\
7.92999982833862	-325.564300537109\\
7.93499994277954	-325.169830322266\\
7.94000005722046	-324.782379150391\\
7.94500017166138	-324.427337646484\\
7.94999980926514	-324.119781494141\\
7.95499992370605	-323.859527587891\\
7.96000003814697	-323.654846191406\\
7.96500015258789	-323.536529541016\\
7.96999979019165	-323.517181396484\\
7.97499990463257	-323.61669921875\\
7.98000001907349	-323.8603515625\\
7.9850001335144	-324.180541992188\\
7.98999977111816	-324.590179443359\\
7.99499988555908	-325.087036132813\\
8	-325.716613769531\\
8.00500011444092	-326.467895507813\\
8.01000022888184	-327.343139648438\\
8.01500034332275	-328.370269775391\\
8.02000045776367	-329.558776855469\\
8.02499961853027	-330.933441162109\\
8.02999973297119	-332.480682373047\\
8.03499984741211	-334.184814453125\\
8.03999996185303	-336.055511474609\\
8.04500007629395	-338.072082519531\\
8.05000019073486	-340.258636474609\\
8.05500030517578	-342.611907958984\\
8.0600004196167	-345.123474121094\\
8.0649995803833	-347.802856445313\\
8.06999969482422	-350.648010253906\\
8.07499980926514	-353.639617919922\\
8.07999992370605	-356.781829833984\\
8.08500003814697	-360.231536865234\\
8.09000015258789	-363.857879638672\\
8.09500026702881	-367.6728515625\\
8.10000038146973	-371.665710449219\\
8.10499954223633	-375.816101074219\\
8.10999965667725	-380.129852294922\\
8.11499977111816	-384.640228271484\\
8.11999988555908	-389.348663330078\\
8.125	-394.264434814453\\
8.13000011444092	-399.355804443359\\
8.13500022888184	-404.626647949219\\
8.14000034332275	-410.054138183594\\
8.14500045776367	-415.689697265625\\
8.14999961853027	-421.516021728516\\
8.15499973297119	-427.557891845703\\
8.15999984741211	-433.798828125\\
8.16499996185303	-440.192596435547\\
8.17000007629395	-446.725250244141\\
8.17500019073486	-453.437377929688\\
8.18000030517578	-460.400939941406\\
8.1850004196167	-467.661834716797\\
8.1899995803833	-475.251403808594\\
8.19499969482422	-482.984558105469\\
8.19999980926514	-490.842224121094\\
8.20499992370605	-498.837371826172\\
8.21000003814697	-507.024536132813\\
8.21500015258789	-515.441528320313\\
8.22000026702881	-524.064758300781\\
8.22500038146973	-532.857727050781\\
8.22999954223633	-541.578369140625\\
8.23499965667725	-550.468566894531\\
8.23999977111816	-560.281127929688\\
8.24499988555908	-570.838134765625\\
8.25	-581.152404785156\\
8.25500011444092	-591.464538574219\\
8.26000022888184	-602.078186035156\\
8.26500034332275	-612.972839355469\\
8.27000045776367	-623.948120117188\\
8.27499961853027	-635.630004882813\\
8.27999973297119	-647.640197753906\\
8.28499984741211	-659.934936523438\\
8.28999996185303	-672.278381347656\\
8.29500007629395	-685.192443847656\\
8.30000019073486	-698.587646484375\\
8.30500030517578	-712.415100097656\\
8.3100004196167	-726.283447265625\\
8.3149995803833	-737.961120605469\\
8.31999969482422	-755.498168945313\\
8.32499980926514	-771.337158203125\\
8.32999992370605	-787.272094726563\\
8.33500003814697	-803.495910644531\\
8.34000015258789	-820.448913574219\\
8.34500026702881	-838.067138671875\\
8.35000038146973	-855.93603515625\\
8.35499954223633	-874.895446777344\\
8.35999965667725	-892.661926269531\\
8.36499977111816	-912.446350097656\\
8.36999988555908	-934.093566894531\\
8.375	-955.406494140625\\
8.38000011444092	-976.940246582031\\
8.38500022888184	-999.809326171875\\
8.39000034332275	-1022.99798583984\\
8.39500045776367	-1047.33276367188\\
8.39999961853027	-1069.3359375\\
8.40499973297119	-1096.9462890625\\
8.40999984741211	-1123.58581542969\\
8.41499996185303	-1150.40283203125\\
8.42000007629395	-1178.16284179688\\
8.42500019073486	-1206.25354003906\\
8.43000030517578	-1235.10620117188\\
8.4350004196167	-1263.45190429688\\
8.4399995803833	-1291.46899414063\\
8.44499969482422	-1320.98657226563\\
8.44999980926514	-1348.80322265625\\
8.45499992370605	-1375.29479980469\\
8.46000003814697	-1401.46020507813\\
8.46500015258789	-1424.25427246094\\
8.47000026702881	-1446.08020019531\\
8.47500038146973	-1465.77380371094\\
8.47999954223633	-1483.52185058594\\
8.48499965667725	-1500.18762207031\\
8.48999977111816	-1515.10217285156\\
8.49499988555908	-1525.92797851563\\
8.5	-1536.34191894531\\
8.50500011444092	-1543.94360351563\\
8.51000022888184	-1550.47375488281\\
8.51500034332275	-1554.20495605469\\
8.52000045776367	-1556.38903808594\\
8.52499961853027	-1562.08654785156\\
8.52999973297119	-1559.30346679688\\
8.53499984741211	-1556.83251953125\\
8.53999996185303	-1554.74816894531\\
8.54500007629395	-1550.95776367188\\
8.55000019073486	-1545.0400390625\\
8.55500030517578	-1538.51013183594\\
8.5600004196167	-1531.421875\\
8.5649995803833	-1523.11486816406\\
8.56999969482422	-1514.51403808594\\
8.57499980926514	-1506.36096191406\\
8.57999992370605	-1498.08740234375\\
8.58500003814697	-1489.76684570313\\
8.59000015258789	-1480.94006347656\\
8.59500026702881	-1471.52612304688\\
8.60000038146973	-1462.05895996094\\
8.60499954223633	-1453.34411621094\\
8.60999965667725	-1444.7275390625\\
8.61499977111816	-1436.97082519531\\
8.61999988555908	-1430.26721191406\\
8.625	-1424.98474121094\\
8.63000011444092	-1417.50256347656\\
8.63500022888184	-1409.69287109375\\
8.64000034332275	-1401.86340332031\\
8.64500045776367	-1391.86437988281\\
8.64999961853027	-1381.27453613281\\
8.65499973297119	-1365.74853515625\\
8.65999984741211	-1352.6865234375\\
8.66499996185303	-1331.76220703125\\
8.67000007629395	-1311.15393066406\\
8.67500019073486	-1287.81872558594\\
8.68000030517578	-1264.35107421875\\
8.6850004196167	-1239.59924316406\\
8.6899995803833	-1214.17822265625\\
8.69499969482422	-1188.64465332031\\
8.69999980926514	-1164.74426269531\\
8.70499992370605	-1141.47094726563\\
8.71000003814697	-1118.58032226563\\
8.71500015258789	-1096.02392578125\\
8.72000026702881	-1075.18115234375\\
8.72500038146973	-1053.91711425781\\
8.72999954223633	-1032.39916992188\\
8.73499965667725	-1010.95098876953\\
8.73999977111816	-989.057861328125\\
8.74499988555908	-966.133728027344\\
8.75	-942.285400390625\\
8.75500011444092	-916.634216308594\\
8.76000022888184	-892.757629394531\\
8.76500034332275	-867.576782226563\\
8.77000045776367	-843.312194824219\\
8.77499961853027	-819.685791015625\\
8.77999973297119	-797.34765625\\
8.78499984741211	-776.141845703125\\
8.78999996185303	-756.089599609375\\
8.79500007629395	-737.437255859375\\
8.80000019073486	-720.106567382813\\
8.80500030517578	-703.961303710938\\
8.8100004196167	-688.843505859375\\
8.8149995803833	-674.617797851563\\
8.81999969482422	-660.946350097656\\
8.82499980926514	-647.763854980469\\
8.82999992370605	-634.9765625\\
8.83500003814697	-622.42919921875\\
8.84000015258789	-610.23291015625\\
8.84500026702881	-598.568481445313\\
8.85000038146973	-587.405822753906\\
8.85499954223633	-577.259704589844\\
8.85999965667725	-568.070983886719\\
8.86499977111816	-560.044677734375\\
8.86999988555908	-553.2509765625\\
8.875	-547.731140136719\\
8.88000011444092	-543.422119140625\\
8.88500022888184	-540.191589355469\\
8.89000034332275	-537.856018066406\\
8.89500045776367	-536.168762207031\\
8.89999961853027	-535.197570800781\\
8.90499973297119	-534.560302734375\\
8.90999984741211	-533.8798828125\\
8.91499996185303	-532.873107910156\\
8.92000007629395	-532.757141113281\\
8.92500019073486	-532.9892578125\\
8.93000030517578	-533.375854492188\\
8.9350004196167	-534.369812011719\\
8.9399995803833	-535.906921386719\\
8.94499969482422	-538.125305175781\\
8.94999980926514	-540.957885742188\\
8.95499992370605	-544.22021484375\\
8.96000003814697	-547.988220214844\\
8.96500015258789	-552.221923828125\\
8.97000026702881	-556.833618164063\\
8.97500038146973	-561.484313964844\\
8.97999954223633	-566.15478515625\\
8.98499965667725	-571.34228515625\\
8.98999977111816	-575.957458496094\\
8.99499988555908	-580.266357421875\\
9	-585.265441894531\\
9.00500011444092	-590.015258789063\\
9.01000022888184	-594.620300292969\\
9.01500034332275	-599.212219238281\\
9.02000045776367	-603.846862792969\\
9.02499961853027	-608.401000976563\\
9.02999973297119	-612.857421875\\
9.03499984741211	-617.233642578125\\
9.03999996185303	-621.47900390625\\
9.04500007629395	-625.643188476563\\
9.05000019073486	-629.437194824219\\
9.05500030517578	-633.03662109375\\
9.0600004196167	-636.372741699219\\
9.0649995803833	-639.434448242188\\
9.06999969482422	-642.159484863281\\
9.07499980926514	-644.563781738281\\
9.07999992370605	-646.639221191406\\
9.08500003814697	-648.379089355469\\
9.09000015258789	-649.734558105469\\
9.09500026702881	-650.6181640625\\
9.10000038146973	-651.136840820313\\
9.10499954223633	-651.569030761719\\
9.10999965667725	-651.627136230469\\
9.11499977111816	-651.219055175781\\
9.11999988555908	-650.470825195313\\
9.125	-649.547485351563\\
9.13000011444092	-648.452453613281\\
9.13500022888184	-646.992553710938\\
9.14000034332275	-645.316528320313\\
9.14500045776367	-643.307495117188\\
9.14999961853027	-640.976989746094\\
9.15499973297119	-638.338256835938\\
9.15999984741211	-635.504699707031\\
9.16499996185303	-632.263244628906\\
9.17000007629395	-628.625549316406\\
9.17500019073486	-624.974365234375\\
9.18000030517578	-621.097595214844\\
9.1850004196167	-617.108032226563\\
9.1899995803833	-613.080017089844\\
9.19499969482422	-609.02001953125\\
9.19999980926514	-604.967102050781\\
9.20499992370605	-600.964172363281\\
9.21000003814697	-597.088989257813\\
9.21500015258789	-593.461608886719\\
9.22000026702881	-589.783874511719\\
9.22500038146973	-586.203430175781\\
9.22999954223633	-582.682373046875\\
9.23499965667725	-579.041198730469\\
9.23999977111816	-575.122619628906\\
9.24499988555908	-571.043518066406\\
9.25	-569.171081542969\\
9.25500011444092	-570.854431152344\\
9.26000022888184	-575.684936523438\\
9.26500034332275	-581.841674804688\\
9.27000045776367	-592.614013671875\\
9.27499961853027	-605.410522460938\\
9.27999973297119	-620.606994628906\\
9.28499984741211	-638.875305175781\\
9.28999996185303	-659.783325195313\\
9.29500007629395	-683.900085449219\\
9.30000019073486	-711.811950683594\\
9.30500030517578	-738.860717773438\\
9.3100004196167	-777.628845214844\\
9.3149995803833	-819.566711425781\\
9.31999969482422	-864.494812011719\\
9.32499980926514	-917.715270996094\\
9.32999992370605	-973.913879394531\\
9.33500003814697	-1040.61474609375\\
9.34000015258789	-1114.2919921875\\
9.34500026702881	-1188.03869628906\\
9.35000038146973	-1282.55517578125\\
9.35499954223633	-1382.11083984375\\
9.35999965667725	-1478.95593261719\\
9.36499977111816	-1589.02270507813\\
9.36999988555908	-1700.57836914063\\
9.375	-1809.12658691406\\
9.38000011444092	-1916.14465332031\\
9.38500022888184	-2018.87561035156\\
9.39000034332275	-2117.21752929688\\
9.39500045776367	-2210.916015625\\
9.39999961853027	-2303.61962890625\\
9.40499973297119	-2389.53662109375\\
9.40999984741211	-2469.75268554688\\
9.41499996185303	-2543.4306640625\\
9.42000007629395	-2607.6728515625\\
9.42500019073486	-2657.59448242188\\
9.43000030517578	-2681.12255859375\\
9.4350004196167	-2712.9091796875\\
9.4399995803833	-2730.09814453125\\
9.44499969482422	-2719.3369140625\\
9.44999980926514	-2679.015625\\
9.45499992370605	-2596.66943359375\\
9.46000003814697	-2476.69799804688\\
9.46500015258789	-2297.24951171875\\
9.47000026702881	-2064.53002929688\\
9.47500038146973	-1736.98059082031\\
9.47999954223633	-1336.37866210938\\
9.48499965667725	-669.354614257813\\
9.48999977111816	87.3506546020508\\
9.49499988555908	112.511848449707\\
9.5	74.9907913208008\\
9.50500011444092	74.6545715332031\\
9.51000022888184	91.7128372192383\\
9.51500034332275	116.236869812012\\
9.52000045776367	147.286697387695\\
9.52499961853027	172.096939086914\\
9.52999973297119	212.776809692383\\
9.53499984741211	239.808319091797\\
9.53999996185303	259.416778564453\\
9.54500007629395	273.845336914063\\
9.55000019073486	274.397735595703\\
9.55500030517578	107.572380065918\\
9.5600004196167	58.4787063598633\\
9.5649995803833	37.3371772766113\\
9.56999969482422	22.9610118865967\\
9.57499980926514	13.6799554824829\\
9.57999992370605	8.55732822418213\\
9.58500003814697	5.79377365112305\\
9.59000015258789	3.91836738586426\\
9.59500026702881	3.45720720291138\\
9.60000038146973	3.07975339889526\\
9.60499954223633	3.25853681564331\\
9.60999965667725	2.80006289482117\\
9.61499977111816	3.00728845596313\\
9.61999988555908	3.04014873504639\\
9.625	2.97157287597656\\
9.63000011444092	2.85795831680298\\
9.63500022888184	2.75514554977417\\
9.64000034332275	2.59522724151611\\
9.64500045776367	2.438716173172\\
9.64999961853027	2.34074187278748\\
9.65499973297119	2.20446372032166\\
9.65999984741211	2.04168820381165\\
9.66499996185303	1.91774892807007\\
9.67000007629395	1.7903243303299\\
9.67500019073486	1.65970861911774\\
9.68000030517578	1.54140794277191\\
9.6850004196167	1.42662286758423\\
9.6899995803833	1.31881785392761\\
9.69499969482422	1.21575403213501\\
9.69999980926514	1.1179301738739\\
9.70499992370605	1.02433884143829\\
9.71000003814697	0.941645979881287\\
9.71500015258789	0.866914451122284\\
9.72000026702881	0.794409215450287\\
9.72500038146973	0.729629456996918\\
9.72999954223633	0.669434130191803\\
9.73499965667725	0.610916256904602\\
9.73999977111816	0.554586589336395\\
9.74499988555908	0.49921378493309\\
9.75	0.465546697378159\\
9.75500011444092	0.433034747838974\\
9.76000022888184	0.405597001314163\\
9.76500034332275	0.373790681362152\\
9.77000045776367	0.334637582302094\\
9.77499961853027	0.303972661495209\\
9.77999973297119	0.275028556585312\\
9.78499984741211	0.257883578538895\\
9.78999996185303	0.235431402921677\\
9.79500007629395	0.21394194662571\\
9.80000019073486	0.194295689463615\\
9.80500030517578	0.177395835518837\\
9.8100004196167	0.164111196994781\\
9.8149995803833	0.153031453490257\\
9.81999969482422	0.139377251267433\\
9.82499980926514	0.127098992466927\\
9.82999992370605	0.116103522479534\\
9.83500003814697	0.108244255185127\\
9.84000015258789	0.103384114801884\\
9.84500026702881	0.0982505679130554\\
9.85000038146973	0.0933027490973473\\
9.85499954223633	0.0772441849112511\\
9.85999965667725	0.0683412179350853\\
9.86499977111816	0.0609700158238411\\
9.86999988555908	0.059698324650526\\
9.875	0.0618107430636883\\
9.88000011444092	0.0598164051771164\\
9.88500022888184	0.0572427064180374\\
9.89000034332275	0.0447159856557846\\
9.89500045776367	0.0393539853394032\\
9.89999961853027	0.0349498614668846\\
9.90499973297119	0.0381641313433647\\
9.90999984741211	0.0456239953637123\\
9.91499996185303	0.0484268218278885\\
9.92000007629395	0.0506765134632587\\
9.92500019073486	0.0321044474840164\\
9.93000030517578	0.0249314978718758\\
9.9350004196167	0.0164546314626932\\
9.9399995803833	0.0203482341021299\\
9.94499969482422	0.0268654525279999\\
9.94999980926514	0.0246825125068426\\
9.95499992370605	0.0226905643939972\\
9.96000003814697	0.0215073544532061\\
9.96500015258789	0.0205165036022663\\
9.97000026702881	0.0196048114448786\\
9.97500038146973	0.0183063093572855\\
9.97999954223633	0.0190571267157793\\
9.98499965667725	0.0208779778331518\\
9.98999977111816	0.0227325204759836\\
9.99499988555908	0.0243946835398674\\
10	0.025297237560153\\
};
\addlegendentry{CF}

\end{axis}
\end{tikzpicture}%
    \end{tikzpicture}}
    \caption{Loads at point C of CF under PD Feedback Control}
    \label{fig:pureFeedbkPDC}
\end{figure}


\subsection*{PID Control}
Next, a PID Controller is tried with small integral action and the same proportional and derivative quantities. This controller doesn't lead to very significant modifications in the performance of the PD Controller, which might be expected since the integral action activates on the basis of accumulation of error and for a small intergral gain value, the contribution from the integral action has quite a small effect. To observe some sort of changes in the performance, so as to get a better idea of whether the integral action is at all helpful, the PID Controller for the angular position of the bike is modified to:
$$K_z = 10 + 100s + \frac{1000}{s}$$

\begin{table}[h!]
	\centering
	\begin{tabular}{ |c|c|c|c| } 
		\hline
		Forces & Mean Error (\%) & RMSE & $R_2$\\ 
		\hline
		FX & 2&114&0.99\\ 
		FY & 56&488&0.5 \\ 
		\hline
	\end{tabular}
	\caption{Error Tabulation of Loads at Point A of CF under PID Feedback Control with modified Z-Feedback}
	\label{tab:pureFeedbkPIDA}
\end{table}

\begin{table}[h!]
	\centering
	\begin{tabular}{ |c|c|c|c| } 
		\hline
		Forces & Mean Error (\%) & RMSE & $R_2$\\ 
		\hline
		FX & 3&14&0.99\\
		FY&3&2&0.99\\
		\hline
	\end{tabular}
	\caption{Error Tabulation of Loads at Point B of CF under PID Feedback Control with modified Z-Feedback}
	\label{tab:pureFeedbkPIDB}
\end{table}

\begin{table}[h!]
	\centering
	\begin{tabular}{ |c|c|c|c| } 
		\hline
		Forces & Mean Error (\%) & RMSE & $R_2$\\ 
		\hline
		FX & 3&101&0.98\\ 
		FY & 35&516&0.52 \\ 
		\hline
	\end{tabular}
	\caption{Error Tabulation of Loads at Point C of CF under PID Feedback Control with modified Z-Feedback}
	\label{tab:pureFeedbkPIDC}
\end{table}


It is quite clear from Tables \ref{tab:pureFeedbkPIDA} to \ref{tab:pureFeedbkPIDC} that the Y force components correlation at A and C improves by quite a lot, while the X force components and the forces at B have not very large changes. 

The correlation between the control effort and the imbalances estimated might help in understanding the control requirements. It can be seen in Figure \ref{fig:imbVsCont} that the control effort in X and Y resemble the corresponding imbalances quite a lot. 
\begin{figure}[h!]
	\centering
	\scalebox{1}{
		\begin{tikzpicture}
			% This file was created by matlab2tikz.
%
%The latest updates can be retrieved from
%  http://www.mathworks.com/matlabcentral/fileexchange/22022-matlab2tikz-matlab2tikz
%where you can also make suggestions and rate matlab2tikz.
%
\begin{tikzpicture}

\begin{axis}[%
width=4.521in,
height=0.823in,
at={(0.758in,3.224in)},
scale only axis,
xmin=0,
xmax=10,
xlabel style={font=\color{white!15!black}},
xlabel={Time (s)},
ymin=-500,
ymax=500,
ylabel style={font=\color{white!15!black}},
ylabel={FX (N)},
axis background/.style={fill=white},
axis x line*=bottom,
axis y line*=left,
xmajorgrids,
ymajorgrids,
legend style={at={(0.35,1)}, anchor=north east, legend cell align=left, align=left, draw=black}
]
\addplot [color=red, line width=1.5pt]
  table[row sep=crcr]{%
0.0949999988079071	-3.81241369247437\\
0.100000001490116	-3.3780517578125\\
0.104999996721745	-3.02275562286377\\
0.109999999403954	-2.74130487442017\\
0.115000002086163	-2.50688576698303\\
0.119999997317791	-3.50171184539795\\
0.125	-7.56024789810181\\
0.129999995231628	-38.5924911499023\\
0.135000005364418	-65.3207550048828\\
0.140000000596046	-83.6230239868164\\
0.144999995827675	-94.6410140991211\\
0.150000005960464	-100.119064331055\\
0.155000001192093	-101.285369873047\\
0.159999996423721	-99.871826171875\\
0.165000006556511	-96.994384765625\\
0.170000001788139	-93.6680068969727\\
0.174999997019768	-89.7039794921875\\
0.180000007152557	-84.6482009887695\\
0.185000002384186	-78.6278381347656\\
0.189999997615814	-71.7985382080078\\
0.194999992847443	-58.2733497619629\\
0.200000002980232	-6.2598443031311\\
0.204999998211861	96.4931793212891\\
0.209999993443489	196.028198242188\\
0.215000003576279	260.831115722656\\
0.219999998807907	277.715972900391\\
0.224999994039536	255.091232299805\\
0.230000004172325	203.554794311523\\
0.234999999403954	133.345275878906\\
0.239999994635582	56.1056747436523\\
0.245000004768372	-16.0835437774658\\
0.25	-73.2415390014648\\
0.254999995231628	-108.291145324707\\
0.259999990463257	-117.419319152832\\
0.264999985694885	-102.604789733887\\
0.270000010728836	-71.0218811035156\\
0.275000005960464	-29.2378940582275\\
0.280000001192093	15.2165021896362\\
0.284999996423721	54.9710121154785\\
0.28999999165535	85.5328826904297\\
0.294999986886978	103.375267028809\\
0.300000011920929	108.224716186523\\
0.305000007152557	102.504570007324\\
0.310000002384186	89.3121185302734\\
0.314999997615814	70.9584579467773\\
0.319999992847443	50.1598510742188\\
0.324999988079071	29.8966255187988\\
0.330000013113022	12.6254892349243\\
0.33500000834465	0.042132280766964\\
0.340000003576279	-6.71820640563965\\
0.344999998807907	-6.37971591949463\\
0.349999994039536	0.863392174243927\\
0.354999989271164	12.470178604126\\
0.360000014305115	25.5600318908691\\
0.365000009536743	37.6813468933105\\
0.370000004768372	47.0138893127441\\
0.375	52.9364852905273\\
0.379999995231628	54.7546424865723\\
0.384999990463257	52.1401443481445\\
0.389999985694885	45.4394454956055\\
0.395000010728836	35.4722862243652\\
0.400000005960464	23.8612384796143\\
0.405000001192093	12.2208433151245\\
0.409999996423721	2.16164875030518\\
0.41499999165535	-4.90084838867188\\
0.419999986886978	-7.97381353378296\\
0.425000011920929	-7.34672021865845\\
0.430000007152557	-3.14297533035278\\
0.435000002384186	3.54557037353516\\
0.439999997615814	10.9576606750488\\
0.444999992847443	17.8444881439209\\
0.449999988079071	23.0575122833252\\
0.455000013113022	25.6603107452393\\
0.46000000834465	25.2681179046631\\
0.465000003576279	21.9174976348877\\
0.469999998807907	16.2951965332031\\
0.474999994039536	9.21124458312988\\
0.479999989271164	1.90070307254791\\
0.485000014305115	-4.52938556671143\\
0.490000009536743	-9.26993656158447\\
0.495000004768372	-11.2350044250488\\
0.5	-10.4693632125854\\
0.504999995231628	-7.23577165603638\\
0.509999990463257	-2.41954445838928\\
0.514999985694885	2.84694600105286\\
0.519999980926514	7.55986738204956\\
0.524999976158142	11.0842456817627\\
0.529999971389771	12.8232269287109\\
0.535000026226044	12.7868032455444\\
0.540000021457672	11.4206886291504\\
0.545000016689301	9.07893466949463\\
0.550000011920929	6.12254619598389\\
0.555000007152557	3.14145874977112\\
0.560000002384186	0.750715732574463\\
0.564999997615814	-0.482109814882278\\
0.569999992847443	-0.670082688331604\\
0.574999988079071	-0.036297682672739\\
0.579999983310699	1.37397313117981\\
0.584999978542328	3.48725152015686\\
0.589999973773956	5.37979936599731\\
0.595000028610229	7.25802373886108\\
0.600000023841858	8.70273685455322\\
0.605000019073486	9.74080276489258\\
0.610000014305115	10.4353437423706\\
0.615000009536743	10.6562080383301\\
0.620000004768372	10.5354585647583\\
0.625	10.3141899108887\\
0.629999995231628	10.0344800949097\\
0.634999990463257	9.80208492279053\\
0.639999985694885	9.72895812988281\\
0.644999980926514	9.87786769866943\\
0.649999976158142	10.2593383789063\\
0.654999971389771	10.9108896255493\\
0.660000026226044	11.6664819717407\\
0.665000021457672	12.4790077209473\\
0.670000016689301	13.373272895813\\
0.675000011920929	14.1844730377197\\
0.680000007152557	14.9620265960693\\
0.685000002384186	15.5990104675293\\
0.689999997615814	16.1367855072021\\
0.694999992847443	16.547513961792\\
0.699999988079071	16.8790473937988\\
0.704999983310699	17.1442604064941\\
0.709999978542328	17.3799724578857\\
0.714999973773956	17.6072311401367\\
0.720000028610229	17.8552627563477\\
0.725000023841858	18.1153450012207\\
0.730000019073486	18.4143886566162\\
0.735000014305115	18.7048244476318\\
0.740000009536743	19.015079498291\\
0.745000004768372	19.2968463897705\\
0.75	19.5646934509277\\
0.754999995231628	19.7840042114258\\
0.759999990463257	19.9632205963135\\
0.764999985694885	20.0693054199219\\
0.769999980926514	20.1333503723145\\
0.774999976158142	20.1316337585449\\
0.779999971389771	20.1005153656006\\
0.785000026226044	20.0411739349365\\
0.790000021457672	19.967716217041\\
0.795000016689301	19.8572864532471\\
0.800000011920929	19.7295360565186\\
0.805000007152557	19.5648574829102\\
0.810000002384186	19.4038696289063\\
0.814999997615814	19.2040424346924\\
0.819999992847443	18.9689235687256\\
0.824999988079071	18.7667045593262\\
0.829999983310699	18.5256423950195\\
0.834999978542328	18.2512435913086\\
0.839999973773956	18.0028057098389\\
0.845000028610229	17.7073440551758\\
0.850000023841858	17.4330081939697\\
0.855000019073486	17.1099529266357\\
0.860000014305115	16.7736644744873\\
0.865000009536743	16.4240379333496\\
0.870000004768372	16.0675182342529\\
0.875	15.7139673233032\\
0.879999995231628	15.3692693710327\\
0.884999990463257	15.0381193161011\\
0.889999985694885	14.7214317321777\\
0.894999980926514	14.417441368103\\
0.899999976158142	14.1262302398682\\
0.904999971389771	13.8272285461426\\
0.910000026226044	13.5744848251343\\
0.915000021457672	13.3321361541748\\
0.920000016689301	13.0918483734131\\
0.925000011920929	12.8711290359497\\
0.930000007152557	12.6495389938354\\
0.935000002384186	12.4443655014038\\
0.939999997615814	12.2345571517944\\
0.944999992847443	12.0371122360229\\
0.949999988079071	11.8387241363525\\
0.954999983310699	11.660267829895\\
0.959999978542328	11.4925193786621\\
0.964999973773956	11.3509664535522\\
0.970000028610229	11.2299461364746\\
0.975000023841858	11.1373691558838\\
0.980000019073486	11.0689249038696\\
0.985000014305115	11.0231170654297\\
0.990000009536743	11.0002870559692\\
0.995000004768372	10.9917831420898\\
1	10.999587059021\\
1.00499999523163	11.011682510376\\
1.00999999046326	11.0328140258789\\
1.01499998569489	11.0622825622559\\
1.01999998092651	11.0965518951416\\
1.02499997615814	11.1369333267212\\
1.02999997138977	11.1891765594482\\
1.0349999666214	11.2558336257935\\
1.03999996185303	11.334056854248\\
1.04499995708466	11.4226579666138\\
1.04999995231628	11.5202493667603\\
1.05499994754791	11.6263341903687\\
1.05999994277954	11.7405338287354\\
1.06500005722046	11.8620529174805\\
1.07000005245209	11.9914493560791\\
1.07500004768372	12.1282176971436\\
1.08000004291534	12.271430015564\\
1.08500003814697	12.4192943572998\\
1.0900000333786	12.570424079895\\
1.09500002861023	12.7236051559448\\
1.10000002384186	12.8774452209473\\
1.10500001907349	13.0306100845337\\
1.11000001430511	13.1824970245361\\
1.11500000953674	13.3334951400757\\
1.12000000476837	13.4830265045166\\
1.125	13.6303882598877\\
1.12999999523163	13.7762107849121\\
1.13499999046326	13.920844078064\\
1.13999998569489	14.0639009475708\\
1.14499998092651	14.2043209075928\\
1.14999997615814	14.3421058654785\\
1.15499997138977	14.4769010543823\\
1.1599999666214	14.6074266433716\\
1.16499996185303	14.7328329086304\\
1.16999995708466	14.8527040481567\\
1.17499995231628	14.9663410186768\\
1.17999994754791	15.072850227356\\
1.18499994277954	15.1733379364014\\
1.19000005722046	15.267951965332\\
1.19500005245209	15.3557395935059\\
1.20000004768372	15.4351778030396\\
1.20500004291534	15.5050468444824\\
1.21000003814697	15.5641412734985\\
1.2150000333786	15.6122856140137\\
1.22000002861023	15.6509103775024\\
1.22500002384186	15.6808338165283\\
1.23000001907349	15.7022848129272\\
1.23500001430511	15.7153749465942\\
1.24000000953674	15.7209901809692\\
1.24500000476837	15.7201128005981\\
1.25	15.7162742614746\\
1.25499999523163	15.7070922851563\\
1.25999999046326	15.6920776367188\\
1.26499998569489	15.6709280014038\\
1.26999998092651	15.6440162658691\\
1.27499997615814	15.6132669448853\\
1.27999997138977	15.5782079696655\\
1.2849999666214	15.535852432251\\
1.28999996185303	15.485050201416\\
1.29499995708466	15.4255332946777\\
1.29999995231628	15.3570289611816\\
1.30499994754791	15.2808399200439\\
1.30999994277954	15.1976528167725\\
1.31500005722046	15.109278678894\\
1.32000005245209	15.0180320739746\\
1.32500004768372	14.9256267547607\\
1.33000004291534	14.8337678909302\\
1.33500003814697	14.743426322937\\
1.3400000333786	14.6554250717163\\
1.34500002861023	14.5708160400391\\
1.35000002384186	14.4905681610107\\
1.35500001907349	14.4155941009521\\
1.36000001430511	14.3461589813232\\
1.36500000953674	14.2800903320313\\
1.37000000476837	14.2132883071899\\
1.375	14.1420679092407\\
1.37999999523163	14.0658693313599\\
1.38499999046326	13.9847087860107\\
1.38999998569489	13.8978404998779\\
1.39499998092651	13.8086929321289\\
1.39999997615814	13.7229032516479\\
1.40499997138977	13.6432418823242\\
1.4099999666214	13.5712604522705\\
1.41499996185303	13.5081214904785\\
1.41999995708466	13.4555330276489\\
1.42499995231628	13.4134111404419\\
1.42999994754791	13.379789352417\\
1.43499994277954	13.3536586761475\\
1.44000005722046	13.3341836929321\\
1.44500005245209	13.3198833465576\\
1.45000004768372	13.3096017837524\\
1.45500004291534	13.3021030426025\\
1.46000003814697	13.2955713272095\\
1.4650000333786	13.287974357605\\
1.47000002861023	13.2770128250122\\
1.47500002384186	13.2603921890259\\
1.48000001907349	13.2397289276123\\
1.48500001430511	13.2207126617432\\
1.49000000953674	13.210916519165\\
1.49500000476837	13.211256980896\\
1.5	13.2197360992432\\
1.50499999523163	13.2330617904663\\
1.50999999046326	13.2466201782227\\
1.51499998569489	13.2540473937988\\
1.51999998092651	13.2567434310913\\
1.52499997615814	13.2614212036133\\
1.52999997138977	13.2731523513794\\
1.5349999666214	13.29075050354\\
1.53999996185303	13.3100490570068\\
1.54499995708466	13.3304109573364\\
1.54999995231628	13.3530359268188\\
1.55499994754791	13.3791818618774\\
1.55999994277954	13.4092979431152\\
1.56500005722046	13.443455696106\\
1.57000005245209	13.8861351013184\\
1.57500004768372	14.7198295593262\\
1.58000004291534	15.1066484451294\\
1.58500003814697	14.9925508499146\\
1.5900000333786	14.6452398300171\\
1.59500002861023	14.251368522644\\
1.60000002384186	13.890772819519\\
1.60500001907349	13.604736328125\\
1.61000001430511	13.4154253005981\\
1.61500000953674	13.3384590148926\\
1.62000000476837	13.3850955963135\\
1.625	13.5351629257202\\
1.62999999523163	13.7701902389526\\
1.63499999046326	14.0430593490601\\
1.63999998569489	14.2903661727905\\
1.64499998092651	14.4744319915771\\
1.64999997615814	14.5715522766113\\
1.65499997138977	14.6099033355713\\
1.6599999666214	14.6125497817993\\
1.66499996185303	14.5828876495361\\
1.66999995708466	14.528769493103\\
1.67499995231628	14.4576807022095\\
1.67999994754791	14.3770961761475\\
1.68499994277954	14.3051462173462\\
1.69000005722046	14.2633209228516\\
1.69500005245209	14.2493963241577\\
1.70000004768372	14.2541933059692\\
1.70500004291534	14.2731294631958\\
1.71000003814697	14.298828125\\
1.7150000333786	14.3223009109497\\
1.72000002861023	14.3431167602539\\
1.72500002384186	14.3617496490479\\
1.73000001907349	14.3770380020142\\
1.73500001430511	14.3872976303101\\
1.74000000953674	14.3908262252808\\
1.74500000476837	14.3867616653442\\
1.75	14.3748140335083\\
1.75499999523163	14.3557214736938\\
1.75999999046326	14.330002784729\\
1.76499998569489	14.2988510131836\\
1.76999998092651	14.2632694244385\\
1.77499997615814	14.2248735427856\\
1.77999997138977	14.1864471435547\\
1.7849999666214	14.1498394012451\\
1.78999996185303	14.115909576416\\
1.79499995708466	14.085259437561\\
1.79999995231628	14.0580081939697\\
1.80499994754791	14.034158706665\\
1.80999994277954	14.0131559371948\\
1.81500005722046	13.9937963485718\\
1.82000005245209	13.9749765396118\\
1.82500004768372	13.9555587768555\\
1.83000004291534	13.9343843460083\\
1.83500003814697	13.9104127883911\\
1.8400000333786	13.8833751678467\\
1.84500002861023	13.8532466888428\\
1.85000002384186	13.8208045959473\\
1.85500001907349	13.7879447937012\\
1.86000001430511	13.7554063796997\\
1.86500000953674	13.7235269546509\\
1.87000000476837	13.6927585601807\\
1.875	13.6633377075195\\
1.87999999523163	13.6366300582886\\
1.88499999046326	13.6145839691162\\
1.88999998569489	13.597749710083\\
1.89499998092651	13.584662437439\\
1.89999997615814	13.5722160339355\\
1.90499997138977	13.5597267150879\\
1.9099999666214	13.5468769073486\\
1.91499996185303	13.5334529876709\\
1.91999995708466	13.5195112228394\\
1.92499995231628	13.505274772644\\
1.92999994754791	13.4919309616089\\
1.93499994277954	13.4803113937378\\
1.94000005722046	13.4707412719727\\
1.94500005245209	13.4630699157715\\
1.95000004768372	13.456974029541\\
1.95500004291534	13.4528360366821\\
1.96000003814697	13.452564239502\\
1.9650000333786	13.4590034484863\\
1.97000002861023	13.4733734130859\\
1.97500002384186	13.4926738739014\\
1.98000001907349	13.5098867416382\\
1.98500001430511	13.5209703445435\\
1.99000000953674	13.5252208709717\\
1.99500000476837	13.5245723724365\\
2	13.519458770752\\
2.00500011444092	13.5118236541748\\
2.00999999046326	13.5069904327393\\
2.01500010490417	13.5057277679443\\
2.01999998092651	13.5135145187378\\
2.02500009536743	13.5385103225708\\
2.02999997138977	13.5874004364014\\
2.03500008583069	13.6598806381226\\
2.03999996185303	13.7452487945557\\
2.04500007629395	13.8305835723877\\
2.04999995231628	13.9084148406982\\
2.0550000667572	13.9751281738281\\
2.05999994277954	14.0292367935181\\
2.06500005722046	14.0675277709961\\
2.0699999332428	14.0843257904053\\
2.07500004768372	14.0725259780884\\
2.07999992370605	14.0268297195435\\
2.08500003814697	13.9440050125122\\
2.08999991416931	13.8173913955688\\
2.09500002861023	13.6604290008545\\
2.09999990463257	13.5065279006958\\
2.10500001907349	13.3133430480957\\
2.10999989509583	13.1220102310181\\
2.11500000953674	12.9476919174194\\
2.11999988555908	12.7590341567993\\
2.125	12.5843267440796\\
2.13000011444092	12.4543600082397\\
2.13499999046326	12.4037103652954\\
2.14000010490417	12.4419393539429\\
2.14499998092651	12.5427360534668\\
2.15000009536743	12.6384515762329\\
2.15499997138977	12.6968545913696\\
2.16000008583069	12.7527322769165\\
2.16499996185303	12.7960367202759\\
2.17000007629395	12.7496299743652\\
2.17499995231628	12.6484184265137\\
2.1800000667572	12.5259275436401\\
2.18499994277954	12.3784189224243\\
2.19000005722046	12.2102766036987\\
2.1949999332428	12.0463438034058\\
2.20000004768372	11.9420347213745\\
2.20499992370605	11.9290018081665\\
2.21000003814697	11.9936485290527\\
2.21499991416931	12.1057901382446\\
2.22000002861023	12.2569599151611\\
2.22499990463257	12.4615621566772\\
2.23000001907349	12.6845054626465\\
2.23499989509583	12.8918085098267\\
2.24000000953674	13.0590534210205\\
2.24499988555908	13.1692562103271\\
2.25	13.2286205291748\\
2.25500011444092	13.2428960800171\\
2.25999999046326	13.216552734375\\
2.26500010490417	13.1774377822876\\
2.26999998092651	13.1606140136719\\
2.27500009536743	13.2074489593506\\
2.27999997138977	13.3113431930542\\
2.28500008583069	13.4874658584595\\
2.28999996185303	13.7237739562988\\
2.29500007629395	13.9477949142456\\
2.29999995231628	14.1423606872559\\
2.3050000667572	14.3955001831055\\
2.30999994277954	14.6963396072388\\
2.31500005722046	14.9905300140381\\
2.3199999332428	15.281665802002\\
2.32500004768372	15.5787324905396\\
2.32999992370605	15.9206762313843\\
2.33500003814697	16.2479858398438\\
2.33999991416931	16.5387687683105\\
2.34500002861023	16.7351322174072\\
2.34999990463257	16.8542213439941\\
2.35500001907349	17.0260887145996\\
2.35999989509583	17.2957057952881\\
2.36500000953674	17.5921516418457\\
2.36999988555908	17.9301567077637\\
2.375	18.3129062652588\\
2.38000011444092	18.6982040405273\\
2.38499999046326	19.1492748260498\\
2.39000010490417	19.6199283599854\\
2.39499998092651	20.0545654296875\\
2.40000009536743	20.4528369903564\\
2.40499997138977	20.7663440704346\\
2.41000008583069	21.0352058410645\\
2.41499996185303	21.1893920898438\\
2.42000007629395	21.2057304382324\\
2.42499995231628	21.092658996582\\
2.4300000667572	20.8708801269531\\
2.43499994277954	20.5603504180908\\
2.44000005722046	20.1800498962402\\
2.4449999332428	19.7468395233154\\
2.45000004768372	19.2621097564697\\
2.45499992370605	18.7548084259033\\
2.46000003814697	18.2781200408936\\
2.46499991416931	17.8077354431152\\
2.47000002861023	17.4557285308838\\
2.47499990463257	17.1932258605957\\
2.48000001907349	16.9959964752197\\
2.48499989509583	16.854829788208\\
2.49000000953674	16.7585697174072\\
2.49499988555908	16.7080402374268\\
2.5	16.6931533813477\\
2.50500011444092	16.7071094512939\\
2.50999999046326	16.6854228973389\\
2.51500010490417	16.5997753143311\\
2.51999998092651	16.3779411315918\\
2.52500009536743	16.0245780944824\\
2.52999997138977	15.6435813903809\\
2.53500008583069	15.2588872909546\\
2.53999996185303	14.8121767044067\\
2.54500007629395	14.4249105453491\\
2.54999995231628	14.0492267608643\\
2.5550000667572	13.7312002182007\\
2.55999994277954	13.3760557174683\\
2.56500005722046	13.0674905776978\\
2.5699999332428	12.782567024231\\
2.57500004768372	12.4830284118652\\
2.57999992370605	12.1604566574097\\
2.58500003814697	11.7642574310303\\
2.58999991416931	11.3329763412476\\
2.59500002861023	10.847638130188\\
2.59999990463257	10.3734140396118\\
2.60500001907349	9.83280754089355\\
2.60999989509583	9.31587314605713\\
2.61500000953674	8.84933090209961\\
2.61999988555908	8.46195507049561\\
2.625	8.18166351318359\\
2.63000011444092	8.0264778137207\\
2.63499999046326	7.99591302871704\\
2.64000010490417	8.07890796661377\\
2.64499998092651	8.27976989746094\\
2.65000009536743	8.58994483947754\\
2.65499997138977	8.93937683105469\\
2.66000008583069	9.26860809326172\\
2.66499996185303	9.56218242645264\\
2.67000007629395	9.79686641693115\\
2.67499995231628	9.93738842010498\\
2.6800000667572	10.0233201980591\\
2.68499994277954	10.0906581878662\\
2.69000005722046	10.1266422271729\\
2.6949999332428	10.1737089157104\\
2.70000004768372	10.2739305496216\\
2.70499992370605	10.4561614990234\\
2.71000003814697	10.736231803894\\
2.71499991416931	11.1359424591064\\
2.72000002861023	11.5976724624634\\
2.72499990463257	12.0810356140137\\
2.73000001907349	12.5971870422363\\
2.73499989509583	13.1146860122681\\
2.74000000953674	13.5907669067383\\
2.74499988555908	14.0589361190796\\
2.75	14.4467306137085\\
2.75500011444092	14.784125328064\\
2.75999999046326	15.0829458236694\\
2.76500010490417	15.4209012985229\\
2.76999998092651	15.779990196228\\
2.77500009536743	16.1828765869141\\
2.77999997138977	16.6361351013184\\
2.78500008583069	17.1360683441162\\
2.78999996185303	17.7948246002197\\
2.79500007629395	18.5213623046875\\
2.79999995231628	19.3144989013672\\
2.8050000667572	20.1001377105713\\
2.80999994277954	20.9178123474121\\
2.81500005722046	21.5753135681152\\
2.8199999332428	22.0443706512451\\
2.82500004768372	22.2656745910645\\
2.82999992370605	22.2338676452637\\
2.83500003814697	21.9043350219727\\
2.83999991416931	21.2633876800537\\
2.84500002861023	20.3755931854248\\
2.84999990463257	19.2101230621338\\
2.85500001907349	18.0543651580811\\
2.85999989509583	16.8916454315186\\
2.86500000953674	15.8031044006348\\
2.86999988555908	14.4265022277832\\
2.875	13.1272449493408\\
2.88000011444092	11.7853507995605\\
2.88499999046326	10.3160839080811\\
2.89000010490417	8.86307048797607\\
2.89499998092651	7.07807683944702\\
2.90000009536743	5.25930404663086\\
2.90499997138977	3.63577389717102\\
2.91000008583069	2.40881085395813\\
2.91499996185303	1.7541800737381\\
2.92000007629395	1.79772865772247\\
2.92499995231628	2.58175182342529\\
2.9300000667572	4.02959728240967\\
2.93499994277954	6.12749576568604\\
2.94000005722046	8.46991729736328\\
2.9449999332428	10.7364110946655\\
2.95000004768372	12.7321577072144\\
2.95499992370605	14.7535667419434\\
2.96000003814697	16.3806800842285\\
2.96499991416931	17.5446033477783\\
2.97000002861023	18.1946849822998\\
2.97499990463257	18.5957775115967\\
2.98000001907349	18.7848110198975\\
2.98499989509583	18.972993850708\\
2.99000000953674	19.1149368286133\\
2.99499988555908	18.2025260925293\\
3	17.4858913421631\\
3.00500011444092	17.0773735046387\\
3.00999999046326	16.9130420684814\\
3.01500010490417	16.964807510376\\
3.01999998092651	17.2424373626709\\
3.02500009536743	17.7592182159424\\
3.02999997138977	18.2875556945801\\
3.03500008583069	18.6711139678955\\
3.03999996185303	18.862865447998\\
3.04500007629395	18.8131847381592\\
3.04999995231628	18.4111938476563\\
3.0550000667572	17.6740531921387\\
3.05999994277954	16.7134647369385\\
3.06500005722046	15.660623550415\\
3.0699999332428	14.6718044281006\\
3.07500004768372	13.9133863449097\\
3.07999992370605	13.4580373764038\\
3.08500003814697	13.3372211456299\\
3.08999991416931	13.5412273406982\\
3.09500002861023	13.9761142730713\\
3.09999990463257	14.535737991333\\
3.10500001907349	15.1379289627075\\
3.10999989509583	15.7213487625122\\
3.11500000953674	16.24342918396\\
3.11999988555908	16.6919612884521\\
3.125	17.1048030853271\\
3.13000011444092	17.5957145690918\\
3.13499999046326	18.3122386932373\\
3.14000010490417	19.5806560516357\\
3.14499998092651	21.4909191131592\\
3.15000009536743	23.9442405700684\\
3.15499997138977	26.5635013580322\\
3.16000008583069	28.889087677002\\
3.16499996185303	30.6648464202881\\
3.17000007629395	31.6412563323975\\
3.17499995231628	31.6649532318115\\
3.1800000667572	30.5515995025635\\
3.18499994277954	28.4222030639648\\
3.19000005722046	25.4061603546143\\
3.1949999332428	21.7417602539063\\
3.20000004768372	17.6523723602295\\
3.20499992370605	13.2076463699341\\
3.21000003814697	8.67442798614502\\
3.21499991416931	4.2484917640686\\
3.22000002861023	0.333435297012329\\
3.22499990463257	-2.95939254760742\\
3.23000001907349	-5.50233840942383\\
3.23499989509583	-7.30611228942871\\
3.24000000953674	-8.26162338256836\\
3.24499988555908	-8.41799736022949\\
3.25	-7.80265331268311\\
3.25500011444092	-6.48008060455322\\
3.25999999046326	-4.56796407699585\\
3.26500010490417	-2.21645522117615\\
3.26999998092651	0.410067617893219\\
3.27500009536743	2.91736221313477\\
3.27999997138977	5.28065395355225\\
3.28500008583069	7.35287857055664\\
3.28999996185303	8.86666202545166\\
3.29500007629395	9.9426097869873\\
3.29999995231628	10.6753787994385\\
3.3050000667572	11.1971750259399\\
3.30999994277954	11.6009473800659\\
3.31500005722046	12.0093259811401\\
3.3199999332428	12.4893169403076\\
3.32500004768372	12.9906158447266\\
3.32999992370605	13.5902824401855\\
3.33500003814697	14.4153718948364\\
3.33999991416931	15.4703769683838\\
3.34500002861023	16.661096572876\\
3.34999990463257	17.918701171875\\
3.35500001907349	19.2631206512451\\
3.35999989509583	20.3364963531494\\
3.36500000953674	21.1431617736816\\
3.36999988555908	21.605676651001\\
3.375	21.8001976013184\\
3.38000011444092	21.6842975616455\\
3.38499999046326	21.30832862854\\
3.39000010490417	20.8422985076904\\
3.39499998092651	20.4319877624512\\
3.40000009536743	20.1458549499512\\
3.40499997138977	20.0418682098389\\
3.41000008583069	20.0899391174316\\
3.41499996185303	20.322883605957\\
3.42000007629395	20.7230052947998\\
3.42499995231628	21.1949100494385\\
3.4300000667572	21.6589527130127\\
3.43499994277954	22.0838317871094\\
3.44000005722046	22.4601192474365\\
3.4449999332428	22.8235549926758\\
3.45000004768372	23.2443580627441\\
3.45499992370605	23.7986927032471\\
3.46000003814697	24.6504802703857\\
3.46499991416931	26.1204681396484\\
3.47000002861023	28.6234912872314\\
3.47499990463257	31.5640926361084\\
3.48000001907349	34.6494369506836\\
3.48499989509583	37.6603050231934\\
3.49000000953674	39.9190483093262\\
3.49499988555908	40.9286422729492\\
3.5	40.6933097839355\\
3.50500011444092	39.2869720458984\\
3.50999999046326	36.9078636169434\\
3.51500010490417	33.3788185119629\\
3.51999998092651	28.8805084228516\\
3.52500009536743	23.2647895812988\\
3.52999997138977	16.4273509979248\\
3.53500008583069	8.2565803527832\\
3.53999996185303	-1.30618846416473\\
3.54500007629395	-12.2317771911621\\
3.54999995231628	-24.1891422271729\\
3.5550000667572	-35.5783500671387\\
3.55999994277954	-43.9878540039063\\
3.56500005722046	-47.885009765625\\
3.5699999332428	-46.7194213867188\\
3.57500004768372	-40.3613624572754\\
3.57999992370605	-30.0624523162842\\
3.58500003814697	-17.2593879699707\\
3.58999991416931	-3.90793180465698\\
3.59500002861023	8.86421489715576\\
3.59999990463257	19.3517036437988\\
3.60500001907349	26.1028156280518\\
3.60999989509583	28.879430770874\\
3.61500000953674	28.7035980224609\\
3.61999988555908	26.4233264923096\\
3.625	22.6909351348877\\
3.63000011444092	18.0625686645508\\
3.63499999046326	13.5092821121216\\
3.64000010490417	9.61331367492676\\
3.64499998092651	7.04532527923584\\
3.65000009536743	6.63253974914551\\
3.65499997138977	8.45427322387695\\
3.66000008583069	12.3379096984863\\
3.66499996185303	17.1914615631104\\
3.67000007629395	22.2036991119385\\
3.67499995231628	26.3822002410889\\
3.6800000667572	29.1517696380615\\
3.68499994277954	29.980525970459\\
3.69000005722046	28.823221206665\\
3.6949999332428	26.0650615692139\\
3.70000004768372	22.1576194763184\\
3.70499992370605	18.0728816986084\\
3.71000003814697	14.5573844909668\\
3.71499991416931	12.4539842605591\\
3.72000002861023	11.764328956604\\
3.72499990463257	12.7555494308472\\
3.73000001907349	15.1669139862061\\
3.73499989509583	18.4507141113281\\
3.74000000953674	22.1958980560303\\
3.74499988555908	25.7111053466797\\
3.75	28.7810020446777\\
3.75500011444092	31.5454196929932\\
3.75999999046326	34.5754928588867\\
3.76500010490417	38.3472061157227\\
3.76999998092651	42.979118347168\\
3.77500009536743	47.8526496887207\\
3.77999997138977	52.234504699707\\
3.78500008583069	55.0734252929688\\
3.78999996185303	56.062614440918\\
3.79500007629395	55.1339340209961\\
3.79999995231628	52.3423385620117\\
3.8050000667572	47.5926246643066\\
3.80999994277954	40.705696105957\\
3.81500005722046	31.4942111968994\\
3.8199999332428	19.9757289886475\\
3.82500004768372	6.20778656005859\\
3.82999992370605	-9.45774078369141\\
3.83500003814697	-26.3209476470947\\
3.83999991416931	-43.0674171447754\\
3.84500002861023	-56.9509582519531\\
3.84999990463257	-64.9076080322266\\
3.85500001907349	-65.810905456543\\
3.85999989509583	-59.3585319519043\\
3.86500000953674	-47.0546073913574\\
3.86999988555908	-30.9839267730713\\
3.875	-12.9954605102539\\
3.88000011444092	4.4370551109314\\
3.88499999046326	19.0430240631104\\
3.89000010490417	28.7012748718262\\
3.89499998092651	33.1046829223633\\
3.90000009536743	32.303409576416\\
3.90499997138977	28.4752597808838\\
3.91000008583069	22.7891979217529\\
3.91499996185303	16.1153240203857\\
3.92000007629395	9.95900726318359\\
3.92499995231628	5.21206521987915\\
3.9300000667572	2.44485449790955\\
3.93499994277954	2.68633651733398\\
3.94000005722046	6.11612987518311\\
3.9449999332428	11.7996826171875\\
3.95000004768372	18.3859348297119\\
3.95499992370605	24.6786403656006\\
3.96000003814697	29.8748226165771\\
3.96499991416931	33.2726898193359\\
3.97000002861023	34.2201728820801\\
3.97499990463257	32.7213134765625\\
3.98000001907349	29.1812133789063\\
3.98499989509583	24.5265579223633\\
3.99000000953674	19.670970916748\\
3.99499988555908	15.506196975708\\
4	12.9467144012451\\
4.00500011444092	12.3319807052612\\
4.01000022888184	13.6557426452637\\
4.0149998664856	16.7404022216797\\
4.01999998092651	20.9162712097168\\
4.02500009536743	25.5857372283936\\
4.03000020980835	30.1708850860596\\
4.03499984741211	34.539379119873\\
4.03999996185303	39.1223373413086\\
4.04500007629395	44.345832824707\\
4.05000019073486	50.0467567443848\\
4.05499982833862	56.0052719116211\\
4.05999994277954	61.5267333984375\\
4.06500005722046	65.6780548095703\\
4.07000017166138	67.5138626098633\\
4.07499980926514	67.2543182373047\\
4.07999992370605	65.1199569702148\\
4.08500003814697	61.0135879516602\\
4.09000015258789	54.2512512207031\\
4.09499979019165	44.0368194580078\\
4.09999990463257	29.0306053161621\\
4.10500001907349	7.5391788482666\\
4.1100001335144	-22.4935741424561\\
4.11499977111816	-60.3292198181152\\
4.11999988555908	-98.895378112793\\
4.125	-128.749862670898\\
4.13000011444092	-141.96044921875\\
4.13500022888184	-133.181335449219\\
4.1399998664856	-108.093605041504\\
4.14499998092651	-71.0451889038086\\
4.15000009536743	-27.8920116424561\\
4.15500020980835	14.3772993087769\\
4.15999984741211	49.5836334228516\\
4.16499996185303	72.2282791137695\\
4.17000007629395	79.6492004394531\\
4.17500019073486	75.1273193359375\\
4.17999982833862	62.2615547180176\\
4.18499994277954	43.4180946350098\\
4.19000005722046	22.1947574615479\\
4.19500017166138	2.31736135482788\\
4.19999980926514	-13.0421676635742\\
4.20499992370605	-21.3998718261719\\
4.21000003814697	-20.9841690063477\\
4.21500015258789	-11.5529546737671\\
4.21999979019165	3.53352284431458\\
4.22499990463257	20.2455787658691\\
4.23000001907349	34.9830665588379\\
4.2350001335144	45.5661811828613\\
4.23999977111816	50.9357757568359\\
4.24499988555908	50.6055946350098\\
4.25	44.6548919677734\\
4.25500011444092	34.4172630310059\\
4.26000022888184	21.9204349517822\\
4.2649998664856	9.72199535369873\\
4.26999998092651	0.557688772678375\\
4.27500009536743	-4.10888910293579\\
4.28000020980835	-3.48353028297424\\
4.28499984741211	2.67712211608887\\
4.28999996185303	12.7242574691772\\
4.29500007629395	24.2432613372803\\
4.30000019073486	35.4953422546387\\
4.30499982833862	46.3463363647461\\
4.30999994277954	57.181510925293\\
4.31500005722046	67.7696533203125\\
4.32000017166138	77.3041305541992\\
4.32499980926514	84.6580352783203\\
4.32999992370605	89.212516784668\\
4.33500003814697	90.4490509033203\\
4.34000015258789	88.0380935668945\\
4.34499979019165	82.9969787597656\\
4.34999990463257	74.4833679199219\\
4.35500001907349	61.3791847229004\\
4.3600001335144	42.2145309448242\\
4.36499977111816	15.2168025970459\\
4.36999988555908	-21.2056350708008\\
4.375	-65.1369934082031\\
4.38000011444092	-107.107803344727\\
4.38500022888184	-139.911361694336\\
4.3899998664856	-161.332183837891\\
4.39499998092651	-168.703872680664\\
4.40000009536743	-153.739013671875\\
4.40500020980835	-113.084243774414\\
4.40999984741211	-57.2380447387695\\
4.41499996185303	1.83397889137268\\
4.42000007629395	53.9246864318848\\
4.42500019073486	89.7109756469727\\
4.42999982833862	104.848243713379\\
4.43499994277954	101.013244628906\\
4.44000005722046	83.8194732666016\\
4.44500017166138	57.5948753356934\\
4.44999980926514	27.4522819519043\\
4.45499992370605	-1.17765438556671\\
4.46000003814697	-23.8738059997559\\
4.46500015258789	-36.3211936950684\\
4.46999979019165	-36.7243156433105\\
4.47499990463257	-24.3082656860352\\
4.48000001907349	-3.69723296165466\\
4.4850001335144	19.3133945465088\\
4.48999977111816	39.7178153991699\\
4.49499988555908	54.3290939331055\\
4.5	61.8470458984375\\
4.50500011444092	61.9339561462402\\
4.51000022888184	54.610897064209\\
4.5149998664856	41.2329521179199\\
4.51999998092651	24.8690567016602\\
4.52500009536743	8.64414596557617\\
4.53000020980835	-3.67818427085876\\
4.53499984741211	-10.0692081451416\\
4.53999996185303	-9.54603862762451\\
4.54500007629395	-1.74285268783569\\
4.55000019073486	11.1221885681152\\
4.55499982833862	25.9252948760986\\
4.55999994277954	40.4540519714355\\
4.56500005722046	53.7178382873535\\
4.57000017166138	63.7352333068848\\
4.57499980926514	72.1884994506836\\
4.57999992370605	79.6463623046875\\
4.58500003814697	85.9086380004883\\
4.59000015258789	90.4205551147461\\
4.59499979019165	93.2821578979492\\
4.59999990463257	94.8465881347656\\
4.60500001907349	95.3401947021484\\
4.6100001335144	93.8684005737305\\
4.61499977111816	89.4003982543945\\
4.61999988555908	75.9716567993164\\
4.625	44.6160163879395\\
4.63000011444092	-22.5421943664551\\
4.63500022888184	-104.572082519531\\
4.6399998664856	-174.900405883789\\
4.64499998092651	-220.102691650391\\
4.65000009536743	-240.251480102539\\
4.65500020980835	-239.824493408203\\
4.65999984741211	-227.494613647461\\
4.66499996185303	-197.391662597656\\
4.67000007629395	-118.306739807129\\
4.67500019073486	-9.78057956695557\\
4.67999982833862	90.1796112060547\\
4.68499994277954	161.452194213867\\
4.69000005722046	194.232086181641\\
4.69500017166138	189.1513671875\\
4.69999980926514	156.315719604492\\
4.70499992370605	103.702590942383\\
4.71000003814697	41.8081398010254\\
4.71500015258789	-17.5506992340088\\
4.71999979019165	-64.7988662719727\\
4.72499990463257	-92.0670623779297\\
4.73000001907349	-92.9908218383789\\
4.7350001335144	-68.8510818481445\\
4.73999977111816	-28.6708450317383\\
4.74499988555908	16.0475540161133\\
4.75	55.1220436096191\\
4.75500011444092	81.778923034668\\
4.76000022888184	94.4963073730469\\
4.7649998664856	92.6857299804688\\
4.76999998092651	76.6567306518555\\
4.77500009536743	49.8209266662598\\
4.78000020980835	17.8764038085938\\
4.78499984741211	-12.748101234436\\
4.78999996185303	-33.8157348632813\\
4.79500007629395	-42.049976348877\\
4.80000019073486	-38.3991165161133\\
4.80499982833862	-21.4100742340088\\
4.80999994277954	4.9152364730835\\
4.81500005722046	35.7884483337402\\
4.82000017166138	64.1900863647461\\
4.82499980926514	87.4602127075195\\
4.82999992370605	105.555114746094\\
4.83500003814697	118.984436035156\\
4.84000015258789	128.274200439453\\
4.84499979019165	134.415405273438\\
4.84999990463257	136.313522338867\\
4.85500001907349	131.777984619141\\
4.8600001335144	118.185020446777\\
4.86499977111816	93.2461547851563\\
4.86999988555908	50.5056037902832\\
4.875	0.736502528190613\\
4.88000011444092	-65.9027328491211\\
4.88500022888184	-139.376724243164\\
4.8899998664856	-200.507629394531\\
4.89499998092651	-238.398056030273\\
4.90000009536743	-252.489059448242\\
4.90500020980835	-248.54133605957\\
4.90999984741211	-232.884521484375\\
4.91499996185303	-208.303558349609\\
4.92000007629395	-170.771286010742\\
4.92500019073486	-88.3598403930664\\
4.92999982833862	36.2874298095703\\
4.93499994277954	155.122848510742\\
4.94000005722046	235.062942504883\\
4.94500017166138	256.614501953125\\
4.94999980926514	231.488555908203\\
4.95499992370605	174.68408203125\\
4.96000003814697	99.7792358398438\\
4.96500015258789	22.768253326416\\
4.96999979019165	-41.6360626220703\\
4.97499990463257	-85.1828155517578\\
4.98000001907349	-102.212791442871\\
4.9850001335144	-93.5943374633789\\
4.98999977111816	-63.5765190124512\\
4.99499988555908	-21.1792583465576\\
5	23.1891841888428\\
5.00500011444092	61.9671821594238\\
5.01000022888184	89.9578704833984\\
5.0149998664856	105.388214111328\\
5.01999998092651	109.072357177734\\
5.02500009536743	102.305381774902\\
5.03000020980835	88.0640411376953\\
5.03499984741211	69.8765869140625\\
5.03999996185303	50.5421943664551\\
5.04500007629395	32.5781936645508\\
5.05000019073486	17.2840251922607\\
5.05499982833862	6.54157829284668\\
5.05999994277954	0.30882939696312\\
5.06500005722046	-0.66469019651413\\
5.07000017166138	2.31176352500916\\
5.07499980926514	8.10561084747314\\
5.07999992370605	15.3003835678101\\
5.08500003814697	20.8938541412354\\
5.09000015258789	24.2746925354004\\
5.09499979019165	22.7062950134277\\
5.09999990463257	17.8292446136475\\
5.10500001907349	11.116847038269\\
5.1100001335144	5.15850782394409\\
5.11499977111816	1.88211488723755\\
5.11999988555908	0.222312867641449\\
5.125	-0.468132495880127\\
5.13000011444092	-0.772791624069214\\
5.13500022888184	-10.5626649856567\\
5.1399998664856	-36.7919502258301\\
5.14499998092651	-81.3309783935547\\
5.15000009536743	-116.200347900391\\
5.15500020980835	-134.732284545898\\
5.15999984741211	-135.238418579102\\
5.16499996185303	-105.1962890625\\
5.17000007629395	-51.371021270752\\
5.17500019073486	8.56664180755615\\
5.17999982833862	61.8024215698242\\
5.18499994277954	98.5104827880859\\
5.19000005722046	114.861297607422\\
5.19500017166138	112.53621673584\\
5.19999980926514	98.6324996948242\\
5.20499992370605	77.5761413574219\\
5.21000003814697	53.4613609313965\\
5.21500015258789	30.3098335266113\\
5.21999979019165	11.6833982467651\\
5.22499990463257	-0.407345861196518\\
5.23000001907349	-5.06433391571045\\
5.2350001335144	-2.78331923484802\\
5.23999977111816	4.45655918121338\\
5.24499988555908	14.8728523254395\\
5.25	26.4961318969727\\
5.25500011444092	37.5445938110352\\
5.26000022888184	46.4459571838379\\
5.2649998664856	52.6636009216309\\
5.26999998092651	55.5189552307129\\
5.27500009536743	54.9261283874512\\
5.28000020980835	50.6174926757813\\
5.28499984741211	43.1970024108887\\
5.28999996185303	33.3644256591797\\
5.29500007629395	22.176456451416\\
5.30000019073486	10.6498136520386\\
5.30499982833862	-0.381906360387802\\
5.30999994277954	-9.85058689117432\\
5.31500005722046	-16.3443088531494\\
5.32000017166138	-19.2610378265381\\
5.32499980926514	-18.2046241760254\\
5.32999992370605	-13.3924741744995\\
5.33500003814697	-6.63382863998413\\
5.34000015258789	0.579433083534241\\
5.34499979019165	6.87112617492676\\
5.34999990463257	11.5002393722534\\
5.35500001907349	13.5198373794556\\
5.3600001335144	12.4546909332275\\
5.36499977111816	8.11277103424072\\
5.36999988555908	1.0876978635788\\
5.375	-7.43859052658081\\
5.38000011444092	-16.0015125274658\\
5.38500022888184	-22.6430549621582\\
5.3899998664856	-26.3633785247803\\
5.39499998092651	-26.3372917175293\\
5.40000009536743	-22.5466213226318\\
5.40500020980835	-16.3645401000977\\
5.40999984741211	-9.66730213165283\\
5.41499996185303	-3.90858507156372\\
5.42000007629395	-0.227553173899651\\
5.42500019073486	0.0870256945490837\\
5.42999982833862	-2.69905996322632\\
5.43499994277954	-6.84904527664185\\
5.44000005722046	-8.66058254241943\\
5.44500017166138	-8.28981685638428\\
5.44999980926514	-7.13328075408936\\
5.45499992370605	-6.3591160774231\\
5.46000003814697	-5.76097679138184\\
5.46500015258789	-5.22617530822754\\
5.46999979019165	-4.68203020095825\\
5.47499990463257	-4.19507455825806\\
5.48000001907349	-3.75738120079041\\
5.4850001335144	-3.34024214744568\\
5.48999977111816	-3.00074791908264\\
5.49499988555908	-2.68399715423584\\
5.5	-2.4114248752594\\
5.50500011444092	-2.17769527435303\\
5.51000022888184	-1.99484145641327\\
5.5149998664856	-1.83645606040955\\
5.51999998092651	-1.99794542789459\\
5.52500009536743	-4.1229567527771\\
5.53000020980835	-10.4650259017944\\
5.53499984741211	-16.7727508544922\\
5.53999996185303	-21.9695682525635\\
5.54500007629395	-25.8999557495117\\
5.55000019073486	-28.9204635620117\\
5.55499982833862	-30.9715633392334\\
5.55999994277954	-32.1703262329102\\
5.56500005722046	-32.8653526306152\\
5.57000017166138	-33.013729095459\\
5.57499980926514	-32.6650123596191\\
5.57999992370605	-31.6957855224609\\
5.58500003814697	-30.1737976074219\\
5.59000015258789	-28.1308212280273\\
5.59499979019165	-25.6615581512451\\
5.59999990463257	-22.9865188598633\\
5.60500001907349	-20.1103286743164\\
5.6100001335144	-17.2488842010498\\
5.61499977111816	-14.6226797103882\\
5.61999988555908	-12.262752532959\\
5.625	-10.2100114822388\\
5.63000011444092	-8.58419132232666\\
5.63500022888184	-7.34681701660156\\
5.6399998664856	-6.35813140869141\\
5.64499998092651	-5.69404029846191\\
5.65000009536743	-5.28879642486572\\
5.65500020980835	-5.01367473602295\\
5.65999984741211	-4.79592132568359\\
5.66499996185303	-4.60314893722534\\
5.67000007629395	-4.21642637252808\\
5.67500019073486	0.851646602153778\\
5.67999982833862	25.731315612793\\
5.68499994277954	53.6877403259277\\
5.69000005722046	67.7666702270508\\
5.69500017166138	63.8119354248047\\
5.69999980926514	49.0636787414551\\
5.70499992370605	30.4799423217773\\
5.71000003814697	12.6292867660522\\
5.71500015258789	-0.427082091569901\\
5.71999979019165	-6.4401798248291\\
5.72499990463257	-5.09118795394897\\
5.73000001907349	2.48332166671753\\
5.7350001335144	14.3413934707642\\
5.73999977111816	27.1794567108154\\
5.74499988555908	38.7320022583008\\
5.75	47.365406036377\\
5.75500011444092	52.553466796875\\
5.76000022888184	54.685905456543\\
5.7649998664856	54.8935546875\\
5.76999998092651	54.4256477355957\\
5.77500009536743	54.5371398925781\\
5.78000020980835	55.6397094726563\\
5.78499984741211	57.7104034423828\\
5.78999996185303	60.4158821105957\\
5.79500007629395	63.2842826843262\\
5.80000019073486	66.0148468017578\\
5.80499982833862	68.2247009277344\\
5.80999994277954	69.7783584594727\\
5.81500005722046	70.7487564086914\\
5.82000017166138	71.1946105957031\\
5.82499980926514	71.4034118652344\\
5.82999992370605	71.7542037963867\\
5.83500003814697	72.3246536254883\\
5.84000015258789	73.5429458618164\\
5.84499979019165	75.5746765136719\\
5.84999990463257	78.6579742431641\\
5.85500001907349	83.0317153930664\\
5.8600001335144	88.7255783081055\\
5.86499977111816	95.5280227661133\\
5.86999988555908	102.242805480957\\
5.875	107.60913848877\\
5.88000011444092	109.147567749023\\
5.88500022888184	104.450645446777\\
5.8899998664856	90.1572265625\\
5.89499998092651	66.2388381958008\\
5.90000009536743	43.2063331604004\\
5.90500020980835	24.80104637146\\
5.90999984741211	5.08878564834595\\
5.91499996185303	-23.7833995819092\\
5.92000007629395	-65.8712387084961\\
5.92500019073486	-116.660583496094\\
5.92999982833862	-167.189010620117\\
5.93499994277954	-209.292419433594\\
5.94000005722046	-237.196472167969\\
5.94500017166138	-248.873138427734\\
5.94999980926514	-246.03141784668\\
5.95499992370605	-232.216567993164\\
5.96000003814697	-213.120193481445\\
5.96500015258789	-193.080230712891\\
5.96999979019165	-176.709091186523\\
5.97499990463257	-165.242538452148\\
5.98000001907349	-158.089218139648\\
5.9850001335144	-110.318176269531\\
5.98999977111816	46.4561042785645\\
5.99499988555908	221.081985473633\\
6	361.681610107422\\
6.00500011444092	438.889312744141\\
6.01000022888184	446.922973632813\\
6.0149998664856	398.642608642578\\
6.01999998092651	314.269592285156\\
6.02500009536743	211.142761230469\\
6.03000020980835	107.207588195801\\
6.03499984741211	16.9939002990723\\
6.03999996185303	-51.2597312927246\\
6.04500007629395	-94.7028503417969\\
6.05000019073486	-111.230079650879\\
6.05499982833862	-98.4726486206055\\
6.05999994277954	-60.9971885681152\\
6.06500005722046	-8.67083740234375\\
6.07000017166138	47.6354713439941\\
6.07499980926514	98.8062362670898\\
6.07999992370605	137.949035644531\\
6.08500003814697	161.419799804688\\
6.09000015258789	169.129470825195\\
6.09499979019165	162.797439575195\\
6.09999990463257	142.425979614258\\
6.10500001907349	111.736145019531\\
6.1100001335144	75.4760208129883\\
6.11499977111816	37.5182723999023\\
6.11999988555908	1.91739296913147\\
6.125	-26.906681060791\\
6.13000011444092	-44.1142501831055\\
6.13500022888184	-47.6310729980469\\
6.1399998664856	-36.8542709350586\\
6.14499998092651	-16.1810302734375\\
6.15000009536743	9.0445442199707\\
6.15500020980835	33.5014228820801\\
6.15999984741211	53.615234375\\
6.16499996185303	67.3294372558594\\
6.17000007629395	72.1822967529297\\
6.17500019073486	67.1854248046875\\
6.17999982833862	52.7266807556152\\
6.18499994277954	30.7738380432129\\
6.19000005722046	6.0717248916626\\
6.19500017166138	-13.1048135757446\\
6.19999980926514	-22.0036125183105\\
6.20499992370605	-23.3477001190186\\
6.21000003814697	-20.0788440704346\\
6.21500015258789	-14.937819480896\\
6.21999979019165	-9.95428848266602\\
6.22499990463257	-7.51641941070557\\
6.23000001907349	-6.19535636901855\\
6.2350001335144	-5.25976133346558\\
6.23999977111816	-4.5326828956604\\
6.24499988555908	-3.83431339263916\\
6.25	-3.20930433273315\\
6.25500011444092	-2.68195676803589\\
6.26000022888184	-2.24186849594116\\
6.2649998664856	-1.87485671043396\\
6.26999998092651	-1.57086038589478\\
6.27500009536743	-1.31903266906738\\
6.28000020980835	-1.10829162597656\\
6.28499984741211	-0.93100243806839\\
6.28999996185303	-0.783238232135773\\
6.29500007629395	-0.661689281463623\\
6.30000019073486	-0.560480296611786\\
6.30499982833862	-0.475040942430496\\
6.30999994277954	-0.402269631624222\\
6.31500005722046	-0.340233713388443\\
6.32000017166138	-0.286983698606491\\
6.32499980926514	-0.239489197731018\\
6.32999992370605	-0.196573913097382\\
6.33500003814697	-0.159614577889442\\
6.34000015258789	-0.132668018341064\\
6.34499979019165	-0.116938516497612\\
6.34999990463257	-0.106173798441887\\
6.35500001907349	-0.094655878841877\\
6.3600001335144	3.10338830947876\\
6.36499977111816	19.291259765625\\
6.36999988555908	40.9978065490723\\
6.375	60.0613136291504\\
6.38000011444092	73.3197860717773\\
6.38500022888184	83.2616806030273\\
6.3899998664856	83.4956665039063\\
6.39499998092651	56.5291023254395\\
6.40000009536743	23.0149154663086\\
6.40500020980835	-7.62400150299072\\
6.40999984741211	-30.7103633880615\\
6.41499996185303	-43.1507759094238\\
6.42000007629395	-44.7611045837402\\
6.42500019073486	-38.032585144043\\
6.42999982833862	-24.9750022888184\\
6.43499994277954	-8.5321741104126\\
6.44000005722046	8.73071193695068\\
6.44500017166138	24.5815525054932\\
6.44999980926514	37.6362915039063\\
6.45499992370605	46.6713638305664\\
6.46000003814697	51.48974609375\\
6.46500015258789	52.8477439880371\\
6.46999979019165	51.3276405334473\\
6.47499990463257	47.8554801940918\\
6.48000001907349	43.5436668395996\\
6.4850001335144	39.1823959350586\\
6.48999977111816	35.5128173828125\\
6.49499988555908	33.0741500854492\\
6.5	32.1461982727051\\
6.50500011444092	32.6514701843262\\
6.51000022888184	34.3332710266113\\
6.5149998664856	36.947093963623\\
6.51999998092651	40.169994354248\\
6.52500009536743	43.5589141845703\\
6.53000020980835	46.8544692993164\\
6.53499984741211	49.7870063781738\\
6.53999996185303	52.1305503845215\\
6.54500007629395	53.8908042907715\\
6.55000019073486	54.9432487487793\\
6.55499982833862	55.473331451416\\
6.55999994277954	55.5626983642578\\
6.56500005722046	55.2600059509277\\
6.57000017166138	54.7937927246094\\
6.57499980926514	54.2074279785156\\
6.57999992370605	53.6077156066895\\
6.58500003814697	53.0420799255371\\
6.59000015258789	52.6055564880371\\
6.59499979019165	52.3557968139648\\
6.59999990463257	52.2812004089355\\
6.60500001907349	52.378345489502\\
6.6100001335144	52.633113861084\\
6.61499977111816	52.9883499145508\\
6.61999988555908	53.3930320739746\\
6.625	53.7723083496094\\
6.63000011444092	54.0501708984375\\
6.63500022888184	54.2069931030273\\
6.6399998664856	54.2195510864258\\
6.64499998092651	54.0902137756348\\
6.65000009536743	53.847541809082\\
6.65500020980835	53.433895111084\\
6.65999984741211	52.9205055236816\\
6.66499996185303	52.3999786376953\\
6.67000007629395	51.8799171447754\\
6.67500019073486	51.3944282531738\\
6.67999982833862	50.9749221801758\\
6.68499994277954	50.6595497131348\\
6.69000005722046	50.3731689453125\\
6.69500017166138	50.1203117370605\\
6.69999980926514	50.006965637207\\
6.70499992370605	49.8641662597656\\
6.71000003814697	49.7473907470703\\
6.71500015258789	49.6156425476074\\
6.71999979019165	49.4237861633301\\
6.72499990463257	49.1167984008789\\
6.73000001907349	48.674129486084\\
6.7350001335144	48.091194152832\\
6.73999977111816	47.3774757385254\\
6.74499988555908	46.5814590454102\\
6.75	45.8414001464844\\
6.75500011444092	45.0785179138184\\
6.76000022888184	44.3347663879395\\
6.7649998664856	43.5583648681641\\
6.76999998092651	42.784049987793\\
6.77500009536743	42.2890663146973\\
6.78000020980835	41.7910423278809\\
6.78499984741211	41.4384727478027\\
6.78999996185303	41.2889556884766\\
6.79500007629395	41.2616958618164\\
6.80000019073486	41.3345222473145\\
6.80499982833862	41.4888534545898\\
6.80999994277954	41.6621322631836\\
6.81500005722046	41.8397674560547\\
6.82000017166138	41.9941902160645\\
6.82499980926514	42.1155014038086\\
6.82999992370605	42.160717010498\\
6.83500003814697	42.0989532470703\\
6.84000015258789	42.0285987854004\\
6.84499979019165	41.9663734436035\\
6.84999990463257	41.8292007446289\\
6.85500001907349	41.7137985229492\\
6.8600001335144	41.6779518127441\\
6.86499977111816	41.7623977661133\\
6.86999988555908	41.9970474243164\\
6.875	42.3732452392578\\
6.88000011444092	42.7973823547363\\
6.88500022888184	43.2221450805664\\
6.8899998664856	43.5854873657227\\
6.89499998092651	43.8539848327637\\
6.90000009536743	44.0062255859375\\
6.90500020980835	44.0249824523926\\
6.90999984741211	43.8119964599609\\
6.91499996185303	43.3392372131348\\
6.92000007629395	42.6539192199707\\
6.92500019073486	41.7645416259766\\
6.92999982833862	40.6969299316406\\
6.93499994277954	39.5040473937988\\
6.94000005722046	38.2389030456543\\
6.94500017166138	36.9461441040039\\
6.94999980926514	35.6596336364746\\
6.95499992370605	34.4371147155762\\
6.96000003814697	33.310131072998\\
6.96500015258789	32.2824172973633\\
6.96999979019165	31.2704582214355\\
6.97499990463257	30.2655906677246\\
6.98000001907349	29.3443336486816\\
6.9850001335144	28.3420867919922\\
6.98999977111816	27.2876510620117\\
6.99499988555908	26.172248840332\\
7	24.9689826965332\\
7.00500011444092	23.6874046325684\\
7.01000022888184	22.328311920166\\
7.0149998664856	20.9338512420654\\
7.01999998092651	19.5366840362549\\
7.02500009536743	18.170446395874\\
7.03000020980835	16.8595390319824\\
7.03499984741211	15.6278524398804\\
7.03999996185303	14.4940042495728\\
7.04500007629395	13.4720315933228\\
7.05000019073486	12.5666284561157\\
7.05499982833862	11.7697553634644\\
7.05999994277954	11.0678615570068\\
7.06500005722046	10.4436950683594\\
7.07000017166138	9.87429141998291\\
7.07499980926514	9.3263988494873\\
7.07999992370605	8.79757881164551\\
7.08500003814697	8.25664329528809\\
7.09000015258789	7.69996070861816\\
7.09499979019165	7.13476657867432\\
7.09999990463257	6.5631365776062\\
7.10500001907349	5.99403953552246\\
7.1100001335144	5.45608425140381\\
7.11499977111816	4.96697616577148\\
7.11999988555908	4.59038829803467\\
7.125	4.27325487136841\\
7.13000011444092	4.01876449584961\\
7.13500022888184	3.88515281677246\\
7.1399998664856	3.85381317138672\\
7.14499998092651	3.88437700271606\\
7.15000009536743	3.95612335205078\\
7.15500020980835	4.08527898788452\\
7.15999984741211	4.22294235229492\\
7.16499996185303	4.32218647003174\\
7.17000007629395	4.35340118408203\\
7.17500019073486	4.34277153015137\\
7.17999982833862	4.34975099563599\\
7.18499994277954	4.3902645111084\\
7.19000005722046	4.44836282730103\\
7.19500017166138	4.5366358757019\\
7.19999980926514	4.66672611236572\\
7.20499992370605	4.84256315231323\\
7.21000003814697	5.05661201477051\\
7.21500015258789	5.31096506118774\\
7.21999979019165	5.5932502746582\\
7.22499990463257	5.88517904281616\\
7.23000001907349	6.19211912155151\\
7.2350001335144	6.4994101524353\\
7.23999977111816	6.79867172241211\\
7.24499988555908	7.09049844741821\\
7.25	7.36971998214722\\
7.25500011444092	7.63721132278442\\
7.26000022888184	7.8988823890686\\
7.2649998664856	8.15360355377197\\
7.26999998092651	8.40262889862061\\
7.27500009536743	8.64839744567871\\
7.28000020980835	8.89196968078613\\
7.28499984741211	9.13443660736084\\
7.28999996185303	9.37796211242676\\
7.29500007629395	9.62195682525635\\
7.30000019073486	9.86664867401123\\
7.30499982833862	10.1061096191406\\
7.30999994277954	10.3384294509888\\
7.31500005722046	10.5615253448486\\
7.32000017166138	10.7726898193359\\
7.32499980926514	10.96812915802\\
7.32999992370605	11.1455583572388\\
7.33500003814697	11.3049087524414\\
7.34000015258789	11.4459009170532\\
7.34499979019165	11.5679655075073\\
7.34999990463257	11.671181678772\\
7.35500001907349	11.7540111541748\\
7.3600001335144	11.8151693344116\\
7.36499977111816	11.8565349578857\\
7.36999988555908	11.8794269561768\\
7.375	11.8880758285522\\
7.38000011444092	11.885422706604\\
7.38500022888184	11.8743257522583\\
7.3899998664856	11.8543167114258\\
7.39499998092651	11.8213834762573\\
7.40000009536743	11.7731800079346\\
7.40500020980835	11.7161426544189\\
7.40999984741211	11.648853302002\\
7.41499996185303	11.5685062408447\\
7.42000007629395	11.4722700119019\\
7.42500019073486	11.3584327697754\\
7.42999982833862	11.2284574508667\\
7.43499994277954	11.084641456604\\
7.44000005722046	10.9289951324463\\
7.44500017166138	10.7635507583618\\
7.44999980926514	10.5888938903809\\
7.45499992370605	10.4071960449219\\
7.46000003814697	10.2231197357178\\
7.46500015258789	10.0406436920166\\
7.46999979019165	9.86179351806641\\
7.47499990463257	9.68655490875244\\
7.48000001907349	9.51436805725098\\
7.4850001335144	9.34317970275879\\
7.48999977111816	9.17197704315186\\
7.49499988555908	9.00252723693848\\
7.5	8.83560752868652\\
7.50500011444092	8.67115783691406\\
7.51000022888184	8.51022434234619\\
7.5149998664856	8.35221767425537\\
7.51999998092651	8.19722557067871\\
7.52500009536743	8.04399967193604\\
7.53000020980835	7.89048290252686\\
7.53499984741211	7.73407363891602\\
7.53999996185303	7.59742116928101\\
7.54500007629395	7.47113370895386\\
7.55000019073486	7.35822343826294\\
7.55499982833862	7.26014089584351\\
7.55999994277954	7.17560577392578\\
7.56500005722046	7.10222387313843\\
7.57000017166138	7.03889656066895\\
7.57499980926514	6.98401832580566\\
7.57999992370605	6.93653202056885\\
7.58500003814697	6.89340591430664\\
7.59000015258789	6.85342741012573\\
7.59499979019165	6.81777143478394\\
7.59999990463257	6.7861533164978\\
7.60500001907349	6.75861692428589\\
7.6100001335144	6.73518562316895\\
7.61499977111816	6.71901512145996\\
7.61999988555908	6.7174711227417\\
7.625	6.72936296463013\\
7.63000011444092	6.75023365020752\\
7.63500022888184	6.78331565856934\\
7.6399998664856	6.83103609085083\\
7.64499998092651	6.88795757293701\\
7.65000009536743	6.95028686523438\\
7.65500020980835	7.01673793792725\\
7.65999984741211	7.08568954467773\\
7.66499996185303	7.15561819076538\\
7.67000007629395	7.22556591033936\\
7.67500019073486	7.2945704460144\\
7.67999982833862	7.36268424987793\\
7.68499994277954	7.43173694610596\\
7.69000005722046	7.50425338745117\\
7.69500017166138	7.58137893676758\\
7.69999980926514	7.66214323043823\\
7.70499992370605	7.74471187591553\\
7.71000003814697	7.82776260375977\\
7.71500015258789	7.91289138793945\\
7.71999979019165	8.00293445587158\\
7.72499990463257	8.09780788421631\\
7.73000001907349	8.19417953491211\\
7.7350001335144	8.2879524230957\\
7.73999977111816	8.37826251983643\\
7.74499988555908	8.46715641021729\\
7.75	8.55465412139893\\
7.75500011444092	8.63789939880371\\
7.76000022888184	8.71637153625488\\
7.7649998664856	8.79218864440918\\
7.76999998092651	8.86519050598145\\
7.77500009536743	8.93367004394531\\
7.78000020980835	8.99709320068359\\
7.78499984741211	9.0564546585083\\
7.78999996185303	9.11348819732666\\
7.79500007629395	9.16848468780518\\
7.80000019073486	9.22120094299316\\
7.80499982833862	9.27102088928223\\
7.80999994277954	9.31736087799072\\
7.81500005722046	9.35991287231445\\
7.82000017166138	9.39842224121094\\
7.82499980926514	9.43245029449463\\
7.82999992370605	9.46156597137451\\
7.83500003814697	9.4865779876709\\
7.84000015258789	9.50824165344238\\
7.84499979019165	9.52600288391113\\
7.84999990463257	9.53941059112549\\
7.85500001907349	9.5490837097168\\
7.8600001335144	9.55521488189697\\
7.86499977111816	9.55763721466064\\
7.86999988555908	9.55624198913574\\
7.875	9.55162811279297\\
7.88000011444092	9.54352283477783\\
7.88500022888184	9.53231811523438\\
7.8899998664856	9.51895904541016\\
7.89499998092651	9.50413227081299\\
7.90000009536743	9.48824691772461\\
7.90500020980835	9.47116184234619\\
7.90999984741211	9.45313167572021\\
7.91499996185303	9.43427658081055\\
7.92000007629395	9.41386699676514\\
7.92500019073486	9.3915319442749\\
7.92999982833862	9.36660385131836\\
7.93499994277954	9.34190082550049\\
7.94000005722046	9.31822204589844\\
7.94500017166138	9.29362869262695\\
7.94999980926514	9.26972579956055\\
7.95499992370605	9.24799156188965\\
7.96000003814697	9.22654342651367\\
7.96500015258789	9.2041482925415\\
7.96999979019165	9.18255519866943\\
7.97499990463257	9.16187858581543\\
7.98000001907349	9.14285182952881\\
7.9850001335144	9.12846851348877\\
7.98999977111816	9.1194019317627\\
7.99499988555908	9.1140832901001\\
8	9.11309623718262\\
8.00500011444092	9.11609745025635\\
8.01000022888184	9.12140560150146\\
8.01500034332275	9.12970542907715\\
8.02000045776367	9.14170837402344\\
8.02499961853027	9.15748977661133\\
8.02999973297119	9.17741298675537\\
8.03499984741211	9.19964504241943\\
8.03999996185303	9.22743225097656\\
8.04500007629395	9.25996017456055\\
8.05000019073486	9.2990837097168\\
8.05500030517578	9.34487628936768\\
8.0600004196167	9.39756679534912\\
8.0649995803833	9.45721912384033\\
8.06999969482422	9.52267360687256\\
8.07499980926514	9.59366512298584\\
8.07999992370605	9.66148090362549\\
8.08500003814697	9.74791526794434\\
8.09000015258789	9.83201217651367\\
8.09500026702881	9.9196252822876\\
8.10000038146973	10.0129070281982\\
8.10499954223633	10.1116180419922\\
8.10999965667725	10.2166957855225\\
8.11499977111816	10.327166557312\\
8.11999988555908	10.4445180892944\\
8.125	10.568097114563\\
8.13000011444092	10.698540687561\\
8.13500022888184	10.8354082107544\\
8.14000034332275	10.9781293869019\\
8.14500045776367	11.1253290176392\\
8.14999961853027	11.28382396698\\
8.15499973297119	11.4446382522583\\
8.15999984741211	11.6131258010864\\
8.16499996185303	11.7898073196411\\
8.17000007629395	11.9713573455811\\
8.17500019073486	12.1572532653809\\
8.18000030517578	12.3458948135376\\
8.1850004196167	12.5384883880615\\
8.1899995803833	12.7407321929932\\
8.19499969482422	12.9550266265869\\
8.19999980926514	13.1821775436401\\
8.20499992370605	13.417064666748\\
8.21000003814697	13.6549673080444\\
8.21500015258789	13.8955450057983\\
8.22000026702881	14.1506185531616\\
8.22500038146973	14.429479598999\\
8.22999954223633	14.7230272293091\\
8.23499965667725	15.01282787323\\
8.23999977111816	15.2917585372925\\
8.24499988555908	15.5695371627808\\
8.25	15.842170715332\\
8.25500011444092	16.1603488922119\\
8.26000022888184	16.5075550079346\\
8.26500034332275	16.8594303131104\\
8.27000045776367	17.1909122467041\\
8.27499961853027	17.5959339141846\\
8.27999973297119	17.9880313873291\\
8.28499984741211	18.3856754302979\\
8.28999996185303	18.8015232086182\\
8.29500007629395	19.2356700897217\\
8.30000019073486	19.6837577819824\\
8.30500030517578	20.1526393890381\\
8.3100004196167	20.6621437072754\\
8.3149995803833	21.1719722747803\\
8.31999969482422	21.686824798584\\
8.32499980926514	22.1664524078369\\
8.32999992370605	22.728157043457\\
8.33500003814697	23.3356895446777\\
8.34000015258789	23.9589614868164\\
8.34500026702881	24.6123294830322\\
8.35000038146973	25.3060646057129\\
8.35499954223633	26.0602359771729\\
8.35999965667725	26.8635101318359\\
8.36499977111816	27.6746845245361\\
8.36999988555908	28.4826984405518\\
8.375	29.3690204620361\\
8.38000011444092	30.3049602508545\\
8.38500022888184	31.2791290283203\\
8.39000034332275	32.3046913146973\\
8.39500045776367	33.3252372741699\\
8.39999961853027	34.5181274414063\\
8.40499973297119	35.8097496032715\\
8.40999984741211	37.030143737793\\
8.41499996185303	38.4054412841797\\
8.42000007629395	39.7671127319336\\
8.42500019073486	41.1513595581055\\
8.43000030517578	42.6190528869629\\
8.4350004196167	44.1368179321289\\
8.4399995803833	45.6896209716797\\
8.44499969482422	47.2693786621094\\
8.44999980926514	48.8674278259277\\
8.45499992370605	50.5383796691895\\
8.46000003814697	52.2531433105469\\
8.46500015258789	53.9474067687988\\
8.47000026702881	55.6532249450684\\
8.47500038146973	57.3148765563965\\
8.47999954223633	58.9039649963379\\
8.48499965667725	60.400806427002\\
8.48999977111816	61.7893867492676\\
8.49499988555908	63.0835571289063\\
8.5	64.2815017700195\\
8.50500011444092	65.3331832885742\\
8.51000022888184	66.2619476318359\\
8.51500034332275	67.0551452636719\\
8.52000045776367	67.6104583740234\\
8.52499961853027	68.1274719238281\\
8.52999973297119	68.4574584960938\\
8.53499984741211	68.7582931518555\\
8.53999996185303	68.9763259887695\\
8.54500007629395	69.1196517944336\\
8.55000019073486	69.1838684082031\\
8.55500030517578	69.2209777832031\\
8.5600004196167	69.2270355224609\\
8.5649995803833	69.2104187011719\\
8.56999969482422	69.1728057861328\\
8.57499980926514	69.1022186279297\\
8.57999992370605	68.9801254272461\\
8.58500003814697	68.8034973144531\\
8.59000015258789	68.5784225463867\\
8.59500026702881	68.2941131591797\\
8.60000038146973	67.9172973632813\\
8.60499954223633	67.3681793212891\\
8.60999965667725	66.5692367553711\\
8.61499977111816	65.4834823608398\\
8.61999988555908	64.1327133178711\\
8.625	62.4678916931152\\
8.63000011444092	60.7208824157715\\
8.63500022888184	59.0027770996094\\
8.64000034332275	57.3081169128418\\
8.64500045776367	55.8658714294434\\
8.64999961853027	54.2537727355957\\
8.65499973297119	53.0526084899902\\
8.65999984741211	52.0685195922852\\
8.66499996185303	51.3312568664551\\
8.67000007629395	50.7712783813477\\
8.67500019073486	50.3259201049805\\
8.68000030517578	49.863941192627\\
8.6850004196167	49.3460998535156\\
8.6899995803833	48.7458114624023\\
8.69499969482422	47.9741630554199\\
8.69999980926514	46.9425849914551\\
8.70499992370605	45.7088775634766\\
8.71000003814697	44.3000411987305\\
8.71500015258789	42.7564239501953\\
8.72000026702881	41.1123123168945\\
8.72500038146973	39.4420318603516\\
8.72999954223633	37.8431625366211\\
8.73499965667725	36.3735122680664\\
8.73999977111816	35.0751075744629\\
8.74499988555908	34.0011024475098\\
8.75	33.1774101257324\\
8.75500011444092	32.5552139282227\\
8.76000022888184	32.0335311889648\\
8.76500034332275	31.5632305145264\\
8.77000045776367	31.127950668335\\
8.77499961853027	30.6585559844971\\
8.77999973297119	30.0697174072266\\
8.78499984741211	29.3958282470703\\
8.78999996185303	28.6274127960205\\
8.79500007629395	27.7671337127686\\
8.80000019073486	26.8476219177246\\
8.80500030517578	25.9067687988281\\
8.8100004196167	24.9893836975098\\
8.8149995803833	24.1478252410889\\
8.81999969482422	23.4268321990967\\
8.82499980926514	22.8553867340088\\
8.82999992370605	22.4420757293701\\
8.83500003814697	22.183744430542\\
8.84000015258789	22.0516872406006\\
8.84500026702881	22.0140533447266\\
8.85000038146973	22.0209980010986\\
8.85499954223633	22.0128135681152\\
8.85999965667725	21.9679832458496\\
8.86499977111816	21.8505573272705\\
8.86999988555908	21.6530151367188\\
8.875	21.3785934448242\\
8.88000011444092	21.0482120513916\\
8.88500022888184	20.6940402984619\\
8.89000034332275	20.3534355163574\\
8.89500045776367	20.0800609588623\\
8.89999961853027	19.8313407897949\\
8.90499973297119	19.683219909668\\
8.90999984741211	19.661153793335\\
8.91499996185303	19.7696075439453\\
8.92000007629395	19.9228801727295\\
8.92500019073486	20.0886192321777\\
8.93000030517578	20.2639980316162\\
8.9350004196167	20.4450702667236\\
8.9399995803833	20.5540313720703\\
8.94499969482422	20.5357265472412\\
8.94999980926514	20.4891681671143\\
8.95499992370605	20.4295501708984\\
8.96000003814697	20.3163509368896\\
8.96500015258789	20.1902484893799\\
8.97000026702881	20.0819454193115\\
8.97500038146973	20.0075378417969\\
8.97999954223633	19.9739837646484\\
8.98499965667725	19.9852123260498\\
8.98999977111816	20.0442237854004\\
8.99499988555908	20.1224308013916\\
9	20.1967906951904\\
9.00500011444092	20.2893314361572\\
9.01000022888184	20.3975315093994\\
9.01500034332275	20.5015468597412\\
9.02000045776367	20.5919399261475\\
9.02499961853027	20.6670894622803\\
9.02999973297119	20.7268543243408\\
9.03499984741211	20.7723751068115\\
9.03999996185303	20.8035640716553\\
9.04500007629395	20.826021194458\\
9.05000019073486	20.8367519378662\\
9.05500030517578	20.850700378418\\
9.0600004196167	20.8656482696533\\
9.0649995803833	20.8822193145752\\
9.06999969482422	20.9010219573975\\
9.07499980926514	20.9183788299561\\
9.07999992370605	20.934289932251\\
9.08500003814697	20.9489917755127\\
9.09000015258789	20.9639720916748\\
9.09500026702881	20.9745864868164\\
9.10000038146973	20.9735813140869\\
9.10499954223633	20.9557876586914\\
9.10999965667725	20.925500869751\\
9.11499977111816	20.8838233947754\\
9.11999988555908	20.8227996826172\\
9.125	20.7387866973877\\
9.13000011444092	20.6329364776611\\
9.13500022888184	20.5083618164063\\
9.14000034332275	20.3743629455566\\
9.14500045776367	20.2485733032227\\
9.14999961853027	20.1265506744385\\
9.15499973297119	20.0311546325684\\
9.15999984741211	19.9636898040771\\
9.16499996185303	19.907844543457\\
9.17000007629395	19.8740978240967\\
9.17500019073486	19.8565006256104\\
9.18000030517578	19.8367691040039\\
9.1850004196167	19.8071918487549\\
9.1899995803833	19.761251449585\\
9.19499969482422	19.6951866149902\\
9.19999980926514	19.6018505096436\\
9.20499992370605	19.4828853607178\\
9.21000003814697	19.3426342010498\\
9.21500015258789	19.1810665130615\\
9.22000026702881	19.0123100280762\\
9.22500038146973	18.8428802490234\\
9.22999954223633	18.6930084228516\\
9.23499965667725	18.5709209442139\\
9.23999977111816	18.4655437469482\\
9.24499988555908	18.3402099609375\\
9.25	18.1300582885742\\
9.25500011444092	17.7502593994141\\
9.26000022888184	17.1698303222656\\
9.26500034332275	16.4917106628418\\
9.27000045776367	15.8359136581421\\
9.27499961853027	15.1553678512573\\
9.27999973297119	14.6563873291016\\
9.28499984741211	14.3161602020264\\
9.28999996185303	14.1859998703003\\
9.29500007629395	14.2729234695435\\
9.30000019073486	14.5864496231079\\
9.30500030517578	15.0001277923584\\
9.3100004196167	15.6845407485962\\
9.3149995803833	16.7505016326904\\
9.31999969482422	18.149621963501\\
9.32499980926514	20.0861988067627\\
9.32999992370605	22.4349536895752\\
9.33500003814697	25.5542964935303\\
9.34000015258789	29.4095230102539\\
9.34500026702881	34.2598075866699\\
9.35000038146973	40.5982322692871\\
9.35499954223633	48.4103469848633\\
9.35999965667725	57.7158470153809\\
9.36499977111816	68.7373428344727\\
9.36999988555908	81.5149688720703\\
9.375	95.376106262207\\
9.38000011444092	110.045890808105\\
9.38500022888184	124.527854919434\\
9.39000034332275	138.550155639648\\
9.39500045776367	151.480041503906\\
9.39999961853027	162.478363037109\\
9.40499973297119	171.218994140625\\
9.40999984741211	177.509292602539\\
9.41499996185303	181.229721069336\\
9.42000007629395	182.10871887207\\
9.42500019073486	179.823623657227\\
9.43000030517578	174.178604125977\\
9.4350004196167	165.886047363281\\
9.4399995803833	154.442428588867\\
9.44499969482422	140.147598266602\\
9.44999980926514	123.822250366211\\
9.45499992370605	105.620651245117\\
9.46000003814697	86.3202514648438\\
9.46500015258789	66.4226684570313\\
9.47000026702881	48.3211936950684\\
9.47500038146973	33.5265464782715\\
9.47999954223633	24.9952411651611\\
9.48499965667725	27.3160514831543\\
9.48999977111816	38.5125160217285\\
9.49499988555908	41.1536102294922\\
9.5	46.3644218444824\\
9.50500011444092	57.9435119628906\\
9.51000022888184	72.6196670532227\\
9.51500034332275	86.3707580566406\\
9.52000045776367	96.986457824707\\
9.52499961853027	103.312942504883\\
9.52999973297119	104.937088012695\\
9.53499984741211	101.581436157227\\
9.53999996185303	93.002067565918\\
9.54500007629395	79.2852630615234\\
9.55000019073486	60.7830696105957\\
9.55500030517578	39.1994743347168\\
9.5600004196167	19.8478584289551\\
9.5649995803833	8.67628288269043\\
9.56999969482422	3.20424866676331\\
9.57499980926514	0.585750937461853\\
9.57999992370605	-0.403141766786575\\
9.58500003814697	-0.636226952075958\\
9.59000015258789	-0.668074071407318\\
9.59500026702881	-0.601570963859558\\
9.60000038146973	-0.472604840993881\\
9.60499954223633	-0.323626935482025\\
9.60999965667725	-0.229456409811974\\
9.61499977111816	-0.189447358250618\\
9.61999988555908	-0.14441029727459\\
9.625	-0.110869273543358\\
9.63000011444092	-0.0868997201323509\\
9.63500022888184	-0.0874980390071869\\
9.64000034332275	-0.0855077430605888\\
9.64500045776367	-0.102112598717213\\
9.64999961853027	-0.141099601984024\\
9.65499973297119	-0.157008782029152\\
9.65999984741211	-0.159177556633949\\
9.66499996185303	-0.178402736783028\\
9.67000007629395	-0.199377477169037\\
9.67500019073486	-0.215607956051826\\
9.68000030517578	-0.232985898852348\\
9.6850004196167	-0.250708192586899\\
9.6899995803833	-0.265989929437637\\
9.69499969482422	-0.278032809495926\\
9.69999980926514	-0.287353962659836\\
9.70499992370605	-0.297029376029968\\
9.71000003814697	-0.308991551399231\\
9.71500015258789	-0.319166034460068\\
9.72000026702881	-0.323284327983856\\
9.72500038146973	-0.326690584421158\\
9.72999954223633	-0.335674375295639\\
9.73499965667725	-0.349440544843674\\
9.73999977111816	-0.365268707275391\\
9.74499988555908	-0.378878206014633\\
9.75	-0.387035965919495\\
9.75500011444092	-0.392063558101654\\
9.76000022888184	-0.392382889986038\\
9.76500034332275	-0.386714041233063\\
9.77000045776367	-0.378391265869141\\
9.77499961853027	-0.372911512851715\\
9.77999973297119	-0.374195694923401\\
9.78499984741211	-0.38208132982254\\
9.78999996185303	-0.389105767011642\\
9.79500007629395	-0.392393201589584\\
9.80000019073486	-0.390770554542542\\
9.80500030517578	-0.386774390935898\\
9.8100004196167	-0.384567737579346\\
9.8149995803833	-0.386149257421494\\
9.81999969482422	-0.391292214393616\\
9.82499980926514	-0.396219044923782\\
9.82999992370605	-0.399135261774063\\
9.83500003814697	-0.399175524711609\\
9.84000015258789	-0.397201418876648\\
9.84500026702881	-0.394840508699417\\
9.85000038146973	-0.392790049314499\\
9.85499954223633	-0.391656190156937\\
9.85999965667725	-0.391738891601563\\
9.86499977111816	-0.39316526055336\\
9.86999988555908	-0.395730346441269\\
9.875	-0.397808402776718\\
9.88000011444092	-0.397800356149673\\
9.88500022888184	-0.395262598991394\\
9.89000034332275	-0.39189350605011\\
9.89500045776367	-0.392037183046341\\
9.89999961853027	-0.397233903408051\\
9.90499973297119	-0.406535089015961\\
9.90999984741211	-0.413759768009186\\
9.91499996185303	-0.412101924419403\\
9.92000007629395	-0.399295061826706\\
9.92500019073486	-0.380255520343781\\
9.93000030517578	-0.369642555713654\\
9.9350004196167	-0.372257322072983\\
9.9399995803833	-0.387704700231552\\
9.94499969482422	-0.405271917581558\\
9.94999980926514	-0.416944473981857\\
9.95499992370605	-0.422976553440094\\
9.96000003814697	-0.424539983272552\\
9.96500015258789	-0.422263115644455\\
9.97000026702881	-0.416427671909332\\
9.97500038146973	-0.407052904367447\\
9.97999954223633	-0.401521891355515\\
9.98499965667725	-0.398633509874344\\
9.98999977111816	-0.398021161556244\\
9.99499988555908	-0.397272825241089\\
10	-0.397420465946198\\
};
\addlegendentry{Control}

\addplot [color=black, dashed, line width=2.0pt]
  table[row sep=crcr]{%
0.0949999988079071	2.42140296117631\\
0.100000001490116	2.0444961292241\\
0.104999996721745	1.65344370483741\\
0.109999999403954	1.27317240407065\\
0.115000002086163	0.914757632254684\\
0.119999997317791	-43.7094322932182\\
0.125	-112.324780590789\\
0.129999995231628	-123.170382910354\\
0.135000005364418	-123.03782956715\\
0.140000000596046	-117.116433047812\\
0.144999995827675	-108.342185636906\\
0.150000005960464	-98.2992077140649\\
0.155000001192093	-88.4381174220212\\
0.159999996423721	-82.5251208808467\\
0.165000006556511	-77.1766728932876\\
0.170000001788139	-70.5948365597196\\
0.174999997019768	-63.4346459480265\\
0.180000007152557	-56.0237486232768\\
0.185000002384186	-48.9677318646687\\
0.189999997615814	-41.9207393947725\\
0.194999992847443	-9.12027548253536\\
0.200000002980232	149.870100814849\\
0.204999998211861	252.007701553404\\
0.209999993443489	282.087768455967\\
0.215000003576279	252.767896549776\\
0.219999998807907	193.658821191639\\
0.224999994039536	116.106348931789\\
0.230000004172325	30.8472526967525\\
0.234999999403954	-47.6924924850464\\
0.239999994635582	-107.798231601715\\
0.245000004768372	-144.634235262871\\
0.25	-154.201780796051\\
0.254999995231628	-139.053717970848\\
0.259999990463257	-109.468836307526\\
0.264999985694885	-66.7106179594994\\
0.270000010728836	-17.2152684926987\\
0.275000005960464	28.6249681711197\\
0.280000001192093	63.6410534977913\\
0.284999996423721	84.8853171765804\\
0.28999999165535	90.9185044467449\\
0.294999986886978	80.3544898629189\\
0.300000011920929	62.5839357972145\\
0.305000007152557	42.4832604527473\\
0.310000002384186	21.2318732142448\\
0.314999997615814	0.815957903862\\
0.319999992847443	-16.1169533729553\\
0.324999988079071	-27.6590649485588\\
0.330000013113022	-35.7166925668716\\
0.33500000834465	-37.591690659523\\
0.340000003576279	-30.4383845925331\\
0.344999998807907	-16.2930896282196\\
0.349999994039536	-0.00677949190139771\\
0.354999989271164	15.0923603773117\\
0.360000014305115	26.7279219031334\\
0.365000009536743	33.2899678349495\\
0.370000004768372	36.1706037521362\\
0.375	34.568922996521\\
0.379999995231628	28.3847386837006\\
0.384999990463257	18.6495280861855\\
0.389999985694885	6.95088338851929\\
0.395000010728836	-4.70474094152451\\
0.400000005960464	-14.7047437429428\\
0.405000001192093	-22.0365894436836\\
0.409999996423721	-24.8984689116478\\
0.41499999165535	-23.5380859971046\\
0.419999986886978	-18.5485727787018\\
0.425000011920929	-10.5670009255409\\
0.430000007152557	-1.24182969331741\\
0.435000002384186	7.13203296065331\\
0.439999997615814	13.5284509956837\\
0.444999992847443	17.3131544291973\\
0.449999988079071	17.8178732097149\\
0.455000013113022	14.9449652731419\\
0.46000000834465	9.33958202600479\\
0.465000003576279	1.92606097459793\\
0.469999998807907	-5.74521264433861\\
0.474999994039536	-12.6944379210472\\
0.479999989271164	-17.7163471281528\\
0.485000014305115	-20.0963931679726\\
0.490000009536743	-19.5946452319622\\
0.495000004768372	-16.1337162852287\\
0.5	-10.3332686424255\\
0.504999995231628	-4.05897858738899\\
0.509999990463257	1.56807665526867\\
0.514999985694885	5.68708418309689\\
0.519999980926514	7.84694877266884\\
0.524999976158142	7.83816969394684\\
0.529999971389771	5.19429738819599\\
0.535000026226044	2.54460909962654\\
0.540000021457672	-0.685934036970139\\
0.545000016689301	-4.11540430784225\\
0.550000011920929	-7.00301140546799\\
0.555000007152557	-8.84424903988838\\
0.560000002384186	-9.3395811021328\\
0.564999997615814	-8.734219789505\\
0.569999992847443	-7.37800800800323\\
0.574999988079071	-5.56445747613907\\
0.579999983310699	-3.6444263458252\\
0.584999978542328	-1.91560918092728\\
0.589999973773956	-0.582508444786072\\
0.595000028610229	0.243278980255127\\
0.600000023841858	0.548720419406891\\
0.605000019073486	0.364620804786682\\
0.610000014305115	-0.147062540054321\\
0.615000009536743	-0.842725515365601\\
0.620000004768372	-1.5681060552597\\
0.625	-2.23378801345825\\
0.629999995231628	-2.68905389308929\\
0.634999990463257	-2.96321082115173\\
0.639999985694885	-2.96159517765045\\
0.644999980926514	-2.72817635536194\\
0.649999976158142	-2.31878888607025\\
0.654999971389771	-1.79908382892609\\
0.660000026226044	-1.2355785369873\\
0.665000021457672	-0.692900776863098\\
0.670000016689301	-0.221860527992249\\
0.675000011920929	0.139987468719482\\
0.680000007152557	0.378649711608887\\
0.685000002384186	0.502459645271301\\
0.689999997615814	0.537814974784851\\
0.694999992847443	0.527533650398254\\
0.699999988079071	0.506013870239258\\
0.704999983310699	0.491771578788757\\
0.709999978542328	0.511655926704407\\
0.714999973773956	0.587408781051636\\
0.720000028610229	0.693294763565063\\
0.725000023841858	0.894605994224548\\
0.730000019073486	1.02811658382416\\
0.735000014305115	1.251833319664\\
0.740000009536743	1.4229291677475\\
0.745000004768372	1.60334599018097\\
0.75	1.73467612266541\\
0.754999995231628	1.85267812013626\\
0.759999990463257	1.90575069189072\\
0.764999985694885	1.86844849586487\\
0.769999980926514	1.78263574838638\\
0.774999976158142	1.74036061763763\\
0.779999971389771	1.7488951086998\\
0.785000026226044	1.76699072122574\\
0.790000021457672	1.76752549409866\\
0.795000016689301	1.74784368276596\\
0.800000011920929	1.71308940649033\\
0.805000007152557	1.6682935655117\\
0.810000002384186	1.61637768149376\\
0.814999997615814	1.5613216906786\\
0.819999992847443	1.50077606737614\\
0.824999988079071	1.4371619746089\\
0.829999983310699	1.36832127347589\\
0.834999978542328	1.29245575703681\\
0.839999973773956	1.19906793534756\\
0.845000028610229	1.08819128572941\\
0.850000023841858	0.966853976249695\\
0.855000019073486	0.836013376712799\\
0.860000014305115	0.694391459226608\\
0.865000009536743	0.54865163564682\\
0.870000004768372	0.410459131002426\\
0.875	0.28434494137764\\
0.879999995231628	0.170927107334137\\
0.884999990463257	0.0685217380523682\\
0.889999985694885	-0.0279619097709656\\
0.894999980926514	-0.118727803230286\\
0.899999976158142	-0.203508734703064\\
0.904999971389771	-0.28107362985611\\
0.910000026226044	-0.35408890247345\\
0.915000021457672	-0.424038350582123\\
0.920000016689301	-0.492837905883789\\
0.925000011920929	-0.562176167964935\\
0.930000007152557	-0.634541273117065\\
0.935000002384186	-0.714413642883301\\
0.939999997615814	-0.803653478622437\\
0.944999992847443	-0.896448254585266\\
0.949999988079071	-0.976009964942932\\
0.954999983310699	-1.04639410972595\\
0.959999978542328	-1.10440325737\\
0.964999973773956	-1.14684748649597\\
0.970000028610229	-1.17022514343262\\
0.975000023841858	-1.18137502670288\\
0.980000019073486	-1.18147706985474\\
0.985000014305115	-1.17050623893738\\
0.990000009536743	-1.16211438179016\\
0.995000004768372	-1.15587210655212\\
1	-1.1597375869751\\
1.00499999523163	-1.18103003501892\\
1.00999999046326	-1.22009122371674\\
1.01499998569489	-1.22765576839447\\
1.01999998092651	-1.20952832698822\\
1.02499997615814	-1.18403673171997\\
1.02999997138977	-1.1516420841217\\
1.0349999666214	-1.11942505836487\\
1.03999996185303	-1.08861398696899\\
1.04499995708466	-1.05772161483765\\
1.04999995231628	-1.02668762207031\\
1.05499994754791	-0.993675231933594\\
1.05999994277954	-0.957669496536255\\
1.06500005722046	-0.918111324310303\\
1.07000005245209	-0.873643159866333\\
1.07500004768372	-0.825416326522827\\
1.08000004291534	-0.776174783706665\\
1.08500003814697	-0.726063966751099\\
1.0900000333786	-0.676214694976807\\
1.09500002861023	-0.627707719802856\\
1.10000002384186	-0.581660270690918\\
1.10500001907349	-0.537456274032593\\
1.11000001430511	-0.491320371627808\\
1.11500000953674	-0.445846796035767\\
1.12000000476837	-0.401285886764526\\
1.125	-0.355082750320435\\
1.12999999523163	-0.307239532470703\\
1.13499999046326	-0.258074760437012\\
1.13999998569489	-0.207897663116455\\
1.14499998092651	-0.157958984375\\
1.14999997615814	-0.107980251312256\\
1.15499997138977	-0.0582470893859863\\
1.1599999666214	-0.00977468490600586\\
1.16499996185303	0.0367789268493652\\
1.16999995708466	0.0810251235961914\\
1.17499995231628	0.122840404510498\\
1.17999994754791	0.16563606262207\\
1.18499994277954	0.207706451416016\\
1.19000005722046	0.24913501739502\\
1.19500005245209	0.287163257598877\\
1.20000004768372	0.319168567657471\\
1.20500004291534	0.34434700012207\\
1.21000003814697	0.361253261566162\\
1.2150000333786	0.378564357757568\\
1.22000002861023	0.393496513366699\\
1.22500002384186	0.407667636871338\\
1.23000001907349	0.418486595153809\\
1.23500001430511	0.422338962554932\\
1.24000000953674	0.420601367950439\\
1.24500000476837	0.412259101867676\\
1.25	0.410991191864014\\
1.25499999523163	0.414987087249756\\
1.25999999046326	0.428375720977783\\
1.26499998569489	0.446660995483398\\
1.26999998092651	0.460062980651855\\
1.27499997615814	0.475192546844482\\
1.27999997138977	0.488712310791016\\
1.2849999666214	0.488628387451172\\
1.28999996185303	0.477060794830322\\
1.29499995708466	0.452243328094482\\
1.29999995231628	0.426882266998291\\
1.30499994754791	0.39525032043457\\
1.30999994277954	0.357760906219482\\
1.31500005722046	0.323147773742676\\
1.32000005245209	0.291556358337402\\
1.32500004768372	0.26470422744751\\
1.33000004291534	0.241031169891357\\
1.33500003814697	0.221286296844482\\
1.3400000333786	0.207462310791016\\
1.34500002861023	0.199287891387939\\
1.35000002384186	0.19572114944458\\
1.35500001907349	0.197071552276611\\
1.36000001430511	0.199856758117676\\
1.36500000953674	0.193292140960693\\
1.37000000476837	0.174482822418213\\
1.375	0.139679431915283\\
1.37999999523163	0.109077453613281\\
1.38499999046326	0.072265625\\
1.38999998569489	0.0290608406066895\\
1.39499998092651	-0.00333166122436523\\
1.39999997615814	-0.0276665687561035\\
1.40499997138977	-0.0417971611022949\\
1.4099999666214	-0.0527853965759277\\
1.41499996185303	-0.056126594543457\\
1.41999995708466	-0.0522050857543945\\
1.42499995231628	-0.0463614463806152\\
1.42999994754791	-0.0372390747070313\\
1.43499994277954	-0.0255861282348633\\
1.44000005722046	-0.0121607780456543\\
1.44500005245209	0.00225591659545898\\
1.45000004768372	0.0162549018859863\\
1.45500004291534	0.028174877166748\\
1.46000003814697	0.0371909141540527\\
1.4650000333786	0.0387735366821289\\
1.47000002861023	0.0274624824523926\\
1.47500002384186	-0.000165939331054688\\
1.48000001907349	-0.0213255882263184\\
1.48500001430511	-0.0208330154418945\\
1.49000000953674	0.01214599609375\\
1.49500000476837	0.0268959999084473\\
1.5	0.0310544967651367\\
1.50499999523163	0.0272951126098633\\
1.50999999046326	0.0192446708679199\\
1.51499998569489	0.00271177291870117\\
1.51999998092651	-0.00861501693725586\\
1.52499997615814	-0.00752687454223633\\
1.52999997138977	0.004913330078125\\
1.5349999666214	0.00888681411743164\\
1.53999996185303	0.00952482223510742\\
1.54499995708466	0.0116586685180664\\
1.54999995231628	0.0218749046325684\\
1.55499994754791	0.0388355255126953\\
1.55999994277954	0.0589399337768555\\
1.56500005722046	0.0839800834655762\\
1.57000005245209	3.68389129638672\\
1.57500004768372	2.23132228851318\\
1.58000004291534	1.34562158584595\\
1.58500003814697	0.758204460144043\\
1.5900000333786	0.323260307312012\\
1.59500002861023	0.0211539268493652\\
1.60000002384186	-0.156994342803955\\
1.60500001907349	-0.228591918945313\\
1.61000001430511	-0.197527885437012\\
1.61500000953674	-0.0421242713928223\\
1.62000000476837	0.166327953338623\\
1.625	0.438634872436523\\
1.62999999523163	0.728136539459229\\
1.63499999046326	0.937882900238037\\
1.63999998569489	1.08333873748779\\
1.64499998092651	1.08419275283813\\
1.64999997615814	1.0419659614563\\
1.65499997138977	1.00541877746582\\
1.6599999666214	0.942756175994873\\
1.66499996185303	0.872861385345459\\
1.66999995708466	0.804239749908447\\
1.67499995231628	0.736320018768311\\
1.67999994754791	0.676591873168945\\
1.68499994277954	0.658730983734131\\
1.69000005722046	0.677008152008057\\
1.69500005245209	0.704109191894531\\
1.70000004768372	0.738269805908203\\
1.70500004291534	0.776620864868164\\
1.71000003814697	0.804420709609985\\
1.7150000333786	0.829968214035034\\
1.72000002861023	0.854914903640747\\
1.72500002384186	0.877328395843506\\
1.73000001907349	0.897476434707642\\
1.73500001430511	0.908495426177979\\
1.74000000953674	0.911363363265991\\
1.74500000476837	0.907158136367798\\
1.75	0.89603590965271\\
1.75499999523163	0.87797737121582\\
1.75999999046326	0.856823682785034\\
1.76499998569489	0.831873893737793\\
1.76999998092651	0.804518938064575\\
1.77499997615814	0.780245542526245\\
1.77999997138977	0.759442567825317\\
1.7849999666214	0.741706132888794\\
1.78999996185303	0.72834324836731\\
1.79499995708466	0.719430685043335\\
1.79999995231628	0.712019324302673\\
1.80499994754791	0.707132935523987\\
1.80999994277954	0.701401591300964\\
1.81500005722046	0.693789839744568\\
1.82000005245209	0.683499574661255\\
1.82500004768372	0.669661998748779\\
1.83000004291534	0.651058614253998\\
1.83500003814697	0.628148257732391\\
1.8400000333786	0.601007699966431\\
1.84500002861023	0.56995815038681\\
1.85000002384186	0.540683925151825\\
1.85500001907349	0.510327249765396\\
1.86000001430511	0.479332119226456\\
1.86500000953674	0.44871923327446\\
1.87000000476837	0.417901337146759\\
1.875	0.389287263154984\\
1.87999999523163	0.364801585674286\\
1.88499999046326	0.342898190021515\\
1.88999998569489	0.323310077190399\\
1.89499998092651	0.297908067703247\\
1.89999997615814	0.271249413490295\\
1.90499997138977	0.243402123451233\\
1.9099999666214	0.212171077728271\\
1.91499996185303	0.178399085998535\\
1.91999995708466	0.142187476158142\\
1.92499995231628	0.105482339859009\\
1.92999994754791	0.0681405067443848\\
1.93499994277954	0.0296809673309326\\
1.94000005722046	-0.00913333892822266\\
1.94500005245209	-0.0481171607971191\\
1.95000004768372	-0.0875446796417236\\
1.95500004291534	-0.124898433685303\\
1.96000003814697	-0.156066656112671\\
1.9650000333786	-0.181480646133423\\
1.97000002861023	-0.204932689666748\\
1.97500002384186	-0.240138053894043\\
1.98000001907349	-0.283454895019531\\
1.98500001430511	-0.336214542388916\\
1.99000000953674	-0.392317295074463\\
1.99500000476837	-0.455850601196289\\
2	-0.527021884918213\\
2.00500011444092	-0.592230796813965\\
2.00999999046326	-0.654590129852295\\
2.01500010490417	-0.711995124816895\\
2.01999998092651	-0.748790264129639\\
2.02500009536743	-0.763010501861572\\
2.02999997138977	-0.752107620239258\\
2.03500008583069	-0.741212844848633\\
2.03999996185303	-0.721396923065186\\
2.04500007629395	-0.702662467956543\\
2.04999995231628	-0.697279930114746\\
2.0550000667572	-0.705598831176758\\
2.05999994277954	-0.723671913146973\\
2.06500005722046	-0.757299423217773\\
2.0699999332428	-0.811579704284668\\
2.07500004768372	-0.888809204101563\\
2.07999992370605	-0.990077018737793\\
2.08500003814697	-1.11579895019531\\
2.08999991416931	-1.24906158447266\\
2.09500002861023	-1.38891315460205\\
2.09999990463257	-1.53057384490967\\
2.10500001907349	-1.67002964019775\\
2.10999989509583	-1.81629657745361\\
2.11500000953674	-1.97003269195557\\
2.11999988555908	-2.06518173217773\\
2.125	-2.06762599945068\\
2.13000011444092	-1.97577667236328\\
2.13499999046326	-1.76670932769775\\
2.14000010490417	-1.52560234069824\\
2.14499998092651	-1.30643558502197\\
2.15000009536743	-1.20281505584717\\
2.15499997138977	-1.13497352600098\\
2.16000008583069	-0.920764923095703\\
2.16499996185303	-0.89216423034668\\
2.17000007629395	-1.017005443573\\
2.17499995231628	-1.03248596191406\\
2.1800000667572	-1.07280886173248\\
2.18499994277954	-1.1261465549469\\
2.19000005722046	-1.18647575378418\\
2.1949999332428	-1.14458465576172\\
2.20000004768372	-0.97826099395752\\
2.20499992370605	-0.747465133666992\\
2.21000003814697	-0.50590991973877\\
2.21499991416931	-0.271903991699219\\
2.22000002861023	0.0477075576782227\\
2.22499990463257	0.330559730529785\\
2.23000001907349	0.557920455932617\\
2.23499989509583	0.739049911499023\\
2.24000000953674	0.850364685058594\\
2.24499988555908	0.899394989013672\\
2.25	0.90913200378418\\
2.25500011444092	0.879728317260742\\
2.25999999046326	0.844079971313477\\
2.26500010490417	0.845844268798828\\
2.26999998092651	0.925134658813477\\
2.27500009536743	1.08535575866699\\
2.27999997138977	1.29531478881836\\
2.28500008583069	1.4732666015625\\
2.28999996185303	1.59890747070313\\
2.29500007629395	1.70081520080566\\
2.29999995231628	1.95803642272949\\
2.3050000667572	2.21002960205078\\
2.30999994277954	2.42663764953613\\
2.31500005722046	2.61535835266113\\
2.3199999332428	2.83462524414063\\
2.32500004768372	3.06576156616211\\
2.32999992370605	3.25847434997559\\
2.33500003814697	3.4154224395752\\
2.33999991416931	3.30189895629883\\
2.34500002861023	2.90591430664063\\
2.34999990463257	3.0317268371582\\
2.35500001907349	3.21782112121582\\
2.35999989509583	3.24929046630859\\
2.36500000953674	3.35573577880859\\
2.36999988555908	3.47579002380371\\
2.375	3.49889373779297\\
2.38000011444092	3.5922966003418\\
2.38499999046326	3.68006896972656\\
2.39000010490417	3.73517799377441\\
2.39499998092651	3.72791433334351\\
2.40000009536743	3.65719556808472\\
2.40499997138977	3.48786449432373\\
2.41000008583069	3.21054458618164\\
2.41499996185303	2.82480049133301\\
2.42000007629395	2.34289360046387\\
2.42499995231628	1.80247497558594\\
2.4300000667572	1.21417045593262\\
2.43499994277954	0.602694511413574\\
2.44000005722046	-0.01666259765625\\
2.4449999332428	-0.666476249694824\\
2.45000004768372	-1.30529642105103\\
2.45499992370605	-1.87166452407837\\
2.46000003814697	-2.36615753173828\\
2.46499991416931	-2.77886772155762\\
2.47000002861023	-3.08443021774292\\
2.47499990463257	-3.30529028177261\\
2.48000001907349	-3.4650651216507\\
2.48499989509583	-3.57516956329346\\
2.49000000953674	-3.65771198272705\\
2.49499988555908	-3.70162057876587\\
2.5	-3.70015525817871\\
2.50500011444092	-3.70737171173096\\
2.50999999046326	-3.83599853515625\\
2.51500010490417	-4.14728927612305\\
2.51999998092651	-4.50792694091797\\
2.52500009536743	-4.79704093933105\\
2.52999997138977	-5.05979347229004\\
2.53500008583069	-5.30242729187012\\
2.53999996185303	-5.4587459564209\\
2.54500007629395	-5.62848091125488\\
2.54999995231628	-5.72932624816895\\
2.5550000667572	-5.78349494934082\\
2.55999994277954	-5.80016708374023\\
2.56500005722046	-5.80961036682129\\
2.5699999332428	-5.83242416381836\\
2.57500004768372	-5.89159202575684\\
2.57999992370605	-6.00176906585693\\
2.58500003814697	-6.16534471511841\\
2.58999991416931	-6.41287469863892\\
2.59500002861023	-6.65946355462074\\
2.59999990463257	-6.87172150611877\\
2.60500001907349	-7.04556798934937\\
2.60999989509583	-7.13887310028076\\
2.61500000953674	-7.12348937988281\\
2.61999988555908	-6.97713279724121\\
2.625	-6.69077110290527\\
2.63000011444092	-6.28514099121094\\
2.63499999046326	-5.8057975769043\\
2.64000010490417	-5.17611312866211\\
2.64499998092651	-4.44412612915039\\
2.65000009536743	-3.77899742126465\\
2.65499997138977	-3.15755271911621\\
2.66000008583069	-2.561767578125\\
2.66499996185303	-2.04315185546875\\
2.67000007629395	-1.65237998962402\\
2.67499995231628	-1.26304817199707\\
2.6800000667572	-0.875898361206055\\
2.68499994277954	-0.543292999267578\\
2.69000005722046	-0.166996002197266\\
2.6949999332428	0.275180816650391\\
2.70000004768372	0.788845062255859\\
2.70499992370605	1.40604972839355\\
2.71000003814697	2.09829330444336\\
2.71499991416931	2.80020809173584\\
2.72000002861023	3.52865314483643\\
2.72499990463257	4.24597501754761\\
2.73000001907349	4.91373491287231\\
2.73499989509583	5.51150453090668\\
2.74000000953674	6.01663845777512\\
2.74499988555908	6.41497111320496\\
2.75	6.74248266220093\\
2.75500011444092	7.03109931945801\\
2.75999999046326	7.32913875579834\\
2.76500010490417	7.61466026306152\\
2.76999998092651	7.87722682952881\\
2.77500009536743	8.1302375793457\\
2.77999997138977	8.44176292419434\\
2.78500008583069	8.84730243682861\\
2.78999996185303	9.31101560592651\\
2.79500007629395	9.78265941143036\\
2.79999995231628	10.2031998634338\\
2.8050000667572	10.5172567367554\\
2.80999994277954	10.6653089523315\\
2.81500005722046	10.5443906784058\\
2.8199999332428	10.1202068328857\\
2.82500004768372	9.36167144775391\\
2.82999992370605	8.2509880065918\\
2.83500003814697	6.84503173828125\\
2.83999991416931	5.21757507324219\\
2.84500002861023	3.43689346313477\\
2.84999990463257	1.57417297363281\\
2.85500001907349	-0.300342559814453\\
2.85999989509583	-2.18221282958984\\
2.86500000953674	-4.10723114013672\\
2.86999988555908	-6.04072570800781\\
2.875	-8.12285232543945\\
2.88000011444092	-10.3734703063965\\
2.88499999046326	-12.8175735473633\\
2.89000010490417	-15.4463195800781\\
2.89499998092651	-17.9450359344482\\
2.90000009536743	-19.8948831558228\\
2.90499997138977	-21.233916759491\\
2.91000008583069	-21.7645921707153\\
2.91499996185303	-21.3615455627441\\
2.92000007629395	-20.1367168426514\\
2.92499995231628	-18.2686462402344\\
2.9300000667572	-16.0014877319336\\
2.93499994277954	-13.5304336547852\\
2.94000005722046	-11.1787490844727\\
2.9449999332428	-9.01774978637695\\
2.95000004768372	-7.27095794677734\\
2.95499992370605	-5.91158676147461\\
2.96000003814697	-5.18611526489258\\
2.96499991416931	-4.61782074719667\\
2.97000002861023	-4.10333919525146\\
2.97499990463257	-3.6835241317749\\
2.98000001907349	-3.07557487487793\\
2.98499989509583	-2.60763740539551\\
2.99000000953674	-3.83297348022461\\
2.99499988555908	-3.56430816650391\\
3	-3.1253547668457\\
3.00500011444092	-2.45606231689453\\
3.00999999046326	-1.55691528320313\\
3.01500010490417	-0.471363067626953\\
3.01999998092651	0.693447113037109\\
3.02500009536743	1.82150268554688\\
3.02999997138977	2.75882720947266\\
3.03500008583069	3.36050415039063\\
3.03999996185303	3.62835693359375\\
3.04500007629395	3.52888107299805\\
3.04999995231628	3.16950988769531\\
3.0550000667572	2.73284530639648\\
3.05999994277954	2.30459403991699\\
3.06500005722046	2.0748348236084\\
3.0699999332428	2.20971488952637\\
3.07500004768372	2.63185501098633\\
3.07999992370605	3.35914134979248\\
3.08500003814697	4.33614301681519\\
3.08999991416931	5.40090131759644\\
3.09500002861023	6.43497824668884\\
3.09999990463257	7.38655519485474\\
3.10500001907349	8.20941305160522\\
3.10999989509583	8.8928918838501\\
3.11500000953674	9.39717102050781\\
3.11999988555908	9.77307796478271\\
3.125	10.1911334991455\\
3.13000011444092	10.8597602844238\\
3.13499999046326	12.1283416748047\\
3.14000010490417	14.0039129257202\\
3.14499998092651	16.1987426877022\\
3.15000009536743	18.2089505195618\\
3.15499997138977	19.364447593689\\
3.16000008583069	20.1749305725098\\
3.16499996185303	20.1208190917969\\
3.17000007629395	19.063606262207\\
3.17499995231628	16.8240566253662\\
3.1800000667572	13.4474067687988\\
3.18499994277954	9.37517166137695\\
3.19000005722046	5.05586624145508\\
3.1949999332428	0.342166900634766\\
3.20000004768372	-4.45927047729492\\
3.20499992370605	-9.42422103881836\\
3.21000003814697	-13.8907890319824\\
3.21499991416931	-17.9282913208008\\
3.22000002861023	-21.2418022155762\\
3.22499990463257	-23.7223587036133\\
3.23000001907349	-25.6725234985352\\
3.23499989509583	-26.5725727081299\\
3.24000000953674	-26.9062261581421\\
3.24499988555908	-26.2904391288757\\
3.25	-25.0010261535645\\
3.25500011444092	-23.095178604126\\
3.25999999046326	-20.888484954834\\
3.26500010490417	-18.5621643066406\\
3.26999998092651	-16.3722267150879\\
3.27500009536743	-14.5094451904297\\
3.27999997138977	-13.0847053527832\\
3.28500008583069	-12.1123819351196\\
3.28999996185303	-11.4564385414124\\
3.29500007629395	-11.0442858040333\\
3.29999995231628	-10.6746945381165\\
3.3050000667572	-10.2308883666992\\
3.30999994277954	-9.62346267700195\\
3.31500005722046	-8.74794006347656\\
3.3199999332428	-8.04729461669922\\
3.32500004768372	-7.29548263549805\\
3.32999992370605	-5.96433258056641\\
3.33500003814697	-4.29378509521484\\
3.33999991416931	-2.44507217407227\\
3.34500002861023	-0.581764221191406\\
3.34999990463257	1.18997192382813\\
3.35500001907349	2.65094757080078\\
3.35999989509583	3.83127593994141\\
3.36500000953674	4.65168762207031\\
3.36999988555908	5.11528015136719\\
3.375	5.44145202636719\\
3.38000011444092	5.64337158203125\\
3.38499999046326	5.87194061279297\\
3.39000010490417	6.22048950195313\\
3.39499998092651	6.74193572998047\\
3.40000009536743	7.50613403320313\\
3.40499997138977	8.42556571960449\\
3.41000008583069	9.45379638671875\\
3.41499996185303	10.5457911491394\\
3.42000007629395	11.5945575237274\\
3.42499995231628	12.5294818878174\\
3.4300000667572	13.3437967300415\\
3.43499994277954	14.0405521392822\\
3.44000005722046	14.6832237243652\\
3.4449999332428	15.3486042022705\\
3.45000004768372	16.0547046661377\\
3.45499992370605	16.9432792663574\\
3.46000003814697	18.4832801818848\\
3.46499991416931	20.9449424743652\\
3.47000002861023	23.8343238830566\\
3.47499990463257	26.6652669906616\\
3.48000001907349	29.0171527862549\\
3.48499989509583	30.468334197998\\
3.49000000953674	29.5704917907715\\
3.49499988555908	27.7876281738281\\
3.5	24.7851715087891\\
3.50500011444092	21.2367515563965\\
3.50999999046326	16.2316551208496\\
3.51500010490417	10.5272903442383\\
3.51999998092651	3.55055236816406\\
3.52500009536743	-5.04254913330078\\
3.52999997138977	-14.8537292480469\\
3.53500008583069	-26.5630722045898\\
3.53999996185303	-40.1033630371094\\
3.54500007629395	-54.6526393890381\\
3.54999995231628	-67.7010622024536\\
3.5550000667572	-74.3145484924316\\
3.55999994277954	-74.2877464294434\\
3.56500005722046	-67.8099594116211\\
3.5699999332428	-56.4465179443359\\
3.57500004768372	-42.1509017944336\\
3.57999992370605	-28.0594367980957\\
3.58500003814697	-14.2724514007568\\
3.58999991416931	-2.14948463439941\\
3.59500002861023	6.19817543029785\\
3.59999990463257	8.64661645889282\\
3.60500001907349	7.90643453598022\\
3.60999989509583	4.73290061950684\\
3.61500000953674	0.952651977539063\\
3.61999988555908	-3.30937576293945\\
3.625	-7.44098281860352\\
3.63000011444092	-10.5881309509277\\
3.63499999046326	-12.6416931152344\\
3.64000010490417	-13.9463424682617\\
3.64499998092651	-11.6763153076172\\
3.65000009536743	-7.03762817382813\\
3.65499997138977	-0.655227661132813\\
3.66000008583069	5.63877868652344\\
3.66499996185303	10.8415985107422\\
3.67000007629395	14.6605987548828\\
3.67499995231628	16.3701171875\\
3.6800000667572	15.9268379211426\\
3.68499994277954	13.5585861206055\\
3.69000005722046	9.95706558227539\\
3.6949999332428	6.13347244262695\\
3.70000004768372	2.8691349029541\\
3.70499992370605	0.891189575195313\\
3.71000003814697	0.765008926391602\\
3.71499991416931	2.29597079753876\\
3.72000002861023	5.47070932388306\\
3.72499990463257	9.57439613342285\\
3.73000001907349	14.0259265899658\\
3.73499989509583	18.0897560119629\\
3.74000000953674	21.3607120513916\\
3.74499988555908	23.7119579315186\\
3.75	25.2381992340088\\
3.75500011444092	27.2647571563721\\
3.75999999046326	30.1252434253693\\
3.76500010490417	34.0051078796387\\
3.76999998092651	38.1770706176758\\
3.77500009536743	41.1061115264893\\
3.77999997138977	42.0768146514893\\
3.78500008583069	41.3755989074707\\
3.78999996185303	38.5826683044434\\
3.79500007629395	34.1948547363281\\
3.79999995231628	27.7736206054688\\
3.8050000667572	19.0727729797363\\
3.80999994277954	8.11155700683594\\
3.81500005722046	-5.34535598754883\\
3.8199999332428	-21.1588973999023\\
3.82500004768372	-38.9385223388672\\
3.82999992370605	-57.9831657409668\\
3.83500003814697	-76.1643657684326\\
3.83999991416931	-89.6059184074402\\
3.84500002861023	-93.665620803833\\
3.84999990463257	-88.160083770752\\
3.85500001907349	-74.9393081665039\\
3.85999989509583	-58.1975898742676\\
3.86500000953674	-39.7770233154297\\
3.86999988555908	-20.1781959533691\\
3.875	-3.30749320983887\\
3.88000011444092	8.4836483001709\\
3.88499999046326	12.3470067977905\\
3.89000010490417	10.1850786209106\\
3.89499998092651	4.86728668212891\\
3.90000009536743	-1.28978729248047\\
3.90499997138977	-7.55210876464844\\
3.91000008583069	-13.1925621032715\\
3.91499996185303	-17.0655174255371\\
3.92000007629395	-18.8676528930664\\
3.92499995231628	-19.693244934082\\
3.9300000667572	-15.8657608032227\\
3.93499994277954	-8.42948913574219\\
3.94000005722046	0.169570922851563\\
3.9449999332428	8.19140625\\
3.95000004768372	14.4943389892578\\
3.95499992370605	19.0080337524414\\
3.96000003814697	21.0233612060547\\
3.96499991416931	20.3848342895508\\
3.97000002861023	17.3775291442871\\
3.97499990463257	12.8767280578613\\
3.98000001907349	8.26663970947266\\
3.98499989509583	4.46706771850586\\
3.99000000953674	2.28827571868896\\
3.99499988555908	2.17781519889832\\
4	4.20523643493652\\
4.00500011444092	8.22319507598877\\
4.01000022888184	13.3550205230713\\
4.0149998664856	18.6788291931152\\
4.01999998092651	23.791862487793\\
4.02500009536743	27.97119140625\\
4.03000020980835	31.3188362121582\\
4.03499984741211	34.8724346160889\\
4.03999996185303	39.0183319449425\\
4.04500007629395	43.8243017196655\\
4.05000019073486	48.7350215911865\\
4.05499982833862	53.0719404220581\\
4.05999994277954	55.8575706481934\\
4.06500005722046	54.9784488677979\\
4.07000017166138	52.1957168579102\\
4.07499980926514	48.1612720489502\\
4.07999992370605	42.1501421928406\\
4.08500003814697	33.1676483154297\\
4.09000015258789	19.8840255737305\\
4.09499979019165	0.400382995605469\\
4.09999990463257	-28.1942329406738\\
4.10500001907349	-70.8958148956299\\
4.1100001335144	-123.493404865265\\
4.11499977111816	-165.471616841614\\
4.11999988555908	-181.730566902301\\
4.125	-179.901673316956\\
4.13000011444092	-144.998245239258\\
4.13500022888184	-107.802764892578\\
4.1399998664856	-58.5017776489258\\
4.14499998092651	-8.39140892028809\\
4.15000009536743	32.5401554107666\\
4.15500020980835	59.5531129837036\\
4.15999984741211	68.828209400177\\
4.16499996185303	55.7046775817871\\
4.17000007629395	38.3441276550293\\
4.17500019073486	18.6627502441406\\
4.17999982833862	-2.38710784912109\\
4.18499994277954	-21.3372268676758\\
4.19000005722046	-34.9347152709961\\
4.19500017166138	-41.2893371582031\\
4.19999980926514	-44.3118591308594\\
4.20499992370605	-34.3701324462891\\
4.21000003814697	-15.899284362793\\
4.21500015258789	4.97520446777344\\
4.21999979019165	23.0312652587891\\
4.22499990463257	35.2092819213867\\
4.23000001907349	40.9890747070313\\
4.2350001335144	41.4239158630371\\
4.23999977111816	35.9136734008789\\
4.24499988555908	25.0464324951172\\
4.25	12.1948299407959\\
4.25500011444092	-0.376081466674805\\
4.26000022888184	-10.0833034515381\\
4.2649998664856	-14.8291654586792\\
4.26999998092651	-13.7992277145386\\
4.27500009536743	-7.22724723815918\\
4.28000020980835	4.62609100341797\\
4.28499984741211	17.8772640228271\\
4.28999996185303	29.8979072570801\\
4.29500007629395	39.3625612258911\\
4.30000019073486	46.8984756469727\\
4.30499982833862	54.5085687637329\\
4.30999994277954	62.0200810432434\\
4.31500005722046	68.835721231997\\
4.32000017166138	73.7924890746036\\
4.32499980926514	76.545521736145\\
4.32999992370605	73.5510196685791\\
4.33500003814697	67.6833229064941\\
4.34000015258789	61.3761157989502\\
4.34499979019165	50.8786697387695\\
4.34999990463257	34.6839904785156\\
4.35500001907349	10.6656646728516\\
4.3600001335144	-23.3931846618652\\
4.36499977111816	-71.9617252349854\\
4.36999988555908	-135.614947363843\\
4.375	-180.974769738384\\
4.38000011444092	-200.144338575017\\
4.38500022888184	-197.711759484378\\
4.3899998664856	-180.232618291863\\
4.39499998092651	-161.834389686584\\
4.40000009536743	-96.8514556884766\\
4.40500020980835	-26.85426902771\\
4.40999984741211	35.8889541625977\\
4.41499996185303	78.6287455558777\\
4.42000007629395	97.7004222869873\\
4.42500019073486	83.9674072265625\\
4.42999982833862	58.234489440918\\
4.43499994277954	29.3986206054688\\
4.44000005722046	-1.18909454345703\\
4.44500017166138	-28.9121551513672\\
4.44999980926514	-48.9449310302734\\
4.45499992370605	-58.0335845947266\\
4.46000003814697	-61.9093780517578\\
4.46500015258789	-49.3530120849609\\
4.46999979019165	-23.9404678344727\\
4.47499990463257	4.56002807617188\\
4.48000001907349	29.3509292602539\\
4.4850001335144	46.3892288208008\\
4.48999977111816	53.9428634643555\\
4.49499988555908	54.6730651855469\\
4.5	47.2912330627441\\
4.50500011444092	33.5852165222168\\
4.51000022888184	16.0994205474854\\
4.5149998664856	-0.832637786865234\\
4.51999998092651	-14.0472207069397\\
4.52500009536743	-20.4666805267334\\
4.53000020980835	-19.7079200744629\\
4.53499984741211	-11.6637382507324\\
4.53999996185303	3.63829803466797\\
4.54500007629395	20.4909286499023\\
4.55000019073486	35.9889669418335\\
4.55499982833862	48.7173256874084\\
4.55999994277954	55.2276319026066\\
4.56500005722046	56.1053212131496\\
4.57000017166138	61.071147173232\\
4.57499980926514	67.4411422088469\\
4.57999992370605	73.4302511700645\\
4.58500003814697	77.8983526067482\\
4.59000015258789	80.6706227558825\\
4.59499979019165	81.7901371684998\\
4.59999990463257	83.5054615289501\\
4.60500001907349	82.7810276221262\\
4.6100001335144	79.0944660825958\\
4.61499977111816	70.0698308976419\\
4.61999988555908	2.10214996337891\\
4.625	-148.668675840808\\
4.63000011444092	-268.037969185548\\
4.63500022888184	-314.152539436985\\
4.6399998664856	-306.252804889948\\
4.64499998092651	-266.784207617719\\
4.65000009536743	-226.388669423976\\
4.65500020980835	-185.303006745416\\
4.65999984741211	-137.872831212493\\
4.66499996185303	-34.6219956874847\\
4.67000007629395	79.3830575942993\\
4.67500019073486	162.12606048584\\
4.67999982833862	201.457649230957\\
4.68499994277954	181.093399047852\\
4.69000005722046	132.036262512207\\
4.69500017166138	74.1932907104492\\
4.69999980926514	10.3971328735352\\
4.70499992370605	-48.7514419555664\\
4.71000003814697	-92.1871490478516\\
4.71500015258789	-113.045913696289\\
4.71999979019165	-120.488807678223\\
4.72499990463257	-95.3216247558594\\
4.73000001907349	-46.9097137451172\\
4.7350001335144	7.31495666503906\\
4.73999977111816	53.5327835083008\\
4.74499988555908	83.9581909179688\\
4.75	95.0337944030762\\
4.75500011444092	93.0000610351563\\
4.76000022888184	76.2151260375977\\
4.7649998664856	45.7585797309875\\
4.76999998092651	11.6352655887604\\
4.77500009536743	-21.7909002304077\\
4.78000020980835	-46.6136913299561\\
4.78499984741211	-55.8741920184102\\
4.78999996185303	-47.4060252416191\\
4.79500007629395	-34.3748885383346\\
4.80000019073486	-4.25009918212891\\
4.80499982833862	27.2310073673725\\
4.80999994277954	45.2009061014391\\
4.81500005722046	65.0544005398936\\
4.82000017166138	83.7840718432846\\
4.82499980926514	100.200201754215\\
4.82999992370605	113.678221334964\\
4.83500003814697	121.669096474688\\
4.84000015258789	129.12389984634\\
4.84499979019165	131.127638213877\\
4.84999990463257	125.403593472989\\
4.85500001907349	108.883459703156\\
4.8600001335144	78.7305555783415\\
4.86499977111816	30.7134834575476\\
4.86999988555908	-85.4952611348524\\
4.875	-211.785785670842\\
4.88000011444092	-300.549603768549\\
4.88500022888184	-328.961514505888\\
4.8899998664856	-311.209530864004\\
4.89499998092651	-264.736722193315\\
4.90000009536743	-227.721677131078\\
4.90500020980835	-181.063321612887\\
4.90999984741211	-131.851041619329\\
4.91499996185303	-88.8987256761089\\
4.92000007629395	-17.2723255157471\\
4.92500019073486	132.749431610107\\
4.92999982833862	235.154964447021\\
4.93499994277954	271.215274810791\\
4.94000005722046	218.966289520264\\
4.94500017166138	148.230293273926\\
4.94999980926514	64.8222045898438\\
4.95499992370605	-17.823902130127\\
4.96000003814697	-83.7893753051758\\
4.96500015258789	-122.830101013184\\
4.96999979019165	-133.004066467285\\
4.97499990463257	-126.471538543701\\
4.98000001907349	-95.130199432373\\
4.9850001335144	-44.699291229248\\
4.98999977111816	8.7354621887207\\
4.99499988555908	53.3667640686035\\
5	85.1533737182617\\
5.00500011444092	100.949378967285\\
5.01000022888184	102.035823822021\\
5.0149998664856	94.7647333145142\\
5.01999998092651	76.2685415744781\\
5.02500009536743	57.0757741928101\\
5.03000020980835	38.7176966667175\\
5.03499984741211	22.6402721405029\\
5.03999996185303	7.91751313209534\\
5.04500007629395	-2.45036435127258\\
5.05000019073486	-7.30770421028137\\
5.05499982833862	-7.18219923973083\\
5.05999994277954	-3.10103917121887\\
5.06500005722046	3.85437679290771\\
5.07000017166138	12.3709845542908\\
5.07499980926514	20.8603866100311\\
5.07999992370605	28.1414352655411\\
5.08500003814697	20.9036390315087\\
5.09000015258789	11.5649695741519\\
5.09499979019165	4.13903299555477\\
5.09999990463257	-3.59032665426366\\
5.10500001907349	-9.09802802004003\\
5.1100001335144	-10.2228296492357\\
5.11499977111816	-9.9714096365417\\
5.11999988555908	-9.18708643482784\\
5.125	-8.27150474643441\\
5.13000011444092	-7.36251008585843\\
5.13500022888184	-130.721603750797\\
5.1399998664856	-182.763220198257\\
5.14499998092651	-185.515455468626\\
5.15000009536743	-168.01071000504\\
5.15500020980835	-141.230085724791\\
5.15999984741211	-93.4144468307495\\
5.16499996185303	-19.5834250450134\\
5.17000007629395	45.8845686912537\\
5.17500019073486	91.6918382644653\\
5.17999982833862	112.972893238068\\
5.18499994277954	102.671513557434\\
5.19000005722046	81.1324408054352\\
5.19500017166138	57.086802482605\\
5.19999980926514	31.3950958251953\\
5.20499992370605	7.62295150756836\\
5.21000003814697	-11.0958347320557\\
5.21500015258789	-22.7826557159424\\
5.21999979019165	-27.2157039642334\\
5.22499990463257	-24.8751564025879\\
5.23000001907349	-17.711877822876\\
5.2350001335144	-6.93331050872803\\
5.23999977111816	6.45397412776947\\
5.24499988555908	18.8168182373047\\
5.25	29.2436428070068\\
5.25500011444092	35.9956169128418\\
5.26000022888184	38.9580879211426\\
5.2649998664856	39.1101226806641\\
5.26999998092651	34.7015724182129\\
5.27500009536743	27.484245300293\\
5.28000020980835	18.2725162506104\\
5.28499984741211	7.52405738830566\\
5.28999996185303	-3.50880432128906\\
5.29500007629395	-14.0917754173279\\
5.30000019073486	-24.0519607067108\\
5.30499982833862	-31.8866605758667\\
5.30999994277954	-35.7272357940674\\
5.31500005722046	-35.415355682373\\
5.32000017166138	-30.490406036377\\
5.32499980926514	-22.1764106750488\\
5.32999992370605	-13.2808322906494\\
5.33500003814697	-5.40276908874512\\
5.34000015258789	0.267654895782471\\
5.34499979019165	3.64921045303345\\
5.34999990463257	3.94527862966061\\
5.35500001907349	0.617146372795105\\
5.3600001335144	-5.88338088989258\\
5.36499977111816	-14.3150553703308\\
5.36999988555908	-23.0409340858459\\
5.375	-30.0666933059692\\
5.38000011444092	-33.8945536613464\\
5.38500022888184	-33.7220969200134\\
5.3899998664856	-29.2258949279785\\
5.39499998092651	-20.9464588165283\\
5.40000009536743	-12.0517148971558\\
5.40500020980835	-5.07697939872742\\
5.40999984741211	0.0755689144134521\\
5.41499996185303	1.64852249622345\\
5.42000007629395	-1.01108103990555\\
5.42500019073486	-7.31449562311172\\
5.42999982833862	-13.4680877559118\\
5.43499994277954	-11.6979919225059\\
5.44000005722046	-8.7775613686116\\
5.44500017166138	-7.30713577443702\\
5.44999980926514	-7.59561180090069\\
5.45499992370605	-7.0496891044869\\
5.46000003814697	-6.23760896718768\\
5.46500015258789	-5.41486793526042\\
5.46999979019165	-4.64196860735643\\
5.47499990463257	-4.09309418168567\\
5.48000001907349	-3.37087101876185\\
5.4850001335144	-3.00757144442954\\
5.48999977111816	-2.53633440792587\\
5.49499988555908	-2.206852550721\\
5.5	-1.90114626159759\\
5.50500011444092	-1.68507162699845\\
5.51000022888184	-1.53017026287041\\
5.5149998664856	-1.31123921144799\\
5.51999998092651	-2.56058225657508\\
5.52500009536743	-17.4647207401064\\
5.53000020980835	-22.4688334630229\\
5.53499984741211	-26.6436007817478\\
5.53999996185303	-29.8174564000714\\
5.54500007629395	-32.1099622284715\\
5.55000019073486	-33.5141572781037\\
5.55499982833862	-34.1697306352276\\
5.55999994277954	-34.3707475314804\\
5.56500005722046	-34.3201145765997\\
5.57000017166138	-33.44125411372\\
5.57499980926514	-31.8843903491929\\
5.57999992370605	-29.8880429493771\\
5.58500003814697	-27.2975068676535\\
5.59000015258789	-24.0839866203684\\
5.59499979019165	-20.9979414000061\\
5.59999990463257	-17.3680971927807\\
5.60500001907349	-13.8156765041378\\
5.6100001335144	-10.6815744971603\\
5.61499977111816	-7.75572235188118\\
5.61999988555908	-5.06811529209914\\
5.625	-2.99898034099422\\
5.63000011444092	-1.49052695885811\\
5.63500022888184	0.049072111221605\\
5.6399998664856	0.93361488165894\\
5.64499998092651	1.41624656668274\\
5.65000009536743	1.86029429810783\\
5.65500020980835	2.21171533465184\\
5.65999984741211	2.51370868775712\\
5.66499996185303	2.73549965805845\\
5.67000007629395	7.84232537011788\\
5.67500019073486	92.8333716932055\\
5.67999982833862	106.338847696363\\
5.68499994277954	106.24098867178\\
5.69000005722046	91.270733833313\\
5.69500017166138	63.3153839111328\\
5.69999980926514	32.5933570861816\\
5.70499992370605	6.56179046630859\\
5.71000003814697	-9.55574798583984\\
5.71500015258789	-9.22650527954102\\
5.71999979019165	-1.65446472167969\\
5.72499990463257	9.96947479248047\\
5.73000001907349	24.0262145996094\\
5.7350001335144	36.8760871887207\\
5.73999977111816	47.8283443450928\\
5.74499988555908	56.0362663269043\\
5.75	61.628267288208\\
5.75500011444092	65.4939937591553\\
5.76000022888184	69.3142776489258\\
5.7649998664856	73.0890331268311\\
5.76999998092651	78.0538158416748\\
5.77500009536743	84.7485389709473\\
5.78000020980835	91.6258563995361\\
5.78499984741211	98.6065979003906\\
5.78999996185303	105.835821151733\\
5.79500007629395	112.406533718109\\
5.80000019073486	117.628842353821\\
5.80499982833862	122.012594223022\\
5.80999994277954	125.490653991699\\
5.81500005722046	127.725963592529\\
5.82000017166138	129.422939300537\\
5.82499980926514	131.417610168457\\
5.82999992370605	131.79369354248\\
5.83500003814697	132.350341796875\\
5.84000015258789	132.360610961914\\
5.84499979019165	132.308471679688\\
5.84999990463257	132.467620849609\\
5.85500001907349	135.354881286621\\
5.8600001335144	136.348346710205\\
5.86499977111816	132.81177520752\\
5.86999988555908	125.946466445923\\
5.875	112.094533443451\\
5.88000011444092	91.9942578077316\\
5.88500022888184	62.2342889308929\\
5.8899998664856	28.9209355694593\\
5.89499998092651	32.7237619447009\\
5.90000009536743	23.3236523765072\\
5.90500020980835	0.527061381793601\\
5.90999984741211	-50.4676209157396\\
5.91499996185303	-130.081960127537\\
5.92000007629395	-215.196881128705\\
5.92500019073486	-284.353329084422\\
5.92999982833862	-326.778743904603\\
5.93499994277954	-339.238582727353\\
5.94000005722046	-325.458370397164\\
5.94500017166138	-294.128780675225\\
5.94999980926514	-254.148067892882\\
5.95499992370605	-212.768185131951\\
5.96000003814697	-186.320186965432\\
5.96500015258789	-165.629757077022\\
5.96999979019165	-149.40887062323\\
5.97499990463257	-141.851680482968\\
5.98000001907349	-189.37956237793\\
5.9850001335144	114.317092895508\\
5.98999977111816	325.554321289063\\
5.99499988555908	456.411590576172\\
6	497.971923828125\\
6.00500011444092	435.856964111328\\
6.01000022888184	331.8173828125\\
6.0149998664856	202.324996948242\\
6.01999998092651	71.6717834472656\\
6.02500009536743	-37.11376953125\\
6.03000020980835	-111.544288635254\\
6.03499984741211	-150.955856323242\\
6.03999996185303	-159.91321182251\\
6.04500007629395	-158.232669830322\\
6.05000019073486	-123.377738952637\\
6.05499982833862	-63.0050201416016\\
6.05999994277954	4.36958312988281\\
6.06500005722046	65.9987640380859\\
6.07000017166138	114.572387695313\\
6.07499980926514	144.743026733398\\
6.07999992370605	154.3603515625\\
6.08500003814697	149.926467895508\\
6.09000015258789	127.041519165039\\
6.09499979019165	91.6378707885742\\
6.09999990463257	52.6701736450195\\
6.10500001907349	13.9150314331055\\
6.1100001335144	-19.968921661377\\
6.11499977111816	-51.3684043884277\\
6.11999988555908	-71.4140129089355\\
6.125	-75.9624290466309\\
6.13000011444092	-64.4471549987793\\
6.13500022888184	-38.6458435058594\\
6.1399998664856	-7.19305038452148\\
6.14499998092651	23.459114074707\\
6.15000009536743	47.8932647705078\\
6.15500020980835	63.6946563720703\\
6.15999984741211	72.1306056976318\\
6.16499996185303	69.0403022766113\\
6.17000007629395	54.8527965545654\\
6.17500019073486	31.8165016174316\\
6.17999982833862	3.92879772186279\\
6.18499994277954	-24.0385810136795\\
6.19000005722046	-37.8807652546976\\
6.19500017166138	-36.7320542064952\\
6.19999980926514	-29.8822625918872\\
6.20499992370605	-20.7194388748226\\
6.21000003814697	-11.1506400999392\\
6.21500015258789	-7.99718743550995\\
6.21999979019165	-10.5476406110075\\
6.22499990463257	-10.5722253959538\\
6.23000001907349	-9.688048953561\\
6.2350001335144	-8.49652846698095\\
6.23999977111816	-7.32244123287862\\
6.24499988555908	-6.26982775357451\\
6.25	-5.342028141116\\
6.25500011444092	-4.54176087083782\\
6.26000022888184	-3.8507122299117\\
6.2649998664856	-3.26714487155009\\
6.26999998092651	-2.77187859150534\\
6.27500009536743	-2.34562340086426\\
6.28000020980835	-1.98249160476984\\
6.28499984741211	-1.67517901206589\\
6.28999996185303	-1.41615502591297\\
6.29500007629395	-1.19658089797673\\
6.30000019073486	-1.00769309581262\\
6.30499982833862	-0.845001635068602\\
6.30999994277954	-0.706507780949821\\
6.31500005722046	-0.587846251711461\\
6.32000017166138	-0.483378264275602\\
6.32499980926514	-0.391296877435807\\
6.32999992370605	-0.312966978238278\\
6.33500003814697	-0.25449658811488\\
6.34000015258789	-0.219055260974077\\
6.34499979019165	-0.186748044864372\\
6.34999990463257	-0.15406878129738\\
6.35500001907349	-0.117966379188952\\
6.3600001335144	29.7757736020517\\
6.36499977111816	52.40922250838\\
6.36999988555908	68.5487024668193\\
6.375	77.8941799537105\\
6.38000011444092	80.8414159777532\\
6.38500022888184	96.8441123962402\\
6.3899998664856	-3.20666456222534\\
6.39499998092651	-36.0830030441284\\
6.40000009536743	-60.1464338302612\\
6.40500020980835	-74.0564517974854\\
6.40999984741211	-75.5594606399536\\
6.41499996185303	-67.4950828552246\\
6.42000007629395	-53.1130714416504\\
6.42500019073486	-35.5368366241455\\
6.42999982833862	-17.7257690429688\\
6.43499994277954	-1.78342056274414\\
6.44000005722046	10.6279830932617\\
6.44500017166138	18.6960296630859\\
6.44999980926514	22.2299995422363\\
6.45499992370605	21.7542991638184\\
6.46000003814697	18.1256637573242\\
6.46500015258789	12.4022026062012\\
6.46999979019165	5.77883148193359\\
6.47499990463257	-0.636199951171875\\
6.48000001907349	-6.27806854248047\\
6.4850001335144	-10.4779472351074\\
6.48999977111816	-13.1246643066406\\
6.49499988555908	-14.2140464782715\\
6.5	-13.9650688171387\\
6.50500011444092	-12.7395820617676\\
6.51000022888184	-10.8991394042969\\
6.5149998664856	-8.85422325134277\\
6.51999998092651	-6.94477081298828\\
6.52500009536743	-5.41062355041504\\
6.53000020980835	-4.40123748779297\\
6.53499984741211	-4.10614204406738\\
6.53999996185303	-4.38287758827209\\
6.54500007629395	-5.16177225112915\\
6.55000019073486	-6.41592693328857\\
6.55499982833862	-8.04833221435547\\
6.55999994277954	-9.96014213562012\\
6.56500005722046	-11.8356323242188\\
6.57000017166138	-13.7173385620117\\
6.57499980926514	-15.2330894470215\\
6.57999992370605	-16.9492301940918\\
6.58500003814697	-18.4206695556641\\
6.59000015258789	-19.6717529296875\\
6.59499979019165	-20.8456878662109\\
6.59999990463257	-21.981689453125\\
6.60500001907349	-22.9477386474609\\
6.6100001335144	-23.9926834106445\\
6.61499977111816	-25.1395568847656\\
6.61999988555908	-26.2489776611328\\
6.625	-27.6947174072266\\
6.63000011444092	-29.2681045532227\\
6.63500022888184	-31.0409088134766\\
6.6399998664856	-33.1941375732422\\
6.64499998092651	-35.3557891845703\\
6.65000009536743	-37.6656341552734\\
6.65500020980835	-40.3836059570313\\
6.65999984741211	-42.8773345947266\\
6.66499996185303	-45.4783477783203\\
6.67000007629395	-48.4582824707031\\
6.67500019073486	-51.1357116699219\\
6.67999982833862	-53.7135772705078\\
6.68499994277954	-55.8799438476563\\
6.69000005722046	-58.0943603515625\\
6.69500017166138	-60.7055053710938\\
6.69999980926514	-64.0725402832031\\
6.70499992370605	-66.7475280761719\\
6.71000003814697	-69.0628051757813\\
6.71500015258789	-71.0830993652344\\
6.71999979019165	-73.2670593261719\\
6.72499990463257	-75.5205383300781\\
6.73000001907349	-77.7982177734375\\
6.7350001335144	-79.9526977539063\\
6.73999977111816	-82.3940124511719\\
6.74499988555908	-84.5464172363281\\
6.75	-86.8674621582031\\
6.75500011444092	-89.0736083984375\\
6.76000022888184	-90.8281860351563\\
6.7649998664856	-93.2888793945313\\
6.76999998092651	-95.0545806884766\\
6.77500009536743	-96.9807739257813\\
6.78000020980835	-98.7571258544922\\
6.78499984741211	-100.426361083984\\
6.78999996185303	-101.955749511719\\
6.79500007629395	-103.355010986328\\
6.80000019073486	-104.620590209961\\
6.80499982833862	-105.809539794922\\
6.80999994277954	-107.0068359375\\
6.81500005722046	-108.029251098633\\
6.82000017166138	-109.041839599609\\
6.82499980926514	-109.978858947754\\
6.82999992370605	-110.78849029541\\
6.83500003814697	-111.677940368652\\
6.84000015258789	-111.911491394043\\
6.84499979019165	-112.322273254395\\
6.84999990463257	-112.315288543701\\
6.85500001907349	-111.973617553711\\
6.8600001335144	-111.493606567383\\
6.86499977111816	-109.946870803833\\
6.86999988555908	-108.497126579285\\
6.875	-107.610729217529\\
6.88000011444092	-106.109340667725\\
6.88500022888184	-104.598859786987\\
6.8899998664856	-102.972255706787\\
6.89499998092651	-101.332683563232\\
6.90000009536743	-99.6882400512695\\
6.90500020980835	-98.0881080627441\\
6.90999984741211	-96.2735824584961\\
6.91499996185303	-94.5304260253906\\
6.92000007629395	-92.8068313598633\\
6.92500019073486	-91.0309524536133\\
6.92999982833862	-89.2721405029297\\
6.93499994277954	-87.4615859985352\\
6.94000005722046	-85.6952972412109\\
6.94500017166138	-83.8775482177734\\
6.94999980926514	-81.9597473144531\\
6.95499992370605	-79.4567108154297\\
6.96000003814697	-77.3406753540039\\
6.96500015258789	-75.8069000244141\\
6.96999979019165	-73.6543121337891\\
6.97499990463257	-71.6433334350586\\
6.98000001907349	-69.648681640625\\
6.9850001335144	-67.8035888671875\\
6.98999977111816	-65.7802886962891\\
6.99499988555908	-63.6703186035156\\
7	-61.9060745239258\\
7.00500011444092	-60.0534057617188\\
7.01000022888184	-58.2161636352539\\
7.0149998664856	-56.3894500732422\\
7.01999998092651	-54.6119613647461\\
7.02500009536743	-52.9251174926758\\
7.03000020980835	-51.2829895019531\\
7.03499984741211	-49.6723785400391\\
7.03999996185303	-48.1072959899902\\
7.04500007629395	-46.5984230041504\\
7.05000019073486	-45.1575546264648\\
7.05499982833862	-43.8112182617188\\
7.05999994277954	-42.4597206115723\\
7.06500005722046	-41.2005500793457\\
7.07000017166138	-40.0607032775879\\
7.07499980926514	-38.9808807373047\\
7.07999992370605	-37.9579811096191\\
7.08500003814697	-37.0121955871582\\
7.09000015258789	-36.1604347229004\\
7.09499979019165	-35.3992538452148\\
7.09999990463257	-34.6796073913574\\
7.10500001907349	-34.0067367553711\\
7.1100001335144	-33.382152557373\\
7.11499977111816	-32.8640880584717\\
7.11999988555908	-32.4206523895264\\
7.125	-32.0163116455078\\
7.13000011444092	-31.6767807006836\\
7.13500022888184	-31.3963356018066\\
7.1399998664856	-31.1717796325684\\
7.14499998092651	-31.0073165893555\\
7.15000009536743	-30.9667835235596\\
7.15500020980835	-31.277027130127\\
7.15999984741211	-31.6262969970703\\
7.16499996185303	-31.7808418273926\\
7.17000007629395	-31.5794639587402\\
7.17500019073486	-31.4028453826904\\
7.17999982833862	-31.430793762207\\
7.18499994277954	-31.5385398864746\\
7.19000005722046	-31.7113418579102\\
7.19500017166138	-31.9399604797363\\
7.19999980926514	-32.2040405273438\\
7.20499992370605	-32.4494552612305\\
7.21000003814697	-32.7676048278809\\
7.21500015258789	-33.1285552978516\\
7.21999979019165	-33.3945655822754\\
7.22499990463257	-33.7423667907715\\
7.23000001907349	-34.0960540771484\\
7.2350001335144	-34.4351539611816\\
7.23999977111816	-34.795524597168\\
7.24499988555908	-35.1300506591797\\
7.25	-35.4375\\
7.25500011444092	-35.770938873291\\
7.26000022888184	-36.081844329834\\
7.2649998664856	-36.3983955383301\\
7.26999998092651	-36.7417030334473\\
7.27500009536743	-37.0341720581055\\
7.28000020980835	-37.3063659667969\\
7.28499984741211	-37.5582275390625\\
7.28999996185303	-37.7970504760742\\
7.29500007629395	-38.0193862915039\\
7.30000019073486	-38.1850204467773\\
7.30499982833862	-38.3103332519531\\
7.30999994277954	-38.4429931640625\\
7.31500005722046	-38.565559387207\\
7.32000017166138	-38.6626052856445\\
7.32499980926514	-38.7380599975586\\
7.32999992370605	-38.7741775512695\\
7.33500003814697	-38.771614074707\\
7.34000015258789	-38.7548675537109\\
7.34499979019165	-38.7208480834961\\
7.34999990463257	-38.660026550293\\
7.35500001907349	-38.5799942016602\\
7.3600001335144	-38.5022354125977\\
7.36499977111816	-38.4312896728516\\
7.36999988555908	-38.3611297607422\\
7.375	-38.2753067016602\\
7.38000011444092	-38.1810836791992\\
7.38500022888184	-38.0027923583984\\
7.3899998664856	-37.688117980957\\
7.39499998092651	-37.2485427856445\\
7.40000009536743	-36.8667678833008\\
7.40500020980835	-36.5067367553711\\
7.40999984741211	-36.1493453979492\\
7.41499996185303	-35.8076324462891\\
7.42000007629395	-35.4831314086914\\
7.42500019073486	-35.1681289672852\\
7.42999982833862	-34.8603286743164\\
7.43499994277954	-34.5249404907227\\
7.44000005722046	-34.1599731445313\\
7.44500017166138	-33.7919311523438\\
7.44999980926514	-33.4115829467773\\
7.45499992370605	-33.0228729248047\\
7.46000003814697	-32.6302947998047\\
7.46500015258789	-32.2424468994141\\
7.46999979019165	-31.8773193359375\\
7.47499990463257	-31.5443649291992\\
7.48000001907349	-31.2182159423828\\
7.4850001335144	-30.8991966247559\\
7.48999977111816	-30.5835456848145\\
7.49499988555908	-30.2605400085449\\
7.5	-29.9290580749512\\
7.50500011444092	-29.5948791503906\\
7.51000022888184	-29.2621002197266\\
7.5149998664856	-28.9623107910156\\
7.51999998092651	-28.7233810424805\\
7.52500009536743	-28.6608047485352\\
7.53000020980835	-28.3369636535645\\
7.53499984741211	-28.1299362182617\\
7.53999996185303	-27.9048271179199\\
7.54500007629395	-27.7317314147949\\
7.55000019073486	-27.5774192810059\\
7.55499982833862	-27.4350929260254\\
7.55999994277954	-27.3193817138672\\
7.56500005722046	-27.2250213623047\\
7.57000017166138	-27.1403503417969\\
7.57499980926514	-27.0674057006836\\
7.57999992370605	-27.0136260986328\\
7.58500003814697	-26.9763298034668\\
7.59000015258789	-26.9574394226074\\
7.59499979019165	-26.9574813842773\\
7.59999990463257	-26.9704246520996\\
7.60500001907349	-27.0103225708008\\
7.6100001335144	-27.1042594909668\\
7.61499977111816	-27.2768630981445\\
7.61999988555908	-27.4370765686035\\
7.625	-27.4800186157227\\
7.63000011444092	-27.4828948974609\\
7.63500022888184	-27.5273056030273\\
7.6399998664856	-27.6026344299316\\
7.64499998092651	-27.7291221618652\\
7.65000009536743	-27.8618469238281\\
7.65500020980835	-28.0038185119629\\
7.65999984741211	-28.1578826904297\\
7.66499996185303	-28.3176574707031\\
7.67000007629395	-28.4777183532715\\
7.67500019073486	-28.6397361755371\\
7.67999982833862	-28.8116340637207\\
7.68499994277954	-28.9927139282227\\
7.69000005722046	-29.1837120056152\\
7.69500017166138	-29.3630599975586\\
7.69999980926514	-29.5290298461914\\
7.70499992370605	-29.6917610168457\\
7.71000003814697	-29.8599662780762\\
7.71500015258789	-30.0638236999512\\
7.71999979019165	-30.2723655700684\\
7.72499990463257	-30.4495010375977\\
7.73000001907349	-30.6097183227539\\
7.7350001335144	-30.7523498535156\\
7.73999977111816	-30.9326820373535\\
7.74499988555908	-31.1194648742676\\
7.75	-31.2674942016602\\
7.75500011444092	-31.4084014892578\\
7.76000022888184	-31.5611267089844\\
7.7649998664856	-31.7036209106445\\
7.76999998092651	-31.8311882019043\\
7.77500009536743	-31.9471092224121\\
7.78000020980835	-32.0539169311523\\
7.78499984741211	-32.1720848083496\\
7.78999996185303	-32.289176940918\\
7.79500007629395	-32.4052352905273\\
7.80000019073486	-32.5064010620117\\
7.80499982833862	-32.5979919433594\\
7.80999994277954	-32.6812210083008\\
7.81500005722046	-32.758171081543\\
7.82000017166138	-32.8291931152344\\
7.82499980926514	-32.8934936523438\\
7.82999992370605	-32.9529037475586\\
7.83500003814697	-33.0094985961914\\
7.84000015258789	-33.061653137207\\
7.84499979019165	-33.1103210449219\\
7.84999990463257	-33.1602096557617\\
7.85500001907349	-33.2105178833008\\
7.8600001335144	-33.2622756958008\\
7.86499977111816	-33.3021774291992\\
7.86999988555908	-33.3338623046875\\
7.875	-33.355712890625\\
7.88000011444092	-33.3796997070313\\
7.88500022888184	-33.4081039428711\\
7.8899998664856	-33.4402542114258\\
7.89499998092651	-33.473747253418\\
7.90000009536743	-33.5055618286133\\
7.90500020980835	-33.5382080078125\\
7.90999984741211	-33.5689544677734\\
7.91499996185303	-33.5879974365234\\
7.92000007629395	-33.5972137451172\\
7.92500019073486	-33.5962219238281\\
7.92999982833862	-33.6144523620605\\
7.93499994277954	-33.6363182067871\\
7.94000005722046	-33.6617088317871\\
7.94500017166138	-33.7043113708496\\
7.94999980926514	-33.7602348327637\\
7.95499992370605	-33.8276405334473\\
7.96000003814697	-33.9089698791504\\
7.96500015258789	-34.0049209594727\\
7.96999979019165	-34.114387512207\\
7.97499990463257	-34.2317085266113\\
7.98000001907349	-34.3479614257813\\
7.9850001335144	-34.4695663452148\\
7.98999977111816	-34.5966529846191\\
7.99499988555908	-34.7298736572266\\
8	-34.8687171936035\\
8.00500011444092	-35.0132293701172\\
8.01000022888184	-35.1811180114746\\
8.01500034332275	-35.3710670471191\\
8.02000045776367	-35.5858154296875\\
8.02499961853027	-35.8069839477539\\
8.02999973297119	-36.0278244018555\\
8.03499984741211	-36.2514038085938\\
8.03999996185303	-36.481071472168\\
8.04500007629395	-36.7237510681152\\
8.05000019073486	-36.9750137329102\\
8.05500030517578	-37.2355918884277\\
8.0600004196167	-37.5081405639648\\
8.0649995803833	-37.792293548584\\
8.06999969482422	-38.0889930725098\\
8.07499980926514	-38.4164428710938\\
8.07999992370605	-38.7719078063965\\
8.08500003814697	-39.1594047546387\\
8.09000015258789	-39.547306060791\\
8.09500026702881	-39.9312705993652\\
8.10000038146973	-40.312255859375\\
8.10499954223633	-40.7053070068359\\
8.10999965667725	-41.1262969970703\\
8.11499977111816	-41.5678253173828\\
8.11999988555908	-42.0276031494141\\
8.125	-42.4933586120605\\
8.13000011444092	-42.9704742431641\\
8.13500022888184	-43.4589080810547\\
8.14000034332275	-43.9721298217773\\
8.14500045776367	-44.5061874389648\\
8.14999961853027	-45.0629348754883\\
8.15499973297119	-45.6227874755859\\
8.15999984741211	-46.1866683959961\\
8.16499996185303	-46.7534561157227\\
8.17000007629395	-47.3525314331055\\
8.17500019073486	-48.0024871826172\\
8.18000030517578	-48.6994400024414\\
8.1850004196167	-49.4169921875\\
8.1899995803833	-50.0939331054688\\
8.19499969482422	-50.7485656738281\\
8.19999980926514	-51.4035949707031\\
8.20499992370605	-52.0966644287109\\
8.21000003814697	-52.8133316040039\\
8.21500015258789	-53.5412139892578\\
8.22000026702881	-54.2685775756836\\
8.22500038146973	-55.0002059936523\\
8.22999954223633	-55.8646774291992\\
8.23499965667725	-56.9375\\
8.23999977111816	-57.9630165100098\\
8.24499988555908	-58.7309913635254\\
8.25	-59.4566040039063\\
8.25500011444092	-60.3188285827637\\
8.26000022888184	-61.2221450805664\\
8.26500034332275	-62.1780319213867\\
8.27000045776367	-63.1999206542969\\
8.27499961853027	-64.2768020629883\\
8.27999973297119	-65.2877082824707\\
8.28499984741211	-66.275821685791\\
8.28999996185303	-67.3812789916992\\
8.29500007629395	-68.4750900268555\\
8.30000019073486	-69.4792938232422\\
8.30500030517578	-70.3612480163574\\
8.3100004196167	-71.1792335510254\\
8.3149995803833	-73.6513900756836\\
8.31999969482422	-74.5739479064941\\
8.32499980926514	-75.5565299987793\\
8.32999992370605	-76.7150459289551\\
8.33500003814697	-77.9951286315918\\
8.34000015258789	-79.2901039123535\\
8.34500026702881	-80.5282497406006\\
8.35000038146973	-81.856164932251\\
8.35499954223633	-82.8983497619629\\
8.35999965667725	-84.6975746154785\\
8.36499977111816	-86.4843001365662\\
8.36999988555908	-87.5807963609695\\
8.375	-88.8218350410461\\
8.38000011444092	-90.2657604217529\\
8.38500022888184	-91.5428047180176\\
8.39000034332275	-92.8642635345459\\
8.39500045776367	-93.8235702514648\\
8.39999961853027	-96.0806083679199\\
8.40499973297119	-97.271842956543\\
8.40999984741211	-98.3203201293945\\
8.41499996185303	-99.5886764526367\\
8.42000007629395	-100.625045776367\\
8.42500019073486	-101.596969604492\\
8.43000030517578	-102.236305236816\\
8.4350004196167	-102.883422851563\\
8.4399995803833	-103.726928710938\\
8.44499969482422	-103.70002746582\\
8.44999980926514	-103.429077148438\\
8.45499992370605	-103.070022583008\\
8.46000003814697	-102.203491210938\\
8.46500015258789	-101.521820068359\\
8.47000026702881	-100.544586181641\\
8.47500038146973	-99.4097900390625\\
8.47999954223633	-98.109375\\
8.48499965667725	-96.8262634277344\\
8.48999977111816	-95.2239685058594\\
8.49499988555908	-93.401611328125\\
8.5	-91.6222229003906\\
8.50500011444092	-89.6122741699219\\
8.51000022888184	-87.5286254882813\\
8.51500034332275	-85.4976196289063\\
8.52000045776367	-83.7492980957031\\
8.52499961853027	-81.7786254882813\\
8.52999973297119	-79.3797302246094\\
8.53499984741211	-76.9634704589844\\
8.53999996185303	-74.3304748535156\\
8.54500007629395	-71.8573608398438\\
8.55000019073486	-69.1973876953125\\
8.55500030517578	-66.5471801757813\\
8.5600004196167	-63.9610900878906\\
8.5649995803833	-61.3893737792969\\
8.56999969482422	-58.8339538574219\\
8.57499980926514	-56.3602600097656\\
8.57999992370605	-53.8860473632813\\
8.58500003814697	-51.5420837402344\\
8.59000015258789	-49.3972778320313\\
8.59500026702881	-47.0743103027344\\
8.60000038146973	-44.6921844482422\\
8.60499954223633	-42.5679168701172\\
8.60999965667725	-40.6155395507813\\
8.61499977111816	-38.8770751953125\\
8.61999988555908	-37.3833618164063\\
8.625	-36.0296020507813\\
8.63000011444092	-34.7677154541016\\
8.63500022888184	-33.5429077148438\\
8.64000034332275	-32.1493530273438\\
8.64500045776367	-30.6784591674805\\
8.64999961853027	-29.0948944091797\\
8.65499973297119	-27.2659759521484\\
8.65999984741211	-25.5329666137695\\
8.66499996185303	-23.3798828125\\
8.67000007629395	-21.3162841796875\\
8.67500019073486	-19.3185615539551\\
8.68000030517578	-17.5785713195801\\
8.6850004196167	-15.9602813720703\\
8.6899995803833	-14.6294136047363\\
8.69499969482422	-13.5705718994141\\
8.69999980926514	-12.7807521820068\\
8.70499992370605	-12.1941184997559\\
8.71000003814697	-11.7110815048218\\
8.71500015258789	-11.3402452468872\\
8.72000026702881	-11.080517411232\\
8.72500038146973	-10.7390179634094\\
8.72999954223633	-10.2818403244019\\
8.73499965667725	-9.72150611877441\\
8.73999977111816	-9.01115798950195\\
8.74499988555908	-8.07327079772949\\
8.75	-7.01551055908203\\
8.75500011444092	-5.98971939086914\\
8.76000022888184	-5.07698631286621\\
8.76500034332275	-4.16665267944336\\
8.77000045776367	-3.39160919189453\\
8.77499961853027	-2.78017807006836\\
8.77999973297119	-2.3358268737793\\
8.78499984741211	-2.06756591796875\\
8.78999996185303	-1.90755462646484\\
8.79500007629395	-1.82984161376953\\
8.80000019073486	-1.80405807495117\\
8.80500030517578	-1.77246856689453\\
8.8100004196167	-1.67264175415039\\
8.8149995803833	-1.46360015869141\\
8.81999969482422	-1.12149047851563\\
8.82499980926514	-0.668426513671875\\
8.82999992370605	-0.0977535247802734\\
8.83500003814697	0.532968521118164\\
8.84000015258789	1.18869590759277\\
8.84500026702881	1.84012985229492\\
8.85000038146973	2.35671234130859\\
8.85499954223633	2.80341720581055\\
8.85999965667725	3.111741065979\\
8.86499977111816	3.30084228515625\\
8.86999988555908	3.39032554626465\\
8.875	3.40934085845947\\
8.88000011444092	3.39821243286133\\
8.88500022888184	3.41071653366089\\
8.89000034332275	3.48219013214111\\
8.89500045776367	3.61507987976074\\
8.89999961853027	3.83333373069763\\
8.90499973297119	4.18112814426422\\
8.90999984741211	4.60291039943695\\
8.91499996185303	5.01985692977905\\
8.92000007629395	5.4183030128479\\
8.92500019073486	5.75872564315796\\
8.93000030517578	6.06904602050781\\
8.9350004196167	6.27861309051514\\
8.9399995803833	6.28277111053467\\
8.94499969482422	6.32563018798828\\
8.94999980926514	6.41941356658936\\
8.95499992370605	6.33371067047119\\
8.96000003814697	6.25577449798584\\
8.96500015258789	6.23030185699463\\
8.97000026702881	6.23604583740234\\
8.97500038146973	6.28714370727539\\
8.97999954223633	6.37807083129883\\
8.98499965667725	6.52265739440918\\
8.98999977111816	6.65454292297363\\
8.99499988555908	6.74770545959473\\
9	6.88217163085938\\
9.00500011444092	7.00693702697754\\
9.01000022888184	7.11214256286621\\
9.01500034332275	7.19761848449707\\
9.02000045776367	7.26884078979492\\
9.02499961853027	7.32300186157227\\
9.02999973297119	7.3602294921875\\
9.03499984741211	7.38762855529785\\
9.03999996185303	7.40803337097168\\
9.04500007629395	7.42629241943359\\
9.05000019073486	7.44699859619141\\
9.05500030517578	7.47268295288086\\
9.0600004196167	7.50273513793945\\
9.0649995803833	7.54086303710938\\
9.06999969482422	7.57691192626953\\
9.07499980926514	7.61254501342773\\
9.07999992370605	7.65041732788086\\
9.08500003814697	7.69283294677734\\
9.09000015258789	7.73464584350586\\
9.09500026702881	7.76022338867188\\
9.10000038146973	7.76885986328125\\
9.10499954223633	7.77299499511719\\
9.10999965667725	7.76774978637695\\
9.11499977111816	7.73912811279297\\
9.11999988555908	7.68336486816406\\
9.125	7.61276245117188\\
9.13000011444092	7.54544067382813\\
9.13500022888184	7.47956466674805\\
9.14000034332275	7.43798828125\\
9.14500045776367	7.41405487060547\\
9.14999961853027	7.41950607299805\\
9.15499973297119	7.45008850097656\\
9.15999984741211	7.50377655029297\\
9.16499996185303	7.57477569580078\\
9.17000007629395	7.64689636230469\\
9.17500019073486	7.71418380737305\\
9.18000030517578	7.76539611816406\\
9.1850004196167	7.79254150390625\\
9.1899995803833	7.79704284667969\\
9.19499969482422	7.77546310424805\\
9.19999980926514	7.71178817749023\\
9.20499992370605	7.62642288208008\\
9.21000003814697	7.54843521118164\\
9.21500015258789	7.4714469909668\\
9.22000026702881	7.39645385742188\\
9.22500038146973	7.36023712158203\\
9.22999954223633	7.35862350463867\\
9.23499965667725	7.36588287353516\\
9.23999977111816	7.31011581420898\\
9.24499988555908	6.97055053710938\\
9.25	6.26971817016602\\
9.25500011444092	5.45309448242188\\
9.26000022888184	4.55222702026367\\
9.26500034332275	3.57521438598633\\
9.27000045776367	2.67372131347656\\
9.27499961853027	2.03693771362305\\
9.27999973297119	1.48619842529297\\
9.28499984741211	1.22903442382813\\
9.28999996185303	1.15479278564453\\
9.29500007629395	1.26551055908203\\
9.30000019073486	1.65462875366211\\
9.30500030517578	1.76268005371094\\
9.3100004196167	2.8023567199707\\
9.3149995803833	4.07488250732422\\
9.31999969482422	5.91177368164063\\
9.32499980926514	8.36624526977539\\
9.32999992370605	11.534008026123\\
9.33500003814697	15.7464370727539\\
9.34000015258789	21.6022644042969\\
9.34500026702881	28.022403717041\\
9.35000038146973	37.2892150878906\\
9.35499954223633	48.6466369628906\\
9.35999965667725	61.3156509399414\\
9.36499977111816	75.7546768188477\\
9.36999988555908	91.4872665405273\\
9.375	107.219337463379\\
9.38000011444092	122.407188415527\\
9.38500022888184	136.590179443359\\
9.39000034332275	148.593826293945\\
9.39500045776367	158.917221069336\\
9.39999961853027	166.108261108398\\
9.40499973297119	170.871803283691\\
9.40999984741211	173.125244140625\\
9.41499996185303	172.689582824707\\
9.42000007629395	169.080848693848\\
9.42500019073486	162.767143249512\\
9.43000030517578	156.222747802734\\
9.4350004196167	144.200859069824\\
9.4399995803833	129.850372314453\\
9.44499969482422	111.986789703369\\
9.44999980926514	90.5132522583008\\
9.45499992370605	65.5664978027344\\
9.46000003814697	36.0090026855469\\
9.46500015258789	8.255615234375\\
9.47000026702881	-16.3336486816406\\
9.47500038146973	-29.3076782226563\\
9.47999954223633	-19.9739837646484\\
9.48499965667725	76.7194136989939\\
9.48999977111816	159.14171571631\\
9.49499988555908	201.751327351081\\
9.5	214.575496596755\\
9.50500011444092	210.708011649065\\
9.51000022888184	196.847454175208\\
9.51500034332275	179.210680161306\\
9.52000045776367	158.786871242836\\
9.52499961853027	137.509871630961\\
9.52999973297119	113.493318662523\\
9.53499984741211	88.640311287101\\
9.53999996185303	62.4601244756808\\
9.54500007629395	35.3192865698815\\
9.55000019073486	8.91889250114907\\
9.55500030517578	45.8512606343893\\
9.5600004196167	57.6257598092187\\
9.5649995803833	58.4196048345202\\
9.56999969482422	54.6111511964584\\
9.57499980926514	49.3712993332363\\
9.57999992370605	44.1204372479619\\
9.58500003814697	38.7332456061827\\
9.59000015258789	33.9196398129905\\
9.59500026702881	29.7033208240269\\
9.60000038146973	26.0542801626935\\
9.60499954223633	22.7781759134513\\
9.60999965667725	19.8419392706559\\
9.61499977111816	17.3579697107594\\
9.61999988555908	15.1787852252419\\
9.625	13.3183539567021\\
9.63000011444092	11.6064829351487\\
9.63500022888184	10.1898438888057\\
9.64000034332275	8.91546413294412\\
9.64500045776367	7.74887762042368\\
9.64999961853027	6.81746913726077\\
9.65499973297119	6.03418861850687\\
9.65999984741211	5.28383564259867\\
9.66499996185303	4.65295900416154\\
9.67000007629395	4.11631653019602\\
9.67500019073486	3.63342210933991\\
9.68000030517578	3.21165817170036\\
9.6850004196167	2.84755131932826\\
9.6899995803833	2.53219244205458\\
9.69499969482422	2.25890468326072\\
9.69999980926514	2.01740422229089\\
9.70499992370605	1.79849368162517\\
9.71000003814697	1.60812146441666\\
9.71500015258789	1.45372659244221\\
9.72000026702881	1.31437806138422\\
9.72500038146973	1.17562017120311\\
9.72999954223633	1.04902911164088\\
9.73499965667725	0.935203932102487\\
9.73999977111816	0.839150728625\\
9.74499988555908	0.768616192057272\\
9.75	0.69897099802948\\
9.75500011444092	0.642468792256103\\
9.76000022888184	0.599622825145603\\
9.76500034332275	0.563946051793513\\
9.77000045776367	0.526564680579069\\
9.77499961853027	0.485685843760802\\
9.77999973297119	0.436388325669068\\
9.78499984741211	0.401306746416013\\
9.78999996185303	0.372511714261655\\
9.79500007629395	0.353208876578446\\
9.80000019073486	0.336936054704818\\
9.80500030517578	0.315343427233588\\
9.8100004196167	0.290603473544829\\
9.8149995803833	0.26044532335623\\
9.81999969482422	0.238529176493834\\
9.82499980926514	0.220806307959998\\
9.82999992370605	0.209726563425888\\
9.83500003814697	0.200726099957382\\
9.84000015258789	0.188412714631523\\
9.84500026702881	0.174467310126152\\
9.85000038146973	0.157513135239438\\
9.85499954223633	0.144195191647135\\
9.85999965667725	0.130973355994652\\
9.86499977111816	0.11767759589257\\
9.86999988555908	0.105290615544039\\
9.875	0.0947285159647393\\
9.88000011444092	0.0858557431334139\\
9.88500022888184	0.0784401144298892\\
9.89000034332275	0.063984840271198\\
9.89500045776367	0.0449180167522325\\
9.89999961853027	0.0199041206722939\\
9.90499973297119	0.00134811130357093\\
9.90999984741211	-0.0017413308872154\\
9.91499996185303	0.00783469015304128\\
9.92000007629395	0.0295970348629755\\
9.92500019073486	0.0232132050870364\\
9.93000030517578	0.00252007411940713\\
9.9350004196167	-0.0372745338398621\\
9.9399995803833	-0.0702463844917789\\
9.94499969482422	-0.0863353506020452\\
9.94999980926514	-0.0991631491477811\\
9.95499992370605	-0.108710453323368\\
9.96000003814697	-0.114958575210474\\
9.96500015258789	-0.11788948067921\\
9.97000026702881	-0.117485755860529\\
9.97500038146973	-0.122283819667169\\
9.97999954223633	-0.131825763604999\\
9.98499965667725	-0.141704740141873\\
9.98999977111816	-0.15193888818182\\
9.99499988555908	-0.162546298200093\\
10	-0.173544982906508\\
};
\addlegendentry{Imbalance}

\end{axis}

\begin{axis}[%
width=4.521in,
height=0.823in,
at={(0.758in,1.913in)},
scale only axis,
xmin=0,
xmax=10,
xlabel style={font=\color{white!15!black}},
xlabel={Time (s)},
ymin=-272.142865214614,
ymax=549.37109375,
ylabel style={font=\color{white!15!black}},
ylabel={FY (N)},
axis background/.style={fill=white},
axis x line*=bottom,
axis y line*=left,
xmajorgrids,
ymajorgrids,
legend style={at={(0.35,1)}, anchor=north east, legend cell align=left, align=left, draw=black}
]
\addplot [color=red, line width=1.5pt]
  table[row sep=crcr]{%
0.0949999988079071	-37.0662422180176\\
0.100000001490116	-31.5252342224121\\
0.104999996721745	-26.7178440093994\\
0.109999999403954	-22.5751724243164\\
0.115000002086163	-18.938892364502\\
0.119999997317791	-2.23375368118286\\
0.125	62.1557922363281\\
0.129999995231628	99.3116912841797\\
0.135000005364418	103.427597045898\\
0.140000000596046	87.1696090698242\\
0.144999995827675	61.9259033203125\\
0.150000005960464	34.1235198974609\\
0.155000001192093	9.10157489776611\\
0.159999996423721	-11.8704986572266\\
0.165000006556511	-26.3391189575195\\
0.170000001788139	-37.3769645690918\\
0.174999997019768	-46.4719390869141\\
0.180000007152557	-54.2302665710449\\
0.185000002384186	-59.6873397827148\\
0.189999997615814	-62.620906829834\\
0.194999992847443	-37.080322265625\\
0.200000002980232	114.983596801758\\
0.204999998211861	225.555160522461\\
0.209999993443489	279.370147705078\\
0.215000003576279	303.037506103516\\
0.219999998807907	303.979156494141\\
0.224999994039536	297.296936035156\\
0.230000004172325	272.017303466797\\
0.234999999403954	228.312561035156\\
0.239999994635582	174.660705566406\\
0.245000004768372	120.717864990234\\
0.25	75.4415740966797\\
0.254999995231628	45.3877983093262\\
0.259999990463257	42.631893157959\\
0.264999985694885	59.386661529541\\
0.270000010728836	79.7584609985352\\
0.275000005960464	97.2398681640625\\
0.280000001192093	111.266975402832\\
0.284999996423721	122.523963928223\\
0.28999999165535	131.552917480469\\
0.294999986886978	139.547454833984\\
0.300000011920929	150.067260742188\\
0.305000007152557	157.162368774414\\
0.310000002384186	155.12190246582\\
0.314999997615814	144.378875732422\\
0.319999992847443	128.521148681641\\
0.324999988079071	111.428527832031\\
0.330000013113022	97.053092956543\\
0.33500000834465	89.1527481079102\\
0.340000003576279	86.252082824707\\
0.344999998807907	87.8781356811523\\
0.349999994039536	90.5200271606445\\
0.354999989271164	92.8365325927734\\
0.360000014305115	94.7438278198242\\
0.365000009536743	96.4510498046875\\
0.370000004768372	98.6735382080078\\
0.375	100.988571166992\\
0.379999995231628	101.318389892578\\
0.384999990463257	98.8782348632813\\
0.389999985694885	93.845100402832\\
0.395000010728836	86.7827072143555\\
0.400000005960464	78.6092681884766\\
0.405000001192093	70.3083419799805\\
0.409999996423721	62.9529991149902\\
0.41499999165535	57.0850257873535\\
0.419999986886978	53.1882209777832\\
0.425000011920929	51.322380065918\\
0.430000007152557	51.557933807373\\
0.435000002384186	52.3433265686035\\
0.439999997615814	53.1564674377441\\
0.444999992847443	54.4606246948242\\
0.449999988079071	55.3283958435059\\
0.455000013113022	55.1263618469238\\
0.46000000834465	53.7050743103027\\
0.465000003576279	51.0949401855469\\
0.469999998807907	47.624870300293\\
0.474999994039536	43.5023574829102\\
0.479999989271164	39.3917808532715\\
0.485000014305115	35.7486381530762\\
0.490000009536743	32.8715629577637\\
0.495000004768372	31.4448070526123\\
0.5	31.728666305542\\
0.504999995231628	32.6903038024902\\
0.509999990463257	33.8093299865723\\
0.514999985694885	34.9559288024902\\
0.519999980926514	36.0620918273926\\
0.524999976158142	37.2138748168945\\
0.529999971389771	38.6567230224609\\
0.535000026226044	40.3115730285645\\
0.540000021457672	40.916088104248\\
0.545000016689301	40.4710121154785\\
0.550000011920929	39.491886138916\\
0.555000007152557	38.5087852478027\\
0.560000002384186	38.2572860717773\\
0.564999997615814	38.5579872131348\\
0.569999992847443	39.332935333252\\
0.574999988079071	40.4100761413574\\
0.579999983310699	41.8927116394043\\
0.584999978542328	43.7611198425293\\
0.589999973773956	45.4540901184082\\
0.595000028610229	47.3374633789063\\
0.600000023841858	49.0112762451172\\
0.605000019073486	50.5827522277832\\
0.610000014305115	52.1441535949707\\
0.615000009536743	53.634838104248\\
0.620000004768372	54.8417778015137\\
0.625	55.9028511047363\\
0.629999995231628	57.0677185058594\\
0.634999990463257	58.2394218444824\\
0.639999985694885	59.4387893676758\\
0.644999980926514	60.6863403320313\\
0.649999976158142	61.9895515441895\\
0.654999971389771	63.4557685852051\\
0.660000026226044	64.8061141967773\\
0.665000021457672	66.1525192260742\\
0.670000016689301	67.5541534423828\\
0.675000011920929	68.8421020507813\\
0.680000007152557	70.1262741088867\\
0.685000002384186	71.2712173461914\\
0.689999997615814	72.3596878051758\\
0.694999992847443	73.2992630004883\\
0.699999988079071	74.1809921264648\\
0.704999983310699	74.9387969970703\\
0.709999978542328	75.6499481201172\\
0.714999973773956	76.2631530761719\\
0.720000028610229	76.8492889404297\\
0.725000023841858	77.3546142578125\\
0.730000019073486	77.8233032226563\\
0.735000014305115	78.2296295166016\\
0.740000009536743	78.5824584960938\\
0.745000004768372	78.8646469116211\\
0.75	79.0785446166992\\
0.754999995231628	79.2196731567383\\
0.759999990463257	79.2787170410156\\
0.764999985694885	79.2963104248047\\
0.769999980926514	79.3448486328125\\
0.774999976158142	79.3969497680664\\
0.779999971389771	79.3154449462891\\
0.785000026226044	79.0234756469727\\
0.790000021457672	78.6122283935547\\
0.795000016689301	78.0800323486328\\
0.800000011920929	77.5305709838867\\
0.805000007152557	76.9147109985352\\
0.810000002384186	76.3548126220703\\
0.814999997615814	75.7048034667969\\
0.819999992847443	74.9827499389648\\
0.824999988079071	74.3821716308594\\
0.829999983310699	73.685417175293\\
0.834999978542328	72.9029922485352\\
0.839999973773956	72.217658996582\\
0.845000028610229	71.4421844482422\\
0.850000023841858	70.7481460571289\\
0.855000019073486	69.9535369873047\\
0.860000014305115	69.1502685546875\\
0.865000009536743	68.3498382568359\\
0.870000004768372	67.5558700561523\\
0.875	66.7695846557617\\
0.879999995231628	65.9949035644531\\
0.884999990463257	65.2367706298828\\
0.889999985694885	64.4984970092773\\
0.894999980926514	63.7814025878906\\
0.899999976158142	63.0888137817383\\
0.904999971389771	62.3717422485352\\
0.910000026226044	61.7607650756836\\
0.915000021457672	61.1798057556152\\
0.920000016689301	60.5991668701172\\
0.925000011920929	60.0749702453613\\
0.930000007152557	59.5539855957031\\
0.935000002384186	59.0893630981445\\
0.939999997615814	58.63671875\\
0.944999992847443	58.2388763427734\\
0.949999988079071	57.8604011535645\\
0.954999983310699	57.5330429077148\\
0.959999978542328	57.2333221435547\\
0.964999973773956	56.9842796325684\\
0.970000028610229	56.7686805725098\\
0.975000023841858	56.6009368896484\\
0.980000019073486	56.4721412658691\\
0.985000014305115	56.3864402770996\\
0.990000009536743	56.3463745117188\\
0.995000004768372	56.3523139953613\\
1	56.4182662963867\\
1.00499999523163	56.5335540771484\\
1.00999999046326	56.7306938171387\\
1.01499998569489	56.9685325622559\\
1.01999998092651	57.1726531982422\\
1.02499997615814	57.3245735168457\\
1.02999997138977	57.4703407287598\\
1.0349999666214	57.6374816894531\\
1.03999996185303	57.8364219665527\\
1.04499995708466	58.071460723877\\
1.04999995231628	58.3401298522949\\
1.05499994754791	58.6393508911133\\
1.05999994277954	58.9641342163086\\
1.06500005722046	59.3088531494141\\
1.07000005245209	59.6706199645996\\
1.07500004768372	60.0447731018066\\
1.08000004291534	60.427906036377\\
1.08500003814697	60.8170776367188\\
1.0900000333786	61.2102203369141\\
1.09500002861023	61.6062698364258\\
1.10000002384186	62.0038223266602\\
1.10500001907349	62.4011039733887\\
1.11000001430511	62.795295715332\\
1.11500000953674	63.184211730957\\
1.12000000476837	63.5662879943848\\
1.125	63.9392776489258\\
1.12999999523163	64.3024215698242\\
1.13499999046326	64.6551513671875\\
1.13999998569489	64.99609375\\
1.14499998092651	65.3233032226563\\
1.14999997615814	65.6372146606445\\
1.15499997138977	65.9378204345703\\
1.1599999666214	66.2232437133789\\
1.16499996185303	66.4921569824219\\
1.16999995708466	66.7444534301758\\
1.17499995231628	66.9795303344727\\
1.17999994754791	67.1947937011719\\
1.18499994277954	67.3896255493164\\
1.19000005722046	67.5638046264648\\
1.19500005245209	67.7161102294922\\
1.20000004768372	67.8468170166016\\
1.20500004291534	67.9577255249023\\
1.21000003814697	68.0501708984375\\
1.2150000333786	68.123649597168\\
1.22000002861023	68.1761779785156\\
1.22500002384186	68.2070693969727\\
1.23000001907349	68.2168807983398\\
1.23500001430511	68.2117919921875\\
1.24000000953674	68.201789855957\\
1.24500000476837	68.1949996948242\\
1.25	68.1908874511719\\
1.25499999523163	68.1621780395508\\
1.25999999046326	68.0930023193359\\
1.26499998569489	67.9737777709961\\
1.26999998092651	67.8155975341797\\
1.27499997615814	67.6378479003906\\
1.27999997138977	67.4420623779297\\
1.2849999666214	67.232795715332\\
1.28999996185303	67.0241622924805\\
1.29499995708466	66.8249664306641\\
1.29999995231628	66.6337127685547\\
1.30499994754791	66.4422454833984\\
1.30999994277954	66.2509536743164\\
1.31500005722046	66.0591812133789\\
1.32000005245209	65.8629989624023\\
1.32500004768372	65.6593704223633\\
1.33000004291534	65.4479217529297\\
1.33500003814697	65.2308807373047\\
1.3400000333786	65.0091934204102\\
1.34500002861023	64.7835083007813\\
1.35000002384186	64.5568389892578\\
1.35500001907349	64.3335266113281\\
1.36000001430511	64.1172103881836\\
1.36500000953674	63.9107246398926\\
1.37000000476837	63.7171936035156\\
1.375	63.540599822998\\
1.37999999523163	63.378719329834\\
1.38499999046326	63.2197074890137\\
1.38999998569489	63.0584373474121\\
1.39499998092651	62.8956298828125\\
1.39999997615814	62.7347030639648\\
1.40499997138977	62.5749359130859\\
1.4099999666214	62.4201049804688\\
1.41499996185303	62.2798271179199\\
1.41999995708466	62.1588859558105\\
1.42499995231628	62.0578804016113\\
1.42999994754791	61.9728088378906\\
1.43499994277954	61.9035453796387\\
1.44000005722046	61.8488883972168\\
1.44500005245209	61.8051834106445\\
1.45000004768372	61.7703666687012\\
1.45500004291534	61.7430763244629\\
1.46000003814697	61.7211990356445\\
1.4650000333786	61.7038230895996\\
1.47000002861023	61.695182800293\\
1.47500002384186	61.7028007507324\\
1.48000001907349	61.7294044494629\\
1.48500001430511	61.7637481689453\\
1.49000000953674	61.7902069091797\\
1.49500000476837	61.80810546875\\
1.5	61.8453330993652\\
1.50499999523163	61.917064666748\\
1.50999999046326	62.0172996520996\\
1.51499998569489	62.1253547668457\\
1.51999998092651	62.2347373962402\\
1.52499997615814	62.3458290100098\\
1.52999997138977	62.4585418701172\\
1.5349999666214	62.5739250183105\\
1.53999996185303	62.6972999572754\\
1.54499995708466	62.830265045166\\
1.54999995231628	62.9686698913574\\
1.55499994754791	63.1056213378906\\
1.55999994277954	63.2393760681152\\
1.56500005722046	63.371166229248\\
1.57000005245209	63.1134490966797\\
1.57500004768372	61.7290191650391\\
1.58000004291534	62.0410461425781\\
1.58500003814697	63.0444297790527\\
1.5900000333786	63.8246917724609\\
1.59500002861023	64.215934753418\\
1.60000002384186	64.3116302490234\\
1.60500001907349	64.2426147460938\\
1.61000001430511	64.138427734375\\
1.61500000953674	64.0907974243164\\
1.62000000476837	64.1158142089844\\
1.625	64.2655792236328\\
1.62999999523163	64.4746704101563\\
1.63499999046326	64.6725692749023\\
1.63999998569489	64.7938690185547\\
1.64499998092651	64.8171844482422\\
1.64999997615814	64.9091491699219\\
1.65499997138977	65.0521545410156\\
1.6599999666214	65.1771850585938\\
1.66499996185303	65.2668685913086\\
1.66999995708466	65.3197326660156\\
1.67499995231628	65.3075332641602\\
1.67999994754791	65.2734832763672\\
1.68499994277954	65.2867202758789\\
1.69000005722046	65.3255844116211\\
1.69500005245209	65.3657455444336\\
1.70000004768372	65.4024658203125\\
1.70500004291534	65.4414215087891\\
1.71000003814697	65.479621887207\\
1.7150000333786	65.5204162597656\\
1.72000002861023	65.5611801147461\\
1.72500002384186	65.5985794067383\\
1.73000001907349	65.6271591186523\\
1.73500001430511	65.6459655761719\\
1.74000000953674	65.65869140625\\
1.74500000476837	65.6667938232422\\
1.75	65.6710739135742\\
1.75499999523163	65.6736373901367\\
1.75999999046326	65.6739501953125\\
1.76499998569489	65.6714401245117\\
1.76999998092651	65.6662826538086\\
1.77499997615814	65.6589279174805\\
1.77999997138977	65.6513519287109\\
1.7849999666214	65.644157409668\\
1.78999996185303	65.6356353759766\\
1.79499995708466	65.6248092651367\\
1.79999995231628	65.6125793457031\\
1.80499994754791	65.6011810302734\\
1.80999994277954	65.5918045043945\\
1.81500005722046	65.5852355957031\\
1.82000005245209	65.5822296142578\\
1.82500004768372	65.5825881958008\\
1.83000004291534	65.5854949951172\\
1.83500003814697	65.5897903442383\\
1.8400000333786	65.5942993164063\\
1.84500002861023	65.5989151000977\\
1.85000002384186	65.6037368774414\\
1.85500001907349	65.6109008789063\\
1.86000001430511	65.6206970214844\\
1.86500000953674	65.6325073242188\\
1.87000000476837	65.6464614868164\\
1.875	65.6628723144531\\
1.87999999523163	65.6839599609375\\
1.88499999046326	65.7129058837891\\
1.88999998569489	65.7502365112305\\
1.89499998092651	65.7934951782227\\
1.89999997615814	65.8370361328125\\
1.90499997138977	65.8806991577148\\
1.9099999666214	65.9252853393555\\
1.91499996185303	65.9725341796875\\
1.91999995708466	66.0229721069336\\
1.92499995231628	66.0770034790039\\
1.92999994754791	66.1373825073242\\
1.93499994277954	66.2057113647461\\
1.94000005722046	66.2814178466797\\
1.94500005245209	66.3617553710938\\
1.95000004768372	66.4446487426758\\
1.95500004291534	66.5303268432617\\
1.96000003814697	66.621711730957\\
1.9650000333786	66.723388671875\\
1.97000002861023	66.8369064331055\\
1.97500002384186	66.9570083618164\\
1.98000001907349	67.0704040527344\\
1.98500001430511	67.1717529296875\\
1.99000000953674	67.2630386352539\\
1.99500000476837	67.3537826538086\\
2	67.445915222168\\
2.00500011444092	67.5412368774414\\
2.00999999046326	67.6375427246094\\
2.01500010490417	67.7584915161133\\
2.01999998092651	67.8964538574219\\
2.02500009536743	68.0529479980469\\
2.02999997138977	68.2309036254883\\
2.03500008583069	68.4254608154297\\
2.03999996185303	68.5842819213867\\
2.04500007629395	68.6754455566406\\
2.04999995231628	68.7089538574219\\
2.0550000667572	68.7077255249023\\
2.05999994277954	68.6733551025391\\
2.06500005722046	68.5882339477539\\
2.0699999332428	68.438362121582\\
2.07500004768372	68.208984375\\
2.07999992370605	67.8897247314453\\
2.08500003814697	67.4883499145508\\
2.08999991416931	66.9775314331055\\
2.09500002861023	66.4211883544922\\
2.09999990463257	65.8390579223633\\
2.10500001907349	65.1714706420898\\
2.10999989509583	64.5058441162109\\
2.11500000953674	63.9086151123047\\
2.11999988555908	63.2772903442383\\
2.125	62.6502456665039\\
2.13000011444092	61.9855537414551\\
2.13499999046326	61.2983016967773\\
2.14000010490417	60.6053733825684\\
2.14499998092651	59.870792388916\\
2.15000009536743	59.2108268737793\\
2.15499997138977	58.6365509033203\\
2.16000008583069	58.053352355957\\
2.16499996185303	57.3880844116211\\
2.17000007629395	56.7158355712891\\
2.17499995231628	56.1278953552246\\
2.1800000667572	55.4671821594238\\
2.18499994277954	54.8183708190918\\
2.19000005722046	54.2070732116699\\
2.1949999332428	53.6602630615234\\
2.20000004768372	53.1807708740234\\
2.20499992370605	52.7620697021484\\
2.21000003814697	52.3967247009277\\
2.21499991416931	52.0852966308594\\
2.22000002861023	51.8390693664551\\
2.22499990463257	51.6755981445313\\
2.23000001907349	51.5840797424316\\
2.23499989509583	51.5163917541504\\
2.24000000953674	51.4217567443848\\
2.24499988555908	51.3465347290039\\
2.25	51.3170852661133\\
2.25500011444092	51.3067016601563\\
2.25999999046326	51.3213233947754\\
2.26500010490417	51.359992980957\\
2.26999998092651	51.4662399291992\\
2.27500009536743	51.6588325500488\\
2.27999997138977	51.9124870300293\\
2.28500008583069	52.2779998779297\\
2.28999996185303	52.7546653747559\\
2.29500007629395	53.2718963623047\\
2.29999995231628	53.7239761352539\\
2.3050000667572	54.1693687438965\\
2.30999994277954	54.6355781555176\\
2.31500005722046	55.15380859375\\
2.3199999332428	55.7000312805176\\
2.32500004768372	56.2631378173828\\
2.32999992370605	56.9292984008789\\
2.33500003814697	57.6452217102051\\
2.33999991416931	58.5084037780762\\
2.34500002861023	59.6449317932129\\
2.34999990463257	60.7507133483887\\
2.35500001907349	61.8018569946289\\
2.35999989509583	62.8447570800781\\
2.36500000953674	63.9740524291992\\
2.36999988555908	65.2144775390625\\
2.375	66.6086807250977\\
2.38000011444092	68.142707824707\\
2.38499999046326	69.921516418457\\
2.39000010490417	71.7118759155273\\
2.39499998092651	73.4688262939453\\
2.40000009536743	75.2031555175781\\
2.40499997138977	76.7361297607422\\
2.41000008583069	78.3429336547852\\
2.41499996185303	79.7728805541992\\
2.42000007629395	80.9985427856445\\
2.42499995231628	82.0293884277344\\
2.4300000667572	82.8881072998047\\
2.43499994277954	83.552734375\\
2.44000005722046	84.0347595214844\\
2.4449999332428	84.3416213989258\\
2.45000004768372	84.4158325195313\\
2.45499992370605	84.4525146484375\\
2.46000003814697	84.4119491577148\\
2.46499991416931	84.2845001220703\\
2.47000002861023	84.1548690795898\\
2.47499990463257	83.9963226318359\\
2.48000001907349	83.798454284668\\
2.48499989509583	83.5627059936523\\
2.49000000953674	83.2943572998047\\
2.49499988555908	83.0471420288086\\
2.5	82.8221206665039\\
2.50500011444092	82.5531005859375\\
2.50999999046326	82.1198577880859\\
2.51500010490417	81.6401519775391\\
2.51999998092651	81.0400238037109\\
2.52500009536743	80.3050537109375\\
2.52999997138977	79.4330215454102\\
2.53500008583069	78.4777221679688\\
2.53999996185303	77.36181640625\\
2.54500007629395	76.3038177490234\\
2.54999995231628	75.2509765625\\
2.5550000667572	74.2820892333984\\
2.55999994277954	73.1135101318359\\
2.56500005722046	72.011116027832\\
2.5699999332428	70.9424209594727\\
2.57500004768372	69.8576126098633\\
2.57999992370605	68.799690246582\\
2.58500003814697	67.655891418457\\
2.58999991416931	66.6017837524414\\
2.59500002861023	65.5730133056641\\
2.59999990463257	64.6520233154297\\
2.60500001907349	63.6356430053711\\
2.60999989509583	62.6397666931152\\
2.61500000953674	61.6651763916016\\
2.61999988555908	60.7100448608398\\
2.625	59.7850074768066\\
2.63000011444092	58.8990707397461\\
2.63499999046326	58.0542411804199\\
2.64000010490417	57.2316741943359\\
2.64499998092651	56.457405090332\\
2.65000009536743	55.7226943969727\\
2.65499997138977	54.9630851745605\\
2.66000008583069	54.1662101745605\\
2.66499996185303	53.3241767883301\\
2.67000007629395	52.402717590332\\
2.67499995231628	51.3914413452148\\
2.6800000667572	50.304370880127\\
2.68499994277954	49.3351173400879\\
2.69000005722046	48.373363494873\\
2.6949999332428	47.5162582397461\\
2.70000004768372	46.6032066345215\\
2.70499992370605	45.7708625793457\\
2.71000003814697	45.033935546875\\
2.71499991416931	44.3453330993652\\
2.72000002861023	43.7767677307129\\
2.72499990463257	43.3064880371094\\
2.73000001907349	42.911548614502\\
2.73499989509583	42.5839462280273\\
2.74000000953674	42.3231925964355\\
2.74499988555908	42.1442565917969\\
2.75	42.0830345153809\\
2.75500011444092	42.1143417358398\\
2.75999999046326	42.2426567077637\\
2.76500010490417	42.6017303466797\\
2.76999998092651	43.2388801574707\\
2.77500009536743	44.1667900085449\\
2.77999997138977	45.3270721435547\\
2.78500008583069	46.6296844482422\\
2.78999996185303	48.2260971069336\\
2.79500007629395	49.9566612243652\\
2.79999995231628	51.8393440246582\\
2.8050000667572	53.7726631164551\\
2.80999994277954	55.8369789123535\\
2.81500005722046	57.7697257995605\\
2.8199999332428	59.609733581543\\
2.82500004768372	61.3150253295898\\
2.82999992370605	62.7673301696777\\
2.83500003814697	64.2870178222656\\
2.83999991416931	65.7488403320313\\
2.84500002861023	67.1981811523438\\
2.84999990463257	68.8159790039063\\
2.85500001907349	70.3773727416992\\
2.85999989509583	71.9466552734375\\
2.86500000953674	73.3993606567383\\
2.86999988555908	75.1488647460938\\
2.875	76.6951751708984\\
2.88000011444092	78.1699905395508\\
2.88499999046326	79.6454849243164\\
2.89000010490417	81.0306015014648\\
2.89499998092651	82.6076736450195\\
2.90000009536743	84.0389633178711\\
2.90499997138977	85.1088180541992\\
2.91000008583069	85.8712921142578\\
2.91499996185303	86.3819198608398\\
2.92000007629395	86.675407409668\\
2.92499995231628	86.8366928100586\\
2.9300000667572	86.9261856079102\\
2.93499994277954	86.9771881103516\\
2.94000005722046	86.963493347168\\
2.9449999332428	86.8873825073242\\
2.95000004768372	86.7455139160156\\
2.95499992370605	86.4142608642578\\
2.96000003814697	85.8770904541016\\
2.96499991416931	85.1763381958008\\
2.97000002861023	84.432014465332\\
2.97499990463257	83.4697875976563\\
2.98000001907349	81.8471908569336\\
2.98499989509583	80.3837051391602\\
2.99000000953674	79.3161239624023\\
2.99499988555908	77.1087875366211\\
3	74.7026214599609\\
3.00500011444092	72.5093765258789\\
3.00999999046326	70.5152893066406\\
3.01500010490417	68.8612670898438\\
3.01999998092651	67.1899719238281\\
3.02500009536743	65.5732803344727\\
3.02999997138977	64.2219390869141\\
3.03500008583069	62.915340423584\\
3.03999996185303	61.5009498596191\\
3.04500007629395	59.8372993469238\\
3.04999995231628	58.0096855163574\\
3.0550000667572	56.042781829834\\
3.05999994277954	53.9724235534668\\
3.06500005722046	51.8241500854492\\
3.0699999332428	49.7138824462891\\
3.07500004768372	47.7697563171387\\
3.07999992370605	45.9780654907227\\
3.08500003814697	44.4003410339355\\
3.08999991416931	43.0766792297363\\
3.09500002861023	41.9691276550293\\
3.09999990463257	41.080150604248\\
3.10500001907349	40.3998260498047\\
3.10999989509583	39.8960380554199\\
3.11500000953674	39.6204795837402\\
3.11999988555908	39.6350555419922\\
3.125	39.9988250732422\\
3.13000011444092	40.9385566711426\\
3.13499999046326	42.6039009094238\\
3.14000010490417	45.4990692138672\\
3.14499998092651	49.2272033691406\\
3.15000009536743	53.668628692627\\
3.15499997138977	58.4805450439453\\
3.16000008583069	62.9560356140137\\
3.16499996185303	66.3673782348633\\
3.17000007629395	68.4790267944336\\
3.17499995231628	69.3925094604492\\
3.1800000667572	69.34130859375\\
3.18499994277954	68.771614074707\\
3.19000005722046	68.0009689331055\\
3.1949999332428	66.9769897460938\\
3.20000004768372	65.5979385375977\\
3.20499992370605	64.0701446533203\\
3.21000003814697	62.9363441467285\\
3.21499991416931	62.080249786377\\
3.22000002861023	61.8919830322266\\
3.22499990463257	62.3754081726074\\
3.23000001907349	63.3928909301758\\
3.23499989509583	64.8272018432617\\
3.24000000953674	66.7213287353516\\
3.24499988555908	68.6827621459961\\
3.25	70.7333374023438\\
3.25500011444092	72.6821212768555\\
3.25999999046326	74.5479278564453\\
3.26500010490417	76.2825317382813\\
3.26999998092651	77.81298828125\\
3.27500009536743	79.0393905639648\\
3.27999997138977	80.0176467895508\\
3.28500008583069	80.7517013549805\\
3.28999996185303	81.1611480712891\\
3.29500007629395	81.2868881225586\\
3.29999995231628	81.1707382202148\\
3.3050000667572	80.8411178588867\\
3.30999994277954	80.2532730102539\\
3.31500005722046	79.6041259765625\\
3.3199999332428	78.9643096923828\\
3.32500004768372	78.5818328857422\\
3.32999992370605	78.0957489013672\\
3.33500003814697	77.2499389648438\\
3.33999991416931	76.1852035522461\\
3.34500002861023	75.1652755737305\\
3.34999990463257	74.2044525146484\\
3.35500001907349	73.0820922851563\\
3.35999989509583	71.8234481811523\\
3.36500000953674	70.355339050293\\
3.36999988555908	68.7985382080078\\
3.375	66.7867431640625\\
3.38000011444092	64.5487976074219\\
3.38499999046326	62.0190162658691\\
3.39000010490417	59.5619812011719\\
3.39499998092651	57.1972618103027\\
3.40000009536743	54.8890991210938\\
3.40499997138977	52.7008743286133\\
3.41000008583069	50.8125038146973\\
3.41499996185303	48.8586578369141\\
3.42000007629395	47.1181297302246\\
3.42499995231628	45.5791435241699\\
3.4300000667572	44.1960716247559\\
3.43499994277954	43.0006065368652\\
3.44000005722046	42.0366592407227\\
3.4449999332428	41.3983917236328\\
3.45000004768372	41.2143745422363\\
3.45499992370605	41.6616325378418\\
3.46000003814697	43.0670928955078\\
3.46499991416931	45.9719200134277\\
3.47000002861023	50.3130302429199\\
3.47499990463257	54.3110618591309\\
3.48000001907349	58.1959037780762\\
3.48499989509583	61.6040115356445\\
3.49000000953674	64.366943359375\\
3.49499988555908	66.4452209472656\\
3.5	67.726432800293\\
3.50500011444092	68.571891784668\\
3.50999999046326	68.8864212036133\\
3.51500010490417	68.655876159668\\
3.51999998092651	68.2525634765625\\
3.52500009536743	67.5520324707031\\
3.52999997138977	67.174919128418\\
3.53500008583069	67.0515823364258\\
3.53999996185303	67.488655090332\\
3.54500007629395	69.0079345703125\\
3.54999995231628	71.3953018188477\\
3.5550000667572	74.5266189575195\\
3.55999994277954	76.8132629394531\\
3.56500005722046	77.6151885986328\\
3.5699999332428	77.2845306396484\\
3.57500004768372	77.2090225219727\\
3.57999992370605	77.84375\\
3.58500003814697	79.5248489379883\\
3.58999991416931	81.0552597045898\\
3.59500002861023	82.1545333862305\\
3.59999990463257	83.3251037597656\\
3.60500001907349	84.5602722167969\\
3.60999989509583	84.9738540649414\\
3.61500000953674	84.6925582885742\\
3.61999988555908	83.1988143920898\\
3.625	79.9928741455078\\
3.63000011444092	75.3558044433594\\
3.63499999046326	70.3874664306641\\
3.64000010490417	66.2963714599609\\
3.64499998092651	63.4235496520996\\
3.65000009536743	61.4460258483887\\
3.65499997138977	60.2314872741699\\
3.66000008583069	59.4852027893066\\
3.66499996185303	59.0868453979492\\
3.67000007629395	59.177734375\\
3.67499995231628	59.4405555725098\\
3.6800000667572	59.2649383544922\\
3.68499994277954	58.2322196960449\\
3.69000005722046	56.2314643859863\\
3.6949999332428	53.361888885498\\
3.70000004768372	49.6741485595703\\
3.70499992370605	45.8038482666016\\
3.71000003814697	42.1058502197266\\
3.71499991416931	39.2385559082031\\
3.72000002861023	37.0214157104492\\
3.72499990463257	35.8268661499023\\
3.73000001907349	35.5572242736816\\
3.73499989509583	36.0979614257813\\
3.74000000953674	37.3291015625\\
3.74499988555908	38.9749565124512\\
3.75	41.3929328918457\\
3.75500011444092	45.6668281555176\\
3.75999999046326	52.2191200256348\\
3.76500010490417	60.9609031677246\\
3.76999998092651	70.3853149414063\\
3.77500009536743	78.9336090087891\\
3.77999997138977	86.2141647338867\\
3.78500008583069	90.3300476074219\\
3.78999996185303	91.0380630493164\\
3.79500007629395	89.8952026367188\\
3.79999995231628	87.7005157470703\\
3.8050000667572	84.2033233642578\\
3.80999994277954	79.5856628417969\\
3.81500005722046	74.3575057983398\\
3.8199999332428	69.2209014892578\\
3.82500004768372	64.779411315918\\
3.82999992370605	61.709602355957\\
3.83500003814697	60.5070648193359\\
3.83999991416931	61.0045585632324\\
3.84500002861023	63.3471946716309\\
3.84999990463257	66.8921737670898\\
3.85500001907349	70.0661849975586\\
3.85999989509583	72.7394332885742\\
3.86500000953674	76.5250015258789\\
3.86999988555908	80.2814559936523\\
3.875	83.1594772338867\\
3.88000011444092	85.3593521118164\\
3.88499999046326	87.5648422241211\\
3.89000010490417	90.6158218383789\\
3.89499998092651	92.9769134521484\\
3.90000009536743	93.4398574829102\\
3.90499997138977	92.4104995727539\\
3.91000008583069	88.6841735839844\\
3.91499996185303	82.6686477661133\\
3.92000007629395	76.0398712158203\\
3.92499995231628	70.5678787231445\\
3.9300000667572	66.9560852050781\\
3.93499994277954	64.7666625976563\\
3.94000005722046	63.283878326416\\
3.9449999332428	62.0898323059082\\
3.95000004768372	61.1491279602051\\
3.95499992370605	60.674633026123\\
3.96000003814697	60.7979698181152\\
3.96499991416931	60.4693031311035\\
3.97000002861023	59.0705413818359\\
3.97499990463257	56.2845764160156\\
3.98000001907349	52.3387298583984\\
3.98499989509583	47.5824699401855\\
3.99000000953674	42.4753456115723\\
3.99499988555908	37.5711250305176\\
4	33.6046333312988\\
4.00500011444092	30.814302444458\\
4.01000022888184	29.1427326202393\\
4.0149998664856	28.7920417785645\\
4.01999998092651	29.6027679443359\\
4.02500009536743	31.2194709777832\\
4.03000020980835	33.7792129516602\\
4.03499984741211	38.0514869689941\\
4.03999996185303	44.8179779052734\\
4.04500007629395	53.755054473877\\
4.05000019073486	63.3069343566895\\
4.05499982833862	72.1533737182617\\
4.05999994277954	78.6229858398438\\
4.06500005722046	82.194694519043\\
4.07000017166138	83.9380722045898\\
4.07499980926514	84.2111892700195\\
4.07999992370605	83.4595489501953\\
4.08500003814697	81.1190872192383\\
4.09000015258789	77.6002960205078\\
4.09499979019165	73.8763122558594\\
4.09999990463257	71.0492401123047\\
4.10500001907349	71.4267730712891\\
4.1100001335144	77.1618957519531\\
4.11499977111816	84.9041213989258\\
4.11999988555908	88.390495300293\\
4.125	82.1005783081055\\
4.13000011444092	77.2587738037109\\
4.13500022888184	75.1635818481445\\
4.1399998664856	76.4441223144531\\
4.14499998092651	77.2553176879883\\
4.15000009536743	77.8289566040039\\
4.15500020980835	80.0418014526367\\
4.15999984741211	84.3068695068359\\
4.16499996185303	91.3820343017578\\
4.17000007629395	100.561721801758\\
4.17500019073486	107.993980407715\\
4.17999982833862	107.181541442871\\
4.18499994277954	97.5104904174805\\
4.19000005722046	82.0913009643555\\
4.19500017166138	64.9702682495117\\
4.19999980926514	50.8312187194824\\
4.20499992370605	43.9547424316406\\
4.21000003814697	42.864860534668\\
4.21500015258789	43.9609718322754\\
4.21999979019165	45.6061897277832\\
4.22499990463257	47.6940879821777\\
4.23000001907349	50.2813110351563\\
4.2350001335144	53.7653198242188\\
4.23999977111816	56.6183128356934\\
4.24499988555908	56.8767890930176\\
4.25	53.4037857055664\\
4.25500011444092	47.3216781616211\\
4.26000022888184	39.2881851196289\\
4.2649998664856	30.7564697265625\\
4.26999998092651	23.4445171356201\\
4.27500009536743	18.5063667297363\\
4.28000020980835	16.7975997924805\\
4.28499984741211	19.0849075317383\\
4.28999996185303	23.4406471252441\\
4.29500007629395	29.0549945831299\\
4.30000019073486	36.9611625671387\\
4.30499982833862	50.0061874389648\\
4.30999994277954	68.1240310668945\\
4.31500005722046	86.8039245605469\\
4.32000017166138	101.858573913574\\
4.32499980926514	110.70565032959\\
4.32999992370605	113.99503326416\\
4.33500003814697	114.408798217773\\
4.34000015258789	111.667053222656\\
4.34499979019165	104.974502563477\\
4.34999990463257	95.3994750976563\\
4.35500001907349	84.7392272949219\\
4.3600001335144	74.9164276123047\\
4.36499977111816	68.3603897094727\\
4.36999988555908	69.1788635253906\\
4.375	81.1848602294922\\
4.38000011444092	94.9386291503906\\
4.38500022888184	91.3496398925781\\
4.3899998664856	66.7111968994141\\
4.39499998092651	39.1350402832031\\
4.40000009536743	52.0180625915527\\
4.40500020980835	66.660774230957\\
4.40999984741211	75.6476745605469\\
4.41499996185303	81.9358749389648\\
4.42000007629395	89.1412124633789\\
4.42500019073486	98.9837799072266\\
4.42999982833862	114.097129821777\\
4.43499994277954	127.797302246094\\
4.44000005722046	130.272842407227\\
4.44500017166138	119.232048034668\\
4.44999980926514	98.6469802856445\\
4.45499992370605	74.0430755615234\\
4.46000003814697	52.0872344970703\\
4.46500015258789	40.6498947143555\\
4.46999979019165	37.8343391418457\\
4.47499990463257	38.9225883483887\\
4.48000001907349	41.1820449829102\\
4.4850001335144	44.2392578125\\
4.48999977111816	47.9659461975098\\
4.49499988555908	52.5899238586426\\
4.5	57.3375282287598\\
4.50500011444092	58.8221855163574\\
4.51000022888184	55.4599533081055\\
4.5149998664856	47.5940246582031\\
4.51999998092651	36.9815673828125\\
4.52500009536743	25.4962959289551\\
4.53000020980835	15.5747184753418\\
4.53499984741211	9.03302574157715\\
4.53999996185303	7.01027202606201\\
4.54500007629395	10.2322750091553\\
4.55000019073486	16.1710720062256\\
4.55499982833862	24.0778617858887\\
4.55999994277954	37.4699897766113\\
4.56500005722046	63.4537391662598\\
4.57000017166138	91.081901550293\\
4.57499980926514	110.892395019531\\
4.57999992370605	120.527603149414\\
4.58500003814697	120.778167724609\\
4.59000015258789	113.852401733398\\
4.59499979019165	103.336402893066\\
4.59999990463257	92.7426071166992\\
4.60500001907349	79.652099609375\\
4.6100001335144	63.5602989196777\\
4.61499977111816	44.255184173584\\
4.61999988555908	33.3011054992676\\
4.625	73.8254547119141\\
4.63000011444092	142.950302124023\\
4.63500022888184	202.37776184082\\
4.6399998664856	209.501007080078\\
4.64499998092651	165.979644775391\\
4.65000009536743	94.108772277832\\
4.65500020980835	24.3990478515625\\
4.65999984741211	-39.3318519592285\\
4.66499996185303	-83.9083786010742\\
4.67000007629395	-26.8800392150879\\
4.67500019073486	32.0814361572266\\
4.67999982833862	72.382682800293\\
4.68499994277954	104.61181640625\\
4.69000005722046	139.203094482422\\
4.69500017166138	169.493408203125\\
4.69999980926514	175.380966186523\\
4.70499992370605	153.591354370117\\
4.71000003814697	112.993156433105\\
4.71500015258789	65.4368515014648\\
4.71999979019165	23.3928928375244\\
4.72499990463257	1.19342684745789\\
4.73000001907349	-0.686737656593323\\
4.7350001335144	4.89217901229858\\
4.73999977111816	13.3215084075928\\
4.74499988555908	23.3780345916748\\
4.75	33.9057884216309\\
4.75500011444092	43.8745422363281\\
4.76000022888184	54.8972473144531\\
4.7649998664856	60.3435821533203\\
4.76999998092651	57.1170654296875\\
4.77500009536743	44.5440635681152\\
4.78000020980835	27.6614017486572\\
4.78499984741211	10.0699901580811\\
4.78999996185303	-1.51013255119324\\
4.79500007629395	-7.72052907943726\\
4.80000019073486	-10.0934858322144\\
4.80499982833862	4.90682649612427\\
4.80999994277954	34.104419708252\\
4.81500005722046	95.3024291992188\\
4.82000017166138	144.394439697266\\
4.82499980926514	167.028198242188\\
4.82999992370605	168.521469116211\\
4.83500003814697	155.519638061523\\
4.84000015258789	134.660720825195\\
4.84499979019165	110.314521789551\\
4.84999990463257	80.7961730957031\\
4.85500001907349	47.2072830200195\\
4.8600001335144	10.1545152664185\\
4.86499977111816	-24.1842060089111\\
4.86999988555908	-23.3426856994629\\
4.875	76.9485549926758\\
4.88000011444092	184.421478271484\\
4.88500022888184	236.102905273438\\
4.8899998664856	222.427398681641\\
4.89499998092651	161.257751464844\\
4.90000009536743	80.1420288085938\\
4.90500020980835	8.01033306121826\\
4.90999984741211	-57.8422889709473\\
4.91499996185303	-114.185821533203\\
4.92000007629395	-135.940093994141\\
4.92500019073486	-32.1807556152344\\
4.92999982833862	67.0636291503906\\
4.93499994277954	137.361267089844\\
4.94000005722046	192.231063842773\\
4.94500017166138	234.840118408203\\
4.94999980926514	260.160583496094\\
4.95499992370605	251.003631591797\\
4.96000003814697	210.125701904297\\
4.96500015258789	151.668548583984\\
4.96999979019165	91.0716934204102\\
4.97499990463257	42.711296081543\\
4.98000001907349	24.4970951080322\\
4.9850001335144	25.6628856658936\\
4.98999977111816	32.8986854553223\\
4.99499988555908	42.1588172912598\\
5	52.7155075073242\\
5.00500011444092	63.9110946655273\\
5.01000022888184	74.8972091674805\\
5.0149998664856	86.8044738769531\\
5.01999998092651	95.6152038574219\\
5.02500009536743	96.8754196166992\\
5.03000020980835	87.4503402709961\\
5.03499984741211	70.1357879638672\\
5.03999996185303	49.9589996337891\\
5.04500007629395	32.6608352661133\\
5.05000019073486	18.4640979766846\\
5.05499982833862	7.9794750213623\\
5.05999994277954	0.972096800804138\\
5.06500005722046	-1.95562124252319\\
5.07000017166138	-2.16487765312195\\
5.07499980926514	-0.830978989601135\\
5.07999992370605	0.945532381534576\\
5.08500003814697	5.17418909072876\\
5.09000015258789	20.8385162353516\\
5.09499979019165	32.9726524353027\\
5.09999990463257	36.9427223205566\\
5.10500001907349	35.574577331543\\
5.1100001335144	36.4559898376465\\
5.11499977111816	37.5675086975098\\
5.11999988555908	37.0474433898926\\
5.125	35.2087135314941\\
5.13000011444092	32.3352127075195\\
5.13500022888184	47.7832221984863\\
5.1399998664856	131.813812255859\\
5.14499998092651	165.041351318359\\
5.15000009536743	144.349502563477\\
5.15500020980835	95.8686676025391\\
5.15999984741211	50.4581527709961\\
5.16499996185303	63.750415802002\\
5.17000007629395	80.4497833251953\\
5.17500019073486	89.27783203125\\
5.17999982833862	95.1530303955078\\
5.18499994277954	102.08277130127\\
5.19000005722046	112.484619140625\\
5.19500017166138	120.141159057617\\
5.19999980926514	121.634307861328\\
5.20499992370605	114.348419189453\\
5.21000003814697	100.48558807373\\
5.21500015258789	84.2379760742188\\
5.21999979019165	69.6565170288086\\
5.22499990463257	59.464916229248\\
5.23000001907349	56.2648811340332\\
5.2350001335144	59.5739860534668\\
5.23999977111816	64.3972244262695\\
5.24499988555908	68.2152481079102\\
5.25	71.0464248657227\\
5.25500011444092	73.2291030883789\\
5.26000022888184	74.8978958129883\\
5.2649998664856	77.1675033569336\\
5.26999998092651	78.4151382446289\\
5.27500009536743	78.1626815795898\\
5.28000020980835	75.191032409668\\
5.28499984741211	69.7527008056641\\
5.28999996185303	62.783748626709\\
5.29500007629395	55.6021957397461\\
5.30000019073486	49.4545783996582\\
5.30499982833862	45.3422431945801\\
5.30999994277954	43.0440483093262\\
5.31500005722046	42.0773735046387\\
5.32000017166138	42.0298118591309\\
5.32499980926514	43.7863693237305\\
5.32999992370605	45.8150367736816\\
5.33500003814697	46.9781379699707\\
5.34000015258789	47.2850189208984\\
5.34499979019165	47.0196762084961\\
5.34999990463257	46.7962608337402\\
5.35500001907349	45.6983032226563\\
5.3600001335144	43.2777214050293\\
5.36499977111816	39.5378684997559\\
5.36999988555908	34.7892265319824\\
5.375	29.4481773376465\\
5.38000011444092	24.0891666412354\\
5.38500022888184	19.3417797088623\\
5.3899998664856	15.5799856185913\\
5.39499998092651	13.237232208252\\
5.40000009536743	12.4610528945923\\
5.40500020980835	12.2437801361084\\
5.40999984741211	12.1732921600342\\
5.41499996185303	12.7267684936523\\
5.42000007629395	13.1440486907959\\
5.42500019073486	12.7390480041504\\
5.42999982833862	11.6010103225708\\
5.43499994277954	11.1359634399414\\
5.44000005722046	13.0681123733521\\
5.44500017166138	15.4166345596313\\
5.44999980926514	18.9365634918213\\
5.45499992370605	21.9662837982178\\
5.46000003814697	23.2829933166504\\
5.46500015258789	23.048490524292\\
5.46999979019165	21.7843761444092\\
5.47499990463257	20.0880107879639\\
5.48000001907349	18.3181018829346\\
5.4850001335144	16.576940536499\\
5.48999977111816	14.9484367370605\\
5.49499988555908	13.4726324081421\\
5.5	12.1568460464478\\
5.50500011444092	10.9812431335449\\
5.51000022888184	9.95047569274902\\
5.5149998664856	9.05428123474121\\
5.51999998092651	10.1525630950928\\
5.52500009536743	17.6063938140869\\
5.53000020980835	25.282621383667\\
5.53499984741211	26.3623943328857\\
5.53999996185303	23.0369987487793\\
5.54500007629395	18.1570110321045\\
5.55000019073486	12.8271226882935\\
5.55499982833862	8.04466533660889\\
5.55999994277954	4.46115589141846\\
5.56500005722046	1.88783252239227\\
5.57000017166138	0.354039251804352\\
5.57499980926514	-0.783860266208649\\
5.57999992370605	-1.53600025177002\\
5.58500003814697	-1.77474570274353\\
5.59000015258789	-1.51368534564972\\
5.59499979019165	-0.765204727649689\\
5.59999990463257	0.471465855836868\\
5.60500001907349	2.18896007537842\\
5.6100001335144	4.28011274337769\\
5.61499977111816	6.64467859268188\\
5.61999988555908	9.10853958129883\\
5.625	11.4813575744629\\
5.63000011444092	13.6335315704346\\
5.63500022888184	15.4339437484741\\
5.6399998664856	16.8316879272461\\
5.64499998092651	17.8190460205078\\
5.65000009536743	18.3632202148438\\
5.65500020980835	18.5710964202881\\
5.65999984741211	18.5105152130127\\
5.66499996185303	18.2502307891846\\
5.67000007629395	18.3261528015137\\
5.67500019073486	22.2961978912354\\
5.67999982833862	47.3717155456543\\
5.68499994277954	60.1822738647461\\
5.69000005722046	61.7361869812012\\
5.69500017166138	69.1031341552734\\
5.69999980926514	67.3271026611328\\
5.70499992370605	56.3642768859863\\
5.71000003814697	40.9111480712891\\
5.71500015258789	26.128927230835\\
5.71999979019165	16.7791843414307\\
5.72499990463257	12.8972120285034\\
5.73000001907349	11.0677471160889\\
5.7350001335144	10.5208406448364\\
5.73999977111816	11.3120183944702\\
5.74499988555908	13.647346496582\\
5.75	17.4557304382324\\
5.75500011444092	22.4557914733887\\
5.76000022888184	28.313009262085\\
5.7649998664856	35.2925796508789\\
5.76999998092651	42.4065399169922\\
5.77500009536743	49.5700416564941\\
5.78000020980835	56.925464630127\\
5.78499984741211	63.950740814209\\
5.78999996185303	71.2018432617188\\
5.79500007629395	78.6846923828125\\
5.80000019073486	86.3299713134766\\
5.80499982833862	93.4421005249023\\
5.80999994277954	100.502487182617\\
5.81500005722046	107.127159118652\\
5.82000017166138	113.24633026123\\
5.82499980926514	118.795043945313\\
5.82999992370605	123.890533447266\\
5.83500003814697	127.905067443848\\
5.84000015258789	130.586700439453\\
5.84499979019165	131.718978881836\\
5.84999990463257	131.229858398438\\
5.85500001907349	129.27653503418\\
5.8600001335144	127.025405883789\\
5.86499977111816	121.637855529785\\
5.86999988555908	111.310333251953\\
5.875	95.3773574829102\\
5.88000011444092	74.0922012329102\\
5.88500022888184	49.31884765625\\
5.8899998664856	26.3759460449219\\
5.89499998092651	25.084192276001\\
5.90000009536743	72.6195526123047\\
5.90500020980835	135.54133605957\\
5.90999984741211	202.245071411133\\
5.91499996185303	274.070831298828\\
5.92000007629395	337.792724609375\\
5.92500019073486	376.502471923828\\
5.92999982833862	381.497436523438\\
5.93499994277954	353.686553955078\\
5.94000005722046	301.348907470703\\
5.94500017166138	233.977630615234\\
5.94999980926514	160.413055419922\\
5.95499992370605	88.981575012207\\
5.96000003814697	25.3406410217285\\
5.96500015258789	-21.6000499725342\\
5.96999979019165	-60.9108047485352\\
5.97499990463257	-98.2411422729492\\
5.98000001907349	-78.5232086181641\\
5.9850001335144	165.286972045898\\
5.98999977111816	328.993713378906\\
5.99499988555908	414.750152587891\\
6	440.417572021484\\
6.00500011444092	438.637298583984\\
6.01000022888184	456.599700927734\\
6.0149998664856	461.897644042969\\
6.01999998092651	428.718841552734\\
6.02500009536743	360.858093261719\\
6.03000020980835	274.089569091797\\
6.03499984741211	184.840957641602\\
6.03999996185303	107.417190551758\\
6.04500007629395	56.9161720275879\\
6.05000019073486	57.9048385620117\\
6.05499982833862	81.6680908203125\\
6.05999994277954	106.511734008789\\
6.06500005722046	127.692878723145\\
6.07000017166138	145.735214233398\\
6.07499980926514	161.088119506836\\
6.07999992370605	174.031280517578\\
6.08500003814697	186.837768554688\\
6.09000015258789	200.326477050781\\
6.09499979019165	207.34098815918\\
6.09999990463257	197.934204101563\\
6.10500001907349	172.651153564453\\
6.1100001335144	138.283813476563\\
6.11499977111816	103.346748352051\\
6.11999988555908	77.0151672363281\\
6.125	58.4585189819336\\
6.13000011444092	47.6211585998535\\
6.13500022888184	47.3393516540527\\
6.1399998664856	51.3563385009766\\
6.14499998092651	55.9274406433105\\
6.15000009536743	60.1284141540527\\
6.15500020980835	64.1522750854492\\
6.15999984741211	70.0011138916016\\
6.16499996185303	74.6583099365234\\
6.17000007629395	75.1308822631836\\
6.17500019073486	70.5335922241211\\
6.17999982833862	61.3431015014648\\
6.18499994277954	48.9107437133789\\
6.19000005722046	37.5742149353027\\
6.19500017166138	40.4131851196289\\
6.19999980926514	44.0287437438965\\
6.20499992370605	42.2204742431641\\
6.21000003814697	36.6889114379883\\
6.21500015258789	32.1123046875\\
6.21999979019165	35.655086517334\\
6.22499990463257	41.2673797607422\\
6.23000001907349	44.2269706726074\\
6.2350001335144	43.9089393615723\\
6.23999977111816	41.6236534118652\\
6.24499988555908	38.1309547424316\\
6.25	34.1350326538086\\
6.25500011444092	30.1868896484375\\
6.26000022888184	26.5098190307617\\
6.2649998664856	23.1832828521729\\
6.26999998092651	20.2212905883789\\
6.27500009536743	17.6138134002686\\
6.28000020980835	15.3312616348267\\
6.28499984741211	13.3365697860718\\
6.28999996185303	11.5964040756226\\
6.29500007629395	10.0795907974243\\
6.30000019073486	8.75443840026855\\
6.30499982833862	7.59923791885376\\
6.30999994277954	6.59221458435059\\
6.31500005722046	5.71394491195679\\
6.32000017166138	4.94971227645874\\
6.32499980926514	4.28384494781494\\
6.32999992370605	3.70432209968567\\
6.33500003814697	3.20123076438904\\
6.34000015258789	2.76399707794189\\
6.34499979019165	2.38440752029419\\
6.34999990463257	2.05542302131653\\
6.35500001907349	1.77113664150238\\
6.3600001335144	5.56419706344604\\
6.36499977111816	37.6908073425293\\
6.36999988555908	58.103385925293\\
6.375	62.4128265380859\\
6.38000011444092	57.2109413146973\\
6.38500022888184	50.7199172973633\\
6.3899998664856	59.9711112976074\\
6.39499998092651	98.9516067504883\\
6.40000009536743	116.647872924805\\
6.40500020980835	113.627700805664\\
6.40999984741211	103.30069732666\\
6.41499996185303	95.3634719848633\\
6.42000007629395	91.0085983276367\\
6.42500019073486	90.0146942138672\\
6.42999982833862	92.1333465576172\\
6.43499994277954	96.7186889648438\\
6.44000005722046	102.848861694336\\
6.44500017166138	109.611801147461\\
6.44999980926514	116.339920043945\\
6.45499992370605	122.274475097656\\
6.46000003814697	127.012954711914\\
6.46500015258789	130.698501586914\\
6.46999979019165	133.246643066406\\
6.47499990463257	134.91877746582\\
6.48000001907349	135.907089233398\\
6.4850001335144	136.544723510742\\
6.48999977111816	137.124099731445\\
6.49499988555908	137.90071105957\\
6.5	138.975860595703\\
6.50500011444092	140.464538574219\\
6.51000022888184	142.286666870117\\
6.5149998664856	144.438919067383\\
6.51999998092651	146.823257446289\\
6.52500009536743	149.217666625977\\
6.53000020980835	151.539443969727\\
6.53499984741211	153.624114990234\\
6.53999996185303	155.445831298828\\
6.54500007629395	156.963729858398\\
6.55000019073486	158.027465820313\\
6.55499982833862	158.797515869141\\
6.55999994277954	159.480331420898\\
6.56500005722046	160.039733886719\\
6.57000017166138	160.377075195313\\
6.57499980926514	160.352340698242\\
6.57999992370605	159.998809814453\\
6.58500003814697	159.700424194336\\
6.59000015258789	159.392532348633\\
6.59499979019165	159.146911621094\\
6.59999990463257	159.034255981445\\
6.60500001907349	159.050354003906\\
6.6100001335144	159.146850585938\\
6.61499977111816	159.375946044922\\
6.61999988555908	159.682998657227\\
6.625	159.955062866211\\
6.63000011444092	160.247497558594\\
6.63500022888184	160.504928588867\\
6.6399998664856	160.779861450195\\
6.64499998092651	161.133804321289\\
6.65000009536743	161.393127441406\\
6.65500020980835	161.691207885742\\
6.65999984741211	162.08203125\\
6.66499996185303	162.40217590332\\
6.67000007629395	162.829360961914\\
6.67500019073486	163.476257324219\\
6.67999982833862	164.080459594727\\
6.68499994277954	164.564041137695\\
6.69000005722046	164.703231811523\\
6.69500017166138	164.708190917969\\
6.69999980926514	165.209671020508\\
6.70499992370605	166.084655761719\\
6.71000003814697	166.674911499023\\
6.71500015258789	166.797348022461\\
6.71999979019165	166.467330932617\\
6.72499990463257	165.885528564453\\
6.73000001907349	165.131118774414\\
6.7350001335144	164.202529907227\\
6.73999977111816	163.137268066406\\
6.74499988555908	162.077590942383\\
6.75	161.076522827148\\
6.75500011444092	160.141708374023\\
6.76000022888184	159.177047729492\\
6.7649998664856	158.124786376953\\
6.76999998092651	157.255935668945\\
6.77500009536743	156.546371459961\\
6.78000020980835	155.847518920898\\
6.78499984741211	155.292358398438\\
6.78999996185303	154.857269287109\\
6.79500007629395	154.476547241211\\
6.80000019073486	154.108688354492\\
6.80499982833862	153.702484130859\\
6.80999994277954	153.325576782227\\
6.81500005722046	152.968673706055\\
6.82000017166138	152.560668945313\\
6.82499980926514	152.136627197266\\
6.82999992370605	151.656188964844\\
6.83500003814697	151.106628417969\\
6.84000015258789	150.501190185547\\
6.84499979019165	149.735809326172\\
6.84999990463257	148.854537963867\\
6.85500001907349	147.800308227539\\
6.8600001335144	146.587203979492\\
6.86499977111816	145.156845092773\\
6.86999988555908	143.360549926758\\
6.875	141.52619934082\\
6.88000011444092	139.775192260742\\
6.88500022888184	137.895431518555\\
6.8899998664856	135.894454956055\\
6.89499998092651	133.771728515625\\
6.90000009536743	131.751770019531\\
6.90500020980835	129.411773681641\\
6.90999984741211	126.797233581543\\
6.91499996185303	124.190132141113\\
6.92000007629395	121.584213256836\\
6.92500019073486	118.883201599121\\
6.92999982833862	116.100486755371\\
6.93499994277954	113.272262573242\\
6.94000005722046	110.426239013672\\
6.94500017166138	107.596382141113\\
6.94999980926514	104.771621704102\\
6.95499992370605	101.87629699707\\
6.96000003814697	98.8523559570313\\
6.96500015258789	95.9722061157227\\
6.96999979019165	93.271484375\\
6.97499990463257	90.5152740478516\\
6.98000001907349	87.7876129150391\\
6.9850001335144	85.0713653564453\\
6.98999977111816	82.3797225952148\\
6.99499988555908	79.6347274780273\\
7	76.8640060424805\\
7.00500011444092	74.166130065918\\
7.01000022888184	71.4906234741211\\
7.0149998664856	68.8399505615234\\
7.01999998092651	66.2285995483398\\
7.02500009536743	63.6866874694824\\
7.03000020980835	61.2401275634766\\
7.03499984741211	58.8905487060547\\
7.03999996185303	56.6351432800293\\
7.04500007629395	54.4784927368164\\
7.05000019073486	52.4262733459473\\
7.05499982833862	50.484790802002\\
7.05999994277954	48.6488075256348\\
7.06500005722046	46.8959274291992\\
7.07000017166138	45.2567367553711\\
7.07499980926514	43.6543846130371\\
7.07999992370605	42.2005386352539\\
7.08500003814697	40.8152923583984\\
7.09000015258789	39.5063781738281\\
7.09499979019165	38.2871208190918\\
7.09999990463257	37.1571884155273\\
7.10500001907349	36.1050834655762\\
7.1100001335144	35.1366539001465\\
7.11499977111816	34.2266273498535\\
7.11999988555908	33.4688606262207\\
7.125	32.8090705871582\\
7.13000011444092	32.238655090332\\
7.13500022888184	31.7696475982666\\
7.1399998664856	31.3965873718262\\
7.14499998092651	31.1075763702393\\
7.15000009536743	30.8991832733154\\
7.15500020980835	30.8303604125977\\
7.15999984741211	30.9613304138184\\
7.16499996185303	31.1739158630371\\
7.17000007629395	31.3107147216797\\
7.17500019073486	31.307279586792\\
7.17999982833862	31.2708473205566\\
7.18499994277954	31.3029174804688\\
7.19000005722046	31.4143962860107\\
7.19500017166138	31.6035537719727\\
7.19999980926514	31.8637561798096\\
7.20499992370605	32.1781845092773\\
7.21000003814697	32.5298233032227\\
7.21500015258789	32.9344215393066\\
7.21999979019165	33.372013092041\\
7.22499990463257	33.8087043762207\\
7.23000001907349	34.2686157226563\\
7.2350001335144	34.7411918640137\\
7.23999977111816	35.2170143127441\\
7.24499988555908	35.6956596374512\\
7.25	36.1623802185059\\
7.25500011444092	36.6160545349121\\
7.26000022888184	37.0674896240234\\
7.2649998664856	37.511173248291\\
7.26999998092651	37.9537734985352\\
7.27500009536743	38.3959503173828\\
7.28000020980835	38.8210220336914\\
7.28499984741211	39.2238578796387\\
7.28999996185303	39.6056861877441\\
7.29500007629395	39.9662170410156\\
7.30000019073486	40.303092956543\\
7.30499982833862	40.6001777648926\\
7.30999994277954	40.8608665466309\\
7.31500005722046	41.0983505249023\\
7.32000017166138	41.3148155212402\\
7.32499980926514	41.5048179626465\\
7.32999992370605	41.6646614074707\\
7.33500003814697	41.787425994873\\
7.34000015258789	41.8719367980957\\
7.34499979019165	41.9250869750977\\
7.34999990463257	41.9487686157227\\
7.35500001907349	41.9416542053223\\
7.3600001335144	41.9063377380371\\
7.36499977111816	41.8533554077148\\
7.36999988555908	41.7902488708496\\
7.375	41.7174758911133\\
7.38000011444092	41.6315994262695\\
7.38500022888184	41.5250511169434\\
7.3899998664856	41.3623428344727\\
7.39499998092651	41.1037673950195\\
7.40000009536743	40.7561531066895\\
7.40500020980835	40.379825592041\\
7.40999984741211	39.9948539733887\\
7.41499996185303	39.6058464050293\\
7.42000007629395	39.2187767028809\\
7.42500019073486	38.8353500366211\\
7.42999982833862	38.455379486084\\
7.43499994277954	38.0739860534668\\
7.44000005722046	37.6770362854004\\
7.44500017166138	37.2601470947266\\
7.44999980926514	36.8280830383301\\
7.45499992370605	36.3821678161621\\
7.46000003814697	35.9259605407715\\
7.46500015258789	35.4643020629883\\
7.46999979019165	35.0042724609375\\
7.47499990463257	34.5562934875488\\
7.48000001907349	34.1261901855469\\
7.4850001335144	33.7092170715332\\
7.48999977111816	33.302116394043\\
7.49499988555908	32.9008331298828\\
7.5	32.4997291564941\\
7.50500011444092	32.0964889526367\\
7.51000022888184	31.6924171447754\\
7.5149998664856	31.2930316925049\\
7.51999998092651	30.9158554077148\\
7.52500009536743	30.5942707061768\\
7.53000020980835	30.3217449188232\\
7.53499984741211	30.0176467895508\\
7.53999996185303	29.7271060943604\\
7.54500007629395	29.4498481750488\\
7.55000019073486	29.2009048461914\\
7.55499982833862	28.9779224395752\\
7.55999994277954	28.7779941558838\\
7.56500005722046	28.6036262512207\\
7.57000017166138	28.4532642364502\\
7.57499980926514	28.3218727111816\\
7.57999992370605	28.2085494995117\\
7.58500003814697	28.1143245697021\\
7.59000015258789	28.0394763946533\\
7.59499979019165	27.9852428436279\\
7.59999990463257	27.9512710571289\\
7.60500001907349	27.937536239624\\
7.6100001335144	27.9507389068604\\
7.61499977111816	28.0068187713623\\
7.61999988555908	28.1147727966309\\
7.625	28.2371196746826\\
7.63000011444092	28.3257656097412\\
7.63500022888184	28.3888835906982\\
7.6399998664856	28.4567852020264\\
7.64499998092651	28.5468521118164\\
7.65000009536743	28.6673488616943\\
7.65500020980835	28.8085918426514\\
7.65999984741211	28.9653301239014\\
7.66499996185303	29.1354064941406\\
7.67000007629395	29.3153743743896\\
7.67500019073486	29.5011253356934\\
7.67999982833862	29.6918754577637\\
7.68499994277954	29.8902778625488\\
7.69000005722046	30.0980796813965\\
7.69500017166138	30.3141593933105\\
7.69999980926514	30.5296859741211\\
7.70499992370605	30.7401828765869\\
7.71000003814697	30.9473896026611\\
7.71500015258789	31.1592216491699\\
7.71999979019165	31.3871326446533\\
7.72499990463257	31.6221885681152\\
7.73000001907349	31.8482246398926\\
7.7350001335144	32.0602684020996\\
7.73999977111816	32.2628211975098\\
7.74499988555908	32.4735298156738\\
7.75	32.6869735717773\\
7.75500011444092	32.8863677978516\\
7.76000022888184	33.0754737854004\\
7.7649998664856	33.262638092041\\
7.76999998092651	33.4437255859375\\
7.77500009536743	33.6130790710449\\
7.78000020980835	33.7692718505859\\
7.78499984741211	33.9160194396973\\
7.78999996185303	34.0609359741211\\
7.79500007629395	34.2042121887207\\
7.80000019073486	34.3445701599121\\
7.80499982833862	34.4763870239258\\
7.80999994277954	34.5978355407715\\
7.81500005722046	34.7095680236816\\
7.82000017166138	34.812614440918\\
7.82499980926514	34.9072914123535\\
7.82999992370605	34.9936180114746\\
7.83500003814697	35.0726051330566\\
7.84000015258789	35.1458969116211\\
7.84499979019165	35.2137603759766\\
7.84999990463257	35.2762031555176\\
7.85500001907349	35.3355712890625\\
7.8600001335144	35.3934860229492\\
7.86499977111816	35.4489097595215\\
7.86999988555908	35.497631072998\\
7.875	35.5380554199219\\
7.88000011444092	35.5697975158691\\
7.88500022888184	35.5970916748047\\
7.8899998664856	35.6239891052246\\
7.89499998092651	35.652271270752\\
7.90000009536743	35.6816711425781\\
7.90500020980835	35.7110366821289\\
7.90999984741211	35.740234375\\
7.91499996185303	35.7673149108887\\
7.92000007629395	35.7878456115723\\
7.92500019073486	35.7995948791504\\
7.92999982833862	35.8040580749512\\
7.93499994277954	35.8123168945313\\
7.94000005722046	35.8264541625977\\
7.94500017166138	35.8447647094727\\
7.94999980926514	35.8729362487793\\
7.95499992370605	35.9148178100586\\
7.96000003814697	35.9683113098145\\
7.96500015258789	36.0331077575684\\
7.96999979019165	36.112663269043\\
7.97499990463257	36.206413269043\\
7.98000001907349	36.3093795776367\\
7.9850001335144	36.4200401306152\\
7.98999977111816	36.5388793945313\\
7.99499988555908	36.664005279541\\
8	36.7963829040527\\
8.00500011444092	36.9370460510254\\
8.01000022888184	37.0863800048828\\
8.01500034332275	37.2513046264648\\
8.02000045776367	37.4368705749512\\
8.02499961853027	37.6436233520508\\
8.02999973297119	37.8652381896973\\
8.03499984741211	38.090202331543\\
8.03999996185303	38.336109161377\\
8.04500007629395	38.5819129943848\\
8.05000019073486	38.8401565551758\\
8.05500030517578	39.1106605529785\\
8.0600004196167	39.3943214416504\\
8.0649995803833	39.6920356750488\\
8.06999969482422	40.0040550231934\\
8.07499980926514	40.3325958251953\\
8.07999992370605	40.6518974304199\\
8.08500003814697	41.0645980834961\\
8.09000015258789	41.483097076416\\
8.09500026702881	41.9018821716309\\
8.10000038146973	42.3313484191895\\
8.10499954223633	42.7682800292969\\
8.10999965667725	43.2185173034668\\
8.11499977111816	43.6892623901367\\
8.11999988555908	44.1828193664551\\
8.125	44.6974639892578\\
8.13000011444092	45.2292098999023\\
8.13500022888184	45.776912689209\\
8.14000034332275	46.341796875\\
8.14500045776367	46.9185523986816\\
8.14999961853027	47.5500221252441\\
8.15499973297119	48.1796607971191\\
8.15999984741211	48.8288879394531\\
8.16499996185303	49.4971694946289\\
8.17000007629395	50.1720237731934\\
8.17500019073486	50.8734970092773\\
8.18000030517578	51.6145057678223\\
8.1850004196167	52.3988075256348\\
8.1899995803833	53.212890625\\
8.19499969482422	54.0314636230469\\
8.19999980926514	54.8474159240723\\
8.20499992370605	55.6699600219727\\
8.21000003814697	56.5194702148438\\
8.21500015258789	57.3845329284668\\
8.22000026702881	58.2745742797852\\
8.22500038146973	59.1860656738281\\
8.22999954223633	60.1296272277832\\
8.23499965667725	61.157470703125\\
8.23999977111816	62.2945709228516\\
8.24499988555908	63.4521751403809\\
8.25	64.4968032836914\\
8.25500011444092	65.5862655639648\\
8.26000022888184	66.7165908813477\\
8.26500034332275	67.8547821044922\\
8.27000045776367	68.9559936523438\\
8.27499961853027	70.342414855957\\
8.27999973297119	71.7008285522461\\
8.28499984741211	73.0244064331055\\
8.28999996185303	74.388801574707\\
8.29500007629395	75.8010711669922\\
8.30000019073486	77.2461700439453\\
8.30500030517578	78.6792068481445\\
8.3100004196167	80.0752105712891\\
8.3149995803833	81.6477661132813\\
8.31999969482422	83.6180419921875\\
8.32499980926514	85.3854598999023\\
8.32999992370605	87.0831298828125\\
8.33500003814697	88.8447265625\\
8.34000015258789	90.6727523803711\\
8.34500026702881	92.5718078613281\\
8.35000038146973	94.5272445678711\\
8.35499954223633	96.5330429077148\\
8.35999965667725	98.6123123168945\\
8.36499977111816	100.955490112305\\
8.36999988555908	103.375503540039\\
8.375	105.700950622559\\
8.38000011444092	108.057403564453\\
8.38500022888184	110.505325317383\\
8.39000034332275	113.004600524902\\
8.39500045776367	115.351699829102\\
8.39999961853027	118.059448242188\\
8.40499973297119	121.15828704834\\
8.40999984741211	124.159469604492\\
8.41499996185303	127.247917175293\\
8.42000007629395	130.212265014648\\
8.42500019073486	133.111450195313\\
8.43000030517578	136.049758911133\\
8.4350004196167	138.956146240234\\
8.4399995803833	141.925064086914\\
8.44499969482422	144.925582885742\\
8.44999980926514	147.68603515625\\
8.45499992370605	150.300018310547\\
8.46000003814697	152.797546386719\\
8.46500015258789	155.144180297852\\
8.47000026702881	157.492065429688\\
8.47500038146973	159.761276245117\\
8.47999954223633	161.934539794922\\
8.48499965667725	164.022399902344\\
8.48999977111816	166.049255371094\\
8.49499988555908	167.949203491211\\
8.5	169.705581665039\\
8.50500011444092	171.357360839844\\
8.51000022888184	172.859939575195\\
8.51500034332275	174.268447875977\\
8.52000045776367	175.539154052734\\
8.52499961853027	176.978591918945\\
8.52999973297119	178.334411621094\\
8.53499984741211	179.483337402344\\
8.53999996185303	180.277252197266\\
8.54500007629395	180.943450927734\\
8.55000019073486	181.523071289063\\
8.55500030517578	181.983963012695\\
8.5600004196167	182.370162963867\\
8.5649995803833	182.738464355469\\
8.56999969482422	183.062850952148\\
8.57499980926514	183.323104858398\\
8.57999992370605	183.496826171875\\
8.58500003814697	183.593185424805\\
8.59000015258789	183.625869750977\\
8.59500026702881	183.594207763672\\
8.60000038146973	183.252471923828\\
8.60499954223633	182.427810668945\\
8.60999965667725	181.186538696289\\
8.61499977111816	179.539123535156\\
8.61999988555908	177.567840576172\\
8.625	175.278991699219\\
8.63000011444092	173.001953125\\
8.63500022888184	170.717987060547\\
8.64000034332275	168.390655517578\\
8.64500045776367	166.256088256836\\
8.64999961853027	163.686431884766\\
8.65499973297119	161.456253051758\\
8.65999984741211	159.273376464844\\
8.66499996185303	157.391586303711\\
8.67000007629395	155.200576782227\\
8.67500019073486	153.046951293945\\
8.68000030517578	150.791442871094\\
8.6850004196167	148.538757324219\\
8.6899995803833	146.303604125977\\
8.69499969482422	143.927459716797\\
8.69999980926514	141.247192382813\\
8.70499992370605	138.566360473633\\
8.71000003814697	135.85871887207\\
8.71500015258789	133.062438964844\\
8.72000026702881	130.195602416992\\
8.72500038146973	127.416839599609\\
8.72999954223633	124.726371765137\\
8.73499965667725	122.101203918457\\
8.73999977111816	119.591667175293\\
8.74499988555908	117.238479614258\\
8.75	115.00212097168\\
8.75500011444092	112.818099975586\\
8.76000022888184	110.61759185791\\
8.76500034332275	108.554183959961\\
8.77000045776367	106.54615020752\\
8.77499961853027	104.559509277344\\
8.77999973297119	102.451507568359\\
8.78499984741211	100.490478515625\\
8.78999996185303	98.5745468139648\\
8.79500007629395	96.6332778930664\\
8.80000019073486	94.712890625\\
8.80500030517578	92.8494262695313\\
8.8100004196167	91.0587310791016\\
8.8149995803833	89.3542938232422\\
8.81999969482422	87.7546463012695\\
8.82499980926514	86.2669982910156\\
8.82999992370605	84.8975601196289\\
8.83500003814697	83.6445236206055\\
8.84000015258789	82.4867401123047\\
8.84500026702881	81.415283203125\\
8.85000038146973	80.4176254272461\\
8.85499954223633	79.4650802612305\\
8.85999965667725	78.5624465942383\\
8.86499977111816	77.6894073486328\\
8.86999988555908	76.8377380371094\\
8.875	76.0075378417969\\
8.88000011444092	75.2113571166992\\
8.88500022888184	74.4637298583984\\
8.89000034332275	73.7717132568359\\
8.89500045776367	73.1779632568359\\
8.89999961853027	72.5743103027344\\
8.90499973297119	72.0659713745117\\
8.90999984741211	71.6671752929688\\
8.91499996185303	71.3131637573242\\
8.92000007629395	71.0021896362305\\
8.92500019073486	70.7386703491211\\
8.93000030517578	70.5993881225586\\
8.9350004196167	70.6064682006836\\
8.9399995803833	70.5923538208008\\
8.94499969482422	70.3885192871094\\
8.94999980926514	70.1702041625977\\
8.95499992370605	70.0306549072266\\
8.96000003814697	69.9440994262695\\
8.96500015258789	69.9094924926758\\
8.97000026702881	69.9197540283203\\
8.97500038146973	69.9713363647461\\
8.97999954223633	70.0364685058594\\
8.98499965667725	70.1220016479492\\
8.98999977111816	70.2717971801758\\
8.99499988555908	70.4143600463867\\
9	70.5189056396484\\
9.00500011444092	70.6677627563477\\
9.01000022888184	70.8563232421875\\
9.01500034332275	71.0375366210938\\
9.02000045776367	71.1964416503906\\
9.02499961853027	71.3410263061523\\
9.02999973297119	71.4740447998047\\
9.03499984741211	71.5927810668945\\
9.03999996185303	71.695671081543\\
9.04500007629395	71.7883071899414\\
9.05000019073486	71.8594818115234\\
9.05500030517578	71.9177703857422\\
9.0600004196167	71.9651718139648\\
9.0649995803833	71.9987640380859\\
9.06999969482422	72.0170364379883\\
9.07499980926514	72.0190811157227\\
9.07999992370605	72.004997253418\\
9.08500003814697	71.976448059082\\
9.09000015258789	71.9357528686523\\
9.09500026702881	71.8784790039063\\
9.10000038146973	71.7934646606445\\
9.10499954223633	71.6816177368164\\
9.10999965667725	71.5701522827148\\
9.11499977111816	71.4602661132813\\
9.11999988555908	71.3355941772461\\
9.125	71.1842422485352\\
9.13000011444092	70.993522644043\\
9.13500022888184	70.7472305297852\\
9.14000034332275	70.4594192504883\\
9.14500045776367	70.1673355102539\\
9.14999961853027	69.8628997802734\\
9.15499973297119	69.5876083374023\\
9.15999984741211	69.352180480957\\
9.16499996185303	69.1080474853516\\
9.17000007629395	68.8646774291992\\
9.17500019073486	68.6591644287109\\
9.18000030517578	68.4539260864258\\
9.1850004196167	68.2412490844727\\
9.1899995803833	68.019172668457\\
9.19499969482422	67.7855072021484\\
9.19999980926514	67.5366058349609\\
9.20499992370605	67.2854843139648\\
9.21000003814697	67.0249633789063\\
9.21500015258789	66.7409820556641\\
9.22000026702881	66.4503021240234\\
9.22500038146973	66.1514511108398\\
9.22999954223633	65.8634033203125\\
9.23499965667725	65.6082153320313\\
9.23999977111816	65.4569473266602\\
9.24499988555908	65.5844802856445\\
9.25	66.1891403198242\\
9.25500011444092	67.2377319335938\\
9.26000022888184	68.6105728149414\\
9.26500034332275	70.1242828369141\\
9.27000045776367	71.7828140258789\\
9.27499961853027	73.8419494628906\\
9.27999973297119	76.0581893920898\\
9.28499984741211	78.4834747314453\\
9.28999996185303	81.1609420776367\\
9.29500007629395	84.1500854492188\\
9.30000019073486	87.4412231445313\\
9.30500030517578	90.8997344970703\\
9.3100004196167	95.1487426757813\\
9.3149995803833	99.7544479370117\\
9.31999969482422	104.532073974609\\
9.32499980926514	109.832389831543\\
9.32999992370605	115.296913146973\\
9.33500003814697	121.433151245117\\
9.34000015258789	127.538146972656\\
9.34500026702881	133.929992675781\\
9.35000038146973	141.405822753906\\
9.35499954223633	148.22248840332\\
9.35999965667725	154.561264038086\\
9.36499977111816	161.182708740234\\
9.36999988555908	167.00129699707\\
9.375	171.651992797852\\
9.38000011444092	175.438629150391\\
9.38500022888184	178.31266784668\\
9.39000034332275	180.620819091797\\
9.39500045776367	182.848327636719\\
9.39999961853027	185.227752685547\\
9.40499973297119	187.873123168945\\
9.40999984741211	190.985717773438\\
9.41499996185303	194.755996704102\\
9.42000007629395	199.042907714844\\
9.42500019073486	203.517013549805\\
9.43000030517578	208.791854858398\\
9.4350004196167	216.556594848633\\
9.4399995803833	222.702377319336\\
9.44499969482422	223.827438354492\\
9.44999980926514	218.210494995117\\
9.45499992370605	204.155960083008\\
9.46000003814697	180.777465820313\\
9.46500015258789	147.569091796875\\
9.47000026702881	106.846565246582\\
9.47500038146973	63.2432518005371\\
9.47999954223633	24.2057113647461\\
9.48499965667725	4.55841684341431\\
9.48999977111816	36.6474609375\\
9.49499988555908	122.041618347168\\
9.5	191.89208984375\\
9.50500011444092	221.220108032227\\
9.51000022888184	218.616027832031\\
9.51500034332275	197.565948486328\\
9.52000045776367	168.613189697266\\
9.52499961853027	137.820892333984\\
9.52999973297119	108.330238342285\\
9.53499984741211	80.8770065307617\\
9.53999996185303	56.3273544311523\\
9.54500007629395	35.1298332214355\\
9.55000019073486	17.3985233306885\\
9.55500030517578	7.99449348449707\\
9.5600004196167	30.367956161499\\
9.5649995803833	54.5276374816895\\
9.56999969482422	67.8906631469727\\
9.57499980926514	72.2704391479492\\
9.57999992370605	70.6799850463867\\
9.58500003814697	65.9190292358398\\
9.59000015258789	60.2299118041992\\
9.59500026702881	53.7873306274414\\
9.60000038146973	47.7894096374512\\
9.60499954223633	41.9111061096191\\
9.60999965667725	37.0691719055176\\
9.61499977111816	32.4956703186035\\
9.61999988555908	28.3613262176514\\
9.625	24.8151302337646\\
9.63000011444092	21.7629508972168\\
9.63500022888184	19.1082382202148\\
9.64000034332275	16.8024253845215\\
9.64500045776367	14.7925453186035\\
9.64999961853027	13.0464477539063\\
9.65499973297119	11.5364198684692\\
9.65999984741211	10.228892326355\\
9.66499996185303	9.09788990020752\\
9.67000007629395	8.11735439300537\\
9.67500019073486	7.26817083358765\\
9.68000030517578	6.52836132049561\\
9.6850004196167	5.88531970977783\\
9.6899995803833	5.32627105712891\\
9.69499969482422	4.8414478302002\\
9.69999980926514	4.42006540298462\\
9.70499992370605	4.05399751663208\\
9.71000003814697	3.7352020740509\\
9.71500015258789	3.45783162117004\\
9.72000026702881	3.2163507938385\\
9.72500038146973	3.00424075126648\\
9.72999954223633	2.81647801399231\\
9.73499965667725	2.65093636512756\\
9.73999977111816	2.50661253929138\\
9.74499988555908	2.38446474075317\\
9.75	2.28320074081421\\
9.75500011444092	2.19570970535278\\
9.76000022888184	2.11899733543396\\
9.76500034332275	2.05082583427429\\
9.77000045776367	1.99083781242371\\
9.77499961853027	1.93957161903381\\
9.77999973297119	1.8964227437973\\
9.78499984741211	1.86038029193878\\
9.78999996185303	1.82954585552216\\
9.79500007629395	1.80298626422882\\
9.80000019073486	1.78000748157501\\
9.80500030517578	1.75972354412079\\
9.8100004196167	1.74126958847046\\
9.8149995803833	1.72396862506866\\
9.81999969482422	1.70805060863495\\
9.82499980926514	1.69519019126892\\
9.82999992370605	1.68655598163605\\
9.83500003814697	1.68248784542084\\
9.84000015258789	1.68046152591705\\
9.84500026702881	1.67670941352844\\
9.85000038146973	1.66889011859894\\
9.85499954223633	1.65813148021698\\
9.85999965667725	1.6508070230484\\
9.86499977111816	1.64863049983978\\
9.86999988555908	1.65114724636078\\
9.875	1.65478777885437\\
9.88000011444092	1.65570294857025\\
9.88500022888184	1.65253865718842\\
9.89000034332275	1.64633560180664\\
9.89500045776367	1.64111661911011\\
9.89999961853027	1.63799571990967\\
9.90499973297119	1.6371465921402\\
9.90999984741211	1.63785624504089\\
9.91499996185303	1.63935148715973\\
9.92000007629395	1.64163076877594\\
9.92500019073486	1.64413964748383\\
9.93000030517578	1.64578056335449\\
9.9350004196167	1.64591431617737\\
9.9399995803833	1.64436745643616\\
9.94499969482422	1.64184010028839\\
9.94999980926514	1.63887786865234\\
9.95499992370605	1.63554155826569\\
9.96000003814697	1.63178706169128\\
9.96500015258789	1.62755978107452\\
9.97000026702881	1.62283670902252\\
9.97500038146973	1.6209704875946\\
9.97999954223633	1.6166068315506\\
9.98499965667725	1.61460828781128\\
9.98999977111816	1.61126828193665\\
9.99499988555908	1.61362612247467\\
10	1.61707663536072\\
};
\addlegendentry{Control}

\addplot [color=black, dashed, line width=2.0pt]
  table[row sep=crcr]{%
0.0949999988079071	-33.0273152189994\\
0.100000001490116	-28.3116866495482\\
0.104999996721745	-24.2017834837699\\
0.109999999403954	-20.6311173388899\\
0.115000002086163	-17.5311709278664\\
0.119999997317791	153.663652461598\\
0.125	128.621796147607\\
0.129999995231628	89.7950448341801\\
0.135000005364418	46.4721234186366\\
0.140000000596046	7.21806484629178\\
0.144999995827675	-25.7509901383938\\
0.150000005960464	-51.206983670921\\
0.155000001192093	-70.0335217037684\\
0.159999996423721	-76.2461681874539\\
0.165000006556511	-83.7782727208308\\
0.170000001788139	-91.2478162710502\\
0.174999997019768	-96.4333785528758\\
0.180000007152557	-98.151823573945\\
0.185000002384186	-96.5465534281509\\
0.189999997615814	-91.2697897833473\\
0.194999992847443	319.336700439453\\
0.200000002980232	378.788131713867\\
0.204999998211861	382.755111694336\\
0.209999993443489	386.795516967773\\
0.215000003576279	357.266571044922\\
0.219999998807907	344.72265625\\
0.224999994039536	284.940368652344\\
0.230000004172325	205.20166015625\\
0.234999999403954	123.077331542969\\
0.239999994635582	51.296142578125\\
0.245000004768372	1.11874389648438\\
0.25	-24.4195556640625\\
0.254999995231628	4.6871337890625\\
0.259999990463257	51.7826538085938\\
0.264999985694885	82.9118041992188\\
0.270000010728836	105.353210449219\\
0.275000005960464	123.383056640625\\
0.280000001192093	138.037231445313\\
0.284999996423721	148.802551269531\\
0.28999999165535	155.312072753906\\
0.294999986886978	173.62109375\\
0.300000011920929	177.571411132813\\
0.305000007152557	162.01318359375\\
0.310000002384186	137.787231445313\\
0.314999997615814	111.881225585938\\
0.319999992847443	89.0755004882813\\
0.324999988079071	72.6040649414063\\
0.330000013113022	71.345458984375\\
0.33500000834465	72.239501953125\\
0.340000003576279	81.03759765625\\
0.344999998807907	86.4461059570313\\
0.349999994039536	90.9376220703125\\
0.354999989271164	95.1261596679688\\
0.360000014305115	98.6153564453125\\
0.365000009536743	103.636779785156\\
0.370000004768372	108.473724365234\\
0.375	107.442504882813\\
0.379999995231628	101.847137451172\\
0.384999990463257	92.8620300292969\\
0.389999985694885	82.1797790527344\\
0.395000010728836	70.6890869140625\\
0.400000005960464	60.2054748535156\\
0.405000001192093	52.183349609375\\
0.409999996423721	47.034912109375\\
0.41499999165535	44.6195983886719\\
0.419999986886978	45.6673278808594\\
0.425000011920929	50.2373962402344\\
0.430000007152557	53.1614379882813\\
0.435000002384186	55.5064392089844\\
0.439999997615814	59.4962310791016\\
0.444999992847443	61.2991790771484\\
0.449999988079071	60.7980346679688\\
0.455000013113022	58.2913208007813\\
0.46000000834465	53.9759368896484\\
0.465000003576279	48.6276397705078\\
0.469999998807907	42.5501174926758\\
0.474999994039536	37.132209777832\\
0.479999989271164	32.8298568725586\\
0.485000014305115	30.2137298583984\\
0.490000009536743	29.493782043457\\
0.495000004768372	32.7774963378906\\
0.5	35.3054428100586\\
0.504999995231628	37.5588684082031\\
0.509999990463257	39.7823028564453\\
0.514999985694885	41.7399215698242\\
0.519999980926514	43.3062210083008\\
0.524999976158142	45.5817337036133\\
0.529999971389771	48.0795593261719\\
0.535000026226044	47.1603546142578\\
0.540000021457672	44.8402786254883\\
0.545000016689301	42.5366516113281\\
0.550000011920929	40.596923828125\\
0.555000007152557	40.7313766479492\\
0.560000002384186	41.2847747802734\\
0.564999997615814	42.3346939086914\\
0.569999992847443	43.9469451904297\\
0.574999988079071	45.8925399780273\\
0.579999983310699	48.0345077514648\\
0.584999978542328	50.2076721191406\\
0.589999973773956	52.2217864990234\\
0.595000028610229	54.0167541503906\\
0.600000023841858	55.5427703857422\\
0.605000019073486	56.8181915283203\\
0.610000014305115	57.8633728027344\\
0.615000009536743	58.7534027099609\\
0.620000004768372	59.5875091552734\\
0.625	60.4351959228516\\
0.629999995231628	61.3263397216797\\
0.634999990463257	62.3394012451172\\
0.639999985694885	63.4562072753906\\
0.644999980926514	64.6869659423828\\
0.649999976158142	66.0091552734375\\
0.654999971389771	67.386474609375\\
0.660000026226044	68.77001953125\\
0.665000021457672	70.1144104003906\\
0.670000016689301	71.3809814453125\\
0.675000011920929	72.5406188964844\\
0.680000007152557	73.5764465332031\\
0.685000002384186	74.4765319824219\\
0.689999997615814	75.247314453125\\
0.694999992847443	75.9271545410156\\
0.699999988079071	76.5359802246094\\
0.704999983310699	77.0627746582031\\
0.709999978542328	77.5354919433594\\
0.714999973773956	78.0002136230469\\
0.720000028610229	78.411865234375\\
0.725000023841858	78.7760925292969\\
0.730000019073486	79.1265258789063\\
0.735000014305115	79.4010620117188\\
0.740000009536743	79.630126953125\\
0.745000004768372	79.7745971679688\\
0.75	79.8570251464844\\
0.754999995231628	79.8473205566406\\
0.759999990463257	79.7745056152344\\
0.764999985694885	79.8960266113281\\
0.769999980926514	80.0375671386719\\
0.774999976158142	79.8501281738281\\
0.779999971389771	79.3773193359375\\
0.785000026226044	78.7987060546875\\
0.790000021457672	78.1942443847656\\
0.795000016689301	77.5823364257813\\
0.800000011920929	76.963623046875\\
0.805000007152557	76.3390197753906\\
0.810000002384186	75.7095031738281\\
0.814999997615814	75.0743713378906\\
0.819999992847443	74.4253234863281\\
0.824999988079071	73.7604370117188\\
0.829999983310699	73.0747680664063\\
0.834999978542328	72.3681030273438\\
0.839999973773956	71.6505432128906\\
0.845000028610229	70.9205627441406\\
0.850000023841858	70.1762084960938\\
0.855000019073486	69.4213256835938\\
0.860000014305115	68.6695251464844\\
0.865000009536743	67.9258117675781\\
0.870000004768372	67.193115234375\\
0.875	66.478515625\\
0.879999995231628	65.7869873046875\\
0.884999990463257	65.1194458007813\\
0.889999985694885	64.4755859375\\
0.894999980926514	63.8582763671875\\
0.899999976158142	63.2696533203125\\
0.904999971389771	62.7118530273438\\
0.910000026226044	62.1866302490234\\
0.915000021457672	61.6935424804688\\
0.920000016689301	61.2283020019531\\
0.925000011920929	60.7937774658203\\
0.930000007152557	60.3901977539063\\
0.935000002384186	60.0185394287109\\
0.939999997615814	59.6788635253906\\
0.944999992847443	59.3728332519531\\
0.949999988079071	59.1035308837891\\
0.954999983310699	58.8712615966797\\
0.959999978542328	58.6781311035156\\
0.964999973773956	58.5265197753906\\
0.970000028610229	58.4177398681641\\
0.975000023841858	58.3497161865234\\
0.980000019073486	58.3223571777344\\
0.985000014305115	58.3385162353516\\
0.990000009536743	58.4084320068359\\
0.995000004768372	58.5323638916016\\
1	58.7201538085938\\
1.00499999523163	58.9808959960938\\
1.00999999046326	59.3036193847656\\
1.01499998569489	59.4891815185547\\
1.01999998092651	59.5459747314453\\
1.02499997615814	59.6590881347656\\
1.02999997138977	59.82470703125\\
1.0349999666214	60.0303344726563\\
1.03999996185303	60.2786712646484\\
1.04499995708466	60.5575103759766\\
1.04999995231628	60.8645324707031\\
1.05499994754791	61.19189453125\\
1.05999994277954	61.5345916748047\\
1.06500005722046	61.8915405273438\\
1.07000005245209	62.2554016113281\\
1.07500004768372	62.6244659423828\\
1.08000004291534	62.9955902099609\\
1.08500003814697	63.3668975830078\\
1.0900000333786	63.7394866943359\\
1.09500002861023	64.1108856201172\\
1.10000002384186	64.4796447753906\\
1.10500001907349	64.8437805175781\\
1.11000001430511	65.1996765136719\\
1.11500000953674	65.5464782714844\\
1.12000000476837	65.8824157714844\\
1.125	66.2088317871094\\
1.12999999523163	66.5244445800781\\
1.13499999046326	66.8272399902344\\
1.13999998569489	67.11669921875\\
1.14499998092651	67.3943786621094\\
1.14999997615814	67.6582641601563\\
1.15499997138977	67.9078063964844\\
1.1599999666214	68.1415100097656\\
1.16499996185303	68.3583374023438\\
1.16999995708466	68.5577087402344\\
1.17499995231628	68.7386169433594\\
1.17999994754791	68.8994750976563\\
1.18499994277954	69.0396728515625\\
1.19000005722046	69.1580810546875\\
1.19500005245209	69.2569580078125\\
1.20000004768372	69.3384399414063\\
1.20500004291534	69.4025268554688\\
1.21000003814697	69.4504089355469\\
1.2150000333786	69.4756469726563\\
1.22000002861023	69.4805603027344\\
1.22500002384186	69.4645385742188\\
1.23000001907349	69.4397888183594\\
1.23500001430511	69.4230651855469\\
1.24000000953674	69.4173889160156\\
1.24500000476837	69.4332580566406\\
1.25	69.38671875\\
1.25499999523163	69.2937622070313\\
1.25999999046326	69.1329650878906\\
1.26499998569489	68.9304809570313\\
1.26999998092651	68.7422180175781\\
1.27499997615814	68.5338745117188\\
1.27999997138977	68.314453125\\
1.2849999666214	68.1117553710938\\
1.28999996185303	67.9235534667969\\
1.29499995708466	67.7552185058594\\
1.29999995231628	67.572265625\\
1.30499994754791	67.3907775878906\\
1.30999994277954	67.2105102539063\\
1.31500005722046	67.022216796875\\
1.32000005245209	66.8256225585938\\
1.32500004768372	66.6189575195313\\
1.33000004291534	66.4090881347656\\
1.33500003814697	66.1976318359375\\
1.3400000333786	65.9825744628906\\
1.34500002861023	65.7687377929688\\
1.35000002384186	65.5633850097656\\
1.35500001907349	65.3692626953125\\
1.36000001430511	65.1897277832031\\
1.36500000953674	65.0242919921875\\
1.37000000476837	64.8785400390625\\
1.375	64.7514343261719\\
1.37999999523163	64.6080932617188\\
1.38499999046326	64.4588317871094\\
1.38999998569489	64.2989501953125\\
1.39499998092651	64.1484222412109\\
1.39999997615814	63.9987335205078\\
1.40499997138977	63.8516845703125\\
1.4099999666214	63.7318420410156\\
1.41499996185303	63.6365356445313\\
1.41999995708466	63.5669860839844\\
1.42499995231628	63.5065002441406\\
1.42999994754791	63.4628143310547\\
1.43499994277954	63.4331207275391\\
1.44000005722046	63.4096832275391\\
1.44500005245209	63.3927612304688\\
1.45000004768372	63.3802032470703\\
1.45500004291534	63.3702239990234\\
1.46000003814697	63.3603210449219\\
1.4650000333786	63.3594818115234\\
1.47000002861023	63.3793182373047\\
1.47500002384186	63.4254302978516\\
1.48000001907349	63.4679718017578\\
1.48500001430511	63.4893035888672\\
1.49000000953674	63.4791870117188\\
1.49500000476837	63.5272674560547\\
1.5	63.6277465820313\\
1.50499999523163	63.7651977539063\\
1.50999999046326	63.8774261474609\\
1.51499998569489	63.9786224365234\\
1.51999998092651	64.0815734863281\\
1.52499997615814	64.189208984375\\
1.52999997138977	64.2974090576172\\
1.5349999666214	64.418701171875\\
1.53999996185303	64.5520629882813\\
1.54499995708466	64.6880493164063\\
1.54999995231628	64.8145599365234\\
1.55499994754791	64.9325256347656\\
1.55999994277954	65.0513610839844\\
1.56500005722046	65.1667175292969\\
1.57000005245209	60.8254699707031\\
1.57500004768372	64.3569641113281\\
1.58000004291534	65.8841705322266\\
1.58500003814697	66.2549285888672\\
1.5900000333786	66.1963806152344\\
1.59500002861023	65.9465637207031\\
1.60000002384186	65.6512756347656\\
1.60500001907349	65.4535827636719\\
1.61000001430511	65.4418640136719\\
1.61500000953674	65.4972534179688\\
1.62000000476837	65.8268432617188\\
1.625	66.1192932128906\\
1.62999999523163	66.3527221679688\\
1.63499999046326	66.4307250976563\\
1.63999998569489	66.2906799316406\\
1.64499998092651	66.5184631347656\\
1.64999997615814	66.7041320800781\\
1.65499997138977	66.7938537597656\\
1.6599999666214	66.837158203125\\
1.66499996185303	66.8455505371094\\
1.66999995708466	66.724365234375\\
1.67499995231628	66.6391296386719\\
1.67999994754791	66.7070617675781\\
1.68499994277954	66.7427062988281\\
1.69000005722046	66.7713623046875\\
1.69500005245209	66.804931640625\\
1.70000004768372	66.845703125\\
1.70500004291534	66.8860778808594\\
1.71000003814697	66.9396667480469\\
1.7150000333786	66.9890747070313\\
1.72000002861023	67.0299072265625\\
1.72500002384186	67.0543212890625\\
1.73000001907349	67.0651245117188\\
1.73500001430511	67.0733642578125\\
1.74000000953674	67.0773315429688\\
1.74500000476837	67.0772399902344\\
1.75	67.0766906738281\\
1.75499999523163	67.0738220214844\\
1.75999999046326	67.0665893554688\\
1.76499998569489	67.0554504394531\\
1.76999998092651	67.042236328125\\
1.77499997615814	67.0316467285156\\
1.77999997138977	67.0238952636719\\
1.7849999666214	67.0135192871094\\
1.78999996185303	67.0003662109375\\
1.79499995708466	66.986328125\\
1.79999995231628	66.976806640625\\
1.80499994754791	66.9712524414063\\
1.80999994277954	66.9693603515625\\
1.81500005722046	66.9721069335938\\
1.82000005245209	66.9790649414063\\
1.82500004768372	66.9869995117188\\
1.83000004291534	66.9964294433594\\
1.83500003814697	67.0032348632813\\
1.8400000333786	67.0084533691406\\
1.84500002861023	67.0129699707031\\
1.85000002384186	67.0224914550781\\
1.85500001907349	67.0339965820313\\
1.86000001430511	67.0474700927734\\
1.86500000953674	67.0627593994141\\
1.87000000476837	67.0799560546875\\
1.875	67.1049499511719\\
1.87999999523163	67.1421203613281\\
1.88499999046326	67.1878204345703\\
1.88999998569489	67.2415466308594\\
1.89499998092651	67.2855224609375\\
1.89999997615814	67.3297119140625\\
1.90499997138977	67.3740081787109\\
1.9099999666214	67.4231567382813\\
1.91499996185303	67.4754180908203\\
1.91999995708466	67.5303649902344\\
1.92499995231628	67.5953979492188\\
1.92999994754791	67.6697540283203\\
1.93499994277954	67.7519683837891\\
1.94000005722046	67.8357849121094\\
1.94500005245209	67.9192657470703\\
1.95000004768372	68.0052947998047\\
1.95500004291534	68.0994567871094\\
1.96000003814697	68.2105712890625\\
1.9650000333786	68.3377685546875\\
1.97000002861023	68.4718627929688\\
1.97500002384186	68.5800476074219\\
1.98000001907349	68.6704711914063\\
1.98500001430511	68.7424926757813\\
1.99000000953674	68.8277587890625\\
1.99500000476837	68.9131774902344\\
2	69.0014343261719\\
2.00500011444092	69.1086120605469\\
2.00999999046326	69.2344970703125\\
2.01500010490417	69.384033203125\\
2.01999998092651	69.5592651367188\\
2.02500009536743	69.7652587890625\\
2.02999997138977	69.9984741210938\\
2.03500008583069	70.1214904785156\\
2.03999996185303	70.154541015625\\
2.04500007629395	70.1306762695313\\
2.04999995231628	70.0992431640625\\
2.0550000667572	70.0292053222656\\
2.05999994277954	69.8807983398438\\
2.06500005722046	69.6528930664063\\
2.0699999332428	69.3169250488281\\
2.07500004768372	68.8681640625\\
2.07999992370605	68.3438415527344\\
2.08500003814697	67.7380676269531\\
2.08999991416931	67.0662841796875\\
2.09500002861023	66.3370971679688\\
2.09999990463257	65.5744934082031\\
2.10500001907349	64.7953186035156\\
2.10999989509583	64.0867614746094\\
2.11500000953674	63.4969787597656\\
2.11999988555908	62.8529968261719\\
2.125	62.1019287109375\\
2.13000011444092	61.3955688476563\\
2.13499999046326	60.7250671386719\\
2.14000010490417	60.0605163574219\\
2.14499998092651	59.4412536621094\\
2.15000009536743	58.958740234375\\
2.15499997138977	58.4697875976563\\
2.16000008583069	57.7256164550781\\
2.16499996185303	57.0540008544922\\
2.17000007629395	56.4618530273438\\
2.17499995231628	55.8438568115234\\
2.1800000667572	55.2441711425781\\
2.18499994277954	54.6805114746094\\
2.19000005722046	54.2181701660156\\
2.1949999332428	53.8458099365234\\
2.20000004768372	53.5626068115234\\
2.20499992370605	53.3378295898438\\
2.21000003814697	53.168212890625\\
2.21499991416931	53.0740203857422\\
2.22000002861023	53.1234893798828\\
2.22499990463257	53.23388671875\\
2.23000001907349	53.3109893798828\\
2.23499989509583	53.2535400390625\\
2.24000000953674	53.2951507568359\\
2.24499988555908	53.3943481445313\\
2.25	53.4440155029297\\
2.25500011444092	53.5138702392578\\
2.25999999046326	53.616943359375\\
2.26500010490417	53.7996368408203\\
2.26999998092651	54.1265716552734\\
2.27500009536743	54.5846557617188\\
2.27999997138977	55.1251831054688\\
2.28500008583069	55.7244567871094\\
2.28999996185303	56.3812713623047\\
2.29500007629395	57.0061798095703\\
2.29999995231628	57.2686614990234\\
2.3050000667572	57.8504180908203\\
2.30999994277954	58.5283355712891\\
2.31500005722046	59.1414794921875\\
2.3199999332428	59.7689666748047\\
2.32500004768372	60.5274963378906\\
2.32999992370605	61.3629302978516\\
2.33500003814697	62.2563171386719\\
2.33999991416931	63.8256072998047\\
2.34500002861023	65.294677734375\\
2.34999990463257	65.9004058837891\\
2.35500001907349	67.0263061523438\\
2.35999989509583	68.2614593505859\\
2.36500000953674	69.6114959716797\\
2.36999988555908	71.2323913574219\\
2.375	72.9109802246094\\
2.38000011444092	74.772216796875\\
2.38499999046326	76.731689453125\\
2.39000010490417	78.621826171875\\
2.39499998092651	80.3759460449219\\
2.40000009536743	81.9833984375\\
2.40499997138977	83.3640747070313\\
2.41000008583069	84.5106811523438\\
2.41499996185303	85.4231567382813\\
2.42000007629395	86.0868225097656\\
2.42499995231628	86.6040954589844\\
2.4300000667572	86.843505859375\\
2.43499994277954	86.900390625\\
2.44000005722046	86.82177734375\\
2.4449999332428	86.2686462402344\\
2.45000004768372	86.0988159179688\\
2.45499992370605	85.6725463867188\\
2.46000003814697	85.2343444824219\\
2.46499991416931	84.8980407714844\\
2.47000002861023	84.5241394042969\\
2.47499990463257	84.1189880371094\\
2.48000001907349	83.7082824707031\\
2.48499989509583	83.2599792480469\\
2.49000000953674	82.89599609375\\
2.49499988555908	82.6882019042969\\
2.5	82.3996276855469\\
2.50500011444092	81.7293701171875\\
2.50999999046326	80.963134765625\\
2.51500010490417	80.43212890625\\
2.51999998092651	79.4562377929688\\
2.52500009536743	78.2732543945313\\
2.52999997138977	77.1453857421875\\
2.53500008583069	75.7731018066406\\
2.53999996185303	74.6932067871094\\
2.54500007629395	73.5989074707031\\
2.54999995231628	72.5052490234375\\
2.5550000667572	71.43017578125\\
2.55999994277954	70.3355102539063\\
2.56500005722046	69.2579650878906\\
2.5699999332428	68.1801147460938\\
2.57500004768372	67.0971374511719\\
2.57999992370605	66.0242919921875\\
2.58500003814697	65.01025390625\\
2.58999991416931	64.0163269042969\\
2.59500002861023	63.0781860351563\\
2.59999990463257	62.1539306640625\\
2.60500001907349	61.2140197753906\\
2.60999989509583	60.3094482421875\\
2.61500000953674	59.4314880371094\\
2.61999988555908	58.6239624023438\\
2.625	57.8876037597656\\
2.63000011444092	57.2358703613281\\
2.63499999046326	56.5711822509766\\
2.64000010490417	56.0347595214844\\
2.64499998092651	55.5626068115234\\
2.65000009536743	54.9518585205078\\
2.65499997138977	54.3023681640625\\
2.66000008583069	53.5847778320313\\
2.66499996185303	52.7317962646484\\
2.67000007629395	51.7002105712891\\
2.67499995231628	50.7295837402344\\
2.6800000667572	49.9710388183594\\
2.68499994277954	48.9624938964844\\
2.69000005722046	48.1505432128906\\
2.6949999332428	47.4209747314453\\
2.70000004768372	46.7854766845703\\
2.70499992370605	46.2608947753906\\
2.71000003814697	45.8296051025391\\
2.71499991416931	45.4409942626953\\
2.72000002861023	45.1763000488281\\
2.72499990463257	45.0136108398438\\
2.73000001907349	44.9005889892578\\
2.73499989509583	44.847412109375\\
2.74000000953674	44.8700408935547\\
2.74499988555908	45.04833984375\\
2.75	45.2864837646484\\
2.75500011444092	45.5987548828125\\
2.75999999046326	46.2358703613281\\
2.76500010490417	47.2809448242188\\
2.76999998092651	48.6716156005859\\
2.77500009536743	50.2719879150391\\
2.77999997138977	51.9459075927734\\
2.78500008583069	53.8839569091797\\
2.78999996185303	56.0438537597656\\
2.79500007629395	58.3191223144531\\
2.79999995231628	60.6071624755859\\
2.8050000667572	62.8387298583984\\
2.80999994277954	64.9119873046875\\
2.81500005722046	66.7615051269531\\
2.8199999332428	68.3424682617188\\
2.82500004768372	69.6933898925781\\
2.82999992370605	70.8486328125\\
2.83500003814697	72.0334014892578\\
2.83999991416931	73.1895294189453\\
2.84500002861023	74.3914337158203\\
2.84999990463257	75.7759399414063\\
2.85500001907349	77.0287628173828\\
2.85999989509583	78.2627716064453\\
2.86500000953674	79.4267578125\\
2.86999988555908	80.537841796875\\
2.875	81.4707641601563\\
2.88000011444092	82.4168395996094\\
2.88499999046326	83.3766479492188\\
2.89000010490417	84.4027404785156\\
2.89499998092651	85.0588073730469\\
2.90000009536743	85.05419921875\\
2.90499997138977	84.9708557128906\\
2.91000008583069	84.78125\\
2.91499996185303	84.5177917480469\\
2.92000007629395	84.4027709960938\\
2.92499995231628	84.4331359863281\\
2.9300000667572	84.568115234375\\
2.93499994277954	84.8017272949219\\
2.94000005722046	84.9219360351563\\
2.9449999332428	84.8916625976563\\
2.95000004768372	84.4933471679688\\
2.95499992370605	83.8920593261719\\
2.96000003814697	83.1684265136719\\
2.96499991416931	82.5073852539063\\
2.97000002861023	81.3760986328125\\
2.97499990463257	79.1353454589844\\
2.98000001907349	76.4956359863281\\
2.98499989509583	77.6094360351563\\
2.99000000953674	73.2052307128906\\
2.99499988555908	70.5969848632813\\
3	68.4452819824219\\
3.00500011444092	66.7007751464844\\
3.00999999046326	65.2935791015625\\
3.01500010490417	64.1371154785156\\
3.01999998092651	63.1266479492188\\
3.02500009536743	62.2004089355469\\
3.02999997138977	61.1544494628906\\
3.03500008583069	59.8258972167969\\
3.03999996185303	58.2877197265625\\
3.04500007629395	56.4483337402344\\
3.04999995231628	54.4290771484375\\
3.0550000667572	52.328369140625\\
3.05999994277954	50.1131591796875\\
3.06500005722046	48.0889282226563\\
3.0699999332428	46.4869537353516\\
3.07500004768372	45.0174713134766\\
3.07999992370605	43.904541015625\\
3.08500003814697	43.1687469482422\\
3.08999991416931	42.6425018310547\\
3.09500002861023	42.3619689941406\\
3.09999990463257	42.3000335693359\\
3.10500001907349	42.3257446289063\\
3.10999989509583	42.6600036621094\\
3.11500000953674	43.3571090698242\\
3.11999988555908	44.2858428955078\\
3.125	45.8560180664063\\
3.13000011444092	48.8534622192383\\
3.13499999046326	53.9274978637695\\
3.14000010490417	59.4673538208008\\
3.14499998092651	65.3217163085938\\
3.15000009536743	71.6093597412109\\
3.15499997138977	76.7478866577148\\
3.16000008583069	79.6033782958984\\
3.16499996185303	80.5327911376953\\
3.17000007629395	79.9743957519531\\
3.17499995231628	78.3206787109375\\
3.1800000667572	76.3425140380859\\
3.18499994277954	74.60205078125\\
3.19000005722046	72.379150390625\\
3.1949999332428	69.3408813476563\\
3.20000004768372	66.2212524414063\\
3.20499992370605	64.420654296875\\
3.21000003814697	62.3851928710938\\
3.21499991416931	61.8544158935547\\
3.22000002861023	62.0965881347656\\
3.22499990463257	62.9850158691406\\
3.23000001907349	64.1661376953125\\
3.23499989509583	66.0837707519531\\
3.24000000953674	68.1308288574219\\
3.24499988555908	69.994873046875\\
3.25	71.9838562011719\\
3.25500011444092	74.0289916992188\\
3.25999999046326	75.8633117675781\\
3.26500010490417	77.3682556152344\\
3.26999998092651	78.6549987792969\\
3.27500009536743	79.4790649414063\\
3.27999997138977	80.0404052734375\\
3.28500008583069	80.2330017089844\\
3.28999996185303	79.9777526855469\\
3.29500007629395	79.4879455566406\\
3.29999995231628	78.7308044433594\\
3.3050000667572	77.704345703125\\
3.30999994277954	76.9226684570313\\
3.31500005722046	75.9544067382813\\
3.3199999332428	76.0295715332031\\
3.32500004768372	75.5554504394531\\
3.32999992370605	74.4963989257813\\
3.33500003814697	73.3287658691406\\
3.33999991416931	72.5506591796875\\
3.34500002861023	71.9338073730469\\
3.34999990463257	71.0890502929688\\
3.35500001907349	69.9164733886719\\
3.35999989509583	68.4745178222656\\
3.36500000953674	66.6541442871094\\
3.36999988555908	64.433837890625\\
3.375	62.0472106933594\\
3.38000011444092	59.5554809570313\\
3.38499999046326	57.0582580566406\\
3.39000010490417	54.6864318847656\\
3.39499998092651	52.5121154785156\\
3.40000009536743	50.5990600585938\\
3.40499997138977	48.952392578125\\
3.41000008583069	47.5452117919922\\
3.41499996185303	46.3708190917969\\
3.42000007629395	45.4247894287109\\
3.42499995231628	44.5412292480469\\
3.4300000667572	43.9254455566406\\
3.43499994277954	43.5628662109375\\
3.44000005722046	43.6346435546875\\
3.4449999332428	44.3318939208984\\
3.45000004768372	45.8849029541016\\
3.45499992370605	48.6344909667969\\
3.46000003814697	54.1957092285156\\
3.46499991416931	60.6329650878906\\
3.47000002861023	66.2200622558594\\
3.47499990463257	71.0586395263672\\
3.48000001907349	74.7043762207031\\
3.48499989509583	77.3860931396484\\
3.49000000953674	79.5529479980469\\
3.49499988555908	79.7333374023438\\
3.5	80.0635070800781\\
3.50500011444092	79.3648986816406\\
3.50999999046326	77.5770721435547\\
3.51500010490417	76.2646789550781\\
3.51999998092651	73.5686187744141\\
3.52500009536743	72.0569915771484\\
3.52999997138977	70.1116790771484\\
3.53500008583069	68.5241546630859\\
3.53999996185303	68.6616668701172\\
3.54500007629395	68.6820373535156\\
3.54999995231628	70.1696472167969\\
3.5550000667572	69.2765350341797\\
3.55999994277954	67.2176818847656\\
3.56500005722046	65.1009521484375\\
3.5699999332428	66.8902587890625\\
3.57500004768372	70.2777099609375\\
3.57999992370605	76.4124450683594\\
3.58500003814697	80.3584594726563\\
3.58999991416931	82.9733276367188\\
3.59500002861023	86.0809936523438\\
3.59999990463257	88.5993347167969\\
3.60500001907349	87.7859802246094\\
3.60999989509583	86.3313293457031\\
3.61500000953674	82.8121643066406\\
3.61999988555908	76.052001953125\\
3.625	68.2838134765625\\
3.63000011444092	60.9854125976563\\
3.63499999046326	57.2356567382813\\
3.64000010490417	55.1966857910156\\
3.64499998092651	53.7584838867188\\
3.65000009536743	54.4007873535156\\
3.65499997138977	55.2311401367188\\
3.66000008583069	56.3066711425781\\
3.66499996185303	58.2928466796875\\
3.67000007629395	60.0240478515625\\
3.67499995231628	60.1795349121094\\
3.6800000667572	58.6249694824219\\
3.68499994277954	55.4293212890625\\
3.69000005722046	51.007568359375\\
3.6949999332428	46.01708984375\\
3.70000004768372	41.1455993652344\\
3.70499992370605	37.2395172119141\\
3.71000003814697	34.7409973144531\\
3.71499991416931	33.6273803710938\\
3.72000002861023	34.0793762207031\\
3.72499990463257	35.7404327392578\\
3.73000001907349	38.2232513427734\\
3.73499989509583	41.3578186035156\\
3.74000000953674	44.4505767822266\\
3.74499988555908	48.2219085693359\\
3.75	55.6395721435547\\
3.75500011444092	65.9635314941406\\
3.75999999046326	79.2845764160156\\
3.76500010490417	91.5764770507813\\
3.76999998092651	100.515365600586\\
3.77500009536743	108.495101928711\\
3.77999997138977	110.054168701172\\
3.78500008583069	106.395172119141\\
3.78999996185303	102.864562988281\\
3.79500007629395	99.1100158691406\\
3.79999995231628	92.5629425048828\\
3.8050000667572	84.6946868896484\\
3.80999994277954	76.3275299072266\\
3.81500005722046	68.2722473144531\\
3.8199999332428	60.9221801757813\\
3.82500004768372	55.1465148925781\\
3.82999992370605	51.7483825683594\\
3.83500003814697	49.3672943115234\\
3.83999991416931	50.2295837402344\\
3.84500002861023	54.0285034179688\\
3.84999990463257	56.1463623046875\\
3.85500001907349	59.67626953125\\
3.85999989509583	68.9979553222656\\
3.86500000953674	76.6009216308594\\
3.86999988555908	82.4068908691406\\
3.875	87.4634704589844\\
3.88000011444092	91.7136535644531\\
3.88499999046326	98.2648315429688\\
3.89000010490417	99.5986633300781\\
3.89499998092651	96.9612121582031\\
3.90000009536743	93.5461120605469\\
3.90499997138977	84.4549255371094\\
3.91000008583069	73.7891845703125\\
3.91499996185303	63.9744873046875\\
3.92000007629395	58.4570922851563\\
3.92499995231628	56.291259765625\\
3.9300000667572	56.0848083496094\\
3.93499994277954	56.3724060058594\\
3.94000005722046	56.8851928710938\\
3.9449999332428	57.6390686035156\\
3.95000004768372	58.9277954101563\\
3.95499992370605	61.3341369628906\\
3.96000003814697	61.4292297363281\\
3.96499991416931	59.1826782226563\\
3.97000002861023	54.7298278808594\\
3.97499990463257	48.8820190429688\\
3.98000001907349	42.2522888183594\\
3.98499989509583	35.890869140625\\
3.99000000953674	30.7064666748047\\
3.99499988555908	27.4119110107422\\
4	25.9596557617188\\
4.00500011444092	26.32080078125\\
4.01000022888184	28.4308929443359\\
4.0149998664856	31.8773040771484\\
4.01999998092651	35.6315612792969\\
4.02500009536743	40.2418212890625\\
4.03000020980835	47.571907043457\\
4.03499984741211	58.9309310913086\\
4.03999996185303	72.4245376586914\\
4.04500007629395	85.4010696411133\\
4.05000019073486	95.8228034973145\\
4.05499982833862	101.282829284668\\
4.05999994277954	102.071159362793\\
4.06500005722046	102.806522369385\\
4.07000017166138	101.229774475098\\
4.07499980926514	99.5155487060547\\
4.07999992370605	93.9699401855469\\
4.08500003814697	87.4451141357422\\
4.09000015258789	80.8513946533203\\
4.09499979019165	74.8604583740234\\
4.09999990463257	72.5017242431641\\
4.10500001907349	78.1604614257813\\
4.1100001335144	78.2114582061768\\
4.11499977111816	67.7713944684795\\
4.11999988555908	34.8035296813806\\
4.125	31.0277328491211\\
4.13000011444092	37.3235168457031\\
4.13500022888184	51.5232238769531\\
4.1399998664856	61.5628051757813\\
4.14499998092651	72.3412170410156\\
4.15000009536743	86.2821960449219\\
4.15500020980835	98.9654846191406\\
4.15999984741211	113.119201660156\\
4.16499996185303	125.296264648438\\
4.17000007629395	130.747650146484\\
4.17500019073486	115.357086181641\\
4.17999982833862	89.9228210449219\\
4.18499994277954	61.5860900878906\\
4.19000005722046	36.6621398925781\\
4.19500017166138	20.8050842285156\\
4.19999980926514	20.8793334960938\\
4.20499992370605	28.6411743164063\\
4.21000003814697	34.6852416992188\\
4.21500015258789	40.8407287597656\\
4.21999979019165	47.4003295898438\\
4.22499990463257	53.1894226074219\\
4.23000001907349	60.6217956542969\\
4.2350001335144	64.660400390625\\
4.23999977111816	62.6019897460938\\
4.24499988555908	53.12890625\\
4.25	42.5394897460938\\
4.25500011444092	29.8885955810547\\
4.26000022888184	18.6747589111328\\
4.2649998664856	10.7915344238281\\
4.26999998092651	7.80047607421875\\
4.27500009536743	10.2980041503906\\
4.28000020980835	19.6538391113281\\
4.28499984741211	28.5290374755859\\
4.28999996185303	37.8160629272461\\
4.29500007629395	50.0331649780273\\
4.30000019073486	71.7326431274414\\
4.30499982833862	100.524341583252\\
4.30999994277954	123.729174613953\\
4.31500005722046	137.85582691431\\
4.32000017166138	141.998596435663\\
4.32499980926514	138.55269908905\\
4.32999992370605	137.383491516113\\
4.33500003814697	130.769248962402\\
4.34000015258789	117.366302490234\\
4.34499979019165	102.099517822266\\
4.34999990463257	86.5880737304688\\
4.35500001907349	72.4025573730469\\
4.3600001335144	61.5314788818359\\
4.36499977111816	59.1753005981445\\
4.36999988555908	75.6261760963432\\
4.375	84.737031727886\\
4.38000011444092	49.3333456958135\\
4.38500022888184	-7.36509447701269\\
4.3899998664856	-60.5778628183441\\
4.39499998092651	30.9561157226563\\
4.40000009536743	52.7711181640625\\
4.40500020980835	72.7567749023438\\
4.40999984741211	91.1765747070313\\
4.41499996185303	110.858856201172\\
4.42000007629395	128.075347900391\\
4.42500019073486	152.941375732422\\
4.42999982833862	165.163177490234\\
4.43499994277954	148.616271972656\\
4.44000005722046	113.974822998047\\
4.44500017166138	73.3323669433594\\
4.44999980926514	36.2294616699219\\
4.45499992370605	9.51507568359375\\
4.46000003814697	8.30020141601563\\
4.46500015258789	17.7458190917969\\
4.46999979019165	26.3580017089844\\
4.47499990463257	34.9700927734375\\
4.48000001907349	44.0977478027344\\
4.4850001335144	52.5057067871094\\
4.48999977111816	60.9818420410156\\
4.49499988555908	69.1546020507813\\
4.5	67.1763000488281\\
4.50500011444092	57.1049499511719\\
4.51000022888184	41.7003479003906\\
4.5149998664856	24.7098846435547\\
4.51999998092651	9.57417297363281\\
4.52500009536743	-1.28556823730469\\
4.53000020980835	-5.2127685546875\\
4.53499984741211	-1.53923034667969\\
4.53999996185303	10.7357559204102\\
4.54500007629395	22.1701049804688\\
4.55000019073486	35.1859741210938\\
4.55499982833862	53.7613887786865\\
4.55999994277954	103.443657354241\\
4.56500005722046	139.621254364512\\
4.57000017166138	153.756866893358\\
4.57499980926514	153.735870875538\\
4.57999992370605	144.329895618788\\
4.58500003814697	128.699783997307\\
4.59000015258789	112.42231822523\\
4.59499979019165	102.733368660344\\
4.59999990463257	85.5385139275473\\
4.60500001907349	65.1142127477144\\
4.6100001335144	40.4108435250347\\
4.61499977111816	10.8735966749411\\
4.61999988555908	125.405883789063\\
4.625	218.384193835814\\
4.63000011444092	251.768113655023\\
4.63500022888184	168.043886754583\\
4.6399998664856	51.2897676666305\\
4.64499998092651	-62.7307091647378\\
4.65000009536743	-128.163967050618\\
4.65500020980835	-185.633312212734\\
4.65999984741211	-230.996883718859\\
4.66499996185303	30.3623962402344\\
4.67000007629395	93.8028869628906\\
4.67500019073486	141.182739257813\\
4.67999982833862	175.413024902344\\
4.68499994277954	221.122619628906\\
4.69000005722046	250.545104980469\\
4.69500017166138	215.96923828125\\
4.69999980926514	147.224639892578\\
4.70499992370605	67.0578308105469\\
4.71000003814697	-5.37164306640625\\
4.71500015258789	-56.7756958007813\\
4.71999979019165	-60.1600494384766\\
4.72499990463257	-33.4666748046875\\
4.73000001907349	-13.1946411132813\\
4.7350001335144	9.09237670898438\\
4.73999977111816	31.0466613769531\\
4.74499988555908	49.955810546875\\
4.75	62.7159423828125\\
4.75500011444092	81.0509033203125\\
4.76000022888184	79.6475524902344\\
4.7649998664856	64.9964294433594\\
4.76999998092651	35.0040740966797\\
4.77500009536743	8.49066162109375\\
4.78000020980835	-17.0377311706543\\
4.78499984741211	-24.2157483306864\\
4.78999996185303	-23.2228615369334\\
4.79500007629395	-28.4033353601714\\
4.80000019073486	17.2140197753906\\
4.80499982833862	56.7204988002777\\
4.80999994277954	187.528152761187\\
4.81500005722046	222.098190753663\\
4.82000017166138	218.561875046883\\
4.82499980926514	199.067688859527\\
4.82999992370605	168.26912032951\\
4.83500003814697	137.479005114968\\
4.84000015258789	108.476579049152\\
4.84499979019165	71.1151441200668\\
4.84999990463257	29.4967968480687\\
4.85500001907349	-15.6540365237535\\
4.8600001335144	-61.1164583280888\\
4.86499977111816	-54.7681036871823\\
4.86999988555908	206.500015122268\\
4.875	300.156021768519\\
4.88000011444092	262.38972648252\\
4.88500022888184	151.367678568102\\
4.8899998664856	23.0553624353697\\
4.89499998092651	-91.0206259292036\\
4.90000009536743	-149.610683172285\\
4.90500020980835	-209.435924412881\\
4.90999984741211	-252.72221025131\\
4.91499996185303	-272.142865214614\\
4.92000007629395	78.8900756835938\\
4.92500019073486	186.728607177734\\
4.92999982833862	251.018432617188\\
4.93499994277954	307.186294555664\\
4.94000005722046	341.680938720703\\
4.94500017166138	347.878753662109\\
4.94999980926514	282.043914794922\\
4.95499992370605	185.9951171875\\
4.96000003814697	87.2883605957031\\
4.96500015258789	6.18984985351563\\
4.96999979019165	-45.1042175292969\\
4.97499990463257	-26.5996704101563\\
4.98000001907349	0.629547119140625\\
4.9850001335144	19.9991455078125\\
4.98999977111816	40.2846984863281\\
4.99499988555908	60.46142578125\\
5	79.0851135253906\\
5.00500011444092	93.1911010742188\\
5.01000022888184	110.807373046875\\
5.0149998664856	119.433441162109\\
5.01999998092651	112.785522460938\\
5.02500009536743	88.1769714355469\\
5.03000020980835	58.3761901855469\\
5.03499984741211	30.186767578125\\
5.03999996185303	13.6554718017578\\
5.04500007629395	0.0894622802734375\\
5.05000019073486	-8.83932495117188\\
5.05499982833862	-11.7126922607422\\
5.05999994277954	-10.2024078369141\\
5.06500005722046	-5.70022583007813\\
5.07000017166138	-0.421501159667969\\
5.07499980926514	4.20491027832031\\
5.07999992370605	7.97334766387939\\
5.08500003814697	47.6244068846858\\
5.09000015258789	55.8533119307902\\
5.09499979019165	49.4546358064646\\
5.09999990463257	38.4069368973404\\
5.10500001907349	43.3515542072061\\
5.1100001335144	44.8423014218479\\
5.11499977111816	42.3533330778345\\
5.11999988555908	38.4042657607181\\
5.125	34.1949280871589\\
5.13000011444092	30.27310795837\\
5.13500022888184	258.381209631096\\
5.1399998664856	191.601433783548\\
5.14499998092651	100.325624092938\\
5.15000009536743	13.5688495188879\\
5.15500020980835	-50.8999690333037\\
5.15999984741211	73.0744323730469\\
5.16499996185303	92.8545227050781\\
5.17000007629395	106.14111328125\\
5.17500019073486	117.678863525391\\
5.17999982833862	130.442993164063\\
5.18499994277954	148.099426269531\\
5.19000005722046	150.375122070313\\
5.19500017166138	142.188812255859\\
5.19999980926514	118.623046875\\
5.20499992370605	90.9640808105469\\
5.21000003814697	66.185791015625\\
5.21500015258789	48.4153442382813\\
5.21999979019165	39.2608947753906\\
5.22499990463257	42.568603515625\\
5.23000001907349	56.3580627441406\\
5.2350001335144	64.8274536132813\\
5.23999977111816	69.7402648925781\\
5.24499988555908	73.783203125\\
5.25	77.8544921875\\
5.25500011444092	80.0763549804688\\
5.26000022888184	85.343017578125\\
5.2649998664856	86.0976867675781\\
5.26999998092651	84.7965087890625\\
5.27500009536743	77.7803649902344\\
5.28000020980835	67.7606506347656\\
5.28499984741211	56.6133117675781\\
5.28999996185303	46.6590270996094\\
5.29500007629395	38.7579040527344\\
5.30000019073486	35.0224914550781\\
5.30499982833862	32.9923706054688\\
5.30999994277954	32.5396118164063\\
5.31500005722046	33.0608215332031\\
5.32000017166138	38.7451171875\\
5.32499980926514	42.8331604003906\\
5.32999992370605	45.1653747558594\\
5.33500003814697	46.4205932617188\\
5.34000015258789	46.7860260009766\\
5.34499979019165	48.3372039794922\\
5.34999990463257	46.6546173095703\\
5.35500001907349	42.3370819091797\\
5.3600001335144	35.9456481933594\\
5.36499977111816	28.3576812744141\\
5.36999988555908	20.0790405273438\\
5.375	13.1263885498047\\
5.38000011444092	7.34717559814453\\
5.38500022888184	4.26284790039063\\
5.3899998664856	3.68657684326172\\
5.39499998092651	6.39743804931641\\
5.40000009536743	8.33016967773438\\
5.40500020980835	9.59230804443359\\
5.40999984741211	12.5757293701172\\
5.41499996185303	13.8230590820313\\
5.42000007629395	12.9180793762207\\
5.42500019073486	10.0429916381836\\
5.42999982833862	9.5510211156091\\
5.43499994277954	15.74013161405\\
5.44000005722046	17.7671286888713\\
5.44500017166138	24.4491360773192\\
5.44999980926514	28.0579314697764\\
5.45499992370605	27.7786068098142\\
5.46000003814697	25.9076650581615\\
5.46500015258789	23.5241587090635\\
5.46999979019165	21.2080235242601\\
5.47499990463257	19.0931145027138\\
5.48000001907349	17.1697890537306\\
5.4850001335144	15.427741284402\\
5.48999977111816	13.9019159220218\\
5.49499988555908	12.5871760291107\\
5.5	11.382070143179\\
5.50500011444092	10.3381444633865\\
5.51000022888184	9.44427045865235\\
5.5149998664856	8.65450845119109\\
5.51999998092651	28.937220522409\\
5.52500009536743	40.8459000747152\\
5.53000020980835	31.7871228156462\\
5.53499984741211	23.3460606623284\\
5.53999996185303	15.8989448072092\\
5.54500007629395	9.6812882222942\\
5.55000019073486	4.78137730828172\\
5.55499982833862	1.38716300565898\\
5.55999994277954	0.258545320615184\\
5.56500005722046	-0.394222654109228\\
5.57000017166138	-1.26472527078514\\
5.57499980926514	-1.5373342982401\\
5.57999992370605	-1.3319004897125\\
5.58500003814697	-0.618603164281609\\
5.59000015258789	0.527956507682902\\
5.59499979019165	2.19697425627834\\
5.59999990463257	4.35733598114229\\
5.60500001907349	6.77759589487748\\
5.6100001335144	9.46762166777414\\
5.61499977111816	12.096838787166\\
5.61999988555908	14.4145402225166\\
5.625	16.4680385931892\\
5.63000011444092	18.0279658061289\\
5.63500022888184	19.0308861869282\\
5.6399998664856	19.822102463753\\
5.64499998092651	19.9214602781434\\
5.65000009536743	19.7805985744872\\
5.65500020980835	19.4002605982036\\
5.65999984741211	18.8673038865605\\
5.66499996185303	18.2942380581638\\
5.67000007629395	21.2722895764766\\
5.67500019073486	51.3632007517412\\
5.67999982833862	32.2661408327654\\
5.68499994277954	16.2522234916687\\
5.69000005722046	56.6831398010254\\
5.69500017166138	50.7744445800781\\
5.69999980926514	40.5987854003906\\
5.70499992370605	28.7182521820068\\
5.71000003814697	16.9381046295166\\
5.71500015258789	11.9789009094238\\
5.71999979019165	8.9215145111084\\
5.72499990463257	0.162282943725586\\
5.73000001907349	-8.51020050048828\\
5.7350001335144	-14.5391540527344\\
5.73999977111816	-16.7731971740723\\
5.74499988555908	-15.1940231323242\\
5.75	-10.6391334533691\\
5.75500011444092	-4.05853652954102\\
5.76000022888184	4.35475921630859\\
5.7649998664856	11.8228187561035\\
5.76999998092651	17.2969551086426\\
5.77500009536743	22.8167877197266\\
5.78000020980835	26.7586975097656\\
5.78499984741211	30.7944412231445\\
5.78999996185303	35.7306861877441\\
5.79500007629395	40.9526634216309\\
5.80000019073486	45.7563972473145\\
5.80499982833862	51.5732002258301\\
5.80999994277954	57.141357421875\\
5.81500005722046	62.3736724853516\\
5.82000017166138	67.5077590942383\\
5.82499980926514	72.1002578735352\\
5.82999992370605	75.0294418334961\\
5.83500003814697	76.5810241699219\\
5.84000015258789	75.949089050293\\
5.84499979019165	73.3754959106445\\
5.84999990463257	68.9979248046875\\
5.85500001907349	67.6220932006836\\
5.8600001335144	58.4179000854492\\
5.86499977111816	44.4210014343262\\
5.86999988555908	25.4460601806641\\
5.875	4.0571403503418\\
5.88000011444092	-16.6953315734863\\
5.88500022888184	-33.8216400146484\\
5.8899998664856	-3.25038086519112\\
5.89499998092651	123.882431165729\\
5.90000009536743	201.969574039035\\
5.90500020980835	280.441221340393\\
5.90999984741211	373.122373897499\\
5.91499996185303	434.069548480025\\
5.92000007629395	438.694976523035\\
5.92500019073486	391.439514862971\\
5.92999982833862	307.692750789095\\
5.93499994277954	210.99237339288\\
5.94000005722046	113.358585253425\\
5.94500017166138	26.1654097543451\\
5.94999980926514	-46.4550730606325\\
5.95499992370605	-105.796137128638\\
5.96000003814697	-129.496820528199\\
5.96500015258789	-163.240228736449\\
5.96999979019165	-202.752680178341\\
5.97499990463257	-241.685969321843\\
5.98000001907349	454.068176269531\\
5.9850001335144	523.911254882813\\
5.98999977111816	549.37109375\\
5.99499988555908	503.947814941406\\
6	450.966186523438\\
6.00500011444092	517.701293945313\\
6.01000022888184	509.463256835938\\
6.0149998664856	425.56396484375\\
6.01999998092651	310.223510742188\\
6.02500009536743	187.73974609375\\
6.03000020980835	78.0029296875\\
6.03499984741211	-1.89825439453125\\
6.03999996185303	-43.0479125976563\\
6.04500007629395	30.5547485351563\\
6.05000019073486	80.8829956054688\\
6.05499982833862	112.275146484375\\
6.05999994277954	137.300903320313\\
6.06500005722046	158.801147460938\\
6.07000017166138	176.202392578125\\
6.07499980926514	188.928894042969\\
6.07999992370605	204.919372558594\\
6.08500003814697	225.590576171875\\
6.09000015258789	230.672790527344\\
6.09499979019165	201.468994140625\\
6.09999990463257	154.889221191406\\
6.10500001907349	105.215576171875\\
6.1100001335144	61.7442016601563\\
6.11499977111816	42.7283020019531\\
6.11999988555908	27.357666015625\\
6.125	19.8351745605469\\
6.13000011444092	34.7830200195313\\
6.13500022888184	45.4046020507813\\
6.1399998664856	54.0984497070313\\
6.14499998092651	61.9868774414063\\
6.15000009536743	68.64208984375\\
6.15500020980835	81.9720458984375\\
6.15999984741211	89.1870727539063\\
6.16499996185303	87.25927734375\\
6.17000007629395	77.5075988769531\\
6.17500019073486	62.4862213134766\\
6.17999982833862	43.9809951782227\\
6.18499994277954	25.3760471343994\\
6.19000005722046	50.7009316300309\\
6.19500017166138	53.4630356026078\\
6.19999980926514	44.6504055802546\\
6.20499992370605	34.7040802876813\\
6.21000003814697	27.6710587328303\\
6.21500015258789	45.9074356507393\\
6.21999979019165	55.4843686280911\\
6.22499990463257	55.4827874519662\\
6.23000001907349	51.7349248926214\\
6.2350001335144	46.5931730792915\\
6.23999977111816	41.3311967311308\\
6.24499988555908	36.4564556570706\\
6.25	32.0562560226674\\
6.25500011444092	28.1313353130408\\
6.26000022888184	24.6544439318784\\
6.2649998664856	21.5925565244416\\
6.26999998092651	18.9079708784903\\
6.27500009536743	16.5588961168848\\
6.28000020980835	14.5012329682372\\
6.28499984741211	12.6988696110187\\
6.28999996185303	11.1168702085357\\
6.29500007629395	9.73009786400939\\
6.30000019073486	8.51511267084825\\
6.30499982833862	7.45005460770697\\
6.30999994277954	6.51545233225109\\
6.31500005722046	5.69821333533074\\
6.32000017166138	4.98160139148648\\
6.32499980926514	4.35646096601648\\
6.32999992370605	3.81089516953106\\
6.33500003814697	3.33086604980338\\
6.34000015258789	2.90938093843202\\
6.34499979019165	2.54171962604459\\
6.34999990463257	2.2229909081476\\
6.35500001907349	1.94806899699576\\
6.3600001335144	92.3499814571266\\
6.36499977111816	91.5449985546114\\
6.36999988555908	81.6543127110043\\
6.375	67.2717521093857\\
6.38000011444092	51.5574577359984\\
6.38500022888184	80.6441459655762\\
6.3899998664856	174.777069091797\\
6.39499998092651	144.989562988281\\
6.40000009536743	111.769927978516\\
6.40500020980835	87.9262237548828\\
6.40999984741211	80.5839233398438\\
6.41499996185303	78.3063659667969\\
6.42000007629395	81.8143005371094\\
6.42500019073486	89.5720520019531\\
6.42999982833862	99.7881164550781\\
6.43499994277954	110.675659179688\\
6.44000005722046	120.855072021484\\
6.44500017166138	129.360137939453\\
6.44999980926514	135.63330078125\\
6.45499992370605	139.668151855469\\
6.46000003814697	141.619506835938\\
6.46500015258789	141.851501464844\\
6.46999979019165	141.162048339844\\
6.47499990463257	139.841674804688\\
6.48000001907349	138.612365722656\\
6.4850001335144	137.892822265625\\
6.48999977111816	137.990295410156\\
6.49499988555908	138.975158691406\\
6.5	140.872131347656\\
6.50500011444092	143.528991699219\\
6.51000022888184	146.656433105469\\
6.5149998664856	150.049072265625\\
6.51999998092651	153.39208984375\\
6.52500009536743	156.464782714844\\
6.53000020980835	158.9189453125\\
6.53499984741211	161.039367675781\\
6.53999996185303	162.545776367188\\
6.54500007629395	163.496826171875\\
6.55000019073486	163.933837890625\\
6.55499982833862	164.239440917969\\
6.55999994277954	164.843017578125\\
6.56500005722046	164.917663574219\\
6.57000017166138	164.423583984375\\
6.57499980926514	163.293273925781\\
6.57999992370605	163.011474609375\\
6.58500003814697	162.507690429688\\
6.59000015258789	162.300537109375\\
6.59499979019165	162.352172851563\\
6.59999990463257	162.649597167969\\
6.60500001907349	162.952697753906\\
6.6100001335144	163.507141113281\\
6.61499977111816	164.214050292969\\
6.61999988555908	164.564392089844\\
6.625	165.004333496094\\
6.63000011444092	165.237976074219\\
6.63500022888184	165.335021972656\\
6.6399998664856	165.72509765625\\
6.64499998092651	165.580627441406\\
6.65000009536743	165.401306152344\\
6.65500020980835	165.474060058594\\
6.65999984741211	165.108520507813\\
6.66499996185303	164.8642578125\\
6.67000007629395	165.240966796875\\
6.67500019073486	165.034912109375\\
6.67999982833862	164.903625488281\\
6.68499994277954	164.123229980469\\
6.69000005722046	162.658935546875\\
6.69500017166138	163.068908691406\\
6.69999980926514	163.397644042969\\
6.70499992370605	162.828308105469\\
6.71000003814697	161.553405761719\\
6.71500015258789	159.660766601563\\
6.71999979019165	157.602416992188\\
6.72499990463257	155.380859375\\
6.73000001907349	152.808715820313\\
6.7350001335144	150.036437988281\\
6.73999977111816	147.15234375\\
6.74499988555908	144.673828125\\
6.75	142.058044433594\\
6.75500011444092	139.513244628906\\
6.76000022888184	136.546752929688\\
6.7649998664856	133.862243652344\\
6.76999998092651	131.737182617188\\
6.77500009536743	129.52490234375\\
6.78000020980835	127.623413085938\\
6.78499984741211	125.934326171875\\
6.78999996185303	124.275573730469\\
6.79500007629395	122.68896484375\\
6.80000019073486	121.1435546875\\
6.80499982833862	119.582641601563\\
6.80999994277954	118.142822265625\\
6.81500005722046	116.501159667969\\
6.82000017166138	114.983764648438\\
6.82499980926514	113.310180664063\\
6.82999992370605	111.521209716797\\
6.83500003814697	109.656311035156\\
6.84000015258789	107.921752929688\\
6.84499979019165	105.957153320313\\
6.84999990463257	103.978240966797\\
6.85500001907349	102.110809326172\\
6.8600001335144	100.408142089844\\
6.86499977111816	98.3860778808594\\
6.86999988555908	96.9726867675781\\
6.875	95.7157592773438\\
6.88000011444092	94.2854919433594\\
6.88500022888184	92.8218994140625\\
6.8899998664856	91.18408203125\\
6.89499998092651	89.547607421875\\
6.90000009536743	87.7247619628906\\
6.90500020980835	85.6915893554688\\
6.90999984741211	83.3719787597656\\
6.91499996185303	80.9373474121094\\
6.92000007629395	78.345703125\\
6.92500019073486	75.6290893554688\\
6.92999982833862	72.8764495849609\\
6.93499994277954	70.1627655029297\\
6.94000005722046	67.5541687011719\\
6.94500017166138	65.0139465332031\\
6.94999980926514	62.5877685546875\\
6.95499992370605	60.1653442382813\\
6.96000003814697	58.1535034179688\\
6.96500015258789	56.3564758300781\\
6.96999979019165	54.1176452636719\\
6.97499990463257	52.4886016845703\\
6.98000001907349	50.4623260498047\\
6.9850001335144	48.48583984375\\
6.98999977111816	46.4439315795898\\
6.99499988555908	44.2677841186523\\
7	42.1552886962891\\
7.00500011444092	39.9224395751953\\
7.01000022888184	37.7289199829102\\
7.0149998664856	35.5842819213867\\
7.01999998092651	33.5519638061523\\
7.02500009536743	31.6407165527344\\
7.03000020980835	29.8625411987305\\
7.03499984741211	28.232234954834\\
7.03999996185303	26.7663154602051\\
7.04500007629395	25.4594650268555\\
7.05000019073486	24.2934417724609\\
7.05499982833862	23.2443046569824\\
7.05999994277954	22.2689628601074\\
7.06500005722046	21.3690223693848\\
7.07000017166138	20.5058898925781\\
7.07499980926514	19.6340637207031\\
7.07999992370605	18.7560386657715\\
7.08500003814697	17.8789768218994\\
7.09000015258789	17.0171165466309\\
7.09499979019165	16.1760311126709\\
7.09999990463257	15.3501586914063\\
7.10500001907349	14.5816555023193\\
7.1100001335144	13.9121856689453\\
7.11499977111816	13.4208850860596\\
7.11999988555908	13.0212154388428\\
7.125	12.6188383102417\\
7.13000011444092	12.4381618499756\\
7.13500022888184	12.4034881591797\\
7.1399998664856	12.3784646987915\\
7.14499998092651	12.4139461517334\\
7.15000009536743	12.5038509368896\\
7.15500020980835	12.697735786438\\
7.15999984741211	12.8008184432983\\
7.16499996185303	12.6936712265015\\
7.17000007629395	12.4766845703125\\
7.17500019073486	12.3312435150146\\
7.17999982833862	12.3267393112183\\
7.18499994277954	12.3187065124512\\
7.19000005722046	12.3739395141602\\
7.19500017166138	12.4936904907227\\
7.19999980926514	12.674919128418\\
7.20499992370605	12.8826017379761\\
7.21000003814697	13.1535530090332\\
7.21500015258789	13.4552736282349\\
7.21999979019165	13.7072219848633\\
7.22499990463257	14.0115079879761\\
7.23000001907349	14.3054885864258\\
7.2350001335144	14.568943977356\\
7.23999977111816	14.8335027694702\\
7.24499988555908	15.0744590759277\\
7.25	15.2930154800415\\
7.25500011444092	15.5189294815063\\
7.26000022888184	15.7319593429565\\
7.2649998664856	15.9449901580811\\
7.26999998092651	16.1673555374146\\
7.27500009536743	16.3695392608643\\
7.28000020980835	16.5746364593506\\
7.28499984741211	16.781777381897\\
7.28999996185303	16.9867544174194\\
7.29500007629395	17.1898336410522\\
7.30000019073486	17.3775215148926\\
7.30499982833862	17.5538921356201\\
7.30999994277954	17.7215700149536\\
7.31500005722046	17.8779621124268\\
7.32000017166138	18.0129661560059\\
7.32499980926514	18.1288394927979\\
7.32999992370605	18.2193803787231\\
7.33500003814697	18.2845640182495\\
7.34000015258789	18.3342089653015\\
7.34499979019165	18.3665418624878\\
7.34999990463257	18.3750863075256\\
7.35500001907349	18.3624811172485\\
7.3600001335144	18.3371577262878\\
7.36499977111816	18.3001108169556\\
7.36999988555908	18.2536191940308\\
7.375	18.1994767189026\\
7.38000011444092	18.1408319473267\\
7.38500022888184	18.0560274124146\\
7.3899998664856	17.9312858581543\\
7.39499998092651	17.7720899581909\\
7.40000009536743	17.6236581802368\\
7.40500020980835	17.4780836105347\\
7.40999984741211	17.3191871643066\\
7.41499996185303	17.1469237804413\\
7.42000007629395	16.963470697403\\
7.42500019073486	16.7693471908569\\
7.42999982833862	16.5638291835785\\
7.43499994277954	16.3466558456421\\
7.44000005722046	16.1176378726959\\
7.44500017166138	15.8817310333252\\
7.44999980926514	15.6390359401703\\
7.45499992370605	15.3966026306152\\
7.46000003814697	15.1625232696533\\
7.46500015258789	14.9394381046295\\
7.46999979019165	14.7234508991241\\
7.47499990463257	14.516798377037\\
7.48000001907349	14.3132230043411\\
7.4850001335144	14.1082977056503\\
7.48999977111816	13.9064126014709\\
7.49499988555908	13.7064657211304\\
7.5	13.5080137252808\\
7.50500011444092	13.3118348121643\\
7.51000022888184	13.1197193861008\\
7.5149998664856	12.9366462230682\\
7.51999998092651	12.7681757211685\\
7.52500009536743	12.6299592256546\\
7.53000020980835	12.4115824699402\\
7.53499984741211	12.2832176685333\\
7.53999996185303	12.1422398090363\\
7.54500007629395	12.0275994539261\\
7.55000019073486	11.9314497709274\\
7.55499982833862	11.8497619628906\\
7.55999994277954	11.7817133665085\\
7.56500005722046	11.7243086099625\\
7.57000017166138	11.6739529371262\\
7.57499980926514	11.6287488937378\\
7.57999992370605	11.5895076990128\\
7.58500003814697	11.5537465810776\\
7.59000015258789	11.524134516716\\
7.59499979019165	11.4999628067017\\
7.59999990463257	11.4817770719528\\
7.60500001907349	11.4715216159821\\
7.6100001335144	11.4804508686066\\
7.61499977111816	11.5213875770569\\
7.61999988555908	11.5707626342773\\
7.625	11.6008632183075\\
7.63000011444092	11.6354843378067\\
7.63500022888184	11.6981518268585\\
7.6399998664856	11.7726744413376\\
7.64499998092651	11.8559085130692\\
7.65000009536743	11.9413645267487\\
7.65500020980835	12.0268141031265\\
7.65999984741211	12.1156979799271\\
7.66499996185303	12.2045772075653\\
7.67000007629395	12.2890963554382\\
7.67500019073486	12.3715052604675\\
7.67999982833862	12.4580421447754\\
7.68499994277954	12.5516893863678\\
7.69000005722046	12.6554056406021\\
7.69500017166138	12.7572487592697\\
7.69999980926514	12.8569811582565\\
7.70499992370605	12.9551231861115\\
7.71000003814697	13.0560866594315\\
7.71500015258789	13.1752519607544\\
7.71999979019165	13.3001050949097\\
7.72499990463257	13.4169554710388\\
7.73000001907349	13.5251088142395\\
7.7350001335144	13.6214764118195\\
7.73999977111816	13.7320694923401\\
7.74499988555908	13.8430516719818\\
7.75	13.9354546070099\\
7.75500011444092	14.0232424736023\\
7.76000022888184	14.1152169704437\\
7.7649998664856	14.2016949653625\\
7.76999998092651	14.2806403636932\\
7.77500009536743	14.353328704834\\
7.78000020980835	14.4209847450256\\
7.78499984741211	14.4933264255524\\
7.78999996185303	14.5643208026886\\
7.79500007629395	14.6339001655579\\
7.80000019073486	14.6977109909058\\
7.80499982833862	14.7572593688965\\
7.80999994277954	14.8128514289856\\
7.81500005722046	14.8638265132904\\
7.82000017166138	14.9100489616394\\
7.82499980926514	14.9516389369965\\
7.82999992370605	14.9896216392517\\
7.83500003814697	15.0251505374908\\
7.84000015258789	15.0571789741516\\
7.84499979019165	15.0859112739563\\
7.84999990463257	15.1123623847961\\
7.85500001907349	15.1363296508789\\
7.8600001335144	15.1580200195313\\
7.86499977111816	15.1753211021423\\
7.86999988555908	15.1888866424561\\
7.875	15.1984791755676\\
7.88000011444092	15.2075982093811\\
7.88500022888184	15.2171568870544\\
7.8899998664856	15.2272272109985\\
7.89499998092651	15.2374610900879\\
7.90000009536743	15.2472553253174\\
7.90500020980835	15.257643699646\\
7.90999984741211	15.2677888870239\\
7.91499996185303	15.2740864753723\\
7.92000007629395	15.2775297164917\\
7.92500019073486	15.2780952453613\\
7.92999982833862	15.2857327461243\\
7.93499994277954	15.2951483726501\\
7.94000005722046	15.3063716888428\\
7.94500017166138	15.3236284255981\\
7.94999980926514	15.3457260131836\\
7.95499992370605	15.3721733093262\\
7.96000003814697	15.401704788208\\
7.96500015258789	15.4340000152588\\
7.96999979019165	15.4696588516235\\
7.97499990463257	15.5110235214233\\
7.98000001907349	15.561728477478\\
7.9850001335144	15.6191778182983\\
7.98999977111816	15.6830539703369\\
7.99499988555908	15.7512178421021\\
8	15.8249483108521\\
8.00500011444092	15.9042863845825\\
8.01000022888184	15.992714881897\\
8.01500034332275	16.0900592803955\\
8.02000045776367	16.1970148086548\\
8.02499961853027	16.3102693557739\\
8.02999973297119	16.4286651611328\\
8.03499984741211	16.552903175354\\
8.03999996185303	16.6853103637695\\
8.04500007629395	16.8297481536865\\
8.05000019073486	16.9852676391602\\
8.05500030517578	17.1521949768066\\
8.0600004196167	17.3286762237549\\
8.0649995803833	17.5162563323975\\
8.06999969482422	17.7153186798096\\
8.07499980926514	17.925329208374\\
8.07999992370605	18.1470012664795\\
8.08500003814697	18.3806571960449\\
8.09000015258789	18.6231994628906\\
8.09500026702881	18.8740520477295\\
8.10000038146973	19.1331596374512\\
8.10499954223633	19.4039611816406\\
8.10999965667725	19.69016456604\\
8.11499977111816	19.9909152984619\\
8.11999988555908	20.306116104126\\
8.125	20.6334075927734\\
8.13000011444092	20.974479675293\\
8.13500022888184	21.3296966552734\\
8.14000034332275	21.7001914978027\\
8.14500045776367	22.0861473083496\\
8.14999961853027	22.4882049560547\\
8.15499973297119	22.9048271179199\\
8.15999984741211	23.3364639282227\\
8.16499996185303	23.7832565307617\\
8.17000007629395	24.247501373291\\
8.17500019073486	24.7310485839844\\
8.18000030517578	25.2344093322754\\
8.1850004196167	25.7578392028809\\
8.1899995803833	26.300163269043\\
8.19499969482422	26.8624534606934\\
8.19999980926514	27.4417495727539\\
8.20499992370605	28.0343322753906\\
8.21000003814697	28.6408767700195\\
8.21500015258789	29.2782211303711\\
8.22000026702881	29.962272644043\\
8.22500038146973	30.6926803588867\\
8.22999954223633	31.4533462524414\\
8.23499965667725	32.2393188476563\\
8.23999977111816	33.0252227783203\\
8.24499988555908	33.7769393920898\\
8.25	34.5639724731445\\
8.25500011444092	35.4599075317383\\
8.26000022888184	36.3895721435547\\
8.26500034332275	37.3401260375977\\
8.27000045776367	38.3494033813477\\
8.27499961853027	39.4111862182617\\
8.27999973297119	40.4895782470703\\
8.28499984741211	41.6013488769531\\
8.28999996185303	42.7965240478516\\
8.29500007629395	44.0173034667969\\
8.30000019073486	45.2277069091797\\
8.30500030517578	46.4965209960938\\
8.3100004196167	47.8420562744141\\
8.3149995803833	49.7643737792969\\
8.31999969482422	50.9357452392578\\
8.32499980926514	52.4950714111328\\
8.32999992370605	54.1678009033203\\
8.33500003814697	55.8921356201172\\
8.34000015258789	57.6899566650391\\
8.34500026702881	59.5790405273438\\
8.35000038146973	61.6019744873047\\
8.35499954223633	63.6374816894531\\
8.35999965667725	66.1414489746094\\
8.36499977111816	68.4034423828125\\
8.36999988555908	70.7382507324219\\
8.375	73.250244140625\\
8.38000011444092	75.8736877441406\\
8.38500022888184	78.5851745605469\\
8.39000034332275	81.4790954589844\\
8.39500045776367	84.3841552734375\\
8.39999961853027	88.2413024902344\\
8.40499973297119	91.2187805175781\\
8.40999984741211	94.6243591308594\\
8.41499996185303	98.1980895996094\\
8.42000007629395	101.783477783203\\
8.42500019073486	105.546936035156\\
8.43000030517578	109.242126464844\\
8.4350004196167	113.109466552734\\
8.4399995803833	117.203735351563\\
8.44499969482422	120.796813964844\\
8.44999980926514	124.596374511719\\
8.45499992370605	128.483764648438\\
8.46000003814697	132.035095214844\\
8.46500015258789	135.927124023438\\
8.47000026702881	139.632934570313\\
8.47500038146973	143.227111816406\\
8.47999954223633	146.699035644531\\
8.48499965667725	150.102111816406\\
8.48999977111816	153.361450195313\\
8.49499988555908	156.400695800781\\
8.5	159.343200683594\\
8.50500011444092	161.967102050781\\
8.51000022888184	164.610412597656\\
8.51500034332275	166.98291015625\\
8.52000045776367	169.432739257813\\
8.52499961853027	171.695556640625\\
8.52999973297119	173.6220703125\\
8.53499984741211	175.209411621094\\
8.53999996185303	176.899475097656\\
8.54500007629395	178.368713378906\\
8.55000019073486	179.712524414063\\
8.55500030517578	180.938903808594\\
8.5600004196167	182.244262695313\\
8.5649995803833	183.42822265625\\
8.56999969482422	184.514465332031\\
8.57499980926514	185.410400390625\\
8.57999992370605	186.134521484375\\
8.58500003814697	186.7685546875\\
8.59000015258789	187.419067382813\\
8.59500026702881	187.33154296875\\
8.60000038146973	186.441650390625\\
8.60499954223633	185.129638671875\\
8.60999965667725	183.180541992188\\
8.61499977111816	180.813354492188\\
8.61999988555908	178.406860351563\\
8.625	175.782653808594\\
8.63000011444092	173.131713867188\\
8.63500022888184	170.479797363281\\
8.64000034332275	168.093872070313\\
8.64500045776367	165.591979980469\\
8.64999961853027	163.561645507813\\
8.65499973297119	161.154846191406\\
8.65999984741211	160.222900390625\\
8.66499996185303	157.929443359375\\
8.67000007629395	156.139831542969\\
8.67500019073486	153.933532714844\\
8.68000030517578	151.951110839844\\
8.6850004196167	149.725402832031\\
8.6899995803833	147.185485839844\\
8.69499969482422	144.368591308594\\
8.69999980926514	141.649108886719\\
8.70499992370605	138.761169433594\\
8.71000003814697	135.668273925781\\
8.71500015258789	132.431884765625\\
8.72000026702881	129.588623046875\\
8.72500038146973	126.790832519531\\
8.72999954223633	124.075744628906\\
8.73499965667725	121.598327636719\\
8.73999977111816	119.413940429688\\
8.74499988555908	117.367431640625\\
8.75	115.398376464844\\
8.75500011444092	113.169738769531\\
8.76000022888184	111.460632324219\\
8.76500034332275	109.586883544922\\
8.77000045776367	107.725341796875\\
8.77499961853027	105.773590087891\\
8.77999973297119	103.846496582031\\
8.78499984741211	101.817596435547\\
8.78999996185303	99.7725219726563\\
8.79500007629395	97.7763061523438\\
8.80000019073486	95.8710327148438\\
8.80500030517578	94.0582275390625\\
8.8100004196167	92.3502197265625\\
8.8149995803833	90.8033752441406\\
8.81999969482422	89.4072875976563\\
8.82499980926514	88.1583862304688\\
8.82999992370605	87.0720825195313\\
8.83500003814697	86.068603515625\\
8.84000015258789	85.1370544433594\\
8.84500026702881	84.3057250976563\\
8.85000038146973	83.4178466796875\\
8.85499954223633	82.6029052734375\\
8.85999965667725	81.7579345703125\\
8.86499977111816	80.8959045410156\\
8.86999988555908	80.0254821777344\\
8.875	79.1787719726563\\
8.88000011444092	78.3868713378906\\
8.88500022888184	77.6564025878906\\
8.89000034332275	76.9995422363281\\
8.89500045776367	76.4156188964844\\
8.89999961853027	75.9223022460938\\
8.90499973297119	75.6033630371094\\
8.90999984741211	75.3289947509766\\
8.91499996185303	74.9457092285156\\
8.92000007629395	74.6560516357422\\
8.92500019073486	74.5646667480469\\
8.93000030517578	74.7477722167969\\
8.9350004196167	74.8531341552734\\
8.9399995803833	74.2566528320313\\
8.94499969482422	73.8726806640625\\
8.94999980926514	73.7698059082031\\
8.95499992370605	73.6179504394531\\
8.96000003814697	73.5396575927734\\
8.96500015258789	73.5009307861328\\
8.97000026702881	73.5259704589844\\
8.97500038146973	73.5425109863281\\
8.97999954223633	73.5364074707031\\
8.98499965667725	73.7318725585938\\
8.98999977111816	73.8447723388672\\
8.99499988555908	73.7905731201172\\
9	73.9671173095703\\
9.00500011444092	74.1338806152344\\
9.01000022888184	74.2521667480469\\
9.01500034332275	74.3298950195313\\
9.02000045776367	74.4081420898438\\
9.02499961853027	74.473876953125\\
9.02999973297119	74.5193176269531\\
9.03499984741211	74.5498657226563\\
9.03999996185303	74.5652160644531\\
9.04500007629395	74.5690307617188\\
9.05000019073486	74.5649108886719\\
9.05500030517578	74.5516662597656\\
9.0600004196167	74.5230407714844\\
9.0649995803833	74.4806823730469\\
9.06999969482422	74.4237670898438\\
9.07499980926514	74.3514404296875\\
9.07999992370605	74.2691955566406\\
9.08500003814697	74.1816101074219\\
9.09000015258789	74.0807495117188\\
9.09500026702881	73.9388427734375\\
9.10000038146973	73.7562866210938\\
9.10499954223633	73.6177978515625\\
9.10999965667725	73.4896850585938\\
9.11499977111816	73.3313293457031\\
9.11999988555908	73.1319580078125\\
9.125	72.8851013183594\\
9.13000011444092	72.5494079589844\\
9.13500022888184	72.1731262207031\\
9.14000034332275	71.8374633789063\\
9.14500045776367	71.5161743164063\\
9.14999961853027	71.2406616210938\\
9.15499973297119	71.0000610351563\\
9.15999984741211	70.7908325195313\\
9.16499996185303	70.6023864746094\\
9.17000007629395	70.4175415039063\\
9.17500019073486	70.2311706542969\\
9.18000030517578	70.0339660644531\\
9.1850004196167	69.8229675292969\\
9.1899995803833	69.59521484375\\
9.19499969482422	69.3434295654297\\
9.19999980926514	69.0746459960938\\
9.20499992370605	68.8014373779297\\
9.21000003814697	68.5343170166016\\
9.21500015258789	68.2409057617188\\
9.22000026702881	67.9292449951172\\
9.22500038146973	67.6625366210938\\
9.22999954223633	67.4335784912109\\
9.23499965667725	67.3625030517578\\
9.23999977111816	67.7814178466797\\
9.24499988555908	69.0203094482422\\
9.25	70.6695404052734\\
9.25500011444092	72.2333068847656\\
9.26000022888184	73.937744140625\\
9.26500034332275	76.0252990722656\\
9.27000045776367	78.4295501708984\\
9.27499961853027	81.0623779296875\\
9.27999973297119	83.8434753417969\\
9.28499984741211	86.9114685058594\\
9.28999996185303	90.4256591796875\\
9.29500007629395	94.3202819824219\\
9.30000019073486	98.3840942382813\\
9.30500030517578	104.251983642578\\
9.3100004196167	109.187103271484\\
9.3149995803833	114.754821777344\\
9.31999969482422	120.871612548828\\
9.32499980926514	127.356201171875\\
9.32999992370605	134.913269042969\\
9.33500003814697	142.001098632813\\
9.34000015258789	148.917236328125\\
9.34500026702881	159.18212890625\\
9.35000038146973	166.464660644531\\
9.35499954223633	173.078674316406\\
9.35999965667725	182.435852050781\\
9.36499977111816	188.265991210938\\
9.36999988555908	192.721069335938\\
9.375	196.540161132813\\
9.38000011444092	198.934936523438\\
9.38500022888184	200.649230957031\\
9.39000034332275	202.437133789063\\
9.39500045776367	204.628601074219\\
9.39999961853027	206.582763671875\\
9.40499973297119	208.889465332031\\
9.40999984741211	211.736022949219\\
9.41499996185303	214.59765625\\
9.42000007629395	216.521301269531\\
9.42500019073486	218.622985839844\\
9.43000030517578	228.543151855469\\
9.4350004196167	230.210021972656\\
9.4399995803833	221.992065429688\\
9.44499969482422	204.978698730469\\
9.44999980926514	176.697662353516\\
9.45499992370605	138.172225952148\\
9.46000003814697	89.4794921875\\
9.46500015258789	36.106689453125\\
9.47000026702881	-15.0630187988281\\
9.47500038146973	-56.5291748046875\\
9.47999954223633	-79.6289215087891\\
9.48499965667725	-19.7776456605161\\
9.48999977111816	147.925173332028\\
9.49499988555908	186.148439601318\\
9.5	161.850404905115\\
9.50500011444092	119.62408657309\\
9.51000022888184	76.8059408209974\\
9.51500034332275	40.7123738737496\\
9.52000045776367	12.5820026617848\\
9.52499961853027	-7.69674507689842\\
9.52999973297119	-21.8753341951365\\
9.53499984741211	-28.761427646054\\
9.53999996185303	-29.4393607505959\\
9.54500007629395	-24.4248500497764\\
9.55000019073486	-13.1091962044638\\
9.55500030517578	54.7966238804296\\
9.5600004196167	75.3806103369121\\
9.5649995803833	78.2161416812276\\
9.56999969482422	74.0621143692834\\
9.57499980926514	67.494743528796\\
9.57999992370605	60.2589857522232\\
9.58500003814697	53.5154654581184\\
9.59000015258789	47.312132687546\\
9.59500026702881	41.7047913185005\\
9.60000038146973	36.7208625911558\\
9.60499954223633	32.3787535319451\\
9.60999965667725	28.5874389409233\\
9.61499977111816	25.2385996990413\\
9.61999988555908	22.311579240437\\
9.625	19.7238496378806\\
9.63000011444092	17.4757985484086\\
9.63500022888184	15.4792706802751\\
9.64000034332275	13.7423914426323\\
9.64500045776367	12.2555304653512\\
9.64999961853027	10.9235186675798\\
9.65499973297119	9.75449015296651\\
9.65999984741211	8.7655402764279\\
9.66499996185303	7.89356203639567\\
9.67000007629395	7.12524355034909\\
9.67500019073486	6.45553321329577\\
9.68000030517578	5.86823327308866\\
9.6850004196167	5.35394252359436\\
9.6899995803833	4.90295145934327\\
9.69499969482422	4.50672572001676\\
9.69999980926514	4.16117302872705\\
9.70499992370605	3.86107402437576\\
9.71000003814697	3.59678561991935\\
9.71500015258789	3.35937604163137\\
9.72000026702881	3.150047078204\\
9.72500038146973	2.96838131658482\\
9.72999954223633	2.80917757087469\\
9.73499965667725	2.67125960569766\\
9.73999977111816	2.55386390221734\\
9.74499988555908	2.45648681152486\\
9.75	2.36731321230789\\
9.75500011444092	2.28635851950026\\
9.76000022888184	2.20935365682287\\
9.76500034332275	2.14012671697869\\
9.77000045776367	2.08478370546434\\
9.77499961853027	2.04071505360704\\
9.77999973297119	2.00933157047458\\
9.78499984741211	1.97707597372449\\
9.78999996185303	1.94789214906539\\
9.79500007629395	1.91982605745882\\
9.80000019073486	1.89414946190012\\
9.80500030517578	1.87265720821654\\
9.8100004196167	1.85386418996604\\
9.8149995803833	1.83798974088808\\
9.81999969482422	1.82525564620294\\
9.82499980926514	1.8165351842465\\
9.82999992370605	1.81284584421557\\
9.83500003814697	1.80994331821892\\
9.84000015258789	1.80262917458872\\
9.84500026702881	1.79068112449237\\
9.85000038146973	1.77240937428944\\
9.85499954223633	1.76714533012711\\
9.85999965667725	1.76715859277089\\
9.86499977111816	1.77384897772086\\
9.86999988555908	1.77907004810942\\
9.875	1.77664932425189\\
9.88000011444092	1.76922085521092\\
9.88500022888184	1.75628592883651\\
9.89000034332275	1.75243019359915\\
9.89500045776367	1.75179087550028\\
9.89999961853027	1.75547805198183\\
9.90499973297119	1.75857013510915\\
9.90999984741211	1.75793656783539\\
9.91499996185303	1.7556717389813\\
9.92000007629395	1.75180205322431\\
9.92500019073486	1.75152337071559\\
9.93000030517578	1.75236576274765\\
9.9350004196167	1.7543882084452\\
9.9399995803833	1.7544054346328\\
9.94499969482422	1.75164791507224\\
9.94999980926514	1.74768417634195\\
9.95499992370605	1.74257018720351\\
9.96000003814697	1.73636209523245\\
9.96500015258789	1.72911616721355\\
9.97000026702881	1.72088890835018\\
9.97500038146973	1.71665426656785\\
9.97999954223633	1.71602370899417\\
9.98499965667725	1.71641346372481\\
9.98999977111816	1.71781762989996\\
9.99499988555908	1.72023006824124\\
10	1.72364466927262\\
};
\addlegendentry{Imbalance}

\end{axis}

\begin{axis}[%
width=4.521in,
height=0.823in,
at={(0.758in,0.602in)},
scale only axis,
xmin=0,
xmax=10,
xlabel style={font=\color{white!15!black}},
xlabel={Time (s)},
ymin=-3446.1826171875,
ymax=1481.43334960938,
ylabel style={font=\color{white!15!black}},
ylabel={TZ (N-m)},
axis background/.style={fill=white},
axis x line*=bottom,
axis y line*=left,
xmajorgrids,
ymajorgrids,
legend style={at={(0.9,0.6)}, anchor=north east, legend cell align=left, align=left, draw=black}
]
\addplot [color=red, line width=1.5pt]
  table[row sep=crcr]{%
0.0949999988079071	3.85962867736816\\
0.100000001490116	3.21730661392212\\
0.104999996721745	2.65948390960693\\
0.109999999403954	2.19050407409668\\
0.115000002086163	1.7765017747879\\
0.119999997317791	-14.9887647628784\\
0.125	-99.0188064575195\\
0.129999995231628	-163.96891784668\\
0.135000005364418	-215.662124633789\\
0.140000000596046	-252.81071472168\\
0.144999995827675	-277.689056396484\\
0.150000005960464	-292.741027832031\\
0.155000001192093	-299.20458984375\\
0.159999996423721	-299.216125488281\\
0.165000006556511	-296.085052490234\\
0.170000001788139	-286.469024658203\\
0.174999997019768	-269.698699951172\\
0.180000007152557	-244.485366821289\\
0.185000002384186	-213.141296386719\\
0.189999997615814	-176.060089111328\\
0.194999992847443	-61.0715522766113\\
0.200000002980232	268.891876220703\\
0.204999998211861	553.742309570313\\
0.209999993443489	730.387512207031\\
0.215000003576279	762.326354980469\\
0.219999998807907	685.0048828125\\
0.224999994039536	547.672485351563\\
0.230000004172325	349.876373291016\\
0.234999999403954	118.396362304688\\
0.239999994635582	-110.889266967773\\
0.245000004768372	-304.151031494141\\
0.25	-433.485900878906\\
0.254999995231628	-477.86572265625\\
0.259999990463257	-427.856964111328\\
0.264999985694885	-329.022064208984\\
0.270000010728836	-198.327835083008\\
0.275000005960464	-56.913516998291\\
0.280000001192093	73.162467956543\\
0.284999996423721	172.281143188477\\
0.28999999165535	230.74235534668\\
0.294999986886978	245.673706054688\\
0.300000011920929	223.571914672852\\
0.305000007152557	174.726837158203\\
0.310000002384186	106.175567626953\\
0.314999997615814	28.171854019165\\
0.319999992847443	-45.9491386413574\\
0.324999988079071	-105.824981689453\\
0.330000013113022	-144.276565551758\\
0.33500000834465	-161.425872802734\\
0.340000003576279	-151.061813354492\\
0.344999998807907	-112.879089355469\\
0.349999994039536	-62.5822792053223\\
0.354999989271164	-9.76411151885986\\
0.360000014305115	36.9717140197754\\
0.365000009536743	71.6178207397461\\
0.370000004768372	93.0497360229492\\
0.375	100.697212219238\\
0.379999995231628	92.0357055664063\\
0.384999990463257	69.3412628173828\\
0.389999985694885	37.2348785400391\\
0.395000010728836	0.701812267303467\\
0.400000005960464	-32.9198875427246\\
0.405000001192093	-59.2621383666992\\
0.409999996423721	-74.3174514770508\\
0.41499999165535	-75.8555145263672\\
0.419999986886978	-65.5822982788086\\
0.425000011920929	-45.0392112731934\\
0.430000007152557	-17.4512634277344\\
0.435000002384186	9.79677391052246\\
0.439999997615814	32.6138801574707\\
0.444999992847443	49.5855751037598\\
0.449999988079071	57.6171340942383\\
0.455000013113022	55.9862899780273\\
0.46000000834465	45.7848167419434\\
0.465000003576279	29.093807220459\\
0.469999998807907	9.36760711669922\\
0.474999994039536	-10.4007730484009\\
0.479999989271164	-26.7442073822021\\
0.485000014305115	-37.3670654296875\\
0.490000009536743	-40.8719444274902\\
0.495000004768372	-36.5681762695313\\
0.5	-24.9424419403076\\
0.504999995231628	-10.1691770553589\\
0.509999990463257	4.96419191360474\\
0.514999985694885	17.7531757354736\\
0.519999980926514	26.4782161712646\\
0.524999976158142	30.3951740264893\\
0.529999971389771	29.3198165893555\\
0.535000026226044	24.2373027801514\\
0.540000021457672	16.6259746551514\\
0.545000016689301	7.8694109916687\\
0.550000011920929	-0.868697285652161\\
0.555000007152557	-7.90472364425659\\
0.560000002384186	-11.8156852722168\\
0.564999997615814	-12.5925159454346\\
0.569999992847443	-10.9963674545288\\
0.574999988079071	-7.86548519134521\\
0.579999983310699	-3.55096983909607\\
0.584999978542328	1.03998029232025\\
0.589999973773956	4.38741827011108\\
0.595000028610229	7.27582454681396\\
0.600000023841858	8.56149768829346\\
0.605000019073486	8.834397315979\\
0.610000014305115	7.97562503814697\\
0.615000009536743	6.03977632522583\\
0.620000004768372	3.92855405807495\\
0.625	1.87825870513916\\
0.629999995231628	-0.250868946313858\\
0.634999990463257	-1.83516919612885\\
0.639999985694885	-2.86832475662231\\
0.644999980926514	-3.26159143447876\\
0.649999976158142	-3.10485553741455\\
0.654999971389771	-2.45196437835693\\
0.660000026226044	-1.63692533969879\\
0.665000021457672	-0.737406730651855\\
0.670000016689301	0.0987247228622437\\
0.675000011920929	0.721289217472076\\
0.680000007152557	1.06295919418335\\
0.685000002384186	1.0984228849411\\
0.689999997615814	0.854670584201813\\
0.694999992847443	0.400078564882278\\
0.699999988079071	-0.165123879909515\\
0.704999983310699	-0.743454873561859\\
0.709999978542328	-1.2287665605545\\
0.714999973773956	-1.60903453826904\\
0.720000028610229	-1.83084452152252\\
0.725000023841858	-1.90250110626221\\
0.730000019073486	-1.82533633708954\\
0.735000014305115	-1.67824769020081\\
0.740000009536743	-1.452392578125\\
0.745000004768372	-1.22309458255768\\
0.75	-1.00321233272552\\
0.754999995231628	-0.836131989955902\\
0.759999990463257	-0.72693943977356\\
0.764999985694885	-0.700340569019318\\
0.769999980926514	-0.776643574237823\\
0.774999976158142	-0.841283798217773\\
0.779999971389771	-0.884264349937439\\
0.785000026226044	-0.924356341362\\
0.790000021457672	-0.96444296836853\\
0.795000016689301	-0.989326417446136\\
0.800000011920929	-1.00042939186096\\
0.805000007152557	-0.983866572380066\\
0.810000002384186	-0.947827756404877\\
0.814999997615814	-0.877414464950562\\
0.819999992847443	-0.76276683807373\\
0.824999988079071	-0.644884765148163\\
0.829999983310699	-0.495423406362534\\
0.834999978542328	-0.355769753456116\\
0.839999973773956	-0.254584729671478\\
0.845000028610229	-0.15548937022686\\
0.850000023841858	-0.100833356380463\\
0.855000019073486	-0.0872217565774918\\
0.860000014305115	-0.111905828118324\\
0.865000009536743	-0.122037522494793\\
0.870000004768372	-0.110092662274837\\
0.875	-0.0873701199889183\\
0.879999995231628	-0.0454946123063564\\
0.884999990463257	0.0145019087940454\\
0.889999985694885	0.0890734791755676\\
0.894999980926514	0.176619559526443\\
0.899999976158142	0.272781997919083\\
0.904999971389771	0.380632072687149\\
0.910000026226044	0.478641659021378\\
0.915000021457672	0.588780045509338\\
0.920000016689301	0.678619503974915\\
0.925000011920929	0.756910741329193\\
0.930000007152557	0.803578436374664\\
0.935000002384186	0.828485906124115\\
0.939999997615814	0.812867999076843\\
0.944999992847443	0.774191558361053\\
0.949999988079071	0.714296460151672\\
0.954999983310699	0.677212655544281\\
0.959999978542328	0.650791227817535\\
0.964999973773956	0.645981550216675\\
0.970000028610229	0.653942048549652\\
0.975000023841858	0.688352406024933\\
0.980000019073486	0.734657168388367\\
0.985000014305115	0.801074147224426\\
0.990000009536743	0.86952543258667\\
0.995000004768372	0.934326708316803\\
1	0.977861940860748\\
1.00499999523163	0.998886525630951\\
1.00999999046326	0.966372728347778\\
1.01499998569489	0.898537874221802\\
1.01999998092651	0.841132342815399\\
1.02499997615814	0.792582094669342\\
1.02999997138977	0.746636748313904\\
1.0349999666214	0.700758516788483\\
1.03999996185303	0.659482538700104\\
1.04499995708466	0.621843636035919\\
1.04999995231628	0.587642550468445\\
1.05499994754791	0.555740714073181\\
1.05999994277954	0.526299118995667\\
1.06500005722046	0.50055193901062\\
1.07000005245209	0.477409690618515\\
1.07500004768372	0.457538217306137\\
1.08000004291534	0.437828242778778\\
1.08500003814697	0.41572043299675\\
1.0900000333786	0.38995760679245\\
1.09500002861023	0.359143018722534\\
1.10000002384186	0.321577697992325\\
1.10500001907349	0.276844412088394\\
1.11000001430511	0.228602588176727\\
1.11500000953674	0.179398030042648\\
1.12000000476837	0.128034323453903\\
1.125	0.0769885629415512\\
1.12999999523163	0.029110124334693\\
1.13499999046326	-0.0145742688328028\\
1.13999998569489	-0.0530684925615788\\
1.14499998092651	-0.0861873850226402\\
1.14999997615814	-0.11664354801178\\
1.15499997138977	-0.143301755189896\\
1.1599999666214	-0.165118530392647\\
1.16499996185303	-0.184218138456345\\
1.16999995708466	-0.202066972851753\\
1.17499995231628	-0.218370258808136\\
1.17999994754791	-0.230828046798706\\
1.18499994277954	-0.240531429648399\\
1.19000005722046	-0.247833728790283\\
1.19500005245209	-0.253147006034851\\
1.20000004768372	-0.260083198547363\\
1.20500004291534	-0.272619843482971\\
1.21000003814697	-0.292753428220749\\
1.2150000333786	-0.316273838281631\\
1.22000002861023	-0.335277140140533\\
1.22500002384186	-0.348991841077805\\
1.23000001907349	-0.355896860361099\\
1.23500001430511	-0.359262764453888\\
1.24000000953674	-0.36284402012825\\
1.24500000476837	-0.365775436162949\\
1.25	-0.370015323162079\\
1.25499999523163	-0.363268196582794\\
1.25999999046326	-0.343468844890594\\
1.26499998569489	-0.310464203357697\\
1.26999998092651	-0.272577196359634\\
1.27499997615814	-0.239197984337807\\
1.27999997138977	-0.208143189549446\\
1.2849999666214	-0.179875373840332\\
1.28999996185303	-0.161925747990608\\
1.29499995708466	-0.156635329127312\\
1.29999995231628	-0.159714177250862\\
1.30499994754791	-0.165927439928055\\
1.30999994277954	-0.176760643720627\\
1.31500005722046	-0.189950183033943\\
1.32000005245209	-0.198719039559364\\
1.32500004768372	-0.200579985976219\\
1.33000004291534	-0.194693103432655\\
1.33500003814697	-0.18022209405899\\
1.3400000333786	-0.155104577541351\\
1.34500002861023	-0.117747209966183\\
1.35000002384186	-0.068583220243454\\
1.35500001907349	-0.00871393457055092\\
1.36000001430511	0.0593706369400024\\
1.36500000953674	0.122640945017338\\
1.37000000476837	0.162415876984596\\
1.375	0.168510586023331\\
1.37999999523163	0.149049445986748\\
1.38499999046326	0.126153096556664\\
1.38999998569489	0.0916421860456467\\
1.39499998092651	0.0487776547670364\\
1.39999997615814	0.0127435699105263\\
1.40499997138977	-0.013399014249444\\
1.4099999666214	-0.0258578211069107\\
1.41499996185303	-0.0213463362306356\\
1.41999995708466	0.00324819190427661\\
1.42499995231628	0.0411916188895702\\
1.42999994754791	0.0824158787727356\\
1.43499994277954	0.127093851566315\\
1.44000005722046	0.172227948904037\\
1.44500005245209	0.214736968278885\\
1.45000004768372	0.250991672277451\\
1.45500004291534	0.276370197534561\\
1.46000003814697	0.287188708782196\\
1.4650000333786	0.279163837432861\\
1.47000002861023	0.245120421051979\\
1.47500002384186	0.177050739526749\\
1.48000001907349	0.085660956799984\\
1.48500001430511	0.0166184809058905\\
1.49000000953674	0.0117866024374962\\
1.49500000476837	0.049376867711544\\
1.5	0.0430140420794487\\
1.50499999523163	-0.0106639163568616\\
1.50999999046326	-0.0822324827313423\\
1.51499998569489	-0.139897793531418\\
1.51999998092651	-0.188835665583611\\
1.52499997615814	-0.231162771582603\\
1.52999997138977	-0.264225989580154\\
1.5349999666214	-0.289148211479187\\
1.53999996185303	-0.313512086868286\\
1.54499995708466	-0.335601717233658\\
1.54999995231628	-0.348361283540726\\
1.55499994754791	-0.344853520393372\\
1.55999994277954	-0.326763182878494\\
1.56500005722046	-0.298086196184158\\
1.57000005245209	0.595933258533478\\
1.57500004768372	3.97082448005676\\
1.58000004291534	3.62398314476013\\
1.58500003814697	2.26498126983643\\
1.5900000333786	0.982154369354248\\
1.59500002861023	-0.152407318353653\\
1.60000002384186	-1.0628844499588\\
1.60500001907349	-1.69169592857361\\
1.61000001430511	-1.97217857837677\\
1.61500000953674	-1.88566315174103\\
1.62000000476837	-1.51802623271942\\
1.625	-0.864418745040894\\
1.62999999523163	-0.132862523198128\\
1.63499999046326	0.514520287513733\\
1.63999998569489	0.903713941574097\\
1.64499998092651	0.991884052753448\\
1.64999997615814	1.03716349601746\\
1.65499997138977	0.999494314193726\\
1.6599999666214	0.862979352474213\\
1.66499996185303	0.673070073127747\\
1.66999995708466	0.448097884654999\\
1.67499995231628	0.168526440858841\\
1.67999994754791	-0.0547609068453312\\
1.68499994277954	-0.145566344261169\\
1.69000005722046	-0.165899962186813\\
1.69500005245209	-0.118177846074104\\
1.70000004768372	-0.0295162759721279\\
1.70500004291534	0.090145044028759\\
1.71000003814697	0.201143950223923\\
1.7150000333786	0.303549379110336\\
1.72000002861023	0.408485263586044\\
1.72500002384186	0.508984744548798\\
1.73000001907349	0.598111391067505\\
1.73500001430511	0.673383712768555\\
1.74000000953674	0.73090386390686\\
1.74500000476837	0.771396636962891\\
1.75	0.797666728496552\\
1.75499999523163	0.810138642787933\\
1.75999999046326	0.811669230461121\\
1.76499998569489	0.806558787822723\\
1.76999998092651	0.796236276626587\\
1.77499997615814	0.788393080234528\\
1.77999997138977	0.792628288269043\\
1.7849999666214	0.809837281703949\\
1.78999996185303	0.83891749382019\\
1.79499995708466	0.881159722805023\\
1.79999995231628	0.934614837169647\\
1.80499994754791	0.99513828754425\\
1.80999994277954	1.06080067157745\\
1.81500005722046	1.12690854072571\\
1.82000005245209	1.19051051139832\\
1.82500004768372	1.25016415119171\\
1.83000004291534	1.30455100536346\\
1.83500003814697	1.35096228122711\\
1.8400000333786	1.38666880130768\\
1.84500002861023	1.41252613067627\\
1.85000002384186	1.43542730808258\\
1.85500001907349	1.46243965625763\\
1.86000001430511	1.49092173576355\\
1.86500000953674	1.52058374881744\\
1.87000000476837	1.5514303445816\\
1.875	1.58670127391815\\
1.87999999523163	1.63308358192444\\
1.88499999046326	1.69244349002838\\
1.88999998569489	1.76257264614105\\
1.89499998092651	1.83367609977722\\
1.89999997615814	1.89522051811218\\
1.90499997138977	1.95277512073517\\
1.9099999666214	2.00752282142639\\
1.91499996185303	2.05867552757263\\
1.91999995708466	2.10632729530334\\
1.92499995231628	2.15357232093811\\
1.92999994754791	2.20435905456543\\
1.93499994277954	2.25844097137451\\
1.94000005722046	2.31444954872131\\
1.94500005245209	2.37225532531738\\
1.95000004768372	2.43224215507507\\
1.95500004291534	2.49736881256104\\
1.96000003814697	2.57699680328369\\
1.9650000333786	2.67940449714661\\
1.97000002861023	2.80081653594971\\
1.97500002384186	2.91710615158081\\
1.98000001907349	3.00336194038391\\
1.98500001430511	3.05971169471741\\
1.99000000953674	3.09629416465759\\
1.99500000476837	3.12678694725037\\
2	3.14604234695435\\
2.00500011444092	3.16753339767456\\
2.00999999046326	3.20828580856323\\
2.01500010490417	3.28363108634949\\
2.01999998092651	3.39962816238403\\
2.02500009536743	3.57069873809814\\
2.02999997138977	3.80332827568054\\
2.03500008583069	4.06847524642944\\
2.03999996185303	4.26133251190186\\
2.04500007629395	4.38065624237061\\
2.04999995231628	4.44797086715698\\
2.0550000667572	4.46639204025269\\
2.05999994277954	4.4074592590332\\
2.06500005722046	4.23943758010864\\
2.0699999332428	3.94337582588196\\
2.07500004768372	3.50470876693726\\
2.07999992370605	2.91785860061646\\
2.08500003814697	2.20974040031433\\
2.08999991416931	1.33795344829559\\
2.09500002861023	0.464001685380936\\
2.09999990463257	-0.484288036823273\\
2.10500001907349	-1.46285486221313\\
2.10999989509583	-2.42390370368958\\
2.11500000953674	-3.32504987716675\\
2.11999988555908	-4.29082870483398\\
2.125	-5.00202131271362\\
2.13000011444092	-5.42916393280029\\
2.13499999046326	-5.68517160415649\\
2.14000010490417	-5.80109357833862\\
2.14499998092651	-5.82345914840698\\
2.15000009536743	-5.7905068397522\\
2.15499997138977	-5.76676416397095\\
2.16000008583069	-5.91176843643188\\
2.16499996185303	-6.22635889053345\\
2.17000007629395	-6.58970546722412\\
2.17499995231628	-6.98871803283691\\
2.1800000667572	-7.51782846450806\\
2.18499994277954	-7.98949956893921\\
2.19000005722046	-8.30752086639404\\
2.1949999332428	-8.40865325927734\\
2.20000004768372	-8.30029773712158\\
2.20499992370605	-7.97623109817505\\
2.21000003814697	-7.45500421524048\\
2.21499991416931	-6.78718900680542\\
2.22000002861023	-6.03164768218994\\
2.22499990463257	-5.22581577301025\\
2.23000001907349	-4.43341970443726\\
2.23499989509583	-3.79924130439758\\
2.24000000953674	-3.4002890586853\\
2.24499988555908	-3.12106823921204\\
2.25	-2.95717573165894\\
2.25500011444092	-2.94146251678467\\
2.25999999046326	-3.01165652275085\\
2.26500010490417	-3.04147672653198\\
2.26999998092651	-2.91490936279297\\
2.27500009536743	-2.59434270858765\\
2.27999997138977	-2.13459539413452\\
2.28500008583069	-1.53677773475647\\
2.28999996185303	-1.05097925662994\\
2.29500007629395	-0.708818733692169\\
2.29999995231628	-0.332800567150116\\
2.3050000667572	0.313991785049438\\
2.30999994277954	0.793654799461365\\
2.31500005722046	1.27523291110992\\
2.3199999332428	1.82696878910065\\
2.32500004768372	2.40218949317932\\
2.32999992370605	2.95326900482178\\
2.33500003814697	3.47283458709717\\
2.33999991416931	3.82204937934875\\
2.34500002861023	3.77544283866882\\
2.34999990463257	3.69422316551208\\
2.35500001907349	4.08896398544312\\
2.35999989509583	4.63342046737671\\
2.36500000953674	5.30769538879395\\
2.36999988555908	6.09939956665039\\
2.375	6.96662521362305\\
2.38000011444092	8.02955532073975\\
2.38499999046326	9.28909015655518\\
2.39000010490417	10.441611289978\\
2.39499998092651	11.5706939697266\\
2.40000009536743	12.5780639648438\\
2.40499997138977	13.312292098999\\
2.41000008583069	13.8142251968384\\
2.41499996185303	13.8866729736328\\
2.42000007629395	13.5712022781372\\
2.42499995231628	12.9266557693481\\
2.4300000667572	12.0202255249023\\
2.43499994277954	10.8439960479736\\
2.44000005722046	9.48384094238281\\
2.4449999332428	7.93053293228149\\
2.45000004768372	6.15201234817505\\
2.45499992370605	4.55241918563843\\
2.46000003814697	2.97843980789185\\
2.46499991416931	1.48226571083069\\
2.47000002861023	0.426844418048859\\
2.47499990463257	-0.504794359207153\\
2.48000001907349	-1.28474926948547\\
2.48499989509583	-1.9316668510437\\
2.49000000953674	-2.46622705459595\\
2.49499988555908	-2.86330676078796\\
2.5	-3.07387328147888\\
2.50500011444092	-3.33440184593201\\
2.50999999046326	-3.90103673934937\\
2.51500010490417	-4.74953508377075\\
2.51999998092651	-6.05715227127075\\
2.52500009536743	-7.31399297714233\\
2.52999997138977	-8.59846782684326\\
2.53500008583069	-9.83919906616211\\
2.53999996185303	-10.8184690475464\\
2.54500007629395	-11.6643648147583\\
2.54999995231628	-12.3032960891724\\
2.5550000667572	-12.7410745620728\\
2.55999994277954	-13.0460958480835\\
2.56500005722046	-13.1626005172729\\
2.5699999332428	-13.2601366043091\\
2.57500004768372	-13.3596754074097\\
2.57999992370605	-13.4852027893066\\
2.58500003814697	-13.6835279464722\\
2.58999991416931	-13.9852142333984\\
2.59500002861023	-14.372594833374\\
2.59999990463257	-14.6517553329468\\
2.60500001907349	-14.843599319458\\
2.60999989509583	-14.9063005447388\\
2.61500000953674	-14.7427816390991\\
2.61999988555908	-14.2865810394287\\
2.625	-13.4922504425049\\
2.63000011444092	-12.3649501800537\\
2.63499999046326	-11.0002031326294\\
2.64000010490417	-9.44873237609863\\
2.64499998092651	-7.5907883644104\\
2.65000009536743	-5.68748760223389\\
2.65499997138977	-4.04191112518311\\
2.66000008583069	-2.56325459480286\\
2.66499996185303	-1.2942281961441\\
2.67000007629395	-0.396213203668594\\
2.67499995231628	0.2695472240448\\
2.6800000667572	0.936742901802063\\
2.68499994277954	1.48963582515717\\
2.69000005722046	2.02196884155273\\
2.6949999332428	2.62343788146973\\
2.70000004768372	3.48613905906677\\
2.70499992370605	4.53385162353516\\
2.71000003814697	5.79257583618164\\
2.71499991416931	7.30804538726807\\
2.72000002861023	8.77013301849365\\
2.72499990463257	10.2617349624634\\
2.73000001907349	11.7208032608032\\
2.73499989509583	13.0688962936401\\
2.74000000953674	14.27086353302\\
2.74499988555908	15.2509384155273\\
2.75	16.014181137085\\
2.75500011444092	16.6996154785156\\
2.75999999046326	17.4068489074707\\
2.76500010490417	18.2097396850586\\
2.76999998092651	19.0754528045654\\
2.77500009536743	20.0067939758301\\
2.77999997138977	21.061092376709\\
2.78500008583069	22.3727531433105\\
2.78999996185303	23.9586162567139\\
2.79500007629395	25.6870632171631\\
2.79999995231628	27.4527320861816\\
2.8050000667572	29.1135787963867\\
2.80999994277954	30.3831787109375\\
2.81500005722046	31.1011638641357\\
2.8199999332428	31.1430931091309\\
2.82500004768372	30.3654499053955\\
2.82999992370605	28.9062728881836\\
2.83500003814697	26.4239444732666\\
2.83999991416931	23.3112316131592\\
2.84500002861023	19.7238025665283\\
2.84999990463257	15.5700664520264\\
2.85500001907349	11.788275718689\\
2.85999989509583	7.7492790222168\\
2.86500000953674	3.88265919685364\\
2.86999988555908	-0.983975887298584\\
2.875	-5.4055438041687\\
2.88000011444092	-10.3340177536011\\
2.88499999046326	-15.828652381897\\
2.89000010490417	-21.4332847595215\\
2.89499998092651	-27.947717666626\\
2.90000009536743	-33.8980445861816\\
2.90499997138977	-38.7852935791016\\
2.91000008583069	-42.2060966491699\\
2.91499996185303	-43.7454872131348\\
2.92000007629395	-43.1687889099121\\
2.92499995231628	-40.6620063781738\\
2.9300000667572	-36.6618461608887\\
2.93499994277954	-31.4093227386475\\
2.94000005722046	-26.4211044311523\\
2.9449999332428	-21.7539348602295\\
2.95000004768372	-17.9259834289551\\
2.95499992370605	-14.7111806869507\\
2.96000003814697	-13.1440477371216\\
2.96499991416931	-12.6566934585571\\
2.97000002861023	-12.486855506897\\
2.97499990463257	-12.8459720611572\\
2.98000001907349	-13.5288553237915\\
2.98499989509583	-15.0354537963867\\
2.99000000953674	-17.7106971740723\\
2.99499988555908	-18.5947589874268\\
3	-18.8281002044678\\
3.00500011444092	-18.258731842041\\
3.00999999046326	-16.6912708282471\\
3.01500010490417	-14.4310655593872\\
3.01999998092651	-11.2090864181519\\
3.02500009536743	-7.47566509246826\\
3.02999997138977	-4.37901544570923\\
3.03500008583069	-1.83990824222565\\
3.03999996185303	0.0383524708449841\\
3.04500007629395	0.948413908481598\\
3.04999995231628	0.868299305438995\\
3.0550000667572	0.273162424564362\\
3.05999994277954	-0.419923424720764\\
3.06500005722046	-0.882623612880707\\
3.0699999332428	-0.573673844337463\\
3.07500004768372	0.758308887481689\\
3.07999992370605	2.85356497764587\\
3.08500003814697	5.80367231369019\\
3.08999991416931	9.26621532440186\\
3.09500002861023	12.8038320541382\\
3.09999990463257	16.2385730743408\\
3.10500001907349	19.4347362518311\\
3.10999989509583	22.2008514404297\\
3.11500000953674	24.5679588317871\\
3.11999988555908	26.5273094177246\\
3.125	28.5213603973389\\
3.13000011444092	30.9002323150635\\
3.13499999046326	34.3488616943359\\
3.14000010490417	39.9524841308594\\
3.14499998092651	46.6925086975098\\
3.15000009536743	54.0534782409668\\
3.15499997138977	60.3303260803223\\
3.16000008583069	64.6156311035156\\
3.16499996185303	66.7298431396484\\
3.17000007629395	66.1408004760742\\
3.17499995231628	62.5995674133301\\
3.1800000667572	55.7935485839844\\
3.18499994277954	47.138427734375\\
3.19000005722046	36.5671997070313\\
3.1949999332428	24.6537857055664\\
3.20000004768372	11.8308372497559\\
3.20499992370605	-1.53096973896027\\
3.21000003814697	-14.258415222168\\
3.21499991416931	-26.3561153411865\\
3.22000002861023	-36.2640647888184\\
3.22499990463257	-44.2523002624512\\
3.23000001907349	-50.1356201171875\\
3.23499989509583	-54.0250854492188\\
3.24000000953674	-55.9566268920898\\
3.24499988555908	-55.8310356140137\\
3.25	-53.668529510498\\
3.25500011444092	-50.1350250244141\\
3.25999999046326	-45.5773963928223\\
3.26500010490417	-40.3299331665039\\
3.26999998092651	-34.9493560791016\\
3.27500009536743	-30.4423446655273\\
3.27999997138977	-26.8028812408447\\
3.28500008583069	-24.3358726501465\\
3.28999996185303	-23.2507095336914\\
3.29500007629395	-23.0587787628174\\
3.29999995231628	-23.3785362243652\\
3.3050000667572	-23.9276638031006\\
3.30999994277954	-24.1812000274658\\
3.31500005722046	-23.7961864471436\\
3.3199999332428	-23.1176433563232\\
3.32500004768372	-22.3875427246094\\
3.32999992370605	-20.85422706604\\
3.33500003814697	-18.2003231048584\\
3.33999991416931	-14.9589223861694\\
3.34500002861023	-11.1682147979736\\
3.34999990463257	-7.12969207763672\\
3.35500001907349	-3.57270359992981\\
3.35999989509583	-0.837733268737793\\
3.36500000953674	1.06474316120148\\
3.36999988555908	2.12777924537659\\
3.375	2.60834813117981\\
3.38000011444092	2.60254764556885\\
3.38499999046326	2.5920135974884\\
3.39000010490417	2.97615313529968\\
3.39499998092651	3.92341351509094\\
3.40000009536743	5.5974817276001\\
3.40499997138977	8.0095796585083\\
3.41000008583069	10.7847480773926\\
3.41499996185303	14.3055934906006\\
3.42000007629395	17.9010734558105\\
3.42499995231628	21.4262733459473\\
3.4300000667572	24.7615299224854\\
3.43499994277954	27.834264755249\\
3.44000005722046	30.7644367218018\\
3.4449999332428	33.7582359313965\\
3.45000004768372	36.9879951477051\\
3.45499992370605	40.7638893127441\\
3.46000003814697	45.8998870849609\\
3.46499991416931	53.5608825683594\\
3.47000002861023	63.5760345458984\\
3.47499990463257	71.8838653564453\\
3.48000001907349	80.3932723999023\\
3.48499989509583	87.0419845581055\\
3.49000000953674	89.9817962646484\\
3.49499988555908	88.1343002319336\\
3.5	83.3645782470703\\
3.50500011444092	76.4407043457031\\
3.50999999046326	66.6149520874023\\
3.51500010490417	53.9023895263672\\
3.51999998092651	38.583740234375\\
3.52500009536743	20.1064071655273\\
3.52999997138977	-1.58119225502014\\
3.53500008583069	-26.9023399353027\\
3.53999996185303	-55.7774429321289\\
3.54500007629395	-88.2295532226563\\
3.54999995231628	-120.161659240723\\
3.5550000667572	-145.385833740234\\
3.55999994277954	-158.086700439453\\
3.56500005722046	-156.909957885742\\
3.5699999332428	-141.474273681641\\
3.57500004768372	-115.41805267334\\
3.57999992370605	-82.6202850341797\\
3.58500003814697	-48.2122383117676\\
3.58999991416931	-15.8714218139648\\
3.59500002861023	10.7554473876953\\
3.59999990463257	27.1801261901855\\
3.60500001907349	31.5855350494385\\
3.60999989509583	26.7709083557129\\
3.61500000953674	16.7534065246582\\
3.61999988555908	2.92183208465576\\
3.625	-13.1160287857056\\
3.63000011444092	-28.8869686126709\\
3.63499999046326	-41.4583549499512\\
3.64000010490417	-49.7889595031738\\
3.64499998092651	-51.5143699645996\\
3.65000009536743	-45.9617614746094\\
3.65499997138977	-32.8690223693848\\
3.66000008583069	-16.5076217651367\\
3.66499996185303	-0.569694757461548\\
3.67000007629395	13.448356628418\\
3.67499995231628	22.8851718902588\\
3.6800000667572	26.4839382171631\\
3.68499994277954	24.0959453582764\\
3.69000005722046	17.115629196167\\
3.6949999332428	7.7178111076355\\
3.70000004768372	-2.1881890296936\\
3.70499992370605	-9.30814552307129\\
3.71000003814697	-12.2804889678955\\
3.71499991416931	-10.2487277984619\\
3.72000002861023	-3.18922066688538\\
3.72499990463257	7.38228130340576\\
3.73000001907349	20.5184555053711\\
3.73499989509583	33.533447265625\\
3.74000000953674	45.753044128418\\
3.74499988555908	55.5429878234863\\
3.75	63.6084365844727\\
3.75500011444092	71.9704971313477\\
3.75999999046326	82.5878295898438\\
3.76500010490417	95.9201126098633\\
3.76999998092651	110.18335723877\\
3.77500009536743	122.796859741211\\
3.77999997138977	131.008834838867\\
3.78500008583069	133.351104736328\\
3.78999996185303	129.476257324219\\
3.79500007629395	120.633056640625\\
3.79999995231628	106.878387451172\\
3.8050000667572	87.0522384643555\\
3.80999994277954	60.8378868103027\\
3.81500005722046	28.2064304351807\\
3.8199999332428	-10.3479423522949\\
3.82500004768372	-53.526985168457\\
3.82999992370605	-99.6847610473633\\
3.83500003814697	-146.059127807617\\
3.83999991416931	-186.201583862305\\
3.84500002861023	-210.581451416016\\
3.84999990463257	-213.745864868164\\
3.85500001907349	-196.374038696289\\
3.85999989509583	-163.423477172852\\
3.86500000953674	-120.812210083008\\
3.86999988555908	-73.3461990356445\\
3.875	-26.8290157318115\\
3.88000011444092	12.1686964035034\\
3.88499999046326	38.1360397338867\\
3.89000010490417	46.1325073242188\\
3.89499998092651	39.4361228942871\\
3.90000009536743	23.7665386199951\\
3.90499997138977	4.62692165374756\\
3.91000008583069	-17.5375595092773\\
3.91499996185303	-38.4589233398438\\
3.92000007629395	-53.9998817443848\\
3.92499995231628	-62.7064781188965\\
3.9300000667572	-62.8245544433594\\
3.93499994277954	-51.9224586486816\\
3.94000005722046	-33.4634094238281\\
3.9449999332428	-12.0126800537109\\
3.95000004768372	7.68815231323242\\
3.95499992370605	23.9992179870605\\
3.96000003814697	35.3120155334473\\
3.96499991416931	39.0721130371094\\
3.97000002861023	35.4169731140137\\
3.97499990463257	25.8568115234375\\
3.98000001907349	13.5122175216675\\
3.98499989509583	1.68167674541473\\
3.99000000953674	-7.0212926864624\\
3.99499988555908	-10.4267635345459\\
4	-7.54062175750732\\
4.00500011444092	0.991474509239197\\
4.01000022888184	14.7620439529419\\
4.0149998664856	30.8696346282959\\
4.01999998092651	47.3421058654785\\
4.02500009536743	62.3904228210449\\
4.03000020980835	75.6257095336914\\
4.03499984741211	88.3185501098633\\
4.03999996185303	102.700500488281\\
4.04500007629395	118.846168518066\\
4.05000019073486	135.308395385742\\
4.05499982833862	150.421478271484\\
4.05999994277954	161.617111206055\\
4.06500005722046	166.443649291992\\
4.07000017166138	164.252075195313\\
4.07499980926514	157.380920410156\\
4.07999992370605	145.474533081055\\
4.08500003814697	126.89818572998\\
4.09000015258789	98.9749145507813\\
4.09499979019165	59.6765365600586\\
4.09999990463257	4.49656200408936\\
4.10500001907349	-70.2025909423828\\
4.1100001335144	-168.868804931641\\
4.11499977111816	-277.287445068359\\
4.11999988555908	-364.630981445313\\
4.125	-410.479705810547\\
4.13000011444092	-393.242279052734\\
4.13500022888184	-324.307250976563\\
4.1399998664856	-228.489379882813\\
4.14499998092651	-112.916305541992\\
4.15000009536743	3.11218738555908\\
4.15500020980835	100.420989990234\\
4.15999984741211	163.10417175293\\
4.16499996185303	181.001022338867\\
4.17000007629395	157.36442565918\\
4.17500019073486	114.812614440918\\
4.17999982833862	55.5297698974609\\
4.18499994277954	-13.160906791687\\
4.19000005722046	-76.6983871459961\\
4.19500017166138	-123.633056640625\\
4.19999980926514	-147.415191650391\\
4.20499992370605	-146.722930908203\\
4.21000003814697	-113.076354980469\\
4.21500015258789	-61.7385215759277\\
4.21999979019165	-6.81467056274414\\
4.22499990463257	39.7839736938477\\
4.23000001907349	71.7346343994141\\
4.2350001335144	87.8764114379883\\
4.23999977111816	87.5849533081055\\
4.24499988555908	70.1074295043945\\
4.25	40.3382873535156\\
4.25500011444092	6.07567453384399\\
4.26000022888184	-25.3318252563477\\
4.2649998664856	-46.3661003112793\\
4.26999998092651	-52.3555183410645\\
4.27500009536743	-42.6738815307617\\
4.28000020980835	-17.9408302307129\\
4.28499984741211	17.9694690704346\\
4.28999996185303	54.9786567687988\\
4.29500007629395	87.8600158691406\\
4.30000019073486	116.221366882324\\
4.30499982833862	144.665496826172\\
4.30999994277954	173.765411376953\\
4.31500005722046	199.726837158203\\
4.32000017166138	220.136764526367\\
4.32499980926514	232.315475463867\\
4.32999992370605	234.082870483398\\
4.33500003814697	223.950332641602\\
4.34000015258789	209.275054931641\\
4.34499979019165	185.525100708008\\
4.34999990463257	148.542999267578\\
4.35500001907349	95.4010772705078\\
4.3600001335144	23.0316925048828\\
4.36499977111816	-72.5108795166016\\
4.36999988555908	-190.038192749023\\
4.375	-307.771575927734\\
4.38000011444092	-396.441314697266\\
4.38500022888184	-458.056213378906\\
4.3899998664856	-490.018463134766\\
4.39499998092651	-471.035217285156\\
4.40000009536743	-356.180511474609\\
4.40500020980835	-197.460311889648\\
4.40999984741211	-28.9074401855469\\
4.41499996185303	118.679298400879\\
4.42000007629395	222.108184814453\\
4.42500019073486	264.099700927734\\
4.42999982833862	239.206481933594\\
4.43499994277954	181.60124206543\\
4.44000005722046	98.4889602661133\\
4.44500017166138	2.14944434165955\\
4.44999980926514	-88.6934585571289\\
4.45499992370605	-158.431335449219\\
4.46000003814697	-194.38102722168\\
4.46500015258789	-196.293792724609\\
4.46999979019165	-152.326141357422\\
4.47499990463257	-81.4948883056641\\
4.48000001907349	-5.56525135040283\\
4.4850001335144	59.5951080322266\\
4.48999977111816	102.959999084473\\
4.49499988555908	123.991058349609\\
4.5	123.752288818359\\
4.50500011444092	100.805801391602\\
4.51000022888184	60.699031829834\\
4.5149998664856	13.5263690948486\\
4.51999998092651	-28.897123336792\\
4.52500009536743	-58.3305206298828\\
4.53000020980835	-67.5071334838867\\
4.53499984741211	-56.2279396057129\\
4.53999996185303	-25.1718215942383\\
4.54500007629395	21.1124935150146\\
4.55000019073486	68.3618545532227\\
4.55499982833862	110.58129119873\\
4.55999994277954	147.104751586914\\
4.56500005722046	171.372161865234\\
4.57000017166138	189.217025756836\\
4.57499980926514	206.162460327148\\
4.57999992370605	221.684234619141\\
4.58500003814697	233.847549438477\\
4.59000015258789	240.779296875\\
4.59499979019165	244.281005859375\\
4.59999990463257	245.420944213867\\
4.60500001907349	239.932098388672\\
4.6100001335144	224.557006835938\\
4.61499977111816	195.219009399414\\
4.61999988555908	135.095993041992\\
4.625	-29.030611038208\\
4.63000011444092	-264.393646240234\\
4.63500022888184	-457.548553466797\\
4.6399998664856	-586.554748535156\\
4.64499998092651	-654.747680664063\\
4.65000009536743	-669.617309570313\\
4.65500020980835	-654.7607421875\\
4.65999984741211	-606.843444824219\\
4.66499996185303	-448.181610107422\\
4.67000007629395	-60.449764251709\\
4.67500019073486	249.001373291016\\
4.67999982833862	454.006774902344\\
4.68499994277954	539.628112792969\\
4.69000005722046	498.359252929688\\
4.69500017166138	390.154418945313\\
4.69999980926514	225.820358276367\\
4.70499992370605	31.5400218963623\\
4.71000003814697	-154.313537597656\\
4.71500015258789	-296.365997314453\\
4.71999979019165	-371.283416748047\\
4.72499990463257	-372.17041015625\\
4.73000001907349	-281.256469726563\\
4.7350001335144	-143.378005981445\\
4.73999977111816	3.27137398719788\\
4.74499988555908	125.812347412109\\
4.75	202.612350463867\\
4.75500011444092	230.973037719727\\
4.76000022888184	221.963348388672\\
4.7649998664856	171.248138427734\\
4.76999998092651	89.9079284667969\\
4.77500009536743	0.213827699422836\\
4.78000020980835	-81.0860900878906\\
4.78499984741211	-132.825256347656\\
4.78999996185303	-139.808258056641\\
4.79500007629395	-111.639312744141\\
4.80000019073486	-60.4924583435059\\
4.80499982833862	29.563756942749\\
4.80999994277954	121.913032531738\\
4.81500005722046	195.8095703125\\
4.82000017166138	249.432250976563\\
4.82499980926514	289.543670654297\\
4.82999992370605	321.007629394531\\
4.83500003814697	343.317321777344\\
4.84000015258789	355.863159179688\\
4.84499979019165	357.334899902344\\
4.84999990463257	339.158386230469\\
4.85500001907349	294.542633056641\\
4.8600001335144	217.135864257813\\
4.86499977111816	114.28923034668\\
4.86999988555908	11.8111457824707\\
4.875	-153.756546020508\\
4.88000011444092	-348.208801269531\\
4.88500022888184	-519.091979980469\\
4.8899998664856	-635.652465820313\\
4.89499998092651	-691.459167480469\\
4.90000009536743	-696.018310546875\\
4.90500020980835	-676.208679199219\\
4.90999984741211	-615.356994628906\\
4.91499996185303	-517.470764160156\\
4.92000007629395	-317.962280273438\\
4.92500019073486	107.880661010742\\
4.92999982833862	475.519287109375\\
4.93499994277954	696.762268066406\\
4.94000005722046	740.236755371094\\
4.94500017166138	630.938598632813\\
4.94999980926514	456.028259277344\\
4.95499992370605	229.990356445313\\
4.96000003814697	-6.20174264907837\\
4.96500015258789	-208.418533325195\\
4.96999979019165	-343.072143554688\\
4.97499990463257	-392.900390625\\
4.98000001907349	-360.826507568359\\
4.9850001335144	-257.665863037109\\
4.98999977111816	-115.894065856934\\
4.99499988555908	29.1057281494141\\
5	150.796096801758\\
5.00500011444092	234.05305480957\\
5.01000022888184	273.27001953125\\
5.0149998664856	275.916076660156\\
5.01999998092651	247.440017700195\\
5.02500009536743	192.53791809082\\
5.03000020980835	128.195831298828\\
5.03499984741211	65.9082260131836\\
5.03999996185303	12.068736076355\\
5.04500007629395	-30.7623462677002\\
5.05000019073486	-57.1223297119141\\
5.05499982833862	-65.2118835449219\\
5.05999994277954	-57.2119522094727\\
5.06500005722046	-37.1051940917969\\
5.07000017166138	-9.9479923248291\\
5.07499980926514	20.3507881164551\\
5.07999992370605	48.204963684082\\
5.08500003814697	66.6736679077148\\
5.09000015258789	58.4364852905273\\
5.09499979019165	41.591365814209\\
5.09999990463257	21.1038913726807\\
5.10500001907349	2.47162413597107\\
5.1100001335144	-4.86392164230347\\
5.11499977111816	-6.18615198135376\\
5.11999988555908	-6.18232917785645\\
5.125	-5.7881121635437\\
5.13000011444092	-5.19559383392334\\
5.13500022888184	-36.9086570739746\\
5.1399998664856	-199.449829101563\\
5.14499998092651	-305.218078613281\\
5.15000009536743	-357.841491699219\\
5.15500020980835	-370.286315917969\\
5.15999984741211	-322.130157470703\\
5.16499996185303	-164.499328613281\\
5.17000007629395	0.643980026245117\\
5.17500019073486	143.04801940918\\
5.17999982833862	242.748153686523\\
5.18499994277954	285.706756591797\\
5.19000005722046	270.343292236328\\
5.19500017166138	224.347412109375\\
5.19999980926514	163.938613891602\\
5.20499992370605	93.3099212646484\\
5.21000003814697	24.3782958984375\\
5.21500015258789	-32.6759490966797\\
5.21999979019165	-69.5917739868164\\
5.22499990463257	-83.5669097900391\\
5.23000001907349	-74.8542785644531\\
5.2350001335144	-51.7318496704102\\
5.23999977111816	-19.8731536865234\\
5.24499988555908	15.2372169494629\\
5.25	47.9210052490234\\
5.25500011444092	72.5325317382813\\
5.26000022888184	88.4006500244141\\
5.2649998664856	95.9053573608398\\
5.26999998092651	92.0099334716797\\
5.27500009536743	75.6470718383789\\
5.28000020980835	50.0732917785645\\
5.28499984741211	18.8309173583984\\
5.28999996185303	-15.1249189376831\\
5.29500007629395	-48.6150665283203\\
5.30000019073486	-78.0637741088867\\
5.30499982833862	-101.629867553711\\
5.30999994277954	-115.900505065918\\
5.31500005722046	-118.711067199707\\
5.32000017166138	-110.18204498291\\
5.32499980926514	-89.7167434692383\\
5.32999992370605	-64.370475769043\\
5.33500003814697	-40.1799850463867\\
5.34000015258789	-19.9664459228516\\
5.34499979019165	-5.57290077209473\\
5.34999990463257	2.35449194908142\\
5.35500001907349	0.982197463512421\\
5.3600001335144	-9.33700275421143\\
5.36499977111816	-26.8877010345459\\
5.36999988555908	-47.918212890625\\
5.375	-67.5278396606445\\
5.38000011444092	-81.607048034668\\
5.38500022888184	-86.7593536376953\\
5.3899998664856	-81.5879287719727\\
5.39499998092651	-66.6014633178711\\
5.40000009536743	-44.1911506652832\\
5.40500020980835	-22.4611892700195\\
5.40999984741211	-4.36173391342163\\
5.41499996185303	7.75707483291626\\
5.42000007629395	11.0792427062988\\
5.42500019073486	5.00094890594482\\
5.42999982833862	-6.63594818115234\\
5.43499994277954	-13.9756956100464\\
5.44000005722046	-9.8026237487793\\
5.44500017166138	-4.4943904876709\\
5.44999980926514	-2.85027122497559\\
5.45499992370605	-3.01476240158081\\
5.46000003814697	-3.11816954612732\\
5.46500015258789	-2.97842931747437\\
5.46999979019165	-2.70491790771484\\
5.47499990463257	-2.39645147323608\\
5.48000001907349	-2.11608695983887\\
5.4850001335144	-1.8462907075882\\
5.48999977111816	-1.59816551208496\\
5.49499988555908	-1.38228118419647\\
5.5	-1.19595456123352\\
5.50500011444092	-1.04115676879883\\
5.51000022888184	-0.895747065544128\\
5.5149998664856	-0.775014400482178\\
5.51999998092651	-3.96374797821045\\
5.52500009536743	-19.323148727417\\
5.53000020980835	-40.4588317871094\\
5.53499984741211	-54.2195510864258\\
5.53999996185303	-63.6125869750977\\
5.54500007629395	-69.8382186889648\\
5.55000019073486	-73.8537216186523\\
5.55499982833862	-75.4984893798828\\
5.55999994277954	-75.5831832885742\\
5.56500005722046	-74.6178512573242\\
5.57000017166138	-72.4187469482422\\
5.57499980926514	-68.7286987304688\\
5.57999992370605	-63.455436706543\\
5.58500003814697	-57.030387878418\\
5.59000015258789	-49.8038024902344\\
5.59499979019165	-42.0511512756348\\
5.59999990463257	-34.2406997680664\\
5.60500001907349	-26.8306484222412\\
5.6100001335144	-20.1328773498535\\
5.61499977111816	-14.4018430709839\\
5.61999988555908	-9.79543972015381\\
5.625	-6.33997297286987\\
5.63000011444092	-3.87775373458862\\
5.63500022888184	-2.30265617370605\\
5.6399998664856	-1.41186439990997\\
5.64499998092651	-1.18893539905548\\
5.65000009536743	-1.25572085380554\\
5.65500020980835	-1.31333708763123\\
5.65999984741211	-1.30973839759827\\
5.66499996185303	-1.25742053985596\\
5.67000007629395	0.344172090291977\\
5.67500019073486	34.1429595947266\\
5.67999982833862	125.288497924805\\
5.68499994277954	165.146881103516\\
5.69000005722046	160.434494018555\\
5.69500017166138	105.32292175293\\
5.69999980926514	54.5554618835449\\
5.70499992370605	9.91613292694092\\
5.71000003814697	-22.3692646026611\\
5.71500015258789	-35.9968109130859\\
5.71999979019165	-29.0886554718018\\
5.72499990463257	-6.73745918273926\\
5.73000001907349	25.1553039550781\\
5.7350001335144	59.954704284668\\
5.73999977111816	90.248291015625\\
5.74499988555908	112.431655883789\\
5.75	124.732498168945\\
5.75500011444092	128.78205871582\\
5.76000022888184	127.775619506836\\
5.7649998664856	125.121032714844\\
5.76999998092651	124.995132446289\\
5.77500009536743	129.021911621094\\
5.78000020980835	137.090347290039\\
5.78499984741211	147.294204711914\\
5.78999996185303	159.51904296875\\
5.79500007629395	172.393829345703\\
5.80000019073486	185.444473266602\\
5.80499982833862	197.748718261719\\
5.80999994277954	210.006988525391\\
5.81500005722046	221.915588378906\\
5.82000017166138	234.079528808594\\
5.82499980926514	246.729461669922\\
5.82999992370605	260.675079345703\\
5.83500003814697	275.630279541016\\
5.84000015258789	291.011077880859\\
5.84499979019165	307.218597412109\\
5.84999990463257	323.1787109375\\
5.85500001907349	337.985229492188\\
5.8600001335144	352.122589111328\\
5.86499977111816	359.506195068359\\
5.86999988555908	354.971282958984\\
5.875	332.355865478516\\
5.88000011444092	285.061676025391\\
5.88500022888184	209.099746704102\\
5.8899998664856	108.454193115234\\
5.89499998092651	17.6206474304199\\
5.90000009536743	-16.2423477172852\\
5.90500020980835	-41.5832901000977\\
5.90999984741211	-86.8193359375\\
5.91499996185303	-169.722366333008\\
5.92000007629395	-284.213745117188\\
5.92500019073486	-411.928894042969\\
5.92999982833862	-536.911010742188\\
5.93499994277954	-649.217834472656\\
5.94000005722046	-745.142639160156\\
5.94500017166138	-823.641174316406\\
5.94999980926514	-884.567626953125\\
5.95499992370605	-926.987854003906\\
5.96000003814697	-953.303649902344\\
5.96500015258789	-971.641784667969\\
5.96999979019165	-965.24462890625\\
5.97499990463257	-923.877136230469\\
5.98000001907349	-700.155578613281\\
5.9850001335144	-17.686637878418\\
5.98999977111816	630.925231933594\\
5.99499988555908	1148.26135253906\\
6	1436.86572265625\\
6.00500011444092	1481.43334960938\\
6.01000022888184	1304.87670898438\\
6.0149998664856	999.191101074219\\
6.01999998092651	592.365234375\\
6.02500009536743	139.517288208008\\
6.03000020980835	-287.121917724609\\
6.03499984741211	-627.730407714844\\
6.03999996185303	-837.651672363281\\
6.04500007629395	-894.718688964844\\
6.05000019073486	-799.89013671875\\
6.05499982833862	-586.608520507813\\
6.05999994277954	-313.979614257813\\
6.06500005722046	-35.4845314025879\\
6.07000017166138	204.676223754883\\
6.07499980926514	377.110717773438\\
6.07999992370605	467.091278076172\\
6.08500003814697	484.762817382813\\
6.09000015258789	443.678466796875\\
6.09499979019165	333.555847167969\\
6.09999990463257	184.512680053711\\
6.10500001907349	26.003246307373\\
6.1100001335144	-118.729133605957\\
6.11499977111816	-232.257934570313\\
6.11999988555908	-308.810668945313\\
6.125	-331.223815917969\\
6.13000011444092	-296.348236083984\\
6.13500022888184	-207.85920715332\\
6.1399998664856	-97.9671859741211\\
6.14499998092651	11.5057191848755\\
6.15000009536743	104.68384552002\\
6.15500020980835	169.838455200195\\
6.15999984741211	207.24494934082\\
6.16499996185303	213.079483032227\\
6.17000007629395	186.277328491211\\
6.17500019073486	133.293563842773\\
6.17999982833862	62.4025573730469\\
6.18499994277954	-12.5198583602905\\
6.19000005722046	-70.1089248657227\\
6.19500017166138	-81.4257049560547\\
6.19999980926514	-71.5938720703125\\
6.20499992370605	-53.9568824768066\\
6.21000003814697	-32.895263671875\\
6.21500015258789	-13.8254652023315\\
6.21999979019165	-7.9845142364502\\
6.22499990463257	-7.84895658493042\\
6.23000001907349	-7.88705492019653\\
6.2350001335144	-7.39362096786499\\
6.23999977111816	-6.70616340637207\\
6.24499988555908	-5.87679576873779\\
6.25	-5.09315156936646\\
6.25500011444092	-4.38177824020386\\
6.26000022888184	-3.74720478057861\\
6.2649998664856	-3.19110751152039\\
6.26999998092651	-2.70913863182068\\
6.27500009536743	-2.29627394676208\\
6.28000020980835	-1.94293820858002\\
6.28499984741211	-1.64005589485168\\
6.28999996185303	-1.38088881969452\\
6.29500007629395	-1.1598094701767\\
6.30000019073486	-0.970695436000824\\
6.30499982833862	-0.809978306293488\\
6.30999994277954	-0.672921121120453\\
6.31500005722046	-0.556369006633759\\
6.32000017166138	-0.458145022392273\\
6.32499980926514	-0.374560594558716\\
6.32999992370605	-0.304029673337936\\
6.33500003814697	-0.244223326444626\\
6.34000015258789	-0.19348606467247\\
6.34499979019165	-0.151644557714462\\
6.34999990463257	-0.11788210272789\\
6.35500001907349	-0.0905931070446968\\
6.3600001335144	20.1320972442627\\
6.36499977111816	93.6277236938477\\
6.36999988555908	144.434127807617\\
6.375	175.662750244141\\
6.38000011444092	191.528900146484\\
6.38500022888184	189.883590698242\\
6.3899998664856	155.791397094727\\
6.39499998092651	43.2196807861328\\
6.40000009536743	-45.1984748840332\\
6.40500020980835	-108.515838623047\\
6.40999984741211	-146.249328613281\\
6.41499996185303	-153.634902954102\\
6.42000007629395	-138.886947631836\\
6.42500019073486	-108.830360412598\\
6.42999982833862	-68.5027084350586\\
6.43499994277954	-25.4388847351074\\
6.44000005722046	14.3958444595337\\
6.44500017166138	46.6056480407715\\
6.44999980926514	68.8037643432617\\
6.45499992370605	79.1459732055664\\
6.46000003814697	79.014533996582\\
6.46500015258789	70.2518005371094\\
6.46999979019165	55.420093536377\\
6.47499990463257	37.3322257995605\\
6.48000001907349	19.0443820953369\\
6.4850001335144	2.61967635154724\\
6.48999977111816	-10.174072265625\\
6.49499988555908	-18.1123027801514\\
6.5	-20.8877696990967\\
6.50500011444092	-19.2518653869629\\
6.51000022888184	-14.3012437820435\\
6.5149998664856	-6.86780071258545\\
6.51999998092651	1.873654961586\\
6.52500009536743	10.4129076004028\\
6.53000020980835	18.0969486236572\\
6.53499984741211	24.161693572998\\
6.53999996185303	28.2648296356201\\
6.54500007629395	30.3855762481689\\
6.55000019073486	30.5729999542236\\
6.55499982833862	29.5502815246582\\
6.55999994277954	27.5206432342529\\
6.56500005722046	25.3679218292236\\
6.57000017166138	22.8398952484131\\
6.57499980926514	20.2526912689209\\
6.57999992370605	18.0846557617188\\
6.58500003814697	16.9260997772217\\
6.59000015258789	16.6286945343018\\
6.59499979019165	17.3406143188477\\
6.59999990463257	19.020299911499\\
6.60500001907349	21.4094429016113\\
6.6100001335144	24.233642578125\\
6.61499977111816	27.3125343322754\\
6.61999988555908	30.33957862854\\
6.625	33.0039672851563\\
6.63000011444092	35.1654205322266\\
6.63500022888184	36.6615257263184\\
6.6399998664856	37.5631904602051\\
6.64499998092651	37.9435615539551\\
6.65000009536743	37.8531379699707\\
6.65500020980835	37.2990188598633\\
6.65999984741211	36.643138885498\\
6.66499996185303	35.9057273864746\\
6.67000007629395	35.2850761413574\\
6.67500019073486	34.9598197937012\\
6.67999982833862	34.8793067932129\\
6.68499994277954	34.8727188110352\\
6.69000005722046	34.6620178222656\\
6.69500017166138	34.3044242858887\\
6.69999980926514	33.7001113891602\\
6.70499992370605	33.037181854248\\
6.71000003814697	31.9146137237549\\
6.71500015258789	30.1283550262451\\
6.71999979019165	27.4734210968018\\
6.72499990463257	24.0785961151123\\
6.73000001907349	19.955394744873\\
6.7350001335144	15.1025371551514\\
6.73999977111816	9.79054546356201\\
6.74499988555908	4.39606523513794\\
6.75	-0.936014771461487\\
6.75500011444092	-5.99951314926147\\
6.76000022888184	-11.1712865829468\\
6.7649998664856	-16.3440742492676\\
6.76999998092651	-20.6945686340332\\
6.77500009536743	-24.2693042755127\\
6.78000020980835	-27.4754428863525\\
6.78499984741211	-29.9120922088623\\
6.78999996185303	-31.8191146850586\\
6.79500007629395	-33.0972175598145\\
6.80000019073486	-34.0209808349609\\
6.80499982833862	-34.7515563964844\\
6.80999994277954	-35.2361717224121\\
6.81500005722046	-35.6754989624023\\
6.82000017166138	-36.1449813842773\\
6.82499980926514	-36.598072052002\\
6.82999992370605	-37.1159591674805\\
6.83500003814697	-37.7981643676758\\
6.84000015258789	-38.5980033874512\\
6.84499979019165	-39.074348449707\\
6.84999990463257	-39.2676658630371\\
6.85500001907349	-39.0298385620117\\
6.8600001335144	-38.305248260498\\
6.86499977111816	-36.9434089660645\\
6.86999988555908	-34.7675247192383\\
6.875	-32.2123870849609\\
6.88000011444092	-29.5246067047119\\
6.88500022888184	-26.6404552459717\\
6.8899998664856	-23.8449287414551\\
6.89499998092651	-21.3512268066406\\
6.90000009536743	-19.4216499328613\\
6.90500020980835	-17.8140602111816\\
6.90999984741211	-16.8946571350098\\
6.91499996185303	-16.6013126373291\\
6.92000007629395	-16.7303276062012\\
6.92500019073486	-17.1770668029785\\
6.92999982833862	-17.7953815460205\\
6.93499994277954	-18.4296169281006\\
6.94000005722046	-18.900297164917\\
6.94500017166138	-19.1300582885742\\
6.94999980926514	-19.0317077636719\\
6.95499992370605	-18.4293880462646\\
6.96000003814697	-17.2700824737549\\
6.96500015258789	-16.0068054199219\\
6.96999979019165	-14.7085380554199\\
6.97499990463257	-13.2403345108032\\
6.98000001907349	-11.9124670028687\\
6.9850001335144	-10.8158273696899\\
6.98999977111816	-10.0730485916138\\
6.99499988555908	-9.56701278686523\\
7	-9.35587787628174\\
7.00500011444092	-9.51299858093262\\
7.01000022888184	-9.82858753204346\\
7.0149998664856	-10.2102003097534\\
7.01999998092651	-10.5436038970947\\
7.02500009536743	-10.7695732116699\\
7.03000020980835	-10.8594388961792\\
7.03499984741211	-10.7628269195557\\
7.03999996185303	-10.4550819396973\\
7.04500007629395	-9.94663333892822\\
7.05000019073486	-9.27346611022949\\
7.05499982833862	-8.49882316589355\\
7.05999994277954	-7.68123769760132\\
7.06500005722046	-6.86452198028564\\
7.07000017166138	-6.12384843826294\\
7.07499980926514	-5.57603311538696\\
7.07999992370605	-5.19942378997803\\
7.08500003814697	-4.99870109558105\\
7.09000015258789	-4.97490978240967\\
7.09499979019165	-5.10242795944214\\
7.09999990463257	-5.32981252670288\\
7.10500001907349	-5.58060169219971\\
7.1100001335144	-5.78115224838257\\
7.11499977111816	-5.91377878189087\\
7.11999988555908	-5.93307447433472\\
7.125	-5.79491233825684\\
7.13000011444092	-5.50206232070923\\
7.13500022888184	-5.10594606399536\\
7.1399998664856	-4.62352609634399\\
7.14499998092651	-4.10351848602295\\
7.15000009536743	-3.59473633766174\\
7.15500020980835	-3.16076350212097\\
7.15999984741211	-2.89034676551819\\
7.16499996185303	-2.91741943359375\\
7.17000007629395	-3.1442928314209\\
7.17500019073486	-3.31997513771057\\
7.17999982833862	-3.50445222854614\\
7.18499994277954	-3.69674324989319\\
7.19000005722046	-3.84347701072693\\
7.19500017166138	-3.92321181297302\\
7.19999980926514	-3.92675185203552\\
7.20499992370605	-3.85755705833435\\
7.21000003814697	-3.72431945800781\\
7.21500015258789	-3.56967163085938\\
7.21999979019165	-3.40724349021912\\
7.22499990463257	-3.23800563812256\\
7.23000001907349	-3.1062593460083\\
7.2350001335144	-3.00901007652283\\
7.23999977111816	-2.94734072685242\\
7.24499988555908	-2.92736148834229\\
7.25	-2.9378399848938\\
7.25500011444092	-2.97213912010193\\
7.26000022888184	-3.03098750114441\\
7.2649998664856	-3.09533739089966\\
7.26999998092651	-3.15758466720581\\
7.27500009536743	-3.21255397796631\\
7.28000020980835	-3.24737310409546\\
7.28499984741211	-3.25777912139893\\
7.28999996185303	-3.24861693382263\\
7.29500007629395	-3.22336935997009\\
7.30000019073486	-3.18622398376465\\
7.30499982833862	-3.12172675132751\\
7.30999994277954	-3.04399180412292\\
7.31500005722046	-2.97094535827637\\
7.32000017166138	-2.90385985374451\\
7.32499980926514	-2.84190845489502\\
7.32999992370605	-2.78657507896423\\
7.33500003814697	-2.73493361473083\\
7.34000015258789	-2.68807029724121\\
7.34499979019165	-2.65160417556763\\
7.34999990463257	-2.62285351753235\\
7.35500001907349	-2.59677028656006\\
7.3600001335144	-2.5733904838562\\
7.36499977111816	-2.55978918075562\\
7.36999988555908	-2.55395483970642\\
7.375	-2.54192996025085\\
7.38000011444092	-2.50895571708679\\
7.38500022888184	-2.45857548713684\\
7.3899998664856	-2.39171957969666\\
7.39499998092651	-2.30254483222961\\
7.40000009536743	-2.19412708282471\\
7.40500020980835	-2.09139895439148\\
7.40999984741211	-1.99337828159332\\
7.41499996185303	-1.90033102035522\\
7.42000007629395	-1.81603872776031\\
7.42500019073486	-1.74347770214081\\
7.42999982833862	-1.68571329116821\\
7.43499994277954	-1.64094424247742\\
7.44000005722046	-1.60125374794006\\
7.44500017166138	-1.56668901443481\\
7.44999980926514	-1.53674149513245\\
7.45499992370605	-1.50579726696014\\
7.46000003814697	-1.46025788784027\\
7.46500015258789	-1.38967049121857\\
7.46999979019165	-1.30307948589325\\
7.47499990463257	-1.22033500671387\\
7.48000001907349	-1.14259314537048\\
7.4850001335144	-1.06168127059937\\
7.48999977111816	-0.981948912143707\\
7.49499988555908	-0.906057834625244\\
7.5	-0.835028350353241\\
7.50500011444092	-0.769822001457214\\
7.51000022888184	-0.710823953151703\\
7.5149998664856	-0.658504843711853\\
7.51999998092651	-0.618326902389526\\
7.52500009536743	-0.616928160190582\\
7.53000020980835	-0.643058836460114\\
7.53499984741211	-0.616935729980469\\
7.53999996185303	-0.595307469367981\\
7.54500007629395	-0.565143406391144\\
7.55000019073486	-0.513934731483459\\
7.55499982833862	-0.455496042966843\\
7.55999994277954	-0.394343167543411\\
7.56500005722046	-0.329679280519485\\
7.57000017166138	-0.269028097391129\\
7.57499980926514	-0.220000311732292\\
7.57999992370605	-0.185144275426865\\
7.58500003814697	-0.165394246578217\\
7.59000015258789	-0.16165728867054\\
7.59499979019165	-0.169789955019951\\
7.59999990463257	-0.190236523747444\\
7.60500001907349	-0.223579168319702\\
7.6100001335144	-0.264841675758362\\
7.61499977111816	-0.305845439434052\\
7.61999988555908	-0.343132168054581\\
7.625	-0.362674534320831\\
7.63000011444092	-0.35086053609848\\
7.63500022888184	-0.32751539349556\\
7.6399998664856	-0.309412896633148\\
7.64499998092651	-0.288961380720139\\
7.65000009536743	-0.269949793815613\\
7.65500020980835	-0.263646870851517\\
7.65999984741211	-0.270612835884094\\
7.66499996185303	-0.287367522716522\\
7.67000007629395	-0.317054569721222\\
7.67500019073486	-0.361158519983292\\
7.67999982833862	-0.413708209991455\\
7.68499994277954	-0.465220361948013\\
7.69000005722046	-0.508676946163177\\
7.69500017166138	-0.545455634593964\\
7.69999980926514	-0.585600733757019\\
7.70499992370605	-0.630435347557068\\
7.71000003814697	-0.676764965057373\\
7.71500015258789	-0.711093962192535\\
7.71999979019165	-0.723874747753143\\
7.72499990463257	-0.732233643531799\\
7.73000001907349	-0.75161874294281\\
7.7350001335144	-0.785595715045929\\
7.73999977111816	-0.822432518005371\\
7.74499988555908	-0.842601478099823\\
7.75	-0.867485046386719\\
7.75500011444092	-0.910699129104614\\
7.76000022888184	-0.955556869506836\\
7.7649998664856	-0.995208978652954\\
7.76999998092651	-1.03746175765991\\
7.77500009536743	-1.08399069309235\\
7.78000020980835	-1.13388001918793\\
7.78499984741211	-1.18214046955109\\
7.78999996185303	-1.22088062763214\\
7.79500007629395	-1.25386607646942\\
7.80000019073486	-1.28529477119446\\
7.80499982833862	-1.31732177734375\\
7.80999994277954	-1.34910571575165\\
7.81500005722046	-1.38108611106873\\
7.82000017166138	-1.41427338123322\\
7.82499980926514	-1.44877123832703\\
7.82999992370605	-1.4838011264801\\
7.83500003814697	-1.51818263530731\\
7.84000015258789	-1.55210065841675\\
7.84499979019165	-1.58587825298309\\
7.84999990463257	-1.61950087547302\\
7.85500001907349	-1.6556271314621\\
7.8600001335144	-1.69455754756927\\
7.86499977111816	-1.73487734794617\\
7.86999988555908	-1.77439963817596\\
7.875	-1.81317698955536\\
7.88000011444092	-1.8505083322525\\
7.88500022888184	-1.88768088817596\\
7.8899998664856	-1.92534065246582\\
7.89499998092651	-1.96349012851715\\
7.90000009536743	-2.00159931182861\\
7.90500020980835	-2.03869366645813\\
7.90999984741211	-2.0749340057373\\
7.91499996185303	-2.11119484901428\\
7.92000007629395	-2.14818978309631\\
7.92500019073486	-2.18562531471252\\
7.92999982833862	-2.22112560272217\\
7.93499994277954	-2.25861811637878\\
7.94000005722046	-2.29808378219604\\
7.94500017166138	-2.33574938774109\\
7.94999980926514	-2.37678790092468\\
7.95499992370605	-2.42229270935059\\
7.96000003814697	-2.4700870513916\\
7.96500015258789	-2.52593660354614\\
7.96999979019165	-2.59445953369141\\
7.97499990463257	-2.66955590248108\\
7.98000001907349	-2.73928332328796\\
7.9850001335144	-2.79934692382813\\
7.98999977111816	-2.85386776924133\\
7.99499988555908	-2.90676546096802\\
8	-2.96664905548096\\
8.00500011444092	-3.0306658744812\\
8.01000022888184	-3.09686517715454\\
8.01500034332275	-3.17112565040588\\
8.02000045776367	-3.2548520565033\\
8.02499961853027	-3.34659099578857\\
8.02999973297119	-3.43977022171021\\
8.03499984741211	-3.53035831451416\\
8.03999996185303	-3.62067270278931\\
8.04500007629395	-3.7090437412262\\
8.05000019073486	-3.79734230041504\\
8.05500030517578	-3.88419318199158\\
8.0600004196167	-3.9701783657074\\
8.0649995803833	-4.05579328536987\\
8.06999969482422	-4.1408896446228\\
8.07499980926514	-4.22849225997925\\
8.07999992370605	-4.320725440979\\
8.08500003814697	-4.44606494903564\\
8.09000015258789	-4.5718822479248\\
8.09500026702881	-4.6974949836731\\
8.10000038146973	-4.81984233856201\\
8.10499954223633	-4.94000434875488\\
8.10999965667725	-5.06450939178467\\
8.11499977111816	-5.1982798576355\\
8.11999988555908	-5.33991146087646\\
8.125	-5.48592090606689\\
8.13000011444092	-5.6312575340271\\
8.13500022888184	-5.77649641036987\\
8.14000034332275	-5.92478799819946\\
8.14500045776367	-6.0848536491394\\
8.14999961853027	-6.25332117080688\\
8.15499973297119	-6.42827224731445\\
8.15999984741211	-6.60490655899048\\
8.16499996185303	-6.77311325073242\\
8.17000007629395	-6.94367551803589\\
8.17500019073486	-7.13560247421265\\
8.18000030517578	-7.36332416534424\\
8.1850004196167	-7.62320470809937\\
8.1899995803833	-7.88892459869385\\
8.19499969482422	-8.12735271453857\\
8.19999980926514	-8.34187126159668\\
8.20499992370605	-8.5501127243042\\
8.21000003814697	-8.75927448272705\\
8.21500015258789	-8.97189807891846\\
8.22000026702881	-9.17650699615479\\
8.22500038146973	-9.36295318603516\\
8.22999954223633	-9.56245040893555\\
8.23499965667725	-9.88542461395264\\
8.23999977111816	-10.3495683670044\\
8.24499988555908	-10.7857761383057\\
8.25	-11.0875415802002\\
8.25500011444092	-11.3887310028076\\
8.26000022888184	-11.7040491104126\\
8.26500034332275	-12.0254354476929\\
8.27000045776367	-12.3430976867676\\
8.27499961853027	-12.762375831604\\
8.27999973297119	-13.1610841751099\\
8.28499984741211	-13.5419654846191\\
8.28999996185303	-13.9251070022583\\
8.29500007629395	-14.3706884384155\\
8.30000019073486	-14.8255348205566\\
8.30500030517578	-15.2231168746948\\
8.3100004196167	-15.5032300949097\\
8.3149995803833	-16.0089817047119\\
8.31999969482422	-16.9665241241455\\
8.32499980926514	-17.5486927032471\\
8.32999992370605	-18.0897998809814\\
8.33500003814697	-18.6464767456055\\
8.34000015258789	-19.258716583252\\
8.34500026702881	-19.8668994903564\\
8.35000038146973	-20.4549808502197\\
8.35499954223633	-21.0057125091553\\
8.35999965667725	-21.5489521026611\\
8.36499977111816	-22.3783912658691\\
8.36999988555908	-23.271879196167\\
8.375	-24.0084991455078\\
8.38000011444092	-24.7732620239258\\
8.38500022888184	-25.6061782836914\\
8.39000034332275	-26.3948955535889\\
8.39500045776367	-27.1043758392334\\
8.39999961853027	-27.8341197967529\\
8.40499973297119	-29.0179920196533\\
8.40999984741211	-29.9460353851318\\
8.41499996185303	-30.8874320983887\\
8.42000007629395	-31.8590087890625\\
8.42500019073486	-32.8079566955566\\
8.43000030517578	-33.6857604980469\\
8.4350004196167	-34.4130172729492\\
8.4399995803833	-35.1171188354492\\
8.44499969482422	-35.6854476928711\\
8.44999980926514	-35.7873687744141\\
8.45499992370605	-35.5351600646973\\
8.46000003814697	-34.8279228210449\\
8.46500015258789	-33.5780754089355\\
8.47000026702881	-32.0682678222656\\
8.47500038146973	-30.2256813049316\\
8.47999954223633	-28.1619625091553\\
8.48499965667725	-25.8418197631836\\
8.48999977111816	-23.0584850311279\\
8.49499988555908	-19.9712886810303\\
8.5	-16.8781490325928\\
8.50500011444092	-13.6026544570923\\
8.51000022888184	-10.2310838699341\\
8.51500034332275	-6.71721839904785\\
8.52000045776367	-3.54654741287231\\
8.52499961853027	-0.0624620094895363\\
8.52999973297119	4.21547222137451\\
8.53499984741211	8.44335842132568\\
8.53999996185303	12.3650140762329\\
8.54500007629395	16.5036487579346\\
8.55000019073486	20.7477684020996\\
8.55500030517578	24.910572052002\\
8.5600004196167	29.0332336425781\\
8.5649995803833	33.0352516174316\\
8.56999969482422	36.6917839050293\\
8.57499980926514	39.9775428771973\\
8.57999992370605	42.7643852233887\\
8.58500003814697	45.2307052612305\\
8.59000015258789	47.3430328369141\\
8.59500026702881	48.6468887329102\\
8.60000038146973	48.8152160644531\\
8.60499954223633	47.729866027832\\
8.60999965667725	45.4689979553223\\
8.61499977111816	42.1210784912109\\
8.61999988555908	37.9868583679199\\
8.625	33.2318954467773\\
8.63000011444092	28.975564956665\\
8.63500022888184	25.053581237793\\
8.64000034332275	21.755729675293\\
8.64500045776367	19.4604377746582\\
8.64999961853027	18.0704097747803\\
8.65499973297119	17.8947677612305\\
8.65999984741211	18.7363090515137\\
8.66499996185303	20.6591949462891\\
8.67000007629395	22.5828018188477\\
8.67500019073486	24.4653549194336\\
8.68000030517578	26.0053977966309\\
8.6850004196167	27.1122932434082\\
8.6899995803833	27.4647827148438\\
8.69499969482422	27.0251178741455\\
8.69999980926514	25.6947593688965\\
8.70499992370605	23.8001441955566\\
8.71000003814697	21.4871883392334\\
8.71500015258789	18.8888072967529\\
8.72000026702881	16.2771186828613\\
8.72500038146973	14.100172996521\\
8.72999954223633	12.4423332214355\\
8.73499965667725	11.4067239761353\\
8.73999977111816	11.134238243103\\
8.74499988555908	11.6226081848145\\
8.75	12.658034324646\\
8.75500011444092	13.9115715026855\\
8.76000022888184	15.0563306808472\\
8.76500034332275	16.205509185791\\
8.77000045776367	16.9977073669434\\
8.77499961853027	17.3669528961182\\
8.77999973297119	17.2349071502686\\
8.78499984741211	16.6394710540771\\
8.78999996185303	15.6748828887939\\
8.79500007629395	14.4338026046753\\
8.80000019073486	13.079195022583\\
8.80500030517578	11.772008895874\\
8.8100004196167	10.6570911407471\\
8.8149995803833	9.86140537261963\\
8.81999969482422	9.4607572555542\\
8.82499980926514	9.46516227722168\\
8.82999992370605	9.84458637237549\\
8.83500003814697	10.5044345855713\\
8.84000015258789	11.3029623031616\\
8.84500026702881	12.1042985916138\\
8.85000038146973	12.7676525115967\\
8.85499954223633	13.1735887527466\\
8.85999965667725	13.2859592437744\\
8.86499977111816	13.0729551315308\\
8.86999988555908	12.5656461715698\\
8.875	11.8378534317017\\
8.88000011444092	10.9965744018555\\
8.88500022888184	10.1615524291992\\
8.89000034332275	9.44242763519287\\
8.89500045776367	8.93515300750732\\
8.89999961853027	8.62128734588623\\
8.90499973297119	8.63485813140869\\
8.90999984741211	8.92214965820313\\
8.91499996185303	9.34568881988525\\
8.92000007629395	9.69242286682129\\
8.92500019073486	10.0550184249878\\
8.93000030517578	10.3480606079102\\
8.9350004196167	10.4337158203125\\
8.9399995803833	10.2600040435791\\
8.94499969482422	9.85817337036133\\
8.94999980926514	9.32344818115234\\
8.95499992370605	8.7154016494751\\
8.96000003814697	8.05844974517822\\
8.96500015258789	7.41900634765625\\
8.97000026702881	6.86564207077026\\
8.97500038146973	6.42429971694946\\
8.97999954223633	6.06194257736206\\
8.98499965667725	5.82749462127686\\
8.98999977111816	5.72662448883057\\
8.99499988555908	5.5788836479187\\
9	5.40288877487183\\
9.00500011444092	5.28422927856445\\
9.01000022888184	5.15682554244995\\
9.01500034332275	4.98157787322998\\
9.02000045776367	4.75627326965332\\
9.02499961853027	4.49885988235474\\
9.02999973297119	4.20811128616333\\
9.03499984741211	3.88963675498962\\
9.03999996185303	3.55840754508972\\
9.04500007629395	3.208660364151\\
9.05000019073486	2.90634822845459\\
9.05500030517578	2.61820864677429\\
9.0600004196167	2.35025978088379\\
9.0649995803833	2.10796689987183\\
9.06999969482422	1.89434790611267\\
9.07499980926514	1.70404207706451\\
9.07999992370605	1.53905701637268\\
9.08500003814697	1.40127420425415\\
9.09000015258789	1.28690588474274\\
9.09500026702881	1.17080211639404\\
9.10000038146973	1.0188399553299\\
9.10499954223633	0.856928467750549\\
9.10999965667725	0.722578942775726\\
9.11499977111816	0.583847939968109\\
9.11999988555908	0.392766445875168\\
9.125	0.142931491136551\\
9.13000011444092	-0.124431967735291\\
9.13500022888184	-0.368234634399414\\
9.14000034332275	-0.565297961235046\\
9.14500045776367	-0.695910394191742\\
9.14999961853027	-0.730715870857239\\
9.15499973297119	-0.668537437915802\\
9.15999984741211	-0.538169085979462\\
9.16499996185303	-0.312374949455261\\
9.17000007629395	-0.0210293233394623\\
9.17500019073486	0.241945624351501\\
9.18000030517578	0.495502710342407\\
9.1850004196167	0.706203281879425\\
9.1899995803833	0.853244781494141\\
9.19499969482422	0.93098521232605\\
9.19999980926514	0.9357590675354\\
9.20499992370605	0.874234199523926\\
9.21000003814697	0.762650430202484\\
9.21500015258789	0.626652836799622\\
9.22000026702881	0.524099230766296\\
9.22500038146973	0.464558780193329\\
9.22999954223633	0.453775405883789\\
9.23499965667725	0.477185696363449\\
9.23999977111816	0.425318747758865\\
9.24499988555908	0.00241884705610573\\
9.25	-1.19167959690094\\
9.25500011444092	-3.16176986694336\\
9.26000022888184	-5.79198884963989\\
9.26500034332275	-8.64524364471436\\
9.27000045776367	-11.7273178100586\\
9.27499961853027	-14.6664638519287\\
9.27999973297119	-17.4648838043213\\
9.28499984741211	-20.2240505218506\\
9.28999996185303	-22.883020401001\\
9.29500007629395	-25.511547088623\\
9.30000019073486	-28.08762550354\\
9.30500030517578	-30.688928604126\\
9.3100004196167	-34.1879615783691\\
9.3149995803833	-37.4212532043457\\
9.31999969482422	-40.4924736022949\\
9.32499980926514	-43.7407493591309\\
9.32999992370605	-46.7960166931152\\
9.33500003814697	-50.0652198791504\\
9.34000015258789	-52.5700378417969\\
9.34500026702881	-54.495548248291\\
9.35000038146973	-56.9844093322754\\
9.35499954223633	-57.3464813232422\\
9.35999965667725	-56.4401473999023\\
9.36499977111816	-56.1717414855957\\
9.36999988555908	-55.5027732849121\\
9.375	-55.5537567138672\\
9.38000011444092	-57.729419708252\\
9.38500022888184	-62.8052558898926\\
9.39000034332275	-71.5855255126953\\
9.39500045776367	-85.2589569091797\\
9.39999961853027	-103.502777099609\\
9.40499973297119	-125.953414916992\\
9.40999984741211	-152.266036987305\\
9.41499996185303	-181.878448486328\\
9.42000007629395	-213.543426513672\\
9.42500019073486	-246.00520324707\\
9.43000030517578	-279.137603759766\\
9.4350004196167	-314.813171386719\\
9.4399995803833	-345.939056396484\\
9.44499969482422	-366.507751464844\\
9.44999980926514	-369.265045166016\\
9.45499992370605	-346.573455810547\\
9.46000003814697	-291.713592529297\\
9.46500015258789	-197.773635864258\\
9.47000026702881	-64.4807434082031\\
9.47500038146973	103.902542114258\\
9.47999954223633	293.513671875\\
9.48499965667725	475.964630126953\\
9.48999977111816	575.585693359375\\
9.49499988555908	555.534912109375\\
9.5	532.8671875\\
9.50500011444092	516.142822265625\\
9.51000022888184	496.478332519531\\
9.51500034332275	470.670532226563\\
9.52000045776367	437.449829101563\\
9.52499961853027	395.834289550781\\
9.52999973297119	345.841369628906\\
9.53499984741211	286.685821533203\\
9.53999996185303	220.089691162109\\
9.54500007629395	148.223266601563\\
9.55000019073486	72.7309799194336\\
9.55500030517578	8.23678588867188\\
9.5600004196167	-10.0301637649536\\
9.5649995803833	-15.0317907333374\\
9.56999969482422	-16.6611766815186\\
9.57499980926514	-16.4460315704346\\
9.57999992370605	-15.3253345489502\\
9.58500003814697	-13.8206272125244\\
9.59000015258789	-12.342661857605\\
9.59500026702881	-10.8197765350342\\
9.60000038146973	-9.49080085754395\\
9.60499954223633	-8.20304870605469\\
9.60999965667725	-7.18081617355347\\
9.61499977111816	-6.194176197052\\
9.61999988555908	-5.35353422164917\\
9.625	-4.63597679138184\\
9.63000011444092	-4.01187515258789\\
9.63500022888184	-3.47424125671387\\
9.64000034332275	-3.01061058044434\\
9.64500045776367	-2.601975440979\\
9.64999961853027	-2.25131607055664\\
9.65499973297119	-1.95835304260254\\
9.65999984741211	-1.70133316516876\\
9.66499996185303	-1.47543835639954\\
9.67000007629395	-1.28331613540649\\
9.67500019073486	-1.11793160438538\\
9.68000030517578	-0.975006937980652\\
9.6850004196167	-0.852586030960083\\
9.6899995803833	-0.747479557991028\\
9.69499969482422	-0.656792104244232\\
9.69999980926514	-0.578152418136597\\
9.70499992370605	-0.509199619293213\\
9.71000003814697	-0.44877952337265\\
9.71500015258789	-0.397250652313232\\
9.72000026702881	-0.353056132793427\\
9.72500038146973	-0.314794480800629\\
9.72999954223633	-0.282559603452682\\
9.73499965667725	-0.255337089300156\\
9.73999977111816	-0.231087043881416\\
9.74499988555908	-0.207770958542824\\
9.75	-0.185483261942863\\
9.75500011444092	-0.168371751904488\\
9.76000022888184	-0.156799525022507\\
9.76500034332275	-0.149752199649811\\
9.77000045776367	-0.143700525164604\\
9.77499961853027	-0.136475205421448\\
9.77999973297119	-0.128779113292694\\
9.78499984741211	-0.122251443564892\\
9.78999996185303	-0.118103347718716\\
9.79500007629395	-0.114585973322392\\
9.80000019073486	-0.11111081391573\\
9.80500030517578	-0.107540413737297\\
9.8100004196167	-0.104473374783993\\
9.8149995803833	-0.102584674954414\\
9.81999969482422	-0.101813584566116\\
9.82499980926514	-0.100938223302364\\
9.82999992370605	-0.0994926616549492\\
9.83500003814697	-0.0975295826792717\\
9.84000015258789	-0.0960869714617729\\
9.84500026702881	-0.0963666364550591\\
9.85000038146973	-0.0986578315496445\\
9.85499954223633	-0.101262278854847\\
9.85999965667725	-0.101024933159351\\
9.86499977111816	-0.0987013280391693\\
9.86999988555908	-0.0954413712024689\\
9.875	-0.093592494726181\\
9.88000011444092	-0.0941574349999428\\
9.88500022888184	-0.0963858217000961\\
9.89000034332275	-0.0982861518859863\\
9.89500045776367	-0.0976400151848793\\
9.89999961853027	-0.095592588186264\\
9.90499973297119	-0.0935894846916199\\
9.90999984741211	-0.0940518751740456\\
9.91499996185303	-0.0974226146936417\\
9.92000007629395	-0.102425210177898\\
9.92500019073486	-0.105491645634174\\
9.93000030517578	-0.10382566601038\\
9.9350004196167	-0.0999227613210678\\
9.9399995803833	-0.0958285108208656\\
9.94499969482422	-0.0946148410439491\\
9.94999980926514	-0.0952732041478157\\
9.95499992370605	-0.0956657901406288\\
9.96000003814697	-0.0957915037870407\\
9.96500015258789	-0.0958780199289322\\
9.97000026702881	-0.0959880724549294\\
9.97500038146973	-0.0954968631267548\\
9.97999954223633	-0.0950661450624466\\
9.98499965667725	-0.0947733595967293\\
9.98999977111816	-0.0950092673301697\\
9.99499988555908	-0.0951542779803276\\
10	-0.095707006752491\\
};
\addlegendentry{Control}

\addplot [color=black, dashed, line width=2.0pt]
  table[row sep=crcr]{%
0.0949999988079071	-8.2012767791748\\
0.100000001490116	-7.2652153968811\\
0.104999996721745	-6.44816923141479\\
0.109999999403954	-5.73187160491943\\
0.115000002086163	-5.10071706771851\\
0.119999997317791	-167.77897644043\\
0.125	-373.096038818359\\
0.129999995231628	-520.255310058594\\
0.135000005364418	-633.923583984375\\
0.140000000596046	-715.330200195313\\
0.144999995827675	-767.9384765625\\
0.150000005960464	-792.739501953125\\
0.155000001192093	-793.192016601563\\
0.159999996423721	-781.780883789063\\
0.165000006556511	-752.462219238281\\
0.170000001788139	-701.7548828125\\
0.174999997019768	-632.294189453125\\
0.180000007152557	-547.203491210938\\
0.185000002384186	-451.015991210938\\
0.189999997615814	-348.254455566406\\
0.194999992847443	146.236831665039\\
0.200000002980232	520.5234375\\
0.204999998211861	736.927001953125\\
0.209999993443489	735.912292480469\\
0.215000003576279	565.79638671875\\
0.219999998807907	325.88134765625\\
0.224999994039536	-6.3717041015625\\
0.230000004172325	-381.056213378906\\
0.234999999403954	-744.279052734375\\
0.239999994635582	-1048.63256835938\\
0.245000004768372	-1255.26916503906\\
0.25	-1349.50268554688\\
0.254999995231628	-1297.33349609375\\
0.259999990463257	-1185.94616699219\\
0.264999985694885	-1024.32275390625\\
0.270000010728836	-836.27099609375\\
0.275000005960464	-654.955017089844\\
0.280000001192093	-507.97314453125\\
0.284999996423721	-413.594787597656\\
0.28999999165535	-378.543701171875\\
0.294999986886978	-400.844970703125\\
0.300000011920929	-465.835388183594\\
0.305000007152557	-559.898742675781\\
0.310000002384186	-664.798767089844\\
0.314999997615814	-762.628479003906\\
0.319999992847443	-838.01318359375\\
0.324999988079071	-880.373474121094\\
0.330000013113022	-894.760437011719\\
0.33500000834465	-873.411743164063\\
0.340000003576279	-809.012268066406\\
0.344999998807907	-723.527038574219\\
0.349999994039536	-630.237670898438\\
0.354999989271164	-541.608154296875\\
0.360000014305115	-467.481140136719\\
0.365000009536743	-411.361419677734\\
0.370000004768372	-373.628173828125\\
0.375	-359.863555908203\\
0.379999995231628	-366.716918945313\\
0.384999990463257	-388.458801269531\\
0.389999985694885	-416.040374755859\\
0.395000010728836	-442.371887207031\\
0.400000005960464	-458.840362548828\\
0.405000001192093	-459.710327148438\\
0.409999996423721	-443.017913818359\\
0.41499999165535	-409.241973876953\\
0.419999986886978	-360.926483154297\\
0.425000011920929	-302.290924072266\\
0.430000007152557	-244.117416381836\\
0.435000002384186	-192.216934204102\\
0.439999997615814	-148.332366943359\\
0.444999992847443	-117.864974975586\\
0.449999988079071	-102.027160644531\\
0.455000013113022	-99.5687255859375\\
0.46000000834465	-107.777496337891\\
0.465000003576279	-122.0751953125\\
0.469999998807907	-138.390777587891\\
0.474999994039536	-151.342193603516\\
0.479999989271164	-157.65885925293\\
0.485000014305115	-155.135803222656\\
0.490000009536743	-143.416198730469\\
0.495000004768372	-121.401473999023\\
0.5	-95.5648498535156\\
0.504999995231628	-70.1248321533203\\
0.509999990463257	-48.3503189086914\\
0.514999985694885	-32.8279418945313\\
0.519999980926514	-24.8787002563477\\
0.524999976158142	-25.5267715454102\\
0.529999971389771	-34.0321273803711\\
0.535000026226044	-46.8865356445313\\
0.540000021457672	-62.633415222168\\
0.545000016689301	-78.9098510742188\\
0.550000011920929	-93.835563659668\\
0.555000007152557	-104.756362915039\\
0.560000002384186	-112.471748352051\\
0.564999997615814	-117.465148925781\\
0.569999992847443	-120.551574707031\\
0.574999988079071	-122.838325500488\\
0.579999983310699	-125.376022338867\\
0.584999978542328	-129.067047119141\\
0.589999973773956	-134.573196411133\\
0.595000028610229	-142.227783203125\\
0.600000023841858	-152.035781860352\\
0.605000019073486	-163.700088500977\\
0.610000014305115	-176.858383178711\\
0.615000009536743	-190.861892700195\\
0.620000004768372	-205.138046264648\\
0.625	-219.218490600586\\
0.629999995231628	-232.826293945313\\
0.634999990463257	-245.602508544922\\
0.639999985694885	-257.485076904297\\
0.644999980926514	-268.465698242188\\
0.649999976158142	-278.668579101563\\
0.654999971389771	-288.263366699219\\
0.660000026226044	-297.459350585938\\
0.665000021457672	-306.436889648438\\
0.670000016689301	-315.346954345703\\
0.675000011920929	-324.282073974609\\
0.680000007152557	-333.251678466797\\
0.685000002384186	-342.211761474609\\
0.689999997615814	-351.045043945313\\
0.694999992847443	-359.538360595703\\
0.699999988079071	-367.545043945313\\
0.704999983310699	-374.979400634766\\
0.709999978542328	-381.715301513672\\
0.714999973773956	-387.599273681641\\
0.720000028610229	-392.720886230469\\
0.725000023841858	-397.09619140625\\
0.730000019073486	-400.668273925781\\
0.735000014305115	-403.589233398438\\
0.740000009536743	-405.853363037109\\
0.745000004768372	-407.574279785156\\
0.75	-408.750946044922\\
0.754999995231628	-409.452606201172\\
0.759999990463257	-409.653137207031\\
0.764999985694885	-409.634094238281\\
0.769999980926514	-409.152648925781\\
0.774999976158142	-408.154235839844\\
0.779999971389771	-406.627960205078\\
0.785000026226044	-404.571136474609\\
0.790000021457672	-402.00244140625\\
0.795000016689301	-398.941467285156\\
0.800000011920929	-395.409576416016\\
0.805000007152557	-391.429138183594\\
0.810000002384186	-387.038330078125\\
0.814999997615814	-382.266906738281\\
0.819999992847443	-377.162536621094\\
0.824999988079071	-371.766021728516\\
0.829999983310699	-366.149688720703\\
0.834999978542328	-360.356811523438\\
0.839999973773956	-354.428558349609\\
0.845000028610229	-348.412231445313\\
0.850000023841858	-342.343597412109\\
0.855000019073486	-336.247406005859\\
0.860000014305115	-330.111236572266\\
0.865000009536743	-323.949584960938\\
0.870000004768372	-317.791961669922\\
0.875	-311.649261474609\\
0.879999995231628	-305.545288085938\\
0.884999990463257	-299.512481689453\\
0.889999985694885	-293.586334228516\\
0.894999980926514	-287.794158935547\\
0.899999976158142	-282.158874511719\\
0.904999971389771	-276.700408935547\\
0.910000026226044	-271.448547363281\\
0.915000021457672	-266.432037353516\\
0.920000016689301	-261.683502197266\\
0.925000011920929	-257.227661132813\\
0.930000007152557	-253.085540771484\\
0.935000002384186	-249.252563476563\\
0.939999997615814	-245.761260986328\\
0.944999992847443	-242.594009399414\\
0.949999988079071	-239.685836791992\\
0.954999983310699	-237.061553955078\\
0.959999978542328	-234.727905273438\\
0.964999973773956	-232.693054199219\\
0.970000028610229	-230.950668334961\\
0.975000023841858	-229.503189086914\\
0.980000019073486	-228.351821899414\\
0.985000014305115	-227.499114990234\\
0.990000009536743	-226.962432861328\\
0.995000004768372	-226.74235534668\\
1	-226.865966796875\\
1.00499999523163	-227.357849121094\\
1.00999999046326	-228.215103149414\\
1.01499998569489	-229.282379150391\\
1.01999998092651	-230.577835083008\\
1.02499997615814	-232.074295043945\\
1.02999997138977	-233.779769897461\\
1.0349999666214	-235.670120239258\\
1.03999996185303	-237.742233276367\\
1.04499995708466	-239.971237182617\\
1.04999995231628	-242.347930908203\\
1.05499994754791	-244.857284545898\\
1.05999994277954	-247.487014770508\\
1.06500005722046	-250.235565185547\\
1.07000005245209	-253.086639404297\\
1.07500004768372	-256.031372070313\\
1.08000004291534	-259.053680419922\\
1.08500003814697	-262.139678955078\\
1.0900000333786	-265.270141601563\\
1.09500002861023	-268.430786132813\\
1.10000002384186	-271.605865478516\\
1.10500001907349	-274.780792236328\\
1.11000001430511	-277.943572998047\\
1.11500000953674	-281.077209472656\\
1.12000000476837	-284.166809082031\\
1.125	-287.200500488281\\
1.12999999523163	-290.163482666016\\
1.13499999046326	-293.040740966797\\
1.13999998569489	-295.820587158203\\
1.14499998092651	-298.500061035156\\
1.14999997615814	-301.061859130859\\
1.15499997138977	-303.493682861328\\
1.1599999666214	-305.797607421875\\
1.16499996185303	-307.966461181641\\
1.16999995708466	-309.987823486328\\
1.17499995231628	-311.855987548828\\
1.17999994754791	-313.583557128906\\
1.18499994277954	-315.156402587891\\
1.19000005722046	-316.569915771484\\
1.19500005245209	-317.823974609375\\
1.20000004768372	-318.918395996094\\
1.20500004291534	-319.849853515625\\
1.21000003814697	-320.617614746094\\
1.2150000333786	-321.223785400391\\
1.22000002861023	-321.668212890625\\
1.22500002384186	-321.952453613281\\
1.23000001907349	-322.088562011719\\
1.23500001430511	-322.092346191406\\
1.24000000953674	-321.970336914063\\
1.24500000476837	-321.735534667969\\
1.25	-321.335784912109\\
1.25499999523163	-320.789367675781\\
1.25999999046326	-320.089233398438\\
1.26499998569489	-319.253143310547\\
1.26999998092651	-318.313781738281\\
1.27499997615814	-317.257659912109\\
1.27999997138977	-316.095886230469\\
1.2849999666214	-314.856475830078\\
1.28999996185303	-313.54052734375\\
1.29499995708466	-312.158142089844\\
1.29999995231628	-310.711151123047\\
1.30499994754791	-309.211669921875\\
1.30999994277954	-307.668731689453\\
1.31500005722046	-306.079650878906\\
1.32000005245209	-304.453552246094\\
1.32500004768372	-302.797546386719\\
1.33000004291534	-301.114105224609\\
1.33500003814697	-299.407897949219\\
1.3400000333786	-297.685028076172\\
1.34500002861023	-295.957824707031\\
1.35000002384186	-294.242645263672\\
1.35500001907349	-292.548461914063\\
1.36000001430511	-290.889129638672\\
1.36500000953674	-289.280853271484\\
1.37000000476837	-287.740020751953\\
1.375	-286.279907226563\\
1.37999999523163	-284.881286621094\\
1.38499999046326	-283.566589355469\\
1.38999998569489	-282.344635009766\\
1.39499998092651	-281.170349121094\\
1.39999997615814	-280.060211181641\\
1.40499997138977	-279.012725830078\\
1.4099999666214	-278.020172119141\\
1.41499996185303	-277.081420898438\\
1.41999995708466	-276.198669433594\\
1.42499995231628	-275.397613525391\\
1.42999994754791	-274.67236328125\\
1.43499994277954	-274.029083251953\\
1.44000005722046	-273.476745605469\\
1.44500005245209	-273.019470214844\\
1.45000004768372	-272.663055419922\\
1.45500004291534	-272.4130859375\\
1.46000003814697	-272.275390625\\
1.4650000333786	-272.256958007813\\
1.47000002861023	-272.364807128906\\
1.47500002384186	-272.605529785156\\
1.48000001907349	-272.910217285156\\
1.48500001430511	-273.229309082031\\
1.49000000953674	-273.523254394531\\
1.49500000476837	-273.930541992188\\
1.5	-274.422912597656\\
1.50499999523163	-274.980194091797\\
1.50999999046326	-275.565704345703\\
1.51499998569489	-276.184692382813\\
1.51999998092651	-276.81787109375\\
1.52499997615814	-277.450592041016\\
1.52999997138977	-278.0751953125\\
1.5349999666214	-278.70458984375\\
1.53999996185303	-279.325714111328\\
1.54499995708466	-279.926208496094\\
1.54999995231628	-280.49267578125\\
1.55499994754791	-281.0185546875\\
1.55999994277954	-281.511352539063\\
1.56500005722046	-281.961517333984\\
1.57000005245209	-277.560119628906\\
1.57500004768372	-279.457244873047\\
1.58000004291534	-281.681823730469\\
1.58500003814697	-284.186584472656\\
1.5900000333786	-286.566223144531\\
1.59500002861023	-288.671997070313\\
1.60000002384186	-290.411590576172\\
1.60500001907349	-291.676147460938\\
1.61000001430511	-292.363708496094\\
1.61500000953674	-292.672393798828\\
1.62000000476837	-292.417053222656\\
1.625	-291.996246337891\\
1.62999999523163	-291.579162597656\\
1.63499999046326	-291.46923828125\\
1.63999998569489	-291.857788085938\\
1.64499998092651	-292.127502441406\\
1.64999997615814	-292.466003417969\\
1.65499997138977	-292.889465332031\\
1.6599999666214	-293.346313476563\\
1.66499996185303	-293.783172607422\\
1.66999995708466	-294.333099365234\\
1.67499995231628	-294.772705078125\\
1.67999994754791	-294.926940917969\\
1.68499994277954	-294.955322265625\\
1.69000005722046	-294.792297363281\\
1.69500005245209	-294.544403076172\\
1.70000004768372	-294.214263916016\\
1.70500004291534	-293.816009521484\\
1.71000003814697	-293.400512695313\\
1.7150000333786	-292.944061279297\\
1.72000002861023	-292.454681396484\\
1.72500002384186	-291.937652587891\\
1.73000001907349	-291.397613525391\\
1.73500001430511	-290.839721679688\\
1.74000000953674	-290.263214111328\\
1.74500000476837	-289.669403076172\\
1.75	-289.062286376953\\
1.75499999523163	-288.439727783203\\
1.75999999046326	-287.794769287109\\
1.76499998569489	-287.128723144531\\
1.76999998092651	-286.437194824219\\
1.77499997615814	-285.704467773438\\
1.77999997138977	-284.929931640625\\
1.7849999666214	-284.119201660156\\
1.78999996185303	-283.269683837891\\
1.79499995708466	-282.383636474609\\
1.79999995231628	-281.472839355469\\
1.80499994754791	-280.536712646484\\
1.80999994277954	-279.587463378906\\
1.81500005722046	-278.630676269531\\
1.82000005245209	-277.671051025391\\
1.82500004768372	-276.712768554688\\
1.83000004291534	-275.762023925781\\
1.83500003814697	-274.827331542969\\
1.8400000333786	-273.905944824219\\
1.84500002861023	-272.995513916016\\
1.85000002384186	-272.077819824219\\
1.85500001907349	-271.162078857422\\
1.86000001430511	-270.247680664063\\
1.86500000953674	-269.3330078125\\
1.87000000476837	-268.4189453125\\
1.875	-267.503967285156\\
1.87999999523163	-266.586395263672\\
1.88499999046326	-265.667449951172\\
1.88999998569489	-264.748138427734\\
1.89499998092651	-263.853454589844\\
1.89999997615814	-262.969299316406\\
1.90499997138977	-262.095642089844\\
1.9099999666214	-261.23681640625\\
1.91499996185303	-260.391296386719\\
1.91999995708466	-259.558776855469\\
1.92499995231628	-258.733795166016\\
1.92999994754791	-257.916809082031\\
1.93499994277954	-257.109313964844\\
1.94000005722046	-256.309173583984\\
1.94500005245209	-255.515686035156\\
1.95000004768372	-254.730041503906\\
1.95500004291534	-253.943359375\\
1.96000003814697	-253.142120361328\\
1.9650000333786	-252.327728271484\\
1.97000002861023	-251.513961791992\\
1.97500002384186	-250.750274658203\\
1.98000001907349	-250.024307250977\\
1.98500001430511	-249.338562011719\\
1.99000000953674	-248.655792236328\\
1.99500000476837	-247.993743896484\\
2	-247.349624633789\\
2.00500011444092	-246.670562744141\\
2.00999999046326	-245.964614868164\\
2.01500010490417	-245.220230102539\\
2.01999998092651	-244.468139648438\\
2.02500009536743	-243.695693969727\\
2.02999997138977	-242.914154052734\\
2.03500008583069	-242.40234375\\
2.03999996185303	-242.127075195313\\
2.04500007629395	-242.115692138672\\
2.04999995231628	-242.386383056641\\
2.0550000667572	-243.0009765625\\
2.05999994277954	-244.017303466797\\
2.06500005722046	-245.474792480469\\
2.0699999332428	-247.397033691406\\
2.07500004768372	-249.807586669922\\
2.07999992370605	-252.617126464844\\
2.08500003814697	-255.826812744141\\
2.08999991416931	-259.325286865234\\
2.09500002861023	-263.072265625\\
2.09999990463257	-266.939697265625\\
2.10500001907349	-270.863403320313\\
2.10999989509583	-274.807861328125\\
2.11500000953674	-278.708984375\\
2.11999988555908	-282.305541992188\\
2.125	-285.440063476563\\
2.13000011444092	-288.217407226563\\
2.13499999046326	-290.587280273438\\
2.14000010490417	-292.661041259766\\
2.14499998092651	-294.537322998047\\
2.15000009536743	-296.290405273438\\
2.15499997138977	-297.992584228516\\
2.16000008583069	-299.665466308594\\
2.16499996185303	-301.302520751953\\
2.17000007629395	-302.850494384766\\
2.17499995231628	-304.183654785156\\
2.1800000667572	-305.254669189453\\
2.18499994277954	-305.9638671875\\
2.19000005722046	-306.210754394531\\
2.1949999332428	-305.944152832031\\
2.20000004768372	-305.161437988281\\
2.20499992370605	-303.957733154297\\
2.21000003814697	-302.458374023438\\
2.21499991416931	-300.770568847656\\
2.22000002861023	-298.907135009766\\
2.22499990463257	-297.066467285156\\
2.23000001907349	-295.419677734375\\
2.23499989509583	-294.139770507813\\
2.24000000953674	-293.075805664063\\
2.24499988555908	-292.265899658203\\
2.25	-291.759216308594\\
2.25500011444092	-291.501922607422\\
2.25999999046326	-291.398315429688\\
2.26500010490417	-291.305419921875\\
2.26999998092651	-291.165832519531\\
2.27500009536743	-290.994110107422\\
2.27999997138977	-290.923187255859\\
2.28500008583069	-291.270324707031\\
2.28999996185303	-291.963806152344\\
2.29500007629395	-292.875793457031\\
2.29999995231628	-293.616455078125\\
2.3050000667572	-294.633911132813\\
2.30999994277954	-295.827362060547\\
2.31500005722046	-297.101959228516\\
2.3199999332428	-298.432922363281\\
2.32500004768372	-299.87939453125\\
2.32999992370605	-301.438354492188\\
2.33500003814697	-303.081298828125\\
2.33999991416931	-305.480255126953\\
2.34500002861023	-308.342834472656\\
2.34999990463257	-310.143005371094\\
2.35500001907349	-311.675476074219\\
2.35999989509583	-312.888275146484\\
2.36500000953674	-313.616394042969\\
2.36999988555908	-313.961242675781\\
2.375	-313.749237060547\\
2.38000011444092	-312.940032958984\\
2.38499999046326	-311.731201171875\\
2.39000010490417	-310.221740722656\\
2.39499998092651	-308.598754882813\\
2.40000009536743	-307.065124511719\\
2.40499997138977	-305.837585449219\\
2.41000008583069	-305.051849365234\\
2.41499996185303	-304.812774658203\\
2.42000007629395	-305.186676025391\\
2.42499995231628	-306.085296630859\\
2.4300000667572	-307.614929199219\\
2.43499994277954	-309.650177001953\\
2.44000005722046	-312.091156005859\\
2.4449999332428	-315.198638916016\\
2.45000004768372	-318.050720214844\\
2.45499992370605	-321.052459716797\\
2.46000003814697	-323.874542236328\\
2.46499991416931	-326.324584960938\\
2.47000002861023	-328.470062255859\\
2.47499990463257	-330.320770263672\\
2.48000001907349	-331.883575439453\\
2.48499989509583	-333.216217041016\\
2.49000000953674	-334.255493164063\\
2.49499988555908	-334.873382568359\\
2.5	-335.318908691406\\
2.50500011444092	-336.026092529297\\
2.50999999046326	-337.137329101563\\
2.51500010490417	-338.498718261719\\
2.51999998092651	-339.8330078125\\
2.52500009536743	-340.812561035156\\
2.52999997138977	-341.450469970703\\
2.53500008583069	-341.203460693359\\
2.53999996185303	-340.488647460938\\
2.54500007629395	-339.073425292969\\
2.54999995231628	-336.981353759766\\
2.5550000667572	-334.272827148438\\
2.55999994277954	-330.997192382813\\
2.56500005722046	-327.300933837891\\
2.5699999332428	-323.21484375\\
2.57500004768372	-318.820098876953\\
2.57999992370605	-314.179870605469\\
2.58500003814697	-309.337310791016\\
2.58999991416931	-304.438293457031\\
2.59500002861023	-299.251342773438\\
2.59999990463257	-293.792083740234\\
2.60500001907349	-288.127807617188\\
2.60999989509583	-282.144470214844\\
2.61500000953674	-275.847961425781\\
2.61999988555908	-269.223022460938\\
2.625	-262.319946289063\\
2.63000011444092	-255.285766601563\\
2.63499999046326	-248.305511474609\\
2.64000010490417	-241.206512451172\\
2.64499998092651	-234.283950805664\\
2.65000009536743	-227.974456787109\\
2.65499997138977	-222.156875610352\\
2.66000008583069	-216.843948364258\\
2.66499996185303	-212.174514770508\\
2.67000007629395	-208.031341552734\\
2.67499995231628	-204.156845092773\\
2.6800000667572	-200.387237548828\\
2.68499994277954	-196.962875366211\\
2.69000005722046	-193.52995300293\\
2.6949999332428	-190.027236938477\\
2.70000004768372	-186.439437866211\\
2.70499992370605	-182.724304199219\\
2.71000003814697	-178.941177368164\\
2.71499991416931	-175.183517456055\\
2.72000002861023	-171.531707763672\\
2.72499990463257	-168.069778442383\\
2.73000001907349	-164.87760925293\\
2.73499989509583	-162.025192260742\\
2.74000000953674	-159.592102050781\\
2.74499988555908	-157.675903320313\\
2.75	-156.106903076172\\
2.75500011444092	-154.77815246582\\
2.75999999046326	-153.619232177734\\
2.76500010490417	-152.755340576172\\
2.76999998092651	-152.255844116211\\
2.77500009536743	-152.112548828125\\
2.77999997138977	-152.025497436523\\
2.78500008583069	-151.949829101563\\
2.78999996185303	-152.028137207031\\
2.79500007629395	-152.510467529297\\
2.79999995231628	-153.703323364258\\
2.8050000667572	-155.902389526367\\
2.80999994277954	-159.487152099609\\
2.81500005722046	-164.761962890625\\
2.8199999332428	-171.950820922852\\
2.82500004768372	-181.098297119141\\
2.82999992370605	-192.140808105469\\
2.83500003814697	-204.662658691406\\
2.83999991416931	-218.356414794922\\
2.84500002861023	-232.827178955078\\
2.84999990463257	-247.620697021484\\
2.85500001907349	-262.804656982422\\
2.85999989509583	-278.007202148438\\
2.86500000953674	-293.214080810547\\
2.86999988555908	-308.628021240234\\
2.875	-323.848937988281\\
2.88000011444092	-339.138427734375\\
2.88499999046326	-354.157775878906\\
2.89000010490417	-368.91357421875\\
2.89499998092651	-382.393951416016\\
2.90000009536743	-393.370300292969\\
2.90499997138977	-401.624694824219\\
2.91000008583069	-406.651214599609\\
2.91499996185303	-408.207916259766\\
2.92000007629395	-406.896362304688\\
2.92499995231628	-403.53857421875\\
2.9300000667572	-399.106842041016\\
2.93499994277954	-394.649627685547\\
2.94000005722046	-391.0361328125\\
2.9449999332428	-388.956665039063\\
2.95000004768372	-388.567443847656\\
2.95499992370605	-390.246795654297\\
2.96000003814697	-393.856048583984\\
2.96499991416931	-397.463592529297\\
2.97000002861023	-401.660247802734\\
2.97499990463257	-406.27490234375\\
2.98000001907349	-410.323669433594\\
2.98499989509583	-414.219055175781\\
2.99000000953674	-418.196380615234\\
2.99499988555908	-419.341522216797\\
3	-418.2607421875\\
3.00500011444092	-414.900604248047\\
3.00999999046326	-409.412261962891\\
3.01500010490417	-402.109893798828\\
3.01999998092651	-393.467834472656\\
3.02500009536743	-383.948516845703\\
3.02999997138977	-374.097259521484\\
3.03500008583069	-364.417572021484\\
3.03999996185303	-354.911285400391\\
3.04500007629395	-345.614837646484\\
3.04999995231628	-336.158203125\\
3.0550000667572	-326.080963134766\\
3.05999994277954	-315.177825927734\\
3.06500005722046	-302.858703613281\\
3.0699999332428	-288.747253417969\\
3.07500004768372	-273.375854492188\\
3.07999992370605	-256.652526855469\\
3.08500003814697	-238.940902709961\\
3.08999991416931	-220.886734008789\\
3.09500002861023	-202.934097290039\\
3.09999990463257	-185.295135498047\\
3.10500001907349	-168.255828857422\\
3.10999989509583	-151.929107666016\\
3.11500000953674	-136.75944519043\\
3.11999988555908	-122.469604492188\\
3.125	-108.508880615234\\
3.13000011444092	-94.2756042480469\\
3.13499999046326	-78.2179641723633\\
3.14000010490417	-61.2780914306641\\
3.14499998092651	-44.2243347167969\\
3.15000009536743	-29.8772888183594\\
3.15499997138977	-20.6112442016602\\
3.16000008583069	-16.4018936157227\\
3.16499996185303	-18.7440032958984\\
3.17000007629395	-28.3176574707031\\
3.17499995231628	-45.1533813476563\\
3.1800000667572	-68.0756072998047\\
3.18499994277954	-95.3065338134766\\
3.19000005722046	-126.002868652344\\
3.1949999332428	-159.844223022461\\
3.20000004768372	-195.105895996094\\
3.20499992370605	-228.661376953125\\
3.21000003814697	-260.738464355469\\
3.21499991416931	-288.76806640625\\
3.22000002861023	-311.951599121094\\
3.22499990463257	-330.5546875\\
3.23000001907349	-344.67431640625\\
3.23499989509583	-354.313415527344\\
3.24000000953674	-360.088348388672\\
3.24499988555908	-361.910095214844\\
3.25	-361.128295898438\\
3.25500011444092	-358.364410400391\\
3.25999999046326	-354.698486328125\\
3.26500010490417	-351.193115234375\\
3.26999998092651	-348.628845214844\\
3.27500009536743	-347.707916259766\\
3.27999997138977	-348.555908203125\\
3.28500008583069	-351.086395263672\\
3.28999996185303	-354.999755859375\\
3.29500007629395	-359.621124267578\\
3.29999995231628	-364.520751953125\\
3.3050000667572	-369.316223144531\\
3.30999994277954	-373.065307617188\\
3.31500005722046	-376.180908203125\\
3.3199999332428	-379.518768310547\\
3.32500004768372	-382.199554443359\\
3.32999992370605	-383.552673339844\\
3.33500003814697	-383.710815429688\\
3.33999991416931	-382.778289794922\\
3.34500002861023	-381.134735107422\\
3.34999990463257	-379.456939697266\\
3.35500001907349	-378.044189453125\\
3.35999989509583	-376.823638916016\\
3.36500000953674	-375.762298583984\\
3.36999988555908	-374.448211669922\\
3.375	-372.424652099609\\
3.38000011444092	-369.237548828125\\
3.38499999046326	-364.423187255859\\
3.39000010490417	-357.613952636719\\
3.39499998092651	-348.625885009766\\
3.40000009536743	-337.29248046875\\
3.40499997138977	-323.919891357422\\
3.41000008583069	-308.699401855469\\
3.41499996185303	-291.913421630859\\
3.42000007629395	-273.944244384766\\
3.42499995231628	-255.041641235352\\
3.4300000667572	-235.48616027832\\
3.43499994277954	-215.302810668945\\
3.44000005722046	-194.399490356445\\
3.4449999332428	-172.835861206055\\
3.45000004768372	-150.518127441406\\
3.45499992370605	-127.154693603516\\
3.46000003814697	-100.797836303711\\
3.46499991416931	-72.4869079589844\\
3.47000002861023	-44.7765884399414\\
3.47499990463257	-19.490364074707\\
3.48000001907349	1.01911163330078\\
3.48499989509583	14.3677749633789\\
3.49000000953674	14.7798614501953\\
3.49499988555908	7.51211547851563\\
3.5	-6.35786437988281\\
3.50500011444092	-28.0252685546875\\
3.50999999046326	-58.0296173095703\\
3.51500010490417	-94.3422546386719\\
3.51999998092651	-139.047805786133\\
3.52500009536743	-189.071334838867\\
3.52999997138977	-243.983901977539\\
3.53500008583069	-301.319519042969\\
3.53999996185303	-359.106689453125\\
3.54500007629395	-412.382263183594\\
3.54999995231628	-451.516174316406\\
3.5550000667572	-469.138092041016\\
3.55999994277954	-465.754272460938\\
3.56500005722046	-442.072631835938\\
3.5699999332428	-405.549682617188\\
3.57500004768372	-361.431182861328\\
3.57999992370605	-319.875061035156\\
3.58500003814697	-283.656555175781\\
3.58999991416931	-256.501556396484\\
3.59500002861023	-243.711791992188\\
3.59999990463257	-248.720336914063\\
3.60500001907349	-266.715270996094\\
3.60999989509583	-292.213806152344\\
3.61500000953674	-321.557464599609\\
3.61999988555908	-353.751586914063\\
3.625	-384.321746826172\\
3.63000011444092	-409.537902832031\\
3.63499999046326	-428.297241210938\\
3.64000010490417	-440.363525390625\\
3.64499998092651	-441.881042480469\\
3.65000009536743	-433.547210693359\\
3.65499997138977	-419.780059814453\\
3.66000008583069	-404.332885742188\\
3.66499996185303	-389.526916503906\\
3.67000007629395	-377.385009765625\\
3.67499995231628	-369.949523925781\\
3.6800000667572	-367.075317382813\\
3.68499994277954	-367.502197265625\\
3.69000005722046	-368.988098144531\\
3.6949999332428	-368.824371337891\\
3.70000004768372	-364.313018798828\\
3.70499992370605	-353.990570068359\\
3.71000003814697	-336.303527832031\\
3.71499991416931	-311.936645507813\\
3.72000002861023	-281.844177246094\\
3.72499990463257	-247.919189453125\\
3.73000001907349	-212.18603515625\\
3.73499989509583	-176.918838500977\\
3.74000000953674	-143.04670715332\\
3.74499988555908	-111.290267944336\\
3.75	-79.6751174926758\\
3.75500011444092	-44.4279251098633\\
3.75999999046326	-4.45027923583984\\
3.76500010490417	37.4040298461914\\
3.76999998092651	75.6792755126953\\
3.77500009536743	101.84944152832\\
3.77999997138977	112.170623779297\\
3.78500008583069	104.595436096191\\
3.78999996185303	82.8487701416016\\
3.79500007629395	48.0095825195313\\
3.79999995231628	-3.25238037109375\\
3.8050000667572	-69.2394866943359\\
3.80999994277954	-147.444046020508\\
3.81500005722046	-233.733581542969\\
3.8199999332428	-323.409759521484\\
3.82500004768372	-410.744689941406\\
3.82999992370605	-489.664001464844\\
3.83500003814697	-552.734375\\
3.83999991416931	-586.011901855469\\
3.84500002861023	-582.29833984375\\
3.84999990463257	-549.971435546875\\
3.85500001907349	-495.15966796875\\
3.85999989509583	-431.049713134766\\
3.86500000953674	-365.725158691406\\
3.86999988555908	-304.003570556641\\
3.875	-254.876708984375\\
3.88000011444092	-224.588775634766\\
3.88499999046326	-219.755126953125\\
3.89000010490417	-235.832855224609\\
3.89499998092651	-264.7626953125\\
3.90000009536743	-297.517639160156\\
3.90499997138977	-334.832580566406\\
3.91000008583069	-370.167175292969\\
3.91499996185303	-398.374206542969\\
3.92000007629395	-417.944885253906\\
3.92499995231628	-429.895202636719\\
3.9300000667572	-428.103210449219\\
3.93499994277954	-416.392456054688\\
3.94000005722046	-400.175476074219\\
3.9449999332428	-383.599182128906\\
3.95000004768372	-369.137329101563\\
3.95499992370605	-357.441436767578\\
3.96000003814697	-351.784942626953\\
3.96499991416931	-351.889678955078\\
3.97000002861023	-355.837554931641\\
3.97499990463257	-360.814025878906\\
3.98000001907349	-363.013061523438\\
3.98499989509583	-359.369384765625\\
3.99000000953674	-347.680541992188\\
3.99499988555908	-327.012268066406\\
4	-297.322784423828\\
4.00500011444092	-259.607666015625\\
4.01000022888184	-216.739120483398\\
4.0149998664856	-171.632797241211\\
4.01999998092651	-125.988067626953\\
4.02500009536743	-81.5640258789063\\
4.03000020980835	-37.3494262695313\\
4.03499984741211	10.1581954956055\\
4.03999996185303	61.0985717773438\\
4.04500007629395	113.678466796875\\
4.05000019073486	163.02490234375\\
4.05499982833862	202.730545043945\\
4.05999994277954	226.483581542969\\
4.06500005722046	229.442230224609\\
4.07000017166138	216.390502929688\\
4.07499980926514	189.077224731445\\
4.07999992370605	139.340972900391\\
4.08500003814697	66.7824020385742\\
4.09000015258789	-30.3341369628906\\
4.09499979019165	-152.664001464844\\
4.09999990463257	-298.070587158203\\
4.10500001907349	-462.673187255859\\
4.1100001335144	-627.137817382813\\
4.11499977111816	-740.085693359375\\
4.11999988555908	-794.27392578125\\
4.125	-757.351806640625\\
4.13000011444092	-646.116333007813\\
4.13500022888184	-521.678771972656\\
4.1399998664856	-375.687133789063\\
4.14499998092651	-233.904052734375\\
4.15000009536743	-119.758148193359\\
4.15500020980835	-52.2405090332031\\
4.15999984741211	-39.9619140625\\
4.16499996185303	-89.6195068359375\\
4.17000007629395	-158.387847900391\\
4.17500019073486	-250.955505371094\\
4.17999982833862	-350.857330322266\\
4.18499994277954	-441.783752441406\\
4.19000005722046	-510.008636474609\\
4.19500017166138	-548.586181640625\\
4.19999980926514	-564.838256835938\\
4.20499992370605	-542.565979003906\\
4.21000003814697	-499.073150634766\\
4.21500015258789	-447.244934082031\\
4.21999979019165	-399.021301269531\\
4.22499990463257	-362.779724121094\\
4.23000001907349	-339.856475830078\\
4.2350001335144	-330.939544677734\\
4.23999977111816	-336.821716308594\\
4.24499988555908	-353.368774414063\\
4.25	-370.934722900391\\
4.25500011444092	-382.869201660156\\
4.26000022888184	-382.3515625\\
4.2649998664856	-364.514038085938\\
4.26999998092651	-328.311340332031\\
4.27500009536743	-275.574035644531\\
4.28000020980835	-207.76350402832\\
4.28499984741211	-136.082214355469\\
4.28999996185303	-65.5757064819336\\
4.29500007629395	2.94557189941406\\
4.30000019073486	77.580207824707\\
4.30499982833862	163.14306640625\\
4.30999994277954	248.295227050781\\
4.31500005722046	321.844909667969\\
4.32000017166138	372.433563232422\\
4.32499980926514	391.618957519531\\
4.32999992370605	367.707916259766\\
4.33500003814697	326.919097900391\\
4.34000015258789	256.810546875\\
4.34499979019165	153.356994628906\\
4.34999990463257	17.31298828125\\
4.35500001907349	-149.183959960938\\
4.3600001335144	-338.631195068359\\
4.36499977111816	-543.633544921875\\
4.36999988555908	-724.737548828125\\
4.375	-827.584716796875\\
4.38000011444092	-890.195190429688\\
4.38500022888184	-904.471862792969\\
4.3899998664856	-877.458312988281\\
4.39499998092651	-696.516357421875\\
4.40000009536743	-488.857238769531\\
4.40500020980835	-275.393463134766\\
4.40999984741211	-96.1848754882813\\
4.41499996185303	24.0112609863281\\
4.42000007629395	70.2714538574219\\
4.42500019073486	26.1497802734375\\
4.42999982833862	-57.0284423828125\\
4.43499994277954	-173.139709472656\\
4.44000005722046	-303.873992919922\\
4.44500017166138	-425.664367675781\\
4.44999980926514	-518.809448242188\\
4.45499992370605	-571.035522460938\\
4.46000003814697	-593.057189941406\\
4.46500015258789	-564.836242675781\\
4.46999979019165	-506.241088867188\\
4.47499990463257	-437.345764160156\\
4.48000001907349	-372.829376220703\\
4.4850001335144	-324.901947021484\\
4.48999977111816	-297.548034667969\\
4.49499988555908	-287.475982666016\\
4.5	-298.61181640625\\
4.50500011444092	-323.41748046875\\
4.51000022888184	-352.117370605469\\
4.5149998664856	-373.520660400391\\
4.51999998092651	-378.629211425781\\
4.52500009536743	-360.915893554688\\
4.53000020980835	-319.332153320313\\
4.53499984741211	-254.901000976563\\
4.53999996185303	-170.730178833008\\
4.54500007629395	-80.9774856567383\\
4.55000019073486	9.25482177734375\\
4.55499982833862	100.899772644043\\
4.55999994277954	173.513717651367\\
4.56500005722046	234.4560546875\\
4.57000017166138	298.381164550781\\
4.57499980926514	356.681213378906\\
4.57999992370605	402.276336669922\\
4.58500003814697	430.053527832031\\
4.59000015258789	441.39501953125\\
4.59499979019165	444.330261230469\\
4.59999990463257	425.404052734375\\
4.60500001907349	382.940063476563\\
4.6100001335144	315.678649902344\\
4.61499977111816	223.032287597656\\
4.61999988555908	-101.285987854004\\
4.625	-510.628479003906\\
4.63000011444092	-766.574279785156\\
4.63500022888184	-936.761047363281\\
4.6399998664856	-1025.18359375\\
4.64499998092651	-1035.79614257813\\
4.65000009536743	-1005.54064941406\\
4.65500020980835	-933.996826171875\\
4.65999984741211	-810.490600585938\\
4.66499996185303	-286.089813232422\\
4.67000007629395	48.4723510742188\\
4.67500019073486	275.719299316406\\
4.67999982833862	368.648712158203\\
4.68499994277954	300.675994873047\\
4.69000005722046	155.520263671875\\
4.69500017166138	-59.0437927246094\\
4.69999980926514	-300.935729980469\\
4.70499992370605	-524.694702148438\\
4.71000003814697	-690.605224609375\\
4.71500015258789	-774.929626464844\\
4.71999979019165	-796.358825683594\\
4.72499990463257	-723.115844726563\\
4.73000001907349	-598.663940429688\\
4.7350001335144	-457.075103759766\\
4.73999977111816	-329.587646484375\\
4.74499988555908	-237.599609375\\
4.75	-190.461242675781\\
4.75500011444092	-175.461410522461\\
4.76000022888184	-198.998611450195\\
4.7649998664856	-250.996002197266\\
4.76999998092651	-303.745208740234\\
4.77500009536743	-347.769226074219\\
4.78000020980835	-362.588623046875\\
4.78499984741211	-337.491088867188\\
4.78999996185303	-273.804565429688\\
4.79500007629395	-192.943237304688\\
4.80000019073486	-64.9232788085938\\
4.80499982833862	85.9361801147461\\
4.80999994277954	253.247741699219\\
4.81500005722046	386.240051269531\\
4.82000017166138	493.14306640625\\
4.82499980926514	571.953247070313\\
4.82999992370605	612.623779296875\\
4.83500003814697	617.345642089844\\
4.84000015258789	595.08984375\\
4.84499979019165	530.431762695313\\
4.84999990463257	426.819274902344\\
4.85500001907349	289.286010742188\\
4.8600001335144	127.642883300781\\
4.86499977111816	4.27019500732422\\
4.86999988555908	-335.834289550781\\
4.875	-630.431396484375\\
4.88000011444092	-863.402954101563\\
4.88500022888184	-1007.04901123047\\
4.8899998664856	-1066.25671386719\\
4.89499998092651	-1050.09851074219\\
4.90000009536743	-1007.671875\\
4.90500020980835	-912.054138183594\\
4.90999984741211	-769.034606933594\\
4.91499996185303	-597.318664550781\\
4.92000007629395	-64.0718994140625\\
4.92500019073486	347.883666992188\\
4.92999982833862	602.026184082031\\
4.93499994277954	666.1396484375\\
4.94000005722046	531.961547851563\\
4.94500017166138	337.968170166016\\
4.94999980926514	67.7607116699219\\
4.95499992370605	-221.334930419922\\
4.96000003814697	-476.808685302734\\
4.96500015258789	-657.413818359375\\
4.96999979019165	-741.313354492188\\
4.97499990463257	-734.999267578125\\
4.98000001907349	-641.389404296875\\
4.9850001335144	-494.148223876953\\
4.98999977111816	-329.822875976563\\
4.99499988555908	-181.413757324219\\
5	-71.3558959960938\\
5.00500011444092	-11.923095703125\\
5.01000022888184	4.3013916015625\\
5.0149998664856	-21.4671325683594\\
5.01999998092651	-92.938232421875\\
5.02500009536743	-178.227233886719\\
5.03000020980835	-262.885223388672\\
5.03499984741211	-333.714477539063\\
5.03999996185303	-390.74365234375\\
5.04500007629395	-419.529479980469\\
5.05000019073486	-414.745971679688\\
5.05499982833862	-378.033233642578\\
5.05999994277954	-313.720245361328\\
5.06500005722046	-229.272705078125\\
5.07000017166138	-132.626434326172\\
5.07499980926514	-32.4915542602539\\
5.07999992370605	63.3241844177246\\
5.08500003814697	69.0185089111328\\
5.09000015258789	54.8408699035645\\
5.09499979019165	34.7169151306152\\
5.09999990463257	11.4997787475586\\
5.10500001907349	6.01758432388306\\
5.1100001335144	5.8624701499939\\
5.11499977111816	5.4753041267395\\
5.11999988555908	4.99839019775391\\
5.125	4.50605726242065\\
5.13000011444092	4.06520748138428\\
5.13500022888184	-187.964614868164\\
5.1399998664856	-348.257049560547\\
5.14499998092651	-444.876098632813\\
5.15000009536743	-489.151123046875\\
5.15500020980835	-490.267883300781\\
5.15999984741211	-301.386169433594\\
5.16499996185303	-133.313140869141\\
5.17000007629395	17.6257171630859\\
5.17500019073486	123.861694335938\\
5.17999982833862	168.689682006836\\
5.18499994277954	135.29264831543\\
5.19000005722046	63.0110473632813\\
5.19500017166138	-29.6387634277344\\
5.19999980926514	-140.745758056641\\
5.20499992370605	-252.098663330078\\
5.21000003814697	-348.218353271484\\
5.21500015258789	-417.978973388672\\
5.21999979019165	-456.199615478516\\
5.22499990463257	-459.441772460938\\
5.23000001907349	-440.507110595703\\
5.2350001335144	-408.448028564453\\
5.23999977111816	-367.844299316406\\
5.24499988555908	-328.571258544922\\
5.25	-297.701232910156\\
5.25500011444092	-279.939514160156\\
5.26000022888184	-272.756927490234\\
5.2649998664856	-282.363189697266\\
5.26999998092651	-312.418762207031\\
5.27500009536743	-354.914886474609\\
5.28000020980835	-403.673065185547\\
5.28499984741211	-452.758483886719\\
5.28999996185303	-496.474945068359\\
5.29500007629395	-530.08056640625\\
5.30000019073486	-552.897766113281\\
5.30499982833862	-561.458251953125\\
5.30999994277954	-552.497253417969\\
5.31500005722046	-527.796325683594\\
5.32000017166138	-486.126525878906\\
5.32499980926514	-439.079040527344\\
5.32999992370605	-393.561981201172\\
5.33500003814697	-353.973907470703\\
5.34000015258789	-323.044555664063\\
5.34499979019165	-299.784301757813\\
5.34999990463257	-287.859893798828\\
5.35500001907349	-285.952880859375\\
5.3600001335144	-290.959716796875\\
5.36499977111816	-298.235229492188\\
5.36999988555908	-303.1533203125\\
5.375	-300.615783691406\\
5.38000011444092	-288.387390136719\\
5.38500022888184	-265.472595214844\\
5.3899998664856	-233.157989501953\\
5.39499998092651	-194.179870605469\\
5.40000009536743	-155.398529052734\\
5.40500020980835	-120.623016357422\\
5.40999984741211	-90.6354904174805\\
5.41499996185303	-68.3142395019531\\
5.42000007629395	-52.848934173584\\
5.42500019073486	-42.6944885253906\\
5.42999982833862	-32.8203430175781\\
5.43499994277954	-18.637809753418\\
5.44000005722046	-5.0857629776001\\
5.44500017166138	0.291438698768616\\
5.44999980926514	0.571109235286713\\
5.45499992370605	0.677574217319489\\
5.46000003814697	0.698699772357941\\
5.46500015258789	0.675340712070465\\
5.46999979019165	0.658027112483978\\
5.47499990463257	0.582719385623932\\
5.48000001907349	0.62347424030304\\
5.4850001335144	0.536334037780762\\
5.48999977111816	0.53626149892807\\
5.49499988555908	0.515532732009888\\
5.5	0.481592893600464\\
5.50500011444092	0.440483540296555\\
5.51000022888184	0.407835096120834\\
5.5149998664856	0.410282403230667\\
5.51999998092651	-17.3880996704102\\
5.52500009536743	-53.3413238525391\\
5.53000020980835	-71.1553192138672\\
5.53499984741211	-84.4771041870117\\
5.53999996185303	-93.3805084228516\\
5.54500007629395	-98.2696304321289\\
5.55000019073486	-99.5766525268555\\
5.55499982833862	-98.1045303344727\\
5.55999994277954	-95.3168182373047\\
5.56500005722046	-90.8873748779297\\
5.57000017166138	-84.480827331543\\
5.57499980926514	-76.6903457641602\\
5.57999992370605	-67.8119201660156\\
5.58500003814697	-58.2793579101563\\
5.59000015258789	-48.36279296875\\
5.59499979019165	-38.777946472168\\
5.59999990463257	-29.6600914001465\\
5.60500001907349	-21.5646438598633\\
5.6100001335144	-14.7318029403687\\
5.61499977111816	-9.17388534545898\\
5.61999988555908	-4.95590353012085\\
5.625	-1.95102787017822\\
5.63000011444092	0.000604867935180664\\
5.63500022888184	1.31969320774078\\
5.6399998664856	1.70227658748627\\
5.64499998092651	1.74398672580719\\
5.65000009536743	1.80060458183289\\
5.65500020980835	1.83761322498322\\
5.65999984741211	1.90236794948578\\
5.66499996185303	1.94421362876892\\
5.67000007629395	-0.780243873596191\\
5.67500019073486	-85.4624938964844\\
5.67999982833862	-109.36353302002\\
5.68499994277954	-119.854415893555\\
5.69000005722046	-163.876861572266\\
5.69500017166138	-181.167266845703\\
5.69999980926514	-186.029266357422\\
5.70499992370605	-181.71467590332\\
5.71000003814697	-172.271942138672\\
5.71500015258789	-166.380554199219\\
5.71999979019165	-164.485290527344\\
5.72499990463257	-159.890014648438\\
5.73000001907349	-153.998519897461\\
5.7350001335144	-146.950836181641\\
5.73999977111816	-139.911987304688\\
5.74499988555908	-133.438034057617\\
5.75	-127.986114501953\\
5.75500011444092	-124.031997680664\\
5.76000022888184	-122.618026733398\\
5.7649998664856	-123.019546508789\\
5.76999998092651	-125.001289367676\\
5.77500009536743	-128.398361206055\\
5.78000020980835	-131.967208862305\\
5.78499984741211	-134.640472412109\\
5.78999996185303	-135.940765380859\\
5.79500007629395	-134.468933105469\\
5.80000019073486	-129.504043579102\\
5.80499982833862	-120.198204040527\\
5.80999994277954	-106.036743164063\\
5.81500005722046	-86.4017028808594\\
5.82000017166138	-61.2989044189453\\
5.82499980926514	-30.9296417236328\\
5.82999992370605	4.48492431640625\\
5.83500003814697	43.7237243652344\\
5.84000015258789	84.8995208740234\\
5.84499979019165	125.970016479492\\
5.84999990463257	164.562255859375\\
5.85500001907349	199.075103759766\\
5.8600001335144	223.242431640625\\
5.86499977111816	233.144958496094\\
5.86999988555908	224.635452270508\\
5.875	195.058685302734\\
5.88000011444092	144.393798828125\\
5.88500022888184	71.6737976074219\\
5.8899998664856	-0.242403984069824\\
5.89499998092651	-23.2886505126953\\
5.90000009536743	-107.813385009766\\
5.90500020980835	-302.606353759766\\
5.90999984741211	-650.8740234375\\
5.91499996185303	-1108.81164550781\\
5.92000007629395	-1610.20251464844\\
5.92500019073486	-2101.0517578125\\
5.92999982833862	-2539.39892578125\\
5.93499994277954	-2907.71264648438\\
5.94000005722046	-3183.6201171875\\
5.94500017166138	-3363.25610351563\\
5.94999980926514	-3446.1826171875\\
5.95499992370605	-3426.93994140625\\
5.96000003814697	-3354.6533203125\\
5.96500015258789	-3186.92578125\\
5.96999979019165	-2909.4658203125\\
5.97499990463257	-2528.615234375\\
5.98000001907349	-1531.36010742188\\
5.9850001335144	-573.876403808594\\
5.98999977111816	178.80078125\\
5.99499988555908	619.525512695313\\
6	734.527709960938\\
6.00500011444092	494.614562988281\\
6.01000022888184	72.6339111328125\\
6.0149998664856	-496.454223632813\\
6.01999998092651	-1123.75134277344\\
6.02500009536743	-1716.0810546875\\
6.03000020980835	-2187.0458984375\\
6.03499984741211	-2476.77880859375\\
6.03999996185303	-2556.24853515625\\
6.04500007629395	-2458.76440429688\\
6.05000019073486	-2195.63110351563\\
6.05499982833862	-1821.70727539063\\
6.05999994277954	-1410.46875\\
6.06500005722046	-1027.53723144531\\
6.07000017166138	-721.80517578125\\
6.07499980926514	-518.67431640625\\
6.07999992370605	-419.413391113281\\
6.08500003814697	-402.67431640625\\
6.09000015258789	-497.732849121094\\
6.09499979019165	-649.047424316406\\
6.09999990463257	-813.453308105469\\
6.10500001907349	-958.576232910156\\
6.1100001335144	-1060.45288085938\\
6.11499977111816	-1120.32861328125\\
6.11999988555908	-1112.09350585938\\
6.125	-1029.72900390625\\
6.13000011444092	-871.93310546875\\
6.13500022888184	-679.225402832031\\
6.1399998664856	-480.794860839844\\
6.14499998092651	-300.676696777344\\
6.15000009536743	-156.313415527344\\
6.15500020980835	-49.3683471679688\\
6.15999984741211	12.2393798828125\\
6.16499996185303	26.994873046875\\
6.17000007629395	2.00572204589844\\
6.17500019073486	-49.3865203857422\\
6.17999982833862	-111.658309936523\\
6.18499994277954	-168.204040527344\\
6.19000005722046	-163.087631225586\\
6.19500017166138	-137.270248413086\\
6.19999980926514	-101.029434204102\\
6.20499992370605	-58.0224075317383\\
6.21000003814697	-12.5806827545166\\
6.21500015258789	4.41610956192017\\
6.21999979019165	4.71465492248535\\
6.22499990463257	4.64253234863281\\
6.23000001907349	4.38992547988892\\
6.2350001335144	4.0585503578186\\
6.23999977111816	3.70336580276489\\
6.24499988555908	3.35898923873901\\
6.25	3.03939604759216\\
6.25500011444092	2.74179673194885\\
6.26000022888184	2.46936702728271\\
6.2649998664856	2.21646022796631\\
6.26999998092651	1.98640048503876\\
6.27500009536743	1.78211307525635\\
6.28000020980835	1.59828293323517\\
6.28499984741211	1.43171012401581\\
6.28999996185303	1.27968871593475\\
6.29500007629395	1.14249551296234\\
6.30000019073486	1.02072072029114\\
6.30499982833862	0.912636280059814\\
6.30999994277954	0.815488517284393\\
6.31500005722046	0.729410529136658\\
6.32000017166138	0.654343128204346\\
6.32499980926514	0.589647233486176\\
6.32999992370605	0.532147765159607\\
6.33500003814697	0.475352764129639\\
6.34000015258789	0.416025966405869\\
6.34499979019165	0.36524623632431\\
6.34999990463257	0.324209004640579\\
6.35500001907349	0.293757319450378\\
6.3600001335144	114.228424072266\\
6.36499977111816	161.314407348633\\
6.36999988555908	191.467010498047\\
6.375	205.425354003906\\
6.38000011444092	204.981292724609\\
6.38500022888184	182.05810546875\\
6.3899998664856	-97.3494567871094\\
6.39499998092651	-263.712249755859\\
6.40000009536743	-405.27197265625\\
6.40500020980835	-508.568054199219\\
6.40999984741211	-561.717712402344\\
6.41499996185303	-579.007446289063\\
6.42000007629395	-567.871337890625\\
6.42500019073486	-538.716796875\\
6.42999982833862	-501.876647949219\\
6.43499994277954	-467.111511230469\\
6.44000005722046	-441.653442382813\\
6.44500017166138	-430.459899902344\\
6.44999980926514	-435.663391113281\\
6.45499992370605	-456.733764648438\\
6.46000003814697	-491.099487304688\\
6.46500015258789	-534.819702148438\\
6.46999979019165	-582.938232421875\\
6.47499990463257	-631.05810546875\\
6.48000001907349	-675.015930175781\\
6.4850001335144	-711.706359863281\\
6.48999977111816	-739.236389160156\\
6.49499988555908	-757.002136230469\\
6.5	-765.400817871094\\
6.50500011444092	-765.667419433594\\
6.51000022888184	-759.689270019531\\
6.5149998664856	-749.519653320313\\
6.51999998092651	-737.259643554688\\
6.52500009536743	-724.680603027344\\
6.53000020980835	-713.27734375\\
6.53499984741211	-703.5205078125\\
6.53999996185303	-695.832214355469\\
6.54500007629395	-690.088134765625\\
6.55000019073486	-685.804931640625\\
6.55499982833862	-682.044860839844\\
6.55999994277954	-677.925842285156\\
6.56500005722046	-673.988220214844\\
6.57000017166138	-669.318603515625\\
6.57499980926514	-663.758605957031\\
6.57999992370605	-655.197387695313\\
6.58500003814697	-644.744506835938\\
6.59000015258789	-631.799926757813\\
6.59499979019165	-616.567077636719\\
6.59999990463257	-599.497436523438\\
6.60500001907349	-581.240051269531\\
6.6100001335144	-562.204895019531\\
6.61499977111816	-542.875305175781\\
6.61999988555908	-524.1982421875\\
6.625	-506.386962890625\\
6.63000011444092	-489.881896972656\\
6.63500022888184	-474.708984375\\
6.6399998664856	-460.616027832031\\
6.64499998092651	-447.886962890625\\
6.65000009536743	-436.354187011719\\
6.65500020980835	-425.305786132813\\
6.65999984741211	-414.898559570313\\
6.66499996185303	-404.825927734375\\
6.67000007629395	-394.525817871094\\
6.67500019073486	-384.398468017578\\
6.67999982833862	-374.340270996094\\
6.68499994277954	-365.336242675781\\
6.69000005722046	-357.294342041016\\
6.69500017166138	-349.212158203125\\
6.69999980926514	-341.647064208984\\
6.70499992370605	-335.091217041016\\
6.71000003814697	-329.204925537109\\
6.71500015258789	-324.299285888672\\
6.71999979019165	-319.723968505859\\
6.72499990463257	-315.096832275391\\
6.73000001907349	-310.275146484375\\
6.7350001335144	-304.817291259766\\
6.73999977111816	-297.716766357422\\
6.74499988555908	-288.6845703125\\
6.75	-277.545562744141\\
6.75500011444092	-264.323883056641\\
6.76000022888184	-249.909851074219\\
6.7649998664856	-232.572723388672\\
6.76999998092651	-212.619140625\\
6.77500009536743	-190.518707275391\\
6.78000020980835	-166.43440246582\\
6.78499984741211	-140.684051513672\\
6.78999996185303	-113.372940063477\\
6.79500007629395	-84.9501800537109\\
6.80000019073486	-55.6707229614258\\
6.80499982833862	-25.8024597167969\\
6.80999994277954	4.47338104248047\\
6.81500005722046	34.891975402832\\
6.82000017166138	65.3491058349609\\
6.82499980926514	95.5875778198242\\
6.82999992370605	125.430572509766\\
6.83500003814697	154.705383300781\\
6.84000015258789	183.312515258789\\
6.84499979019165	211.000244140625\\
6.84999990463257	237.620864868164\\
6.85500001907349	263.001586914063\\
6.8600001335144	287.0107421875\\
6.86499977111816	309.255310058594\\
6.86999988555908	329.875671386719\\
6.875	348.831756591797\\
6.88000011444092	365.914916992188\\
6.88500022888184	381.187072753906\\
6.8899998664856	394.650482177734\\
6.89499998092651	406.397033691406\\
6.90000009536743	416.470550537109\\
6.90500020980835	424.973602294922\\
6.90999984741211	431.850341796875\\
6.91499996185303	437.278656005859\\
6.92000007629395	441.277954101563\\
6.92500019073486	443.856201171875\\
6.92999982833862	445.105407714844\\
6.93499994277954	444.970458984375\\
6.94000005722046	443.542663574219\\
6.94500017166138	440.815490722656\\
6.94999980926514	436.797241210938\\
6.95499992370605	431.181518554688\\
6.96000003814697	424.490905761719\\
6.96500015258789	417.089447021484\\
6.96999979019165	408.445007324219\\
6.97499990463257	399.086608886719\\
6.98000001907349	389.037017822266\\
6.9850001335144	378.542541503906\\
6.98999977111816	367.514038085938\\
6.99499988555908	356.014709472656\\
7	344.446868896484\\
7.00500011444092	332.587615966797\\
7.01000022888184	320.551086425781\\
7.0149998664856	308.353088378906\\
7.01999998092651	296.0400390625\\
7.02500009536743	283.691467285156\\
7.03000020980835	271.32177734375\\
7.03499984741211	258.968566894531\\
7.03999996185303	246.693054199219\\
7.04500007629395	234.56494140625\\
7.05000019073486	222.666229248047\\
7.05499982833862	211.099166870117\\
7.05999994277954	199.870422363281\\
7.06500005722046	189.102752685547\\
7.07000017166138	178.890258789063\\
7.07499980926514	169.261077880859\\
7.07999992370605	160.258361816406\\
7.08500003814697	151.920333862305\\
7.09000015258789	144.266738891602\\
7.09499979019165	137.304534912109\\
7.09999990463257	130.994567871094\\
7.10500001907349	125.311248779297\\
7.1100001335144	120.227210998535\\
7.11499977111816	115.765625\\
7.11999988555908	111.894409179688\\
7.125	108.585754394531\\
7.13000011444092	105.853515625\\
7.13500022888184	103.699562072754\\
7.1399998664856	102.141822814941\\
7.14499998092651	101.200012207031\\
7.15000009536743	100.922859191895\\
7.15500020980835	101.445083618164\\
7.15999984741211	102.768562316895\\
7.16499996185303	104.833114624023\\
7.17000007629395	107.202606201172\\
7.17500019073486	109.984519958496\\
7.17999982833862	113.191452026367\\
7.18499994277954	116.771774291992\\
7.19000005722046	120.664215087891\\
7.19500017166138	124.848625183105\\
7.19999980926514	129.310241699219\\
7.20499992370605	133.969161987305\\
7.21000003814697	138.889755249023\\
7.21500015258789	144.027008056641\\
7.21999979019165	149.250839233398\\
7.22499990463257	154.674346923828\\
7.23000001907349	160.222213745117\\
7.2350001335144	165.858215332031\\
7.23999977111816	171.585327148438\\
7.24499988555908	177.332885742188\\
7.25	183.066101074219\\
7.25500011444092	188.792572021484\\
7.26000022888184	194.4140625\\
7.2649998664856	199.936798095703\\
7.26999998092651	205.371765136719\\
7.27500009536743	210.625\\
7.28000020980835	215.661819458008\\
7.28499984741211	220.484985351563\\
7.28999996185303	225.085266113281\\
7.29500007629395	229.462646484375\\
7.30000019073486	233.513793945313\\
7.30499982833862	237.273529052734\\
7.30999994277954	240.762054443359\\
7.31500005722046	243.97021484375\\
7.32000017166138	246.875244140625\\
7.32499980926514	249.481811523438\\
7.32999992370605	251.758087158203\\
7.33500003814697	253.700927734375\\
7.34000015258789	255.324691772461\\
7.34499979019165	256.621459960938\\
7.34999990463257	257.585754394531\\
7.35500001907349	258.223785400391\\
7.3600001335144	258.571411132813\\
7.36499977111816	258.634704589844\\
7.36999988555908	258.415496826172\\
7.375	257.907135009766\\
7.38000011444092	257.122222900391\\
7.38500022888184	256.016448974609\\
7.3899998664856	254.561218261719\\
7.39499998092651	252.766448974609\\
7.40000009536743	250.719482421875\\
7.40500020980835	248.403381347656\\
7.40999984741211	245.850708007813\\
7.41499996185303	243.082138061523\\
7.42000007629395	240.124267578125\\
7.42500019073486	237.007675170898\\
7.42999982833862	233.749176025391\\
7.43499994277954	230.346801757813\\
7.44000005722046	226.823440551758\\
7.44500017166138	223.192169189453\\
7.44999980926514	219.468109130859\\
7.45499992370605	215.661499023438\\
7.46000003814697	211.781494140625\\
7.46500015258789	207.849914550781\\
7.46999979019165	203.909164428711\\
7.47499990463257	199.982940673828\\
7.48000001907349	196.0712890625\\
7.4850001335144	192.198760986328\\
7.48999977111816	188.381210327148\\
7.49499988555908	184.633010864258\\
7.5	180.965667724609\\
7.50500011444092	177.388153076172\\
7.51000022888184	173.912612915039\\
7.5149998664856	170.562789916992\\
7.51999998092651	167.35823059082\\
7.52500009536743	164.41520690918\\
7.53000020980835	161.502197265625\\
7.53499984741211	158.770401000977\\
7.53999996185303	156.20671081543\\
7.54500007629395	153.793884277344\\
7.55000019073486	151.556671142578\\
7.55499982833862	149.505065917969\\
7.55999994277954	147.637939453125\\
7.56500005722046	145.963424682617\\
7.57000017166138	144.49169921875\\
7.57499980926514	143.226608276367\\
7.57999992370605	142.164489746094\\
7.58500003814697	141.311340332031\\
7.59000015258789	140.657821655273\\
7.59499979019165	140.204864501953\\
7.59999990463257	139.951889038086\\
7.60500001907349	139.899368286133\\
7.6100001335144	140.06233215332\\
7.61499977111816	140.458038330078\\
7.61999988555908	141.019317626953\\
7.625	141.669189453125\\
7.63000011444092	142.440200805664\\
7.63500022888184	143.361541748047\\
7.6399998664856	144.409698486328\\
7.64499998092651	145.579299926758\\
7.65000009536743	146.874145507813\\
7.65500020980835	148.289443969727\\
7.65999984741211	149.809158325195\\
7.66499996185303	151.4287109375\\
7.67000007629395	153.142730712891\\
7.67500019073486	154.938781738281\\
7.67999982833862	156.799407958984\\
7.68499994277954	158.70979309082\\
7.69000005722046	160.655578613281\\
7.69500017166138	162.640853881836\\
7.69999980926514	164.655609130859\\
7.70499992370605	166.691040039063\\
7.71000003814697	168.732299804688\\
7.71500015258789	170.752136230469\\
7.71999979019165	172.757263183594\\
7.72499990463257	174.755950927734\\
7.73000001907349	176.739624023438\\
7.7350001335144	178.702331542969\\
7.73999977111816	180.599182128906\\
7.74499988555908	182.44841003418\\
7.75	184.264709472656\\
7.75500011444092	186.027618408203\\
7.76000022888184	187.718444824219\\
7.7649998664856	189.339889526367\\
7.76999998092651	190.890106201172\\
7.77500009536743	192.370407104492\\
7.78000020980835	193.777557373047\\
7.78499984741211	195.085845947266\\
7.78999996185303	196.311004638672\\
7.79500007629395	197.453094482422\\
7.80000019073486	198.511123657227\\
7.80499982833862	199.485473632813\\
7.80999994277954	200.37629699707\\
7.81500005722046	201.18212890625\\
7.82000017166138	201.902786254883\\
7.82499980926514	202.53889465332\\
7.82999992370605	203.092391967773\\
7.83500003814697	203.565521240234\\
7.84000015258789	203.956512451172\\
7.84499979019165	204.267883300781\\
7.84999990463257	204.512268066406\\
7.85500001907349	204.687408447266\\
7.8600001335144	204.796188354492\\
7.86499977111816	204.840362548828\\
7.86999988555908	204.822937011719\\
7.875	204.746673583984\\
7.88000011444092	204.623657226563\\
7.88500022888184	204.459548950195\\
7.8899998664856	204.257598876953\\
7.89499998092651	204.019104003906\\
7.90000009536743	203.744720458984\\
7.90500020980835	203.440216064453\\
7.90999984741211	203.109069824219\\
7.91499996185303	202.754638671875\\
7.92000007629395	202.380493164063\\
7.92500019073486	201.991668701172\\
7.92999982833862	201.615081787109\\
7.93499994277954	201.235076904297\\
7.94000005722046	200.851593017578\\
7.94500017166138	200.502502441406\\
7.94999980926514	200.177200317383\\
7.95499992370605	199.870223999023\\
7.96000003814697	199.613555908203\\
7.96500015258789	199.416931152344\\
7.96999979019165	199.265289306641\\
7.97499990463257	199.160934448242\\
7.98000001907349	199.107757568359\\
7.9850001335144	199.102554321289\\
7.98999977111816	199.148956298828\\
7.99499988555908	199.271392822266\\
8	199.455230712891\\
8.00500011444092	199.700073242188\\
8.01000022888184	200.034698486328\\
8.01500034332275	200.456405639648\\
8.02000045776367	200.969268798828\\
8.02499961853027	201.5625\\
8.02999973297119	202.233810424805\\
8.03499984741211	202.988220214844\\
8.03999996185303	203.825164794922\\
8.04500007629395	204.743469238281\\
8.05000019073486	205.743637084961\\
8.05500030517578	206.825408935547\\
8.0600004196167	207.987594604492\\
8.0649995803833	209.230072021484\\
8.06999969482422	210.552368164063\\
8.07499980926514	211.978179931641\\
8.07999992370605	213.501739501953\\
8.08500003814697	215.126724243164\\
8.09000015258789	216.825820922852\\
8.09500026702881	218.595397949219\\
8.10000038146973	220.436172485352\\
8.10499954223633	222.3544921875\\
8.10999965667725	224.356994628906\\
8.11499977111816	226.437408447266\\
8.11999988555908	228.591003417969\\
8.125	230.804824829102\\
8.13000011444092	233.080017089844\\
8.13500022888184	235.413558959961\\
8.14000034332275	237.816665649414\\
8.14500045776367	240.281646728516\\
8.14999961853027	242.806732177734\\
8.15499973297119	245.369201660156\\
8.15999984741211	247.96630859375\\
8.16499996185303	250.593139648438\\
8.17000007629395	253.277908325195\\
8.17500019073486	256.036346435547\\
8.18000030517578	258.859283447266\\
8.1850004196167	261.715179443359\\
8.1899995803833	264.538818359375\\
8.19499969482422	267.3447265625\\
8.19999980926514	270.145080566406\\
8.20499992370605	272.96240234375\\
8.21000003814697	275.776306152344\\
8.21500015258789	278.560180664063\\
8.22000026702881	281.288848876953\\
8.22500038146973	283.957763671875\\
8.22999954223633	286.703002929688\\
8.23499965667725	289.594360351563\\
8.23999977111816	292.41943359375\\
8.24499988555908	295.018890380859\\
8.25	297.507934570313\\
8.25500011444092	299.965698242188\\
8.26000022888184	302.324157714844\\
8.26500034332275	304.572326660156\\
8.27000045776367	306.742492675781\\
8.27499961853027	308.812866210938\\
8.27999973297119	310.680053710938\\
8.28499984741211	312.359558105469\\
8.28999996185303	313.917205810547\\
8.29500007629395	315.244445800781\\
8.30000019073486	316.269866943359\\
8.30500030517578	316.881744384766\\
8.3100004196167	317.098693847656\\
8.3149995803833	317.996032714844\\
8.31999969482422	317.828491210938\\
8.32499980926514	317.302947998047\\
8.32999992370605	316.363525390625\\
8.33500003814697	314.989898681641\\
8.34000015258789	313.080413818359\\
8.34500026702881	310.550598144531\\
8.35000038146973	307.422302246094\\
8.35499954223633	303.409973144531\\
8.35999965667725	298.972320556641\\
8.36499977111816	293.772338867188\\
8.36999988555908	287.453857421875\\
8.375	280.149291992188\\
8.38000011444092	271.816162109375\\
8.38500022888184	262.218444824219\\
8.39000034332275	251.345504760742\\
8.39500045776367	238.921005249023\\
8.39999961853027	225.415405273438\\
8.40499973297119	209.939636230469\\
8.40999984741211	192.788986206055\\
8.41499996185303	173.889053344727\\
8.42000007629395	153.1220703125\\
8.42500019073486	130.573440551758\\
8.43000030517578	106.216766357422\\
8.4350004196167	80.2500915527344\\
8.4399995803833	52.838249206543\\
8.44499969482422	23.9721374511719\\
8.44999980926514	-5.98745727539063\\
8.45499992370605	-36.8202056884766\\
8.46000003814697	-68.2707214355469\\
8.46500015258789	-100.067611694336\\
8.47000026702881	-131.9326171875\\
8.47500038146973	-163.600997924805\\
8.47999954223633	-194.824249267578\\
8.48499965667725	-225.102142333984\\
8.48999977111816	-253.9638671875\\
8.49499988555908	-281.753356933594\\
8.5	-307.812530517578\\
8.50500011444092	-332.277801513672\\
8.51000022888184	-354.459655761719\\
8.51500034332275	-374.922576904297\\
8.52000045776367	-394.080871582031\\
8.52499961853027	-409.134033203125\\
8.52999973297119	-423.165649414063\\
8.53499984741211	-434.797912597656\\
8.53999996185303	-444.159729003906\\
8.54500007629395	-451.7255859375\\
8.55000019073486	-457.671081542969\\
8.55500030517578	-462.298706054688\\
8.5600004196167	-465.669677734375\\
8.5649995803833	-468.657287597656\\
8.56999969482422	-471.567932128906\\
8.57499980926514	-474.455383300781\\
8.57999992370605	-477.836364746094\\
8.58500003814697	-481.430114746094\\
8.59000015258789	-485.798767089844\\
8.59500026702881	-491.719299316406\\
8.60000038146973	-499.478637695313\\
8.60499954223633	-508.792297363281\\
8.60999965667725	-519.651245117188\\
8.61499977111816	-531.479919433594\\
8.61999988555908	-543.386108398438\\
8.625	-554.869750976563\\
8.63000011444092	-565.027954101563\\
8.63500022888184	-573.452697753906\\
8.64000034332275	-579.160705566406\\
8.64500045776367	-582.479797363281\\
8.64999961853027	-582.674438476563\\
8.65499973297119	-580.894775390625\\
8.65999984741211	-575.34716796875\\
8.66499996185303	-569.499145507813\\
8.67000007629395	-562.372680664063\\
8.67500019073486	-555.346740722656\\
8.68000030517578	-548.263122558594\\
8.6850004196167	-541.837707519531\\
8.6899995803833	-536.457580566406\\
8.69499969482422	-532.012390136719\\
8.69999980926514	-528.070739746094\\
8.70499992370605	-524.484130859375\\
8.71000003814697	-520.940673828125\\
8.71500015258789	-517.148010253906\\
8.72000026702881	-512.313354492188\\
8.72500038146973	-506.348876953125\\
8.72999954223633	-499.080810546875\\
8.73499965667725	-490.361877441406\\
8.73999977111816	-480.191253662109\\
8.74499988555908	-468.854614257813\\
8.75	-456.906860351563\\
8.75500011444092	-445.162506103516\\
8.76000022888184	-433.256713867188\\
8.76500034332275	-421.972595214844\\
8.77000045776367	-411.435974121094\\
8.77499961853027	-401.866760253906\\
8.77999973297119	-393.20849609375\\
8.78499984741211	-385.483917236328\\
8.78999996185303	-378.486083984375\\
8.79500007629395	-371.981994628906\\
8.80000019073486	-365.726104736328\\
8.80500030517578	-359.515441894531\\
8.8100004196167	-353.172790527344\\
8.8149995803833	-346.556640625\\
8.81999969482422	-339.647338867188\\
8.82499980926514	-332.531585693359\\
8.82999992370605	-325.326934814453\\
8.83500003814697	-318.278076171875\\
8.84000015258789	-311.604522705078\\
8.84500026702881	-305.492767333984\\
8.85000038146973	-300.251922607422\\
8.85499954223633	-295.83984375\\
8.85999965667725	-292.374694824219\\
8.86499977111816	-289.794372558594\\
8.86999988555908	-287.989135742188\\
8.875	-286.786682128906\\
8.88000011444092	-285.991180419922\\
8.88500022888184	-285.421173095703\\
8.89000034332275	-284.912963867188\\
8.89500045776367	-284.390991210938\\
8.89999961853027	-283.791625976563\\
8.90499973297119	-283.065155029297\\
8.90999984741211	-282.402404785156\\
8.91499996185303	-282.078308105469\\
8.92000007629395	-282.003265380859\\
8.92500019073486	-282.264984130859\\
8.93000030517578	-283.116760253906\\
8.9350004196167	-284.614410400391\\
8.9399995803833	-286.8017578125\\
8.94499969482422	-289.320190429688\\
8.94999980926514	-292.051239013672\\
8.95499992370605	-295.169830322266\\
8.96000003814697	-298.442169189453\\
8.96500015258789	-301.767944335938\\
8.97000026702881	-305.080108642578\\
8.97500038146973	-308.400970458984\\
8.97999954223633	-311.736663818359\\
8.98499965667725	-314.902313232422\\
8.98999977111816	-318.177947998047\\
8.99499988555908	-321.656433105469\\
9	-324.99609375\\
9.00500011444092	-328.387908935547\\
9.01000022888184	-331.852874755859\\
9.01500034332275	-335.368865966797\\
9.02000045776367	-338.890960693359\\
9.02499961853027	-342.406951904297\\
9.02999973297119	-345.894073486328\\
9.03499984741211	-349.309020996094\\
9.03999996185303	-352.620544433594\\
9.04500007629395	-355.798553466797\\
9.05000019073486	-358.812408447266\\
9.05500030517578	-361.637908935547\\
9.0600004196167	-364.271850585938\\
9.0649995803833	-366.6845703125\\
9.06999969482422	-368.886505126953\\
9.07499980926514	-370.858062744141\\
9.07999992370605	-372.590606689453\\
9.08500003814697	-374.075042724609\\
9.09000015258789	-375.320465087891\\
9.09500026702881	-376.38232421875\\
9.10000038146973	-377.256408691406\\
9.10499954223633	-377.876281738281\\
9.10999965667725	-378.267578125\\
9.11499977111816	-378.512023925781\\
9.11999988555908	-378.631256103516\\
9.125	-378.562927246094\\
9.13000011444092	-378.224548339844\\
9.13500022888184	-377.641143798828\\
9.14000034332275	-376.763610839844\\
9.14500045776367	-375.608825683594\\
9.14999961853027	-374.161956787109\\
9.15499973297119	-372.462463378906\\
9.15999984741211	-370.536682128906\\
9.16499996185303	-368.432769775391\\
9.17000007629395	-366.227874755859\\
9.17500019073486	-363.952667236328\\
9.18000030517578	-361.663635253906\\
9.1850004196167	-359.412078857422\\
9.1899995803833	-357.19189453125\\
9.19499969482422	-355.020629882813\\
9.19999980926514	-352.924652099609\\
9.20499992370605	-350.894683837891\\
9.21000003814697	-348.896179199219\\
9.21500015258789	-346.868011474609\\
9.22000026702881	-344.826721191406\\
9.22500038146973	-342.959625244141\\
9.22999954223633	-341.244842529297\\
9.23499965667725	-339.9521484375\\
9.23999977111816	-339.855651855469\\
9.24499988555908	-341.972869873047\\
9.25	-346.648315429688\\
9.25500011444092	-353.463043212891\\
9.26000022888184	-362.571258544922\\
9.26500034332275	-374.165863037109\\
9.27000045776367	-387.652313232422\\
9.27499961853027	-403.376403808594\\
9.27999973297119	-421.763519287109\\
9.28499984741211	-442.807006835938\\
9.28999996185303	-467.108428955078\\
9.29500007629395	-495.166290283203\\
9.30000019073486	-527.250793457031\\
9.30500030517578	-566.303100585938\\
9.3100004196167	-609.601806640625\\
9.3149995803833	-659.525390625\\
9.31999969482422	-716.902954101563\\
9.32499980926514	-782.759765625\\
9.32999992370605	-859.139587402344\\
9.33500003814697	-945.9853515625\\
9.34000015258789	-1044.29711914063\\
9.34500026702881	-1160.63500976563\\
9.35000038146973	-1289.92651367188\\
9.35499954223633	-1434.38549804688\\
9.35999965667725	-1598.25219726563\\
9.36499977111816	-1771.92504882813\\
9.36999988555908	-1951.71337890625\\
9.375	-2133.30102539063\\
9.38000011444092	-2309.1162109375\\
9.38500022888184	-2472.42138671875\\
9.39000034332275	-2617.43383789063\\
9.39500045776367	-2737.310546875\\
9.39999961853027	-2824.314453125\\
9.40499973297119	-2871.60693359375\\
9.40999984741211	-2872.99487304688\\
9.41499996185303	-2822.720703125\\
9.42000007629395	-2716.04565429688\\
9.42500019073486	-2552.166015625\\
9.43000030517578	-2347.76171875\\
9.4350004196167	-2091.06323242188\\
9.4399995803833	-1783.05249023438\\
9.44499969482422	-1437.51110839844\\
9.44999980926514	-1073.79870605469\\
9.45499992370605	-719.64599609375\\
9.46000003814697	-403.577209472656\\
9.46500015258789	-153.610321044922\\
9.47000026702881	12.3342895507813\\
9.47500038146973	90.54931640625\\
9.47999954223633	90.6701049804688\\
9.48499965667725	22.1504821777344\\
9.48999977111816	-73.8514709472656\\
9.49499988555908	-159.586761474609\\
9.5	-229.632598876953\\
9.50500011444092	-278.916198730469\\
9.51000022888184	-306.023773193359\\
9.51500034332275	-312.531768798828\\
9.52000045776367	-300.161010742188\\
9.52499961853027	-272.685180664063\\
9.52999973297119	-231.253112792969\\
9.53499984741211	-181.066741943359\\
9.53999996185303	-124.834922790527\\
9.54500007629395	-65.6634140014648\\
9.55000019073486	-6.87306880950928\\
9.55500030517578	-4.46394872665405\\
9.5600004196167	-5.64716291427612\\
9.5649995803833	-5.84261417388916\\
9.56999969482422	-5.55730724334717\\
9.57499980926514	-5.03012323379517\\
9.57999992370605	-4.56477212905884\\
9.58500003814697	-3.98586893081665\\
9.59000015258789	-3.4477379322052\\
9.59500026702881	-2.99691486358643\\
9.60000038146973	-2.60591912269592\\
9.60499954223633	-2.23323202133179\\
9.60999965667725	-1.8926295042038\\
9.61499977111816	-1.61023819446564\\
9.61999988555908	-1.35563588142395\\
9.625	-1.13980543613434\\
9.63000011444092	-0.943061709403992\\
9.63500022888184	-0.785603106021881\\
9.64000034332275	-0.639583051204681\\
9.64500045776367	-0.498232185840607\\
9.64999961853027	-0.395867437124252\\
9.65499973297119	-0.312859892845154\\
9.65999984741211	-0.226449772715569\\
9.66499996185303	-0.1550432741642\\
9.67000007629395	-0.0956037789583206\\
9.67500019073486	-0.0439781285822392\\
9.68000030517578	-0.000455858709756285\\
9.6850004196167	0.0366440638899803\\
9.6899995803833	0.0680756419897079\\
9.69499969482422	0.0947906449437141\\
9.69999980926514	0.118509501218796\\
9.70499992370605	0.140522822737694\\
9.71000003814697	0.159736007452011\\
9.71500015258789	0.175179600715637\\
9.72000026702881	0.187430992722511\\
9.72500038146973	0.195728704333305\\
9.72999954223633	0.202163070440292\\
9.73499965667725	0.208391264081001\\
9.73999977111816	0.21707184612751\\
9.74499988555908	0.23142683506012\\
9.75	0.238891273736954\\
9.75500011444092	0.243232890963554\\
9.76000022888184	0.242178440093994\\
9.76500034332275	0.239218637347221\\
9.77000045776367	0.239091485738754\\
9.77499961853027	0.239570066332817\\
9.77999973297119	0.241363301873207\\
9.78499984741211	0.241760119795799\\
9.78999996185303	0.242424517869949\\
9.79500007629395	0.243681102991104\\
9.80000019073486	0.244842559099197\\
9.80500030517578	0.24474324285984\\
9.8100004196167	0.243203401565552\\
9.8149995803833	0.239609003067017\\
9.81999969482422	0.238187700510025\\
9.82499980926514	0.237994745373726\\
9.82999992370605	0.240071535110474\\
9.83500003814697	0.242009773850441\\
9.84000015258789	0.240788042545319\\
9.84500026702881	0.236731812357903\\
9.85000038146973	0.228993102908134\\
9.85499954223633	0.227082684636116\\
9.85999965667725	0.227175712585449\\
9.86499977111816	0.229913175106049\\
9.86999988555908	0.23204343020916\\
9.875	0.231235027313232\\
9.88000011444092	0.22875751554966\\
9.88500022888184	0.224412903189659\\
9.89000034332275	0.221380993723869\\
9.89500045776367	0.217975065112114\\
9.89999961853027	0.214116871356964\\
9.90499973297119	0.210883781313896\\
9.90999984741211	0.209485217928886\\
9.91499996185303	0.210196599364281\\
9.92000007629395	0.213304489850998\\
9.92500019073486	0.212549030780792\\
9.93000030517578	0.209679216146469\\
9.9350004196167	0.203546360135078\\
9.9399995803833	0.19802825152874\\
9.94499969482422	0.19506448507309\\
9.94999980926514	0.192541614174843\\
9.95499992370605	0.19053041934967\\
9.96000003814697	0.189101204276085\\
9.96500015258789	0.188323885202408\\
9.97000026702881	0.188267856836319\\
9.97500038146973	0.187228858470917\\
9.97999954223633	0.185208722949028\\
9.98499965667725	0.18309086561203\\
9.98999977111816	0.180869400501251\\
9.99499988555908	0.178538516163826\\
10	0.176092550158501\\
};
\addlegendentry{Imbalance}

\end{axis}
\end{tikzpicture}%
	\end{tikzpicture}}
	\caption{Comparison of Controller Effort and Net Imbalance on CF}
	\label{fig:imbVsCont}
\end{figure}
Thus, it seems like the control forces as a result of position control try to recreate the imbalance forces experienced by the bike. It is observed after quite a lot of simulations with different controller parameters that better the correlation between the control forces and the imbalance forces is, better are the loads comparison between CF and RS at the validation points. It is also important to note that it was observed that for all classical PID and PD feedback cases, the control torque was always significantly less than the net imbalance torque experienced by the bike, as seen in Figure \ref{fig:imbVsCont}.
